\chapter{Conclusion} 
\label{chapter:conclusion}
The aim of this project was to build an easy-to-use web application that allows Ordnance Survey employees to upload a complex aerial photograph, and train the system to identify and label specific features on this image and other similar images. Employees can train the system with supervised learning data by selecting features on the uploaded image and providing labels for these features. At the click of a button, the system can then propagate this knowledge across any given map image to label other occurrences of these features. Employees can dynamically add new classes to the system, and iteratively provide additional training data to be added to the system’s knowledge-base, enhancing its ability to identify new and existing features. The vast complexity that operates behind the scenes is abstracted from the user, meaning that no previous knowledge in machine learning is required to operate the system. 

These functional requirements of the application were not only achieved, but also optimised to improve the performance of the system while providing an intuitive and enjoyable user experience. Additionally, the project was implemented with a modular and expandable code base, allowing for the prototype to be modified, improved and extended as required by the customer. 

By working under an agile methodology any obstacles that arose during the project could be efficiently tackled, and changes to the initial project goals could be catered for, ultimately creating a stronger, more effective and conclusively successful final product. As a prototype application, the project opens itself up to a wide range of future extensions and improvements that will only enhance the already powerful tool that has been created.