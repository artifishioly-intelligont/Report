 %% ----------------------------------------------------------------
%% Progress.tex
%% ---------------------------------------------------------------- 
\documentclass{ecsprogress}    % Use the progress Style
\graphicspath{{../figs/}}   % Location of your graphics files
    \usepackage{natbib}            % Use Natbib style for the refs.
\hypersetup{colorlinks=true}   % Set to false for black/white printing
\input{Definitions}            % Include your abbreviations



\usepackage{enumitem}% http://ctan.org/pkg/enumitem
\usepackage{multirow}
\usepackage{float}
\usepackage{amsmath}
\usepackage{multicol}
\usepackage{amssymb}
\usepackage[normalem]{ulem}
\useunder{\uline}{\ul}{}
\usepackage{wrapfig}


\usepackage[table,xcdraw]{xcolor}


%% ----------------------------------------------------------------
\begin{document}
\frontmatter
\title      {Heterogeneous Agent-based Model for Supermarket Competition}
\authors    {\texorpdfstring
             {\href{mailto:sc22g13@ecs.soton.ac.uk}{Stefan J. Collier}}
             {Stefan J. Collier}
            }
\addresses  {\groupname\\\deptname\\\univname}
\date       {\today}
\subject    {}
\keywords   {}
\supervisor {Dr. Maria Polukarov}
\examiner   {Professor Sheng Chen}

\maketitle
\begin{abstract}
This project aim was to model and analyse the effects of competitive pricing behaviors of grocery retailers on the British market. 

This was achieved by creating a multi-agent model, containing retailer and consumer agents. The heterogeneous crowd of retailers employs either a uniform pricing strategy or a ‘local price flexing’ strategy. The actions of these retailers are chosen by predicting the profit of each action, using a perceptron. Following on from the consideration of different economic models, a discrete model was developed so that software agents have a discrete environment to operate within. Within the model, it has been observed how supermarkets with differing behaviors affect a heterogeneous crowd of consumer agents. The model was implemented in Java with Python used to evaluate the results. 

The simulation displays good acceptance with real grocery market behavior, i.e. captures the performance of British retailers thus can be used to determine the impact of changes in their behavior on their competitors and consumers.Furthermore it can be used to provide insight into sustainability of volatile pricing strategies, providing a useful insight in volatility of British supermarket retail industry. 
\end{abstract}
\acknowledgements{
I would like to express my sincere gratitude to Dr Maria Polukarov for her guidance and support which provided me the freedom to take this research in the direction of my interest.\\
\\
I would also like to thank my family and friends for their encouragement and support. To those who quietly listened to my software complaints. To those who worked throughout the nights with me. To those who helped me write what I couldn't say. I cannot thank you enough.
}

\declaration{
I, Stefan Collier, declare that this dissertation and the work presented in it are my own and has been generated by me as the result of my own original research.\\
I confirm that:\\
1. This work was done wholly or mainly while in candidature for a degree at this University;\\
2. Where any part of this dissertation has previously been submitted for any other qualification at this University or any other institution, this has been clearly stated;\\
3. Where I have consulted the published work of others, this is always clearly attributed;\\
4. Where I have quoted from the work of others, the source is always given. With the exception of such quotations, this dissertation is entirely my own work;\\
5. I have acknowledged all main sources of help;\\
6. Where the thesis is based on work done by myself jointly with others, I have made clear exactly what was done by others and what I have contributed myself;\\
7. Either none of this work has been published before submission, or parts of this work have been published by :\\
\\
Stefan Collier\\
April 2016
}
\tableofcontents
\listoffigures
\listoftables

\mainmatter
%% ----------------------------------------------------------------
%\include{Introduction}
%\include{Conclusions}
 %% ----------------------------------------------------------------
%% Progress.tex
%% ---------------------------------------------------------------- 
\documentclass{ecsprogress}    % Use the progress Style
\graphicspath{{../figs/}}   % Location of your graphics files
    \usepackage{natbib}            % Use Natbib style for the refs.
\hypersetup{colorlinks=true}   % Set to false for black/white printing
\input{Definitions}            % Include your abbreviations



\usepackage{enumitem}% http://ctan.org/pkg/enumitem
\usepackage{multirow}
\usepackage{float}
\usepackage{amsmath}
\usepackage{multicol}
\usepackage{amssymb}
\usepackage[normalem]{ulem}
\useunder{\uline}{\ul}{}
\usepackage{wrapfig}


\usepackage[table,xcdraw]{xcolor}


%% ----------------------------------------------------------------
\begin{document}
\frontmatter
\title      {Heterogeneous Agent-based Model for Supermarket Competition}
\authors    {\texorpdfstring
             {\href{mailto:sc22g13@ecs.soton.ac.uk}{Stefan J. Collier}}
             {Stefan J. Collier}
            }
\addresses  {\groupname\\\deptname\\\univname}
\date       {\today}
\subject    {}
\keywords   {}
\supervisor {Dr. Maria Polukarov}
\examiner   {Professor Sheng Chen}

\maketitle
\begin{abstract}
This project aim was to model and analyse the effects of competitive pricing behaviors of grocery retailers on the British market. 

This was achieved by creating a multi-agent model, containing retailer and consumer agents. The heterogeneous crowd of retailers employs either a uniform pricing strategy or a ‘local price flexing’ strategy. The actions of these retailers are chosen by predicting the profit of each action, using a perceptron. Following on from the consideration of different economic models, a discrete model was developed so that software agents have a discrete environment to operate within. Within the model, it has been observed how supermarkets with differing behaviors affect a heterogeneous crowd of consumer agents. The model was implemented in Java with Python used to evaluate the results. 

The simulation displays good acceptance with real grocery market behavior, i.e. captures the performance of British retailers thus can be used to determine the impact of changes in their behavior on their competitors and consumers.Furthermore it can be used to provide insight into sustainability of volatile pricing strategies, providing a useful insight in volatility of British supermarket retail industry. 
\end{abstract}
\acknowledgements{
I would like to express my sincere gratitude to Dr Maria Polukarov for her guidance and support which provided me the freedom to take this research in the direction of my interest.\\
\\
I would also like to thank my family and friends for their encouragement and support. To those who quietly listened to my software complaints. To those who worked throughout the nights with me. To those who helped me write what I couldn't say. I cannot thank you enough.
}

\declaration{
I, Stefan Collier, declare that this dissertation and the work presented in it are my own and has been generated by me as the result of my own original research.\\
I confirm that:\\
1. This work was done wholly or mainly while in candidature for a degree at this University;\\
2. Where any part of this dissertation has previously been submitted for any other qualification at this University or any other institution, this has been clearly stated;\\
3. Where I have consulted the published work of others, this is always clearly attributed;\\
4. Where I have quoted from the work of others, the source is always given. With the exception of such quotations, this dissertation is entirely my own work;\\
5. I have acknowledged all main sources of help;\\
6. Where the thesis is based on work done by myself jointly with others, I have made clear exactly what was done by others and what I have contributed myself;\\
7. Either none of this work has been published before submission, or parts of this work have been published by :\\
\\
Stefan Collier\\
April 2016
}
\tableofcontents
\listoffigures
\listoftables

\mainmatter
%% ----------------------------------------------------------------
%\include{Introduction}
%\include{Conclusions}
 %% ----------------------------------------------------------------
%% Progress.tex
%% ---------------------------------------------------------------- 
\documentclass{ecsprogress}    % Use the progress Style
\graphicspath{{../figs/}}   % Location of your graphics files
    \usepackage{natbib}            % Use Natbib style for the refs.
\hypersetup{colorlinks=true}   % Set to false for black/white printing
\input{Definitions}            % Include your abbreviations



\usepackage{enumitem}% http://ctan.org/pkg/enumitem
\usepackage{multirow}
\usepackage{float}
\usepackage{amsmath}
\usepackage{multicol}
\usepackage{amssymb}
\usepackage[normalem]{ulem}
\useunder{\uline}{\ul}{}
\usepackage{wrapfig}


\usepackage[table,xcdraw]{xcolor}


%% ----------------------------------------------------------------
\begin{document}
\frontmatter
\title      {Heterogeneous Agent-based Model for Supermarket Competition}
\authors    {\texorpdfstring
             {\href{mailto:sc22g13@ecs.soton.ac.uk}{Stefan J. Collier}}
             {Stefan J. Collier}
            }
\addresses  {\groupname\\\deptname\\\univname}
\date       {\today}
\subject    {}
\keywords   {}
\supervisor {Dr. Maria Polukarov}
\examiner   {Professor Sheng Chen}

\maketitle
\begin{abstract}
This project aim was to model and analyse the effects of competitive pricing behaviors of grocery retailers on the British market. 

This was achieved by creating a multi-agent model, containing retailer and consumer agents. The heterogeneous crowd of retailers employs either a uniform pricing strategy or a ‘local price flexing’ strategy. The actions of these retailers are chosen by predicting the profit of each action, using a perceptron. Following on from the consideration of different economic models, a discrete model was developed so that software agents have a discrete environment to operate within. Within the model, it has been observed how supermarkets with differing behaviors affect a heterogeneous crowd of consumer agents. The model was implemented in Java with Python used to evaluate the results. 

The simulation displays good acceptance with real grocery market behavior, i.e. captures the performance of British retailers thus can be used to determine the impact of changes in their behavior on their competitors and consumers.Furthermore it can be used to provide insight into sustainability of volatile pricing strategies, providing a useful insight in volatility of British supermarket retail industry. 
\end{abstract}
\acknowledgements{
I would like to express my sincere gratitude to Dr Maria Polukarov for her guidance and support which provided me the freedom to take this research in the direction of my interest.\\
\\
I would also like to thank my family and friends for their encouragement and support. To those who quietly listened to my software complaints. To those who worked throughout the nights with me. To those who helped me write what I couldn't say. I cannot thank you enough.
}

\declaration{
I, Stefan Collier, declare that this dissertation and the work presented in it are my own and has been generated by me as the result of my own original research.\\
I confirm that:\\
1. This work was done wholly or mainly while in candidature for a degree at this University;\\
2. Where any part of this dissertation has previously been submitted for any other qualification at this University or any other institution, this has been clearly stated;\\
3. Where I have consulted the published work of others, this is always clearly attributed;\\
4. Where I have quoted from the work of others, the source is always given. With the exception of such quotations, this dissertation is entirely my own work;\\
5. I have acknowledged all main sources of help;\\
6. Where the thesis is based on work done by myself jointly with others, I have made clear exactly what was done by others and what I have contributed myself;\\
7. Either none of this work has been published before submission, or parts of this work have been published by :\\
\\
Stefan Collier\\
April 2016
}
\tableofcontents
\listoffigures
\listoftables

\mainmatter
%% ----------------------------------------------------------------
%\include{Introduction}
%\include{Conclusions}
 %% ----------------------------------------------------------------
%% Progress.tex
%% ---------------------------------------------------------------- 
\documentclass{ecsprogress}    % Use the progress Style
\graphicspath{{../figs/}}   % Location of your graphics files
    \usepackage{natbib}            % Use Natbib style for the refs.
\hypersetup{colorlinks=true}   % Set to false for black/white printing
\input{Definitions}            % Include your abbreviations



\usepackage{enumitem}% http://ctan.org/pkg/enumitem
\usepackage{multirow}
\usepackage{float}
\usepackage{amsmath}
\usepackage{multicol}
\usepackage{amssymb}
\usepackage[normalem]{ulem}
\useunder{\uline}{\ul}{}
\usepackage{wrapfig}


\usepackage[table,xcdraw]{xcolor}


%% ----------------------------------------------------------------
\begin{document}
\frontmatter
\title      {Heterogeneous Agent-based Model for Supermarket Competition}
\authors    {\texorpdfstring
             {\href{mailto:sc22g13@ecs.soton.ac.uk}{Stefan J. Collier}}
             {Stefan J. Collier}
            }
\addresses  {\groupname\\\deptname\\\univname}
\date       {\today}
\subject    {}
\keywords   {}
\supervisor {Dr. Maria Polukarov}
\examiner   {Professor Sheng Chen}

\maketitle
\begin{abstract}
This project aim was to model and analyse the effects of competitive pricing behaviors of grocery retailers on the British market. 

This was achieved by creating a multi-agent model, containing retailer and consumer agents. The heterogeneous crowd of retailers employs either a uniform pricing strategy or a ‘local price flexing’ strategy. The actions of these retailers are chosen by predicting the profit of each action, using a perceptron. Following on from the consideration of different economic models, a discrete model was developed so that software agents have a discrete environment to operate within. Within the model, it has been observed how supermarkets with differing behaviors affect a heterogeneous crowd of consumer agents. The model was implemented in Java with Python used to evaluate the results. 

The simulation displays good acceptance with real grocery market behavior, i.e. captures the performance of British retailers thus can be used to determine the impact of changes in their behavior on their competitors and consumers.Furthermore it can be used to provide insight into sustainability of volatile pricing strategies, providing a useful insight in volatility of British supermarket retail industry. 
\end{abstract}
\acknowledgements{
I would like to express my sincere gratitude to Dr Maria Polukarov for her guidance and support which provided me the freedom to take this research in the direction of my interest.\\
\\
I would also like to thank my family and friends for their encouragement and support. To those who quietly listened to my software complaints. To those who worked throughout the nights with me. To those who helped me write what I couldn't say. I cannot thank you enough.
}

\declaration{
I, Stefan Collier, declare that this dissertation and the work presented in it are my own and has been generated by me as the result of my own original research.\\
I confirm that:\\
1. This work was done wholly or mainly while in candidature for a degree at this University;\\
2. Where any part of this dissertation has previously been submitted for any other qualification at this University or any other institution, this has been clearly stated;\\
3. Where I have consulted the published work of others, this is always clearly attributed;\\
4. Where I have quoted from the work of others, the source is always given. With the exception of such quotations, this dissertation is entirely my own work;\\
5. I have acknowledged all main sources of help;\\
6. Where the thesis is based on work done by myself jointly with others, I have made clear exactly what was done by others and what I have contributed myself;\\
7. Either none of this work has been published before submission, or parts of this work have been published by :\\
\\
Stefan Collier\\
April 2016
}
\tableofcontents
\listoffigures
\listoftables

\mainmatter
%% ----------------------------------------------------------------
%\include{Introduction}
%\include{Conclusions}
\include{chapters/1Project/main}
\include{chapters/2Lit/main}
\include{chapters/3Design/HighLevel}
\include{chapters/3Design/InDepth}
\include{chapters/4Impl/main}

\include{chapters/5Experiments/1/main}
\include{chapters/5Experiments/2/main}
\include{chapters/5Experiments/3/main}
\include{chapters/5Experiments/4/main}

\include{chapters/6Conclusion/main}

\appendix
\include{appendix/AppendixB}
\include{appendix/D/main}
\include{appendix/AppendixC}

\backmatter
\bibliographystyle{ecs}
\bibliography{ECS}
\end{document}
%% ----------------------------------------------------------------

 %% ----------------------------------------------------------------
%% Progress.tex
%% ---------------------------------------------------------------- 
\documentclass{ecsprogress}    % Use the progress Style
\graphicspath{{../figs/}}   % Location of your graphics files
    \usepackage{natbib}            % Use Natbib style for the refs.
\hypersetup{colorlinks=true}   % Set to false for black/white printing
\input{Definitions}            % Include your abbreviations



\usepackage{enumitem}% http://ctan.org/pkg/enumitem
\usepackage{multirow}
\usepackage{float}
\usepackage{amsmath}
\usepackage{multicol}
\usepackage{amssymb}
\usepackage[normalem]{ulem}
\useunder{\uline}{\ul}{}
\usepackage{wrapfig}


\usepackage[table,xcdraw]{xcolor}


%% ----------------------------------------------------------------
\begin{document}
\frontmatter
\title      {Heterogeneous Agent-based Model for Supermarket Competition}
\authors    {\texorpdfstring
             {\href{mailto:sc22g13@ecs.soton.ac.uk}{Stefan J. Collier}}
             {Stefan J. Collier}
            }
\addresses  {\groupname\\\deptname\\\univname}
\date       {\today}
\subject    {}
\keywords   {}
\supervisor {Dr. Maria Polukarov}
\examiner   {Professor Sheng Chen}

\maketitle
\begin{abstract}
This project aim was to model and analyse the effects of competitive pricing behaviors of grocery retailers on the British market. 

This was achieved by creating a multi-agent model, containing retailer and consumer agents. The heterogeneous crowd of retailers employs either a uniform pricing strategy or a ‘local price flexing’ strategy. The actions of these retailers are chosen by predicting the profit of each action, using a perceptron. Following on from the consideration of different economic models, a discrete model was developed so that software agents have a discrete environment to operate within. Within the model, it has been observed how supermarkets with differing behaviors affect a heterogeneous crowd of consumer agents. The model was implemented in Java with Python used to evaluate the results. 

The simulation displays good acceptance with real grocery market behavior, i.e. captures the performance of British retailers thus can be used to determine the impact of changes in their behavior on their competitors and consumers.Furthermore it can be used to provide insight into sustainability of volatile pricing strategies, providing a useful insight in volatility of British supermarket retail industry. 
\end{abstract}
\acknowledgements{
I would like to express my sincere gratitude to Dr Maria Polukarov for her guidance and support which provided me the freedom to take this research in the direction of my interest.\\
\\
I would also like to thank my family and friends for their encouragement and support. To those who quietly listened to my software complaints. To those who worked throughout the nights with me. To those who helped me write what I couldn't say. I cannot thank you enough.
}

\declaration{
I, Stefan Collier, declare that this dissertation and the work presented in it are my own and has been generated by me as the result of my own original research.\\
I confirm that:\\
1. This work was done wholly or mainly while in candidature for a degree at this University;\\
2. Where any part of this dissertation has previously been submitted for any other qualification at this University or any other institution, this has been clearly stated;\\
3. Where I have consulted the published work of others, this is always clearly attributed;\\
4. Where I have quoted from the work of others, the source is always given. With the exception of such quotations, this dissertation is entirely my own work;\\
5. I have acknowledged all main sources of help;\\
6. Where the thesis is based on work done by myself jointly with others, I have made clear exactly what was done by others and what I have contributed myself;\\
7. Either none of this work has been published before submission, or parts of this work have been published by :\\
\\
Stefan Collier\\
April 2016
}
\tableofcontents
\listoffigures
\listoftables

\mainmatter
%% ----------------------------------------------------------------
%\include{Introduction}
%\include{Conclusions}
\include{chapters/1Project/main}
\include{chapters/2Lit/main}
\include{chapters/3Design/HighLevel}
\include{chapters/3Design/InDepth}
\include{chapters/4Impl/main}

\include{chapters/5Experiments/1/main}
\include{chapters/5Experiments/2/main}
\include{chapters/5Experiments/3/main}
\include{chapters/5Experiments/4/main}

\include{chapters/6Conclusion/main}

\appendix
\include{appendix/AppendixB}
\include{appendix/D/main}
\include{appendix/AppendixC}

\backmatter
\bibliographystyle{ecs}
\bibliography{ECS}
\end{document}
%% ----------------------------------------------------------------

\include{chapters/3Design/HighLevel}
\include{chapters/3Design/InDepth}
 %% ----------------------------------------------------------------
%% Progress.tex
%% ---------------------------------------------------------------- 
\documentclass{ecsprogress}    % Use the progress Style
\graphicspath{{../figs/}}   % Location of your graphics files
    \usepackage{natbib}            % Use Natbib style for the refs.
\hypersetup{colorlinks=true}   % Set to false for black/white printing
\input{Definitions}            % Include your abbreviations



\usepackage{enumitem}% http://ctan.org/pkg/enumitem
\usepackage{multirow}
\usepackage{float}
\usepackage{amsmath}
\usepackage{multicol}
\usepackage{amssymb}
\usepackage[normalem]{ulem}
\useunder{\uline}{\ul}{}
\usepackage{wrapfig}


\usepackage[table,xcdraw]{xcolor}


%% ----------------------------------------------------------------
\begin{document}
\frontmatter
\title      {Heterogeneous Agent-based Model for Supermarket Competition}
\authors    {\texorpdfstring
             {\href{mailto:sc22g13@ecs.soton.ac.uk}{Stefan J. Collier}}
             {Stefan J. Collier}
            }
\addresses  {\groupname\\\deptname\\\univname}
\date       {\today}
\subject    {}
\keywords   {}
\supervisor {Dr. Maria Polukarov}
\examiner   {Professor Sheng Chen}

\maketitle
\begin{abstract}
This project aim was to model and analyse the effects of competitive pricing behaviors of grocery retailers on the British market. 

This was achieved by creating a multi-agent model, containing retailer and consumer agents. The heterogeneous crowd of retailers employs either a uniform pricing strategy or a ‘local price flexing’ strategy. The actions of these retailers are chosen by predicting the profit of each action, using a perceptron. Following on from the consideration of different economic models, a discrete model was developed so that software agents have a discrete environment to operate within. Within the model, it has been observed how supermarkets with differing behaviors affect a heterogeneous crowd of consumer agents. The model was implemented in Java with Python used to evaluate the results. 

The simulation displays good acceptance with real grocery market behavior, i.e. captures the performance of British retailers thus can be used to determine the impact of changes in their behavior on their competitors and consumers.Furthermore it can be used to provide insight into sustainability of volatile pricing strategies, providing a useful insight in volatility of British supermarket retail industry. 
\end{abstract}
\acknowledgements{
I would like to express my sincere gratitude to Dr Maria Polukarov for her guidance and support which provided me the freedom to take this research in the direction of my interest.\\
\\
I would also like to thank my family and friends for their encouragement and support. To those who quietly listened to my software complaints. To those who worked throughout the nights with me. To those who helped me write what I couldn't say. I cannot thank you enough.
}

\declaration{
I, Stefan Collier, declare that this dissertation and the work presented in it are my own and has been generated by me as the result of my own original research.\\
I confirm that:\\
1. This work was done wholly or mainly while in candidature for a degree at this University;\\
2. Where any part of this dissertation has previously been submitted for any other qualification at this University or any other institution, this has been clearly stated;\\
3. Where I have consulted the published work of others, this is always clearly attributed;\\
4. Where I have quoted from the work of others, the source is always given. With the exception of such quotations, this dissertation is entirely my own work;\\
5. I have acknowledged all main sources of help;\\
6. Where the thesis is based on work done by myself jointly with others, I have made clear exactly what was done by others and what I have contributed myself;\\
7. Either none of this work has been published before submission, or parts of this work have been published by :\\
\\
Stefan Collier\\
April 2016
}
\tableofcontents
\listoffigures
\listoftables

\mainmatter
%% ----------------------------------------------------------------
%\include{Introduction}
%\include{Conclusions}
\include{chapters/1Project/main}
\include{chapters/2Lit/main}
\include{chapters/3Design/HighLevel}
\include{chapters/3Design/InDepth}
\include{chapters/4Impl/main}

\include{chapters/5Experiments/1/main}
\include{chapters/5Experiments/2/main}
\include{chapters/5Experiments/3/main}
\include{chapters/5Experiments/4/main}

\include{chapters/6Conclusion/main}

\appendix
\include{appendix/AppendixB}
\include{appendix/D/main}
\include{appendix/AppendixC}

\backmatter
\bibliographystyle{ecs}
\bibliography{ECS}
\end{document}
%% ----------------------------------------------------------------


 %% ----------------------------------------------------------------
%% Progress.tex
%% ---------------------------------------------------------------- 
\documentclass{ecsprogress}    % Use the progress Style
\graphicspath{{../figs/}}   % Location of your graphics files
    \usepackage{natbib}            % Use Natbib style for the refs.
\hypersetup{colorlinks=true}   % Set to false for black/white printing
\input{Definitions}            % Include your abbreviations



\usepackage{enumitem}% http://ctan.org/pkg/enumitem
\usepackage{multirow}
\usepackage{float}
\usepackage{amsmath}
\usepackage{multicol}
\usepackage{amssymb}
\usepackage[normalem]{ulem}
\useunder{\uline}{\ul}{}
\usepackage{wrapfig}


\usepackage[table,xcdraw]{xcolor}


%% ----------------------------------------------------------------
\begin{document}
\frontmatter
\title      {Heterogeneous Agent-based Model for Supermarket Competition}
\authors    {\texorpdfstring
             {\href{mailto:sc22g13@ecs.soton.ac.uk}{Stefan J. Collier}}
             {Stefan J. Collier}
            }
\addresses  {\groupname\\\deptname\\\univname}
\date       {\today}
\subject    {}
\keywords   {}
\supervisor {Dr. Maria Polukarov}
\examiner   {Professor Sheng Chen}

\maketitle
\begin{abstract}
This project aim was to model and analyse the effects of competitive pricing behaviors of grocery retailers on the British market. 

This was achieved by creating a multi-agent model, containing retailer and consumer agents. The heterogeneous crowd of retailers employs either a uniform pricing strategy or a ‘local price flexing’ strategy. The actions of these retailers are chosen by predicting the profit of each action, using a perceptron. Following on from the consideration of different economic models, a discrete model was developed so that software agents have a discrete environment to operate within. Within the model, it has been observed how supermarkets with differing behaviors affect a heterogeneous crowd of consumer agents. The model was implemented in Java with Python used to evaluate the results. 

The simulation displays good acceptance with real grocery market behavior, i.e. captures the performance of British retailers thus can be used to determine the impact of changes in their behavior on their competitors and consumers.Furthermore it can be used to provide insight into sustainability of volatile pricing strategies, providing a useful insight in volatility of British supermarket retail industry. 
\end{abstract}
\acknowledgements{
I would like to express my sincere gratitude to Dr Maria Polukarov for her guidance and support which provided me the freedom to take this research in the direction of my interest.\\
\\
I would also like to thank my family and friends for their encouragement and support. To those who quietly listened to my software complaints. To those who worked throughout the nights with me. To those who helped me write what I couldn't say. I cannot thank you enough.
}

\declaration{
I, Stefan Collier, declare that this dissertation and the work presented in it are my own and has been generated by me as the result of my own original research.\\
I confirm that:\\
1. This work was done wholly or mainly while in candidature for a degree at this University;\\
2. Where any part of this dissertation has previously been submitted for any other qualification at this University or any other institution, this has been clearly stated;\\
3. Where I have consulted the published work of others, this is always clearly attributed;\\
4. Where I have quoted from the work of others, the source is always given. With the exception of such quotations, this dissertation is entirely my own work;\\
5. I have acknowledged all main sources of help;\\
6. Where the thesis is based on work done by myself jointly with others, I have made clear exactly what was done by others and what I have contributed myself;\\
7. Either none of this work has been published before submission, or parts of this work have been published by :\\
\\
Stefan Collier\\
April 2016
}
\tableofcontents
\listoffigures
\listoftables

\mainmatter
%% ----------------------------------------------------------------
%\include{Introduction}
%\include{Conclusions}
\include{chapters/1Project/main}
\include{chapters/2Lit/main}
\include{chapters/3Design/HighLevel}
\include{chapters/3Design/InDepth}
\include{chapters/4Impl/main}

\include{chapters/5Experiments/1/main}
\include{chapters/5Experiments/2/main}
\include{chapters/5Experiments/3/main}
\include{chapters/5Experiments/4/main}

\include{chapters/6Conclusion/main}

\appendix
\include{appendix/AppendixB}
\include{appendix/D/main}
\include{appendix/AppendixC}

\backmatter
\bibliographystyle{ecs}
\bibliography{ECS}
\end{document}
%% ----------------------------------------------------------------

 %% ----------------------------------------------------------------
%% Progress.tex
%% ---------------------------------------------------------------- 
\documentclass{ecsprogress}    % Use the progress Style
\graphicspath{{../figs/}}   % Location of your graphics files
    \usepackage{natbib}            % Use Natbib style for the refs.
\hypersetup{colorlinks=true}   % Set to false for black/white printing
\input{Definitions}            % Include your abbreviations



\usepackage{enumitem}% http://ctan.org/pkg/enumitem
\usepackage{multirow}
\usepackage{float}
\usepackage{amsmath}
\usepackage{multicol}
\usepackage{amssymb}
\usepackage[normalem]{ulem}
\useunder{\uline}{\ul}{}
\usepackage{wrapfig}


\usepackage[table,xcdraw]{xcolor}


%% ----------------------------------------------------------------
\begin{document}
\frontmatter
\title      {Heterogeneous Agent-based Model for Supermarket Competition}
\authors    {\texorpdfstring
             {\href{mailto:sc22g13@ecs.soton.ac.uk}{Stefan J. Collier}}
             {Stefan J. Collier}
            }
\addresses  {\groupname\\\deptname\\\univname}
\date       {\today}
\subject    {}
\keywords   {}
\supervisor {Dr. Maria Polukarov}
\examiner   {Professor Sheng Chen}

\maketitle
\begin{abstract}
This project aim was to model and analyse the effects of competitive pricing behaviors of grocery retailers on the British market. 

This was achieved by creating a multi-agent model, containing retailer and consumer agents. The heterogeneous crowd of retailers employs either a uniform pricing strategy or a ‘local price flexing’ strategy. The actions of these retailers are chosen by predicting the profit of each action, using a perceptron. Following on from the consideration of different economic models, a discrete model was developed so that software agents have a discrete environment to operate within. Within the model, it has been observed how supermarkets with differing behaviors affect a heterogeneous crowd of consumer agents. The model was implemented in Java with Python used to evaluate the results. 

The simulation displays good acceptance with real grocery market behavior, i.e. captures the performance of British retailers thus can be used to determine the impact of changes in their behavior on their competitors and consumers.Furthermore it can be used to provide insight into sustainability of volatile pricing strategies, providing a useful insight in volatility of British supermarket retail industry. 
\end{abstract}
\acknowledgements{
I would like to express my sincere gratitude to Dr Maria Polukarov for her guidance and support which provided me the freedom to take this research in the direction of my interest.\\
\\
I would also like to thank my family and friends for their encouragement and support. To those who quietly listened to my software complaints. To those who worked throughout the nights with me. To those who helped me write what I couldn't say. I cannot thank you enough.
}

\declaration{
I, Stefan Collier, declare that this dissertation and the work presented in it are my own and has been generated by me as the result of my own original research.\\
I confirm that:\\
1. This work was done wholly or mainly while in candidature for a degree at this University;\\
2. Where any part of this dissertation has previously been submitted for any other qualification at this University or any other institution, this has been clearly stated;\\
3. Where I have consulted the published work of others, this is always clearly attributed;\\
4. Where I have quoted from the work of others, the source is always given. With the exception of such quotations, this dissertation is entirely my own work;\\
5. I have acknowledged all main sources of help;\\
6. Where the thesis is based on work done by myself jointly with others, I have made clear exactly what was done by others and what I have contributed myself;\\
7. Either none of this work has been published before submission, or parts of this work have been published by :\\
\\
Stefan Collier\\
April 2016
}
\tableofcontents
\listoffigures
\listoftables

\mainmatter
%% ----------------------------------------------------------------
%\include{Introduction}
%\include{Conclusions}
\include{chapters/1Project/main}
\include{chapters/2Lit/main}
\include{chapters/3Design/HighLevel}
\include{chapters/3Design/InDepth}
\include{chapters/4Impl/main}

\include{chapters/5Experiments/1/main}
\include{chapters/5Experiments/2/main}
\include{chapters/5Experiments/3/main}
\include{chapters/5Experiments/4/main}

\include{chapters/6Conclusion/main}

\appendix
\include{appendix/AppendixB}
\include{appendix/D/main}
\include{appendix/AppendixC}

\backmatter
\bibliographystyle{ecs}
\bibliography{ECS}
\end{document}
%% ----------------------------------------------------------------

 %% ----------------------------------------------------------------
%% Progress.tex
%% ---------------------------------------------------------------- 
\documentclass{ecsprogress}    % Use the progress Style
\graphicspath{{../figs/}}   % Location of your graphics files
    \usepackage{natbib}            % Use Natbib style for the refs.
\hypersetup{colorlinks=true}   % Set to false for black/white printing
\input{Definitions}            % Include your abbreviations



\usepackage{enumitem}% http://ctan.org/pkg/enumitem
\usepackage{multirow}
\usepackage{float}
\usepackage{amsmath}
\usepackage{multicol}
\usepackage{amssymb}
\usepackage[normalem]{ulem}
\useunder{\uline}{\ul}{}
\usepackage{wrapfig}


\usepackage[table,xcdraw]{xcolor}


%% ----------------------------------------------------------------
\begin{document}
\frontmatter
\title      {Heterogeneous Agent-based Model for Supermarket Competition}
\authors    {\texorpdfstring
             {\href{mailto:sc22g13@ecs.soton.ac.uk}{Stefan J. Collier}}
             {Stefan J. Collier}
            }
\addresses  {\groupname\\\deptname\\\univname}
\date       {\today}
\subject    {}
\keywords   {}
\supervisor {Dr. Maria Polukarov}
\examiner   {Professor Sheng Chen}

\maketitle
\begin{abstract}
This project aim was to model and analyse the effects of competitive pricing behaviors of grocery retailers on the British market. 

This was achieved by creating a multi-agent model, containing retailer and consumer agents. The heterogeneous crowd of retailers employs either a uniform pricing strategy or a ‘local price flexing’ strategy. The actions of these retailers are chosen by predicting the profit of each action, using a perceptron. Following on from the consideration of different economic models, a discrete model was developed so that software agents have a discrete environment to operate within. Within the model, it has been observed how supermarkets with differing behaviors affect a heterogeneous crowd of consumer agents. The model was implemented in Java with Python used to evaluate the results. 

The simulation displays good acceptance with real grocery market behavior, i.e. captures the performance of British retailers thus can be used to determine the impact of changes in their behavior on their competitors and consumers.Furthermore it can be used to provide insight into sustainability of volatile pricing strategies, providing a useful insight in volatility of British supermarket retail industry. 
\end{abstract}
\acknowledgements{
I would like to express my sincere gratitude to Dr Maria Polukarov for her guidance and support which provided me the freedom to take this research in the direction of my interest.\\
\\
I would also like to thank my family and friends for their encouragement and support. To those who quietly listened to my software complaints. To those who worked throughout the nights with me. To those who helped me write what I couldn't say. I cannot thank you enough.
}

\declaration{
I, Stefan Collier, declare that this dissertation and the work presented in it are my own and has been generated by me as the result of my own original research.\\
I confirm that:\\
1. This work was done wholly or mainly while in candidature for a degree at this University;\\
2. Where any part of this dissertation has previously been submitted for any other qualification at this University or any other institution, this has been clearly stated;\\
3. Where I have consulted the published work of others, this is always clearly attributed;\\
4. Where I have quoted from the work of others, the source is always given. With the exception of such quotations, this dissertation is entirely my own work;\\
5. I have acknowledged all main sources of help;\\
6. Where the thesis is based on work done by myself jointly with others, I have made clear exactly what was done by others and what I have contributed myself;\\
7. Either none of this work has been published before submission, or parts of this work have been published by :\\
\\
Stefan Collier\\
April 2016
}
\tableofcontents
\listoffigures
\listoftables

\mainmatter
%% ----------------------------------------------------------------
%\include{Introduction}
%\include{Conclusions}
\include{chapters/1Project/main}
\include{chapters/2Lit/main}
\include{chapters/3Design/HighLevel}
\include{chapters/3Design/InDepth}
\include{chapters/4Impl/main}

\include{chapters/5Experiments/1/main}
\include{chapters/5Experiments/2/main}
\include{chapters/5Experiments/3/main}
\include{chapters/5Experiments/4/main}

\include{chapters/6Conclusion/main}

\appendix
\include{appendix/AppendixB}
\include{appendix/D/main}
\include{appendix/AppendixC}

\backmatter
\bibliographystyle{ecs}
\bibliography{ECS}
\end{document}
%% ----------------------------------------------------------------

 %% ----------------------------------------------------------------
%% Progress.tex
%% ---------------------------------------------------------------- 
\documentclass{ecsprogress}    % Use the progress Style
\graphicspath{{../figs/}}   % Location of your graphics files
    \usepackage{natbib}            % Use Natbib style for the refs.
\hypersetup{colorlinks=true}   % Set to false for black/white printing
\input{Definitions}            % Include your abbreviations



\usepackage{enumitem}% http://ctan.org/pkg/enumitem
\usepackage{multirow}
\usepackage{float}
\usepackage{amsmath}
\usepackage{multicol}
\usepackage{amssymb}
\usepackage[normalem]{ulem}
\useunder{\uline}{\ul}{}
\usepackage{wrapfig}


\usepackage[table,xcdraw]{xcolor}


%% ----------------------------------------------------------------
\begin{document}
\frontmatter
\title      {Heterogeneous Agent-based Model for Supermarket Competition}
\authors    {\texorpdfstring
             {\href{mailto:sc22g13@ecs.soton.ac.uk}{Stefan J. Collier}}
             {Stefan J. Collier}
            }
\addresses  {\groupname\\\deptname\\\univname}
\date       {\today}
\subject    {}
\keywords   {}
\supervisor {Dr. Maria Polukarov}
\examiner   {Professor Sheng Chen}

\maketitle
\begin{abstract}
This project aim was to model and analyse the effects of competitive pricing behaviors of grocery retailers on the British market. 

This was achieved by creating a multi-agent model, containing retailer and consumer agents. The heterogeneous crowd of retailers employs either a uniform pricing strategy or a ‘local price flexing’ strategy. The actions of these retailers are chosen by predicting the profit of each action, using a perceptron. Following on from the consideration of different economic models, a discrete model was developed so that software agents have a discrete environment to operate within. Within the model, it has been observed how supermarkets with differing behaviors affect a heterogeneous crowd of consumer agents. The model was implemented in Java with Python used to evaluate the results. 

The simulation displays good acceptance with real grocery market behavior, i.e. captures the performance of British retailers thus can be used to determine the impact of changes in their behavior on their competitors and consumers.Furthermore it can be used to provide insight into sustainability of volatile pricing strategies, providing a useful insight in volatility of British supermarket retail industry. 
\end{abstract}
\acknowledgements{
I would like to express my sincere gratitude to Dr Maria Polukarov for her guidance and support which provided me the freedom to take this research in the direction of my interest.\\
\\
I would also like to thank my family and friends for their encouragement and support. To those who quietly listened to my software complaints. To those who worked throughout the nights with me. To those who helped me write what I couldn't say. I cannot thank you enough.
}

\declaration{
I, Stefan Collier, declare that this dissertation and the work presented in it are my own and has been generated by me as the result of my own original research.\\
I confirm that:\\
1. This work was done wholly or mainly while in candidature for a degree at this University;\\
2. Where any part of this dissertation has previously been submitted for any other qualification at this University or any other institution, this has been clearly stated;\\
3. Where I have consulted the published work of others, this is always clearly attributed;\\
4. Where I have quoted from the work of others, the source is always given. With the exception of such quotations, this dissertation is entirely my own work;\\
5. I have acknowledged all main sources of help;\\
6. Where the thesis is based on work done by myself jointly with others, I have made clear exactly what was done by others and what I have contributed myself;\\
7. Either none of this work has been published before submission, or parts of this work have been published by :\\
\\
Stefan Collier\\
April 2016
}
\tableofcontents
\listoffigures
\listoftables

\mainmatter
%% ----------------------------------------------------------------
%\include{Introduction}
%\include{Conclusions}
\include{chapters/1Project/main}
\include{chapters/2Lit/main}
\include{chapters/3Design/HighLevel}
\include{chapters/3Design/InDepth}
\include{chapters/4Impl/main}

\include{chapters/5Experiments/1/main}
\include{chapters/5Experiments/2/main}
\include{chapters/5Experiments/3/main}
\include{chapters/5Experiments/4/main}

\include{chapters/6Conclusion/main}

\appendix
\include{appendix/AppendixB}
\include{appendix/D/main}
\include{appendix/AppendixC}

\backmatter
\bibliographystyle{ecs}
\bibliography{ECS}
\end{document}
%% ----------------------------------------------------------------


 %% ----------------------------------------------------------------
%% Progress.tex
%% ---------------------------------------------------------------- 
\documentclass{ecsprogress}    % Use the progress Style
\graphicspath{{../figs/}}   % Location of your graphics files
    \usepackage{natbib}            % Use Natbib style for the refs.
\hypersetup{colorlinks=true}   % Set to false for black/white printing
\input{Definitions}            % Include your abbreviations



\usepackage{enumitem}% http://ctan.org/pkg/enumitem
\usepackage{multirow}
\usepackage{float}
\usepackage{amsmath}
\usepackage{multicol}
\usepackage{amssymb}
\usepackage[normalem]{ulem}
\useunder{\uline}{\ul}{}
\usepackage{wrapfig}


\usepackage[table,xcdraw]{xcolor}


%% ----------------------------------------------------------------
\begin{document}
\frontmatter
\title      {Heterogeneous Agent-based Model for Supermarket Competition}
\authors    {\texorpdfstring
             {\href{mailto:sc22g13@ecs.soton.ac.uk}{Stefan J. Collier}}
             {Stefan J. Collier}
            }
\addresses  {\groupname\\\deptname\\\univname}
\date       {\today}
\subject    {}
\keywords   {}
\supervisor {Dr. Maria Polukarov}
\examiner   {Professor Sheng Chen}

\maketitle
\begin{abstract}
This project aim was to model and analyse the effects of competitive pricing behaviors of grocery retailers on the British market. 

This was achieved by creating a multi-agent model, containing retailer and consumer agents. The heterogeneous crowd of retailers employs either a uniform pricing strategy or a ‘local price flexing’ strategy. The actions of these retailers are chosen by predicting the profit of each action, using a perceptron. Following on from the consideration of different economic models, a discrete model was developed so that software agents have a discrete environment to operate within. Within the model, it has been observed how supermarkets with differing behaviors affect a heterogeneous crowd of consumer agents. The model was implemented in Java with Python used to evaluate the results. 

The simulation displays good acceptance with real grocery market behavior, i.e. captures the performance of British retailers thus can be used to determine the impact of changes in their behavior on their competitors and consumers.Furthermore it can be used to provide insight into sustainability of volatile pricing strategies, providing a useful insight in volatility of British supermarket retail industry. 
\end{abstract}
\acknowledgements{
I would like to express my sincere gratitude to Dr Maria Polukarov for her guidance and support which provided me the freedom to take this research in the direction of my interest.\\
\\
I would also like to thank my family and friends for their encouragement and support. To those who quietly listened to my software complaints. To those who worked throughout the nights with me. To those who helped me write what I couldn't say. I cannot thank you enough.
}

\declaration{
I, Stefan Collier, declare that this dissertation and the work presented in it are my own and has been generated by me as the result of my own original research.\\
I confirm that:\\
1. This work was done wholly or mainly while in candidature for a degree at this University;\\
2. Where any part of this dissertation has previously been submitted for any other qualification at this University or any other institution, this has been clearly stated;\\
3. Where I have consulted the published work of others, this is always clearly attributed;\\
4. Where I have quoted from the work of others, the source is always given. With the exception of such quotations, this dissertation is entirely my own work;\\
5. I have acknowledged all main sources of help;\\
6. Where the thesis is based on work done by myself jointly with others, I have made clear exactly what was done by others and what I have contributed myself;\\
7. Either none of this work has been published before submission, or parts of this work have been published by :\\
\\
Stefan Collier\\
April 2016
}
\tableofcontents
\listoffigures
\listoftables

\mainmatter
%% ----------------------------------------------------------------
%\include{Introduction}
%\include{Conclusions}
\include{chapters/1Project/main}
\include{chapters/2Lit/main}
\include{chapters/3Design/HighLevel}
\include{chapters/3Design/InDepth}
\include{chapters/4Impl/main}

\include{chapters/5Experiments/1/main}
\include{chapters/5Experiments/2/main}
\include{chapters/5Experiments/3/main}
\include{chapters/5Experiments/4/main}

\include{chapters/6Conclusion/main}

\appendix
\include{appendix/AppendixB}
\include{appendix/D/main}
\include{appendix/AppendixC}

\backmatter
\bibliographystyle{ecs}
\bibliography{ECS}
\end{document}
%% ----------------------------------------------------------------


\appendix
\include{appendix/AppendixB}
 %% ----------------------------------------------------------------
%% Progress.tex
%% ---------------------------------------------------------------- 
\documentclass{ecsprogress}    % Use the progress Style
\graphicspath{{../figs/}}   % Location of your graphics files
    \usepackage{natbib}            % Use Natbib style for the refs.
\hypersetup{colorlinks=true}   % Set to false for black/white printing
\input{Definitions}            % Include your abbreviations



\usepackage{enumitem}% http://ctan.org/pkg/enumitem
\usepackage{multirow}
\usepackage{float}
\usepackage{amsmath}
\usepackage{multicol}
\usepackage{amssymb}
\usepackage[normalem]{ulem}
\useunder{\uline}{\ul}{}
\usepackage{wrapfig}


\usepackage[table,xcdraw]{xcolor}


%% ----------------------------------------------------------------
\begin{document}
\frontmatter
\title      {Heterogeneous Agent-based Model for Supermarket Competition}
\authors    {\texorpdfstring
             {\href{mailto:sc22g13@ecs.soton.ac.uk}{Stefan J. Collier}}
             {Stefan J. Collier}
            }
\addresses  {\groupname\\\deptname\\\univname}
\date       {\today}
\subject    {}
\keywords   {}
\supervisor {Dr. Maria Polukarov}
\examiner   {Professor Sheng Chen}

\maketitle
\begin{abstract}
This project aim was to model and analyse the effects of competitive pricing behaviors of grocery retailers on the British market. 

This was achieved by creating a multi-agent model, containing retailer and consumer agents. The heterogeneous crowd of retailers employs either a uniform pricing strategy or a ‘local price flexing’ strategy. The actions of these retailers are chosen by predicting the profit of each action, using a perceptron. Following on from the consideration of different economic models, a discrete model was developed so that software agents have a discrete environment to operate within. Within the model, it has been observed how supermarkets with differing behaviors affect a heterogeneous crowd of consumer agents. The model was implemented in Java with Python used to evaluate the results. 

The simulation displays good acceptance with real grocery market behavior, i.e. captures the performance of British retailers thus can be used to determine the impact of changes in their behavior on their competitors and consumers.Furthermore it can be used to provide insight into sustainability of volatile pricing strategies, providing a useful insight in volatility of British supermarket retail industry. 
\end{abstract}
\acknowledgements{
I would like to express my sincere gratitude to Dr Maria Polukarov for her guidance and support which provided me the freedom to take this research in the direction of my interest.\\
\\
I would also like to thank my family and friends for their encouragement and support. To those who quietly listened to my software complaints. To those who worked throughout the nights with me. To those who helped me write what I couldn't say. I cannot thank you enough.
}

\declaration{
I, Stefan Collier, declare that this dissertation and the work presented in it are my own and has been generated by me as the result of my own original research.\\
I confirm that:\\
1. This work was done wholly or mainly while in candidature for a degree at this University;\\
2. Where any part of this dissertation has previously been submitted for any other qualification at this University or any other institution, this has been clearly stated;\\
3. Where I have consulted the published work of others, this is always clearly attributed;\\
4. Where I have quoted from the work of others, the source is always given. With the exception of such quotations, this dissertation is entirely my own work;\\
5. I have acknowledged all main sources of help;\\
6. Where the thesis is based on work done by myself jointly with others, I have made clear exactly what was done by others and what I have contributed myself;\\
7. Either none of this work has been published before submission, or parts of this work have been published by :\\
\\
Stefan Collier\\
April 2016
}
\tableofcontents
\listoffigures
\listoftables

\mainmatter
%% ----------------------------------------------------------------
%\include{Introduction}
%\include{Conclusions}
\include{chapters/1Project/main}
\include{chapters/2Lit/main}
\include{chapters/3Design/HighLevel}
\include{chapters/3Design/InDepth}
\include{chapters/4Impl/main}

\include{chapters/5Experiments/1/main}
\include{chapters/5Experiments/2/main}
\include{chapters/5Experiments/3/main}
\include{chapters/5Experiments/4/main}

\include{chapters/6Conclusion/main}

\appendix
\include{appendix/AppendixB}
\include{appendix/D/main}
\include{appendix/AppendixC}

\backmatter
\bibliographystyle{ecs}
\bibliography{ECS}
\end{document}
%% ----------------------------------------------------------------

\include{appendix/AppendixC}

\backmatter
\bibliographystyle{ecs}
\bibliography{ECS}
\end{document}
%% ----------------------------------------------------------------

 %% ----------------------------------------------------------------
%% Progress.tex
%% ---------------------------------------------------------------- 
\documentclass{ecsprogress}    % Use the progress Style
\graphicspath{{../figs/}}   % Location of your graphics files
    \usepackage{natbib}            % Use Natbib style for the refs.
\hypersetup{colorlinks=true}   % Set to false for black/white printing
\input{Definitions}            % Include your abbreviations



\usepackage{enumitem}% http://ctan.org/pkg/enumitem
\usepackage{multirow}
\usepackage{float}
\usepackage{amsmath}
\usepackage{multicol}
\usepackage{amssymb}
\usepackage[normalem]{ulem}
\useunder{\uline}{\ul}{}
\usepackage{wrapfig}


\usepackage[table,xcdraw]{xcolor}


%% ----------------------------------------------------------------
\begin{document}
\frontmatter
\title      {Heterogeneous Agent-based Model for Supermarket Competition}
\authors    {\texorpdfstring
             {\href{mailto:sc22g13@ecs.soton.ac.uk}{Stefan J. Collier}}
             {Stefan J. Collier}
            }
\addresses  {\groupname\\\deptname\\\univname}
\date       {\today}
\subject    {}
\keywords   {}
\supervisor {Dr. Maria Polukarov}
\examiner   {Professor Sheng Chen}

\maketitle
\begin{abstract}
This project aim was to model and analyse the effects of competitive pricing behaviors of grocery retailers on the British market. 

This was achieved by creating a multi-agent model, containing retailer and consumer agents. The heterogeneous crowd of retailers employs either a uniform pricing strategy or a ‘local price flexing’ strategy. The actions of these retailers are chosen by predicting the profit of each action, using a perceptron. Following on from the consideration of different economic models, a discrete model was developed so that software agents have a discrete environment to operate within. Within the model, it has been observed how supermarkets with differing behaviors affect a heterogeneous crowd of consumer agents. The model was implemented in Java with Python used to evaluate the results. 

The simulation displays good acceptance with real grocery market behavior, i.e. captures the performance of British retailers thus can be used to determine the impact of changes in their behavior on their competitors and consumers.Furthermore it can be used to provide insight into sustainability of volatile pricing strategies, providing a useful insight in volatility of British supermarket retail industry. 
\end{abstract}
\acknowledgements{
I would like to express my sincere gratitude to Dr Maria Polukarov for her guidance and support which provided me the freedom to take this research in the direction of my interest.\\
\\
I would also like to thank my family and friends for their encouragement and support. To those who quietly listened to my software complaints. To those who worked throughout the nights with me. To those who helped me write what I couldn't say. I cannot thank you enough.
}

\declaration{
I, Stefan Collier, declare that this dissertation and the work presented in it are my own and has been generated by me as the result of my own original research.\\
I confirm that:\\
1. This work was done wholly or mainly while in candidature for a degree at this University;\\
2. Where any part of this dissertation has previously been submitted for any other qualification at this University or any other institution, this has been clearly stated;\\
3. Where I have consulted the published work of others, this is always clearly attributed;\\
4. Where I have quoted from the work of others, the source is always given. With the exception of such quotations, this dissertation is entirely my own work;\\
5. I have acknowledged all main sources of help;\\
6. Where the thesis is based on work done by myself jointly with others, I have made clear exactly what was done by others and what I have contributed myself;\\
7. Either none of this work has been published before submission, or parts of this work have been published by :\\
\\
Stefan Collier\\
April 2016
}
\tableofcontents
\listoffigures
\listoftables

\mainmatter
%% ----------------------------------------------------------------
%\include{Introduction}
%\include{Conclusions}
 %% ----------------------------------------------------------------
%% Progress.tex
%% ---------------------------------------------------------------- 
\documentclass{ecsprogress}    % Use the progress Style
\graphicspath{{../figs/}}   % Location of your graphics files
    \usepackage{natbib}            % Use Natbib style for the refs.
\hypersetup{colorlinks=true}   % Set to false for black/white printing
\input{Definitions}            % Include your abbreviations



\usepackage{enumitem}% http://ctan.org/pkg/enumitem
\usepackage{multirow}
\usepackage{float}
\usepackage{amsmath}
\usepackage{multicol}
\usepackage{amssymb}
\usepackage[normalem]{ulem}
\useunder{\uline}{\ul}{}
\usepackage{wrapfig}


\usepackage[table,xcdraw]{xcolor}


%% ----------------------------------------------------------------
\begin{document}
\frontmatter
\title      {Heterogeneous Agent-based Model for Supermarket Competition}
\authors    {\texorpdfstring
             {\href{mailto:sc22g13@ecs.soton.ac.uk}{Stefan J. Collier}}
             {Stefan J. Collier}
            }
\addresses  {\groupname\\\deptname\\\univname}
\date       {\today}
\subject    {}
\keywords   {}
\supervisor {Dr. Maria Polukarov}
\examiner   {Professor Sheng Chen}

\maketitle
\begin{abstract}
This project aim was to model and analyse the effects of competitive pricing behaviors of grocery retailers on the British market. 

This was achieved by creating a multi-agent model, containing retailer and consumer agents. The heterogeneous crowd of retailers employs either a uniform pricing strategy or a ‘local price flexing’ strategy. The actions of these retailers are chosen by predicting the profit of each action, using a perceptron. Following on from the consideration of different economic models, a discrete model was developed so that software agents have a discrete environment to operate within. Within the model, it has been observed how supermarkets with differing behaviors affect a heterogeneous crowd of consumer agents. The model was implemented in Java with Python used to evaluate the results. 

The simulation displays good acceptance with real grocery market behavior, i.e. captures the performance of British retailers thus can be used to determine the impact of changes in their behavior on their competitors and consumers.Furthermore it can be used to provide insight into sustainability of volatile pricing strategies, providing a useful insight in volatility of British supermarket retail industry. 
\end{abstract}
\acknowledgements{
I would like to express my sincere gratitude to Dr Maria Polukarov for her guidance and support which provided me the freedom to take this research in the direction of my interest.\\
\\
I would also like to thank my family and friends for their encouragement and support. To those who quietly listened to my software complaints. To those who worked throughout the nights with me. To those who helped me write what I couldn't say. I cannot thank you enough.
}

\declaration{
I, Stefan Collier, declare that this dissertation and the work presented in it are my own and has been generated by me as the result of my own original research.\\
I confirm that:\\
1. This work was done wholly or mainly while in candidature for a degree at this University;\\
2. Where any part of this dissertation has previously been submitted for any other qualification at this University or any other institution, this has been clearly stated;\\
3. Where I have consulted the published work of others, this is always clearly attributed;\\
4. Where I have quoted from the work of others, the source is always given. With the exception of such quotations, this dissertation is entirely my own work;\\
5. I have acknowledged all main sources of help;\\
6. Where the thesis is based on work done by myself jointly with others, I have made clear exactly what was done by others and what I have contributed myself;\\
7. Either none of this work has been published before submission, or parts of this work have been published by :\\
\\
Stefan Collier\\
April 2016
}
\tableofcontents
\listoffigures
\listoftables

\mainmatter
%% ----------------------------------------------------------------
%\include{Introduction}
%\include{Conclusions}
\include{chapters/1Project/main}
\include{chapters/2Lit/main}
\include{chapters/3Design/HighLevel}
\include{chapters/3Design/InDepth}
\include{chapters/4Impl/main}

\include{chapters/5Experiments/1/main}
\include{chapters/5Experiments/2/main}
\include{chapters/5Experiments/3/main}
\include{chapters/5Experiments/4/main}

\include{chapters/6Conclusion/main}

\appendix
\include{appendix/AppendixB}
\include{appendix/D/main}
\include{appendix/AppendixC}

\backmatter
\bibliographystyle{ecs}
\bibliography{ECS}
\end{document}
%% ----------------------------------------------------------------

 %% ----------------------------------------------------------------
%% Progress.tex
%% ---------------------------------------------------------------- 
\documentclass{ecsprogress}    % Use the progress Style
\graphicspath{{../figs/}}   % Location of your graphics files
    \usepackage{natbib}            % Use Natbib style for the refs.
\hypersetup{colorlinks=true}   % Set to false for black/white printing
\input{Definitions}            % Include your abbreviations



\usepackage{enumitem}% http://ctan.org/pkg/enumitem
\usepackage{multirow}
\usepackage{float}
\usepackage{amsmath}
\usepackage{multicol}
\usepackage{amssymb}
\usepackage[normalem]{ulem}
\useunder{\uline}{\ul}{}
\usepackage{wrapfig}


\usepackage[table,xcdraw]{xcolor}


%% ----------------------------------------------------------------
\begin{document}
\frontmatter
\title      {Heterogeneous Agent-based Model for Supermarket Competition}
\authors    {\texorpdfstring
             {\href{mailto:sc22g13@ecs.soton.ac.uk}{Stefan J. Collier}}
             {Stefan J. Collier}
            }
\addresses  {\groupname\\\deptname\\\univname}
\date       {\today}
\subject    {}
\keywords   {}
\supervisor {Dr. Maria Polukarov}
\examiner   {Professor Sheng Chen}

\maketitle
\begin{abstract}
This project aim was to model and analyse the effects of competitive pricing behaviors of grocery retailers on the British market. 

This was achieved by creating a multi-agent model, containing retailer and consumer agents. The heterogeneous crowd of retailers employs either a uniform pricing strategy or a ‘local price flexing’ strategy. The actions of these retailers are chosen by predicting the profit of each action, using a perceptron. Following on from the consideration of different economic models, a discrete model was developed so that software agents have a discrete environment to operate within. Within the model, it has been observed how supermarkets with differing behaviors affect a heterogeneous crowd of consumer agents. The model was implemented in Java with Python used to evaluate the results. 

The simulation displays good acceptance with real grocery market behavior, i.e. captures the performance of British retailers thus can be used to determine the impact of changes in their behavior on their competitors and consumers.Furthermore it can be used to provide insight into sustainability of volatile pricing strategies, providing a useful insight in volatility of British supermarket retail industry. 
\end{abstract}
\acknowledgements{
I would like to express my sincere gratitude to Dr Maria Polukarov for her guidance and support which provided me the freedom to take this research in the direction of my interest.\\
\\
I would also like to thank my family and friends for their encouragement and support. To those who quietly listened to my software complaints. To those who worked throughout the nights with me. To those who helped me write what I couldn't say. I cannot thank you enough.
}

\declaration{
I, Stefan Collier, declare that this dissertation and the work presented in it are my own and has been generated by me as the result of my own original research.\\
I confirm that:\\
1. This work was done wholly or mainly while in candidature for a degree at this University;\\
2. Where any part of this dissertation has previously been submitted for any other qualification at this University or any other institution, this has been clearly stated;\\
3. Where I have consulted the published work of others, this is always clearly attributed;\\
4. Where I have quoted from the work of others, the source is always given. With the exception of such quotations, this dissertation is entirely my own work;\\
5. I have acknowledged all main sources of help;\\
6. Where the thesis is based on work done by myself jointly with others, I have made clear exactly what was done by others and what I have contributed myself;\\
7. Either none of this work has been published before submission, or parts of this work have been published by :\\
\\
Stefan Collier\\
April 2016
}
\tableofcontents
\listoffigures
\listoftables

\mainmatter
%% ----------------------------------------------------------------
%\include{Introduction}
%\include{Conclusions}
\include{chapters/1Project/main}
\include{chapters/2Lit/main}
\include{chapters/3Design/HighLevel}
\include{chapters/3Design/InDepth}
\include{chapters/4Impl/main}

\include{chapters/5Experiments/1/main}
\include{chapters/5Experiments/2/main}
\include{chapters/5Experiments/3/main}
\include{chapters/5Experiments/4/main}

\include{chapters/6Conclusion/main}

\appendix
\include{appendix/AppendixB}
\include{appendix/D/main}
\include{appendix/AppendixC}

\backmatter
\bibliographystyle{ecs}
\bibliography{ECS}
\end{document}
%% ----------------------------------------------------------------

\include{chapters/3Design/HighLevel}
\include{chapters/3Design/InDepth}
 %% ----------------------------------------------------------------
%% Progress.tex
%% ---------------------------------------------------------------- 
\documentclass{ecsprogress}    % Use the progress Style
\graphicspath{{../figs/}}   % Location of your graphics files
    \usepackage{natbib}            % Use Natbib style for the refs.
\hypersetup{colorlinks=true}   % Set to false for black/white printing
\input{Definitions}            % Include your abbreviations



\usepackage{enumitem}% http://ctan.org/pkg/enumitem
\usepackage{multirow}
\usepackage{float}
\usepackage{amsmath}
\usepackage{multicol}
\usepackage{amssymb}
\usepackage[normalem]{ulem}
\useunder{\uline}{\ul}{}
\usepackage{wrapfig}


\usepackage[table,xcdraw]{xcolor}


%% ----------------------------------------------------------------
\begin{document}
\frontmatter
\title      {Heterogeneous Agent-based Model for Supermarket Competition}
\authors    {\texorpdfstring
             {\href{mailto:sc22g13@ecs.soton.ac.uk}{Stefan J. Collier}}
             {Stefan J. Collier}
            }
\addresses  {\groupname\\\deptname\\\univname}
\date       {\today}
\subject    {}
\keywords   {}
\supervisor {Dr. Maria Polukarov}
\examiner   {Professor Sheng Chen}

\maketitle
\begin{abstract}
This project aim was to model and analyse the effects of competitive pricing behaviors of grocery retailers on the British market. 

This was achieved by creating a multi-agent model, containing retailer and consumer agents. The heterogeneous crowd of retailers employs either a uniform pricing strategy or a ‘local price flexing’ strategy. The actions of these retailers are chosen by predicting the profit of each action, using a perceptron. Following on from the consideration of different economic models, a discrete model was developed so that software agents have a discrete environment to operate within. Within the model, it has been observed how supermarkets with differing behaviors affect a heterogeneous crowd of consumer agents. The model was implemented in Java with Python used to evaluate the results. 

The simulation displays good acceptance with real grocery market behavior, i.e. captures the performance of British retailers thus can be used to determine the impact of changes in their behavior on their competitors and consumers.Furthermore it can be used to provide insight into sustainability of volatile pricing strategies, providing a useful insight in volatility of British supermarket retail industry. 
\end{abstract}
\acknowledgements{
I would like to express my sincere gratitude to Dr Maria Polukarov for her guidance and support which provided me the freedom to take this research in the direction of my interest.\\
\\
I would also like to thank my family and friends for their encouragement and support. To those who quietly listened to my software complaints. To those who worked throughout the nights with me. To those who helped me write what I couldn't say. I cannot thank you enough.
}

\declaration{
I, Stefan Collier, declare that this dissertation and the work presented in it are my own and has been generated by me as the result of my own original research.\\
I confirm that:\\
1. This work was done wholly or mainly while in candidature for a degree at this University;\\
2. Where any part of this dissertation has previously been submitted for any other qualification at this University or any other institution, this has been clearly stated;\\
3. Where I have consulted the published work of others, this is always clearly attributed;\\
4. Where I have quoted from the work of others, the source is always given. With the exception of such quotations, this dissertation is entirely my own work;\\
5. I have acknowledged all main sources of help;\\
6. Where the thesis is based on work done by myself jointly with others, I have made clear exactly what was done by others and what I have contributed myself;\\
7. Either none of this work has been published before submission, or parts of this work have been published by :\\
\\
Stefan Collier\\
April 2016
}
\tableofcontents
\listoffigures
\listoftables

\mainmatter
%% ----------------------------------------------------------------
%\include{Introduction}
%\include{Conclusions}
\include{chapters/1Project/main}
\include{chapters/2Lit/main}
\include{chapters/3Design/HighLevel}
\include{chapters/3Design/InDepth}
\include{chapters/4Impl/main}

\include{chapters/5Experiments/1/main}
\include{chapters/5Experiments/2/main}
\include{chapters/5Experiments/3/main}
\include{chapters/5Experiments/4/main}

\include{chapters/6Conclusion/main}

\appendix
\include{appendix/AppendixB}
\include{appendix/D/main}
\include{appendix/AppendixC}

\backmatter
\bibliographystyle{ecs}
\bibliography{ECS}
\end{document}
%% ----------------------------------------------------------------


 %% ----------------------------------------------------------------
%% Progress.tex
%% ---------------------------------------------------------------- 
\documentclass{ecsprogress}    % Use the progress Style
\graphicspath{{../figs/}}   % Location of your graphics files
    \usepackage{natbib}            % Use Natbib style for the refs.
\hypersetup{colorlinks=true}   % Set to false for black/white printing
\input{Definitions}            % Include your abbreviations



\usepackage{enumitem}% http://ctan.org/pkg/enumitem
\usepackage{multirow}
\usepackage{float}
\usepackage{amsmath}
\usepackage{multicol}
\usepackage{amssymb}
\usepackage[normalem]{ulem}
\useunder{\uline}{\ul}{}
\usepackage{wrapfig}


\usepackage[table,xcdraw]{xcolor}


%% ----------------------------------------------------------------
\begin{document}
\frontmatter
\title      {Heterogeneous Agent-based Model for Supermarket Competition}
\authors    {\texorpdfstring
             {\href{mailto:sc22g13@ecs.soton.ac.uk}{Stefan J. Collier}}
             {Stefan J. Collier}
            }
\addresses  {\groupname\\\deptname\\\univname}
\date       {\today}
\subject    {}
\keywords   {}
\supervisor {Dr. Maria Polukarov}
\examiner   {Professor Sheng Chen}

\maketitle
\begin{abstract}
This project aim was to model and analyse the effects of competitive pricing behaviors of grocery retailers on the British market. 

This was achieved by creating a multi-agent model, containing retailer and consumer agents. The heterogeneous crowd of retailers employs either a uniform pricing strategy or a ‘local price flexing’ strategy. The actions of these retailers are chosen by predicting the profit of each action, using a perceptron. Following on from the consideration of different economic models, a discrete model was developed so that software agents have a discrete environment to operate within. Within the model, it has been observed how supermarkets with differing behaviors affect a heterogeneous crowd of consumer agents. The model was implemented in Java with Python used to evaluate the results. 

The simulation displays good acceptance with real grocery market behavior, i.e. captures the performance of British retailers thus can be used to determine the impact of changes in their behavior on their competitors and consumers.Furthermore it can be used to provide insight into sustainability of volatile pricing strategies, providing a useful insight in volatility of British supermarket retail industry. 
\end{abstract}
\acknowledgements{
I would like to express my sincere gratitude to Dr Maria Polukarov for her guidance and support which provided me the freedom to take this research in the direction of my interest.\\
\\
I would also like to thank my family and friends for their encouragement and support. To those who quietly listened to my software complaints. To those who worked throughout the nights with me. To those who helped me write what I couldn't say. I cannot thank you enough.
}

\declaration{
I, Stefan Collier, declare that this dissertation and the work presented in it are my own and has been generated by me as the result of my own original research.\\
I confirm that:\\
1. This work was done wholly or mainly while in candidature for a degree at this University;\\
2. Where any part of this dissertation has previously been submitted for any other qualification at this University or any other institution, this has been clearly stated;\\
3. Where I have consulted the published work of others, this is always clearly attributed;\\
4. Where I have quoted from the work of others, the source is always given. With the exception of such quotations, this dissertation is entirely my own work;\\
5. I have acknowledged all main sources of help;\\
6. Where the thesis is based on work done by myself jointly with others, I have made clear exactly what was done by others and what I have contributed myself;\\
7. Either none of this work has been published before submission, or parts of this work have been published by :\\
\\
Stefan Collier\\
April 2016
}
\tableofcontents
\listoffigures
\listoftables

\mainmatter
%% ----------------------------------------------------------------
%\include{Introduction}
%\include{Conclusions}
\include{chapters/1Project/main}
\include{chapters/2Lit/main}
\include{chapters/3Design/HighLevel}
\include{chapters/3Design/InDepth}
\include{chapters/4Impl/main}

\include{chapters/5Experiments/1/main}
\include{chapters/5Experiments/2/main}
\include{chapters/5Experiments/3/main}
\include{chapters/5Experiments/4/main}

\include{chapters/6Conclusion/main}

\appendix
\include{appendix/AppendixB}
\include{appendix/D/main}
\include{appendix/AppendixC}

\backmatter
\bibliographystyle{ecs}
\bibliography{ECS}
\end{document}
%% ----------------------------------------------------------------

 %% ----------------------------------------------------------------
%% Progress.tex
%% ---------------------------------------------------------------- 
\documentclass{ecsprogress}    % Use the progress Style
\graphicspath{{../figs/}}   % Location of your graphics files
    \usepackage{natbib}            % Use Natbib style for the refs.
\hypersetup{colorlinks=true}   % Set to false for black/white printing
\input{Definitions}            % Include your abbreviations



\usepackage{enumitem}% http://ctan.org/pkg/enumitem
\usepackage{multirow}
\usepackage{float}
\usepackage{amsmath}
\usepackage{multicol}
\usepackage{amssymb}
\usepackage[normalem]{ulem}
\useunder{\uline}{\ul}{}
\usepackage{wrapfig}


\usepackage[table,xcdraw]{xcolor}


%% ----------------------------------------------------------------
\begin{document}
\frontmatter
\title      {Heterogeneous Agent-based Model for Supermarket Competition}
\authors    {\texorpdfstring
             {\href{mailto:sc22g13@ecs.soton.ac.uk}{Stefan J. Collier}}
             {Stefan J. Collier}
            }
\addresses  {\groupname\\\deptname\\\univname}
\date       {\today}
\subject    {}
\keywords   {}
\supervisor {Dr. Maria Polukarov}
\examiner   {Professor Sheng Chen}

\maketitle
\begin{abstract}
This project aim was to model and analyse the effects of competitive pricing behaviors of grocery retailers on the British market. 

This was achieved by creating a multi-agent model, containing retailer and consumer agents. The heterogeneous crowd of retailers employs either a uniform pricing strategy or a ‘local price flexing’ strategy. The actions of these retailers are chosen by predicting the profit of each action, using a perceptron. Following on from the consideration of different economic models, a discrete model was developed so that software agents have a discrete environment to operate within. Within the model, it has been observed how supermarkets with differing behaviors affect a heterogeneous crowd of consumer agents. The model was implemented in Java with Python used to evaluate the results. 

The simulation displays good acceptance with real grocery market behavior, i.e. captures the performance of British retailers thus can be used to determine the impact of changes in their behavior on their competitors and consumers.Furthermore it can be used to provide insight into sustainability of volatile pricing strategies, providing a useful insight in volatility of British supermarket retail industry. 
\end{abstract}
\acknowledgements{
I would like to express my sincere gratitude to Dr Maria Polukarov for her guidance and support which provided me the freedom to take this research in the direction of my interest.\\
\\
I would also like to thank my family and friends for their encouragement and support. To those who quietly listened to my software complaints. To those who worked throughout the nights with me. To those who helped me write what I couldn't say. I cannot thank you enough.
}

\declaration{
I, Stefan Collier, declare that this dissertation and the work presented in it are my own and has been generated by me as the result of my own original research.\\
I confirm that:\\
1. This work was done wholly or mainly while in candidature for a degree at this University;\\
2. Where any part of this dissertation has previously been submitted for any other qualification at this University or any other institution, this has been clearly stated;\\
3. Where I have consulted the published work of others, this is always clearly attributed;\\
4. Where I have quoted from the work of others, the source is always given. With the exception of such quotations, this dissertation is entirely my own work;\\
5. I have acknowledged all main sources of help;\\
6. Where the thesis is based on work done by myself jointly with others, I have made clear exactly what was done by others and what I have contributed myself;\\
7. Either none of this work has been published before submission, or parts of this work have been published by :\\
\\
Stefan Collier\\
April 2016
}
\tableofcontents
\listoffigures
\listoftables

\mainmatter
%% ----------------------------------------------------------------
%\include{Introduction}
%\include{Conclusions}
\include{chapters/1Project/main}
\include{chapters/2Lit/main}
\include{chapters/3Design/HighLevel}
\include{chapters/3Design/InDepth}
\include{chapters/4Impl/main}

\include{chapters/5Experiments/1/main}
\include{chapters/5Experiments/2/main}
\include{chapters/5Experiments/3/main}
\include{chapters/5Experiments/4/main}

\include{chapters/6Conclusion/main}

\appendix
\include{appendix/AppendixB}
\include{appendix/D/main}
\include{appendix/AppendixC}

\backmatter
\bibliographystyle{ecs}
\bibliography{ECS}
\end{document}
%% ----------------------------------------------------------------

 %% ----------------------------------------------------------------
%% Progress.tex
%% ---------------------------------------------------------------- 
\documentclass{ecsprogress}    % Use the progress Style
\graphicspath{{../figs/}}   % Location of your graphics files
    \usepackage{natbib}            % Use Natbib style for the refs.
\hypersetup{colorlinks=true}   % Set to false for black/white printing
\input{Definitions}            % Include your abbreviations



\usepackage{enumitem}% http://ctan.org/pkg/enumitem
\usepackage{multirow}
\usepackage{float}
\usepackage{amsmath}
\usepackage{multicol}
\usepackage{amssymb}
\usepackage[normalem]{ulem}
\useunder{\uline}{\ul}{}
\usepackage{wrapfig}


\usepackage[table,xcdraw]{xcolor}


%% ----------------------------------------------------------------
\begin{document}
\frontmatter
\title      {Heterogeneous Agent-based Model for Supermarket Competition}
\authors    {\texorpdfstring
             {\href{mailto:sc22g13@ecs.soton.ac.uk}{Stefan J. Collier}}
             {Stefan J. Collier}
            }
\addresses  {\groupname\\\deptname\\\univname}
\date       {\today}
\subject    {}
\keywords   {}
\supervisor {Dr. Maria Polukarov}
\examiner   {Professor Sheng Chen}

\maketitle
\begin{abstract}
This project aim was to model and analyse the effects of competitive pricing behaviors of grocery retailers on the British market. 

This was achieved by creating a multi-agent model, containing retailer and consumer agents. The heterogeneous crowd of retailers employs either a uniform pricing strategy or a ‘local price flexing’ strategy. The actions of these retailers are chosen by predicting the profit of each action, using a perceptron. Following on from the consideration of different economic models, a discrete model was developed so that software agents have a discrete environment to operate within. Within the model, it has been observed how supermarkets with differing behaviors affect a heterogeneous crowd of consumer agents. The model was implemented in Java with Python used to evaluate the results. 

The simulation displays good acceptance with real grocery market behavior, i.e. captures the performance of British retailers thus can be used to determine the impact of changes in their behavior on their competitors and consumers.Furthermore it can be used to provide insight into sustainability of volatile pricing strategies, providing a useful insight in volatility of British supermarket retail industry. 
\end{abstract}
\acknowledgements{
I would like to express my sincere gratitude to Dr Maria Polukarov for her guidance and support which provided me the freedom to take this research in the direction of my interest.\\
\\
I would also like to thank my family and friends for their encouragement and support. To those who quietly listened to my software complaints. To those who worked throughout the nights with me. To those who helped me write what I couldn't say. I cannot thank you enough.
}

\declaration{
I, Stefan Collier, declare that this dissertation and the work presented in it are my own and has been generated by me as the result of my own original research.\\
I confirm that:\\
1. This work was done wholly or mainly while in candidature for a degree at this University;\\
2. Where any part of this dissertation has previously been submitted for any other qualification at this University or any other institution, this has been clearly stated;\\
3. Where I have consulted the published work of others, this is always clearly attributed;\\
4. Where I have quoted from the work of others, the source is always given. With the exception of such quotations, this dissertation is entirely my own work;\\
5. I have acknowledged all main sources of help;\\
6. Where the thesis is based on work done by myself jointly with others, I have made clear exactly what was done by others and what I have contributed myself;\\
7. Either none of this work has been published before submission, or parts of this work have been published by :\\
\\
Stefan Collier\\
April 2016
}
\tableofcontents
\listoffigures
\listoftables

\mainmatter
%% ----------------------------------------------------------------
%\include{Introduction}
%\include{Conclusions}
\include{chapters/1Project/main}
\include{chapters/2Lit/main}
\include{chapters/3Design/HighLevel}
\include{chapters/3Design/InDepth}
\include{chapters/4Impl/main}

\include{chapters/5Experiments/1/main}
\include{chapters/5Experiments/2/main}
\include{chapters/5Experiments/3/main}
\include{chapters/5Experiments/4/main}

\include{chapters/6Conclusion/main}

\appendix
\include{appendix/AppendixB}
\include{appendix/D/main}
\include{appendix/AppendixC}

\backmatter
\bibliographystyle{ecs}
\bibliography{ECS}
\end{document}
%% ----------------------------------------------------------------

 %% ----------------------------------------------------------------
%% Progress.tex
%% ---------------------------------------------------------------- 
\documentclass{ecsprogress}    % Use the progress Style
\graphicspath{{../figs/}}   % Location of your graphics files
    \usepackage{natbib}            % Use Natbib style for the refs.
\hypersetup{colorlinks=true}   % Set to false for black/white printing
\input{Definitions}            % Include your abbreviations



\usepackage{enumitem}% http://ctan.org/pkg/enumitem
\usepackage{multirow}
\usepackage{float}
\usepackage{amsmath}
\usepackage{multicol}
\usepackage{amssymb}
\usepackage[normalem]{ulem}
\useunder{\uline}{\ul}{}
\usepackage{wrapfig}


\usepackage[table,xcdraw]{xcolor}


%% ----------------------------------------------------------------
\begin{document}
\frontmatter
\title      {Heterogeneous Agent-based Model for Supermarket Competition}
\authors    {\texorpdfstring
             {\href{mailto:sc22g13@ecs.soton.ac.uk}{Stefan J. Collier}}
             {Stefan J. Collier}
            }
\addresses  {\groupname\\\deptname\\\univname}
\date       {\today}
\subject    {}
\keywords   {}
\supervisor {Dr. Maria Polukarov}
\examiner   {Professor Sheng Chen}

\maketitle
\begin{abstract}
This project aim was to model and analyse the effects of competitive pricing behaviors of grocery retailers on the British market. 

This was achieved by creating a multi-agent model, containing retailer and consumer agents. The heterogeneous crowd of retailers employs either a uniform pricing strategy or a ‘local price flexing’ strategy. The actions of these retailers are chosen by predicting the profit of each action, using a perceptron. Following on from the consideration of different economic models, a discrete model was developed so that software agents have a discrete environment to operate within. Within the model, it has been observed how supermarkets with differing behaviors affect a heterogeneous crowd of consumer agents. The model was implemented in Java with Python used to evaluate the results. 

The simulation displays good acceptance with real grocery market behavior, i.e. captures the performance of British retailers thus can be used to determine the impact of changes in their behavior on their competitors and consumers.Furthermore it can be used to provide insight into sustainability of volatile pricing strategies, providing a useful insight in volatility of British supermarket retail industry. 
\end{abstract}
\acknowledgements{
I would like to express my sincere gratitude to Dr Maria Polukarov for her guidance and support which provided me the freedom to take this research in the direction of my interest.\\
\\
I would also like to thank my family and friends for their encouragement and support. To those who quietly listened to my software complaints. To those who worked throughout the nights with me. To those who helped me write what I couldn't say. I cannot thank you enough.
}

\declaration{
I, Stefan Collier, declare that this dissertation and the work presented in it are my own and has been generated by me as the result of my own original research.\\
I confirm that:\\
1. This work was done wholly or mainly while in candidature for a degree at this University;\\
2. Where any part of this dissertation has previously been submitted for any other qualification at this University or any other institution, this has been clearly stated;\\
3. Where I have consulted the published work of others, this is always clearly attributed;\\
4. Where I have quoted from the work of others, the source is always given. With the exception of such quotations, this dissertation is entirely my own work;\\
5. I have acknowledged all main sources of help;\\
6. Where the thesis is based on work done by myself jointly with others, I have made clear exactly what was done by others and what I have contributed myself;\\
7. Either none of this work has been published before submission, or parts of this work have been published by :\\
\\
Stefan Collier\\
April 2016
}
\tableofcontents
\listoffigures
\listoftables

\mainmatter
%% ----------------------------------------------------------------
%\include{Introduction}
%\include{Conclusions}
\include{chapters/1Project/main}
\include{chapters/2Lit/main}
\include{chapters/3Design/HighLevel}
\include{chapters/3Design/InDepth}
\include{chapters/4Impl/main}

\include{chapters/5Experiments/1/main}
\include{chapters/5Experiments/2/main}
\include{chapters/5Experiments/3/main}
\include{chapters/5Experiments/4/main}

\include{chapters/6Conclusion/main}

\appendix
\include{appendix/AppendixB}
\include{appendix/D/main}
\include{appendix/AppendixC}

\backmatter
\bibliographystyle{ecs}
\bibliography{ECS}
\end{document}
%% ----------------------------------------------------------------


 %% ----------------------------------------------------------------
%% Progress.tex
%% ---------------------------------------------------------------- 
\documentclass{ecsprogress}    % Use the progress Style
\graphicspath{{../figs/}}   % Location of your graphics files
    \usepackage{natbib}            % Use Natbib style for the refs.
\hypersetup{colorlinks=true}   % Set to false for black/white printing
\input{Definitions}            % Include your abbreviations



\usepackage{enumitem}% http://ctan.org/pkg/enumitem
\usepackage{multirow}
\usepackage{float}
\usepackage{amsmath}
\usepackage{multicol}
\usepackage{amssymb}
\usepackage[normalem]{ulem}
\useunder{\uline}{\ul}{}
\usepackage{wrapfig}


\usepackage[table,xcdraw]{xcolor}


%% ----------------------------------------------------------------
\begin{document}
\frontmatter
\title      {Heterogeneous Agent-based Model for Supermarket Competition}
\authors    {\texorpdfstring
             {\href{mailto:sc22g13@ecs.soton.ac.uk}{Stefan J. Collier}}
             {Stefan J. Collier}
            }
\addresses  {\groupname\\\deptname\\\univname}
\date       {\today}
\subject    {}
\keywords   {}
\supervisor {Dr. Maria Polukarov}
\examiner   {Professor Sheng Chen}

\maketitle
\begin{abstract}
This project aim was to model and analyse the effects of competitive pricing behaviors of grocery retailers on the British market. 

This was achieved by creating a multi-agent model, containing retailer and consumer agents. The heterogeneous crowd of retailers employs either a uniform pricing strategy or a ‘local price flexing’ strategy. The actions of these retailers are chosen by predicting the profit of each action, using a perceptron. Following on from the consideration of different economic models, a discrete model was developed so that software agents have a discrete environment to operate within. Within the model, it has been observed how supermarkets with differing behaviors affect a heterogeneous crowd of consumer agents. The model was implemented in Java with Python used to evaluate the results. 

The simulation displays good acceptance with real grocery market behavior, i.e. captures the performance of British retailers thus can be used to determine the impact of changes in their behavior on their competitors and consumers.Furthermore it can be used to provide insight into sustainability of volatile pricing strategies, providing a useful insight in volatility of British supermarket retail industry. 
\end{abstract}
\acknowledgements{
I would like to express my sincere gratitude to Dr Maria Polukarov for her guidance and support which provided me the freedom to take this research in the direction of my interest.\\
\\
I would also like to thank my family and friends for their encouragement and support. To those who quietly listened to my software complaints. To those who worked throughout the nights with me. To those who helped me write what I couldn't say. I cannot thank you enough.
}

\declaration{
I, Stefan Collier, declare that this dissertation and the work presented in it are my own and has been generated by me as the result of my own original research.\\
I confirm that:\\
1. This work was done wholly or mainly while in candidature for a degree at this University;\\
2. Where any part of this dissertation has previously been submitted for any other qualification at this University or any other institution, this has been clearly stated;\\
3. Where I have consulted the published work of others, this is always clearly attributed;\\
4. Where I have quoted from the work of others, the source is always given. With the exception of such quotations, this dissertation is entirely my own work;\\
5. I have acknowledged all main sources of help;\\
6. Where the thesis is based on work done by myself jointly with others, I have made clear exactly what was done by others and what I have contributed myself;\\
7. Either none of this work has been published before submission, or parts of this work have been published by :\\
\\
Stefan Collier\\
April 2016
}
\tableofcontents
\listoffigures
\listoftables

\mainmatter
%% ----------------------------------------------------------------
%\include{Introduction}
%\include{Conclusions}
\include{chapters/1Project/main}
\include{chapters/2Lit/main}
\include{chapters/3Design/HighLevel}
\include{chapters/3Design/InDepth}
\include{chapters/4Impl/main}

\include{chapters/5Experiments/1/main}
\include{chapters/5Experiments/2/main}
\include{chapters/5Experiments/3/main}
\include{chapters/5Experiments/4/main}

\include{chapters/6Conclusion/main}

\appendix
\include{appendix/AppendixB}
\include{appendix/D/main}
\include{appendix/AppendixC}

\backmatter
\bibliographystyle{ecs}
\bibliography{ECS}
\end{document}
%% ----------------------------------------------------------------


\appendix
\include{appendix/AppendixB}
 %% ----------------------------------------------------------------
%% Progress.tex
%% ---------------------------------------------------------------- 
\documentclass{ecsprogress}    % Use the progress Style
\graphicspath{{../figs/}}   % Location of your graphics files
    \usepackage{natbib}            % Use Natbib style for the refs.
\hypersetup{colorlinks=true}   % Set to false for black/white printing
\input{Definitions}            % Include your abbreviations



\usepackage{enumitem}% http://ctan.org/pkg/enumitem
\usepackage{multirow}
\usepackage{float}
\usepackage{amsmath}
\usepackage{multicol}
\usepackage{amssymb}
\usepackage[normalem]{ulem}
\useunder{\uline}{\ul}{}
\usepackage{wrapfig}


\usepackage[table,xcdraw]{xcolor}


%% ----------------------------------------------------------------
\begin{document}
\frontmatter
\title      {Heterogeneous Agent-based Model for Supermarket Competition}
\authors    {\texorpdfstring
             {\href{mailto:sc22g13@ecs.soton.ac.uk}{Stefan J. Collier}}
             {Stefan J. Collier}
            }
\addresses  {\groupname\\\deptname\\\univname}
\date       {\today}
\subject    {}
\keywords   {}
\supervisor {Dr. Maria Polukarov}
\examiner   {Professor Sheng Chen}

\maketitle
\begin{abstract}
This project aim was to model and analyse the effects of competitive pricing behaviors of grocery retailers on the British market. 

This was achieved by creating a multi-agent model, containing retailer and consumer agents. The heterogeneous crowd of retailers employs either a uniform pricing strategy or a ‘local price flexing’ strategy. The actions of these retailers are chosen by predicting the profit of each action, using a perceptron. Following on from the consideration of different economic models, a discrete model was developed so that software agents have a discrete environment to operate within. Within the model, it has been observed how supermarkets with differing behaviors affect a heterogeneous crowd of consumer agents. The model was implemented in Java with Python used to evaluate the results. 

The simulation displays good acceptance with real grocery market behavior, i.e. captures the performance of British retailers thus can be used to determine the impact of changes in their behavior on their competitors and consumers.Furthermore it can be used to provide insight into sustainability of volatile pricing strategies, providing a useful insight in volatility of British supermarket retail industry. 
\end{abstract}
\acknowledgements{
I would like to express my sincere gratitude to Dr Maria Polukarov for her guidance and support which provided me the freedom to take this research in the direction of my interest.\\
\\
I would also like to thank my family and friends for their encouragement and support. To those who quietly listened to my software complaints. To those who worked throughout the nights with me. To those who helped me write what I couldn't say. I cannot thank you enough.
}

\declaration{
I, Stefan Collier, declare that this dissertation and the work presented in it are my own and has been generated by me as the result of my own original research.\\
I confirm that:\\
1. This work was done wholly or mainly while in candidature for a degree at this University;\\
2. Where any part of this dissertation has previously been submitted for any other qualification at this University or any other institution, this has been clearly stated;\\
3. Where I have consulted the published work of others, this is always clearly attributed;\\
4. Where I have quoted from the work of others, the source is always given. With the exception of such quotations, this dissertation is entirely my own work;\\
5. I have acknowledged all main sources of help;\\
6. Where the thesis is based on work done by myself jointly with others, I have made clear exactly what was done by others and what I have contributed myself;\\
7. Either none of this work has been published before submission, or parts of this work have been published by :\\
\\
Stefan Collier\\
April 2016
}
\tableofcontents
\listoffigures
\listoftables

\mainmatter
%% ----------------------------------------------------------------
%\include{Introduction}
%\include{Conclusions}
\include{chapters/1Project/main}
\include{chapters/2Lit/main}
\include{chapters/3Design/HighLevel}
\include{chapters/3Design/InDepth}
\include{chapters/4Impl/main}

\include{chapters/5Experiments/1/main}
\include{chapters/5Experiments/2/main}
\include{chapters/5Experiments/3/main}
\include{chapters/5Experiments/4/main}

\include{chapters/6Conclusion/main}

\appendix
\include{appendix/AppendixB}
\include{appendix/D/main}
\include{appendix/AppendixC}

\backmatter
\bibliographystyle{ecs}
\bibliography{ECS}
\end{document}
%% ----------------------------------------------------------------

\include{appendix/AppendixC}

\backmatter
\bibliographystyle{ecs}
\bibliography{ECS}
\end{document}
%% ----------------------------------------------------------------

\include{chapters/3Design/HighLevel}
\include{chapters/3Design/InDepth}
 %% ----------------------------------------------------------------
%% Progress.tex
%% ---------------------------------------------------------------- 
\documentclass{ecsprogress}    % Use the progress Style
\graphicspath{{../figs/}}   % Location of your graphics files
    \usepackage{natbib}            % Use Natbib style for the refs.
\hypersetup{colorlinks=true}   % Set to false for black/white printing
\input{Definitions}            % Include your abbreviations



\usepackage{enumitem}% http://ctan.org/pkg/enumitem
\usepackage{multirow}
\usepackage{float}
\usepackage{amsmath}
\usepackage{multicol}
\usepackage{amssymb}
\usepackage[normalem]{ulem}
\useunder{\uline}{\ul}{}
\usepackage{wrapfig}


\usepackage[table,xcdraw]{xcolor}


%% ----------------------------------------------------------------
\begin{document}
\frontmatter
\title      {Heterogeneous Agent-based Model for Supermarket Competition}
\authors    {\texorpdfstring
             {\href{mailto:sc22g13@ecs.soton.ac.uk}{Stefan J. Collier}}
             {Stefan J. Collier}
            }
\addresses  {\groupname\\\deptname\\\univname}
\date       {\today}
\subject    {}
\keywords   {}
\supervisor {Dr. Maria Polukarov}
\examiner   {Professor Sheng Chen}

\maketitle
\begin{abstract}
This project aim was to model and analyse the effects of competitive pricing behaviors of grocery retailers on the British market. 

This was achieved by creating a multi-agent model, containing retailer and consumer agents. The heterogeneous crowd of retailers employs either a uniform pricing strategy or a ‘local price flexing’ strategy. The actions of these retailers are chosen by predicting the profit of each action, using a perceptron. Following on from the consideration of different economic models, a discrete model was developed so that software agents have a discrete environment to operate within. Within the model, it has been observed how supermarkets with differing behaviors affect a heterogeneous crowd of consumer agents. The model was implemented in Java with Python used to evaluate the results. 

The simulation displays good acceptance with real grocery market behavior, i.e. captures the performance of British retailers thus can be used to determine the impact of changes in their behavior on their competitors and consumers.Furthermore it can be used to provide insight into sustainability of volatile pricing strategies, providing a useful insight in volatility of British supermarket retail industry. 
\end{abstract}
\acknowledgements{
I would like to express my sincere gratitude to Dr Maria Polukarov for her guidance and support which provided me the freedom to take this research in the direction of my interest.\\
\\
I would also like to thank my family and friends for their encouragement and support. To those who quietly listened to my software complaints. To those who worked throughout the nights with me. To those who helped me write what I couldn't say. I cannot thank you enough.
}

\declaration{
I, Stefan Collier, declare that this dissertation and the work presented in it are my own and has been generated by me as the result of my own original research.\\
I confirm that:\\
1. This work was done wholly or mainly while in candidature for a degree at this University;\\
2. Where any part of this dissertation has previously been submitted for any other qualification at this University or any other institution, this has been clearly stated;\\
3. Where I have consulted the published work of others, this is always clearly attributed;\\
4. Where I have quoted from the work of others, the source is always given. With the exception of such quotations, this dissertation is entirely my own work;\\
5. I have acknowledged all main sources of help;\\
6. Where the thesis is based on work done by myself jointly with others, I have made clear exactly what was done by others and what I have contributed myself;\\
7. Either none of this work has been published before submission, or parts of this work have been published by :\\
\\
Stefan Collier\\
April 2016
}
\tableofcontents
\listoffigures
\listoftables

\mainmatter
%% ----------------------------------------------------------------
%\include{Introduction}
%\include{Conclusions}
 %% ----------------------------------------------------------------
%% Progress.tex
%% ---------------------------------------------------------------- 
\documentclass{ecsprogress}    % Use the progress Style
\graphicspath{{../figs/}}   % Location of your graphics files
    \usepackage{natbib}            % Use Natbib style for the refs.
\hypersetup{colorlinks=true}   % Set to false for black/white printing
\input{Definitions}            % Include your abbreviations



\usepackage{enumitem}% http://ctan.org/pkg/enumitem
\usepackage{multirow}
\usepackage{float}
\usepackage{amsmath}
\usepackage{multicol}
\usepackage{amssymb}
\usepackage[normalem]{ulem}
\useunder{\uline}{\ul}{}
\usepackage{wrapfig}


\usepackage[table,xcdraw]{xcolor}


%% ----------------------------------------------------------------
\begin{document}
\frontmatter
\title      {Heterogeneous Agent-based Model for Supermarket Competition}
\authors    {\texorpdfstring
             {\href{mailto:sc22g13@ecs.soton.ac.uk}{Stefan J. Collier}}
             {Stefan J. Collier}
            }
\addresses  {\groupname\\\deptname\\\univname}
\date       {\today}
\subject    {}
\keywords   {}
\supervisor {Dr. Maria Polukarov}
\examiner   {Professor Sheng Chen}

\maketitle
\begin{abstract}
This project aim was to model and analyse the effects of competitive pricing behaviors of grocery retailers on the British market. 

This was achieved by creating a multi-agent model, containing retailer and consumer agents. The heterogeneous crowd of retailers employs either a uniform pricing strategy or a ‘local price flexing’ strategy. The actions of these retailers are chosen by predicting the profit of each action, using a perceptron. Following on from the consideration of different economic models, a discrete model was developed so that software agents have a discrete environment to operate within. Within the model, it has been observed how supermarkets with differing behaviors affect a heterogeneous crowd of consumer agents. The model was implemented in Java with Python used to evaluate the results. 

The simulation displays good acceptance with real grocery market behavior, i.e. captures the performance of British retailers thus can be used to determine the impact of changes in their behavior on their competitors and consumers.Furthermore it can be used to provide insight into sustainability of volatile pricing strategies, providing a useful insight in volatility of British supermarket retail industry. 
\end{abstract}
\acknowledgements{
I would like to express my sincere gratitude to Dr Maria Polukarov for her guidance and support which provided me the freedom to take this research in the direction of my interest.\\
\\
I would also like to thank my family and friends for their encouragement and support. To those who quietly listened to my software complaints. To those who worked throughout the nights with me. To those who helped me write what I couldn't say. I cannot thank you enough.
}

\declaration{
I, Stefan Collier, declare that this dissertation and the work presented in it are my own and has been generated by me as the result of my own original research.\\
I confirm that:\\
1. This work was done wholly or mainly while in candidature for a degree at this University;\\
2. Where any part of this dissertation has previously been submitted for any other qualification at this University or any other institution, this has been clearly stated;\\
3. Where I have consulted the published work of others, this is always clearly attributed;\\
4. Where I have quoted from the work of others, the source is always given. With the exception of such quotations, this dissertation is entirely my own work;\\
5. I have acknowledged all main sources of help;\\
6. Where the thesis is based on work done by myself jointly with others, I have made clear exactly what was done by others and what I have contributed myself;\\
7. Either none of this work has been published before submission, or parts of this work have been published by :\\
\\
Stefan Collier\\
April 2016
}
\tableofcontents
\listoffigures
\listoftables

\mainmatter
%% ----------------------------------------------------------------
%\include{Introduction}
%\include{Conclusions}
\include{chapters/1Project/main}
\include{chapters/2Lit/main}
\include{chapters/3Design/HighLevel}
\include{chapters/3Design/InDepth}
\include{chapters/4Impl/main}

\include{chapters/5Experiments/1/main}
\include{chapters/5Experiments/2/main}
\include{chapters/5Experiments/3/main}
\include{chapters/5Experiments/4/main}

\include{chapters/6Conclusion/main}

\appendix
\include{appendix/AppendixB}
\include{appendix/D/main}
\include{appendix/AppendixC}

\backmatter
\bibliographystyle{ecs}
\bibliography{ECS}
\end{document}
%% ----------------------------------------------------------------

 %% ----------------------------------------------------------------
%% Progress.tex
%% ---------------------------------------------------------------- 
\documentclass{ecsprogress}    % Use the progress Style
\graphicspath{{../figs/}}   % Location of your graphics files
    \usepackage{natbib}            % Use Natbib style for the refs.
\hypersetup{colorlinks=true}   % Set to false for black/white printing
\input{Definitions}            % Include your abbreviations



\usepackage{enumitem}% http://ctan.org/pkg/enumitem
\usepackage{multirow}
\usepackage{float}
\usepackage{amsmath}
\usepackage{multicol}
\usepackage{amssymb}
\usepackage[normalem]{ulem}
\useunder{\uline}{\ul}{}
\usepackage{wrapfig}


\usepackage[table,xcdraw]{xcolor}


%% ----------------------------------------------------------------
\begin{document}
\frontmatter
\title      {Heterogeneous Agent-based Model for Supermarket Competition}
\authors    {\texorpdfstring
             {\href{mailto:sc22g13@ecs.soton.ac.uk}{Stefan J. Collier}}
             {Stefan J. Collier}
            }
\addresses  {\groupname\\\deptname\\\univname}
\date       {\today}
\subject    {}
\keywords   {}
\supervisor {Dr. Maria Polukarov}
\examiner   {Professor Sheng Chen}

\maketitle
\begin{abstract}
This project aim was to model and analyse the effects of competitive pricing behaviors of grocery retailers on the British market. 

This was achieved by creating a multi-agent model, containing retailer and consumer agents. The heterogeneous crowd of retailers employs either a uniform pricing strategy or a ‘local price flexing’ strategy. The actions of these retailers are chosen by predicting the profit of each action, using a perceptron. Following on from the consideration of different economic models, a discrete model was developed so that software agents have a discrete environment to operate within. Within the model, it has been observed how supermarkets with differing behaviors affect a heterogeneous crowd of consumer agents. The model was implemented in Java with Python used to evaluate the results. 

The simulation displays good acceptance with real grocery market behavior, i.e. captures the performance of British retailers thus can be used to determine the impact of changes in their behavior on their competitors and consumers.Furthermore it can be used to provide insight into sustainability of volatile pricing strategies, providing a useful insight in volatility of British supermarket retail industry. 
\end{abstract}
\acknowledgements{
I would like to express my sincere gratitude to Dr Maria Polukarov for her guidance and support which provided me the freedom to take this research in the direction of my interest.\\
\\
I would also like to thank my family and friends for their encouragement and support. To those who quietly listened to my software complaints. To those who worked throughout the nights with me. To those who helped me write what I couldn't say. I cannot thank you enough.
}

\declaration{
I, Stefan Collier, declare that this dissertation and the work presented in it are my own and has been generated by me as the result of my own original research.\\
I confirm that:\\
1. This work was done wholly or mainly while in candidature for a degree at this University;\\
2. Where any part of this dissertation has previously been submitted for any other qualification at this University or any other institution, this has been clearly stated;\\
3. Where I have consulted the published work of others, this is always clearly attributed;\\
4. Where I have quoted from the work of others, the source is always given. With the exception of such quotations, this dissertation is entirely my own work;\\
5. I have acknowledged all main sources of help;\\
6. Where the thesis is based on work done by myself jointly with others, I have made clear exactly what was done by others and what I have contributed myself;\\
7. Either none of this work has been published before submission, or parts of this work have been published by :\\
\\
Stefan Collier\\
April 2016
}
\tableofcontents
\listoffigures
\listoftables

\mainmatter
%% ----------------------------------------------------------------
%\include{Introduction}
%\include{Conclusions}
\include{chapters/1Project/main}
\include{chapters/2Lit/main}
\include{chapters/3Design/HighLevel}
\include{chapters/3Design/InDepth}
\include{chapters/4Impl/main}

\include{chapters/5Experiments/1/main}
\include{chapters/5Experiments/2/main}
\include{chapters/5Experiments/3/main}
\include{chapters/5Experiments/4/main}

\include{chapters/6Conclusion/main}

\appendix
\include{appendix/AppendixB}
\include{appendix/D/main}
\include{appendix/AppendixC}

\backmatter
\bibliographystyle{ecs}
\bibliography{ECS}
\end{document}
%% ----------------------------------------------------------------

\include{chapters/3Design/HighLevel}
\include{chapters/3Design/InDepth}
 %% ----------------------------------------------------------------
%% Progress.tex
%% ---------------------------------------------------------------- 
\documentclass{ecsprogress}    % Use the progress Style
\graphicspath{{../figs/}}   % Location of your graphics files
    \usepackage{natbib}            % Use Natbib style for the refs.
\hypersetup{colorlinks=true}   % Set to false for black/white printing
\input{Definitions}            % Include your abbreviations



\usepackage{enumitem}% http://ctan.org/pkg/enumitem
\usepackage{multirow}
\usepackage{float}
\usepackage{amsmath}
\usepackage{multicol}
\usepackage{amssymb}
\usepackage[normalem]{ulem}
\useunder{\uline}{\ul}{}
\usepackage{wrapfig}


\usepackage[table,xcdraw]{xcolor}


%% ----------------------------------------------------------------
\begin{document}
\frontmatter
\title      {Heterogeneous Agent-based Model for Supermarket Competition}
\authors    {\texorpdfstring
             {\href{mailto:sc22g13@ecs.soton.ac.uk}{Stefan J. Collier}}
             {Stefan J. Collier}
            }
\addresses  {\groupname\\\deptname\\\univname}
\date       {\today}
\subject    {}
\keywords   {}
\supervisor {Dr. Maria Polukarov}
\examiner   {Professor Sheng Chen}

\maketitle
\begin{abstract}
This project aim was to model and analyse the effects of competitive pricing behaviors of grocery retailers on the British market. 

This was achieved by creating a multi-agent model, containing retailer and consumer agents. The heterogeneous crowd of retailers employs either a uniform pricing strategy or a ‘local price flexing’ strategy. The actions of these retailers are chosen by predicting the profit of each action, using a perceptron. Following on from the consideration of different economic models, a discrete model was developed so that software agents have a discrete environment to operate within. Within the model, it has been observed how supermarkets with differing behaviors affect a heterogeneous crowd of consumer agents. The model was implemented in Java with Python used to evaluate the results. 

The simulation displays good acceptance with real grocery market behavior, i.e. captures the performance of British retailers thus can be used to determine the impact of changes in their behavior on their competitors and consumers.Furthermore it can be used to provide insight into sustainability of volatile pricing strategies, providing a useful insight in volatility of British supermarket retail industry. 
\end{abstract}
\acknowledgements{
I would like to express my sincere gratitude to Dr Maria Polukarov for her guidance and support which provided me the freedom to take this research in the direction of my interest.\\
\\
I would also like to thank my family and friends for their encouragement and support. To those who quietly listened to my software complaints. To those who worked throughout the nights with me. To those who helped me write what I couldn't say. I cannot thank you enough.
}

\declaration{
I, Stefan Collier, declare that this dissertation and the work presented in it are my own and has been generated by me as the result of my own original research.\\
I confirm that:\\
1. This work was done wholly or mainly while in candidature for a degree at this University;\\
2. Where any part of this dissertation has previously been submitted for any other qualification at this University or any other institution, this has been clearly stated;\\
3. Where I have consulted the published work of others, this is always clearly attributed;\\
4. Where I have quoted from the work of others, the source is always given. With the exception of such quotations, this dissertation is entirely my own work;\\
5. I have acknowledged all main sources of help;\\
6. Where the thesis is based on work done by myself jointly with others, I have made clear exactly what was done by others and what I have contributed myself;\\
7. Either none of this work has been published before submission, or parts of this work have been published by :\\
\\
Stefan Collier\\
April 2016
}
\tableofcontents
\listoffigures
\listoftables

\mainmatter
%% ----------------------------------------------------------------
%\include{Introduction}
%\include{Conclusions}
\include{chapters/1Project/main}
\include{chapters/2Lit/main}
\include{chapters/3Design/HighLevel}
\include{chapters/3Design/InDepth}
\include{chapters/4Impl/main}

\include{chapters/5Experiments/1/main}
\include{chapters/5Experiments/2/main}
\include{chapters/5Experiments/3/main}
\include{chapters/5Experiments/4/main}

\include{chapters/6Conclusion/main}

\appendix
\include{appendix/AppendixB}
\include{appendix/D/main}
\include{appendix/AppendixC}

\backmatter
\bibliographystyle{ecs}
\bibliography{ECS}
\end{document}
%% ----------------------------------------------------------------


 %% ----------------------------------------------------------------
%% Progress.tex
%% ---------------------------------------------------------------- 
\documentclass{ecsprogress}    % Use the progress Style
\graphicspath{{../figs/}}   % Location of your graphics files
    \usepackage{natbib}            % Use Natbib style for the refs.
\hypersetup{colorlinks=true}   % Set to false for black/white printing
\input{Definitions}            % Include your abbreviations



\usepackage{enumitem}% http://ctan.org/pkg/enumitem
\usepackage{multirow}
\usepackage{float}
\usepackage{amsmath}
\usepackage{multicol}
\usepackage{amssymb}
\usepackage[normalem]{ulem}
\useunder{\uline}{\ul}{}
\usepackage{wrapfig}


\usepackage[table,xcdraw]{xcolor}


%% ----------------------------------------------------------------
\begin{document}
\frontmatter
\title      {Heterogeneous Agent-based Model for Supermarket Competition}
\authors    {\texorpdfstring
             {\href{mailto:sc22g13@ecs.soton.ac.uk}{Stefan J. Collier}}
             {Stefan J. Collier}
            }
\addresses  {\groupname\\\deptname\\\univname}
\date       {\today}
\subject    {}
\keywords   {}
\supervisor {Dr. Maria Polukarov}
\examiner   {Professor Sheng Chen}

\maketitle
\begin{abstract}
This project aim was to model and analyse the effects of competitive pricing behaviors of grocery retailers on the British market. 

This was achieved by creating a multi-agent model, containing retailer and consumer agents. The heterogeneous crowd of retailers employs either a uniform pricing strategy or a ‘local price flexing’ strategy. The actions of these retailers are chosen by predicting the profit of each action, using a perceptron. Following on from the consideration of different economic models, a discrete model was developed so that software agents have a discrete environment to operate within. Within the model, it has been observed how supermarkets with differing behaviors affect a heterogeneous crowd of consumer agents. The model was implemented in Java with Python used to evaluate the results. 

The simulation displays good acceptance with real grocery market behavior, i.e. captures the performance of British retailers thus can be used to determine the impact of changes in their behavior on their competitors and consumers.Furthermore it can be used to provide insight into sustainability of volatile pricing strategies, providing a useful insight in volatility of British supermarket retail industry. 
\end{abstract}
\acknowledgements{
I would like to express my sincere gratitude to Dr Maria Polukarov for her guidance and support which provided me the freedom to take this research in the direction of my interest.\\
\\
I would also like to thank my family and friends for their encouragement and support. To those who quietly listened to my software complaints. To those who worked throughout the nights with me. To those who helped me write what I couldn't say. I cannot thank you enough.
}

\declaration{
I, Stefan Collier, declare that this dissertation and the work presented in it are my own and has been generated by me as the result of my own original research.\\
I confirm that:\\
1. This work was done wholly or mainly while in candidature for a degree at this University;\\
2. Where any part of this dissertation has previously been submitted for any other qualification at this University or any other institution, this has been clearly stated;\\
3. Where I have consulted the published work of others, this is always clearly attributed;\\
4. Where I have quoted from the work of others, the source is always given. With the exception of such quotations, this dissertation is entirely my own work;\\
5. I have acknowledged all main sources of help;\\
6. Where the thesis is based on work done by myself jointly with others, I have made clear exactly what was done by others and what I have contributed myself;\\
7. Either none of this work has been published before submission, or parts of this work have been published by :\\
\\
Stefan Collier\\
April 2016
}
\tableofcontents
\listoffigures
\listoftables

\mainmatter
%% ----------------------------------------------------------------
%\include{Introduction}
%\include{Conclusions}
\include{chapters/1Project/main}
\include{chapters/2Lit/main}
\include{chapters/3Design/HighLevel}
\include{chapters/3Design/InDepth}
\include{chapters/4Impl/main}

\include{chapters/5Experiments/1/main}
\include{chapters/5Experiments/2/main}
\include{chapters/5Experiments/3/main}
\include{chapters/5Experiments/4/main}

\include{chapters/6Conclusion/main}

\appendix
\include{appendix/AppendixB}
\include{appendix/D/main}
\include{appendix/AppendixC}

\backmatter
\bibliographystyle{ecs}
\bibliography{ECS}
\end{document}
%% ----------------------------------------------------------------

 %% ----------------------------------------------------------------
%% Progress.tex
%% ---------------------------------------------------------------- 
\documentclass{ecsprogress}    % Use the progress Style
\graphicspath{{../figs/}}   % Location of your graphics files
    \usepackage{natbib}            % Use Natbib style for the refs.
\hypersetup{colorlinks=true}   % Set to false for black/white printing
\input{Definitions}            % Include your abbreviations



\usepackage{enumitem}% http://ctan.org/pkg/enumitem
\usepackage{multirow}
\usepackage{float}
\usepackage{amsmath}
\usepackage{multicol}
\usepackage{amssymb}
\usepackage[normalem]{ulem}
\useunder{\uline}{\ul}{}
\usepackage{wrapfig}


\usepackage[table,xcdraw]{xcolor}


%% ----------------------------------------------------------------
\begin{document}
\frontmatter
\title      {Heterogeneous Agent-based Model for Supermarket Competition}
\authors    {\texorpdfstring
             {\href{mailto:sc22g13@ecs.soton.ac.uk}{Stefan J. Collier}}
             {Stefan J. Collier}
            }
\addresses  {\groupname\\\deptname\\\univname}
\date       {\today}
\subject    {}
\keywords   {}
\supervisor {Dr. Maria Polukarov}
\examiner   {Professor Sheng Chen}

\maketitle
\begin{abstract}
This project aim was to model and analyse the effects of competitive pricing behaviors of grocery retailers on the British market. 

This was achieved by creating a multi-agent model, containing retailer and consumer agents. The heterogeneous crowd of retailers employs either a uniform pricing strategy or a ‘local price flexing’ strategy. The actions of these retailers are chosen by predicting the profit of each action, using a perceptron. Following on from the consideration of different economic models, a discrete model was developed so that software agents have a discrete environment to operate within. Within the model, it has been observed how supermarkets with differing behaviors affect a heterogeneous crowd of consumer agents. The model was implemented in Java with Python used to evaluate the results. 

The simulation displays good acceptance with real grocery market behavior, i.e. captures the performance of British retailers thus can be used to determine the impact of changes in their behavior on their competitors and consumers.Furthermore it can be used to provide insight into sustainability of volatile pricing strategies, providing a useful insight in volatility of British supermarket retail industry. 
\end{abstract}
\acknowledgements{
I would like to express my sincere gratitude to Dr Maria Polukarov for her guidance and support which provided me the freedom to take this research in the direction of my interest.\\
\\
I would also like to thank my family and friends for their encouragement and support. To those who quietly listened to my software complaints. To those who worked throughout the nights with me. To those who helped me write what I couldn't say. I cannot thank you enough.
}

\declaration{
I, Stefan Collier, declare that this dissertation and the work presented in it are my own and has been generated by me as the result of my own original research.\\
I confirm that:\\
1. This work was done wholly or mainly while in candidature for a degree at this University;\\
2. Where any part of this dissertation has previously been submitted for any other qualification at this University or any other institution, this has been clearly stated;\\
3. Where I have consulted the published work of others, this is always clearly attributed;\\
4. Where I have quoted from the work of others, the source is always given. With the exception of such quotations, this dissertation is entirely my own work;\\
5. I have acknowledged all main sources of help;\\
6. Where the thesis is based on work done by myself jointly with others, I have made clear exactly what was done by others and what I have contributed myself;\\
7. Either none of this work has been published before submission, or parts of this work have been published by :\\
\\
Stefan Collier\\
April 2016
}
\tableofcontents
\listoffigures
\listoftables

\mainmatter
%% ----------------------------------------------------------------
%\include{Introduction}
%\include{Conclusions}
\include{chapters/1Project/main}
\include{chapters/2Lit/main}
\include{chapters/3Design/HighLevel}
\include{chapters/3Design/InDepth}
\include{chapters/4Impl/main}

\include{chapters/5Experiments/1/main}
\include{chapters/5Experiments/2/main}
\include{chapters/5Experiments/3/main}
\include{chapters/5Experiments/4/main}

\include{chapters/6Conclusion/main}

\appendix
\include{appendix/AppendixB}
\include{appendix/D/main}
\include{appendix/AppendixC}

\backmatter
\bibliographystyle{ecs}
\bibliography{ECS}
\end{document}
%% ----------------------------------------------------------------

 %% ----------------------------------------------------------------
%% Progress.tex
%% ---------------------------------------------------------------- 
\documentclass{ecsprogress}    % Use the progress Style
\graphicspath{{../figs/}}   % Location of your graphics files
    \usepackage{natbib}            % Use Natbib style for the refs.
\hypersetup{colorlinks=true}   % Set to false for black/white printing
\input{Definitions}            % Include your abbreviations



\usepackage{enumitem}% http://ctan.org/pkg/enumitem
\usepackage{multirow}
\usepackage{float}
\usepackage{amsmath}
\usepackage{multicol}
\usepackage{amssymb}
\usepackage[normalem]{ulem}
\useunder{\uline}{\ul}{}
\usepackage{wrapfig}


\usepackage[table,xcdraw]{xcolor}


%% ----------------------------------------------------------------
\begin{document}
\frontmatter
\title      {Heterogeneous Agent-based Model for Supermarket Competition}
\authors    {\texorpdfstring
             {\href{mailto:sc22g13@ecs.soton.ac.uk}{Stefan J. Collier}}
             {Stefan J. Collier}
            }
\addresses  {\groupname\\\deptname\\\univname}
\date       {\today}
\subject    {}
\keywords   {}
\supervisor {Dr. Maria Polukarov}
\examiner   {Professor Sheng Chen}

\maketitle
\begin{abstract}
This project aim was to model and analyse the effects of competitive pricing behaviors of grocery retailers on the British market. 

This was achieved by creating a multi-agent model, containing retailer and consumer agents. The heterogeneous crowd of retailers employs either a uniform pricing strategy or a ‘local price flexing’ strategy. The actions of these retailers are chosen by predicting the profit of each action, using a perceptron. Following on from the consideration of different economic models, a discrete model was developed so that software agents have a discrete environment to operate within. Within the model, it has been observed how supermarkets with differing behaviors affect a heterogeneous crowd of consumer agents. The model was implemented in Java with Python used to evaluate the results. 

The simulation displays good acceptance with real grocery market behavior, i.e. captures the performance of British retailers thus can be used to determine the impact of changes in their behavior on their competitors and consumers.Furthermore it can be used to provide insight into sustainability of volatile pricing strategies, providing a useful insight in volatility of British supermarket retail industry. 
\end{abstract}
\acknowledgements{
I would like to express my sincere gratitude to Dr Maria Polukarov for her guidance and support which provided me the freedom to take this research in the direction of my interest.\\
\\
I would also like to thank my family and friends for their encouragement and support. To those who quietly listened to my software complaints. To those who worked throughout the nights with me. To those who helped me write what I couldn't say. I cannot thank you enough.
}

\declaration{
I, Stefan Collier, declare that this dissertation and the work presented in it are my own and has been generated by me as the result of my own original research.\\
I confirm that:\\
1. This work was done wholly or mainly while in candidature for a degree at this University;\\
2. Where any part of this dissertation has previously been submitted for any other qualification at this University or any other institution, this has been clearly stated;\\
3. Where I have consulted the published work of others, this is always clearly attributed;\\
4. Where I have quoted from the work of others, the source is always given. With the exception of such quotations, this dissertation is entirely my own work;\\
5. I have acknowledged all main sources of help;\\
6. Where the thesis is based on work done by myself jointly with others, I have made clear exactly what was done by others and what I have contributed myself;\\
7. Either none of this work has been published before submission, or parts of this work have been published by :\\
\\
Stefan Collier\\
April 2016
}
\tableofcontents
\listoffigures
\listoftables

\mainmatter
%% ----------------------------------------------------------------
%\include{Introduction}
%\include{Conclusions}
\include{chapters/1Project/main}
\include{chapters/2Lit/main}
\include{chapters/3Design/HighLevel}
\include{chapters/3Design/InDepth}
\include{chapters/4Impl/main}

\include{chapters/5Experiments/1/main}
\include{chapters/5Experiments/2/main}
\include{chapters/5Experiments/3/main}
\include{chapters/5Experiments/4/main}

\include{chapters/6Conclusion/main}

\appendix
\include{appendix/AppendixB}
\include{appendix/D/main}
\include{appendix/AppendixC}

\backmatter
\bibliographystyle{ecs}
\bibliography{ECS}
\end{document}
%% ----------------------------------------------------------------

 %% ----------------------------------------------------------------
%% Progress.tex
%% ---------------------------------------------------------------- 
\documentclass{ecsprogress}    % Use the progress Style
\graphicspath{{../figs/}}   % Location of your graphics files
    \usepackage{natbib}            % Use Natbib style for the refs.
\hypersetup{colorlinks=true}   % Set to false for black/white printing
\input{Definitions}            % Include your abbreviations



\usepackage{enumitem}% http://ctan.org/pkg/enumitem
\usepackage{multirow}
\usepackage{float}
\usepackage{amsmath}
\usepackage{multicol}
\usepackage{amssymb}
\usepackage[normalem]{ulem}
\useunder{\uline}{\ul}{}
\usepackage{wrapfig}


\usepackage[table,xcdraw]{xcolor}


%% ----------------------------------------------------------------
\begin{document}
\frontmatter
\title      {Heterogeneous Agent-based Model for Supermarket Competition}
\authors    {\texorpdfstring
             {\href{mailto:sc22g13@ecs.soton.ac.uk}{Stefan J. Collier}}
             {Stefan J. Collier}
            }
\addresses  {\groupname\\\deptname\\\univname}
\date       {\today}
\subject    {}
\keywords   {}
\supervisor {Dr. Maria Polukarov}
\examiner   {Professor Sheng Chen}

\maketitle
\begin{abstract}
This project aim was to model and analyse the effects of competitive pricing behaviors of grocery retailers on the British market. 

This was achieved by creating a multi-agent model, containing retailer and consumer agents. The heterogeneous crowd of retailers employs either a uniform pricing strategy or a ‘local price flexing’ strategy. The actions of these retailers are chosen by predicting the profit of each action, using a perceptron. Following on from the consideration of different economic models, a discrete model was developed so that software agents have a discrete environment to operate within. Within the model, it has been observed how supermarkets with differing behaviors affect a heterogeneous crowd of consumer agents. The model was implemented in Java with Python used to evaluate the results. 

The simulation displays good acceptance with real grocery market behavior, i.e. captures the performance of British retailers thus can be used to determine the impact of changes in their behavior on their competitors and consumers.Furthermore it can be used to provide insight into sustainability of volatile pricing strategies, providing a useful insight in volatility of British supermarket retail industry. 
\end{abstract}
\acknowledgements{
I would like to express my sincere gratitude to Dr Maria Polukarov for her guidance and support which provided me the freedom to take this research in the direction of my interest.\\
\\
I would also like to thank my family and friends for their encouragement and support. To those who quietly listened to my software complaints. To those who worked throughout the nights with me. To those who helped me write what I couldn't say. I cannot thank you enough.
}

\declaration{
I, Stefan Collier, declare that this dissertation and the work presented in it are my own and has been generated by me as the result of my own original research.\\
I confirm that:\\
1. This work was done wholly or mainly while in candidature for a degree at this University;\\
2. Where any part of this dissertation has previously been submitted for any other qualification at this University or any other institution, this has been clearly stated;\\
3. Where I have consulted the published work of others, this is always clearly attributed;\\
4. Where I have quoted from the work of others, the source is always given. With the exception of such quotations, this dissertation is entirely my own work;\\
5. I have acknowledged all main sources of help;\\
6. Where the thesis is based on work done by myself jointly with others, I have made clear exactly what was done by others and what I have contributed myself;\\
7. Either none of this work has been published before submission, or parts of this work have been published by :\\
\\
Stefan Collier\\
April 2016
}
\tableofcontents
\listoffigures
\listoftables

\mainmatter
%% ----------------------------------------------------------------
%\include{Introduction}
%\include{Conclusions}
\include{chapters/1Project/main}
\include{chapters/2Lit/main}
\include{chapters/3Design/HighLevel}
\include{chapters/3Design/InDepth}
\include{chapters/4Impl/main}

\include{chapters/5Experiments/1/main}
\include{chapters/5Experiments/2/main}
\include{chapters/5Experiments/3/main}
\include{chapters/5Experiments/4/main}

\include{chapters/6Conclusion/main}

\appendix
\include{appendix/AppendixB}
\include{appendix/D/main}
\include{appendix/AppendixC}

\backmatter
\bibliographystyle{ecs}
\bibliography{ECS}
\end{document}
%% ----------------------------------------------------------------


 %% ----------------------------------------------------------------
%% Progress.tex
%% ---------------------------------------------------------------- 
\documentclass{ecsprogress}    % Use the progress Style
\graphicspath{{../figs/}}   % Location of your graphics files
    \usepackage{natbib}            % Use Natbib style for the refs.
\hypersetup{colorlinks=true}   % Set to false for black/white printing
\input{Definitions}            % Include your abbreviations



\usepackage{enumitem}% http://ctan.org/pkg/enumitem
\usepackage{multirow}
\usepackage{float}
\usepackage{amsmath}
\usepackage{multicol}
\usepackage{amssymb}
\usepackage[normalem]{ulem}
\useunder{\uline}{\ul}{}
\usepackage{wrapfig}


\usepackage[table,xcdraw]{xcolor}


%% ----------------------------------------------------------------
\begin{document}
\frontmatter
\title      {Heterogeneous Agent-based Model for Supermarket Competition}
\authors    {\texorpdfstring
             {\href{mailto:sc22g13@ecs.soton.ac.uk}{Stefan J. Collier}}
             {Stefan J. Collier}
            }
\addresses  {\groupname\\\deptname\\\univname}
\date       {\today}
\subject    {}
\keywords   {}
\supervisor {Dr. Maria Polukarov}
\examiner   {Professor Sheng Chen}

\maketitle
\begin{abstract}
This project aim was to model and analyse the effects of competitive pricing behaviors of grocery retailers on the British market. 

This was achieved by creating a multi-agent model, containing retailer and consumer agents. The heterogeneous crowd of retailers employs either a uniform pricing strategy or a ‘local price flexing’ strategy. The actions of these retailers are chosen by predicting the profit of each action, using a perceptron. Following on from the consideration of different economic models, a discrete model was developed so that software agents have a discrete environment to operate within. Within the model, it has been observed how supermarkets with differing behaviors affect a heterogeneous crowd of consumer agents. The model was implemented in Java with Python used to evaluate the results. 

The simulation displays good acceptance with real grocery market behavior, i.e. captures the performance of British retailers thus can be used to determine the impact of changes in their behavior on their competitors and consumers.Furthermore it can be used to provide insight into sustainability of volatile pricing strategies, providing a useful insight in volatility of British supermarket retail industry. 
\end{abstract}
\acknowledgements{
I would like to express my sincere gratitude to Dr Maria Polukarov for her guidance and support which provided me the freedom to take this research in the direction of my interest.\\
\\
I would also like to thank my family and friends for their encouragement and support. To those who quietly listened to my software complaints. To those who worked throughout the nights with me. To those who helped me write what I couldn't say. I cannot thank you enough.
}

\declaration{
I, Stefan Collier, declare that this dissertation and the work presented in it are my own and has been generated by me as the result of my own original research.\\
I confirm that:\\
1. This work was done wholly or mainly while in candidature for a degree at this University;\\
2. Where any part of this dissertation has previously been submitted for any other qualification at this University or any other institution, this has been clearly stated;\\
3. Where I have consulted the published work of others, this is always clearly attributed;\\
4. Where I have quoted from the work of others, the source is always given. With the exception of such quotations, this dissertation is entirely my own work;\\
5. I have acknowledged all main sources of help;\\
6. Where the thesis is based on work done by myself jointly with others, I have made clear exactly what was done by others and what I have contributed myself;\\
7. Either none of this work has been published before submission, or parts of this work have been published by :\\
\\
Stefan Collier\\
April 2016
}
\tableofcontents
\listoffigures
\listoftables

\mainmatter
%% ----------------------------------------------------------------
%\include{Introduction}
%\include{Conclusions}
\include{chapters/1Project/main}
\include{chapters/2Lit/main}
\include{chapters/3Design/HighLevel}
\include{chapters/3Design/InDepth}
\include{chapters/4Impl/main}

\include{chapters/5Experiments/1/main}
\include{chapters/5Experiments/2/main}
\include{chapters/5Experiments/3/main}
\include{chapters/5Experiments/4/main}

\include{chapters/6Conclusion/main}

\appendix
\include{appendix/AppendixB}
\include{appendix/D/main}
\include{appendix/AppendixC}

\backmatter
\bibliographystyle{ecs}
\bibliography{ECS}
\end{document}
%% ----------------------------------------------------------------


\appendix
\include{appendix/AppendixB}
 %% ----------------------------------------------------------------
%% Progress.tex
%% ---------------------------------------------------------------- 
\documentclass{ecsprogress}    % Use the progress Style
\graphicspath{{../figs/}}   % Location of your graphics files
    \usepackage{natbib}            % Use Natbib style for the refs.
\hypersetup{colorlinks=true}   % Set to false for black/white printing
\input{Definitions}            % Include your abbreviations



\usepackage{enumitem}% http://ctan.org/pkg/enumitem
\usepackage{multirow}
\usepackage{float}
\usepackage{amsmath}
\usepackage{multicol}
\usepackage{amssymb}
\usepackage[normalem]{ulem}
\useunder{\uline}{\ul}{}
\usepackage{wrapfig}


\usepackage[table,xcdraw]{xcolor}


%% ----------------------------------------------------------------
\begin{document}
\frontmatter
\title      {Heterogeneous Agent-based Model for Supermarket Competition}
\authors    {\texorpdfstring
             {\href{mailto:sc22g13@ecs.soton.ac.uk}{Stefan J. Collier}}
             {Stefan J. Collier}
            }
\addresses  {\groupname\\\deptname\\\univname}
\date       {\today}
\subject    {}
\keywords   {}
\supervisor {Dr. Maria Polukarov}
\examiner   {Professor Sheng Chen}

\maketitle
\begin{abstract}
This project aim was to model and analyse the effects of competitive pricing behaviors of grocery retailers on the British market. 

This was achieved by creating a multi-agent model, containing retailer and consumer agents. The heterogeneous crowd of retailers employs either a uniform pricing strategy or a ‘local price flexing’ strategy. The actions of these retailers are chosen by predicting the profit of each action, using a perceptron. Following on from the consideration of different economic models, a discrete model was developed so that software agents have a discrete environment to operate within. Within the model, it has been observed how supermarkets with differing behaviors affect a heterogeneous crowd of consumer agents. The model was implemented in Java with Python used to evaluate the results. 

The simulation displays good acceptance with real grocery market behavior, i.e. captures the performance of British retailers thus can be used to determine the impact of changes in their behavior on their competitors and consumers.Furthermore it can be used to provide insight into sustainability of volatile pricing strategies, providing a useful insight in volatility of British supermarket retail industry. 
\end{abstract}
\acknowledgements{
I would like to express my sincere gratitude to Dr Maria Polukarov for her guidance and support which provided me the freedom to take this research in the direction of my interest.\\
\\
I would also like to thank my family and friends for their encouragement and support. To those who quietly listened to my software complaints. To those who worked throughout the nights with me. To those who helped me write what I couldn't say. I cannot thank you enough.
}

\declaration{
I, Stefan Collier, declare that this dissertation and the work presented in it are my own and has been generated by me as the result of my own original research.\\
I confirm that:\\
1. This work was done wholly or mainly while in candidature for a degree at this University;\\
2. Where any part of this dissertation has previously been submitted for any other qualification at this University or any other institution, this has been clearly stated;\\
3. Where I have consulted the published work of others, this is always clearly attributed;\\
4. Where I have quoted from the work of others, the source is always given. With the exception of such quotations, this dissertation is entirely my own work;\\
5. I have acknowledged all main sources of help;\\
6. Where the thesis is based on work done by myself jointly with others, I have made clear exactly what was done by others and what I have contributed myself;\\
7. Either none of this work has been published before submission, or parts of this work have been published by :\\
\\
Stefan Collier\\
April 2016
}
\tableofcontents
\listoffigures
\listoftables

\mainmatter
%% ----------------------------------------------------------------
%\include{Introduction}
%\include{Conclusions}
\include{chapters/1Project/main}
\include{chapters/2Lit/main}
\include{chapters/3Design/HighLevel}
\include{chapters/3Design/InDepth}
\include{chapters/4Impl/main}

\include{chapters/5Experiments/1/main}
\include{chapters/5Experiments/2/main}
\include{chapters/5Experiments/3/main}
\include{chapters/5Experiments/4/main}

\include{chapters/6Conclusion/main}

\appendix
\include{appendix/AppendixB}
\include{appendix/D/main}
\include{appendix/AppendixC}

\backmatter
\bibliographystyle{ecs}
\bibliography{ECS}
\end{document}
%% ----------------------------------------------------------------

\include{appendix/AppendixC}

\backmatter
\bibliographystyle{ecs}
\bibliography{ECS}
\end{document}
%% ----------------------------------------------------------------


 %% ----------------------------------------------------------------
%% Progress.tex
%% ---------------------------------------------------------------- 
\documentclass{ecsprogress}    % Use the progress Style
\graphicspath{{../figs/}}   % Location of your graphics files
    \usepackage{natbib}            % Use Natbib style for the refs.
\hypersetup{colorlinks=true}   % Set to false for black/white printing
\input{Definitions}            % Include your abbreviations



\usepackage{enumitem}% http://ctan.org/pkg/enumitem
\usepackage{multirow}
\usepackage{float}
\usepackage{amsmath}
\usepackage{multicol}
\usepackage{amssymb}
\usepackage[normalem]{ulem}
\useunder{\uline}{\ul}{}
\usepackage{wrapfig}


\usepackage[table,xcdraw]{xcolor}


%% ----------------------------------------------------------------
\begin{document}
\frontmatter
\title      {Heterogeneous Agent-based Model for Supermarket Competition}
\authors    {\texorpdfstring
             {\href{mailto:sc22g13@ecs.soton.ac.uk}{Stefan J. Collier}}
             {Stefan J. Collier}
            }
\addresses  {\groupname\\\deptname\\\univname}
\date       {\today}
\subject    {}
\keywords   {}
\supervisor {Dr. Maria Polukarov}
\examiner   {Professor Sheng Chen}

\maketitle
\begin{abstract}
This project aim was to model and analyse the effects of competitive pricing behaviors of grocery retailers on the British market. 

This was achieved by creating a multi-agent model, containing retailer and consumer agents. The heterogeneous crowd of retailers employs either a uniform pricing strategy or a ‘local price flexing’ strategy. The actions of these retailers are chosen by predicting the profit of each action, using a perceptron. Following on from the consideration of different economic models, a discrete model was developed so that software agents have a discrete environment to operate within. Within the model, it has been observed how supermarkets with differing behaviors affect a heterogeneous crowd of consumer agents. The model was implemented in Java with Python used to evaluate the results. 

The simulation displays good acceptance with real grocery market behavior, i.e. captures the performance of British retailers thus can be used to determine the impact of changes in their behavior on their competitors and consumers.Furthermore it can be used to provide insight into sustainability of volatile pricing strategies, providing a useful insight in volatility of British supermarket retail industry. 
\end{abstract}
\acknowledgements{
I would like to express my sincere gratitude to Dr Maria Polukarov for her guidance and support which provided me the freedom to take this research in the direction of my interest.\\
\\
I would also like to thank my family and friends for their encouragement and support. To those who quietly listened to my software complaints. To those who worked throughout the nights with me. To those who helped me write what I couldn't say. I cannot thank you enough.
}

\declaration{
I, Stefan Collier, declare that this dissertation and the work presented in it are my own and has been generated by me as the result of my own original research.\\
I confirm that:\\
1. This work was done wholly or mainly while in candidature for a degree at this University;\\
2. Where any part of this dissertation has previously been submitted for any other qualification at this University or any other institution, this has been clearly stated;\\
3. Where I have consulted the published work of others, this is always clearly attributed;\\
4. Where I have quoted from the work of others, the source is always given. With the exception of such quotations, this dissertation is entirely my own work;\\
5. I have acknowledged all main sources of help;\\
6. Where the thesis is based on work done by myself jointly with others, I have made clear exactly what was done by others and what I have contributed myself;\\
7. Either none of this work has been published before submission, or parts of this work have been published by :\\
\\
Stefan Collier\\
April 2016
}
\tableofcontents
\listoffigures
\listoftables

\mainmatter
%% ----------------------------------------------------------------
%\include{Introduction}
%\include{Conclusions}
 %% ----------------------------------------------------------------
%% Progress.tex
%% ---------------------------------------------------------------- 
\documentclass{ecsprogress}    % Use the progress Style
\graphicspath{{../figs/}}   % Location of your graphics files
    \usepackage{natbib}            % Use Natbib style for the refs.
\hypersetup{colorlinks=true}   % Set to false for black/white printing
\input{Definitions}            % Include your abbreviations



\usepackage{enumitem}% http://ctan.org/pkg/enumitem
\usepackage{multirow}
\usepackage{float}
\usepackage{amsmath}
\usepackage{multicol}
\usepackage{amssymb}
\usepackage[normalem]{ulem}
\useunder{\uline}{\ul}{}
\usepackage{wrapfig}


\usepackage[table,xcdraw]{xcolor}


%% ----------------------------------------------------------------
\begin{document}
\frontmatter
\title      {Heterogeneous Agent-based Model for Supermarket Competition}
\authors    {\texorpdfstring
             {\href{mailto:sc22g13@ecs.soton.ac.uk}{Stefan J. Collier}}
             {Stefan J. Collier}
            }
\addresses  {\groupname\\\deptname\\\univname}
\date       {\today}
\subject    {}
\keywords   {}
\supervisor {Dr. Maria Polukarov}
\examiner   {Professor Sheng Chen}

\maketitle
\begin{abstract}
This project aim was to model and analyse the effects of competitive pricing behaviors of grocery retailers on the British market. 

This was achieved by creating a multi-agent model, containing retailer and consumer agents. The heterogeneous crowd of retailers employs either a uniform pricing strategy or a ‘local price flexing’ strategy. The actions of these retailers are chosen by predicting the profit of each action, using a perceptron. Following on from the consideration of different economic models, a discrete model was developed so that software agents have a discrete environment to operate within. Within the model, it has been observed how supermarkets with differing behaviors affect a heterogeneous crowd of consumer agents. The model was implemented in Java with Python used to evaluate the results. 

The simulation displays good acceptance with real grocery market behavior, i.e. captures the performance of British retailers thus can be used to determine the impact of changes in their behavior on their competitors and consumers.Furthermore it can be used to provide insight into sustainability of volatile pricing strategies, providing a useful insight in volatility of British supermarket retail industry. 
\end{abstract}
\acknowledgements{
I would like to express my sincere gratitude to Dr Maria Polukarov for her guidance and support which provided me the freedom to take this research in the direction of my interest.\\
\\
I would also like to thank my family and friends for their encouragement and support. To those who quietly listened to my software complaints. To those who worked throughout the nights with me. To those who helped me write what I couldn't say. I cannot thank you enough.
}

\declaration{
I, Stefan Collier, declare that this dissertation and the work presented in it are my own and has been generated by me as the result of my own original research.\\
I confirm that:\\
1. This work was done wholly or mainly while in candidature for a degree at this University;\\
2. Where any part of this dissertation has previously been submitted for any other qualification at this University or any other institution, this has been clearly stated;\\
3. Where I have consulted the published work of others, this is always clearly attributed;\\
4. Where I have quoted from the work of others, the source is always given. With the exception of such quotations, this dissertation is entirely my own work;\\
5. I have acknowledged all main sources of help;\\
6. Where the thesis is based on work done by myself jointly with others, I have made clear exactly what was done by others and what I have contributed myself;\\
7. Either none of this work has been published before submission, or parts of this work have been published by :\\
\\
Stefan Collier\\
April 2016
}
\tableofcontents
\listoffigures
\listoftables

\mainmatter
%% ----------------------------------------------------------------
%\include{Introduction}
%\include{Conclusions}
\include{chapters/1Project/main}
\include{chapters/2Lit/main}
\include{chapters/3Design/HighLevel}
\include{chapters/3Design/InDepth}
\include{chapters/4Impl/main}

\include{chapters/5Experiments/1/main}
\include{chapters/5Experiments/2/main}
\include{chapters/5Experiments/3/main}
\include{chapters/5Experiments/4/main}

\include{chapters/6Conclusion/main}

\appendix
\include{appendix/AppendixB}
\include{appendix/D/main}
\include{appendix/AppendixC}

\backmatter
\bibliographystyle{ecs}
\bibliography{ECS}
\end{document}
%% ----------------------------------------------------------------

 %% ----------------------------------------------------------------
%% Progress.tex
%% ---------------------------------------------------------------- 
\documentclass{ecsprogress}    % Use the progress Style
\graphicspath{{../figs/}}   % Location of your graphics files
    \usepackage{natbib}            % Use Natbib style for the refs.
\hypersetup{colorlinks=true}   % Set to false for black/white printing
\input{Definitions}            % Include your abbreviations



\usepackage{enumitem}% http://ctan.org/pkg/enumitem
\usepackage{multirow}
\usepackage{float}
\usepackage{amsmath}
\usepackage{multicol}
\usepackage{amssymb}
\usepackage[normalem]{ulem}
\useunder{\uline}{\ul}{}
\usepackage{wrapfig}


\usepackage[table,xcdraw]{xcolor}


%% ----------------------------------------------------------------
\begin{document}
\frontmatter
\title      {Heterogeneous Agent-based Model for Supermarket Competition}
\authors    {\texorpdfstring
             {\href{mailto:sc22g13@ecs.soton.ac.uk}{Stefan J. Collier}}
             {Stefan J. Collier}
            }
\addresses  {\groupname\\\deptname\\\univname}
\date       {\today}
\subject    {}
\keywords   {}
\supervisor {Dr. Maria Polukarov}
\examiner   {Professor Sheng Chen}

\maketitle
\begin{abstract}
This project aim was to model and analyse the effects of competitive pricing behaviors of grocery retailers on the British market. 

This was achieved by creating a multi-agent model, containing retailer and consumer agents. The heterogeneous crowd of retailers employs either a uniform pricing strategy or a ‘local price flexing’ strategy. The actions of these retailers are chosen by predicting the profit of each action, using a perceptron. Following on from the consideration of different economic models, a discrete model was developed so that software agents have a discrete environment to operate within. Within the model, it has been observed how supermarkets with differing behaviors affect a heterogeneous crowd of consumer agents. The model was implemented in Java with Python used to evaluate the results. 

The simulation displays good acceptance with real grocery market behavior, i.e. captures the performance of British retailers thus can be used to determine the impact of changes in their behavior on their competitors and consumers.Furthermore it can be used to provide insight into sustainability of volatile pricing strategies, providing a useful insight in volatility of British supermarket retail industry. 
\end{abstract}
\acknowledgements{
I would like to express my sincere gratitude to Dr Maria Polukarov for her guidance and support which provided me the freedom to take this research in the direction of my interest.\\
\\
I would also like to thank my family and friends for their encouragement and support. To those who quietly listened to my software complaints. To those who worked throughout the nights with me. To those who helped me write what I couldn't say. I cannot thank you enough.
}

\declaration{
I, Stefan Collier, declare that this dissertation and the work presented in it are my own and has been generated by me as the result of my own original research.\\
I confirm that:\\
1. This work was done wholly or mainly while in candidature for a degree at this University;\\
2. Where any part of this dissertation has previously been submitted for any other qualification at this University or any other institution, this has been clearly stated;\\
3. Where I have consulted the published work of others, this is always clearly attributed;\\
4. Where I have quoted from the work of others, the source is always given. With the exception of such quotations, this dissertation is entirely my own work;\\
5. I have acknowledged all main sources of help;\\
6. Where the thesis is based on work done by myself jointly with others, I have made clear exactly what was done by others and what I have contributed myself;\\
7. Either none of this work has been published before submission, or parts of this work have been published by :\\
\\
Stefan Collier\\
April 2016
}
\tableofcontents
\listoffigures
\listoftables

\mainmatter
%% ----------------------------------------------------------------
%\include{Introduction}
%\include{Conclusions}
\include{chapters/1Project/main}
\include{chapters/2Lit/main}
\include{chapters/3Design/HighLevel}
\include{chapters/3Design/InDepth}
\include{chapters/4Impl/main}

\include{chapters/5Experiments/1/main}
\include{chapters/5Experiments/2/main}
\include{chapters/5Experiments/3/main}
\include{chapters/5Experiments/4/main}

\include{chapters/6Conclusion/main}

\appendix
\include{appendix/AppendixB}
\include{appendix/D/main}
\include{appendix/AppendixC}

\backmatter
\bibliographystyle{ecs}
\bibliography{ECS}
\end{document}
%% ----------------------------------------------------------------

\include{chapters/3Design/HighLevel}
\include{chapters/3Design/InDepth}
 %% ----------------------------------------------------------------
%% Progress.tex
%% ---------------------------------------------------------------- 
\documentclass{ecsprogress}    % Use the progress Style
\graphicspath{{../figs/}}   % Location of your graphics files
    \usepackage{natbib}            % Use Natbib style for the refs.
\hypersetup{colorlinks=true}   % Set to false for black/white printing
\input{Definitions}            % Include your abbreviations



\usepackage{enumitem}% http://ctan.org/pkg/enumitem
\usepackage{multirow}
\usepackage{float}
\usepackage{amsmath}
\usepackage{multicol}
\usepackage{amssymb}
\usepackage[normalem]{ulem}
\useunder{\uline}{\ul}{}
\usepackage{wrapfig}


\usepackage[table,xcdraw]{xcolor}


%% ----------------------------------------------------------------
\begin{document}
\frontmatter
\title      {Heterogeneous Agent-based Model for Supermarket Competition}
\authors    {\texorpdfstring
             {\href{mailto:sc22g13@ecs.soton.ac.uk}{Stefan J. Collier}}
             {Stefan J. Collier}
            }
\addresses  {\groupname\\\deptname\\\univname}
\date       {\today}
\subject    {}
\keywords   {}
\supervisor {Dr. Maria Polukarov}
\examiner   {Professor Sheng Chen}

\maketitle
\begin{abstract}
This project aim was to model and analyse the effects of competitive pricing behaviors of grocery retailers on the British market. 

This was achieved by creating a multi-agent model, containing retailer and consumer agents. The heterogeneous crowd of retailers employs either a uniform pricing strategy or a ‘local price flexing’ strategy. The actions of these retailers are chosen by predicting the profit of each action, using a perceptron. Following on from the consideration of different economic models, a discrete model was developed so that software agents have a discrete environment to operate within. Within the model, it has been observed how supermarkets with differing behaviors affect a heterogeneous crowd of consumer agents. The model was implemented in Java with Python used to evaluate the results. 

The simulation displays good acceptance with real grocery market behavior, i.e. captures the performance of British retailers thus can be used to determine the impact of changes in their behavior on their competitors and consumers.Furthermore it can be used to provide insight into sustainability of volatile pricing strategies, providing a useful insight in volatility of British supermarket retail industry. 
\end{abstract}
\acknowledgements{
I would like to express my sincere gratitude to Dr Maria Polukarov for her guidance and support which provided me the freedom to take this research in the direction of my interest.\\
\\
I would also like to thank my family and friends for their encouragement and support. To those who quietly listened to my software complaints. To those who worked throughout the nights with me. To those who helped me write what I couldn't say. I cannot thank you enough.
}

\declaration{
I, Stefan Collier, declare that this dissertation and the work presented in it are my own and has been generated by me as the result of my own original research.\\
I confirm that:\\
1. This work was done wholly or mainly while in candidature for a degree at this University;\\
2. Where any part of this dissertation has previously been submitted for any other qualification at this University or any other institution, this has been clearly stated;\\
3. Where I have consulted the published work of others, this is always clearly attributed;\\
4. Where I have quoted from the work of others, the source is always given. With the exception of such quotations, this dissertation is entirely my own work;\\
5. I have acknowledged all main sources of help;\\
6. Where the thesis is based on work done by myself jointly with others, I have made clear exactly what was done by others and what I have contributed myself;\\
7. Either none of this work has been published before submission, or parts of this work have been published by :\\
\\
Stefan Collier\\
April 2016
}
\tableofcontents
\listoffigures
\listoftables

\mainmatter
%% ----------------------------------------------------------------
%\include{Introduction}
%\include{Conclusions}
\include{chapters/1Project/main}
\include{chapters/2Lit/main}
\include{chapters/3Design/HighLevel}
\include{chapters/3Design/InDepth}
\include{chapters/4Impl/main}

\include{chapters/5Experiments/1/main}
\include{chapters/5Experiments/2/main}
\include{chapters/5Experiments/3/main}
\include{chapters/5Experiments/4/main}

\include{chapters/6Conclusion/main}

\appendix
\include{appendix/AppendixB}
\include{appendix/D/main}
\include{appendix/AppendixC}

\backmatter
\bibliographystyle{ecs}
\bibliography{ECS}
\end{document}
%% ----------------------------------------------------------------


 %% ----------------------------------------------------------------
%% Progress.tex
%% ---------------------------------------------------------------- 
\documentclass{ecsprogress}    % Use the progress Style
\graphicspath{{../figs/}}   % Location of your graphics files
    \usepackage{natbib}            % Use Natbib style for the refs.
\hypersetup{colorlinks=true}   % Set to false for black/white printing
\input{Definitions}            % Include your abbreviations



\usepackage{enumitem}% http://ctan.org/pkg/enumitem
\usepackage{multirow}
\usepackage{float}
\usepackage{amsmath}
\usepackage{multicol}
\usepackage{amssymb}
\usepackage[normalem]{ulem}
\useunder{\uline}{\ul}{}
\usepackage{wrapfig}


\usepackage[table,xcdraw]{xcolor}


%% ----------------------------------------------------------------
\begin{document}
\frontmatter
\title      {Heterogeneous Agent-based Model for Supermarket Competition}
\authors    {\texorpdfstring
             {\href{mailto:sc22g13@ecs.soton.ac.uk}{Stefan J. Collier}}
             {Stefan J. Collier}
            }
\addresses  {\groupname\\\deptname\\\univname}
\date       {\today}
\subject    {}
\keywords   {}
\supervisor {Dr. Maria Polukarov}
\examiner   {Professor Sheng Chen}

\maketitle
\begin{abstract}
This project aim was to model and analyse the effects of competitive pricing behaviors of grocery retailers on the British market. 

This was achieved by creating a multi-agent model, containing retailer and consumer agents. The heterogeneous crowd of retailers employs either a uniform pricing strategy or a ‘local price flexing’ strategy. The actions of these retailers are chosen by predicting the profit of each action, using a perceptron. Following on from the consideration of different economic models, a discrete model was developed so that software agents have a discrete environment to operate within. Within the model, it has been observed how supermarkets with differing behaviors affect a heterogeneous crowd of consumer agents. The model was implemented in Java with Python used to evaluate the results. 

The simulation displays good acceptance with real grocery market behavior, i.e. captures the performance of British retailers thus can be used to determine the impact of changes in their behavior on their competitors and consumers.Furthermore it can be used to provide insight into sustainability of volatile pricing strategies, providing a useful insight in volatility of British supermarket retail industry. 
\end{abstract}
\acknowledgements{
I would like to express my sincere gratitude to Dr Maria Polukarov for her guidance and support which provided me the freedom to take this research in the direction of my interest.\\
\\
I would also like to thank my family and friends for their encouragement and support. To those who quietly listened to my software complaints. To those who worked throughout the nights with me. To those who helped me write what I couldn't say. I cannot thank you enough.
}

\declaration{
I, Stefan Collier, declare that this dissertation and the work presented in it are my own and has been generated by me as the result of my own original research.\\
I confirm that:\\
1. This work was done wholly or mainly while in candidature for a degree at this University;\\
2. Where any part of this dissertation has previously been submitted for any other qualification at this University or any other institution, this has been clearly stated;\\
3. Where I have consulted the published work of others, this is always clearly attributed;\\
4. Where I have quoted from the work of others, the source is always given. With the exception of such quotations, this dissertation is entirely my own work;\\
5. I have acknowledged all main sources of help;\\
6. Where the thesis is based on work done by myself jointly with others, I have made clear exactly what was done by others and what I have contributed myself;\\
7. Either none of this work has been published before submission, or parts of this work have been published by :\\
\\
Stefan Collier\\
April 2016
}
\tableofcontents
\listoffigures
\listoftables

\mainmatter
%% ----------------------------------------------------------------
%\include{Introduction}
%\include{Conclusions}
\include{chapters/1Project/main}
\include{chapters/2Lit/main}
\include{chapters/3Design/HighLevel}
\include{chapters/3Design/InDepth}
\include{chapters/4Impl/main}

\include{chapters/5Experiments/1/main}
\include{chapters/5Experiments/2/main}
\include{chapters/5Experiments/3/main}
\include{chapters/5Experiments/4/main}

\include{chapters/6Conclusion/main}

\appendix
\include{appendix/AppendixB}
\include{appendix/D/main}
\include{appendix/AppendixC}

\backmatter
\bibliographystyle{ecs}
\bibliography{ECS}
\end{document}
%% ----------------------------------------------------------------

 %% ----------------------------------------------------------------
%% Progress.tex
%% ---------------------------------------------------------------- 
\documentclass{ecsprogress}    % Use the progress Style
\graphicspath{{../figs/}}   % Location of your graphics files
    \usepackage{natbib}            % Use Natbib style for the refs.
\hypersetup{colorlinks=true}   % Set to false for black/white printing
\input{Definitions}            % Include your abbreviations



\usepackage{enumitem}% http://ctan.org/pkg/enumitem
\usepackage{multirow}
\usepackage{float}
\usepackage{amsmath}
\usepackage{multicol}
\usepackage{amssymb}
\usepackage[normalem]{ulem}
\useunder{\uline}{\ul}{}
\usepackage{wrapfig}


\usepackage[table,xcdraw]{xcolor}


%% ----------------------------------------------------------------
\begin{document}
\frontmatter
\title      {Heterogeneous Agent-based Model for Supermarket Competition}
\authors    {\texorpdfstring
             {\href{mailto:sc22g13@ecs.soton.ac.uk}{Stefan J. Collier}}
             {Stefan J. Collier}
            }
\addresses  {\groupname\\\deptname\\\univname}
\date       {\today}
\subject    {}
\keywords   {}
\supervisor {Dr. Maria Polukarov}
\examiner   {Professor Sheng Chen}

\maketitle
\begin{abstract}
This project aim was to model and analyse the effects of competitive pricing behaviors of grocery retailers on the British market. 

This was achieved by creating a multi-agent model, containing retailer and consumer agents. The heterogeneous crowd of retailers employs either a uniform pricing strategy or a ‘local price flexing’ strategy. The actions of these retailers are chosen by predicting the profit of each action, using a perceptron. Following on from the consideration of different economic models, a discrete model was developed so that software agents have a discrete environment to operate within. Within the model, it has been observed how supermarkets with differing behaviors affect a heterogeneous crowd of consumer agents. The model was implemented in Java with Python used to evaluate the results. 

The simulation displays good acceptance with real grocery market behavior, i.e. captures the performance of British retailers thus can be used to determine the impact of changes in their behavior on their competitors and consumers.Furthermore it can be used to provide insight into sustainability of volatile pricing strategies, providing a useful insight in volatility of British supermarket retail industry. 
\end{abstract}
\acknowledgements{
I would like to express my sincere gratitude to Dr Maria Polukarov for her guidance and support which provided me the freedom to take this research in the direction of my interest.\\
\\
I would also like to thank my family and friends for their encouragement and support. To those who quietly listened to my software complaints. To those who worked throughout the nights with me. To those who helped me write what I couldn't say. I cannot thank you enough.
}

\declaration{
I, Stefan Collier, declare that this dissertation and the work presented in it are my own and has been generated by me as the result of my own original research.\\
I confirm that:\\
1. This work was done wholly or mainly while in candidature for a degree at this University;\\
2. Where any part of this dissertation has previously been submitted for any other qualification at this University or any other institution, this has been clearly stated;\\
3. Where I have consulted the published work of others, this is always clearly attributed;\\
4. Where I have quoted from the work of others, the source is always given. With the exception of such quotations, this dissertation is entirely my own work;\\
5. I have acknowledged all main sources of help;\\
6. Where the thesis is based on work done by myself jointly with others, I have made clear exactly what was done by others and what I have contributed myself;\\
7. Either none of this work has been published before submission, or parts of this work have been published by :\\
\\
Stefan Collier\\
April 2016
}
\tableofcontents
\listoffigures
\listoftables

\mainmatter
%% ----------------------------------------------------------------
%\include{Introduction}
%\include{Conclusions}
\include{chapters/1Project/main}
\include{chapters/2Lit/main}
\include{chapters/3Design/HighLevel}
\include{chapters/3Design/InDepth}
\include{chapters/4Impl/main}

\include{chapters/5Experiments/1/main}
\include{chapters/5Experiments/2/main}
\include{chapters/5Experiments/3/main}
\include{chapters/5Experiments/4/main}

\include{chapters/6Conclusion/main}

\appendix
\include{appendix/AppendixB}
\include{appendix/D/main}
\include{appendix/AppendixC}

\backmatter
\bibliographystyle{ecs}
\bibliography{ECS}
\end{document}
%% ----------------------------------------------------------------

 %% ----------------------------------------------------------------
%% Progress.tex
%% ---------------------------------------------------------------- 
\documentclass{ecsprogress}    % Use the progress Style
\graphicspath{{../figs/}}   % Location of your graphics files
    \usepackage{natbib}            % Use Natbib style for the refs.
\hypersetup{colorlinks=true}   % Set to false for black/white printing
\input{Definitions}            % Include your abbreviations



\usepackage{enumitem}% http://ctan.org/pkg/enumitem
\usepackage{multirow}
\usepackage{float}
\usepackage{amsmath}
\usepackage{multicol}
\usepackage{amssymb}
\usepackage[normalem]{ulem}
\useunder{\uline}{\ul}{}
\usepackage{wrapfig}


\usepackage[table,xcdraw]{xcolor}


%% ----------------------------------------------------------------
\begin{document}
\frontmatter
\title      {Heterogeneous Agent-based Model for Supermarket Competition}
\authors    {\texorpdfstring
             {\href{mailto:sc22g13@ecs.soton.ac.uk}{Stefan J. Collier}}
             {Stefan J. Collier}
            }
\addresses  {\groupname\\\deptname\\\univname}
\date       {\today}
\subject    {}
\keywords   {}
\supervisor {Dr. Maria Polukarov}
\examiner   {Professor Sheng Chen}

\maketitle
\begin{abstract}
This project aim was to model and analyse the effects of competitive pricing behaviors of grocery retailers on the British market. 

This was achieved by creating a multi-agent model, containing retailer and consumer agents. The heterogeneous crowd of retailers employs either a uniform pricing strategy or a ‘local price flexing’ strategy. The actions of these retailers are chosen by predicting the profit of each action, using a perceptron. Following on from the consideration of different economic models, a discrete model was developed so that software agents have a discrete environment to operate within. Within the model, it has been observed how supermarkets with differing behaviors affect a heterogeneous crowd of consumer agents. The model was implemented in Java with Python used to evaluate the results. 

The simulation displays good acceptance with real grocery market behavior, i.e. captures the performance of British retailers thus can be used to determine the impact of changes in their behavior on their competitors and consumers.Furthermore it can be used to provide insight into sustainability of volatile pricing strategies, providing a useful insight in volatility of British supermarket retail industry. 
\end{abstract}
\acknowledgements{
I would like to express my sincere gratitude to Dr Maria Polukarov for her guidance and support which provided me the freedom to take this research in the direction of my interest.\\
\\
I would also like to thank my family and friends for their encouragement and support. To those who quietly listened to my software complaints. To those who worked throughout the nights with me. To those who helped me write what I couldn't say. I cannot thank you enough.
}

\declaration{
I, Stefan Collier, declare that this dissertation and the work presented in it are my own and has been generated by me as the result of my own original research.\\
I confirm that:\\
1. This work was done wholly or mainly while in candidature for a degree at this University;\\
2. Where any part of this dissertation has previously been submitted for any other qualification at this University or any other institution, this has been clearly stated;\\
3. Where I have consulted the published work of others, this is always clearly attributed;\\
4. Where I have quoted from the work of others, the source is always given. With the exception of such quotations, this dissertation is entirely my own work;\\
5. I have acknowledged all main sources of help;\\
6. Where the thesis is based on work done by myself jointly with others, I have made clear exactly what was done by others and what I have contributed myself;\\
7. Either none of this work has been published before submission, or parts of this work have been published by :\\
\\
Stefan Collier\\
April 2016
}
\tableofcontents
\listoffigures
\listoftables

\mainmatter
%% ----------------------------------------------------------------
%\include{Introduction}
%\include{Conclusions}
\include{chapters/1Project/main}
\include{chapters/2Lit/main}
\include{chapters/3Design/HighLevel}
\include{chapters/3Design/InDepth}
\include{chapters/4Impl/main}

\include{chapters/5Experiments/1/main}
\include{chapters/5Experiments/2/main}
\include{chapters/5Experiments/3/main}
\include{chapters/5Experiments/4/main}

\include{chapters/6Conclusion/main}

\appendix
\include{appendix/AppendixB}
\include{appendix/D/main}
\include{appendix/AppendixC}

\backmatter
\bibliographystyle{ecs}
\bibliography{ECS}
\end{document}
%% ----------------------------------------------------------------

 %% ----------------------------------------------------------------
%% Progress.tex
%% ---------------------------------------------------------------- 
\documentclass{ecsprogress}    % Use the progress Style
\graphicspath{{../figs/}}   % Location of your graphics files
    \usepackage{natbib}            % Use Natbib style for the refs.
\hypersetup{colorlinks=true}   % Set to false for black/white printing
\input{Definitions}            % Include your abbreviations



\usepackage{enumitem}% http://ctan.org/pkg/enumitem
\usepackage{multirow}
\usepackage{float}
\usepackage{amsmath}
\usepackage{multicol}
\usepackage{amssymb}
\usepackage[normalem]{ulem}
\useunder{\uline}{\ul}{}
\usepackage{wrapfig}


\usepackage[table,xcdraw]{xcolor}


%% ----------------------------------------------------------------
\begin{document}
\frontmatter
\title      {Heterogeneous Agent-based Model for Supermarket Competition}
\authors    {\texorpdfstring
             {\href{mailto:sc22g13@ecs.soton.ac.uk}{Stefan J. Collier}}
             {Stefan J. Collier}
            }
\addresses  {\groupname\\\deptname\\\univname}
\date       {\today}
\subject    {}
\keywords   {}
\supervisor {Dr. Maria Polukarov}
\examiner   {Professor Sheng Chen}

\maketitle
\begin{abstract}
This project aim was to model and analyse the effects of competitive pricing behaviors of grocery retailers on the British market. 

This was achieved by creating a multi-agent model, containing retailer and consumer agents. The heterogeneous crowd of retailers employs either a uniform pricing strategy or a ‘local price flexing’ strategy. The actions of these retailers are chosen by predicting the profit of each action, using a perceptron. Following on from the consideration of different economic models, a discrete model was developed so that software agents have a discrete environment to operate within. Within the model, it has been observed how supermarkets with differing behaviors affect a heterogeneous crowd of consumer agents. The model was implemented in Java with Python used to evaluate the results. 

The simulation displays good acceptance with real grocery market behavior, i.e. captures the performance of British retailers thus can be used to determine the impact of changes in their behavior on their competitors and consumers.Furthermore it can be used to provide insight into sustainability of volatile pricing strategies, providing a useful insight in volatility of British supermarket retail industry. 
\end{abstract}
\acknowledgements{
I would like to express my sincere gratitude to Dr Maria Polukarov for her guidance and support which provided me the freedom to take this research in the direction of my interest.\\
\\
I would also like to thank my family and friends for their encouragement and support. To those who quietly listened to my software complaints. To those who worked throughout the nights with me. To those who helped me write what I couldn't say. I cannot thank you enough.
}

\declaration{
I, Stefan Collier, declare that this dissertation and the work presented in it are my own and has been generated by me as the result of my own original research.\\
I confirm that:\\
1. This work was done wholly or mainly while in candidature for a degree at this University;\\
2. Where any part of this dissertation has previously been submitted for any other qualification at this University or any other institution, this has been clearly stated;\\
3. Where I have consulted the published work of others, this is always clearly attributed;\\
4. Where I have quoted from the work of others, the source is always given. With the exception of such quotations, this dissertation is entirely my own work;\\
5. I have acknowledged all main sources of help;\\
6. Where the thesis is based on work done by myself jointly with others, I have made clear exactly what was done by others and what I have contributed myself;\\
7. Either none of this work has been published before submission, or parts of this work have been published by :\\
\\
Stefan Collier\\
April 2016
}
\tableofcontents
\listoffigures
\listoftables

\mainmatter
%% ----------------------------------------------------------------
%\include{Introduction}
%\include{Conclusions}
\include{chapters/1Project/main}
\include{chapters/2Lit/main}
\include{chapters/3Design/HighLevel}
\include{chapters/3Design/InDepth}
\include{chapters/4Impl/main}

\include{chapters/5Experiments/1/main}
\include{chapters/5Experiments/2/main}
\include{chapters/5Experiments/3/main}
\include{chapters/5Experiments/4/main}

\include{chapters/6Conclusion/main}

\appendix
\include{appendix/AppendixB}
\include{appendix/D/main}
\include{appendix/AppendixC}

\backmatter
\bibliographystyle{ecs}
\bibliography{ECS}
\end{document}
%% ----------------------------------------------------------------


 %% ----------------------------------------------------------------
%% Progress.tex
%% ---------------------------------------------------------------- 
\documentclass{ecsprogress}    % Use the progress Style
\graphicspath{{../figs/}}   % Location of your graphics files
    \usepackage{natbib}            % Use Natbib style for the refs.
\hypersetup{colorlinks=true}   % Set to false for black/white printing
\input{Definitions}            % Include your abbreviations



\usepackage{enumitem}% http://ctan.org/pkg/enumitem
\usepackage{multirow}
\usepackage{float}
\usepackage{amsmath}
\usepackage{multicol}
\usepackage{amssymb}
\usepackage[normalem]{ulem}
\useunder{\uline}{\ul}{}
\usepackage{wrapfig}


\usepackage[table,xcdraw]{xcolor}


%% ----------------------------------------------------------------
\begin{document}
\frontmatter
\title      {Heterogeneous Agent-based Model for Supermarket Competition}
\authors    {\texorpdfstring
             {\href{mailto:sc22g13@ecs.soton.ac.uk}{Stefan J. Collier}}
             {Stefan J. Collier}
            }
\addresses  {\groupname\\\deptname\\\univname}
\date       {\today}
\subject    {}
\keywords   {}
\supervisor {Dr. Maria Polukarov}
\examiner   {Professor Sheng Chen}

\maketitle
\begin{abstract}
This project aim was to model and analyse the effects of competitive pricing behaviors of grocery retailers on the British market. 

This was achieved by creating a multi-agent model, containing retailer and consumer agents. The heterogeneous crowd of retailers employs either a uniform pricing strategy or a ‘local price flexing’ strategy. The actions of these retailers are chosen by predicting the profit of each action, using a perceptron. Following on from the consideration of different economic models, a discrete model was developed so that software agents have a discrete environment to operate within. Within the model, it has been observed how supermarkets with differing behaviors affect a heterogeneous crowd of consumer agents. The model was implemented in Java with Python used to evaluate the results. 

The simulation displays good acceptance with real grocery market behavior, i.e. captures the performance of British retailers thus can be used to determine the impact of changes in their behavior on their competitors and consumers.Furthermore it can be used to provide insight into sustainability of volatile pricing strategies, providing a useful insight in volatility of British supermarket retail industry. 
\end{abstract}
\acknowledgements{
I would like to express my sincere gratitude to Dr Maria Polukarov for her guidance and support which provided me the freedom to take this research in the direction of my interest.\\
\\
I would also like to thank my family and friends for their encouragement and support. To those who quietly listened to my software complaints. To those who worked throughout the nights with me. To those who helped me write what I couldn't say. I cannot thank you enough.
}

\declaration{
I, Stefan Collier, declare that this dissertation and the work presented in it are my own and has been generated by me as the result of my own original research.\\
I confirm that:\\
1. This work was done wholly or mainly while in candidature for a degree at this University;\\
2. Where any part of this dissertation has previously been submitted for any other qualification at this University or any other institution, this has been clearly stated;\\
3. Where I have consulted the published work of others, this is always clearly attributed;\\
4. Where I have quoted from the work of others, the source is always given. With the exception of such quotations, this dissertation is entirely my own work;\\
5. I have acknowledged all main sources of help;\\
6. Where the thesis is based on work done by myself jointly with others, I have made clear exactly what was done by others and what I have contributed myself;\\
7. Either none of this work has been published before submission, or parts of this work have been published by :\\
\\
Stefan Collier\\
April 2016
}
\tableofcontents
\listoffigures
\listoftables

\mainmatter
%% ----------------------------------------------------------------
%\include{Introduction}
%\include{Conclusions}
\include{chapters/1Project/main}
\include{chapters/2Lit/main}
\include{chapters/3Design/HighLevel}
\include{chapters/3Design/InDepth}
\include{chapters/4Impl/main}

\include{chapters/5Experiments/1/main}
\include{chapters/5Experiments/2/main}
\include{chapters/5Experiments/3/main}
\include{chapters/5Experiments/4/main}

\include{chapters/6Conclusion/main}

\appendix
\include{appendix/AppendixB}
\include{appendix/D/main}
\include{appendix/AppendixC}

\backmatter
\bibliographystyle{ecs}
\bibliography{ECS}
\end{document}
%% ----------------------------------------------------------------


\appendix
\include{appendix/AppendixB}
 %% ----------------------------------------------------------------
%% Progress.tex
%% ---------------------------------------------------------------- 
\documentclass{ecsprogress}    % Use the progress Style
\graphicspath{{../figs/}}   % Location of your graphics files
    \usepackage{natbib}            % Use Natbib style for the refs.
\hypersetup{colorlinks=true}   % Set to false for black/white printing
\input{Definitions}            % Include your abbreviations



\usepackage{enumitem}% http://ctan.org/pkg/enumitem
\usepackage{multirow}
\usepackage{float}
\usepackage{amsmath}
\usepackage{multicol}
\usepackage{amssymb}
\usepackage[normalem]{ulem}
\useunder{\uline}{\ul}{}
\usepackage{wrapfig}


\usepackage[table,xcdraw]{xcolor}


%% ----------------------------------------------------------------
\begin{document}
\frontmatter
\title      {Heterogeneous Agent-based Model for Supermarket Competition}
\authors    {\texorpdfstring
             {\href{mailto:sc22g13@ecs.soton.ac.uk}{Stefan J. Collier}}
             {Stefan J. Collier}
            }
\addresses  {\groupname\\\deptname\\\univname}
\date       {\today}
\subject    {}
\keywords   {}
\supervisor {Dr. Maria Polukarov}
\examiner   {Professor Sheng Chen}

\maketitle
\begin{abstract}
This project aim was to model and analyse the effects of competitive pricing behaviors of grocery retailers on the British market. 

This was achieved by creating a multi-agent model, containing retailer and consumer agents. The heterogeneous crowd of retailers employs either a uniform pricing strategy or a ‘local price flexing’ strategy. The actions of these retailers are chosen by predicting the profit of each action, using a perceptron. Following on from the consideration of different economic models, a discrete model was developed so that software agents have a discrete environment to operate within. Within the model, it has been observed how supermarkets with differing behaviors affect a heterogeneous crowd of consumer agents. The model was implemented in Java with Python used to evaluate the results. 

The simulation displays good acceptance with real grocery market behavior, i.e. captures the performance of British retailers thus can be used to determine the impact of changes in their behavior on their competitors and consumers.Furthermore it can be used to provide insight into sustainability of volatile pricing strategies, providing a useful insight in volatility of British supermarket retail industry. 
\end{abstract}
\acknowledgements{
I would like to express my sincere gratitude to Dr Maria Polukarov for her guidance and support which provided me the freedom to take this research in the direction of my interest.\\
\\
I would also like to thank my family and friends for their encouragement and support. To those who quietly listened to my software complaints. To those who worked throughout the nights with me. To those who helped me write what I couldn't say. I cannot thank you enough.
}

\declaration{
I, Stefan Collier, declare that this dissertation and the work presented in it are my own and has been generated by me as the result of my own original research.\\
I confirm that:\\
1. This work was done wholly or mainly while in candidature for a degree at this University;\\
2. Where any part of this dissertation has previously been submitted for any other qualification at this University or any other institution, this has been clearly stated;\\
3. Where I have consulted the published work of others, this is always clearly attributed;\\
4. Where I have quoted from the work of others, the source is always given. With the exception of such quotations, this dissertation is entirely my own work;\\
5. I have acknowledged all main sources of help;\\
6. Where the thesis is based on work done by myself jointly with others, I have made clear exactly what was done by others and what I have contributed myself;\\
7. Either none of this work has been published before submission, or parts of this work have been published by :\\
\\
Stefan Collier\\
April 2016
}
\tableofcontents
\listoffigures
\listoftables

\mainmatter
%% ----------------------------------------------------------------
%\include{Introduction}
%\include{Conclusions}
\include{chapters/1Project/main}
\include{chapters/2Lit/main}
\include{chapters/3Design/HighLevel}
\include{chapters/3Design/InDepth}
\include{chapters/4Impl/main}

\include{chapters/5Experiments/1/main}
\include{chapters/5Experiments/2/main}
\include{chapters/5Experiments/3/main}
\include{chapters/5Experiments/4/main}

\include{chapters/6Conclusion/main}

\appendix
\include{appendix/AppendixB}
\include{appendix/D/main}
\include{appendix/AppendixC}

\backmatter
\bibliographystyle{ecs}
\bibliography{ECS}
\end{document}
%% ----------------------------------------------------------------

\include{appendix/AppendixC}

\backmatter
\bibliographystyle{ecs}
\bibliography{ECS}
\end{document}
%% ----------------------------------------------------------------

 %% ----------------------------------------------------------------
%% Progress.tex
%% ---------------------------------------------------------------- 
\documentclass{ecsprogress}    % Use the progress Style
\graphicspath{{../figs/}}   % Location of your graphics files
    \usepackage{natbib}            % Use Natbib style for the refs.
\hypersetup{colorlinks=true}   % Set to false for black/white printing
\input{Definitions}            % Include your abbreviations



\usepackage{enumitem}% http://ctan.org/pkg/enumitem
\usepackage{multirow}
\usepackage{float}
\usepackage{amsmath}
\usepackage{multicol}
\usepackage{amssymb}
\usepackage[normalem]{ulem}
\useunder{\uline}{\ul}{}
\usepackage{wrapfig}


\usepackage[table,xcdraw]{xcolor}


%% ----------------------------------------------------------------
\begin{document}
\frontmatter
\title      {Heterogeneous Agent-based Model for Supermarket Competition}
\authors    {\texorpdfstring
             {\href{mailto:sc22g13@ecs.soton.ac.uk}{Stefan J. Collier}}
             {Stefan J. Collier}
            }
\addresses  {\groupname\\\deptname\\\univname}
\date       {\today}
\subject    {}
\keywords   {}
\supervisor {Dr. Maria Polukarov}
\examiner   {Professor Sheng Chen}

\maketitle
\begin{abstract}
This project aim was to model and analyse the effects of competitive pricing behaviors of grocery retailers on the British market. 

This was achieved by creating a multi-agent model, containing retailer and consumer agents. The heterogeneous crowd of retailers employs either a uniform pricing strategy or a ‘local price flexing’ strategy. The actions of these retailers are chosen by predicting the profit of each action, using a perceptron. Following on from the consideration of different economic models, a discrete model was developed so that software agents have a discrete environment to operate within. Within the model, it has been observed how supermarkets with differing behaviors affect a heterogeneous crowd of consumer agents. The model was implemented in Java with Python used to evaluate the results. 

The simulation displays good acceptance with real grocery market behavior, i.e. captures the performance of British retailers thus can be used to determine the impact of changes in their behavior on their competitors and consumers.Furthermore it can be used to provide insight into sustainability of volatile pricing strategies, providing a useful insight in volatility of British supermarket retail industry. 
\end{abstract}
\acknowledgements{
I would like to express my sincere gratitude to Dr Maria Polukarov for her guidance and support which provided me the freedom to take this research in the direction of my interest.\\
\\
I would also like to thank my family and friends for their encouragement and support. To those who quietly listened to my software complaints. To those who worked throughout the nights with me. To those who helped me write what I couldn't say. I cannot thank you enough.
}

\declaration{
I, Stefan Collier, declare that this dissertation and the work presented in it are my own and has been generated by me as the result of my own original research.\\
I confirm that:\\
1. This work was done wholly or mainly while in candidature for a degree at this University;\\
2. Where any part of this dissertation has previously been submitted for any other qualification at this University or any other institution, this has been clearly stated;\\
3. Where I have consulted the published work of others, this is always clearly attributed;\\
4. Where I have quoted from the work of others, the source is always given. With the exception of such quotations, this dissertation is entirely my own work;\\
5. I have acknowledged all main sources of help;\\
6. Where the thesis is based on work done by myself jointly with others, I have made clear exactly what was done by others and what I have contributed myself;\\
7. Either none of this work has been published before submission, or parts of this work have been published by :\\
\\
Stefan Collier\\
April 2016
}
\tableofcontents
\listoffigures
\listoftables

\mainmatter
%% ----------------------------------------------------------------
%\include{Introduction}
%\include{Conclusions}
 %% ----------------------------------------------------------------
%% Progress.tex
%% ---------------------------------------------------------------- 
\documentclass{ecsprogress}    % Use the progress Style
\graphicspath{{../figs/}}   % Location of your graphics files
    \usepackage{natbib}            % Use Natbib style for the refs.
\hypersetup{colorlinks=true}   % Set to false for black/white printing
\input{Definitions}            % Include your abbreviations



\usepackage{enumitem}% http://ctan.org/pkg/enumitem
\usepackage{multirow}
\usepackage{float}
\usepackage{amsmath}
\usepackage{multicol}
\usepackage{amssymb}
\usepackage[normalem]{ulem}
\useunder{\uline}{\ul}{}
\usepackage{wrapfig}


\usepackage[table,xcdraw]{xcolor}


%% ----------------------------------------------------------------
\begin{document}
\frontmatter
\title      {Heterogeneous Agent-based Model for Supermarket Competition}
\authors    {\texorpdfstring
             {\href{mailto:sc22g13@ecs.soton.ac.uk}{Stefan J. Collier}}
             {Stefan J. Collier}
            }
\addresses  {\groupname\\\deptname\\\univname}
\date       {\today}
\subject    {}
\keywords   {}
\supervisor {Dr. Maria Polukarov}
\examiner   {Professor Sheng Chen}

\maketitle
\begin{abstract}
This project aim was to model and analyse the effects of competitive pricing behaviors of grocery retailers on the British market. 

This was achieved by creating a multi-agent model, containing retailer and consumer agents. The heterogeneous crowd of retailers employs either a uniform pricing strategy or a ‘local price flexing’ strategy. The actions of these retailers are chosen by predicting the profit of each action, using a perceptron. Following on from the consideration of different economic models, a discrete model was developed so that software agents have a discrete environment to operate within. Within the model, it has been observed how supermarkets with differing behaviors affect a heterogeneous crowd of consumer agents. The model was implemented in Java with Python used to evaluate the results. 

The simulation displays good acceptance with real grocery market behavior, i.e. captures the performance of British retailers thus can be used to determine the impact of changes in their behavior on their competitors and consumers.Furthermore it can be used to provide insight into sustainability of volatile pricing strategies, providing a useful insight in volatility of British supermarket retail industry. 
\end{abstract}
\acknowledgements{
I would like to express my sincere gratitude to Dr Maria Polukarov for her guidance and support which provided me the freedom to take this research in the direction of my interest.\\
\\
I would also like to thank my family and friends for their encouragement and support. To those who quietly listened to my software complaints. To those who worked throughout the nights with me. To those who helped me write what I couldn't say. I cannot thank you enough.
}

\declaration{
I, Stefan Collier, declare that this dissertation and the work presented in it are my own and has been generated by me as the result of my own original research.\\
I confirm that:\\
1. This work was done wholly or mainly while in candidature for a degree at this University;\\
2. Where any part of this dissertation has previously been submitted for any other qualification at this University or any other institution, this has been clearly stated;\\
3. Where I have consulted the published work of others, this is always clearly attributed;\\
4. Where I have quoted from the work of others, the source is always given. With the exception of such quotations, this dissertation is entirely my own work;\\
5. I have acknowledged all main sources of help;\\
6. Where the thesis is based on work done by myself jointly with others, I have made clear exactly what was done by others and what I have contributed myself;\\
7. Either none of this work has been published before submission, or parts of this work have been published by :\\
\\
Stefan Collier\\
April 2016
}
\tableofcontents
\listoffigures
\listoftables

\mainmatter
%% ----------------------------------------------------------------
%\include{Introduction}
%\include{Conclusions}
\include{chapters/1Project/main}
\include{chapters/2Lit/main}
\include{chapters/3Design/HighLevel}
\include{chapters/3Design/InDepth}
\include{chapters/4Impl/main}

\include{chapters/5Experiments/1/main}
\include{chapters/5Experiments/2/main}
\include{chapters/5Experiments/3/main}
\include{chapters/5Experiments/4/main}

\include{chapters/6Conclusion/main}

\appendix
\include{appendix/AppendixB}
\include{appendix/D/main}
\include{appendix/AppendixC}

\backmatter
\bibliographystyle{ecs}
\bibliography{ECS}
\end{document}
%% ----------------------------------------------------------------

 %% ----------------------------------------------------------------
%% Progress.tex
%% ---------------------------------------------------------------- 
\documentclass{ecsprogress}    % Use the progress Style
\graphicspath{{../figs/}}   % Location of your graphics files
    \usepackage{natbib}            % Use Natbib style for the refs.
\hypersetup{colorlinks=true}   % Set to false for black/white printing
\input{Definitions}            % Include your abbreviations



\usepackage{enumitem}% http://ctan.org/pkg/enumitem
\usepackage{multirow}
\usepackage{float}
\usepackage{amsmath}
\usepackage{multicol}
\usepackage{amssymb}
\usepackage[normalem]{ulem}
\useunder{\uline}{\ul}{}
\usepackage{wrapfig}


\usepackage[table,xcdraw]{xcolor}


%% ----------------------------------------------------------------
\begin{document}
\frontmatter
\title      {Heterogeneous Agent-based Model for Supermarket Competition}
\authors    {\texorpdfstring
             {\href{mailto:sc22g13@ecs.soton.ac.uk}{Stefan J. Collier}}
             {Stefan J. Collier}
            }
\addresses  {\groupname\\\deptname\\\univname}
\date       {\today}
\subject    {}
\keywords   {}
\supervisor {Dr. Maria Polukarov}
\examiner   {Professor Sheng Chen}

\maketitle
\begin{abstract}
This project aim was to model and analyse the effects of competitive pricing behaviors of grocery retailers on the British market. 

This was achieved by creating a multi-agent model, containing retailer and consumer agents. The heterogeneous crowd of retailers employs either a uniform pricing strategy or a ‘local price flexing’ strategy. The actions of these retailers are chosen by predicting the profit of each action, using a perceptron. Following on from the consideration of different economic models, a discrete model was developed so that software agents have a discrete environment to operate within. Within the model, it has been observed how supermarkets with differing behaviors affect a heterogeneous crowd of consumer agents. The model was implemented in Java with Python used to evaluate the results. 

The simulation displays good acceptance with real grocery market behavior, i.e. captures the performance of British retailers thus can be used to determine the impact of changes in their behavior on their competitors and consumers.Furthermore it can be used to provide insight into sustainability of volatile pricing strategies, providing a useful insight in volatility of British supermarket retail industry. 
\end{abstract}
\acknowledgements{
I would like to express my sincere gratitude to Dr Maria Polukarov for her guidance and support which provided me the freedom to take this research in the direction of my interest.\\
\\
I would also like to thank my family and friends for their encouragement and support. To those who quietly listened to my software complaints. To those who worked throughout the nights with me. To those who helped me write what I couldn't say. I cannot thank you enough.
}

\declaration{
I, Stefan Collier, declare that this dissertation and the work presented in it are my own and has been generated by me as the result of my own original research.\\
I confirm that:\\
1. This work was done wholly or mainly while in candidature for a degree at this University;\\
2. Where any part of this dissertation has previously been submitted for any other qualification at this University or any other institution, this has been clearly stated;\\
3. Where I have consulted the published work of others, this is always clearly attributed;\\
4. Where I have quoted from the work of others, the source is always given. With the exception of such quotations, this dissertation is entirely my own work;\\
5. I have acknowledged all main sources of help;\\
6. Where the thesis is based on work done by myself jointly with others, I have made clear exactly what was done by others and what I have contributed myself;\\
7. Either none of this work has been published before submission, or parts of this work have been published by :\\
\\
Stefan Collier\\
April 2016
}
\tableofcontents
\listoffigures
\listoftables

\mainmatter
%% ----------------------------------------------------------------
%\include{Introduction}
%\include{Conclusions}
\include{chapters/1Project/main}
\include{chapters/2Lit/main}
\include{chapters/3Design/HighLevel}
\include{chapters/3Design/InDepth}
\include{chapters/4Impl/main}

\include{chapters/5Experiments/1/main}
\include{chapters/5Experiments/2/main}
\include{chapters/5Experiments/3/main}
\include{chapters/5Experiments/4/main}

\include{chapters/6Conclusion/main}

\appendix
\include{appendix/AppendixB}
\include{appendix/D/main}
\include{appendix/AppendixC}

\backmatter
\bibliographystyle{ecs}
\bibliography{ECS}
\end{document}
%% ----------------------------------------------------------------

\include{chapters/3Design/HighLevel}
\include{chapters/3Design/InDepth}
 %% ----------------------------------------------------------------
%% Progress.tex
%% ---------------------------------------------------------------- 
\documentclass{ecsprogress}    % Use the progress Style
\graphicspath{{../figs/}}   % Location of your graphics files
    \usepackage{natbib}            % Use Natbib style for the refs.
\hypersetup{colorlinks=true}   % Set to false for black/white printing
\input{Definitions}            % Include your abbreviations



\usepackage{enumitem}% http://ctan.org/pkg/enumitem
\usepackage{multirow}
\usepackage{float}
\usepackage{amsmath}
\usepackage{multicol}
\usepackage{amssymb}
\usepackage[normalem]{ulem}
\useunder{\uline}{\ul}{}
\usepackage{wrapfig}


\usepackage[table,xcdraw]{xcolor}


%% ----------------------------------------------------------------
\begin{document}
\frontmatter
\title      {Heterogeneous Agent-based Model for Supermarket Competition}
\authors    {\texorpdfstring
             {\href{mailto:sc22g13@ecs.soton.ac.uk}{Stefan J. Collier}}
             {Stefan J. Collier}
            }
\addresses  {\groupname\\\deptname\\\univname}
\date       {\today}
\subject    {}
\keywords   {}
\supervisor {Dr. Maria Polukarov}
\examiner   {Professor Sheng Chen}

\maketitle
\begin{abstract}
This project aim was to model and analyse the effects of competitive pricing behaviors of grocery retailers on the British market. 

This was achieved by creating a multi-agent model, containing retailer and consumer agents. The heterogeneous crowd of retailers employs either a uniform pricing strategy or a ‘local price flexing’ strategy. The actions of these retailers are chosen by predicting the profit of each action, using a perceptron. Following on from the consideration of different economic models, a discrete model was developed so that software agents have a discrete environment to operate within. Within the model, it has been observed how supermarkets with differing behaviors affect a heterogeneous crowd of consumer agents. The model was implemented in Java with Python used to evaluate the results. 

The simulation displays good acceptance with real grocery market behavior, i.e. captures the performance of British retailers thus can be used to determine the impact of changes in their behavior on their competitors and consumers.Furthermore it can be used to provide insight into sustainability of volatile pricing strategies, providing a useful insight in volatility of British supermarket retail industry. 
\end{abstract}
\acknowledgements{
I would like to express my sincere gratitude to Dr Maria Polukarov for her guidance and support which provided me the freedom to take this research in the direction of my interest.\\
\\
I would also like to thank my family and friends for their encouragement and support. To those who quietly listened to my software complaints. To those who worked throughout the nights with me. To those who helped me write what I couldn't say. I cannot thank you enough.
}

\declaration{
I, Stefan Collier, declare that this dissertation and the work presented in it are my own and has been generated by me as the result of my own original research.\\
I confirm that:\\
1. This work was done wholly or mainly while in candidature for a degree at this University;\\
2. Where any part of this dissertation has previously been submitted for any other qualification at this University or any other institution, this has been clearly stated;\\
3. Where I have consulted the published work of others, this is always clearly attributed;\\
4. Where I have quoted from the work of others, the source is always given. With the exception of such quotations, this dissertation is entirely my own work;\\
5. I have acknowledged all main sources of help;\\
6. Where the thesis is based on work done by myself jointly with others, I have made clear exactly what was done by others and what I have contributed myself;\\
7. Either none of this work has been published before submission, or parts of this work have been published by :\\
\\
Stefan Collier\\
April 2016
}
\tableofcontents
\listoffigures
\listoftables

\mainmatter
%% ----------------------------------------------------------------
%\include{Introduction}
%\include{Conclusions}
\include{chapters/1Project/main}
\include{chapters/2Lit/main}
\include{chapters/3Design/HighLevel}
\include{chapters/3Design/InDepth}
\include{chapters/4Impl/main}

\include{chapters/5Experiments/1/main}
\include{chapters/5Experiments/2/main}
\include{chapters/5Experiments/3/main}
\include{chapters/5Experiments/4/main}

\include{chapters/6Conclusion/main}

\appendix
\include{appendix/AppendixB}
\include{appendix/D/main}
\include{appendix/AppendixC}

\backmatter
\bibliographystyle{ecs}
\bibliography{ECS}
\end{document}
%% ----------------------------------------------------------------


 %% ----------------------------------------------------------------
%% Progress.tex
%% ---------------------------------------------------------------- 
\documentclass{ecsprogress}    % Use the progress Style
\graphicspath{{../figs/}}   % Location of your graphics files
    \usepackage{natbib}            % Use Natbib style for the refs.
\hypersetup{colorlinks=true}   % Set to false for black/white printing
\input{Definitions}            % Include your abbreviations



\usepackage{enumitem}% http://ctan.org/pkg/enumitem
\usepackage{multirow}
\usepackage{float}
\usepackage{amsmath}
\usepackage{multicol}
\usepackage{amssymb}
\usepackage[normalem]{ulem}
\useunder{\uline}{\ul}{}
\usepackage{wrapfig}


\usepackage[table,xcdraw]{xcolor}


%% ----------------------------------------------------------------
\begin{document}
\frontmatter
\title      {Heterogeneous Agent-based Model for Supermarket Competition}
\authors    {\texorpdfstring
             {\href{mailto:sc22g13@ecs.soton.ac.uk}{Stefan J. Collier}}
             {Stefan J. Collier}
            }
\addresses  {\groupname\\\deptname\\\univname}
\date       {\today}
\subject    {}
\keywords   {}
\supervisor {Dr. Maria Polukarov}
\examiner   {Professor Sheng Chen}

\maketitle
\begin{abstract}
This project aim was to model and analyse the effects of competitive pricing behaviors of grocery retailers on the British market. 

This was achieved by creating a multi-agent model, containing retailer and consumer agents. The heterogeneous crowd of retailers employs either a uniform pricing strategy or a ‘local price flexing’ strategy. The actions of these retailers are chosen by predicting the profit of each action, using a perceptron. Following on from the consideration of different economic models, a discrete model was developed so that software agents have a discrete environment to operate within. Within the model, it has been observed how supermarkets with differing behaviors affect a heterogeneous crowd of consumer agents. The model was implemented in Java with Python used to evaluate the results. 

The simulation displays good acceptance with real grocery market behavior, i.e. captures the performance of British retailers thus can be used to determine the impact of changes in their behavior on their competitors and consumers.Furthermore it can be used to provide insight into sustainability of volatile pricing strategies, providing a useful insight in volatility of British supermarket retail industry. 
\end{abstract}
\acknowledgements{
I would like to express my sincere gratitude to Dr Maria Polukarov for her guidance and support which provided me the freedom to take this research in the direction of my interest.\\
\\
I would also like to thank my family and friends for their encouragement and support. To those who quietly listened to my software complaints. To those who worked throughout the nights with me. To those who helped me write what I couldn't say. I cannot thank you enough.
}

\declaration{
I, Stefan Collier, declare that this dissertation and the work presented in it are my own and has been generated by me as the result of my own original research.\\
I confirm that:\\
1. This work was done wholly or mainly while in candidature for a degree at this University;\\
2. Where any part of this dissertation has previously been submitted for any other qualification at this University or any other institution, this has been clearly stated;\\
3. Where I have consulted the published work of others, this is always clearly attributed;\\
4. Where I have quoted from the work of others, the source is always given. With the exception of such quotations, this dissertation is entirely my own work;\\
5. I have acknowledged all main sources of help;\\
6. Where the thesis is based on work done by myself jointly with others, I have made clear exactly what was done by others and what I have contributed myself;\\
7. Either none of this work has been published before submission, or parts of this work have been published by :\\
\\
Stefan Collier\\
April 2016
}
\tableofcontents
\listoffigures
\listoftables

\mainmatter
%% ----------------------------------------------------------------
%\include{Introduction}
%\include{Conclusions}
\include{chapters/1Project/main}
\include{chapters/2Lit/main}
\include{chapters/3Design/HighLevel}
\include{chapters/3Design/InDepth}
\include{chapters/4Impl/main}

\include{chapters/5Experiments/1/main}
\include{chapters/5Experiments/2/main}
\include{chapters/5Experiments/3/main}
\include{chapters/5Experiments/4/main}

\include{chapters/6Conclusion/main}

\appendix
\include{appendix/AppendixB}
\include{appendix/D/main}
\include{appendix/AppendixC}

\backmatter
\bibliographystyle{ecs}
\bibliography{ECS}
\end{document}
%% ----------------------------------------------------------------

 %% ----------------------------------------------------------------
%% Progress.tex
%% ---------------------------------------------------------------- 
\documentclass{ecsprogress}    % Use the progress Style
\graphicspath{{../figs/}}   % Location of your graphics files
    \usepackage{natbib}            % Use Natbib style for the refs.
\hypersetup{colorlinks=true}   % Set to false for black/white printing
\input{Definitions}            % Include your abbreviations



\usepackage{enumitem}% http://ctan.org/pkg/enumitem
\usepackage{multirow}
\usepackage{float}
\usepackage{amsmath}
\usepackage{multicol}
\usepackage{amssymb}
\usepackage[normalem]{ulem}
\useunder{\uline}{\ul}{}
\usepackage{wrapfig}


\usepackage[table,xcdraw]{xcolor}


%% ----------------------------------------------------------------
\begin{document}
\frontmatter
\title      {Heterogeneous Agent-based Model for Supermarket Competition}
\authors    {\texorpdfstring
             {\href{mailto:sc22g13@ecs.soton.ac.uk}{Stefan J. Collier}}
             {Stefan J. Collier}
            }
\addresses  {\groupname\\\deptname\\\univname}
\date       {\today}
\subject    {}
\keywords   {}
\supervisor {Dr. Maria Polukarov}
\examiner   {Professor Sheng Chen}

\maketitle
\begin{abstract}
This project aim was to model and analyse the effects of competitive pricing behaviors of grocery retailers on the British market. 

This was achieved by creating a multi-agent model, containing retailer and consumer agents. The heterogeneous crowd of retailers employs either a uniform pricing strategy or a ‘local price flexing’ strategy. The actions of these retailers are chosen by predicting the profit of each action, using a perceptron. Following on from the consideration of different economic models, a discrete model was developed so that software agents have a discrete environment to operate within. Within the model, it has been observed how supermarkets with differing behaviors affect a heterogeneous crowd of consumer agents. The model was implemented in Java with Python used to evaluate the results. 

The simulation displays good acceptance with real grocery market behavior, i.e. captures the performance of British retailers thus can be used to determine the impact of changes in their behavior on their competitors and consumers.Furthermore it can be used to provide insight into sustainability of volatile pricing strategies, providing a useful insight in volatility of British supermarket retail industry. 
\end{abstract}
\acknowledgements{
I would like to express my sincere gratitude to Dr Maria Polukarov for her guidance and support which provided me the freedom to take this research in the direction of my interest.\\
\\
I would also like to thank my family and friends for their encouragement and support. To those who quietly listened to my software complaints. To those who worked throughout the nights with me. To those who helped me write what I couldn't say. I cannot thank you enough.
}

\declaration{
I, Stefan Collier, declare that this dissertation and the work presented in it are my own and has been generated by me as the result of my own original research.\\
I confirm that:\\
1. This work was done wholly or mainly while in candidature for a degree at this University;\\
2. Where any part of this dissertation has previously been submitted for any other qualification at this University or any other institution, this has been clearly stated;\\
3. Where I have consulted the published work of others, this is always clearly attributed;\\
4. Where I have quoted from the work of others, the source is always given. With the exception of such quotations, this dissertation is entirely my own work;\\
5. I have acknowledged all main sources of help;\\
6. Where the thesis is based on work done by myself jointly with others, I have made clear exactly what was done by others and what I have contributed myself;\\
7. Either none of this work has been published before submission, or parts of this work have been published by :\\
\\
Stefan Collier\\
April 2016
}
\tableofcontents
\listoffigures
\listoftables

\mainmatter
%% ----------------------------------------------------------------
%\include{Introduction}
%\include{Conclusions}
\include{chapters/1Project/main}
\include{chapters/2Lit/main}
\include{chapters/3Design/HighLevel}
\include{chapters/3Design/InDepth}
\include{chapters/4Impl/main}

\include{chapters/5Experiments/1/main}
\include{chapters/5Experiments/2/main}
\include{chapters/5Experiments/3/main}
\include{chapters/5Experiments/4/main}

\include{chapters/6Conclusion/main}

\appendix
\include{appendix/AppendixB}
\include{appendix/D/main}
\include{appendix/AppendixC}

\backmatter
\bibliographystyle{ecs}
\bibliography{ECS}
\end{document}
%% ----------------------------------------------------------------

 %% ----------------------------------------------------------------
%% Progress.tex
%% ---------------------------------------------------------------- 
\documentclass{ecsprogress}    % Use the progress Style
\graphicspath{{../figs/}}   % Location of your graphics files
    \usepackage{natbib}            % Use Natbib style for the refs.
\hypersetup{colorlinks=true}   % Set to false for black/white printing
\input{Definitions}            % Include your abbreviations



\usepackage{enumitem}% http://ctan.org/pkg/enumitem
\usepackage{multirow}
\usepackage{float}
\usepackage{amsmath}
\usepackage{multicol}
\usepackage{amssymb}
\usepackage[normalem]{ulem}
\useunder{\uline}{\ul}{}
\usepackage{wrapfig}


\usepackage[table,xcdraw]{xcolor}


%% ----------------------------------------------------------------
\begin{document}
\frontmatter
\title      {Heterogeneous Agent-based Model for Supermarket Competition}
\authors    {\texorpdfstring
             {\href{mailto:sc22g13@ecs.soton.ac.uk}{Stefan J. Collier}}
             {Stefan J. Collier}
            }
\addresses  {\groupname\\\deptname\\\univname}
\date       {\today}
\subject    {}
\keywords   {}
\supervisor {Dr. Maria Polukarov}
\examiner   {Professor Sheng Chen}

\maketitle
\begin{abstract}
This project aim was to model and analyse the effects of competitive pricing behaviors of grocery retailers on the British market. 

This was achieved by creating a multi-agent model, containing retailer and consumer agents. The heterogeneous crowd of retailers employs either a uniform pricing strategy or a ‘local price flexing’ strategy. The actions of these retailers are chosen by predicting the profit of each action, using a perceptron. Following on from the consideration of different economic models, a discrete model was developed so that software agents have a discrete environment to operate within. Within the model, it has been observed how supermarkets with differing behaviors affect a heterogeneous crowd of consumer agents. The model was implemented in Java with Python used to evaluate the results. 

The simulation displays good acceptance with real grocery market behavior, i.e. captures the performance of British retailers thus can be used to determine the impact of changes in their behavior on their competitors and consumers.Furthermore it can be used to provide insight into sustainability of volatile pricing strategies, providing a useful insight in volatility of British supermarket retail industry. 
\end{abstract}
\acknowledgements{
I would like to express my sincere gratitude to Dr Maria Polukarov for her guidance and support which provided me the freedom to take this research in the direction of my interest.\\
\\
I would also like to thank my family and friends for their encouragement and support. To those who quietly listened to my software complaints. To those who worked throughout the nights with me. To those who helped me write what I couldn't say. I cannot thank you enough.
}

\declaration{
I, Stefan Collier, declare that this dissertation and the work presented in it are my own and has been generated by me as the result of my own original research.\\
I confirm that:\\
1. This work was done wholly or mainly while in candidature for a degree at this University;\\
2. Where any part of this dissertation has previously been submitted for any other qualification at this University or any other institution, this has been clearly stated;\\
3. Where I have consulted the published work of others, this is always clearly attributed;\\
4. Where I have quoted from the work of others, the source is always given. With the exception of such quotations, this dissertation is entirely my own work;\\
5. I have acknowledged all main sources of help;\\
6. Where the thesis is based on work done by myself jointly with others, I have made clear exactly what was done by others and what I have contributed myself;\\
7. Either none of this work has been published before submission, or parts of this work have been published by :\\
\\
Stefan Collier\\
April 2016
}
\tableofcontents
\listoffigures
\listoftables

\mainmatter
%% ----------------------------------------------------------------
%\include{Introduction}
%\include{Conclusions}
\include{chapters/1Project/main}
\include{chapters/2Lit/main}
\include{chapters/3Design/HighLevel}
\include{chapters/3Design/InDepth}
\include{chapters/4Impl/main}

\include{chapters/5Experiments/1/main}
\include{chapters/5Experiments/2/main}
\include{chapters/5Experiments/3/main}
\include{chapters/5Experiments/4/main}

\include{chapters/6Conclusion/main}

\appendix
\include{appendix/AppendixB}
\include{appendix/D/main}
\include{appendix/AppendixC}

\backmatter
\bibliographystyle{ecs}
\bibliography{ECS}
\end{document}
%% ----------------------------------------------------------------

 %% ----------------------------------------------------------------
%% Progress.tex
%% ---------------------------------------------------------------- 
\documentclass{ecsprogress}    % Use the progress Style
\graphicspath{{../figs/}}   % Location of your graphics files
    \usepackage{natbib}            % Use Natbib style for the refs.
\hypersetup{colorlinks=true}   % Set to false for black/white printing
\input{Definitions}            % Include your abbreviations



\usepackage{enumitem}% http://ctan.org/pkg/enumitem
\usepackage{multirow}
\usepackage{float}
\usepackage{amsmath}
\usepackage{multicol}
\usepackage{amssymb}
\usepackage[normalem]{ulem}
\useunder{\uline}{\ul}{}
\usepackage{wrapfig}


\usepackage[table,xcdraw]{xcolor}


%% ----------------------------------------------------------------
\begin{document}
\frontmatter
\title      {Heterogeneous Agent-based Model for Supermarket Competition}
\authors    {\texorpdfstring
             {\href{mailto:sc22g13@ecs.soton.ac.uk}{Stefan J. Collier}}
             {Stefan J. Collier}
            }
\addresses  {\groupname\\\deptname\\\univname}
\date       {\today}
\subject    {}
\keywords   {}
\supervisor {Dr. Maria Polukarov}
\examiner   {Professor Sheng Chen}

\maketitle
\begin{abstract}
This project aim was to model and analyse the effects of competitive pricing behaviors of grocery retailers on the British market. 

This was achieved by creating a multi-agent model, containing retailer and consumer agents. The heterogeneous crowd of retailers employs either a uniform pricing strategy or a ‘local price flexing’ strategy. The actions of these retailers are chosen by predicting the profit of each action, using a perceptron. Following on from the consideration of different economic models, a discrete model was developed so that software agents have a discrete environment to operate within. Within the model, it has been observed how supermarkets with differing behaviors affect a heterogeneous crowd of consumer agents. The model was implemented in Java with Python used to evaluate the results. 

The simulation displays good acceptance with real grocery market behavior, i.e. captures the performance of British retailers thus can be used to determine the impact of changes in their behavior on their competitors and consumers.Furthermore it can be used to provide insight into sustainability of volatile pricing strategies, providing a useful insight in volatility of British supermarket retail industry. 
\end{abstract}
\acknowledgements{
I would like to express my sincere gratitude to Dr Maria Polukarov for her guidance and support which provided me the freedom to take this research in the direction of my interest.\\
\\
I would also like to thank my family and friends for their encouragement and support. To those who quietly listened to my software complaints. To those who worked throughout the nights with me. To those who helped me write what I couldn't say. I cannot thank you enough.
}

\declaration{
I, Stefan Collier, declare that this dissertation and the work presented in it are my own and has been generated by me as the result of my own original research.\\
I confirm that:\\
1. This work was done wholly or mainly while in candidature for a degree at this University;\\
2. Where any part of this dissertation has previously been submitted for any other qualification at this University or any other institution, this has been clearly stated;\\
3. Where I have consulted the published work of others, this is always clearly attributed;\\
4. Where I have quoted from the work of others, the source is always given. With the exception of such quotations, this dissertation is entirely my own work;\\
5. I have acknowledged all main sources of help;\\
6. Where the thesis is based on work done by myself jointly with others, I have made clear exactly what was done by others and what I have contributed myself;\\
7. Either none of this work has been published before submission, or parts of this work have been published by :\\
\\
Stefan Collier\\
April 2016
}
\tableofcontents
\listoffigures
\listoftables

\mainmatter
%% ----------------------------------------------------------------
%\include{Introduction}
%\include{Conclusions}
\include{chapters/1Project/main}
\include{chapters/2Lit/main}
\include{chapters/3Design/HighLevel}
\include{chapters/3Design/InDepth}
\include{chapters/4Impl/main}

\include{chapters/5Experiments/1/main}
\include{chapters/5Experiments/2/main}
\include{chapters/5Experiments/3/main}
\include{chapters/5Experiments/4/main}

\include{chapters/6Conclusion/main}

\appendix
\include{appendix/AppendixB}
\include{appendix/D/main}
\include{appendix/AppendixC}

\backmatter
\bibliographystyle{ecs}
\bibliography{ECS}
\end{document}
%% ----------------------------------------------------------------


 %% ----------------------------------------------------------------
%% Progress.tex
%% ---------------------------------------------------------------- 
\documentclass{ecsprogress}    % Use the progress Style
\graphicspath{{../figs/}}   % Location of your graphics files
    \usepackage{natbib}            % Use Natbib style for the refs.
\hypersetup{colorlinks=true}   % Set to false for black/white printing
\input{Definitions}            % Include your abbreviations



\usepackage{enumitem}% http://ctan.org/pkg/enumitem
\usepackage{multirow}
\usepackage{float}
\usepackage{amsmath}
\usepackage{multicol}
\usepackage{amssymb}
\usepackage[normalem]{ulem}
\useunder{\uline}{\ul}{}
\usepackage{wrapfig}


\usepackage[table,xcdraw]{xcolor}


%% ----------------------------------------------------------------
\begin{document}
\frontmatter
\title      {Heterogeneous Agent-based Model for Supermarket Competition}
\authors    {\texorpdfstring
             {\href{mailto:sc22g13@ecs.soton.ac.uk}{Stefan J. Collier}}
             {Stefan J. Collier}
            }
\addresses  {\groupname\\\deptname\\\univname}
\date       {\today}
\subject    {}
\keywords   {}
\supervisor {Dr. Maria Polukarov}
\examiner   {Professor Sheng Chen}

\maketitle
\begin{abstract}
This project aim was to model and analyse the effects of competitive pricing behaviors of grocery retailers on the British market. 

This was achieved by creating a multi-agent model, containing retailer and consumer agents. The heterogeneous crowd of retailers employs either a uniform pricing strategy or a ‘local price flexing’ strategy. The actions of these retailers are chosen by predicting the profit of each action, using a perceptron. Following on from the consideration of different economic models, a discrete model was developed so that software agents have a discrete environment to operate within. Within the model, it has been observed how supermarkets with differing behaviors affect a heterogeneous crowd of consumer agents. The model was implemented in Java with Python used to evaluate the results. 

The simulation displays good acceptance with real grocery market behavior, i.e. captures the performance of British retailers thus can be used to determine the impact of changes in their behavior on their competitors and consumers.Furthermore it can be used to provide insight into sustainability of volatile pricing strategies, providing a useful insight in volatility of British supermarket retail industry. 
\end{abstract}
\acknowledgements{
I would like to express my sincere gratitude to Dr Maria Polukarov for her guidance and support which provided me the freedom to take this research in the direction of my interest.\\
\\
I would also like to thank my family and friends for their encouragement and support. To those who quietly listened to my software complaints. To those who worked throughout the nights with me. To those who helped me write what I couldn't say. I cannot thank you enough.
}

\declaration{
I, Stefan Collier, declare that this dissertation and the work presented in it are my own and has been generated by me as the result of my own original research.\\
I confirm that:\\
1. This work was done wholly or mainly while in candidature for a degree at this University;\\
2. Where any part of this dissertation has previously been submitted for any other qualification at this University or any other institution, this has been clearly stated;\\
3. Where I have consulted the published work of others, this is always clearly attributed;\\
4. Where I have quoted from the work of others, the source is always given. With the exception of such quotations, this dissertation is entirely my own work;\\
5. I have acknowledged all main sources of help;\\
6. Where the thesis is based on work done by myself jointly with others, I have made clear exactly what was done by others and what I have contributed myself;\\
7. Either none of this work has been published before submission, or parts of this work have been published by :\\
\\
Stefan Collier\\
April 2016
}
\tableofcontents
\listoffigures
\listoftables

\mainmatter
%% ----------------------------------------------------------------
%\include{Introduction}
%\include{Conclusions}
\include{chapters/1Project/main}
\include{chapters/2Lit/main}
\include{chapters/3Design/HighLevel}
\include{chapters/3Design/InDepth}
\include{chapters/4Impl/main}

\include{chapters/5Experiments/1/main}
\include{chapters/5Experiments/2/main}
\include{chapters/5Experiments/3/main}
\include{chapters/5Experiments/4/main}

\include{chapters/6Conclusion/main}

\appendix
\include{appendix/AppendixB}
\include{appendix/D/main}
\include{appendix/AppendixC}

\backmatter
\bibliographystyle{ecs}
\bibliography{ECS}
\end{document}
%% ----------------------------------------------------------------


\appendix
\include{appendix/AppendixB}
 %% ----------------------------------------------------------------
%% Progress.tex
%% ---------------------------------------------------------------- 
\documentclass{ecsprogress}    % Use the progress Style
\graphicspath{{../figs/}}   % Location of your graphics files
    \usepackage{natbib}            % Use Natbib style for the refs.
\hypersetup{colorlinks=true}   % Set to false for black/white printing
\input{Definitions}            % Include your abbreviations



\usepackage{enumitem}% http://ctan.org/pkg/enumitem
\usepackage{multirow}
\usepackage{float}
\usepackage{amsmath}
\usepackage{multicol}
\usepackage{amssymb}
\usepackage[normalem]{ulem}
\useunder{\uline}{\ul}{}
\usepackage{wrapfig}


\usepackage[table,xcdraw]{xcolor}


%% ----------------------------------------------------------------
\begin{document}
\frontmatter
\title      {Heterogeneous Agent-based Model for Supermarket Competition}
\authors    {\texorpdfstring
             {\href{mailto:sc22g13@ecs.soton.ac.uk}{Stefan J. Collier}}
             {Stefan J. Collier}
            }
\addresses  {\groupname\\\deptname\\\univname}
\date       {\today}
\subject    {}
\keywords   {}
\supervisor {Dr. Maria Polukarov}
\examiner   {Professor Sheng Chen}

\maketitle
\begin{abstract}
This project aim was to model and analyse the effects of competitive pricing behaviors of grocery retailers on the British market. 

This was achieved by creating a multi-agent model, containing retailer and consumer agents. The heterogeneous crowd of retailers employs either a uniform pricing strategy or a ‘local price flexing’ strategy. The actions of these retailers are chosen by predicting the profit of each action, using a perceptron. Following on from the consideration of different economic models, a discrete model was developed so that software agents have a discrete environment to operate within. Within the model, it has been observed how supermarkets with differing behaviors affect a heterogeneous crowd of consumer agents. The model was implemented in Java with Python used to evaluate the results. 

The simulation displays good acceptance with real grocery market behavior, i.e. captures the performance of British retailers thus can be used to determine the impact of changes in their behavior on their competitors and consumers.Furthermore it can be used to provide insight into sustainability of volatile pricing strategies, providing a useful insight in volatility of British supermarket retail industry. 
\end{abstract}
\acknowledgements{
I would like to express my sincere gratitude to Dr Maria Polukarov for her guidance and support which provided me the freedom to take this research in the direction of my interest.\\
\\
I would also like to thank my family and friends for their encouragement and support. To those who quietly listened to my software complaints. To those who worked throughout the nights with me. To those who helped me write what I couldn't say. I cannot thank you enough.
}

\declaration{
I, Stefan Collier, declare that this dissertation and the work presented in it are my own and has been generated by me as the result of my own original research.\\
I confirm that:\\
1. This work was done wholly or mainly while in candidature for a degree at this University;\\
2. Where any part of this dissertation has previously been submitted for any other qualification at this University or any other institution, this has been clearly stated;\\
3. Where I have consulted the published work of others, this is always clearly attributed;\\
4. Where I have quoted from the work of others, the source is always given. With the exception of such quotations, this dissertation is entirely my own work;\\
5. I have acknowledged all main sources of help;\\
6. Where the thesis is based on work done by myself jointly with others, I have made clear exactly what was done by others and what I have contributed myself;\\
7. Either none of this work has been published before submission, or parts of this work have been published by :\\
\\
Stefan Collier\\
April 2016
}
\tableofcontents
\listoffigures
\listoftables

\mainmatter
%% ----------------------------------------------------------------
%\include{Introduction}
%\include{Conclusions}
\include{chapters/1Project/main}
\include{chapters/2Lit/main}
\include{chapters/3Design/HighLevel}
\include{chapters/3Design/InDepth}
\include{chapters/4Impl/main}

\include{chapters/5Experiments/1/main}
\include{chapters/5Experiments/2/main}
\include{chapters/5Experiments/3/main}
\include{chapters/5Experiments/4/main}

\include{chapters/6Conclusion/main}

\appendix
\include{appendix/AppendixB}
\include{appendix/D/main}
\include{appendix/AppendixC}

\backmatter
\bibliographystyle{ecs}
\bibliography{ECS}
\end{document}
%% ----------------------------------------------------------------

\include{appendix/AppendixC}

\backmatter
\bibliographystyle{ecs}
\bibliography{ECS}
\end{document}
%% ----------------------------------------------------------------

 %% ----------------------------------------------------------------
%% Progress.tex
%% ---------------------------------------------------------------- 
\documentclass{ecsprogress}    % Use the progress Style
\graphicspath{{../figs/}}   % Location of your graphics files
    \usepackage{natbib}            % Use Natbib style for the refs.
\hypersetup{colorlinks=true}   % Set to false for black/white printing
\input{Definitions}            % Include your abbreviations



\usepackage{enumitem}% http://ctan.org/pkg/enumitem
\usepackage{multirow}
\usepackage{float}
\usepackage{amsmath}
\usepackage{multicol}
\usepackage{amssymb}
\usepackage[normalem]{ulem}
\useunder{\uline}{\ul}{}
\usepackage{wrapfig}


\usepackage[table,xcdraw]{xcolor}


%% ----------------------------------------------------------------
\begin{document}
\frontmatter
\title      {Heterogeneous Agent-based Model for Supermarket Competition}
\authors    {\texorpdfstring
             {\href{mailto:sc22g13@ecs.soton.ac.uk}{Stefan J. Collier}}
             {Stefan J. Collier}
            }
\addresses  {\groupname\\\deptname\\\univname}
\date       {\today}
\subject    {}
\keywords   {}
\supervisor {Dr. Maria Polukarov}
\examiner   {Professor Sheng Chen}

\maketitle
\begin{abstract}
This project aim was to model and analyse the effects of competitive pricing behaviors of grocery retailers on the British market. 

This was achieved by creating a multi-agent model, containing retailer and consumer agents. The heterogeneous crowd of retailers employs either a uniform pricing strategy or a ‘local price flexing’ strategy. The actions of these retailers are chosen by predicting the profit of each action, using a perceptron. Following on from the consideration of different economic models, a discrete model was developed so that software agents have a discrete environment to operate within. Within the model, it has been observed how supermarkets with differing behaviors affect a heterogeneous crowd of consumer agents. The model was implemented in Java with Python used to evaluate the results. 

The simulation displays good acceptance with real grocery market behavior, i.e. captures the performance of British retailers thus can be used to determine the impact of changes in their behavior on their competitors and consumers.Furthermore it can be used to provide insight into sustainability of volatile pricing strategies, providing a useful insight in volatility of British supermarket retail industry. 
\end{abstract}
\acknowledgements{
I would like to express my sincere gratitude to Dr Maria Polukarov for her guidance and support which provided me the freedom to take this research in the direction of my interest.\\
\\
I would also like to thank my family and friends for their encouragement and support. To those who quietly listened to my software complaints. To those who worked throughout the nights with me. To those who helped me write what I couldn't say. I cannot thank you enough.
}

\declaration{
I, Stefan Collier, declare that this dissertation and the work presented in it are my own and has been generated by me as the result of my own original research.\\
I confirm that:\\
1. This work was done wholly or mainly while in candidature for a degree at this University;\\
2. Where any part of this dissertation has previously been submitted for any other qualification at this University or any other institution, this has been clearly stated;\\
3. Where I have consulted the published work of others, this is always clearly attributed;\\
4. Where I have quoted from the work of others, the source is always given. With the exception of such quotations, this dissertation is entirely my own work;\\
5. I have acknowledged all main sources of help;\\
6. Where the thesis is based on work done by myself jointly with others, I have made clear exactly what was done by others and what I have contributed myself;\\
7. Either none of this work has been published before submission, or parts of this work have been published by :\\
\\
Stefan Collier\\
April 2016
}
\tableofcontents
\listoffigures
\listoftables

\mainmatter
%% ----------------------------------------------------------------
%\include{Introduction}
%\include{Conclusions}
 %% ----------------------------------------------------------------
%% Progress.tex
%% ---------------------------------------------------------------- 
\documentclass{ecsprogress}    % Use the progress Style
\graphicspath{{../figs/}}   % Location of your graphics files
    \usepackage{natbib}            % Use Natbib style for the refs.
\hypersetup{colorlinks=true}   % Set to false for black/white printing
\input{Definitions}            % Include your abbreviations



\usepackage{enumitem}% http://ctan.org/pkg/enumitem
\usepackage{multirow}
\usepackage{float}
\usepackage{amsmath}
\usepackage{multicol}
\usepackage{amssymb}
\usepackage[normalem]{ulem}
\useunder{\uline}{\ul}{}
\usepackage{wrapfig}


\usepackage[table,xcdraw]{xcolor}


%% ----------------------------------------------------------------
\begin{document}
\frontmatter
\title      {Heterogeneous Agent-based Model for Supermarket Competition}
\authors    {\texorpdfstring
             {\href{mailto:sc22g13@ecs.soton.ac.uk}{Stefan J. Collier}}
             {Stefan J. Collier}
            }
\addresses  {\groupname\\\deptname\\\univname}
\date       {\today}
\subject    {}
\keywords   {}
\supervisor {Dr. Maria Polukarov}
\examiner   {Professor Sheng Chen}

\maketitle
\begin{abstract}
This project aim was to model and analyse the effects of competitive pricing behaviors of grocery retailers on the British market. 

This was achieved by creating a multi-agent model, containing retailer and consumer agents. The heterogeneous crowd of retailers employs either a uniform pricing strategy or a ‘local price flexing’ strategy. The actions of these retailers are chosen by predicting the profit of each action, using a perceptron. Following on from the consideration of different economic models, a discrete model was developed so that software agents have a discrete environment to operate within. Within the model, it has been observed how supermarkets with differing behaviors affect a heterogeneous crowd of consumer agents. The model was implemented in Java with Python used to evaluate the results. 

The simulation displays good acceptance with real grocery market behavior, i.e. captures the performance of British retailers thus can be used to determine the impact of changes in their behavior on their competitors and consumers.Furthermore it can be used to provide insight into sustainability of volatile pricing strategies, providing a useful insight in volatility of British supermarket retail industry. 
\end{abstract}
\acknowledgements{
I would like to express my sincere gratitude to Dr Maria Polukarov for her guidance and support which provided me the freedom to take this research in the direction of my interest.\\
\\
I would also like to thank my family and friends for their encouragement and support. To those who quietly listened to my software complaints. To those who worked throughout the nights with me. To those who helped me write what I couldn't say. I cannot thank you enough.
}

\declaration{
I, Stefan Collier, declare that this dissertation and the work presented in it are my own and has been generated by me as the result of my own original research.\\
I confirm that:\\
1. This work was done wholly or mainly while in candidature for a degree at this University;\\
2. Where any part of this dissertation has previously been submitted for any other qualification at this University or any other institution, this has been clearly stated;\\
3. Where I have consulted the published work of others, this is always clearly attributed;\\
4. Where I have quoted from the work of others, the source is always given. With the exception of such quotations, this dissertation is entirely my own work;\\
5. I have acknowledged all main sources of help;\\
6. Where the thesis is based on work done by myself jointly with others, I have made clear exactly what was done by others and what I have contributed myself;\\
7. Either none of this work has been published before submission, or parts of this work have been published by :\\
\\
Stefan Collier\\
April 2016
}
\tableofcontents
\listoffigures
\listoftables

\mainmatter
%% ----------------------------------------------------------------
%\include{Introduction}
%\include{Conclusions}
\include{chapters/1Project/main}
\include{chapters/2Lit/main}
\include{chapters/3Design/HighLevel}
\include{chapters/3Design/InDepth}
\include{chapters/4Impl/main}

\include{chapters/5Experiments/1/main}
\include{chapters/5Experiments/2/main}
\include{chapters/5Experiments/3/main}
\include{chapters/5Experiments/4/main}

\include{chapters/6Conclusion/main}

\appendix
\include{appendix/AppendixB}
\include{appendix/D/main}
\include{appendix/AppendixC}

\backmatter
\bibliographystyle{ecs}
\bibliography{ECS}
\end{document}
%% ----------------------------------------------------------------

 %% ----------------------------------------------------------------
%% Progress.tex
%% ---------------------------------------------------------------- 
\documentclass{ecsprogress}    % Use the progress Style
\graphicspath{{../figs/}}   % Location of your graphics files
    \usepackage{natbib}            % Use Natbib style for the refs.
\hypersetup{colorlinks=true}   % Set to false for black/white printing
\input{Definitions}            % Include your abbreviations



\usepackage{enumitem}% http://ctan.org/pkg/enumitem
\usepackage{multirow}
\usepackage{float}
\usepackage{amsmath}
\usepackage{multicol}
\usepackage{amssymb}
\usepackage[normalem]{ulem}
\useunder{\uline}{\ul}{}
\usepackage{wrapfig}


\usepackage[table,xcdraw]{xcolor}


%% ----------------------------------------------------------------
\begin{document}
\frontmatter
\title      {Heterogeneous Agent-based Model for Supermarket Competition}
\authors    {\texorpdfstring
             {\href{mailto:sc22g13@ecs.soton.ac.uk}{Stefan J. Collier}}
             {Stefan J. Collier}
            }
\addresses  {\groupname\\\deptname\\\univname}
\date       {\today}
\subject    {}
\keywords   {}
\supervisor {Dr. Maria Polukarov}
\examiner   {Professor Sheng Chen}

\maketitle
\begin{abstract}
This project aim was to model and analyse the effects of competitive pricing behaviors of grocery retailers on the British market. 

This was achieved by creating a multi-agent model, containing retailer and consumer agents. The heterogeneous crowd of retailers employs either a uniform pricing strategy or a ‘local price flexing’ strategy. The actions of these retailers are chosen by predicting the profit of each action, using a perceptron. Following on from the consideration of different economic models, a discrete model was developed so that software agents have a discrete environment to operate within. Within the model, it has been observed how supermarkets with differing behaviors affect a heterogeneous crowd of consumer agents. The model was implemented in Java with Python used to evaluate the results. 

The simulation displays good acceptance with real grocery market behavior, i.e. captures the performance of British retailers thus can be used to determine the impact of changes in their behavior on their competitors and consumers.Furthermore it can be used to provide insight into sustainability of volatile pricing strategies, providing a useful insight in volatility of British supermarket retail industry. 
\end{abstract}
\acknowledgements{
I would like to express my sincere gratitude to Dr Maria Polukarov for her guidance and support which provided me the freedom to take this research in the direction of my interest.\\
\\
I would also like to thank my family and friends for their encouragement and support. To those who quietly listened to my software complaints. To those who worked throughout the nights with me. To those who helped me write what I couldn't say. I cannot thank you enough.
}

\declaration{
I, Stefan Collier, declare that this dissertation and the work presented in it are my own and has been generated by me as the result of my own original research.\\
I confirm that:\\
1. This work was done wholly or mainly while in candidature for a degree at this University;\\
2. Where any part of this dissertation has previously been submitted for any other qualification at this University or any other institution, this has been clearly stated;\\
3. Where I have consulted the published work of others, this is always clearly attributed;\\
4. Where I have quoted from the work of others, the source is always given. With the exception of such quotations, this dissertation is entirely my own work;\\
5. I have acknowledged all main sources of help;\\
6. Where the thesis is based on work done by myself jointly with others, I have made clear exactly what was done by others and what I have contributed myself;\\
7. Either none of this work has been published before submission, or parts of this work have been published by :\\
\\
Stefan Collier\\
April 2016
}
\tableofcontents
\listoffigures
\listoftables

\mainmatter
%% ----------------------------------------------------------------
%\include{Introduction}
%\include{Conclusions}
\include{chapters/1Project/main}
\include{chapters/2Lit/main}
\include{chapters/3Design/HighLevel}
\include{chapters/3Design/InDepth}
\include{chapters/4Impl/main}

\include{chapters/5Experiments/1/main}
\include{chapters/5Experiments/2/main}
\include{chapters/5Experiments/3/main}
\include{chapters/5Experiments/4/main}

\include{chapters/6Conclusion/main}

\appendix
\include{appendix/AppendixB}
\include{appendix/D/main}
\include{appendix/AppendixC}

\backmatter
\bibliographystyle{ecs}
\bibliography{ECS}
\end{document}
%% ----------------------------------------------------------------

\include{chapters/3Design/HighLevel}
\include{chapters/3Design/InDepth}
 %% ----------------------------------------------------------------
%% Progress.tex
%% ---------------------------------------------------------------- 
\documentclass{ecsprogress}    % Use the progress Style
\graphicspath{{../figs/}}   % Location of your graphics files
    \usepackage{natbib}            % Use Natbib style for the refs.
\hypersetup{colorlinks=true}   % Set to false for black/white printing
\input{Definitions}            % Include your abbreviations



\usepackage{enumitem}% http://ctan.org/pkg/enumitem
\usepackage{multirow}
\usepackage{float}
\usepackage{amsmath}
\usepackage{multicol}
\usepackage{amssymb}
\usepackage[normalem]{ulem}
\useunder{\uline}{\ul}{}
\usepackage{wrapfig}


\usepackage[table,xcdraw]{xcolor}


%% ----------------------------------------------------------------
\begin{document}
\frontmatter
\title      {Heterogeneous Agent-based Model for Supermarket Competition}
\authors    {\texorpdfstring
             {\href{mailto:sc22g13@ecs.soton.ac.uk}{Stefan J. Collier}}
             {Stefan J. Collier}
            }
\addresses  {\groupname\\\deptname\\\univname}
\date       {\today}
\subject    {}
\keywords   {}
\supervisor {Dr. Maria Polukarov}
\examiner   {Professor Sheng Chen}

\maketitle
\begin{abstract}
This project aim was to model and analyse the effects of competitive pricing behaviors of grocery retailers on the British market. 

This was achieved by creating a multi-agent model, containing retailer and consumer agents. The heterogeneous crowd of retailers employs either a uniform pricing strategy or a ‘local price flexing’ strategy. The actions of these retailers are chosen by predicting the profit of each action, using a perceptron. Following on from the consideration of different economic models, a discrete model was developed so that software agents have a discrete environment to operate within. Within the model, it has been observed how supermarkets with differing behaviors affect a heterogeneous crowd of consumer agents. The model was implemented in Java with Python used to evaluate the results. 

The simulation displays good acceptance with real grocery market behavior, i.e. captures the performance of British retailers thus can be used to determine the impact of changes in their behavior on their competitors and consumers.Furthermore it can be used to provide insight into sustainability of volatile pricing strategies, providing a useful insight in volatility of British supermarket retail industry. 
\end{abstract}
\acknowledgements{
I would like to express my sincere gratitude to Dr Maria Polukarov for her guidance and support which provided me the freedom to take this research in the direction of my interest.\\
\\
I would also like to thank my family and friends for their encouragement and support. To those who quietly listened to my software complaints. To those who worked throughout the nights with me. To those who helped me write what I couldn't say. I cannot thank you enough.
}

\declaration{
I, Stefan Collier, declare that this dissertation and the work presented in it are my own and has been generated by me as the result of my own original research.\\
I confirm that:\\
1. This work was done wholly or mainly while in candidature for a degree at this University;\\
2. Where any part of this dissertation has previously been submitted for any other qualification at this University or any other institution, this has been clearly stated;\\
3. Where I have consulted the published work of others, this is always clearly attributed;\\
4. Where I have quoted from the work of others, the source is always given. With the exception of such quotations, this dissertation is entirely my own work;\\
5. I have acknowledged all main sources of help;\\
6. Where the thesis is based on work done by myself jointly with others, I have made clear exactly what was done by others and what I have contributed myself;\\
7. Either none of this work has been published before submission, or parts of this work have been published by :\\
\\
Stefan Collier\\
April 2016
}
\tableofcontents
\listoffigures
\listoftables

\mainmatter
%% ----------------------------------------------------------------
%\include{Introduction}
%\include{Conclusions}
\include{chapters/1Project/main}
\include{chapters/2Lit/main}
\include{chapters/3Design/HighLevel}
\include{chapters/3Design/InDepth}
\include{chapters/4Impl/main}

\include{chapters/5Experiments/1/main}
\include{chapters/5Experiments/2/main}
\include{chapters/5Experiments/3/main}
\include{chapters/5Experiments/4/main}

\include{chapters/6Conclusion/main}

\appendix
\include{appendix/AppendixB}
\include{appendix/D/main}
\include{appendix/AppendixC}

\backmatter
\bibliographystyle{ecs}
\bibliography{ECS}
\end{document}
%% ----------------------------------------------------------------


 %% ----------------------------------------------------------------
%% Progress.tex
%% ---------------------------------------------------------------- 
\documentclass{ecsprogress}    % Use the progress Style
\graphicspath{{../figs/}}   % Location of your graphics files
    \usepackage{natbib}            % Use Natbib style for the refs.
\hypersetup{colorlinks=true}   % Set to false for black/white printing
\input{Definitions}            % Include your abbreviations



\usepackage{enumitem}% http://ctan.org/pkg/enumitem
\usepackage{multirow}
\usepackage{float}
\usepackage{amsmath}
\usepackage{multicol}
\usepackage{amssymb}
\usepackage[normalem]{ulem}
\useunder{\uline}{\ul}{}
\usepackage{wrapfig}


\usepackage[table,xcdraw]{xcolor}


%% ----------------------------------------------------------------
\begin{document}
\frontmatter
\title      {Heterogeneous Agent-based Model for Supermarket Competition}
\authors    {\texorpdfstring
             {\href{mailto:sc22g13@ecs.soton.ac.uk}{Stefan J. Collier}}
             {Stefan J. Collier}
            }
\addresses  {\groupname\\\deptname\\\univname}
\date       {\today}
\subject    {}
\keywords   {}
\supervisor {Dr. Maria Polukarov}
\examiner   {Professor Sheng Chen}

\maketitle
\begin{abstract}
This project aim was to model and analyse the effects of competitive pricing behaviors of grocery retailers on the British market. 

This was achieved by creating a multi-agent model, containing retailer and consumer agents. The heterogeneous crowd of retailers employs either a uniform pricing strategy or a ‘local price flexing’ strategy. The actions of these retailers are chosen by predicting the profit of each action, using a perceptron. Following on from the consideration of different economic models, a discrete model was developed so that software agents have a discrete environment to operate within. Within the model, it has been observed how supermarkets with differing behaviors affect a heterogeneous crowd of consumer agents. The model was implemented in Java with Python used to evaluate the results. 

The simulation displays good acceptance with real grocery market behavior, i.e. captures the performance of British retailers thus can be used to determine the impact of changes in their behavior on their competitors and consumers.Furthermore it can be used to provide insight into sustainability of volatile pricing strategies, providing a useful insight in volatility of British supermarket retail industry. 
\end{abstract}
\acknowledgements{
I would like to express my sincere gratitude to Dr Maria Polukarov for her guidance and support which provided me the freedom to take this research in the direction of my interest.\\
\\
I would also like to thank my family and friends for their encouragement and support. To those who quietly listened to my software complaints. To those who worked throughout the nights with me. To those who helped me write what I couldn't say. I cannot thank you enough.
}

\declaration{
I, Stefan Collier, declare that this dissertation and the work presented in it are my own and has been generated by me as the result of my own original research.\\
I confirm that:\\
1. This work was done wholly or mainly while in candidature for a degree at this University;\\
2. Where any part of this dissertation has previously been submitted for any other qualification at this University or any other institution, this has been clearly stated;\\
3. Where I have consulted the published work of others, this is always clearly attributed;\\
4. Where I have quoted from the work of others, the source is always given. With the exception of such quotations, this dissertation is entirely my own work;\\
5. I have acknowledged all main sources of help;\\
6. Where the thesis is based on work done by myself jointly with others, I have made clear exactly what was done by others and what I have contributed myself;\\
7. Either none of this work has been published before submission, or parts of this work have been published by :\\
\\
Stefan Collier\\
April 2016
}
\tableofcontents
\listoffigures
\listoftables

\mainmatter
%% ----------------------------------------------------------------
%\include{Introduction}
%\include{Conclusions}
\include{chapters/1Project/main}
\include{chapters/2Lit/main}
\include{chapters/3Design/HighLevel}
\include{chapters/3Design/InDepth}
\include{chapters/4Impl/main}

\include{chapters/5Experiments/1/main}
\include{chapters/5Experiments/2/main}
\include{chapters/5Experiments/3/main}
\include{chapters/5Experiments/4/main}

\include{chapters/6Conclusion/main}

\appendix
\include{appendix/AppendixB}
\include{appendix/D/main}
\include{appendix/AppendixC}

\backmatter
\bibliographystyle{ecs}
\bibliography{ECS}
\end{document}
%% ----------------------------------------------------------------

 %% ----------------------------------------------------------------
%% Progress.tex
%% ---------------------------------------------------------------- 
\documentclass{ecsprogress}    % Use the progress Style
\graphicspath{{../figs/}}   % Location of your graphics files
    \usepackage{natbib}            % Use Natbib style for the refs.
\hypersetup{colorlinks=true}   % Set to false for black/white printing
\input{Definitions}            % Include your abbreviations



\usepackage{enumitem}% http://ctan.org/pkg/enumitem
\usepackage{multirow}
\usepackage{float}
\usepackage{amsmath}
\usepackage{multicol}
\usepackage{amssymb}
\usepackage[normalem]{ulem}
\useunder{\uline}{\ul}{}
\usepackage{wrapfig}


\usepackage[table,xcdraw]{xcolor}


%% ----------------------------------------------------------------
\begin{document}
\frontmatter
\title      {Heterogeneous Agent-based Model for Supermarket Competition}
\authors    {\texorpdfstring
             {\href{mailto:sc22g13@ecs.soton.ac.uk}{Stefan J. Collier}}
             {Stefan J. Collier}
            }
\addresses  {\groupname\\\deptname\\\univname}
\date       {\today}
\subject    {}
\keywords   {}
\supervisor {Dr. Maria Polukarov}
\examiner   {Professor Sheng Chen}

\maketitle
\begin{abstract}
This project aim was to model and analyse the effects of competitive pricing behaviors of grocery retailers on the British market. 

This was achieved by creating a multi-agent model, containing retailer and consumer agents. The heterogeneous crowd of retailers employs either a uniform pricing strategy or a ‘local price flexing’ strategy. The actions of these retailers are chosen by predicting the profit of each action, using a perceptron. Following on from the consideration of different economic models, a discrete model was developed so that software agents have a discrete environment to operate within. Within the model, it has been observed how supermarkets with differing behaviors affect a heterogeneous crowd of consumer agents. The model was implemented in Java with Python used to evaluate the results. 

The simulation displays good acceptance with real grocery market behavior, i.e. captures the performance of British retailers thus can be used to determine the impact of changes in their behavior on their competitors and consumers.Furthermore it can be used to provide insight into sustainability of volatile pricing strategies, providing a useful insight in volatility of British supermarket retail industry. 
\end{abstract}
\acknowledgements{
I would like to express my sincere gratitude to Dr Maria Polukarov for her guidance and support which provided me the freedom to take this research in the direction of my interest.\\
\\
I would also like to thank my family and friends for their encouragement and support. To those who quietly listened to my software complaints. To those who worked throughout the nights with me. To those who helped me write what I couldn't say. I cannot thank you enough.
}

\declaration{
I, Stefan Collier, declare that this dissertation and the work presented in it are my own and has been generated by me as the result of my own original research.\\
I confirm that:\\
1. This work was done wholly or mainly while in candidature for a degree at this University;\\
2. Where any part of this dissertation has previously been submitted for any other qualification at this University or any other institution, this has been clearly stated;\\
3. Where I have consulted the published work of others, this is always clearly attributed;\\
4. Where I have quoted from the work of others, the source is always given. With the exception of such quotations, this dissertation is entirely my own work;\\
5. I have acknowledged all main sources of help;\\
6. Where the thesis is based on work done by myself jointly with others, I have made clear exactly what was done by others and what I have contributed myself;\\
7. Either none of this work has been published before submission, or parts of this work have been published by :\\
\\
Stefan Collier\\
April 2016
}
\tableofcontents
\listoffigures
\listoftables

\mainmatter
%% ----------------------------------------------------------------
%\include{Introduction}
%\include{Conclusions}
\include{chapters/1Project/main}
\include{chapters/2Lit/main}
\include{chapters/3Design/HighLevel}
\include{chapters/3Design/InDepth}
\include{chapters/4Impl/main}

\include{chapters/5Experiments/1/main}
\include{chapters/5Experiments/2/main}
\include{chapters/5Experiments/3/main}
\include{chapters/5Experiments/4/main}

\include{chapters/6Conclusion/main}

\appendix
\include{appendix/AppendixB}
\include{appendix/D/main}
\include{appendix/AppendixC}

\backmatter
\bibliographystyle{ecs}
\bibliography{ECS}
\end{document}
%% ----------------------------------------------------------------

 %% ----------------------------------------------------------------
%% Progress.tex
%% ---------------------------------------------------------------- 
\documentclass{ecsprogress}    % Use the progress Style
\graphicspath{{../figs/}}   % Location of your graphics files
    \usepackage{natbib}            % Use Natbib style for the refs.
\hypersetup{colorlinks=true}   % Set to false for black/white printing
\input{Definitions}            % Include your abbreviations



\usepackage{enumitem}% http://ctan.org/pkg/enumitem
\usepackage{multirow}
\usepackage{float}
\usepackage{amsmath}
\usepackage{multicol}
\usepackage{amssymb}
\usepackage[normalem]{ulem}
\useunder{\uline}{\ul}{}
\usepackage{wrapfig}


\usepackage[table,xcdraw]{xcolor}


%% ----------------------------------------------------------------
\begin{document}
\frontmatter
\title      {Heterogeneous Agent-based Model for Supermarket Competition}
\authors    {\texorpdfstring
             {\href{mailto:sc22g13@ecs.soton.ac.uk}{Stefan J. Collier}}
             {Stefan J. Collier}
            }
\addresses  {\groupname\\\deptname\\\univname}
\date       {\today}
\subject    {}
\keywords   {}
\supervisor {Dr. Maria Polukarov}
\examiner   {Professor Sheng Chen}

\maketitle
\begin{abstract}
This project aim was to model and analyse the effects of competitive pricing behaviors of grocery retailers on the British market. 

This was achieved by creating a multi-agent model, containing retailer and consumer agents. The heterogeneous crowd of retailers employs either a uniform pricing strategy or a ‘local price flexing’ strategy. The actions of these retailers are chosen by predicting the profit of each action, using a perceptron. Following on from the consideration of different economic models, a discrete model was developed so that software agents have a discrete environment to operate within. Within the model, it has been observed how supermarkets with differing behaviors affect a heterogeneous crowd of consumer agents. The model was implemented in Java with Python used to evaluate the results. 

The simulation displays good acceptance with real grocery market behavior, i.e. captures the performance of British retailers thus can be used to determine the impact of changes in their behavior on their competitors and consumers.Furthermore it can be used to provide insight into sustainability of volatile pricing strategies, providing a useful insight in volatility of British supermarket retail industry. 
\end{abstract}
\acknowledgements{
I would like to express my sincere gratitude to Dr Maria Polukarov for her guidance and support which provided me the freedom to take this research in the direction of my interest.\\
\\
I would also like to thank my family and friends for their encouragement and support. To those who quietly listened to my software complaints. To those who worked throughout the nights with me. To those who helped me write what I couldn't say. I cannot thank you enough.
}

\declaration{
I, Stefan Collier, declare that this dissertation and the work presented in it are my own and has been generated by me as the result of my own original research.\\
I confirm that:\\
1. This work was done wholly or mainly while in candidature for a degree at this University;\\
2. Where any part of this dissertation has previously been submitted for any other qualification at this University or any other institution, this has been clearly stated;\\
3. Where I have consulted the published work of others, this is always clearly attributed;\\
4. Where I have quoted from the work of others, the source is always given. With the exception of such quotations, this dissertation is entirely my own work;\\
5. I have acknowledged all main sources of help;\\
6. Where the thesis is based on work done by myself jointly with others, I have made clear exactly what was done by others and what I have contributed myself;\\
7. Either none of this work has been published before submission, or parts of this work have been published by :\\
\\
Stefan Collier\\
April 2016
}
\tableofcontents
\listoffigures
\listoftables

\mainmatter
%% ----------------------------------------------------------------
%\include{Introduction}
%\include{Conclusions}
\include{chapters/1Project/main}
\include{chapters/2Lit/main}
\include{chapters/3Design/HighLevel}
\include{chapters/3Design/InDepth}
\include{chapters/4Impl/main}

\include{chapters/5Experiments/1/main}
\include{chapters/5Experiments/2/main}
\include{chapters/5Experiments/3/main}
\include{chapters/5Experiments/4/main}

\include{chapters/6Conclusion/main}

\appendix
\include{appendix/AppendixB}
\include{appendix/D/main}
\include{appendix/AppendixC}

\backmatter
\bibliographystyle{ecs}
\bibliography{ECS}
\end{document}
%% ----------------------------------------------------------------

 %% ----------------------------------------------------------------
%% Progress.tex
%% ---------------------------------------------------------------- 
\documentclass{ecsprogress}    % Use the progress Style
\graphicspath{{../figs/}}   % Location of your graphics files
    \usepackage{natbib}            % Use Natbib style for the refs.
\hypersetup{colorlinks=true}   % Set to false for black/white printing
\input{Definitions}            % Include your abbreviations



\usepackage{enumitem}% http://ctan.org/pkg/enumitem
\usepackage{multirow}
\usepackage{float}
\usepackage{amsmath}
\usepackage{multicol}
\usepackage{amssymb}
\usepackage[normalem]{ulem}
\useunder{\uline}{\ul}{}
\usepackage{wrapfig}


\usepackage[table,xcdraw]{xcolor}


%% ----------------------------------------------------------------
\begin{document}
\frontmatter
\title      {Heterogeneous Agent-based Model for Supermarket Competition}
\authors    {\texorpdfstring
             {\href{mailto:sc22g13@ecs.soton.ac.uk}{Stefan J. Collier}}
             {Stefan J. Collier}
            }
\addresses  {\groupname\\\deptname\\\univname}
\date       {\today}
\subject    {}
\keywords   {}
\supervisor {Dr. Maria Polukarov}
\examiner   {Professor Sheng Chen}

\maketitle
\begin{abstract}
This project aim was to model and analyse the effects of competitive pricing behaviors of grocery retailers on the British market. 

This was achieved by creating a multi-agent model, containing retailer and consumer agents. The heterogeneous crowd of retailers employs either a uniform pricing strategy or a ‘local price flexing’ strategy. The actions of these retailers are chosen by predicting the profit of each action, using a perceptron. Following on from the consideration of different economic models, a discrete model was developed so that software agents have a discrete environment to operate within. Within the model, it has been observed how supermarkets with differing behaviors affect a heterogeneous crowd of consumer agents. The model was implemented in Java with Python used to evaluate the results. 

The simulation displays good acceptance with real grocery market behavior, i.e. captures the performance of British retailers thus can be used to determine the impact of changes in their behavior on their competitors and consumers.Furthermore it can be used to provide insight into sustainability of volatile pricing strategies, providing a useful insight in volatility of British supermarket retail industry. 
\end{abstract}
\acknowledgements{
I would like to express my sincere gratitude to Dr Maria Polukarov for her guidance and support which provided me the freedom to take this research in the direction of my interest.\\
\\
I would also like to thank my family and friends for their encouragement and support. To those who quietly listened to my software complaints. To those who worked throughout the nights with me. To those who helped me write what I couldn't say. I cannot thank you enough.
}

\declaration{
I, Stefan Collier, declare that this dissertation and the work presented in it are my own and has been generated by me as the result of my own original research.\\
I confirm that:\\
1. This work was done wholly or mainly while in candidature for a degree at this University;\\
2. Where any part of this dissertation has previously been submitted for any other qualification at this University or any other institution, this has been clearly stated;\\
3. Where I have consulted the published work of others, this is always clearly attributed;\\
4. Where I have quoted from the work of others, the source is always given. With the exception of such quotations, this dissertation is entirely my own work;\\
5. I have acknowledged all main sources of help;\\
6. Where the thesis is based on work done by myself jointly with others, I have made clear exactly what was done by others and what I have contributed myself;\\
7. Either none of this work has been published before submission, or parts of this work have been published by :\\
\\
Stefan Collier\\
April 2016
}
\tableofcontents
\listoffigures
\listoftables

\mainmatter
%% ----------------------------------------------------------------
%\include{Introduction}
%\include{Conclusions}
\include{chapters/1Project/main}
\include{chapters/2Lit/main}
\include{chapters/3Design/HighLevel}
\include{chapters/3Design/InDepth}
\include{chapters/4Impl/main}

\include{chapters/5Experiments/1/main}
\include{chapters/5Experiments/2/main}
\include{chapters/5Experiments/3/main}
\include{chapters/5Experiments/4/main}

\include{chapters/6Conclusion/main}

\appendix
\include{appendix/AppendixB}
\include{appendix/D/main}
\include{appendix/AppendixC}

\backmatter
\bibliographystyle{ecs}
\bibliography{ECS}
\end{document}
%% ----------------------------------------------------------------


 %% ----------------------------------------------------------------
%% Progress.tex
%% ---------------------------------------------------------------- 
\documentclass{ecsprogress}    % Use the progress Style
\graphicspath{{../figs/}}   % Location of your graphics files
    \usepackage{natbib}            % Use Natbib style for the refs.
\hypersetup{colorlinks=true}   % Set to false for black/white printing
\input{Definitions}            % Include your abbreviations



\usepackage{enumitem}% http://ctan.org/pkg/enumitem
\usepackage{multirow}
\usepackage{float}
\usepackage{amsmath}
\usepackage{multicol}
\usepackage{amssymb}
\usepackage[normalem]{ulem}
\useunder{\uline}{\ul}{}
\usepackage{wrapfig}


\usepackage[table,xcdraw]{xcolor}


%% ----------------------------------------------------------------
\begin{document}
\frontmatter
\title      {Heterogeneous Agent-based Model for Supermarket Competition}
\authors    {\texorpdfstring
             {\href{mailto:sc22g13@ecs.soton.ac.uk}{Stefan J. Collier}}
             {Stefan J. Collier}
            }
\addresses  {\groupname\\\deptname\\\univname}
\date       {\today}
\subject    {}
\keywords   {}
\supervisor {Dr. Maria Polukarov}
\examiner   {Professor Sheng Chen}

\maketitle
\begin{abstract}
This project aim was to model and analyse the effects of competitive pricing behaviors of grocery retailers on the British market. 

This was achieved by creating a multi-agent model, containing retailer and consumer agents. The heterogeneous crowd of retailers employs either a uniform pricing strategy or a ‘local price flexing’ strategy. The actions of these retailers are chosen by predicting the profit of each action, using a perceptron. Following on from the consideration of different economic models, a discrete model was developed so that software agents have a discrete environment to operate within. Within the model, it has been observed how supermarkets with differing behaviors affect a heterogeneous crowd of consumer agents. The model was implemented in Java with Python used to evaluate the results. 

The simulation displays good acceptance with real grocery market behavior, i.e. captures the performance of British retailers thus can be used to determine the impact of changes in their behavior on their competitors and consumers.Furthermore it can be used to provide insight into sustainability of volatile pricing strategies, providing a useful insight in volatility of British supermarket retail industry. 
\end{abstract}
\acknowledgements{
I would like to express my sincere gratitude to Dr Maria Polukarov for her guidance and support which provided me the freedom to take this research in the direction of my interest.\\
\\
I would also like to thank my family and friends for their encouragement and support. To those who quietly listened to my software complaints. To those who worked throughout the nights with me. To those who helped me write what I couldn't say. I cannot thank you enough.
}

\declaration{
I, Stefan Collier, declare that this dissertation and the work presented in it are my own and has been generated by me as the result of my own original research.\\
I confirm that:\\
1. This work was done wholly or mainly while in candidature for a degree at this University;\\
2. Where any part of this dissertation has previously been submitted for any other qualification at this University or any other institution, this has been clearly stated;\\
3. Where I have consulted the published work of others, this is always clearly attributed;\\
4. Where I have quoted from the work of others, the source is always given. With the exception of such quotations, this dissertation is entirely my own work;\\
5. I have acknowledged all main sources of help;\\
6. Where the thesis is based on work done by myself jointly with others, I have made clear exactly what was done by others and what I have contributed myself;\\
7. Either none of this work has been published before submission, or parts of this work have been published by :\\
\\
Stefan Collier\\
April 2016
}
\tableofcontents
\listoffigures
\listoftables

\mainmatter
%% ----------------------------------------------------------------
%\include{Introduction}
%\include{Conclusions}
\include{chapters/1Project/main}
\include{chapters/2Lit/main}
\include{chapters/3Design/HighLevel}
\include{chapters/3Design/InDepth}
\include{chapters/4Impl/main}

\include{chapters/5Experiments/1/main}
\include{chapters/5Experiments/2/main}
\include{chapters/5Experiments/3/main}
\include{chapters/5Experiments/4/main}

\include{chapters/6Conclusion/main}

\appendix
\include{appendix/AppendixB}
\include{appendix/D/main}
\include{appendix/AppendixC}

\backmatter
\bibliographystyle{ecs}
\bibliography{ECS}
\end{document}
%% ----------------------------------------------------------------


\appendix
\include{appendix/AppendixB}
 %% ----------------------------------------------------------------
%% Progress.tex
%% ---------------------------------------------------------------- 
\documentclass{ecsprogress}    % Use the progress Style
\graphicspath{{../figs/}}   % Location of your graphics files
    \usepackage{natbib}            % Use Natbib style for the refs.
\hypersetup{colorlinks=true}   % Set to false for black/white printing
\input{Definitions}            % Include your abbreviations



\usepackage{enumitem}% http://ctan.org/pkg/enumitem
\usepackage{multirow}
\usepackage{float}
\usepackage{amsmath}
\usepackage{multicol}
\usepackage{amssymb}
\usepackage[normalem]{ulem}
\useunder{\uline}{\ul}{}
\usepackage{wrapfig}


\usepackage[table,xcdraw]{xcolor}


%% ----------------------------------------------------------------
\begin{document}
\frontmatter
\title      {Heterogeneous Agent-based Model for Supermarket Competition}
\authors    {\texorpdfstring
             {\href{mailto:sc22g13@ecs.soton.ac.uk}{Stefan J. Collier}}
             {Stefan J. Collier}
            }
\addresses  {\groupname\\\deptname\\\univname}
\date       {\today}
\subject    {}
\keywords   {}
\supervisor {Dr. Maria Polukarov}
\examiner   {Professor Sheng Chen}

\maketitle
\begin{abstract}
This project aim was to model and analyse the effects of competitive pricing behaviors of grocery retailers on the British market. 

This was achieved by creating a multi-agent model, containing retailer and consumer agents. The heterogeneous crowd of retailers employs either a uniform pricing strategy or a ‘local price flexing’ strategy. The actions of these retailers are chosen by predicting the profit of each action, using a perceptron. Following on from the consideration of different economic models, a discrete model was developed so that software agents have a discrete environment to operate within. Within the model, it has been observed how supermarkets with differing behaviors affect a heterogeneous crowd of consumer agents. The model was implemented in Java with Python used to evaluate the results. 

The simulation displays good acceptance with real grocery market behavior, i.e. captures the performance of British retailers thus can be used to determine the impact of changes in their behavior on their competitors and consumers.Furthermore it can be used to provide insight into sustainability of volatile pricing strategies, providing a useful insight in volatility of British supermarket retail industry. 
\end{abstract}
\acknowledgements{
I would like to express my sincere gratitude to Dr Maria Polukarov for her guidance and support which provided me the freedom to take this research in the direction of my interest.\\
\\
I would also like to thank my family and friends for their encouragement and support. To those who quietly listened to my software complaints. To those who worked throughout the nights with me. To those who helped me write what I couldn't say. I cannot thank you enough.
}

\declaration{
I, Stefan Collier, declare that this dissertation and the work presented in it are my own and has been generated by me as the result of my own original research.\\
I confirm that:\\
1. This work was done wholly or mainly while in candidature for a degree at this University;\\
2. Where any part of this dissertation has previously been submitted for any other qualification at this University or any other institution, this has been clearly stated;\\
3. Where I have consulted the published work of others, this is always clearly attributed;\\
4. Where I have quoted from the work of others, the source is always given. With the exception of such quotations, this dissertation is entirely my own work;\\
5. I have acknowledged all main sources of help;\\
6. Where the thesis is based on work done by myself jointly with others, I have made clear exactly what was done by others and what I have contributed myself;\\
7. Either none of this work has been published before submission, or parts of this work have been published by :\\
\\
Stefan Collier\\
April 2016
}
\tableofcontents
\listoffigures
\listoftables

\mainmatter
%% ----------------------------------------------------------------
%\include{Introduction}
%\include{Conclusions}
\include{chapters/1Project/main}
\include{chapters/2Lit/main}
\include{chapters/3Design/HighLevel}
\include{chapters/3Design/InDepth}
\include{chapters/4Impl/main}

\include{chapters/5Experiments/1/main}
\include{chapters/5Experiments/2/main}
\include{chapters/5Experiments/3/main}
\include{chapters/5Experiments/4/main}

\include{chapters/6Conclusion/main}

\appendix
\include{appendix/AppendixB}
\include{appendix/D/main}
\include{appendix/AppendixC}

\backmatter
\bibliographystyle{ecs}
\bibliography{ECS}
\end{document}
%% ----------------------------------------------------------------

\include{appendix/AppendixC}

\backmatter
\bibliographystyle{ecs}
\bibliography{ECS}
\end{document}
%% ----------------------------------------------------------------

 %% ----------------------------------------------------------------
%% Progress.tex
%% ---------------------------------------------------------------- 
\documentclass{ecsprogress}    % Use the progress Style
\graphicspath{{../figs/}}   % Location of your graphics files
    \usepackage{natbib}            % Use Natbib style for the refs.
\hypersetup{colorlinks=true}   % Set to false for black/white printing
\input{Definitions}            % Include your abbreviations



\usepackage{enumitem}% http://ctan.org/pkg/enumitem
\usepackage{multirow}
\usepackage{float}
\usepackage{amsmath}
\usepackage{multicol}
\usepackage{amssymb}
\usepackage[normalem]{ulem}
\useunder{\uline}{\ul}{}
\usepackage{wrapfig}


\usepackage[table,xcdraw]{xcolor}


%% ----------------------------------------------------------------
\begin{document}
\frontmatter
\title      {Heterogeneous Agent-based Model for Supermarket Competition}
\authors    {\texorpdfstring
             {\href{mailto:sc22g13@ecs.soton.ac.uk}{Stefan J. Collier}}
             {Stefan J. Collier}
            }
\addresses  {\groupname\\\deptname\\\univname}
\date       {\today}
\subject    {}
\keywords   {}
\supervisor {Dr. Maria Polukarov}
\examiner   {Professor Sheng Chen}

\maketitle
\begin{abstract}
This project aim was to model and analyse the effects of competitive pricing behaviors of grocery retailers on the British market. 

This was achieved by creating a multi-agent model, containing retailer and consumer agents. The heterogeneous crowd of retailers employs either a uniform pricing strategy or a ‘local price flexing’ strategy. The actions of these retailers are chosen by predicting the profit of each action, using a perceptron. Following on from the consideration of different economic models, a discrete model was developed so that software agents have a discrete environment to operate within. Within the model, it has been observed how supermarkets with differing behaviors affect a heterogeneous crowd of consumer agents. The model was implemented in Java with Python used to evaluate the results. 

The simulation displays good acceptance with real grocery market behavior, i.e. captures the performance of British retailers thus can be used to determine the impact of changes in their behavior on their competitors and consumers.Furthermore it can be used to provide insight into sustainability of volatile pricing strategies, providing a useful insight in volatility of British supermarket retail industry. 
\end{abstract}
\acknowledgements{
I would like to express my sincere gratitude to Dr Maria Polukarov for her guidance and support which provided me the freedom to take this research in the direction of my interest.\\
\\
I would also like to thank my family and friends for their encouragement and support. To those who quietly listened to my software complaints. To those who worked throughout the nights with me. To those who helped me write what I couldn't say. I cannot thank you enough.
}

\declaration{
I, Stefan Collier, declare that this dissertation and the work presented in it are my own and has been generated by me as the result of my own original research.\\
I confirm that:\\
1. This work was done wholly or mainly while in candidature for a degree at this University;\\
2. Where any part of this dissertation has previously been submitted for any other qualification at this University or any other institution, this has been clearly stated;\\
3. Where I have consulted the published work of others, this is always clearly attributed;\\
4. Where I have quoted from the work of others, the source is always given. With the exception of such quotations, this dissertation is entirely my own work;\\
5. I have acknowledged all main sources of help;\\
6. Where the thesis is based on work done by myself jointly with others, I have made clear exactly what was done by others and what I have contributed myself;\\
7. Either none of this work has been published before submission, or parts of this work have been published by :\\
\\
Stefan Collier\\
April 2016
}
\tableofcontents
\listoffigures
\listoftables

\mainmatter
%% ----------------------------------------------------------------
%\include{Introduction}
%\include{Conclusions}
 %% ----------------------------------------------------------------
%% Progress.tex
%% ---------------------------------------------------------------- 
\documentclass{ecsprogress}    % Use the progress Style
\graphicspath{{../figs/}}   % Location of your graphics files
    \usepackage{natbib}            % Use Natbib style for the refs.
\hypersetup{colorlinks=true}   % Set to false for black/white printing
\input{Definitions}            % Include your abbreviations



\usepackage{enumitem}% http://ctan.org/pkg/enumitem
\usepackage{multirow}
\usepackage{float}
\usepackage{amsmath}
\usepackage{multicol}
\usepackage{amssymb}
\usepackage[normalem]{ulem}
\useunder{\uline}{\ul}{}
\usepackage{wrapfig}


\usepackage[table,xcdraw]{xcolor}


%% ----------------------------------------------------------------
\begin{document}
\frontmatter
\title      {Heterogeneous Agent-based Model for Supermarket Competition}
\authors    {\texorpdfstring
             {\href{mailto:sc22g13@ecs.soton.ac.uk}{Stefan J. Collier}}
             {Stefan J. Collier}
            }
\addresses  {\groupname\\\deptname\\\univname}
\date       {\today}
\subject    {}
\keywords   {}
\supervisor {Dr. Maria Polukarov}
\examiner   {Professor Sheng Chen}

\maketitle
\begin{abstract}
This project aim was to model and analyse the effects of competitive pricing behaviors of grocery retailers on the British market. 

This was achieved by creating a multi-agent model, containing retailer and consumer agents. The heterogeneous crowd of retailers employs either a uniform pricing strategy or a ‘local price flexing’ strategy. The actions of these retailers are chosen by predicting the profit of each action, using a perceptron. Following on from the consideration of different economic models, a discrete model was developed so that software agents have a discrete environment to operate within. Within the model, it has been observed how supermarkets with differing behaviors affect a heterogeneous crowd of consumer agents. The model was implemented in Java with Python used to evaluate the results. 

The simulation displays good acceptance with real grocery market behavior, i.e. captures the performance of British retailers thus can be used to determine the impact of changes in their behavior on their competitors and consumers.Furthermore it can be used to provide insight into sustainability of volatile pricing strategies, providing a useful insight in volatility of British supermarket retail industry. 
\end{abstract}
\acknowledgements{
I would like to express my sincere gratitude to Dr Maria Polukarov for her guidance and support which provided me the freedom to take this research in the direction of my interest.\\
\\
I would also like to thank my family and friends for their encouragement and support. To those who quietly listened to my software complaints. To those who worked throughout the nights with me. To those who helped me write what I couldn't say. I cannot thank you enough.
}

\declaration{
I, Stefan Collier, declare that this dissertation and the work presented in it are my own and has been generated by me as the result of my own original research.\\
I confirm that:\\
1. This work was done wholly or mainly while in candidature for a degree at this University;\\
2. Where any part of this dissertation has previously been submitted for any other qualification at this University or any other institution, this has been clearly stated;\\
3. Where I have consulted the published work of others, this is always clearly attributed;\\
4. Where I have quoted from the work of others, the source is always given. With the exception of such quotations, this dissertation is entirely my own work;\\
5. I have acknowledged all main sources of help;\\
6. Where the thesis is based on work done by myself jointly with others, I have made clear exactly what was done by others and what I have contributed myself;\\
7. Either none of this work has been published before submission, or parts of this work have been published by :\\
\\
Stefan Collier\\
April 2016
}
\tableofcontents
\listoffigures
\listoftables

\mainmatter
%% ----------------------------------------------------------------
%\include{Introduction}
%\include{Conclusions}
\include{chapters/1Project/main}
\include{chapters/2Lit/main}
\include{chapters/3Design/HighLevel}
\include{chapters/3Design/InDepth}
\include{chapters/4Impl/main}

\include{chapters/5Experiments/1/main}
\include{chapters/5Experiments/2/main}
\include{chapters/5Experiments/3/main}
\include{chapters/5Experiments/4/main}

\include{chapters/6Conclusion/main}

\appendix
\include{appendix/AppendixB}
\include{appendix/D/main}
\include{appendix/AppendixC}

\backmatter
\bibliographystyle{ecs}
\bibliography{ECS}
\end{document}
%% ----------------------------------------------------------------

 %% ----------------------------------------------------------------
%% Progress.tex
%% ---------------------------------------------------------------- 
\documentclass{ecsprogress}    % Use the progress Style
\graphicspath{{../figs/}}   % Location of your graphics files
    \usepackage{natbib}            % Use Natbib style for the refs.
\hypersetup{colorlinks=true}   % Set to false for black/white printing
\input{Definitions}            % Include your abbreviations



\usepackage{enumitem}% http://ctan.org/pkg/enumitem
\usepackage{multirow}
\usepackage{float}
\usepackage{amsmath}
\usepackage{multicol}
\usepackage{amssymb}
\usepackage[normalem]{ulem}
\useunder{\uline}{\ul}{}
\usepackage{wrapfig}


\usepackage[table,xcdraw]{xcolor}


%% ----------------------------------------------------------------
\begin{document}
\frontmatter
\title      {Heterogeneous Agent-based Model for Supermarket Competition}
\authors    {\texorpdfstring
             {\href{mailto:sc22g13@ecs.soton.ac.uk}{Stefan J. Collier}}
             {Stefan J. Collier}
            }
\addresses  {\groupname\\\deptname\\\univname}
\date       {\today}
\subject    {}
\keywords   {}
\supervisor {Dr. Maria Polukarov}
\examiner   {Professor Sheng Chen}

\maketitle
\begin{abstract}
This project aim was to model and analyse the effects of competitive pricing behaviors of grocery retailers on the British market. 

This was achieved by creating a multi-agent model, containing retailer and consumer agents. The heterogeneous crowd of retailers employs either a uniform pricing strategy or a ‘local price flexing’ strategy. The actions of these retailers are chosen by predicting the profit of each action, using a perceptron. Following on from the consideration of different economic models, a discrete model was developed so that software agents have a discrete environment to operate within. Within the model, it has been observed how supermarkets with differing behaviors affect a heterogeneous crowd of consumer agents. The model was implemented in Java with Python used to evaluate the results. 

The simulation displays good acceptance with real grocery market behavior, i.e. captures the performance of British retailers thus can be used to determine the impact of changes in their behavior on their competitors and consumers.Furthermore it can be used to provide insight into sustainability of volatile pricing strategies, providing a useful insight in volatility of British supermarket retail industry. 
\end{abstract}
\acknowledgements{
I would like to express my sincere gratitude to Dr Maria Polukarov for her guidance and support which provided me the freedom to take this research in the direction of my interest.\\
\\
I would also like to thank my family and friends for their encouragement and support. To those who quietly listened to my software complaints. To those who worked throughout the nights with me. To those who helped me write what I couldn't say. I cannot thank you enough.
}

\declaration{
I, Stefan Collier, declare that this dissertation and the work presented in it are my own and has been generated by me as the result of my own original research.\\
I confirm that:\\
1. This work was done wholly or mainly while in candidature for a degree at this University;\\
2. Where any part of this dissertation has previously been submitted for any other qualification at this University or any other institution, this has been clearly stated;\\
3. Where I have consulted the published work of others, this is always clearly attributed;\\
4. Where I have quoted from the work of others, the source is always given. With the exception of such quotations, this dissertation is entirely my own work;\\
5. I have acknowledged all main sources of help;\\
6. Where the thesis is based on work done by myself jointly with others, I have made clear exactly what was done by others and what I have contributed myself;\\
7. Either none of this work has been published before submission, or parts of this work have been published by :\\
\\
Stefan Collier\\
April 2016
}
\tableofcontents
\listoffigures
\listoftables

\mainmatter
%% ----------------------------------------------------------------
%\include{Introduction}
%\include{Conclusions}
\include{chapters/1Project/main}
\include{chapters/2Lit/main}
\include{chapters/3Design/HighLevel}
\include{chapters/3Design/InDepth}
\include{chapters/4Impl/main}

\include{chapters/5Experiments/1/main}
\include{chapters/5Experiments/2/main}
\include{chapters/5Experiments/3/main}
\include{chapters/5Experiments/4/main}

\include{chapters/6Conclusion/main}

\appendix
\include{appendix/AppendixB}
\include{appendix/D/main}
\include{appendix/AppendixC}

\backmatter
\bibliographystyle{ecs}
\bibliography{ECS}
\end{document}
%% ----------------------------------------------------------------

\include{chapters/3Design/HighLevel}
\include{chapters/3Design/InDepth}
 %% ----------------------------------------------------------------
%% Progress.tex
%% ---------------------------------------------------------------- 
\documentclass{ecsprogress}    % Use the progress Style
\graphicspath{{../figs/}}   % Location of your graphics files
    \usepackage{natbib}            % Use Natbib style for the refs.
\hypersetup{colorlinks=true}   % Set to false for black/white printing
\input{Definitions}            % Include your abbreviations



\usepackage{enumitem}% http://ctan.org/pkg/enumitem
\usepackage{multirow}
\usepackage{float}
\usepackage{amsmath}
\usepackage{multicol}
\usepackage{amssymb}
\usepackage[normalem]{ulem}
\useunder{\uline}{\ul}{}
\usepackage{wrapfig}


\usepackage[table,xcdraw]{xcolor}


%% ----------------------------------------------------------------
\begin{document}
\frontmatter
\title      {Heterogeneous Agent-based Model for Supermarket Competition}
\authors    {\texorpdfstring
             {\href{mailto:sc22g13@ecs.soton.ac.uk}{Stefan J. Collier}}
             {Stefan J. Collier}
            }
\addresses  {\groupname\\\deptname\\\univname}
\date       {\today}
\subject    {}
\keywords   {}
\supervisor {Dr. Maria Polukarov}
\examiner   {Professor Sheng Chen}

\maketitle
\begin{abstract}
This project aim was to model and analyse the effects of competitive pricing behaviors of grocery retailers on the British market. 

This was achieved by creating a multi-agent model, containing retailer and consumer agents. The heterogeneous crowd of retailers employs either a uniform pricing strategy or a ‘local price flexing’ strategy. The actions of these retailers are chosen by predicting the profit of each action, using a perceptron. Following on from the consideration of different economic models, a discrete model was developed so that software agents have a discrete environment to operate within. Within the model, it has been observed how supermarkets with differing behaviors affect a heterogeneous crowd of consumer agents. The model was implemented in Java with Python used to evaluate the results. 

The simulation displays good acceptance with real grocery market behavior, i.e. captures the performance of British retailers thus can be used to determine the impact of changes in their behavior on their competitors and consumers.Furthermore it can be used to provide insight into sustainability of volatile pricing strategies, providing a useful insight in volatility of British supermarket retail industry. 
\end{abstract}
\acknowledgements{
I would like to express my sincere gratitude to Dr Maria Polukarov for her guidance and support which provided me the freedom to take this research in the direction of my interest.\\
\\
I would also like to thank my family and friends for their encouragement and support. To those who quietly listened to my software complaints. To those who worked throughout the nights with me. To those who helped me write what I couldn't say. I cannot thank you enough.
}

\declaration{
I, Stefan Collier, declare that this dissertation and the work presented in it are my own and has been generated by me as the result of my own original research.\\
I confirm that:\\
1. This work was done wholly or mainly while in candidature for a degree at this University;\\
2. Where any part of this dissertation has previously been submitted for any other qualification at this University or any other institution, this has been clearly stated;\\
3. Where I have consulted the published work of others, this is always clearly attributed;\\
4. Where I have quoted from the work of others, the source is always given. With the exception of such quotations, this dissertation is entirely my own work;\\
5. I have acknowledged all main sources of help;\\
6. Where the thesis is based on work done by myself jointly with others, I have made clear exactly what was done by others and what I have contributed myself;\\
7. Either none of this work has been published before submission, or parts of this work have been published by :\\
\\
Stefan Collier\\
April 2016
}
\tableofcontents
\listoffigures
\listoftables

\mainmatter
%% ----------------------------------------------------------------
%\include{Introduction}
%\include{Conclusions}
\include{chapters/1Project/main}
\include{chapters/2Lit/main}
\include{chapters/3Design/HighLevel}
\include{chapters/3Design/InDepth}
\include{chapters/4Impl/main}

\include{chapters/5Experiments/1/main}
\include{chapters/5Experiments/2/main}
\include{chapters/5Experiments/3/main}
\include{chapters/5Experiments/4/main}

\include{chapters/6Conclusion/main}

\appendix
\include{appendix/AppendixB}
\include{appendix/D/main}
\include{appendix/AppendixC}

\backmatter
\bibliographystyle{ecs}
\bibliography{ECS}
\end{document}
%% ----------------------------------------------------------------


 %% ----------------------------------------------------------------
%% Progress.tex
%% ---------------------------------------------------------------- 
\documentclass{ecsprogress}    % Use the progress Style
\graphicspath{{../figs/}}   % Location of your graphics files
    \usepackage{natbib}            % Use Natbib style for the refs.
\hypersetup{colorlinks=true}   % Set to false for black/white printing
\input{Definitions}            % Include your abbreviations



\usepackage{enumitem}% http://ctan.org/pkg/enumitem
\usepackage{multirow}
\usepackage{float}
\usepackage{amsmath}
\usepackage{multicol}
\usepackage{amssymb}
\usepackage[normalem]{ulem}
\useunder{\uline}{\ul}{}
\usepackage{wrapfig}


\usepackage[table,xcdraw]{xcolor}


%% ----------------------------------------------------------------
\begin{document}
\frontmatter
\title      {Heterogeneous Agent-based Model for Supermarket Competition}
\authors    {\texorpdfstring
             {\href{mailto:sc22g13@ecs.soton.ac.uk}{Stefan J. Collier}}
             {Stefan J. Collier}
            }
\addresses  {\groupname\\\deptname\\\univname}
\date       {\today}
\subject    {}
\keywords   {}
\supervisor {Dr. Maria Polukarov}
\examiner   {Professor Sheng Chen}

\maketitle
\begin{abstract}
This project aim was to model and analyse the effects of competitive pricing behaviors of grocery retailers on the British market. 

This was achieved by creating a multi-agent model, containing retailer and consumer agents. The heterogeneous crowd of retailers employs either a uniform pricing strategy or a ‘local price flexing’ strategy. The actions of these retailers are chosen by predicting the profit of each action, using a perceptron. Following on from the consideration of different economic models, a discrete model was developed so that software agents have a discrete environment to operate within. Within the model, it has been observed how supermarkets with differing behaviors affect a heterogeneous crowd of consumer agents. The model was implemented in Java with Python used to evaluate the results. 

The simulation displays good acceptance with real grocery market behavior, i.e. captures the performance of British retailers thus can be used to determine the impact of changes in their behavior on their competitors and consumers.Furthermore it can be used to provide insight into sustainability of volatile pricing strategies, providing a useful insight in volatility of British supermarket retail industry. 
\end{abstract}
\acknowledgements{
I would like to express my sincere gratitude to Dr Maria Polukarov for her guidance and support which provided me the freedom to take this research in the direction of my interest.\\
\\
I would also like to thank my family and friends for their encouragement and support. To those who quietly listened to my software complaints. To those who worked throughout the nights with me. To those who helped me write what I couldn't say. I cannot thank you enough.
}

\declaration{
I, Stefan Collier, declare that this dissertation and the work presented in it are my own and has been generated by me as the result of my own original research.\\
I confirm that:\\
1. This work was done wholly or mainly while in candidature for a degree at this University;\\
2. Where any part of this dissertation has previously been submitted for any other qualification at this University or any other institution, this has been clearly stated;\\
3. Where I have consulted the published work of others, this is always clearly attributed;\\
4. Where I have quoted from the work of others, the source is always given. With the exception of such quotations, this dissertation is entirely my own work;\\
5. I have acknowledged all main sources of help;\\
6. Where the thesis is based on work done by myself jointly with others, I have made clear exactly what was done by others and what I have contributed myself;\\
7. Either none of this work has been published before submission, or parts of this work have been published by :\\
\\
Stefan Collier\\
April 2016
}
\tableofcontents
\listoffigures
\listoftables

\mainmatter
%% ----------------------------------------------------------------
%\include{Introduction}
%\include{Conclusions}
\include{chapters/1Project/main}
\include{chapters/2Lit/main}
\include{chapters/3Design/HighLevel}
\include{chapters/3Design/InDepth}
\include{chapters/4Impl/main}

\include{chapters/5Experiments/1/main}
\include{chapters/5Experiments/2/main}
\include{chapters/5Experiments/3/main}
\include{chapters/5Experiments/4/main}

\include{chapters/6Conclusion/main}

\appendix
\include{appendix/AppendixB}
\include{appendix/D/main}
\include{appendix/AppendixC}

\backmatter
\bibliographystyle{ecs}
\bibliography{ECS}
\end{document}
%% ----------------------------------------------------------------

 %% ----------------------------------------------------------------
%% Progress.tex
%% ---------------------------------------------------------------- 
\documentclass{ecsprogress}    % Use the progress Style
\graphicspath{{../figs/}}   % Location of your graphics files
    \usepackage{natbib}            % Use Natbib style for the refs.
\hypersetup{colorlinks=true}   % Set to false for black/white printing
\input{Definitions}            % Include your abbreviations



\usepackage{enumitem}% http://ctan.org/pkg/enumitem
\usepackage{multirow}
\usepackage{float}
\usepackage{amsmath}
\usepackage{multicol}
\usepackage{amssymb}
\usepackage[normalem]{ulem}
\useunder{\uline}{\ul}{}
\usepackage{wrapfig}


\usepackage[table,xcdraw]{xcolor}


%% ----------------------------------------------------------------
\begin{document}
\frontmatter
\title      {Heterogeneous Agent-based Model for Supermarket Competition}
\authors    {\texorpdfstring
             {\href{mailto:sc22g13@ecs.soton.ac.uk}{Stefan J. Collier}}
             {Stefan J. Collier}
            }
\addresses  {\groupname\\\deptname\\\univname}
\date       {\today}
\subject    {}
\keywords   {}
\supervisor {Dr. Maria Polukarov}
\examiner   {Professor Sheng Chen}

\maketitle
\begin{abstract}
This project aim was to model and analyse the effects of competitive pricing behaviors of grocery retailers on the British market. 

This was achieved by creating a multi-agent model, containing retailer and consumer agents. The heterogeneous crowd of retailers employs either a uniform pricing strategy or a ‘local price flexing’ strategy. The actions of these retailers are chosen by predicting the profit of each action, using a perceptron. Following on from the consideration of different economic models, a discrete model was developed so that software agents have a discrete environment to operate within. Within the model, it has been observed how supermarkets with differing behaviors affect a heterogeneous crowd of consumer agents. The model was implemented in Java with Python used to evaluate the results. 

The simulation displays good acceptance with real grocery market behavior, i.e. captures the performance of British retailers thus can be used to determine the impact of changes in their behavior on their competitors and consumers.Furthermore it can be used to provide insight into sustainability of volatile pricing strategies, providing a useful insight in volatility of British supermarket retail industry. 
\end{abstract}
\acknowledgements{
I would like to express my sincere gratitude to Dr Maria Polukarov for her guidance and support which provided me the freedom to take this research in the direction of my interest.\\
\\
I would also like to thank my family and friends for their encouragement and support. To those who quietly listened to my software complaints. To those who worked throughout the nights with me. To those who helped me write what I couldn't say. I cannot thank you enough.
}

\declaration{
I, Stefan Collier, declare that this dissertation and the work presented in it are my own and has been generated by me as the result of my own original research.\\
I confirm that:\\
1. This work was done wholly or mainly while in candidature for a degree at this University;\\
2. Where any part of this dissertation has previously been submitted for any other qualification at this University or any other institution, this has been clearly stated;\\
3. Where I have consulted the published work of others, this is always clearly attributed;\\
4. Where I have quoted from the work of others, the source is always given. With the exception of such quotations, this dissertation is entirely my own work;\\
5. I have acknowledged all main sources of help;\\
6. Where the thesis is based on work done by myself jointly with others, I have made clear exactly what was done by others and what I have contributed myself;\\
7. Either none of this work has been published before submission, or parts of this work have been published by :\\
\\
Stefan Collier\\
April 2016
}
\tableofcontents
\listoffigures
\listoftables

\mainmatter
%% ----------------------------------------------------------------
%\include{Introduction}
%\include{Conclusions}
\include{chapters/1Project/main}
\include{chapters/2Lit/main}
\include{chapters/3Design/HighLevel}
\include{chapters/3Design/InDepth}
\include{chapters/4Impl/main}

\include{chapters/5Experiments/1/main}
\include{chapters/5Experiments/2/main}
\include{chapters/5Experiments/3/main}
\include{chapters/5Experiments/4/main}

\include{chapters/6Conclusion/main}

\appendix
\include{appendix/AppendixB}
\include{appendix/D/main}
\include{appendix/AppendixC}

\backmatter
\bibliographystyle{ecs}
\bibliography{ECS}
\end{document}
%% ----------------------------------------------------------------

 %% ----------------------------------------------------------------
%% Progress.tex
%% ---------------------------------------------------------------- 
\documentclass{ecsprogress}    % Use the progress Style
\graphicspath{{../figs/}}   % Location of your graphics files
    \usepackage{natbib}            % Use Natbib style for the refs.
\hypersetup{colorlinks=true}   % Set to false for black/white printing
\input{Definitions}            % Include your abbreviations



\usepackage{enumitem}% http://ctan.org/pkg/enumitem
\usepackage{multirow}
\usepackage{float}
\usepackage{amsmath}
\usepackage{multicol}
\usepackage{amssymb}
\usepackage[normalem]{ulem}
\useunder{\uline}{\ul}{}
\usepackage{wrapfig}


\usepackage[table,xcdraw]{xcolor}


%% ----------------------------------------------------------------
\begin{document}
\frontmatter
\title      {Heterogeneous Agent-based Model for Supermarket Competition}
\authors    {\texorpdfstring
             {\href{mailto:sc22g13@ecs.soton.ac.uk}{Stefan J. Collier}}
             {Stefan J. Collier}
            }
\addresses  {\groupname\\\deptname\\\univname}
\date       {\today}
\subject    {}
\keywords   {}
\supervisor {Dr. Maria Polukarov}
\examiner   {Professor Sheng Chen}

\maketitle
\begin{abstract}
This project aim was to model and analyse the effects of competitive pricing behaviors of grocery retailers on the British market. 

This was achieved by creating a multi-agent model, containing retailer and consumer agents. The heterogeneous crowd of retailers employs either a uniform pricing strategy or a ‘local price flexing’ strategy. The actions of these retailers are chosen by predicting the profit of each action, using a perceptron. Following on from the consideration of different economic models, a discrete model was developed so that software agents have a discrete environment to operate within. Within the model, it has been observed how supermarkets with differing behaviors affect a heterogeneous crowd of consumer agents. The model was implemented in Java with Python used to evaluate the results. 

The simulation displays good acceptance with real grocery market behavior, i.e. captures the performance of British retailers thus can be used to determine the impact of changes in their behavior on their competitors and consumers.Furthermore it can be used to provide insight into sustainability of volatile pricing strategies, providing a useful insight in volatility of British supermarket retail industry. 
\end{abstract}
\acknowledgements{
I would like to express my sincere gratitude to Dr Maria Polukarov for her guidance and support which provided me the freedom to take this research in the direction of my interest.\\
\\
I would also like to thank my family and friends for their encouragement and support. To those who quietly listened to my software complaints. To those who worked throughout the nights with me. To those who helped me write what I couldn't say. I cannot thank you enough.
}

\declaration{
I, Stefan Collier, declare that this dissertation and the work presented in it are my own and has been generated by me as the result of my own original research.\\
I confirm that:\\
1. This work was done wholly or mainly while in candidature for a degree at this University;\\
2. Where any part of this dissertation has previously been submitted for any other qualification at this University or any other institution, this has been clearly stated;\\
3. Where I have consulted the published work of others, this is always clearly attributed;\\
4. Where I have quoted from the work of others, the source is always given. With the exception of such quotations, this dissertation is entirely my own work;\\
5. I have acknowledged all main sources of help;\\
6. Where the thesis is based on work done by myself jointly with others, I have made clear exactly what was done by others and what I have contributed myself;\\
7. Either none of this work has been published before submission, or parts of this work have been published by :\\
\\
Stefan Collier\\
April 2016
}
\tableofcontents
\listoffigures
\listoftables

\mainmatter
%% ----------------------------------------------------------------
%\include{Introduction}
%\include{Conclusions}
\include{chapters/1Project/main}
\include{chapters/2Lit/main}
\include{chapters/3Design/HighLevel}
\include{chapters/3Design/InDepth}
\include{chapters/4Impl/main}

\include{chapters/5Experiments/1/main}
\include{chapters/5Experiments/2/main}
\include{chapters/5Experiments/3/main}
\include{chapters/5Experiments/4/main}

\include{chapters/6Conclusion/main}

\appendix
\include{appendix/AppendixB}
\include{appendix/D/main}
\include{appendix/AppendixC}

\backmatter
\bibliographystyle{ecs}
\bibliography{ECS}
\end{document}
%% ----------------------------------------------------------------

 %% ----------------------------------------------------------------
%% Progress.tex
%% ---------------------------------------------------------------- 
\documentclass{ecsprogress}    % Use the progress Style
\graphicspath{{../figs/}}   % Location of your graphics files
    \usepackage{natbib}            % Use Natbib style for the refs.
\hypersetup{colorlinks=true}   % Set to false for black/white printing
\input{Definitions}            % Include your abbreviations



\usepackage{enumitem}% http://ctan.org/pkg/enumitem
\usepackage{multirow}
\usepackage{float}
\usepackage{amsmath}
\usepackage{multicol}
\usepackage{amssymb}
\usepackage[normalem]{ulem}
\useunder{\uline}{\ul}{}
\usepackage{wrapfig}


\usepackage[table,xcdraw]{xcolor}


%% ----------------------------------------------------------------
\begin{document}
\frontmatter
\title      {Heterogeneous Agent-based Model for Supermarket Competition}
\authors    {\texorpdfstring
             {\href{mailto:sc22g13@ecs.soton.ac.uk}{Stefan J. Collier}}
             {Stefan J. Collier}
            }
\addresses  {\groupname\\\deptname\\\univname}
\date       {\today}
\subject    {}
\keywords   {}
\supervisor {Dr. Maria Polukarov}
\examiner   {Professor Sheng Chen}

\maketitle
\begin{abstract}
This project aim was to model and analyse the effects of competitive pricing behaviors of grocery retailers on the British market. 

This was achieved by creating a multi-agent model, containing retailer and consumer agents. The heterogeneous crowd of retailers employs either a uniform pricing strategy or a ‘local price flexing’ strategy. The actions of these retailers are chosen by predicting the profit of each action, using a perceptron. Following on from the consideration of different economic models, a discrete model was developed so that software agents have a discrete environment to operate within. Within the model, it has been observed how supermarkets with differing behaviors affect a heterogeneous crowd of consumer agents. The model was implemented in Java with Python used to evaluate the results. 

The simulation displays good acceptance with real grocery market behavior, i.e. captures the performance of British retailers thus can be used to determine the impact of changes in their behavior on their competitors and consumers.Furthermore it can be used to provide insight into sustainability of volatile pricing strategies, providing a useful insight in volatility of British supermarket retail industry. 
\end{abstract}
\acknowledgements{
I would like to express my sincere gratitude to Dr Maria Polukarov for her guidance and support which provided me the freedom to take this research in the direction of my interest.\\
\\
I would also like to thank my family and friends for their encouragement and support. To those who quietly listened to my software complaints. To those who worked throughout the nights with me. To those who helped me write what I couldn't say. I cannot thank you enough.
}

\declaration{
I, Stefan Collier, declare that this dissertation and the work presented in it are my own and has been generated by me as the result of my own original research.\\
I confirm that:\\
1. This work was done wholly or mainly while in candidature for a degree at this University;\\
2. Where any part of this dissertation has previously been submitted for any other qualification at this University or any other institution, this has been clearly stated;\\
3. Where I have consulted the published work of others, this is always clearly attributed;\\
4. Where I have quoted from the work of others, the source is always given. With the exception of such quotations, this dissertation is entirely my own work;\\
5. I have acknowledged all main sources of help;\\
6. Where the thesis is based on work done by myself jointly with others, I have made clear exactly what was done by others and what I have contributed myself;\\
7. Either none of this work has been published before submission, or parts of this work have been published by :\\
\\
Stefan Collier\\
April 2016
}
\tableofcontents
\listoffigures
\listoftables

\mainmatter
%% ----------------------------------------------------------------
%\include{Introduction}
%\include{Conclusions}
\include{chapters/1Project/main}
\include{chapters/2Lit/main}
\include{chapters/3Design/HighLevel}
\include{chapters/3Design/InDepth}
\include{chapters/4Impl/main}

\include{chapters/5Experiments/1/main}
\include{chapters/5Experiments/2/main}
\include{chapters/5Experiments/3/main}
\include{chapters/5Experiments/4/main}

\include{chapters/6Conclusion/main}

\appendix
\include{appendix/AppendixB}
\include{appendix/D/main}
\include{appendix/AppendixC}

\backmatter
\bibliographystyle{ecs}
\bibliography{ECS}
\end{document}
%% ----------------------------------------------------------------


 %% ----------------------------------------------------------------
%% Progress.tex
%% ---------------------------------------------------------------- 
\documentclass{ecsprogress}    % Use the progress Style
\graphicspath{{../figs/}}   % Location of your graphics files
    \usepackage{natbib}            % Use Natbib style for the refs.
\hypersetup{colorlinks=true}   % Set to false for black/white printing
\input{Definitions}            % Include your abbreviations



\usepackage{enumitem}% http://ctan.org/pkg/enumitem
\usepackage{multirow}
\usepackage{float}
\usepackage{amsmath}
\usepackage{multicol}
\usepackage{amssymb}
\usepackage[normalem]{ulem}
\useunder{\uline}{\ul}{}
\usepackage{wrapfig}


\usepackage[table,xcdraw]{xcolor}


%% ----------------------------------------------------------------
\begin{document}
\frontmatter
\title      {Heterogeneous Agent-based Model for Supermarket Competition}
\authors    {\texorpdfstring
             {\href{mailto:sc22g13@ecs.soton.ac.uk}{Stefan J. Collier}}
             {Stefan J. Collier}
            }
\addresses  {\groupname\\\deptname\\\univname}
\date       {\today}
\subject    {}
\keywords   {}
\supervisor {Dr. Maria Polukarov}
\examiner   {Professor Sheng Chen}

\maketitle
\begin{abstract}
This project aim was to model and analyse the effects of competitive pricing behaviors of grocery retailers on the British market. 

This was achieved by creating a multi-agent model, containing retailer and consumer agents. The heterogeneous crowd of retailers employs either a uniform pricing strategy or a ‘local price flexing’ strategy. The actions of these retailers are chosen by predicting the profit of each action, using a perceptron. Following on from the consideration of different economic models, a discrete model was developed so that software agents have a discrete environment to operate within. Within the model, it has been observed how supermarkets with differing behaviors affect a heterogeneous crowd of consumer agents. The model was implemented in Java with Python used to evaluate the results. 

The simulation displays good acceptance with real grocery market behavior, i.e. captures the performance of British retailers thus can be used to determine the impact of changes in their behavior on their competitors and consumers.Furthermore it can be used to provide insight into sustainability of volatile pricing strategies, providing a useful insight in volatility of British supermarket retail industry. 
\end{abstract}
\acknowledgements{
I would like to express my sincere gratitude to Dr Maria Polukarov for her guidance and support which provided me the freedom to take this research in the direction of my interest.\\
\\
I would also like to thank my family and friends for their encouragement and support. To those who quietly listened to my software complaints. To those who worked throughout the nights with me. To those who helped me write what I couldn't say. I cannot thank you enough.
}

\declaration{
I, Stefan Collier, declare that this dissertation and the work presented in it are my own and has been generated by me as the result of my own original research.\\
I confirm that:\\
1. This work was done wholly or mainly while in candidature for a degree at this University;\\
2. Where any part of this dissertation has previously been submitted for any other qualification at this University or any other institution, this has been clearly stated;\\
3. Where I have consulted the published work of others, this is always clearly attributed;\\
4. Where I have quoted from the work of others, the source is always given. With the exception of such quotations, this dissertation is entirely my own work;\\
5. I have acknowledged all main sources of help;\\
6. Where the thesis is based on work done by myself jointly with others, I have made clear exactly what was done by others and what I have contributed myself;\\
7. Either none of this work has been published before submission, or parts of this work have been published by :\\
\\
Stefan Collier\\
April 2016
}
\tableofcontents
\listoffigures
\listoftables

\mainmatter
%% ----------------------------------------------------------------
%\include{Introduction}
%\include{Conclusions}
\include{chapters/1Project/main}
\include{chapters/2Lit/main}
\include{chapters/3Design/HighLevel}
\include{chapters/3Design/InDepth}
\include{chapters/4Impl/main}

\include{chapters/5Experiments/1/main}
\include{chapters/5Experiments/2/main}
\include{chapters/5Experiments/3/main}
\include{chapters/5Experiments/4/main}

\include{chapters/6Conclusion/main}

\appendix
\include{appendix/AppendixB}
\include{appendix/D/main}
\include{appendix/AppendixC}

\backmatter
\bibliographystyle{ecs}
\bibliography{ECS}
\end{document}
%% ----------------------------------------------------------------


\appendix
\include{appendix/AppendixB}
 %% ----------------------------------------------------------------
%% Progress.tex
%% ---------------------------------------------------------------- 
\documentclass{ecsprogress}    % Use the progress Style
\graphicspath{{../figs/}}   % Location of your graphics files
    \usepackage{natbib}            % Use Natbib style for the refs.
\hypersetup{colorlinks=true}   % Set to false for black/white printing
\input{Definitions}            % Include your abbreviations



\usepackage{enumitem}% http://ctan.org/pkg/enumitem
\usepackage{multirow}
\usepackage{float}
\usepackage{amsmath}
\usepackage{multicol}
\usepackage{amssymb}
\usepackage[normalem]{ulem}
\useunder{\uline}{\ul}{}
\usepackage{wrapfig}


\usepackage[table,xcdraw]{xcolor}


%% ----------------------------------------------------------------
\begin{document}
\frontmatter
\title      {Heterogeneous Agent-based Model for Supermarket Competition}
\authors    {\texorpdfstring
             {\href{mailto:sc22g13@ecs.soton.ac.uk}{Stefan J. Collier}}
             {Stefan J. Collier}
            }
\addresses  {\groupname\\\deptname\\\univname}
\date       {\today}
\subject    {}
\keywords   {}
\supervisor {Dr. Maria Polukarov}
\examiner   {Professor Sheng Chen}

\maketitle
\begin{abstract}
This project aim was to model and analyse the effects of competitive pricing behaviors of grocery retailers on the British market. 

This was achieved by creating a multi-agent model, containing retailer and consumer agents. The heterogeneous crowd of retailers employs either a uniform pricing strategy or a ‘local price flexing’ strategy. The actions of these retailers are chosen by predicting the profit of each action, using a perceptron. Following on from the consideration of different economic models, a discrete model was developed so that software agents have a discrete environment to operate within. Within the model, it has been observed how supermarkets with differing behaviors affect a heterogeneous crowd of consumer agents. The model was implemented in Java with Python used to evaluate the results. 

The simulation displays good acceptance with real grocery market behavior, i.e. captures the performance of British retailers thus can be used to determine the impact of changes in their behavior on their competitors and consumers.Furthermore it can be used to provide insight into sustainability of volatile pricing strategies, providing a useful insight in volatility of British supermarket retail industry. 
\end{abstract}
\acknowledgements{
I would like to express my sincere gratitude to Dr Maria Polukarov for her guidance and support which provided me the freedom to take this research in the direction of my interest.\\
\\
I would also like to thank my family and friends for their encouragement and support. To those who quietly listened to my software complaints. To those who worked throughout the nights with me. To those who helped me write what I couldn't say. I cannot thank you enough.
}

\declaration{
I, Stefan Collier, declare that this dissertation and the work presented in it are my own and has been generated by me as the result of my own original research.\\
I confirm that:\\
1. This work was done wholly or mainly while in candidature for a degree at this University;\\
2. Where any part of this dissertation has previously been submitted for any other qualification at this University or any other institution, this has been clearly stated;\\
3. Where I have consulted the published work of others, this is always clearly attributed;\\
4. Where I have quoted from the work of others, the source is always given. With the exception of such quotations, this dissertation is entirely my own work;\\
5. I have acknowledged all main sources of help;\\
6. Where the thesis is based on work done by myself jointly with others, I have made clear exactly what was done by others and what I have contributed myself;\\
7. Either none of this work has been published before submission, or parts of this work have been published by :\\
\\
Stefan Collier\\
April 2016
}
\tableofcontents
\listoffigures
\listoftables

\mainmatter
%% ----------------------------------------------------------------
%\include{Introduction}
%\include{Conclusions}
\include{chapters/1Project/main}
\include{chapters/2Lit/main}
\include{chapters/3Design/HighLevel}
\include{chapters/3Design/InDepth}
\include{chapters/4Impl/main}

\include{chapters/5Experiments/1/main}
\include{chapters/5Experiments/2/main}
\include{chapters/5Experiments/3/main}
\include{chapters/5Experiments/4/main}

\include{chapters/6Conclusion/main}

\appendix
\include{appendix/AppendixB}
\include{appendix/D/main}
\include{appendix/AppendixC}

\backmatter
\bibliographystyle{ecs}
\bibliography{ECS}
\end{document}
%% ----------------------------------------------------------------

\include{appendix/AppendixC}

\backmatter
\bibliographystyle{ecs}
\bibliography{ECS}
\end{document}
%% ----------------------------------------------------------------


 %% ----------------------------------------------------------------
%% Progress.tex
%% ---------------------------------------------------------------- 
\documentclass{ecsprogress}    % Use the progress Style
\graphicspath{{../figs/}}   % Location of your graphics files
    \usepackage{natbib}            % Use Natbib style for the refs.
\hypersetup{colorlinks=true}   % Set to false for black/white printing
\input{Definitions}            % Include your abbreviations



\usepackage{enumitem}% http://ctan.org/pkg/enumitem
\usepackage{multirow}
\usepackage{float}
\usepackage{amsmath}
\usepackage{multicol}
\usepackage{amssymb}
\usepackage[normalem]{ulem}
\useunder{\uline}{\ul}{}
\usepackage{wrapfig}


\usepackage[table,xcdraw]{xcolor}


%% ----------------------------------------------------------------
\begin{document}
\frontmatter
\title      {Heterogeneous Agent-based Model for Supermarket Competition}
\authors    {\texorpdfstring
             {\href{mailto:sc22g13@ecs.soton.ac.uk}{Stefan J. Collier}}
             {Stefan J. Collier}
            }
\addresses  {\groupname\\\deptname\\\univname}
\date       {\today}
\subject    {}
\keywords   {}
\supervisor {Dr. Maria Polukarov}
\examiner   {Professor Sheng Chen}

\maketitle
\begin{abstract}
This project aim was to model and analyse the effects of competitive pricing behaviors of grocery retailers on the British market. 

This was achieved by creating a multi-agent model, containing retailer and consumer agents. The heterogeneous crowd of retailers employs either a uniform pricing strategy or a ‘local price flexing’ strategy. The actions of these retailers are chosen by predicting the profit of each action, using a perceptron. Following on from the consideration of different economic models, a discrete model was developed so that software agents have a discrete environment to operate within. Within the model, it has been observed how supermarkets with differing behaviors affect a heterogeneous crowd of consumer agents. The model was implemented in Java with Python used to evaluate the results. 

The simulation displays good acceptance with real grocery market behavior, i.e. captures the performance of British retailers thus can be used to determine the impact of changes in their behavior on their competitors and consumers.Furthermore it can be used to provide insight into sustainability of volatile pricing strategies, providing a useful insight in volatility of British supermarket retail industry. 
\end{abstract}
\acknowledgements{
I would like to express my sincere gratitude to Dr Maria Polukarov for her guidance and support which provided me the freedom to take this research in the direction of my interest.\\
\\
I would also like to thank my family and friends for their encouragement and support. To those who quietly listened to my software complaints. To those who worked throughout the nights with me. To those who helped me write what I couldn't say. I cannot thank you enough.
}

\declaration{
I, Stefan Collier, declare that this dissertation and the work presented in it are my own and has been generated by me as the result of my own original research.\\
I confirm that:\\
1. This work was done wholly or mainly while in candidature for a degree at this University;\\
2. Where any part of this dissertation has previously been submitted for any other qualification at this University or any other institution, this has been clearly stated;\\
3. Where I have consulted the published work of others, this is always clearly attributed;\\
4. Where I have quoted from the work of others, the source is always given. With the exception of such quotations, this dissertation is entirely my own work;\\
5. I have acknowledged all main sources of help;\\
6. Where the thesis is based on work done by myself jointly with others, I have made clear exactly what was done by others and what I have contributed myself;\\
7. Either none of this work has been published before submission, or parts of this work have been published by :\\
\\
Stefan Collier\\
April 2016
}
\tableofcontents
\listoffigures
\listoftables

\mainmatter
%% ----------------------------------------------------------------
%\include{Introduction}
%\include{Conclusions}
 %% ----------------------------------------------------------------
%% Progress.tex
%% ---------------------------------------------------------------- 
\documentclass{ecsprogress}    % Use the progress Style
\graphicspath{{../figs/}}   % Location of your graphics files
    \usepackage{natbib}            % Use Natbib style for the refs.
\hypersetup{colorlinks=true}   % Set to false for black/white printing
\input{Definitions}            % Include your abbreviations



\usepackage{enumitem}% http://ctan.org/pkg/enumitem
\usepackage{multirow}
\usepackage{float}
\usepackage{amsmath}
\usepackage{multicol}
\usepackage{amssymb}
\usepackage[normalem]{ulem}
\useunder{\uline}{\ul}{}
\usepackage{wrapfig}


\usepackage[table,xcdraw]{xcolor}


%% ----------------------------------------------------------------
\begin{document}
\frontmatter
\title      {Heterogeneous Agent-based Model for Supermarket Competition}
\authors    {\texorpdfstring
             {\href{mailto:sc22g13@ecs.soton.ac.uk}{Stefan J. Collier}}
             {Stefan J. Collier}
            }
\addresses  {\groupname\\\deptname\\\univname}
\date       {\today}
\subject    {}
\keywords   {}
\supervisor {Dr. Maria Polukarov}
\examiner   {Professor Sheng Chen}

\maketitle
\begin{abstract}
This project aim was to model and analyse the effects of competitive pricing behaviors of grocery retailers on the British market. 

This was achieved by creating a multi-agent model, containing retailer and consumer agents. The heterogeneous crowd of retailers employs either a uniform pricing strategy or a ‘local price flexing’ strategy. The actions of these retailers are chosen by predicting the profit of each action, using a perceptron. Following on from the consideration of different economic models, a discrete model was developed so that software agents have a discrete environment to operate within. Within the model, it has been observed how supermarkets with differing behaviors affect a heterogeneous crowd of consumer agents. The model was implemented in Java with Python used to evaluate the results. 

The simulation displays good acceptance with real grocery market behavior, i.e. captures the performance of British retailers thus can be used to determine the impact of changes in their behavior on their competitors and consumers.Furthermore it can be used to provide insight into sustainability of volatile pricing strategies, providing a useful insight in volatility of British supermarket retail industry. 
\end{abstract}
\acknowledgements{
I would like to express my sincere gratitude to Dr Maria Polukarov for her guidance and support which provided me the freedom to take this research in the direction of my interest.\\
\\
I would also like to thank my family and friends for their encouragement and support. To those who quietly listened to my software complaints. To those who worked throughout the nights with me. To those who helped me write what I couldn't say. I cannot thank you enough.
}

\declaration{
I, Stefan Collier, declare that this dissertation and the work presented in it are my own and has been generated by me as the result of my own original research.\\
I confirm that:\\
1. This work was done wholly or mainly while in candidature for a degree at this University;\\
2. Where any part of this dissertation has previously been submitted for any other qualification at this University or any other institution, this has been clearly stated;\\
3. Where I have consulted the published work of others, this is always clearly attributed;\\
4. Where I have quoted from the work of others, the source is always given. With the exception of such quotations, this dissertation is entirely my own work;\\
5. I have acknowledged all main sources of help;\\
6. Where the thesis is based on work done by myself jointly with others, I have made clear exactly what was done by others and what I have contributed myself;\\
7. Either none of this work has been published before submission, or parts of this work have been published by :\\
\\
Stefan Collier\\
April 2016
}
\tableofcontents
\listoffigures
\listoftables

\mainmatter
%% ----------------------------------------------------------------
%\include{Introduction}
%\include{Conclusions}
\include{chapters/1Project/main}
\include{chapters/2Lit/main}
\include{chapters/3Design/HighLevel}
\include{chapters/3Design/InDepth}
\include{chapters/4Impl/main}

\include{chapters/5Experiments/1/main}
\include{chapters/5Experiments/2/main}
\include{chapters/5Experiments/3/main}
\include{chapters/5Experiments/4/main}

\include{chapters/6Conclusion/main}

\appendix
\include{appendix/AppendixB}
\include{appendix/D/main}
\include{appendix/AppendixC}

\backmatter
\bibliographystyle{ecs}
\bibliography{ECS}
\end{document}
%% ----------------------------------------------------------------

 %% ----------------------------------------------------------------
%% Progress.tex
%% ---------------------------------------------------------------- 
\documentclass{ecsprogress}    % Use the progress Style
\graphicspath{{../figs/}}   % Location of your graphics files
    \usepackage{natbib}            % Use Natbib style for the refs.
\hypersetup{colorlinks=true}   % Set to false for black/white printing
\input{Definitions}            % Include your abbreviations



\usepackage{enumitem}% http://ctan.org/pkg/enumitem
\usepackage{multirow}
\usepackage{float}
\usepackage{amsmath}
\usepackage{multicol}
\usepackage{amssymb}
\usepackage[normalem]{ulem}
\useunder{\uline}{\ul}{}
\usepackage{wrapfig}


\usepackage[table,xcdraw]{xcolor}


%% ----------------------------------------------------------------
\begin{document}
\frontmatter
\title      {Heterogeneous Agent-based Model for Supermarket Competition}
\authors    {\texorpdfstring
             {\href{mailto:sc22g13@ecs.soton.ac.uk}{Stefan J. Collier}}
             {Stefan J. Collier}
            }
\addresses  {\groupname\\\deptname\\\univname}
\date       {\today}
\subject    {}
\keywords   {}
\supervisor {Dr. Maria Polukarov}
\examiner   {Professor Sheng Chen}

\maketitle
\begin{abstract}
This project aim was to model and analyse the effects of competitive pricing behaviors of grocery retailers on the British market. 

This was achieved by creating a multi-agent model, containing retailer and consumer agents. The heterogeneous crowd of retailers employs either a uniform pricing strategy or a ‘local price flexing’ strategy. The actions of these retailers are chosen by predicting the profit of each action, using a perceptron. Following on from the consideration of different economic models, a discrete model was developed so that software agents have a discrete environment to operate within. Within the model, it has been observed how supermarkets with differing behaviors affect a heterogeneous crowd of consumer agents. The model was implemented in Java with Python used to evaluate the results. 

The simulation displays good acceptance with real grocery market behavior, i.e. captures the performance of British retailers thus can be used to determine the impact of changes in their behavior on their competitors and consumers.Furthermore it can be used to provide insight into sustainability of volatile pricing strategies, providing a useful insight in volatility of British supermarket retail industry. 
\end{abstract}
\acknowledgements{
I would like to express my sincere gratitude to Dr Maria Polukarov for her guidance and support which provided me the freedom to take this research in the direction of my interest.\\
\\
I would also like to thank my family and friends for their encouragement and support. To those who quietly listened to my software complaints. To those who worked throughout the nights with me. To those who helped me write what I couldn't say. I cannot thank you enough.
}

\declaration{
I, Stefan Collier, declare that this dissertation and the work presented in it are my own and has been generated by me as the result of my own original research.\\
I confirm that:\\
1. This work was done wholly or mainly while in candidature for a degree at this University;\\
2. Where any part of this dissertation has previously been submitted for any other qualification at this University or any other institution, this has been clearly stated;\\
3. Where I have consulted the published work of others, this is always clearly attributed;\\
4. Where I have quoted from the work of others, the source is always given. With the exception of such quotations, this dissertation is entirely my own work;\\
5. I have acknowledged all main sources of help;\\
6. Where the thesis is based on work done by myself jointly with others, I have made clear exactly what was done by others and what I have contributed myself;\\
7. Either none of this work has been published before submission, or parts of this work have been published by :\\
\\
Stefan Collier\\
April 2016
}
\tableofcontents
\listoffigures
\listoftables

\mainmatter
%% ----------------------------------------------------------------
%\include{Introduction}
%\include{Conclusions}
\include{chapters/1Project/main}
\include{chapters/2Lit/main}
\include{chapters/3Design/HighLevel}
\include{chapters/3Design/InDepth}
\include{chapters/4Impl/main}

\include{chapters/5Experiments/1/main}
\include{chapters/5Experiments/2/main}
\include{chapters/5Experiments/3/main}
\include{chapters/5Experiments/4/main}

\include{chapters/6Conclusion/main}

\appendix
\include{appendix/AppendixB}
\include{appendix/D/main}
\include{appendix/AppendixC}

\backmatter
\bibliographystyle{ecs}
\bibliography{ECS}
\end{document}
%% ----------------------------------------------------------------

\include{chapters/3Design/HighLevel}
\include{chapters/3Design/InDepth}
 %% ----------------------------------------------------------------
%% Progress.tex
%% ---------------------------------------------------------------- 
\documentclass{ecsprogress}    % Use the progress Style
\graphicspath{{../figs/}}   % Location of your graphics files
    \usepackage{natbib}            % Use Natbib style for the refs.
\hypersetup{colorlinks=true}   % Set to false for black/white printing
\input{Definitions}            % Include your abbreviations



\usepackage{enumitem}% http://ctan.org/pkg/enumitem
\usepackage{multirow}
\usepackage{float}
\usepackage{amsmath}
\usepackage{multicol}
\usepackage{amssymb}
\usepackage[normalem]{ulem}
\useunder{\uline}{\ul}{}
\usepackage{wrapfig}


\usepackage[table,xcdraw]{xcolor}


%% ----------------------------------------------------------------
\begin{document}
\frontmatter
\title      {Heterogeneous Agent-based Model for Supermarket Competition}
\authors    {\texorpdfstring
             {\href{mailto:sc22g13@ecs.soton.ac.uk}{Stefan J. Collier}}
             {Stefan J. Collier}
            }
\addresses  {\groupname\\\deptname\\\univname}
\date       {\today}
\subject    {}
\keywords   {}
\supervisor {Dr. Maria Polukarov}
\examiner   {Professor Sheng Chen}

\maketitle
\begin{abstract}
This project aim was to model and analyse the effects of competitive pricing behaviors of grocery retailers on the British market. 

This was achieved by creating a multi-agent model, containing retailer and consumer agents. The heterogeneous crowd of retailers employs either a uniform pricing strategy or a ‘local price flexing’ strategy. The actions of these retailers are chosen by predicting the profit of each action, using a perceptron. Following on from the consideration of different economic models, a discrete model was developed so that software agents have a discrete environment to operate within. Within the model, it has been observed how supermarkets with differing behaviors affect a heterogeneous crowd of consumer agents. The model was implemented in Java with Python used to evaluate the results. 

The simulation displays good acceptance with real grocery market behavior, i.e. captures the performance of British retailers thus can be used to determine the impact of changes in their behavior on their competitors and consumers.Furthermore it can be used to provide insight into sustainability of volatile pricing strategies, providing a useful insight in volatility of British supermarket retail industry. 
\end{abstract}
\acknowledgements{
I would like to express my sincere gratitude to Dr Maria Polukarov for her guidance and support which provided me the freedom to take this research in the direction of my interest.\\
\\
I would also like to thank my family and friends for their encouragement and support. To those who quietly listened to my software complaints. To those who worked throughout the nights with me. To those who helped me write what I couldn't say. I cannot thank you enough.
}

\declaration{
I, Stefan Collier, declare that this dissertation and the work presented in it are my own and has been generated by me as the result of my own original research.\\
I confirm that:\\
1. This work was done wholly or mainly while in candidature for a degree at this University;\\
2. Where any part of this dissertation has previously been submitted for any other qualification at this University or any other institution, this has been clearly stated;\\
3. Where I have consulted the published work of others, this is always clearly attributed;\\
4. Where I have quoted from the work of others, the source is always given. With the exception of such quotations, this dissertation is entirely my own work;\\
5. I have acknowledged all main sources of help;\\
6. Where the thesis is based on work done by myself jointly with others, I have made clear exactly what was done by others and what I have contributed myself;\\
7. Either none of this work has been published before submission, or parts of this work have been published by :\\
\\
Stefan Collier\\
April 2016
}
\tableofcontents
\listoffigures
\listoftables

\mainmatter
%% ----------------------------------------------------------------
%\include{Introduction}
%\include{Conclusions}
\include{chapters/1Project/main}
\include{chapters/2Lit/main}
\include{chapters/3Design/HighLevel}
\include{chapters/3Design/InDepth}
\include{chapters/4Impl/main}

\include{chapters/5Experiments/1/main}
\include{chapters/5Experiments/2/main}
\include{chapters/5Experiments/3/main}
\include{chapters/5Experiments/4/main}

\include{chapters/6Conclusion/main}

\appendix
\include{appendix/AppendixB}
\include{appendix/D/main}
\include{appendix/AppendixC}

\backmatter
\bibliographystyle{ecs}
\bibliography{ECS}
\end{document}
%% ----------------------------------------------------------------


 %% ----------------------------------------------------------------
%% Progress.tex
%% ---------------------------------------------------------------- 
\documentclass{ecsprogress}    % Use the progress Style
\graphicspath{{../figs/}}   % Location of your graphics files
    \usepackage{natbib}            % Use Natbib style for the refs.
\hypersetup{colorlinks=true}   % Set to false for black/white printing
\input{Definitions}            % Include your abbreviations



\usepackage{enumitem}% http://ctan.org/pkg/enumitem
\usepackage{multirow}
\usepackage{float}
\usepackage{amsmath}
\usepackage{multicol}
\usepackage{amssymb}
\usepackage[normalem]{ulem}
\useunder{\uline}{\ul}{}
\usepackage{wrapfig}


\usepackage[table,xcdraw]{xcolor}


%% ----------------------------------------------------------------
\begin{document}
\frontmatter
\title      {Heterogeneous Agent-based Model for Supermarket Competition}
\authors    {\texorpdfstring
             {\href{mailto:sc22g13@ecs.soton.ac.uk}{Stefan J. Collier}}
             {Stefan J. Collier}
            }
\addresses  {\groupname\\\deptname\\\univname}
\date       {\today}
\subject    {}
\keywords   {}
\supervisor {Dr. Maria Polukarov}
\examiner   {Professor Sheng Chen}

\maketitle
\begin{abstract}
This project aim was to model and analyse the effects of competitive pricing behaviors of grocery retailers on the British market. 

This was achieved by creating a multi-agent model, containing retailer and consumer agents. The heterogeneous crowd of retailers employs either a uniform pricing strategy or a ‘local price flexing’ strategy. The actions of these retailers are chosen by predicting the profit of each action, using a perceptron. Following on from the consideration of different economic models, a discrete model was developed so that software agents have a discrete environment to operate within. Within the model, it has been observed how supermarkets with differing behaviors affect a heterogeneous crowd of consumer agents. The model was implemented in Java with Python used to evaluate the results. 

The simulation displays good acceptance with real grocery market behavior, i.e. captures the performance of British retailers thus can be used to determine the impact of changes in their behavior on their competitors and consumers.Furthermore it can be used to provide insight into sustainability of volatile pricing strategies, providing a useful insight in volatility of British supermarket retail industry. 
\end{abstract}
\acknowledgements{
I would like to express my sincere gratitude to Dr Maria Polukarov for her guidance and support which provided me the freedom to take this research in the direction of my interest.\\
\\
I would also like to thank my family and friends for their encouragement and support. To those who quietly listened to my software complaints. To those who worked throughout the nights with me. To those who helped me write what I couldn't say. I cannot thank you enough.
}

\declaration{
I, Stefan Collier, declare that this dissertation and the work presented in it are my own and has been generated by me as the result of my own original research.\\
I confirm that:\\
1. This work was done wholly or mainly while in candidature for a degree at this University;\\
2. Where any part of this dissertation has previously been submitted for any other qualification at this University or any other institution, this has been clearly stated;\\
3. Where I have consulted the published work of others, this is always clearly attributed;\\
4. Where I have quoted from the work of others, the source is always given. With the exception of such quotations, this dissertation is entirely my own work;\\
5. I have acknowledged all main sources of help;\\
6. Where the thesis is based on work done by myself jointly with others, I have made clear exactly what was done by others and what I have contributed myself;\\
7. Either none of this work has been published before submission, or parts of this work have been published by :\\
\\
Stefan Collier\\
April 2016
}
\tableofcontents
\listoffigures
\listoftables

\mainmatter
%% ----------------------------------------------------------------
%\include{Introduction}
%\include{Conclusions}
\include{chapters/1Project/main}
\include{chapters/2Lit/main}
\include{chapters/3Design/HighLevel}
\include{chapters/3Design/InDepth}
\include{chapters/4Impl/main}

\include{chapters/5Experiments/1/main}
\include{chapters/5Experiments/2/main}
\include{chapters/5Experiments/3/main}
\include{chapters/5Experiments/4/main}

\include{chapters/6Conclusion/main}

\appendix
\include{appendix/AppendixB}
\include{appendix/D/main}
\include{appendix/AppendixC}

\backmatter
\bibliographystyle{ecs}
\bibliography{ECS}
\end{document}
%% ----------------------------------------------------------------

 %% ----------------------------------------------------------------
%% Progress.tex
%% ---------------------------------------------------------------- 
\documentclass{ecsprogress}    % Use the progress Style
\graphicspath{{../figs/}}   % Location of your graphics files
    \usepackage{natbib}            % Use Natbib style for the refs.
\hypersetup{colorlinks=true}   % Set to false for black/white printing
\input{Definitions}            % Include your abbreviations



\usepackage{enumitem}% http://ctan.org/pkg/enumitem
\usepackage{multirow}
\usepackage{float}
\usepackage{amsmath}
\usepackage{multicol}
\usepackage{amssymb}
\usepackage[normalem]{ulem}
\useunder{\uline}{\ul}{}
\usepackage{wrapfig}


\usepackage[table,xcdraw]{xcolor}


%% ----------------------------------------------------------------
\begin{document}
\frontmatter
\title      {Heterogeneous Agent-based Model for Supermarket Competition}
\authors    {\texorpdfstring
             {\href{mailto:sc22g13@ecs.soton.ac.uk}{Stefan J. Collier}}
             {Stefan J. Collier}
            }
\addresses  {\groupname\\\deptname\\\univname}
\date       {\today}
\subject    {}
\keywords   {}
\supervisor {Dr. Maria Polukarov}
\examiner   {Professor Sheng Chen}

\maketitle
\begin{abstract}
This project aim was to model and analyse the effects of competitive pricing behaviors of grocery retailers on the British market. 

This was achieved by creating a multi-agent model, containing retailer and consumer agents. The heterogeneous crowd of retailers employs either a uniform pricing strategy or a ‘local price flexing’ strategy. The actions of these retailers are chosen by predicting the profit of each action, using a perceptron. Following on from the consideration of different economic models, a discrete model was developed so that software agents have a discrete environment to operate within. Within the model, it has been observed how supermarkets with differing behaviors affect a heterogeneous crowd of consumer agents. The model was implemented in Java with Python used to evaluate the results. 

The simulation displays good acceptance with real grocery market behavior, i.e. captures the performance of British retailers thus can be used to determine the impact of changes in their behavior on their competitors and consumers.Furthermore it can be used to provide insight into sustainability of volatile pricing strategies, providing a useful insight in volatility of British supermarket retail industry. 
\end{abstract}
\acknowledgements{
I would like to express my sincere gratitude to Dr Maria Polukarov for her guidance and support which provided me the freedom to take this research in the direction of my interest.\\
\\
I would also like to thank my family and friends for their encouragement and support. To those who quietly listened to my software complaints. To those who worked throughout the nights with me. To those who helped me write what I couldn't say. I cannot thank you enough.
}

\declaration{
I, Stefan Collier, declare that this dissertation and the work presented in it are my own and has been generated by me as the result of my own original research.\\
I confirm that:\\
1. This work was done wholly or mainly while in candidature for a degree at this University;\\
2. Where any part of this dissertation has previously been submitted for any other qualification at this University or any other institution, this has been clearly stated;\\
3. Where I have consulted the published work of others, this is always clearly attributed;\\
4. Where I have quoted from the work of others, the source is always given. With the exception of such quotations, this dissertation is entirely my own work;\\
5. I have acknowledged all main sources of help;\\
6. Where the thesis is based on work done by myself jointly with others, I have made clear exactly what was done by others and what I have contributed myself;\\
7. Either none of this work has been published before submission, or parts of this work have been published by :\\
\\
Stefan Collier\\
April 2016
}
\tableofcontents
\listoffigures
\listoftables

\mainmatter
%% ----------------------------------------------------------------
%\include{Introduction}
%\include{Conclusions}
\include{chapters/1Project/main}
\include{chapters/2Lit/main}
\include{chapters/3Design/HighLevel}
\include{chapters/3Design/InDepth}
\include{chapters/4Impl/main}

\include{chapters/5Experiments/1/main}
\include{chapters/5Experiments/2/main}
\include{chapters/5Experiments/3/main}
\include{chapters/5Experiments/4/main}

\include{chapters/6Conclusion/main}

\appendix
\include{appendix/AppendixB}
\include{appendix/D/main}
\include{appendix/AppendixC}

\backmatter
\bibliographystyle{ecs}
\bibliography{ECS}
\end{document}
%% ----------------------------------------------------------------

 %% ----------------------------------------------------------------
%% Progress.tex
%% ---------------------------------------------------------------- 
\documentclass{ecsprogress}    % Use the progress Style
\graphicspath{{../figs/}}   % Location of your graphics files
    \usepackage{natbib}            % Use Natbib style for the refs.
\hypersetup{colorlinks=true}   % Set to false for black/white printing
\input{Definitions}            % Include your abbreviations



\usepackage{enumitem}% http://ctan.org/pkg/enumitem
\usepackage{multirow}
\usepackage{float}
\usepackage{amsmath}
\usepackage{multicol}
\usepackage{amssymb}
\usepackage[normalem]{ulem}
\useunder{\uline}{\ul}{}
\usepackage{wrapfig}


\usepackage[table,xcdraw]{xcolor}


%% ----------------------------------------------------------------
\begin{document}
\frontmatter
\title      {Heterogeneous Agent-based Model for Supermarket Competition}
\authors    {\texorpdfstring
             {\href{mailto:sc22g13@ecs.soton.ac.uk}{Stefan J. Collier}}
             {Stefan J. Collier}
            }
\addresses  {\groupname\\\deptname\\\univname}
\date       {\today}
\subject    {}
\keywords   {}
\supervisor {Dr. Maria Polukarov}
\examiner   {Professor Sheng Chen}

\maketitle
\begin{abstract}
This project aim was to model and analyse the effects of competitive pricing behaviors of grocery retailers on the British market. 

This was achieved by creating a multi-agent model, containing retailer and consumer agents. The heterogeneous crowd of retailers employs either a uniform pricing strategy or a ‘local price flexing’ strategy. The actions of these retailers are chosen by predicting the profit of each action, using a perceptron. Following on from the consideration of different economic models, a discrete model was developed so that software agents have a discrete environment to operate within. Within the model, it has been observed how supermarkets with differing behaviors affect a heterogeneous crowd of consumer agents. The model was implemented in Java with Python used to evaluate the results. 

The simulation displays good acceptance with real grocery market behavior, i.e. captures the performance of British retailers thus can be used to determine the impact of changes in their behavior on their competitors and consumers.Furthermore it can be used to provide insight into sustainability of volatile pricing strategies, providing a useful insight in volatility of British supermarket retail industry. 
\end{abstract}
\acknowledgements{
I would like to express my sincere gratitude to Dr Maria Polukarov for her guidance and support which provided me the freedom to take this research in the direction of my interest.\\
\\
I would also like to thank my family and friends for their encouragement and support. To those who quietly listened to my software complaints. To those who worked throughout the nights with me. To those who helped me write what I couldn't say. I cannot thank you enough.
}

\declaration{
I, Stefan Collier, declare that this dissertation and the work presented in it are my own and has been generated by me as the result of my own original research.\\
I confirm that:\\
1. This work was done wholly or mainly while in candidature for a degree at this University;\\
2. Where any part of this dissertation has previously been submitted for any other qualification at this University or any other institution, this has been clearly stated;\\
3. Where I have consulted the published work of others, this is always clearly attributed;\\
4. Where I have quoted from the work of others, the source is always given. With the exception of such quotations, this dissertation is entirely my own work;\\
5. I have acknowledged all main sources of help;\\
6. Where the thesis is based on work done by myself jointly with others, I have made clear exactly what was done by others and what I have contributed myself;\\
7. Either none of this work has been published before submission, or parts of this work have been published by :\\
\\
Stefan Collier\\
April 2016
}
\tableofcontents
\listoffigures
\listoftables

\mainmatter
%% ----------------------------------------------------------------
%\include{Introduction}
%\include{Conclusions}
\include{chapters/1Project/main}
\include{chapters/2Lit/main}
\include{chapters/3Design/HighLevel}
\include{chapters/3Design/InDepth}
\include{chapters/4Impl/main}

\include{chapters/5Experiments/1/main}
\include{chapters/5Experiments/2/main}
\include{chapters/5Experiments/3/main}
\include{chapters/5Experiments/4/main}

\include{chapters/6Conclusion/main}

\appendix
\include{appendix/AppendixB}
\include{appendix/D/main}
\include{appendix/AppendixC}

\backmatter
\bibliographystyle{ecs}
\bibliography{ECS}
\end{document}
%% ----------------------------------------------------------------

 %% ----------------------------------------------------------------
%% Progress.tex
%% ---------------------------------------------------------------- 
\documentclass{ecsprogress}    % Use the progress Style
\graphicspath{{../figs/}}   % Location of your graphics files
    \usepackage{natbib}            % Use Natbib style for the refs.
\hypersetup{colorlinks=true}   % Set to false for black/white printing
\input{Definitions}            % Include your abbreviations



\usepackage{enumitem}% http://ctan.org/pkg/enumitem
\usepackage{multirow}
\usepackage{float}
\usepackage{amsmath}
\usepackage{multicol}
\usepackage{amssymb}
\usepackage[normalem]{ulem}
\useunder{\uline}{\ul}{}
\usepackage{wrapfig}


\usepackage[table,xcdraw]{xcolor}


%% ----------------------------------------------------------------
\begin{document}
\frontmatter
\title      {Heterogeneous Agent-based Model for Supermarket Competition}
\authors    {\texorpdfstring
             {\href{mailto:sc22g13@ecs.soton.ac.uk}{Stefan J. Collier}}
             {Stefan J. Collier}
            }
\addresses  {\groupname\\\deptname\\\univname}
\date       {\today}
\subject    {}
\keywords   {}
\supervisor {Dr. Maria Polukarov}
\examiner   {Professor Sheng Chen}

\maketitle
\begin{abstract}
This project aim was to model and analyse the effects of competitive pricing behaviors of grocery retailers on the British market. 

This was achieved by creating a multi-agent model, containing retailer and consumer agents. The heterogeneous crowd of retailers employs either a uniform pricing strategy or a ‘local price flexing’ strategy. The actions of these retailers are chosen by predicting the profit of each action, using a perceptron. Following on from the consideration of different economic models, a discrete model was developed so that software agents have a discrete environment to operate within. Within the model, it has been observed how supermarkets with differing behaviors affect a heterogeneous crowd of consumer agents. The model was implemented in Java with Python used to evaluate the results. 

The simulation displays good acceptance with real grocery market behavior, i.e. captures the performance of British retailers thus can be used to determine the impact of changes in their behavior on their competitors and consumers.Furthermore it can be used to provide insight into sustainability of volatile pricing strategies, providing a useful insight in volatility of British supermarket retail industry. 
\end{abstract}
\acknowledgements{
I would like to express my sincere gratitude to Dr Maria Polukarov for her guidance and support which provided me the freedom to take this research in the direction of my interest.\\
\\
I would also like to thank my family and friends for their encouragement and support. To those who quietly listened to my software complaints. To those who worked throughout the nights with me. To those who helped me write what I couldn't say. I cannot thank you enough.
}

\declaration{
I, Stefan Collier, declare that this dissertation and the work presented in it are my own and has been generated by me as the result of my own original research.\\
I confirm that:\\
1. This work was done wholly or mainly while in candidature for a degree at this University;\\
2. Where any part of this dissertation has previously been submitted for any other qualification at this University or any other institution, this has been clearly stated;\\
3. Where I have consulted the published work of others, this is always clearly attributed;\\
4. Where I have quoted from the work of others, the source is always given. With the exception of such quotations, this dissertation is entirely my own work;\\
5. I have acknowledged all main sources of help;\\
6. Where the thesis is based on work done by myself jointly with others, I have made clear exactly what was done by others and what I have contributed myself;\\
7. Either none of this work has been published before submission, or parts of this work have been published by :\\
\\
Stefan Collier\\
April 2016
}
\tableofcontents
\listoffigures
\listoftables

\mainmatter
%% ----------------------------------------------------------------
%\include{Introduction}
%\include{Conclusions}
\include{chapters/1Project/main}
\include{chapters/2Lit/main}
\include{chapters/3Design/HighLevel}
\include{chapters/3Design/InDepth}
\include{chapters/4Impl/main}

\include{chapters/5Experiments/1/main}
\include{chapters/5Experiments/2/main}
\include{chapters/5Experiments/3/main}
\include{chapters/5Experiments/4/main}

\include{chapters/6Conclusion/main}

\appendix
\include{appendix/AppendixB}
\include{appendix/D/main}
\include{appendix/AppendixC}

\backmatter
\bibliographystyle{ecs}
\bibliography{ECS}
\end{document}
%% ----------------------------------------------------------------


 %% ----------------------------------------------------------------
%% Progress.tex
%% ---------------------------------------------------------------- 
\documentclass{ecsprogress}    % Use the progress Style
\graphicspath{{../figs/}}   % Location of your graphics files
    \usepackage{natbib}            % Use Natbib style for the refs.
\hypersetup{colorlinks=true}   % Set to false for black/white printing
\input{Definitions}            % Include your abbreviations



\usepackage{enumitem}% http://ctan.org/pkg/enumitem
\usepackage{multirow}
\usepackage{float}
\usepackage{amsmath}
\usepackage{multicol}
\usepackage{amssymb}
\usepackage[normalem]{ulem}
\useunder{\uline}{\ul}{}
\usepackage{wrapfig}


\usepackage[table,xcdraw]{xcolor}


%% ----------------------------------------------------------------
\begin{document}
\frontmatter
\title      {Heterogeneous Agent-based Model for Supermarket Competition}
\authors    {\texorpdfstring
             {\href{mailto:sc22g13@ecs.soton.ac.uk}{Stefan J. Collier}}
             {Stefan J. Collier}
            }
\addresses  {\groupname\\\deptname\\\univname}
\date       {\today}
\subject    {}
\keywords   {}
\supervisor {Dr. Maria Polukarov}
\examiner   {Professor Sheng Chen}

\maketitle
\begin{abstract}
This project aim was to model and analyse the effects of competitive pricing behaviors of grocery retailers on the British market. 

This was achieved by creating a multi-agent model, containing retailer and consumer agents. The heterogeneous crowd of retailers employs either a uniform pricing strategy or a ‘local price flexing’ strategy. The actions of these retailers are chosen by predicting the profit of each action, using a perceptron. Following on from the consideration of different economic models, a discrete model was developed so that software agents have a discrete environment to operate within. Within the model, it has been observed how supermarkets with differing behaviors affect a heterogeneous crowd of consumer agents. The model was implemented in Java with Python used to evaluate the results. 

The simulation displays good acceptance with real grocery market behavior, i.e. captures the performance of British retailers thus can be used to determine the impact of changes in their behavior on their competitors and consumers.Furthermore it can be used to provide insight into sustainability of volatile pricing strategies, providing a useful insight in volatility of British supermarket retail industry. 
\end{abstract}
\acknowledgements{
I would like to express my sincere gratitude to Dr Maria Polukarov for her guidance and support which provided me the freedom to take this research in the direction of my interest.\\
\\
I would also like to thank my family and friends for their encouragement and support. To those who quietly listened to my software complaints. To those who worked throughout the nights with me. To those who helped me write what I couldn't say. I cannot thank you enough.
}

\declaration{
I, Stefan Collier, declare that this dissertation and the work presented in it are my own and has been generated by me as the result of my own original research.\\
I confirm that:\\
1. This work was done wholly or mainly while in candidature for a degree at this University;\\
2. Where any part of this dissertation has previously been submitted for any other qualification at this University or any other institution, this has been clearly stated;\\
3. Where I have consulted the published work of others, this is always clearly attributed;\\
4. Where I have quoted from the work of others, the source is always given. With the exception of such quotations, this dissertation is entirely my own work;\\
5. I have acknowledged all main sources of help;\\
6. Where the thesis is based on work done by myself jointly with others, I have made clear exactly what was done by others and what I have contributed myself;\\
7. Either none of this work has been published before submission, or parts of this work have been published by :\\
\\
Stefan Collier\\
April 2016
}
\tableofcontents
\listoffigures
\listoftables

\mainmatter
%% ----------------------------------------------------------------
%\include{Introduction}
%\include{Conclusions}
\include{chapters/1Project/main}
\include{chapters/2Lit/main}
\include{chapters/3Design/HighLevel}
\include{chapters/3Design/InDepth}
\include{chapters/4Impl/main}

\include{chapters/5Experiments/1/main}
\include{chapters/5Experiments/2/main}
\include{chapters/5Experiments/3/main}
\include{chapters/5Experiments/4/main}

\include{chapters/6Conclusion/main}

\appendix
\include{appendix/AppendixB}
\include{appendix/D/main}
\include{appendix/AppendixC}

\backmatter
\bibliographystyle{ecs}
\bibliography{ECS}
\end{document}
%% ----------------------------------------------------------------


\appendix
\include{appendix/AppendixB}
 %% ----------------------------------------------------------------
%% Progress.tex
%% ---------------------------------------------------------------- 
\documentclass{ecsprogress}    % Use the progress Style
\graphicspath{{../figs/}}   % Location of your graphics files
    \usepackage{natbib}            % Use Natbib style for the refs.
\hypersetup{colorlinks=true}   % Set to false for black/white printing
\input{Definitions}            % Include your abbreviations



\usepackage{enumitem}% http://ctan.org/pkg/enumitem
\usepackage{multirow}
\usepackage{float}
\usepackage{amsmath}
\usepackage{multicol}
\usepackage{amssymb}
\usepackage[normalem]{ulem}
\useunder{\uline}{\ul}{}
\usepackage{wrapfig}


\usepackage[table,xcdraw]{xcolor}


%% ----------------------------------------------------------------
\begin{document}
\frontmatter
\title      {Heterogeneous Agent-based Model for Supermarket Competition}
\authors    {\texorpdfstring
             {\href{mailto:sc22g13@ecs.soton.ac.uk}{Stefan J. Collier}}
             {Stefan J. Collier}
            }
\addresses  {\groupname\\\deptname\\\univname}
\date       {\today}
\subject    {}
\keywords   {}
\supervisor {Dr. Maria Polukarov}
\examiner   {Professor Sheng Chen}

\maketitle
\begin{abstract}
This project aim was to model and analyse the effects of competitive pricing behaviors of grocery retailers on the British market. 

This was achieved by creating a multi-agent model, containing retailer and consumer agents. The heterogeneous crowd of retailers employs either a uniform pricing strategy or a ‘local price flexing’ strategy. The actions of these retailers are chosen by predicting the profit of each action, using a perceptron. Following on from the consideration of different economic models, a discrete model was developed so that software agents have a discrete environment to operate within. Within the model, it has been observed how supermarkets with differing behaviors affect a heterogeneous crowd of consumer agents. The model was implemented in Java with Python used to evaluate the results. 

The simulation displays good acceptance with real grocery market behavior, i.e. captures the performance of British retailers thus can be used to determine the impact of changes in their behavior on their competitors and consumers.Furthermore it can be used to provide insight into sustainability of volatile pricing strategies, providing a useful insight in volatility of British supermarket retail industry. 
\end{abstract}
\acknowledgements{
I would like to express my sincere gratitude to Dr Maria Polukarov for her guidance and support which provided me the freedom to take this research in the direction of my interest.\\
\\
I would also like to thank my family and friends for their encouragement and support. To those who quietly listened to my software complaints. To those who worked throughout the nights with me. To those who helped me write what I couldn't say. I cannot thank you enough.
}

\declaration{
I, Stefan Collier, declare that this dissertation and the work presented in it are my own and has been generated by me as the result of my own original research.\\
I confirm that:\\
1. This work was done wholly or mainly while in candidature for a degree at this University;\\
2. Where any part of this dissertation has previously been submitted for any other qualification at this University or any other institution, this has been clearly stated;\\
3. Where I have consulted the published work of others, this is always clearly attributed;\\
4. Where I have quoted from the work of others, the source is always given. With the exception of such quotations, this dissertation is entirely my own work;\\
5. I have acknowledged all main sources of help;\\
6. Where the thesis is based on work done by myself jointly with others, I have made clear exactly what was done by others and what I have contributed myself;\\
7. Either none of this work has been published before submission, or parts of this work have been published by :\\
\\
Stefan Collier\\
April 2016
}
\tableofcontents
\listoffigures
\listoftables

\mainmatter
%% ----------------------------------------------------------------
%\include{Introduction}
%\include{Conclusions}
\include{chapters/1Project/main}
\include{chapters/2Lit/main}
\include{chapters/3Design/HighLevel}
\include{chapters/3Design/InDepth}
\include{chapters/4Impl/main}

\include{chapters/5Experiments/1/main}
\include{chapters/5Experiments/2/main}
\include{chapters/5Experiments/3/main}
\include{chapters/5Experiments/4/main}

\include{chapters/6Conclusion/main}

\appendix
\include{appendix/AppendixB}
\include{appendix/D/main}
\include{appendix/AppendixC}

\backmatter
\bibliographystyle{ecs}
\bibliography{ECS}
\end{document}
%% ----------------------------------------------------------------

\include{appendix/AppendixC}

\backmatter
\bibliographystyle{ecs}
\bibliography{ECS}
\end{document}
%% ----------------------------------------------------------------


\appendix
\include{appendix/AppendixB}
 %% ----------------------------------------------------------------
%% Progress.tex
%% ---------------------------------------------------------------- 
\documentclass{ecsprogress}    % Use the progress Style
\graphicspath{{../figs/}}   % Location of your graphics files
    \usepackage{natbib}            % Use Natbib style for the refs.
\hypersetup{colorlinks=true}   % Set to false for black/white printing
\input{Definitions}            % Include your abbreviations



\usepackage{enumitem}% http://ctan.org/pkg/enumitem
\usepackage{multirow}
\usepackage{float}
\usepackage{amsmath}
\usepackage{multicol}
\usepackage{amssymb}
\usepackage[normalem]{ulem}
\useunder{\uline}{\ul}{}
\usepackage{wrapfig}


\usepackage[table,xcdraw]{xcolor}


%% ----------------------------------------------------------------
\begin{document}
\frontmatter
\title      {Heterogeneous Agent-based Model for Supermarket Competition}
\authors    {\texorpdfstring
             {\href{mailto:sc22g13@ecs.soton.ac.uk}{Stefan J. Collier}}
             {Stefan J. Collier}
            }
\addresses  {\groupname\\\deptname\\\univname}
\date       {\today}
\subject    {}
\keywords   {}
\supervisor {Dr. Maria Polukarov}
\examiner   {Professor Sheng Chen}

\maketitle
\begin{abstract}
This project aim was to model and analyse the effects of competitive pricing behaviors of grocery retailers on the British market. 

This was achieved by creating a multi-agent model, containing retailer and consumer agents. The heterogeneous crowd of retailers employs either a uniform pricing strategy or a ‘local price flexing’ strategy. The actions of these retailers are chosen by predicting the profit of each action, using a perceptron. Following on from the consideration of different economic models, a discrete model was developed so that software agents have a discrete environment to operate within. Within the model, it has been observed how supermarkets with differing behaviors affect a heterogeneous crowd of consumer agents. The model was implemented in Java with Python used to evaluate the results. 

The simulation displays good acceptance with real grocery market behavior, i.e. captures the performance of British retailers thus can be used to determine the impact of changes in their behavior on their competitors and consumers.Furthermore it can be used to provide insight into sustainability of volatile pricing strategies, providing a useful insight in volatility of British supermarket retail industry. 
\end{abstract}
\acknowledgements{
I would like to express my sincere gratitude to Dr Maria Polukarov for her guidance and support which provided me the freedom to take this research in the direction of my interest.\\
\\
I would also like to thank my family and friends for their encouragement and support. To those who quietly listened to my software complaints. To those who worked throughout the nights with me. To those who helped me write what I couldn't say. I cannot thank you enough.
}

\declaration{
I, Stefan Collier, declare that this dissertation and the work presented in it are my own and has been generated by me as the result of my own original research.\\
I confirm that:\\
1. This work was done wholly or mainly while in candidature for a degree at this University;\\
2. Where any part of this dissertation has previously been submitted for any other qualification at this University or any other institution, this has been clearly stated;\\
3. Where I have consulted the published work of others, this is always clearly attributed;\\
4. Where I have quoted from the work of others, the source is always given. With the exception of such quotations, this dissertation is entirely my own work;\\
5. I have acknowledged all main sources of help;\\
6. Where the thesis is based on work done by myself jointly with others, I have made clear exactly what was done by others and what I have contributed myself;\\
7. Either none of this work has been published before submission, or parts of this work have been published by :\\
\\
Stefan Collier\\
April 2016
}
\tableofcontents
\listoffigures
\listoftables

\mainmatter
%% ----------------------------------------------------------------
%\include{Introduction}
%\include{Conclusions}
 %% ----------------------------------------------------------------
%% Progress.tex
%% ---------------------------------------------------------------- 
\documentclass{ecsprogress}    % Use the progress Style
\graphicspath{{../figs/}}   % Location of your graphics files
    \usepackage{natbib}            % Use Natbib style for the refs.
\hypersetup{colorlinks=true}   % Set to false for black/white printing
\input{Definitions}            % Include your abbreviations



\usepackage{enumitem}% http://ctan.org/pkg/enumitem
\usepackage{multirow}
\usepackage{float}
\usepackage{amsmath}
\usepackage{multicol}
\usepackage{amssymb}
\usepackage[normalem]{ulem}
\useunder{\uline}{\ul}{}
\usepackage{wrapfig}


\usepackage[table,xcdraw]{xcolor}


%% ----------------------------------------------------------------
\begin{document}
\frontmatter
\title      {Heterogeneous Agent-based Model for Supermarket Competition}
\authors    {\texorpdfstring
             {\href{mailto:sc22g13@ecs.soton.ac.uk}{Stefan J. Collier}}
             {Stefan J. Collier}
            }
\addresses  {\groupname\\\deptname\\\univname}
\date       {\today}
\subject    {}
\keywords   {}
\supervisor {Dr. Maria Polukarov}
\examiner   {Professor Sheng Chen}

\maketitle
\begin{abstract}
This project aim was to model and analyse the effects of competitive pricing behaviors of grocery retailers on the British market. 

This was achieved by creating a multi-agent model, containing retailer and consumer agents. The heterogeneous crowd of retailers employs either a uniform pricing strategy or a ‘local price flexing’ strategy. The actions of these retailers are chosen by predicting the profit of each action, using a perceptron. Following on from the consideration of different economic models, a discrete model was developed so that software agents have a discrete environment to operate within. Within the model, it has been observed how supermarkets with differing behaviors affect a heterogeneous crowd of consumer agents. The model was implemented in Java with Python used to evaluate the results. 

The simulation displays good acceptance with real grocery market behavior, i.e. captures the performance of British retailers thus can be used to determine the impact of changes in their behavior on their competitors and consumers.Furthermore it can be used to provide insight into sustainability of volatile pricing strategies, providing a useful insight in volatility of British supermarket retail industry. 
\end{abstract}
\acknowledgements{
I would like to express my sincere gratitude to Dr Maria Polukarov for her guidance and support which provided me the freedom to take this research in the direction of my interest.\\
\\
I would also like to thank my family and friends for their encouragement and support. To those who quietly listened to my software complaints. To those who worked throughout the nights with me. To those who helped me write what I couldn't say. I cannot thank you enough.
}

\declaration{
I, Stefan Collier, declare that this dissertation and the work presented in it are my own and has been generated by me as the result of my own original research.\\
I confirm that:\\
1. This work was done wholly or mainly while in candidature for a degree at this University;\\
2. Where any part of this dissertation has previously been submitted for any other qualification at this University or any other institution, this has been clearly stated;\\
3. Where I have consulted the published work of others, this is always clearly attributed;\\
4. Where I have quoted from the work of others, the source is always given. With the exception of such quotations, this dissertation is entirely my own work;\\
5. I have acknowledged all main sources of help;\\
6. Where the thesis is based on work done by myself jointly with others, I have made clear exactly what was done by others and what I have contributed myself;\\
7. Either none of this work has been published before submission, or parts of this work have been published by :\\
\\
Stefan Collier\\
April 2016
}
\tableofcontents
\listoffigures
\listoftables

\mainmatter
%% ----------------------------------------------------------------
%\include{Introduction}
%\include{Conclusions}
\include{chapters/1Project/main}
\include{chapters/2Lit/main}
\include{chapters/3Design/HighLevel}
\include{chapters/3Design/InDepth}
\include{chapters/4Impl/main}

\include{chapters/5Experiments/1/main}
\include{chapters/5Experiments/2/main}
\include{chapters/5Experiments/3/main}
\include{chapters/5Experiments/4/main}

\include{chapters/6Conclusion/main}

\appendix
\include{appendix/AppendixB}
\include{appendix/D/main}
\include{appendix/AppendixC}

\backmatter
\bibliographystyle{ecs}
\bibliography{ECS}
\end{document}
%% ----------------------------------------------------------------

 %% ----------------------------------------------------------------
%% Progress.tex
%% ---------------------------------------------------------------- 
\documentclass{ecsprogress}    % Use the progress Style
\graphicspath{{../figs/}}   % Location of your graphics files
    \usepackage{natbib}            % Use Natbib style for the refs.
\hypersetup{colorlinks=true}   % Set to false for black/white printing
\input{Definitions}            % Include your abbreviations



\usepackage{enumitem}% http://ctan.org/pkg/enumitem
\usepackage{multirow}
\usepackage{float}
\usepackage{amsmath}
\usepackage{multicol}
\usepackage{amssymb}
\usepackage[normalem]{ulem}
\useunder{\uline}{\ul}{}
\usepackage{wrapfig}


\usepackage[table,xcdraw]{xcolor}


%% ----------------------------------------------------------------
\begin{document}
\frontmatter
\title      {Heterogeneous Agent-based Model for Supermarket Competition}
\authors    {\texorpdfstring
             {\href{mailto:sc22g13@ecs.soton.ac.uk}{Stefan J. Collier}}
             {Stefan J. Collier}
            }
\addresses  {\groupname\\\deptname\\\univname}
\date       {\today}
\subject    {}
\keywords   {}
\supervisor {Dr. Maria Polukarov}
\examiner   {Professor Sheng Chen}

\maketitle
\begin{abstract}
This project aim was to model and analyse the effects of competitive pricing behaviors of grocery retailers on the British market. 

This was achieved by creating a multi-agent model, containing retailer and consumer agents. The heterogeneous crowd of retailers employs either a uniform pricing strategy or a ‘local price flexing’ strategy. The actions of these retailers are chosen by predicting the profit of each action, using a perceptron. Following on from the consideration of different economic models, a discrete model was developed so that software agents have a discrete environment to operate within. Within the model, it has been observed how supermarkets with differing behaviors affect a heterogeneous crowd of consumer agents. The model was implemented in Java with Python used to evaluate the results. 

The simulation displays good acceptance with real grocery market behavior, i.e. captures the performance of British retailers thus can be used to determine the impact of changes in their behavior on their competitors and consumers.Furthermore it can be used to provide insight into sustainability of volatile pricing strategies, providing a useful insight in volatility of British supermarket retail industry. 
\end{abstract}
\acknowledgements{
I would like to express my sincere gratitude to Dr Maria Polukarov for her guidance and support which provided me the freedom to take this research in the direction of my interest.\\
\\
I would also like to thank my family and friends for their encouragement and support. To those who quietly listened to my software complaints. To those who worked throughout the nights with me. To those who helped me write what I couldn't say. I cannot thank you enough.
}

\declaration{
I, Stefan Collier, declare that this dissertation and the work presented in it are my own and has been generated by me as the result of my own original research.\\
I confirm that:\\
1. This work was done wholly or mainly while in candidature for a degree at this University;\\
2. Where any part of this dissertation has previously been submitted for any other qualification at this University or any other institution, this has been clearly stated;\\
3. Where I have consulted the published work of others, this is always clearly attributed;\\
4. Where I have quoted from the work of others, the source is always given. With the exception of such quotations, this dissertation is entirely my own work;\\
5. I have acknowledged all main sources of help;\\
6. Where the thesis is based on work done by myself jointly with others, I have made clear exactly what was done by others and what I have contributed myself;\\
7. Either none of this work has been published before submission, or parts of this work have been published by :\\
\\
Stefan Collier\\
April 2016
}
\tableofcontents
\listoffigures
\listoftables

\mainmatter
%% ----------------------------------------------------------------
%\include{Introduction}
%\include{Conclusions}
\include{chapters/1Project/main}
\include{chapters/2Lit/main}
\include{chapters/3Design/HighLevel}
\include{chapters/3Design/InDepth}
\include{chapters/4Impl/main}

\include{chapters/5Experiments/1/main}
\include{chapters/5Experiments/2/main}
\include{chapters/5Experiments/3/main}
\include{chapters/5Experiments/4/main}

\include{chapters/6Conclusion/main}

\appendix
\include{appendix/AppendixB}
\include{appendix/D/main}
\include{appendix/AppendixC}

\backmatter
\bibliographystyle{ecs}
\bibliography{ECS}
\end{document}
%% ----------------------------------------------------------------

\include{chapters/3Design/HighLevel}
\include{chapters/3Design/InDepth}
 %% ----------------------------------------------------------------
%% Progress.tex
%% ---------------------------------------------------------------- 
\documentclass{ecsprogress}    % Use the progress Style
\graphicspath{{../figs/}}   % Location of your graphics files
    \usepackage{natbib}            % Use Natbib style for the refs.
\hypersetup{colorlinks=true}   % Set to false for black/white printing
\input{Definitions}            % Include your abbreviations



\usepackage{enumitem}% http://ctan.org/pkg/enumitem
\usepackage{multirow}
\usepackage{float}
\usepackage{amsmath}
\usepackage{multicol}
\usepackage{amssymb}
\usepackage[normalem]{ulem}
\useunder{\uline}{\ul}{}
\usepackage{wrapfig}


\usepackage[table,xcdraw]{xcolor}


%% ----------------------------------------------------------------
\begin{document}
\frontmatter
\title      {Heterogeneous Agent-based Model for Supermarket Competition}
\authors    {\texorpdfstring
             {\href{mailto:sc22g13@ecs.soton.ac.uk}{Stefan J. Collier}}
             {Stefan J. Collier}
            }
\addresses  {\groupname\\\deptname\\\univname}
\date       {\today}
\subject    {}
\keywords   {}
\supervisor {Dr. Maria Polukarov}
\examiner   {Professor Sheng Chen}

\maketitle
\begin{abstract}
This project aim was to model and analyse the effects of competitive pricing behaviors of grocery retailers on the British market. 

This was achieved by creating a multi-agent model, containing retailer and consumer agents. The heterogeneous crowd of retailers employs either a uniform pricing strategy or a ‘local price flexing’ strategy. The actions of these retailers are chosen by predicting the profit of each action, using a perceptron. Following on from the consideration of different economic models, a discrete model was developed so that software agents have a discrete environment to operate within. Within the model, it has been observed how supermarkets with differing behaviors affect a heterogeneous crowd of consumer agents. The model was implemented in Java with Python used to evaluate the results. 

The simulation displays good acceptance with real grocery market behavior, i.e. captures the performance of British retailers thus can be used to determine the impact of changes in their behavior on their competitors and consumers.Furthermore it can be used to provide insight into sustainability of volatile pricing strategies, providing a useful insight in volatility of British supermarket retail industry. 
\end{abstract}
\acknowledgements{
I would like to express my sincere gratitude to Dr Maria Polukarov for her guidance and support which provided me the freedom to take this research in the direction of my interest.\\
\\
I would also like to thank my family and friends for their encouragement and support. To those who quietly listened to my software complaints. To those who worked throughout the nights with me. To those who helped me write what I couldn't say. I cannot thank you enough.
}

\declaration{
I, Stefan Collier, declare that this dissertation and the work presented in it are my own and has been generated by me as the result of my own original research.\\
I confirm that:\\
1. This work was done wholly or mainly while in candidature for a degree at this University;\\
2. Where any part of this dissertation has previously been submitted for any other qualification at this University or any other institution, this has been clearly stated;\\
3. Where I have consulted the published work of others, this is always clearly attributed;\\
4. Where I have quoted from the work of others, the source is always given. With the exception of such quotations, this dissertation is entirely my own work;\\
5. I have acknowledged all main sources of help;\\
6. Where the thesis is based on work done by myself jointly with others, I have made clear exactly what was done by others and what I have contributed myself;\\
7. Either none of this work has been published before submission, or parts of this work have been published by :\\
\\
Stefan Collier\\
April 2016
}
\tableofcontents
\listoffigures
\listoftables

\mainmatter
%% ----------------------------------------------------------------
%\include{Introduction}
%\include{Conclusions}
\include{chapters/1Project/main}
\include{chapters/2Lit/main}
\include{chapters/3Design/HighLevel}
\include{chapters/3Design/InDepth}
\include{chapters/4Impl/main}

\include{chapters/5Experiments/1/main}
\include{chapters/5Experiments/2/main}
\include{chapters/5Experiments/3/main}
\include{chapters/5Experiments/4/main}

\include{chapters/6Conclusion/main}

\appendix
\include{appendix/AppendixB}
\include{appendix/D/main}
\include{appendix/AppendixC}

\backmatter
\bibliographystyle{ecs}
\bibliography{ECS}
\end{document}
%% ----------------------------------------------------------------


 %% ----------------------------------------------------------------
%% Progress.tex
%% ---------------------------------------------------------------- 
\documentclass{ecsprogress}    % Use the progress Style
\graphicspath{{../figs/}}   % Location of your graphics files
    \usepackage{natbib}            % Use Natbib style for the refs.
\hypersetup{colorlinks=true}   % Set to false for black/white printing
\input{Definitions}            % Include your abbreviations



\usepackage{enumitem}% http://ctan.org/pkg/enumitem
\usepackage{multirow}
\usepackage{float}
\usepackage{amsmath}
\usepackage{multicol}
\usepackage{amssymb}
\usepackage[normalem]{ulem}
\useunder{\uline}{\ul}{}
\usepackage{wrapfig}


\usepackage[table,xcdraw]{xcolor}


%% ----------------------------------------------------------------
\begin{document}
\frontmatter
\title      {Heterogeneous Agent-based Model for Supermarket Competition}
\authors    {\texorpdfstring
             {\href{mailto:sc22g13@ecs.soton.ac.uk}{Stefan J. Collier}}
             {Stefan J. Collier}
            }
\addresses  {\groupname\\\deptname\\\univname}
\date       {\today}
\subject    {}
\keywords   {}
\supervisor {Dr. Maria Polukarov}
\examiner   {Professor Sheng Chen}

\maketitle
\begin{abstract}
This project aim was to model and analyse the effects of competitive pricing behaviors of grocery retailers on the British market. 

This was achieved by creating a multi-agent model, containing retailer and consumer agents. The heterogeneous crowd of retailers employs either a uniform pricing strategy or a ‘local price flexing’ strategy. The actions of these retailers are chosen by predicting the profit of each action, using a perceptron. Following on from the consideration of different economic models, a discrete model was developed so that software agents have a discrete environment to operate within. Within the model, it has been observed how supermarkets with differing behaviors affect a heterogeneous crowd of consumer agents. The model was implemented in Java with Python used to evaluate the results. 

The simulation displays good acceptance with real grocery market behavior, i.e. captures the performance of British retailers thus can be used to determine the impact of changes in their behavior on their competitors and consumers.Furthermore it can be used to provide insight into sustainability of volatile pricing strategies, providing a useful insight in volatility of British supermarket retail industry. 
\end{abstract}
\acknowledgements{
I would like to express my sincere gratitude to Dr Maria Polukarov for her guidance and support which provided me the freedom to take this research in the direction of my interest.\\
\\
I would also like to thank my family and friends for their encouragement and support. To those who quietly listened to my software complaints. To those who worked throughout the nights with me. To those who helped me write what I couldn't say. I cannot thank you enough.
}

\declaration{
I, Stefan Collier, declare that this dissertation and the work presented in it are my own and has been generated by me as the result of my own original research.\\
I confirm that:\\
1. This work was done wholly or mainly while in candidature for a degree at this University;\\
2. Where any part of this dissertation has previously been submitted for any other qualification at this University or any other institution, this has been clearly stated;\\
3. Where I have consulted the published work of others, this is always clearly attributed;\\
4. Where I have quoted from the work of others, the source is always given. With the exception of such quotations, this dissertation is entirely my own work;\\
5. I have acknowledged all main sources of help;\\
6. Where the thesis is based on work done by myself jointly with others, I have made clear exactly what was done by others and what I have contributed myself;\\
7. Either none of this work has been published before submission, or parts of this work have been published by :\\
\\
Stefan Collier\\
April 2016
}
\tableofcontents
\listoffigures
\listoftables

\mainmatter
%% ----------------------------------------------------------------
%\include{Introduction}
%\include{Conclusions}
\include{chapters/1Project/main}
\include{chapters/2Lit/main}
\include{chapters/3Design/HighLevel}
\include{chapters/3Design/InDepth}
\include{chapters/4Impl/main}

\include{chapters/5Experiments/1/main}
\include{chapters/5Experiments/2/main}
\include{chapters/5Experiments/3/main}
\include{chapters/5Experiments/4/main}

\include{chapters/6Conclusion/main}

\appendix
\include{appendix/AppendixB}
\include{appendix/D/main}
\include{appendix/AppendixC}

\backmatter
\bibliographystyle{ecs}
\bibliography{ECS}
\end{document}
%% ----------------------------------------------------------------

 %% ----------------------------------------------------------------
%% Progress.tex
%% ---------------------------------------------------------------- 
\documentclass{ecsprogress}    % Use the progress Style
\graphicspath{{../figs/}}   % Location of your graphics files
    \usepackage{natbib}            % Use Natbib style for the refs.
\hypersetup{colorlinks=true}   % Set to false for black/white printing
\input{Definitions}            % Include your abbreviations



\usepackage{enumitem}% http://ctan.org/pkg/enumitem
\usepackage{multirow}
\usepackage{float}
\usepackage{amsmath}
\usepackage{multicol}
\usepackage{amssymb}
\usepackage[normalem]{ulem}
\useunder{\uline}{\ul}{}
\usepackage{wrapfig}


\usepackage[table,xcdraw]{xcolor}


%% ----------------------------------------------------------------
\begin{document}
\frontmatter
\title      {Heterogeneous Agent-based Model for Supermarket Competition}
\authors    {\texorpdfstring
             {\href{mailto:sc22g13@ecs.soton.ac.uk}{Stefan J. Collier}}
             {Stefan J. Collier}
            }
\addresses  {\groupname\\\deptname\\\univname}
\date       {\today}
\subject    {}
\keywords   {}
\supervisor {Dr. Maria Polukarov}
\examiner   {Professor Sheng Chen}

\maketitle
\begin{abstract}
This project aim was to model and analyse the effects of competitive pricing behaviors of grocery retailers on the British market. 

This was achieved by creating a multi-agent model, containing retailer and consumer agents. The heterogeneous crowd of retailers employs either a uniform pricing strategy or a ‘local price flexing’ strategy. The actions of these retailers are chosen by predicting the profit of each action, using a perceptron. Following on from the consideration of different economic models, a discrete model was developed so that software agents have a discrete environment to operate within. Within the model, it has been observed how supermarkets with differing behaviors affect a heterogeneous crowd of consumer agents. The model was implemented in Java with Python used to evaluate the results. 

The simulation displays good acceptance with real grocery market behavior, i.e. captures the performance of British retailers thus can be used to determine the impact of changes in their behavior on their competitors and consumers.Furthermore it can be used to provide insight into sustainability of volatile pricing strategies, providing a useful insight in volatility of British supermarket retail industry. 
\end{abstract}
\acknowledgements{
I would like to express my sincere gratitude to Dr Maria Polukarov for her guidance and support which provided me the freedom to take this research in the direction of my interest.\\
\\
I would also like to thank my family and friends for their encouragement and support. To those who quietly listened to my software complaints. To those who worked throughout the nights with me. To those who helped me write what I couldn't say. I cannot thank you enough.
}

\declaration{
I, Stefan Collier, declare that this dissertation and the work presented in it are my own and has been generated by me as the result of my own original research.\\
I confirm that:\\
1. This work was done wholly or mainly while in candidature for a degree at this University;\\
2. Where any part of this dissertation has previously been submitted for any other qualification at this University or any other institution, this has been clearly stated;\\
3. Where I have consulted the published work of others, this is always clearly attributed;\\
4. Where I have quoted from the work of others, the source is always given. With the exception of such quotations, this dissertation is entirely my own work;\\
5. I have acknowledged all main sources of help;\\
6. Where the thesis is based on work done by myself jointly with others, I have made clear exactly what was done by others and what I have contributed myself;\\
7. Either none of this work has been published before submission, or parts of this work have been published by :\\
\\
Stefan Collier\\
April 2016
}
\tableofcontents
\listoffigures
\listoftables

\mainmatter
%% ----------------------------------------------------------------
%\include{Introduction}
%\include{Conclusions}
\include{chapters/1Project/main}
\include{chapters/2Lit/main}
\include{chapters/3Design/HighLevel}
\include{chapters/3Design/InDepth}
\include{chapters/4Impl/main}

\include{chapters/5Experiments/1/main}
\include{chapters/5Experiments/2/main}
\include{chapters/5Experiments/3/main}
\include{chapters/5Experiments/4/main}

\include{chapters/6Conclusion/main}

\appendix
\include{appendix/AppendixB}
\include{appendix/D/main}
\include{appendix/AppendixC}

\backmatter
\bibliographystyle{ecs}
\bibliography{ECS}
\end{document}
%% ----------------------------------------------------------------

 %% ----------------------------------------------------------------
%% Progress.tex
%% ---------------------------------------------------------------- 
\documentclass{ecsprogress}    % Use the progress Style
\graphicspath{{../figs/}}   % Location of your graphics files
    \usepackage{natbib}            % Use Natbib style for the refs.
\hypersetup{colorlinks=true}   % Set to false for black/white printing
\input{Definitions}            % Include your abbreviations



\usepackage{enumitem}% http://ctan.org/pkg/enumitem
\usepackage{multirow}
\usepackage{float}
\usepackage{amsmath}
\usepackage{multicol}
\usepackage{amssymb}
\usepackage[normalem]{ulem}
\useunder{\uline}{\ul}{}
\usepackage{wrapfig}


\usepackage[table,xcdraw]{xcolor}


%% ----------------------------------------------------------------
\begin{document}
\frontmatter
\title      {Heterogeneous Agent-based Model for Supermarket Competition}
\authors    {\texorpdfstring
             {\href{mailto:sc22g13@ecs.soton.ac.uk}{Stefan J. Collier}}
             {Stefan J. Collier}
            }
\addresses  {\groupname\\\deptname\\\univname}
\date       {\today}
\subject    {}
\keywords   {}
\supervisor {Dr. Maria Polukarov}
\examiner   {Professor Sheng Chen}

\maketitle
\begin{abstract}
This project aim was to model and analyse the effects of competitive pricing behaviors of grocery retailers on the British market. 

This was achieved by creating a multi-agent model, containing retailer and consumer agents. The heterogeneous crowd of retailers employs either a uniform pricing strategy or a ‘local price flexing’ strategy. The actions of these retailers are chosen by predicting the profit of each action, using a perceptron. Following on from the consideration of different economic models, a discrete model was developed so that software agents have a discrete environment to operate within. Within the model, it has been observed how supermarkets with differing behaviors affect a heterogeneous crowd of consumer agents. The model was implemented in Java with Python used to evaluate the results. 

The simulation displays good acceptance with real grocery market behavior, i.e. captures the performance of British retailers thus can be used to determine the impact of changes in their behavior on their competitors and consumers.Furthermore it can be used to provide insight into sustainability of volatile pricing strategies, providing a useful insight in volatility of British supermarket retail industry. 
\end{abstract}
\acknowledgements{
I would like to express my sincere gratitude to Dr Maria Polukarov for her guidance and support which provided me the freedom to take this research in the direction of my interest.\\
\\
I would also like to thank my family and friends for their encouragement and support. To those who quietly listened to my software complaints. To those who worked throughout the nights with me. To those who helped me write what I couldn't say. I cannot thank you enough.
}

\declaration{
I, Stefan Collier, declare that this dissertation and the work presented in it are my own and has been generated by me as the result of my own original research.\\
I confirm that:\\
1. This work was done wholly or mainly while in candidature for a degree at this University;\\
2. Where any part of this dissertation has previously been submitted for any other qualification at this University or any other institution, this has been clearly stated;\\
3. Where I have consulted the published work of others, this is always clearly attributed;\\
4. Where I have quoted from the work of others, the source is always given. With the exception of such quotations, this dissertation is entirely my own work;\\
5. I have acknowledged all main sources of help;\\
6. Where the thesis is based on work done by myself jointly with others, I have made clear exactly what was done by others and what I have contributed myself;\\
7. Either none of this work has been published before submission, or parts of this work have been published by :\\
\\
Stefan Collier\\
April 2016
}
\tableofcontents
\listoffigures
\listoftables

\mainmatter
%% ----------------------------------------------------------------
%\include{Introduction}
%\include{Conclusions}
\include{chapters/1Project/main}
\include{chapters/2Lit/main}
\include{chapters/3Design/HighLevel}
\include{chapters/3Design/InDepth}
\include{chapters/4Impl/main}

\include{chapters/5Experiments/1/main}
\include{chapters/5Experiments/2/main}
\include{chapters/5Experiments/3/main}
\include{chapters/5Experiments/4/main}

\include{chapters/6Conclusion/main}

\appendix
\include{appendix/AppendixB}
\include{appendix/D/main}
\include{appendix/AppendixC}

\backmatter
\bibliographystyle{ecs}
\bibliography{ECS}
\end{document}
%% ----------------------------------------------------------------

 %% ----------------------------------------------------------------
%% Progress.tex
%% ---------------------------------------------------------------- 
\documentclass{ecsprogress}    % Use the progress Style
\graphicspath{{../figs/}}   % Location of your graphics files
    \usepackage{natbib}            % Use Natbib style for the refs.
\hypersetup{colorlinks=true}   % Set to false for black/white printing
\input{Definitions}            % Include your abbreviations



\usepackage{enumitem}% http://ctan.org/pkg/enumitem
\usepackage{multirow}
\usepackage{float}
\usepackage{amsmath}
\usepackage{multicol}
\usepackage{amssymb}
\usepackage[normalem]{ulem}
\useunder{\uline}{\ul}{}
\usepackage{wrapfig}


\usepackage[table,xcdraw]{xcolor}


%% ----------------------------------------------------------------
\begin{document}
\frontmatter
\title      {Heterogeneous Agent-based Model for Supermarket Competition}
\authors    {\texorpdfstring
             {\href{mailto:sc22g13@ecs.soton.ac.uk}{Stefan J. Collier}}
             {Stefan J. Collier}
            }
\addresses  {\groupname\\\deptname\\\univname}
\date       {\today}
\subject    {}
\keywords   {}
\supervisor {Dr. Maria Polukarov}
\examiner   {Professor Sheng Chen}

\maketitle
\begin{abstract}
This project aim was to model and analyse the effects of competitive pricing behaviors of grocery retailers on the British market. 

This was achieved by creating a multi-agent model, containing retailer and consumer agents. The heterogeneous crowd of retailers employs either a uniform pricing strategy or a ‘local price flexing’ strategy. The actions of these retailers are chosen by predicting the profit of each action, using a perceptron. Following on from the consideration of different economic models, a discrete model was developed so that software agents have a discrete environment to operate within. Within the model, it has been observed how supermarkets with differing behaviors affect a heterogeneous crowd of consumer agents. The model was implemented in Java with Python used to evaluate the results. 

The simulation displays good acceptance with real grocery market behavior, i.e. captures the performance of British retailers thus can be used to determine the impact of changes in their behavior on their competitors and consumers.Furthermore it can be used to provide insight into sustainability of volatile pricing strategies, providing a useful insight in volatility of British supermarket retail industry. 
\end{abstract}
\acknowledgements{
I would like to express my sincere gratitude to Dr Maria Polukarov for her guidance and support which provided me the freedom to take this research in the direction of my interest.\\
\\
I would also like to thank my family and friends for their encouragement and support. To those who quietly listened to my software complaints. To those who worked throughout the nights with me. To those who helped me write what I couldn't say. I cannot thank you enough.
}

\declaration{
I, Stefan Collier, declare that this dissertation and the work presented in it are my own and has been generated by me as the result of my own original research.\\
I confirm that:\\
1. This work was done wholly or mainly while in candidature for a degree at this University;\\
2. Where any part of this dissertation has previously been submitted for any other qualification at this University or any other institution, this has been clearly stated;\\
3. Where I have consulted the published work of others, this is always clearly attributed;\\
4. Where I have quoted from the work of others, the source is always given. With the exception of such quotations, this dissertation is entirely my own work;\\
5. I have acknowledged all main sources of help;\\
6. Where the thesis is based on work done by myself jointly with others, I have made clear exactly what was done by others and what I have contributed myself;\\
7. Either none of this work has been published before submission, or parts of this work have been published by :\\
\\
Stefan Collier\\
April 2016
}
\tableofcontents
\listoffigures
\listoftables

\mainmatter
%% ----------------------------------------------------------------
%\include{Introduction}
%\include{Conclusions}
\include{chapters/1Project/main}
\include{chapters/2Lit/main}
\include{chapters/3Design/HighLevel}
\include{chapters/3Design/InDepth}
\include{chapters/4Impl/main}

\include{chapters/5Experiments/1/main}
\include{chapters/5Experiments/2/main}
\include{chapters/5Experiments/3/main}
\include{chapters/5Experiments/4/main}

\include{chapters/6Conclusion/main}

\appendix
\include{appendix/AppendixB}
\include{appendix/D/main}
\include{appendix/AppendixC}

\backmatter
\bibliographystyle{ecs}
\bibliography{ECS}
\end{document}
%% ----------------------------------------------------------------


 %% ----------------------------------------------------------------
%% Progress.tex
%% ---------------------------------------------------------------- 
\documentclass{ecsprogress}    % Use the progress Style
\graphicspath{{../figs/}}   % Location of your graphics files
    \usepackage{natbib}            % Use Natbib style for the refs.
\hypersetup{colorlinks=true}   % Set to false for black/white printing
\input{Definitions}            % Include your abbreviations



\usepackage{enumitem}% http://ctan.org/pkg/enumitem
\usepackage{multirow}
\usepackage{float}
\usepackage{amsmath}
\usepackage{multicol}
\usepackage{amssymb}
\usepackage[normalem]{ulem}
\useunder{\uline}{\ul}{}
\usepackage{wrapfig}


\usepackage[table,xcdraw]{xcolor}


%% ----------------------------------------------------------------
\begin{document}
\frontmatter
\title      {Heterogeneous Agent-based Model for Supermarket Competition}
\authors    {\texorpdfstring
             {\href{mailto:sc22g13@ecs.soton.ac.uk}{Stefan J. Collier}}
             {Stefan J. Collier}
            }
\addresses  {\groupname\\\deptname\\\univname}
\date       {\today}
\subject    {}
\keywords   {}
\supervisor {Dr. Maria Polukarov}
\examiner   {Professor Sheng Chen}

\maketitle
\begin{abstract}
This project aim was to model and analyse the effects of competitive pricing behaviors of grocery retailers on the British market. 

This was achieved by creating a multi-agent model, containing retailer and consumer agents. The heterogeneous crowd of retailers employs either a uniform pricing strategy or a ‘local price flexing’ strategy. The actions of these retailers are chosen by predicting the profit of each action, using a perceptron. Following on from the consideration of different economic models, a discrete model was developed so that software agents have a discrete environment to operate within. Within the model, it has been observed how supermarkets with differing behaviors affect a heterogeneous crowd of consumer agents. The model was implemented in Java with Python used to evaluate the results. 

The simulation displays good acceptance with real grocery market behavior, i.e. captures the performance of British retailers thus can be used to determine the impact of changes in their behavior on their competitors and consumers.Furthermore it can be used to provide insight into sustainability of volatile pricing strategies, providing a useful insight in volatility of British supermarket retail industry. 
\end{abstract}
\acknowledgements{
I would like to express my sincere gratitude to Dr Maria Polukarov for her guidance and support which provided me the freedom to take this research in the direction of my interest.\\
\\
I would also like to thank my family and friends for their encouragement and support. To those who quietly listened to my software complaints. To those who worked throughout the nights with me. To those who helped me write what I couldn't say. I cannot thank you enough.
}

\declaration{
I, Stefan Collier, declare that this dissertation and the work presented in it are my own and has been generated by me as the result of my own original research.\\
I confirm that:\\
1. This work was done wholly or mainly while in candidature for a degree at this University;\\
2. Where any part of this dissertation has previously been submitted for any other qualification at this University or any other institution, this has been clearly stated;\\
3. Where I have consulted the published work of others, this is always clearly attributed;\\
4. Where I have quoted from the work of others, the source is always given. With the exception of such quotations, this dissertation is entirely my own work;\\
5. I have acknowledged all main sources of help;\\
6. Where the thesis is based on work done by myself jointly with others, I have made clear exactly what was done by others and what I have contributed myself;\\
7. Either none of this work has been published before submission, or parts of this work have been published by :\\
\\
Stefan Collier\\
April 2016
}
\tableofcontents
\listoffigures
\listoftables

\mainmatter
%% ----------------------------------------------------------------
%\include{Introduction}
%\include{Conclusions}
\include{chapters/1Project/main}
\include{chapters/2Lit/main}
\include{chapters/3Design/HighLevel}
\include{chapters/3Design/InDepth}
\include{chapters/4Impl/main}

\include{chapters/5Experiments/1/main}
\include{chapters/5Experiments/2/main}
\include{chapters/5Experiments/3/main}
\include{chapters/5Experiments/4/main}

\include{chapters/6Conclusion/main}

\appendix
\include{appendix/AppendixB}
\include{appendix/D/main}
\include{appendix/AppendixC}

\backmatter
\bibliographystyle{ecs}
\bibliography{ECS}
\end{document}
%% ----------------------------------------------------------------


\appendix
\include{appendix/AppendixB}
 %% ----------------------------------------------------------------
%% Progress.tex
%% ---------------------------------------------------------------- 
\documentclass{ecsprogress}    % Use the progress Style
\graphicspath{{../figs/}}   % Location of your graphics files
    \usepackage{natbib}            % Use Natbib style for the refs.
\hypersetup{colorlinks=true}   % Set to false for black/white printing
\input{Definitions}            % Include your abbreviations



\usepackage{enumitem}% http://ctan.org/pkg/enumitem
\usepackage{multirow}
\usepackage{float}
\usepackage{amsmath}
\usepackage{multicol}
\usepackage{amssymb}
\usepackage[normalem]{ulem}
\useunder{\uline}{\ul}{}
\usepackage{wrapfig}


\usepackage[table,xcdraw]{xcolor}


%% ----------------------------------------------------------------
\begin{document}
\frontmatter
\title      {Heterogeneous Agent-based Model for Supermarket Competition}
\authors    {\texorpdfstring
             {\href{mailto:sc22g13@ecs.soton.ac.uk}{Stefan J. Collier}}
             {Stefan J. Collier}
            }
\addresses  {\groupname\\\deptname\\\univname}
\date       {\today}
\subject    {}
\keywords   {}
\supervisor {Dr. Maria Polukarov}
\examiner   {Professor Sheng Chen}

\maketitle
\begin{abstract}
This project aim was to model and analyse the effects of competitive pricing behaviors of grocery retailers on the British market. 

This was achieved by creating a multi-agent model, containing retailer and consumer agents. The heterogeneous crowd of retailers employs either a uniform pricing strategy or a ‘local price flexing’ strategy. The actions of these retailers are chosen by predicting the profit of each action, using a perceptron. Following on from the consideration of different economic models, a discrete model was developed so that software agents have a discrete environment to operate within. Within the model, it has been observed how supermarkets with differing behaviors affect a heterogeneous crowd of consumer agents. The model was implemented in Java with Python used to evaluate the results. 

The simulation displays good acceptance with real grocery market behavior, i.e. captures the performance of British retailers thus can be used to determine the impact of changes in their behavior on their competitors and consumers.Furthermore it can be used to provide insight into sustainability of volatile pricing strategies, providing a useful insight in volatility of British supermarket retail industry. 
\end{abstract}
\acknowledgements{
I would like to express my sincere gratitude to Dr Maria Polukarov for her guidance and support which provided me the freedom to take this research in the direction of my interest.\\
\\
I would also like to thank my family and friends for their encouragement and support. To those who quietly listened to my software complaints. To those who worked throughout the nights with me. To those who helped me write what I couldn't say. I cannot thank you enough.
}

\declaration{
I, Stefan Collier, declare that this dissertation and the work presented in it are my own and has been generated by me as the result of my own original research.\\
I confirm that:\\
1. This work was done wholly or mainly while in candidature for a degree at this University;\\
2. Where any part of this dissertation has previously been submitted for any other qualification at this University or any other institution, this has been clearly stated;\\
3. Where I have consulted the published work of others, this is always clearly attributed;\\
4. Where I have quoted from the work of others, the source is always given. With the exception of such quotations, this dissertation is entirely my own work;\\
5. I have acknowledged all main sources of help;\\
6. Where the thesis is based on work done by myself jointly with others, I have made clear exactly what was done by others and what I have contributed myself;\\
7. Either none of this work has been published before submission, or parts of this work have been published by :\\
\\
Stefan Collier\\
April 2016
}
\tableofcontents
\listoffigures
\listoftables

\mainmatter
%% ----------------------------------------------------------------
%\include{Introduction}
%\include{Conclusions}
\include{chapters/1Project/main}
\include{chapters/2Lit/main}
\include{chapters/3Design/HighLevel}
\include{chapters/3Design/InDepth}
\include{chapters/4Impl/main}

\include{chapters/5Experiments/1/main}
\include{chapters/5Experiments/2/main}
\include{chapters/5Experiments/3/main}
\include{chapters/5Experiments/4/main}

\include{chapters/6Conclusion/main}

\appendix
\include{appendix/AppendixB}
\include{appendix/D/main}
\include{appendix/AppendixC}

\backmatter
\bibliographystyle{ecs}
\bibliography{ECS}
\end{document}
%% ----------------------------------------------------------------

\include{appendix/AppendixC}

\backmatter
\bibliographystyle{ecs}
\bibliography{ECS}
\end{document}
%% ----------------------------------------------------------------

\include{appendix/AppendixC}

\backmatter
\bibliographystyle{ecs}
\bibliography{ECS}
\end{document}
%% ----------------------------------------------------------------

 %% ----------------------------------------------------------------
%% Progress.tex
%% ---------------------------------------------------------------- 
\documentclass{ecsprogress}    % Use the progress Style
\graphicspath{{../figs/}}   % Location of your graphics files
    \usepackage{natbib}            % Use Natbib style for the refs.
\hypersetup{colorlinks=true}   % Set to false for black/white printing
\input{Definitions}            % Include your abbreviations



\usepackage{enumitem}% http://ctan.org/pkg/enumitem
\usepackage{multirow}
\usepackage{float}
\usepackage{amsmath}
\usepackage{multicol}
\usepackage{amssymb}
\usepackage[normalem]{ulem}
\useunder{\uline}{\ul}{}
\usepackage{wrapfig}


\usepackage[table,xcdraw]{xcolor}


%% ----------------------------------------------------------------
\begin{document}
\frontmatter
\title      {Heterogeneous Agent-based Model for Supermarket Competition}
\authors    {\texorpdfstring
             {\href{mailto:sc22g13@ecs.soton.ac.uk}{Stefan J. Collier}}
             {Stefan J. Collier}
            }
\addresses  {\groupname\\\deptname\\\univname}
\date       {\today}
\subject    {}
\keywords   {}
\supervisor {Dr. Maria Polukarov}
\examiner   {Professor Sheng Chen}

\maketitle
\begin{abstract}
This project aim was to model and analyse the effects of competitive pricing behaviors of grocery retailers on the British market. 

This was achieved by creating a multi-agent model, containing retailer and consumer agents. The heterogeneous crowd of retailers employs either a uniform pricing strategy or a ‘local price flexing’ strategy. The actions of these retailers are chosen by predicting the profit of each action, using a perceptron. Following on from the consideration of different economic models, a discrete model was developed so that software agents have a discrete environment to operate within. Within the model, it has been observed how supermarkets with differing behaviors affect a heterogeneous crowd of consumer agents. The model was implemented in Java with Python used to evaluate the results. 

The simulation displays good acceptance with real grocery market behavior, i.e. captures the performance of British retailers thus can be used to determine the impact of changes in their behavior on their competitors and consumers.Furthermore it can be used to provide insight into sustainability of volatile pricing strategies, providing a useful insight in volatility of British supermarket retail industry. 
\end{abstract}
\acknowledgements{
I would like to express my sincere gratitude to Dr Maria Polukarov for her guidance and support which provided me the freedom to take this research in the direction of my interest.\\
\\
I would also like to thank my family and friends for their encouragement and support. To those who quietly listened to my software complaints. To those who worked throughout the nights with me. To those who helped me write what I couldn't say. I cannot thank you enough.
}

\declaration{
I, Stefan Collier, declare that this dissertation and the work presented in it are my own and has been generated by me as the result of my own original research.\\
I confirm that:\\
1. This work was done wholly or mainly while in candidature for a degree at this University;\\
2. Where any part of this dissertation has previously been submitted for any other qualification at this University or any other institution, this has been clearly stated;\\
3. Where I have consulted the published work of others, this is always clearly attributed;\\
4. Where I have quoted from the work of others, the source is always given. With the exception of such quotations, this dissertation is entirely my own work;\\
5. I have acknowledged all main sources of help;\\
6. Where the thesis is based on work done by myself jointly with others, I have made clear exactly what was done by others and what I have contributed myself;\\
7. Either none of this work has been published before submission, or parts of this work have been published by :\\
\\
Stefan Collier\\
April 2016
}
\tableofcontents
\listoffigures
\listoftables

\mainmatter
%% ----------------------------------------------------------------
%\include{Introduction}
%\include{Conclusions}
 %% ----------------------------------------------------------------
%% Progress.tex
%% ---------------------------------------------------------------- 
\documentclass{ecsprogress}    % Use the progress Style
\graphicspath{{../figs/}}   % Location of your graphics files
    \usepackage{natbib}            % Use Natbib style for the refs.
\hypersetup{colorlinks=true}   % Set to false for black/white printing
\input{Definitions}            % Include your abbreviations



\usepackage{enumitem}% http://ctan.org/pkg/enumitem
\usepackage{multirow}
\usepackage{float}
\usepackage{amsmath}
\usepackage{multicol}
\usepackage{amssymb}
\usepackage[normalem]{ulem}
\useunder{\uline}{\ul}{}
\usepackage{wrapfig}


\usepackage[table,xcdraw]{xcolor}


%% ----------------------------------------------------------------
\begin{document}
\frontmatter
\title      {Heterogeneous Agent-based Model for Supermarket Competition}
\authors    {\texorpdfstring
             {\href{mailto:sc22g13@ecs.soton.ac.uk}{Stefan J. Collier}}
             {Stefan J. Collier}
            }
\addresses  {\groupname\\\deptname\\\univname}
\date       {\today}
\subject    {}
\keywords   {}
\supervisor {Dr. Maria Polukarov}
\examiner   {Professor Sheng Chen}

\maketitle
\begin{abstract}
This project aim was to model and analyse the effects of competitive pricing behaviors of grocery retailers on the British market. 

This was achieved by creating a multi-agent model, containing retailer and consumer agents. The heterogeneous crowd of retailers employs either a uniform pricing strategy or a ‘local price flexing’ strategy. The actions of these retailers are chosen by predicting the profit of each action, using a perceptron. Following on from the consideration of different economic models, a discrete model was developed so that software agents have a discrete environment to operate within. Within the model, it has been observed how supermarkets with differing behaviors affect a heterogeneous crowd of consumer agents. The model was implemented in Java with Python used to evaluate the results. 

The simulation displays good acceptance with real grocery market behavior, i.e. captures the performance of British retailers thus can be used to determine the impact of changes in their behavior on their competitors and consumers.Furthermore it can be used to provide insight into sustainability of volatile pricing strategies, providing a useful insight in volatility of British supermarket retail industry. 
\end{abstract}
\acknowledgements{
I would like to express my sincere gratitude to Dr Maria Polukarov for her guidance and support which provided me the freedom to take this research in the direction of my interest.\\
\\
I would also like to thank my family and friends for their encouragement and support. To those who quietly listened to my software complaints. To those who worked throughout the nights with me. To those who helped me write what I couldn't say. I cannot thank you enough.
}

\declaration{
I, Stefan Collier, declare that this dissertation and the work presented in it are my own and has been generated by me as the result of my own original research.\\
I confirm that:\\
1. This work was done wholly or mainly while in candidature for a degree at this University;\\
2. Where any part of this dissertation has previously been submitted for any other qualification at this University or any other institution, this has been clearly stated;\\
3. Where I have consulted the published work of others, this is always clearly attributed;\\
4. Where I have quoted from the work of others, the source is always given. With the exception of such quotations, this dissertation is entirely my own work;\\
5. I have acknowledged all main sources of help;\\
6. Where the thesis is based on work done by myself jointly with others, I have made clear exactly what was done by others and what I have contributed myself;\\
7. Either none of this work has been published before submission, or parts of this work have been published by :\\
\\
Stefan Collier\\
April 2016
}
\tableofcontents
\listoffigures
\listoftables

\mainmatter
%% ----------------------------------------------------------------
%\include{Introduction}
%\include{Conclusions}
 %% ----------------------------------------------------------------
%% Progress.tex
%% ---------------------------------------------------------------- 
\documentclass{ecsprogress}    % Use the progress Style
\graphicspath{{../figs/}}   % Location of your graphics files
    \usepackage{natbib}            % Use Natbib style for the refs.
\hypersetup{colorlinks=true}   % Set to false for black/white printing
\input{Definitions}            % Include your abbreviations



\usepackage{enumitem}% http://ctan.org/pkg/enumitem
\usepackage{multirow}
\usepackage{float}
\usepackage{amsmath}
\usepackage{multicol}
\usepackage{amssymb}
\usepackage[normalem]{ulem}
\useunder{\uline}{\ul}{}
\usepackage{wrapfig}


\usepackage[table,xcdraw]{xcolor}


%% ----------------------------------------------------------------
\begin{document}
\frontmatter
\title      {Heterogeneous Agent-based Model for Supermarket Competition}
\authors    {\texorpdfstring
             {\href{mailto:sc22g13@ecs.soton.ac.uk}{Stefan J. Collier}}
             {Stefan J. Collier}
            }
\addresses  {\groupname\\\deptname\\\univname}
\date       {\today}
\subject    {}
\keywords   {}
\supervisor {Dr. Maria Polukarov}
\examiner   {Professor Sheng Chen}

\maketitle
\begin{abstract}
This project aim was to model and analyse the effects of competitive pricing behaviors of grocery retailers on the British market. 

This was achieved by creating a multi-agent model, containing retailer and consumer agents. The heterogeneous crowd of retailers employs either a uniform pricing strategy or a ‘local price flexing’ strategy. The actions of these retailers are chosen by predicting the profit of each action, using a perceptron. Following on from the consideration of different economic models, a discrete model was developed so that software agents have a discrete environment to operate within. Within the model, it has been observed how supermarkets with differing behaviors affect a heterogeneous crowd of consumer agents. The model was implemented in Java with Python used to evaluate the results. 

The simulation displays good acceptance with real grocery market behavior, i.e. captures the performance of British retailers thus can be used to determine the impact of changes in their behavior on their competitors and consumers.Furthermore it can be used to provide insight into sustainability of volatile pricing strategies, providing a useful insight in volatility of British supermarket retail industry. 
\end{abstract}
\acknowledgements{
I would like to express my sincere gratitude to Dr Maria Polukarov for her guidance and support which provided me the freedom to take this research in the direction of my interest.\\
\\
I would also like to thank my family and friends for their encouragement and support. To those who quietly listened to my software complaints. To those who worked throughout the nights with me. To those who helped me write what I couldn't say. I cannot thank you enough.
}

\declaration{
I, Stefan Collier, declare that this dissertation and the work presented in it are my own and has been generated by me as the result of my own original research.\\
I confirm that:\\
1. This work was done wholly or mainly while in candidature for a degree at this University;\\
2. Where any part of this dissertation has previously been submitted for any other qualification at this University or any other institution, this has been clearly stated;\\
3. Where I have consulted the published work of others, this is always clearly attributed;\\
4. Where I have quoted from the work of others, the source is always given. With the exception of such quotations, this dissertation is entirely my own work;\\
5. I have acknowledged all main sources of help;\\
6. Where the thesis is based on work done by myself jointly with others, I have made clear exactly what was done by others and what I have contributed myself;\\
7. Either none of this work has been published before submission, or parts of this work have been published by :\\
\\
Stefan Collier\\
April 2016
}
\tableofcontents
\listoffigures
\listoftables

\mainmatter
%% ----------------------------------------------------------------
%\include{Introduction}
%\include{Conclusions}
\include{chapters/1Project/main}
\include{chapters/2Lit/main}
\include{chapters/3Design/HighLevel}
\include{chapters/3Design/InDepth}
\include{chapters/4Impl/main}

\include{chapters/5Experiments/1/main}
\include{chapters/5Experiments/2/main}
\include{chapters/5Experiments/3/main}
\include{chapters/5Experiments/4/main}

\include{chapters/6Conclusion/main}

\appendix
\include{appendix/AppendixB}
\include{appendix/D/main}
\include{appendix/AppendixC}

\backmatter
\bibliographystyle{ecs}
\bibliography{ECS}
\end{document}
%% ----------------------------------------------------------------

 %% ----------------------------------------------------------------
%% Progress.tex
%% ---------------------------------------------------------------- 
\documentclass{ecsprogress}    % Use the progress Style
\graphicspath{{../figs/}}   % Location of your graphics files
    \usepackage{natbib}            % Use Natbib style for the refs.
\hypersetup{colorlinks=true}   % Set to false for black/white printing
\input{Definitions}            % Include your abbreviations



\usepackage{enumitem}% http://ctan.org/pkg/enumitem
\usepackage{multirow}
\usepackage{float}
\usepackage{amsmath}
\usepackage{multicol}
\usepackage{amssymb}
\usepackage[normalem]{ulem}
\useunder{\uline}{\ul}{}
\usepackage{wrapfig}


\usepackage[table,xcdraw]{xcolor}


%% ----------------------------------------------------------------
\begin{document}
\frontmatter
\title      {Heterogeneous Agent-based Model for Supermarket Competition}
\authors    {\texorpdfstring
             {\href{mailto:sc22g13@ecs.soton.ac.uk}{Stefan J. Collier}}
             {Stefan J. Collier}
            }
\addresses  {\groupname\\\deptname\\\univname}
\date       {\today}
\subject    {}
\keywords   {}
\supervisor {Dr. Maria Polukarov}
\examiner   {Professor Sheng Chen}

\maketitle
\begin{abstract}
This project aim was to model and analyse the effects of competitive pricing behaviors of grocery retailers on the British market. 

This was achieved by creating a multi-agent model, containing retailer and consumer agents. The heterogeneous crowd of retailers employs either a uniform pricing strategy or a ‘local price flexing’ strategy. The actions of these retailers are chosen by predicting the profit of each action, using a perceptron. Following on from the consideration of different economic models, a discrete model was developed so that software agents have a discrete environment to operate within. Within the model, it has been observed how supermarkets with differing behaviors affect a heterogeneous crowd of consumer agents. The model was implemented in Java with Python used to evaluate the results. 

The simulation displays good acceptance with real grocery market behavior, i.e. captures the performance of British retailers thus can be used to determine the impact of changes in their behavior on their competitors and consumers.Furthermore it can be used to provide insight into sustainability of volatile pricing strategies, providing a useful insight in volatility of British supermarket retail industry. 
\end{abstract}
\acknowledgements{
I would like to express my sincere gratitude to Dr Maria Polukarov for her guidance and support which provided me the freedom to take this research in the direction of my interest.\\
\\
I would also like to thank my family and friends for their encouragement and support. To those who quietly listened to my software complaints. To those who worked throughout the nights with me. To those who helped me write what I couldn't say. I cannot thank you enough.
}

\declaration{
I, Stefan Collier, declare that this dissertation and the work presented in it are my own and has been generated by me as the result of my own original research.\\
I confirm that:\\
1. This work was done wholly or mainly while in candidature for a degree at this University;\\
2. Where any part of this dissertation has previously been submitted for any other qualification at this University or any other institution, this has been clearly stated;\\
3. Where I have consulted the published work of others, this is always clearly attributed;\\
4. Where I have quoted from the work of others, the source is always given. With the exception of such quotations, this dissertation is entirely my own work;\\
5. I have acknowledged all main sources of help;\\
6. Where the thesis is based on work done by myself jointly with others, I have made clear exactly what was done by others and what I have contributed myself;\\
7. Either none of this work has been published before submission, or parts of this work have been published by :\\
\\
Stefan Collier\\
April 2016
}
\tableofcontents
\listoffigures
\listoftables

\mainmatter
%% ----------------------------------------------------------------
%\include{Introduction}
%\include{Conclusions}
\include{chapters/1Project/main}
\include{chapters/2Lit/main}
\include{chapters/3Design/HighLevel}
\include{chapters/3Design/InDepth}
\include{chapters/4Impl/main}

\include{chapters/5Experiments/1/main}
\include{chapters/5Experiments/2/main}
\include{chapters/5Experiments/3/main}
\include{chapters/5Experiments/4/main}

\include{chapters/6Conclusion/main}

\appendix
\include{appendix/AppendixB}
\include{appendix/D/main}
\include{appendix/AppendixC}

\backmatter
\bibliographystyle{ecs}
\bibliography{ECS}
\end{document}
%% ----------------------------------------------------------------

\include{chapters/3Design/HighLevel}
\include{chapters/3Design/InDepth}
 %% ----------------------------------------------------------------
%% Progress.tex
%% ---------------------------------------------------------------- 
\documentclass{ecsprogress}    % Use the progress Style
\graphicspath{{../figs/}}   % Location of your graphics files
    \usepackage{natbib}            % Use Natbib style for the refs.
\hypersetup{colorlinks=true}   % Set to false for black/white printing
\input{Definitions}            % Include your abbreviations



\usepackage{enumitem}% http://ctan.org/pkg/enumitem
\usepackage{multirow}
\usepackage{float}
\usepackage{amsmath}
\usepackage{multicol}
\usepackage{amssymb}
\usepackage[normalem]{ulem}
\useunder{\uline}{\ul}{}
\usepackage{wrapfig}


\usepackage[table,xcdraw]{xcolor}


%% ----------------------------------------------------------------
\begin{document}
\frontmatter
\title      {Heterogeneous Agent-based Model for Supermarket Competition}
\authors    {\texorpdfstring
             {\href{mailto:sc22g13@ecs.soton.ac.uk}{Stefan J. Collier}}
             {Stefan J. Collier}
            }
\addresses  {\groupname\\\deptname\\\univname}
\date       {\today}
\subject    {}
\keywords   {}
\supervisor {Dr. Maria Polukarov}
\examiner   {Professor Sheng Chen}

\maketitle
\begin{abstract}
This project aim was to model and analyse the effects of competitive pricing behaviors of grocery retailers on the British market. 

This was achieved by creating a multi-agent model, containing retailer and consumer agents. The heterogeneous crowd of retailers employs either a uniform pricing strategy or a ‘local price flexing’ strategy. The actions of these retailers are chosen by predicting the profit of each action, using a perceptron. Following on from the consideration of different economic models, a discrete model was developed so that software agents have a discrete environment to operate within. Within the model, it has been observed how supermarkets with differing behaviors affect a heterogeneous crowd of consumer agents. The model was implemented in Java with Python used to evaluate the results. 

The simulation displays good acceptance with real grocery market behavior, i.e. captures the performance of British retailers thus can be used to determine the impact of changes in their behavior on their competitors and consumers.Furthermore it can be used to provide insight into sustainability of volatile pricing strategies, providing a useful insight in volatility of British supermarket retail industry. 
\end{abstract}
\acknowledgements{
I would like to express my sincere gratitude to Dr Maria Polukarov for her guidance and support which provided me the freedom to take this research in the direction of my interest.\\
\\
I would also like to thank my family and friends for their encouragement and support. To those who quietly listened to my software complaints. To those who worked throughout the nights with me. To those who helped me write what I couldn't say. I cannot thank you enough.
}

\declaration{
I, Stefan Collier, declare that this dissertation and the work presented in it are my own and has been generated by me as the result of my own original research.\\
I confirm that:\\
1. This work was done wholly or mainly while in candidature for a degree at this University;\\
2. Where any part of this dissertation has previously been submitted for any other qualification at this University or any other institution, this has been clearly stated;\\
3. Where I have consulted the published work of others, this is always clearly attributed;\\
4. Where I have quoted from the work of others, the source is always given. With the exception of such quotations, this dissertation is entirely my own work;\\
5. I have acknowledged all main sources of help;\\
6. Where the thesis is based on work done by myself jointly with others, I have made clear exactly what was done by others and what I have contributed myself;\\
7. Either none of this work has been published before submission, or parts of this work have been published by :\\
\\
Stefan Collier\\
April 2016
}
\tableofcontents
\listoffigures
\listoftables

\mainmatter
%% ----------------------------------------------------------------
%\include{Introduction}
%\include{Conclusions}
\include{chapters/1Project/main}
\include{chapters/2Lit/main}
\include{chapters/3Design/HighLevel}
\include{chapters/3Design/InDepth}
\include{chapters/4Impl/main}

\include{chapters/5Experiments/1/main}
\include{chapters/5Experiments/2/main}
\include{chapters/5Experiments/3/main}
\include{chapters/5Experiments/4/main}

\include{chapters/6Conclusion/main}

\appendix
\include{appendix/AppendixB}
\include{appendix/D/main}
\include{appendix/AppendixC}

\backmatter
\bibliographystyle{ecs}
\bibliography{ECS}
\end{document}
%% ----------------------------------------------------------------


 %% ----------------------------------------------------------------
%% Progress.tex
%% ---------------------------------------------------------------- 
\documentclass{ecsprogress}    % Use the progress Style
\graphicspath{{../figs/}}   % Location of your graphics files
    \usepackage{natbib}            % Use Natbib style for the refs.
\hypersetup{colorlinks=true}   % Set to false for black/white printing
\input{Definitions}            % Include your abbreviations



\usepackage{enumitem}% http://ctan.org/pkg/enumitem
\usepackage{multirow}
\usepackage{float}
\usepackage{amsmath}
\usepackage{multicol}
\usepackage{amssymb}
\usepackage[normalem]{ulem}
\useunder{\uline}{\ul}{}
\usepackage{wrapfig}


\usepackage[table,xcdraw]{xcolor}


%% ----------------------------------------------------------------
\begin{document}
\frontmatter
\title      {Heterogeneous Agent-based Model for Supermarket Competition}
\authors    {\texorpdfstring
             {\href{mailto:sc22g13@ecs.soton.ac.uk}{Stefan J. Collier}}
             {Stefan J. Collier}
            }
\addresses  {\groupname\\\deptname\\\univname}
\date       {\today}
\subject    {}
\keywords   {}
\supervisor {Dr. Maria Polukarov}
\examiner   {Professor Sheng Chen}

\maketitle
\begin{abstract}
This project aim was to model and analyse the effects of competitive pricing behaviors of grocery retailers on the British market. 

This was achieved by creating a multi-agent model, containing retailer and consumer agents. The heterogeneous crowd of retailers employs either a uniform pricing strategy or a ‘local price flexing’ strategy. The actions of these retailers are chosen by predicting the profit of each action, using a perceptron. Following on from the consideration of different economic models, a discrete model was developed so that software agents have a discrete environment to operate within. Within the model, it has been observed how supermarkets with differing behaviors affect a heterogeneous crowd of consumer agents. The model was implemented in Java with Python used to evaluate the results. 

The simulation displays good acceptance with real grocery market behavior, i.e. captures the performance of British retailers thus can be used to determine the impact of changes in their behavior on their competitors and consumers.Furthermore it can be used to provide insight into sustainability of volatile pricing strategies, providing a useful insight in volatility of British supermarket retail industry. 
\end{abstract}
\acknowledgements{
I would like to express my sincere gratitude to Dr Maria Polukarov for her guidance and support which provided me the freedom to take this research in the direction of my interest.\\
\\
I would also like to thank my family and friends for their encouragement and support. To those who quietly listened to my software complaints. To those who worked throughout the nights with me. To those who helped me write what I couldn't say. I cannot thank you enough.
}

\declaration{
I, Stefan Collier, declare that this dissertation and the work presented in it are my own and has been generated by me as the result of my own original research.\\
I confirm that:\\
1. This work was done wholly or mainly while in candidature for a degree at this University;\\
2. Where any part of this dissertation has previously been submitted for any other qualification at this University or any other institution, this has been clearly stated;\\
3. Where I have consulted the published work of others, this is always clearly attributed;\\
4. Where I have quoted from the work of others, the source is always given. With the exception of such quotations, this dissertation is entirely my own work;\\
5. I have acknowledged all main sources of help;\\
6. Where the thesis is based on work done by myself jointly with others, I have made clear exactly what was done by others and what I have contributed myself;\\
7. Either none of this work has been published before submission, or parts of this work have been published by :\\
\\
Stefan Collier\\
April 2016
}
\tableofcontents
\listoffigures
\listoftables

\mainmatter
%% ----------------------------------------------------------------
%\include{Introduction}
%\include{Conclusions}
\include{chapters/1Project/main}
\include{chapters/2Lit/main}
\include{chapters/3Design/HighLevel}
\include{chapters/3Design/InDepth}
\include{chapters/4Impl/main}

\include{chapters/5Experiments/1/main}
\include{chapters/5Experiments/2/main}
\include{chapters/5Experiments/3/main}
\include{chapters/5Experiments/4/main}

\include{chapters/6Conclusion/main}

\appendix
\include{appendix/AppendixB}
\include{appendix/D/main}
\include{appendix/AppendixC}

\backmatter
\bibliographystyle{ecs}
\bibliography{ECS}
\end{document}
%% ----------------------------------------------------------------

 %% ----------------------------------------------------------------
%% Progress.tex
%% ---------------------------------------------------------------- 
\documentclass{ecsprogress}    % Use the progress Style
\graphicspath{{../figs/}}   % Location of your graphics files
    \usepackage{natbib}            % Use Natbib style for the refs.
\hypersetup{colorlinks=true}   % Set to false for black/white printing
\input{Definitions}            % Include your abbreviations



\usepackage{enumitem}% http://ctan.org/pkg/enumitem
\usepackage{multirow}
\usepackage{float}
\usepackage{amsmath}
\usepackage{multicol}
\usepackage{amssymb}
\usepackage[normalem]{ulem}
\useunder{\uline}{\ul}{}
\usepackage{wrapfig}


\usepackage[table,xcdraw]{xcolor}


%% ----------------------------------------------------------------
\begin{document}
\frontmatter
\title      {Heterogeneous Agent-based Model for Supermarket Competition}
\authors    {\texorpdfstring
             {\href{mailto:sc22g13@ecs.soton.ac.uk}{Stefan J. Collier}}
             {Stefan J. Collier}
            }
\addresses  {\groupname\\\deptname\\\univname}
\date       {\today}
\subject    {}
\keywords   {}
\supervisor {Dr. Maria Polukarov}
\examiner   {Professor Sheng Chen}

\maketitle
\begin{abstract}
This project aim was to model and analyse the effects of competitive pricing behaviors of grocery retailers on the British market. 

This was achieved by creating a multi-agent model, containing retailer and consumer agents. The heterogeneous crowd of retailers employs either a uniform pricing strategy or a ‘local price flexing’ strategy. The actions of these retailers are chosen by predicting the profit of each action, using a perceptron. Following on from the consideration of different economic models, a discrete model was developed so that software agents have a discrete environment to operate within. Within the model, it has been observed how supermarkets with differing behaviors affect a heterogeneous crowd of consumer agents. The model was implemented in Java with Python used to evaluate the results. 

The simulation displays good acceptance with real grocery market behavior, i.e. captures the performance of British retailers thus can be used to determine the impact of changes in their behavior on their competitors and consumers.Furthermore it can be used to provide insight into sustainability of volatile pricing strategies, providing a useful insight in volatility of British supermarket retail industry. 
\end{abstract}
\acknowledgements{
I would like to express my sincere gratitude to Dr Maria Polukarov for her guidance and support which provided me the freedom to take this research in the direction of my interest.\\
\\
I would also like to thank my family and friends for their encouragement and support. To those who quietly listened to my software complaints. To those who worked throughout the nights with me. To those who helped me write what I couldn't say. I cannot thank you enough.
}

\declaration{
I, Stefan Collier, declare that this dissertation and the work presented in it are my own and has been generated by me as the result of my own original research.\\
I confirm that:\\
1. This work was done wholly or mainly while in candidature for a degree at this University;\\
2. Where any part of this dissertation has previously been submitted for any other qualification at this University or any other institution, this has been clearly stated;\\
3. Where I have consulted the published work of others, this is always clearly attributed;\\
4. Where I have quoted from the work of others, the source is always given. With the exception of such quotations, this dissertation is entirely my own work;\\
5. I have acknowledged all main sources of help;\\
6. Where the thesis is based on work done by myself jointly with others, I have made clear exactly what was done by others and what I have contributed myself;\\
7. Either none of this work has been published before submission, or parts of this work have been published by :\\
\\
Stefan Collier\\
April 2016
}
\tableofcontents
\listoffigures
\listoftables

\mainmatter
%% ----------------------------------------------------------------
%\include{Introduction}
%\include{Conclusions}
\include{chapters/1Project/main}
\include{chapters/2Lit/main}
\include{chapters/3Design/HighLevel}
\include{chapters/3Design/InDepth}
\include{chapters/4Impl/main}

\include{chapters/5Experiments/1/main}
\include{chapters/5Experiments/2/main}
\include{chapters/5Experiments/3/main}
\include{chapters/5Experiments/4/main}

\include{chapters/6Conclusion/main}

\appendix
\include{appendix/AppendixB}
\include{appendix/D/main}
\include{appendix/AppendixC}

\backmatter
\bibliographystyle{ecs}
\bibliography{ECS}
\end{document}
%% ----------------------------------------------------------------

 %% ----------------------------------------------------------------
%% Progress.tex
%% ---------------------------------------------------------------- 
\documentclass{ecsprogress}    % Use the progress Style
\graphicspath{{../figs/}}   % Location of your graphics files
    \usepackage{natbib}            % Use Natbib style for the refs.
\hypersetup{colorlinks=true}   % Set to false for black/white printing
\input{Definitions}            % Include your abbreviations



\usepackage{enumitem}% http://ctan.org/pkg/enumitem
\usepackage{multirow}
\usepackage{float}
\usepackage{amsmath}
\usepackage{multicol}
\usepackage{amssymb}
\usepackage[normalem]{ulem}
\useunder{\uline}{\ul}{}
\usepackage{wrapfig}


\usepackage[table,xcdraw]{xcolor}


%% ----------------------------------------------------------------
\begin{document}
\frontmatter
\title      {Heterogeneous Agent-based Model for Supermarket Competition}
\authors    {\texorpdfstring
             {\href{mailto:sc22g13@ecs.soton.ac.uk}{Stefan J. Collier}}
             {Stefan J. Collier}
            }
\addresses  {\groupname\\\deptname\\\univname}
\date       {\today}
\subject    {}
\keywords   {}
\supervisor {Dr. Maria Polukarov}
\examiner   {Professor Sheng Chen}

\maketitle
\begin{abstract}
This project aim was to model and analyse the effects of competitive pricing behaviors of grocery retailers on the British market. 

This was achieved by creating a multi-agent model, containing retailer and consumer agents. The heterogeneous crowd of retailers employs either a uniform pricing strategy or a ‘local price flexing’ strategy. The actions of these retailers are chosen by predicting the profit of each action, using a perceptron. Following on from the consideration of different economic models, a discrete model was developed so that software agents have a discrete environment to operate within. Within the model, it has been observed how supermarkets with differing behaviors affect a heterogeneous crowd of consumer agents. The model was implemented in Java with Python used to evaluate the results. 

The simulation displays good acceptance with real grocery market behavior, i.e. captures the performance of British retailers thus can be used to determine the impact of changes in their behavior on their competitors and consumers.Furthermore it can be used to provide insight into sustainability of volatile pricing strategies, providing a useful insight in volatility of British supermarket retail industry. 
\end{abstract}
\acknowledgements{
I would like to express my sincere gratitude to Dr Maria Polukarov for her guidance and support which provided me the freedom to take this research in the direction of my interest.\\
\\
I would also like to thank my family and friends for their encouragement and support. To those who quietly listened to my software complaints. To those who worked throughout the nights with me. To those who helped me write what I couldn't say. I cannot thank you enough.
}

\declaration{
I, Stefan Collier, declare that this dissertation and the work presented in it are my own and has been generated by me as the result of my own original research.\\
I confirm that:\\
1. This work was done wholly or mainly while in candidature for a degree at this University;\\
2. Where any part of this dissertation has previously been submitted for any other qualification at this University or any other institution, this has been clearly stated;\\
3. Where I have consulted the published work of others, this is always clearly attributed;\\
4. Where I have quoted from the work of others, the source is always given. With the exception of such quotations, this dissertation is entirely my own work;\\
5. I have acknowledged all main sources of help;\\
6. Where the thesis is based on work done by myself jointly with others, I have made clear exactly what was done by others and what I have contributed myself;\\
7. Either none of this work has been published before submission, or parts of this work have been published by :\\
\\
Stefan Collier\\
April 2016
}
\tableofcontents
\listoffigures
\listoftables

\mainmatter
%% ----------------------------------------------------------------
%\include{Introduction}
%\include{Conclusions}
\include{chapters/1Project/main}
\include{chapters/2Lit/main}
\include{chapters/3Design/HighLevel}
\include{chapters/3Design/InDepth}
\include{chapters/4Impl/main}

\include{chapters/5Experiments/1/main}
\include{chapters/5Experiments/2/main}
\include{chapters/5Experiments/3/main}
\include{chapters/5Experiments/4/main}

\include{chapters/6Conclusion/main}

\appendix
\include{appendix/AppendixB}
\include{appendix/D/main}
\include{appendix/AppendixC}

\backmatter
\bibliographystyle{ecs}
\bibliography{ECS}
\end{document}
%% ----------------------------------------------------------------

 %% ----------------------------------------------------------------
%% Progress.tex
%% ---------------------------------------------------------------- 
\documentclass{ecsprogress}    % Use the progress Style
\graphicspath{{../figs/}}   % Location of your graphics files
    \usepackage{natbib}            % Use Natbib style for the refs.
\hypersetup{colorlinks=true}   % Set to false for black/white printing
\input{Definitions}            % Include your abbreviations



\usepackage{enumitem}% http://ctan.org/pkg/enumitem
\usepackage{multirow}
\usepackage{float}
\usepackage{amsmath}
\usepackage{multicol}
\usepackage{amssymb}
\usepackage[normalem]{ulem}
\useunder{\uline}{\ul}{}
\usepackage{wrapfig}


\usepackage[table,xcdraw]{xcolor}


%% ----------------------------------------------------------------
\begin{document}
\frontmatter
\title      {Heterogeneous Agent-based Model for Supermarket Competition}
\authors    {\texorpdfstring
             {\href{mailto:sc22g13@ecs.soton.ac.uk}{Stefan J. Collier}}
             {Stefan J. Collier}
            }
\addresses  {\groupname\\\deptname\\\univname}
\date       {\today}
\subject    {}
\keywords   {}
\supervisor {Dr. Maria Polukarov}
\examiner   {Professor Sheng Chen}

\maketitle
\begin{abstract}
This project aim was to model and analyse the effects of competitive pricing behaviors of grocery retailers on the British market. 

This was achieved by creating a multi-agent model, containing retailer and consumer agents. The heterogeneous crowd of retailers employs either a uniform pricing strategy or a ‘local price flexing’ strategy. The actions of these retailers are chosen by predicting the profit of each action, using a perceptron. Following on from the consideration of different economic models, a discrete model was developed so that software agents have a discrete environment to operate within. Within the model, it has been observed how supermarkets with differing behaviors affect a heterogeneous crowd of consumer agents. The model was implemented in Java with Python used to evaluate the results. 

The simulation displays good acceptance with real grocery market behavior, i.e. captures the performance of British retailers thus can be used to determine the impact of changes in their behavior on their competitors and consumers.Furthermore it can be used to provide insight into sustainability of volatile pricing strategies, providing a useful insight in volatility of British supermarket retail industry. 
\end{abstract}
\acknowledgements{
I would like to express my sincere gratitude to Dr Maria Polukarov for her guidance and support which provided me the freedom to take this research in the direction of my interest.\\
\\
I would also like to thank my family and friends for their encouragement and support. To those who quietly listened to my software complaints. To those who worked throughout the nights with me. To those who helped me write what I couldn't say. I cannot thank you enough.
}

\declaration{
I, Stefan Collier, declare that this dissertation and the work presented in it are my own and has been generated by me as the result of my own original research.\\
I confirm that:\\
1. This work was done wholly or mainly while in candidature for a degree at this University;\\
2. Where any part of this dissertation has previously been submitted for any other qualification at this University or any other institution, this has been clearly stated;\\
3. Where I have consulted the published work of others, this is always clearly attributed;\\
4. Where I have quoted from the work of others, the source is always given. With the exception of such quotations, this dissertation is entirely my own work;\\
5. I have acknowledged all main sources of help;\\
6. Where the thesis is based on work done by myself jointly with others, I have made clear exactly what was done by others and what I have contributed myself;\\
7. Either none of this work has been published before submission, or parts of this work have been published by :\\
\\
Stefan Collier\\
April 2016
}
\tableofcontents
\listoffigures
\listoftables

\mainmatter
%% ----------------------------------------------------------------
%\include{Introduction}
%\include{Conclusions}
\include{chapters/1Project/main}
\include{chapters/2Lit/main}
\include{chapters/3Design/HighLevel}
\include{chapters/3Design/InDepth}
\include{chapters/4Impl/main}

\include{chapters/5Experiments/1/main}
\include{chapters/5Experiments/2/main}
\include{chapters/5Experiments/3/main}
\include{chapters/5Experiments/4/main}

\include{chapters/6Conclusion/main}

\appendix
\include{appendix/AppendixB}
\include{appendix/D/main}
\include{appendix/AppendixC}

\backmatter
\bibliographystyle{ecs}
\bibliography{ECS}
\end{document}
%% ----------------------------------------------------------------


 %% ----------------------------------------------------------------
%% Progress.tex
%% ---------------------------------------------------------------- 
\documentclass{ecsprogress}    % Use the progress Style
\graphicspath{{../figs/}}   % Location of your graphics files
    \usepackage{natbib}            % Use Natbib style for the refs.
\hypersetup{colorlinks=true}   % Set to false for black/white printing
\input{Definitions}            % Include your abbreviations



\usepackage{enumitem}% http://ctan.org/pkg/enumitem
\usepackage{multirow}
\usepackage{float}
\usepackage{amsmath}
\usepackage{multicol}
\usepackage{amssymb}
\usepackage[normalem]{ulem}
\useunder{\uline}{\ul}{}
\usepackage{wrapfig}


\usepackage[table,xcdraw]{xcolor}


%% ----------------------------------------------------------------
\begin{document}
\frontmatter
\title      {Heterogeneous Agent-based Model for Supermarket Competition}
\authors    {\texorpdfstring
             {\href{mailto:sc22g13@ecs.soton.ac.uk}{Stefan J. Collier}}
             {Stefan J. Collier}
            }
\addresses  {\groupname\\\deptname\\\univname}
\date       {\today}
\subject    {}
\keywords   {}
\supervisor {Dr. Maria Polukarov}
\examiner   {Professor Sheng Chen}

\maketitle
\begin{abstract}
This project aim was to model and analyse the effects of competitive pricing behaviors of grocery retailers on the British market. 

This was achieved by creating a multi-agent model, containing retailer and consumer agents. The heterogeneous crowd of retailers employs either a uniform pricing strategy or a ‘local price flexing’ strategy. The actions of these retailers are chosen by predicting the profit of each action, using a perceptron. Following on from the consideration of different economic models, a discrete model was developed so that software agents have a discrete environment to operate within. Within the model, it has been observed how supermarkets with differing behaviors affect a heterogeneous crowd of consumer agents. The model was implemented in Java with Python used to evaluate the results. 

The simulation displays good acceptance with real grocery market behavior, i.e. captures the performance of British retailers thus can be used to determine the impact of changes in their behavior on their competitors and consumers.Furthermore it can be used to provide insight into sustainability of volatile pricing strategies, providing a useful insight in volatility of British supermarket retail industry. 
\end{abstract}
\acknowledgements{
I would like to express my sincere gratitude to Dr Maria Polukarov for her guidance and support which provided me the freedom to take this research in the direction of my interest.\\
\\
I would also like to thank my family and friends for their encouragement and support. To those who quietly listened to my software complaints. To those who worked throughout the nights with me. To those who helped me write what I couldn't say. I cannot thank you enough.
}

\declaration{
I, Stefan Collier, declare that this dissertation and the work presented in it are my own and has been generated by me as the result of my own original research.\\
I confirm that:\\
1. This work was done wholly or mainly while in candidature for a degree at this University;\\
2. Where any part of this dissertation has previously been submitted for any other qualification at this University or any other institution, this has been clearly stated;\\
3. Where I have consulted the published work of others, this is always clearly attributed;\\
4. Where I have quoted from the work of others, the source is always given. With the exception of such quotations, this dissertation is entirely my own work;\\
5. I have acknowledged all main sources of help;\\
6. Where the thesis is based on work done by myself jointly with others, I have made clear exactly what was done by others and what I have contributed myself;\\
7. Either none of this work has been published before submission, or parts of this work have been published by :\\
\\
Stefan Collier\\
April 2016
}
\tableofcontents
\listoffigures
\listoftables

\mainmatter
%% ----------------------------------------------------------------
%\include{Introduction}
%\include{Conclusions}
\include{chapters/1Project/main}
\include{chapters/2Lit/main}
\include{chapters/3Design/HighLevel}
\include{chapters/3Design/InDepth}
\include{chapters/4Impl/main}

\include{chapters/5Experiments/1/main}
\include{chapters/5Experiments/2/main}
\include{chapters/5Experiments/3/main}
\include{chapters/5Experiments/4/main}

\include{chapters/6Conclusion/main}

\appendix
\include{appendix/AppendixB}
\include{appendix/D/main}
\include{appendix/AppendixC}

\backmatter
\bibliographystyle{ecs}
\bibliography{ECS}
\end{document}
%% ----------------------------------------------------------------


\appendix
\include{appendix/AppendixB}
 %% ----------------------------------------------------------------
%% Progress.tex
%% ---------------------------------------------------------------- 
\documentclass{ecsprogress}    % Use the progress Style
\graphicspath{{../figs/}}   % Location of your graphics files
    \usepackage{natbib}            % Use Natbib style for the refs.
\hypersetup{colorlinks=true}   % Set to false for black/white printing
\input{Definitions}            % Include your abbreviations



\usepackage{enumitem}% http://ctan.org/pkg/enumitem
\usepackage{multirow}
\usepackage{float}
\usepackage{amsmath}
\usepackage{multicol}
\usepackage{amssymb}
\usepackage[normalem]{ulem}
\useunder{\uline}{\ul}{}
\usepackage{wrapfig}


\usepackage[table,xcdraw]{xcolor}


%% ----------------------------------------------------------------
\begin{document}
\frontmatter
\title      {Heterogeneous Agent-based Model for Supermarket Competition}
\authors    {\texorpdfstring
             {\href{mailto:sc22g13@ecs.soton.ac.uk}{Stefan J. Collier}}
             {Stefan J. Collier}
            }
\addresses  {\groupname\\\deptname\\\univname}
\date       {\today}
\subject    {}
\keywords   {}
\supervisor {Dr. Maria Polukarov}
\examiner   {Professor Sheng Chen}

\maketitle
\begin{abstract}
This project aim was to model and analyse the effects of competitive pricing behaviors of grocery retailers on the British market. 

This was achieved by creating a multi-agent model, containing retailer and consumer agents. The heterogeneous crowd of retailers employs either a uniform pricing strategy or a ‘local price flexing’ strategy. The actions of these retailers are chosen by predicting the profit of each action, using a perceptron. Following on from the consideration of different economic models, a discrete model was developed so that software agents have a discrete environment to operate within. Within the model, it has been observed how supermarkets with differing behaviors affect a heterogeneous crowd of consumer agents. The model was implemented in Java with Python used to evaluate the results. 

The simulation displays good acceptance with real grocery market behavior, i.e. captures the performance of British retailers thus can be used to determine the impact of changes in their behavior on their competitors and consumers.Furthermore it can be used to provide insight into sustainability of volatile pricing strategies, providing a useful insight in volatility of British supermarket retail industry. 
\end{abstract}
\acknowledgements{
I would like to express my sincere gratitude to Dr Maria Polukarov for her guidance and support which provided me the freedom to take this research in the direction of my interest.\\
\\
I would also like to thank my family and friends for their encouragement and support. To those who quietly listened to my software complaints. To those who worked throughout the nights with me. To those who helped me write what I couldn't say. I cannot thank you enough.
}

\declaration{
I, Stefan Collier, declare that this dissertation and the work presented in it are my own and has been generated by me as the result of my own original research.\\
I confirm that:\\
1. This work was done wholly or mainly while in candidature for a degree at this University;\\
2. Where any part of this dissertation has previously been submitted for any other qualification at this University or any other institution, this has been clearly stated;\\
3. Where I have consulted the published work of others, this is always clearly attributed;\\
4. Where I have quoted from the work of others, the source is always given. With the exception of such quotations, this dissertation is entirely my own work;\\
5. I have acknowledged all main sources of help;\\
6. Where the thesis is based on work done by myself jointly with others, I have made clear exactly what was done by others and what I have contributed myself;\\
7. Either none of this work has been published before submission, or parts of this work have been published by :\\
\\
Stefan Collier\\
April 2016
}
\tableofcontents
\listoffigures
\listoftables

\mainmatter
%% ----------------------------------------------------------------
%\include{Introduction}
%\include{Conclusions}
\include{chapters/1Project/main}
\include{chapters/2Lit/main}
\include{chapters/3Design/HighLevel}
\include{chapters/3Design/InDepth}
\include{chapters/4Impl/main}

\include{chapters/5Experiments/1/main}
\include{chapters/5Experiments/2/main}
\include{chapters/5Experiments/3/main}
\include{chapters/5Experiments/4/main}

\include{chapters/6Conclusion/main}

\appendix
\include{appendix/AppendixB}
\include{appendix/D/main}
\include{appendix/AppendixC}

\backmatter
\bibliographystyle{ecs}
\bibliography{ECS}
\end{document}
%% ----------------------------------------------------------------

\include{appendix/AppendixC}

\backmatter
\bibliographystyle{ecs}
\bibliography{ECS}
\end{document}
%% ----------------------------------------------------------------

 %% ----------------------------------------------------------------
%% Progress.tex
%% ---------------------------------------------------------------- 
\documentclass{ecsprogress}    % Use the progress Style
\graphicspath{{../figs/}}   % Location of your graphics files
    \usepackage{natbib}            % Use Natbib style for the refs.
\hypersetup{colorlinks=true}   % Set to false for black/white printing
\input{Definitions}            % Include your abbreviations



\usepackage{enumitem}% http://ctan.org/pkg/enumitem
\usepackage{multirow}
\usepackage{float}
\usepackage{amsmath}
\usepackage{multicol}
\usepackage{amssymb}
\usepackage[normalem]{ulem}
\useunder{\uline}{\ul}{}
\usepackage{wrapfig}


\usepackage[table,xcdraw]{xcolor}


%% ----------------------------------------------------------------
\begin{document}
\frontmatter
\title      {Heterogeneous Agent-based Model for Supermarket Competition}
\authors    {\texorpdfstring
             {\href{mailto:sc22g13@ecs.soton.ac.uk}{Stefan J. Collier}}
             {Stefan J. Collier}
            }
\addresses  {\groupname\\\deptname\\\univname}
\date       {\today}
\subject    {}
\keywords   {}
\supervisor {Dr. Maria Polukarov}
\examiner   {Professor Sheng Chen}

\maketitle
\begin{abstract}
This project aim was to model and analyse the effects of competitive pricing behaviors of grocery retailers on the British market. 

This was achieved by creating a multi-agent model, containing retailer and consumer agents. The heterogeneous crowd of retailers employs either a uniform pricing strategy or a ‘local price flexing’ strategy. The actions of these retailers are chosen by predicting the profit of each action, using a perceptron. Following on from the consideration of different economic models, a discrete model was developed so that software agents have a discrete environment to operate within. Within the model, it has been observed how supermarkets with differing behaviors affect a heterogeneous crowd of consumer agents. The model was implemented in Java with Python used to evaluate the results. 

The simulation displays good acceptance with real grocery market behavior, i.e. captures the performance of British retailers thus can be used to determine the impact of changes in their behavior on their competitors and consumers.Furthermore it can be used to provide insight into sustainability of volatile pricing strategies, providing a useful insight in volatility of British supermarket retail industry. 
\end{abstract}
\acknowledgements{
I would like to express my sincere gratitude to Dr Maria Polukarov for her guidance and support which provided me the freedom to take this research in the direction of my interest.\\
\\
I would also like to thank my family and friends for their encouragement and support. To those who quietly listened to my software complaints. To those who worked throughout the nights with me. To those who helped me write what I couldn't say. I cannot thank you enough.
}

\declaration{
I, Stefan Collier, declare that this dissertation and the work presented in it are my own and has been generated by me as the result of my own original research.\\
I confirm that:\\
1. This work was done wholly or mainly while in candidature for a degree at this University;\\
2. Where any part of this dissertation has previously been submitted for any other qualification at this University or any other institution, this has been clearly stated;\\
3. Where I have consulted the published work of others, this is always clearly attributed;\\
4. Where I have quoted from the work of others, the source is always given. With the exception of such quotations, this dissertation is entirely my own work;\\
5. I have acknowledged all main sources of help;\\
6. Where the thesis is based on work done by myself jointly with others, I have made clear exactly what was done by others and what I have contributed myself;\\
7. Either none of this work has been published before submission, or parts of this work have been published by :\\
\\
Stefan Collier\\
April 2016
}
\tableofcontents
\listoffigures
\listoftables

\mainmatter
%% ----------------------------------------------------------------
%\include{Introduction}
%\include{Conclusions}
 %% ----------------------------------------------------------------
%% Progress.tex
%% ---------------------------------------------------------------- 
\documentclass{ecsprogress}    % Use the progress Style
\graphicspath{{../figs/}}   % Location of your graphics files
    \usepackage{natbib}            % Use Natbib style for the refs.
\hypersetup{colorlinks=true}   % Set to false for black/white printing
\input{Definitions}            % Include your abbreviations



\usepackage{enumitem}% http://ctan.org/pkg/enumitem
\usepackage{multirow}
\usepackage{float}
\usepackage{amsmath}
\usepackage{multicol}
\usepackage{amssymb}
\usepackage[normalem]{ulem}
\useunder{\uline}{\ul}{}
\usepackage{wrapfig}


\usepackage[table,xcdraw]{xcolor}


%% ----------------------------------------------------------------
\begin{document}
\frontmatter
\title      {Heterogeneous Agent-based Model for Supermarket Competition}
\authors    {\texorpdfstring
             {\href{mailto:sc22g13@ecs.soton.ac.uk}{Stefan J. Collier}}
             {Stefan J. Collier}
            }
\addresses  {\groupname\\\deptname\\\univname}
\date       {\today}
\subject    {}
\keywords   {}
\supervisor {Dr. Maria Polukarov}
\examiner   {Professor Sheng Chen}

\maketitle
\begin{abstract}
This project aim was to model and analyse the effects of competitive pricing behaviors of grocery retailers on the British market. 

This was achieved by creating a multi-agent model, containing retailer and consumer agents. The heterogeneous crowd of retailers employs either a uniform pricing strategy or a ‘local price flexing’ strategy. The actions of these retailers are chosen by predicting the profit of each action, using a perceptron. Following on from the consideration of different economic models, a discrete model was developed so that software agents have a discrete environment to operate within. Within the model, it has been observed how supermarkets with differing behaviors affect a heterogeneous crowd of consumer agents. The model was implemented in Java with Python used to evaluate the results. 

The simulation displays good acceptance with real grocery market behavior, i.e. captures the performance of British retailers thus can be used to determine the impact of changes in their behavior on their competitors and consumers.Furthermore it can be used to provide insight into sustainability of volatile pricing strategies, providing a useful insight in volatility of British supermarket retail industry. 
\end{abstract}
\acknowledgements{
I would like to express my sincere gratitude to Dr Maria Polukarov for her guidance and support which provided me the freedom to take this research in the direction of my interest.\\
\\
I would also like to thank my family and friends for their encouragement and support. To those who quietly listened to my software complaints. To those who worked throughout the nights with me. To those who helped me write what I couldn't say. I cannot thank you enough.
}

\declaration{
I, Stefan Collier, declare that this dissertation and the work presented in it are my own and has been generated by me as the result of my own original research.\\
I confirm that:\\
1. This work was done wholly or mainly while in candidature for a degree at this University;\\
2. Where any part of this dissertation has previously been submitted for any other qualification at this University or any other institution, this has been clearly stated;\\
3. Where I have consulted the published work of others, this is always clearly attributed;\\
4. Where I have quoted from the work of others, the source is always given. With the exception of such quotations, this dissertation is entirely my own work;\\
5. I have acknowledged all main sources of help;\\
6. Where the thesis is based on work done by myself jointly with others, I have made clear exactly what was done by others and what I have contributed myself;\\
7. Either none of this work has been published before submission, or parts of this work have been published by :\\
\\
Stefan Collier\\
April 2016
}
\tableofcontents
\listoffigures
\listoftables

\mainmatter
%% ----------------------------------------------------------------
%\include{Introduction}
%\include{Conclusions}
\include{chapters/1Project/main}
\include{chapters/2Lit/main}
\include{chapters/3Design/HighLevel}
\include{chapters/3Design/InDepth}
\include{chapters/4Impl/main}

\include{chapters/5Experiments/1/main}
\include{chapters/5Experiments/2/main}
\include{chapters/5Experiments/3/main}
\include{chapters/5Experiments/4/main}

\include{chapters/6Conclusion/main}

\appendix
\include{appendix/AppendixB}
\include{appendix/D/main}
\include{appendix/AppendixC}

\backmatter
\bibliographystyle{ecs}
\bibliography{ECS}
\end{document}
%% ----------------------------------------------------------------

 %% ----------------------------------------------------------------
%% Progress.tex
%% ---------------------------------------------------------------- 
\documentclass{ecsprogress}    % Use the progress Style
\graphicspath{{../figs/}}   % Location of your graphics files
    \usepackage{natbib}            % Use Natbib style for the refs.
\hypersetup{colorlinks=true}   % Set to false for black/white printing
\input{Definitions}            % Include your abbreviations



\usepackage{enumitem}% http://ctan.org/pkg/enumitem
\usepackage{multirow}
\usepackage{float}
\usepackage{amsmath}
\usepackage{multicol}
\usepackage{amssymb}
\usepackage[normalem]{ulem}
\useunder{\uline}{\ul}{}
\usepackage{wrapfig}


\usepackage[table,xcdraw]{xcolor}


%% ----------------------------------------------------------------
\begin{document}
\frontmatter
\title      {Heterogeneous Agent-based Model for Supermarket Competition}
\authors    {\texorpdfstring
             {\href{mailto:sc22g13@ecs.soton.ac.uk}{Stefan J. Collier}}
             {Stefan J. Collier}
            }
\addresses  {\groupname\\\deptname\\\univname}
\date       {\today}
\subject    {}
\keywords   {}
\supervisor {Dr. Maria Polukarov}
\examiner   {Professor Sheng Chen}

\maketitle
\begin{abstract}
This project aim was to model and analyse the effects of competitive pricing behaviors of grocery retailers on the British market. 

This was achieved by creating a multi-agent model, containing retailer and consumer agents. The heterogeneous crowd of retailers employs either a uniform pricing strategy or a ‘local price flexing’ strategy. The actions of these retailers are chosen by predicting the profit of each action, using a perceptron. Following on from the consideration of different economic models, a discrete model was developed so that software agents have a discrete environment to operate within. Within the model, it has been observed how supermarkets with differing behaviors affect a heterogeneous crowd of consumer agents. The model was implemented in Java with Python used to evaluate the results. 

The simulation displays good acceptance with real grocery market behavior, i.e. captures the performance of British retailers thus can be used to determine the impact of changes in their behavior on their competitors and consumers.Furthermore it can be used to provide insight into sustainability of volatile pricing strategies, providing a useful insight in volatility of British supermarket retail industry. 
\end{abstract}
\acknowledgements{
I would like to express my sincere gratitude to Dr Maria Polukarov for her guidance and support which provided me the freedom to take this research in the direction of my interest.\\
\\
I would also like to thank my family and friends for their encouragement and support. To those who quietly listened to my software complaints. To those who worked throughout the nights with me. To those who helped me write what I couldn't say. I cannot thank you enough.
}

\declaration{
I, Stefan Collier, declare that this dissertation and the work presented in it are my own and has been generated by me as the result of my own original research.\\
I confirm that:\\
1. This work was done wholly or mainly while in candidature for a degree at this University;\\
2. Where any part of this dissertation has previously been submitted for any other qualification at this University or any other institution, this has been clearly stated;\\
3. Where I have consulted the published work of others, this is always clearly attributed;\\
4. Where I have quoted from the work of others, the source is always given. With the exception of such quotations, this dissertation is entirely my own work;\\
5. I have acknowledged all main sources of help;\\
6. Where the thesis is based on work done by myself jointly with others, I have made clear exactly what was done by others and what I have contributed myself;\\
7. Either none of this work has been published before submission, or parts of this work have been published by :\\
\\
Stefan Collier\\
April 2016
}
\tableofcontents
\listoffigures
\listoftables

\mainmatter
%% ----------------------------------------------------------------
%\include{Introduction}
%\include{Conclusions}
\include{chapters/1Project/main}
\include{chapters/2Lit/main}
\include{chapters/3Design/HighLevel}
\include{chapters/3Design/InDepth}
\include{chapters/4Impl/main}

\include{chapters/5Experiments/1/main}
\include{chapters/5Experiments/2/main}
\include{chapters/5Experiments/3/main}
\include{chapters/5Experiments/4/main}

\include{chapters/6Conclusion/main}

\appendix
\include{appendix/AppendixB}
\include{appendix/D/main}
\include{appendix/AppendixC}

\backmatter
\bibliographystyle{ecs}
\bibliography{ECS}
\end{document}
%% ----------------------------------------------------------------

\include{chapters/3Design/HighLevel}
\include{chapters/3Design/InDepth}
 %% ----------------------------------------------------------------
%% Progress.tex
%% ---------------------------------------------------------------- 
\documentclass{ecsprogress}    % Use the progress Style
\graphicspath{{../figs/}}   % Location of your graphics files
    \usepackage{natbib}            % Use Natbib style for the refs.
\hypersetup{colorlinks=true}   % Set to false for black/white printing
\input{Definitions}            % Include your abbreviations



\usepackage{enumitem}% http://ctan.org/pkg/enumitem
\usepackage{multirow}
\usepackage{float}
\usepackage{amsmath}
\usepackage{multicol}
\usepackage{amssymb}
\usepackage[normalem]{ulem}
\useunder{\uline}{\ul}{}
\usepackage{wrapfig}


\usepackage[table,xcdraw]{xcolor}


%% ----------------------------------------------------------------
\begin{document}
\frontmatter
\title      {Heterogeneous Agent-based Model for Supermarket Competition}
\authors    {\texorpdfstring
             {\href{mailto:sc22g13@ecs.soton.ac.uk}{Stefan J. Collier}}
             {Stefan J. Collier}
            }
\addresses  {\groupname\\\deptname\\\univname}
\date       {\today}
\subject    {}
\keywords   {}
\supervisor {Dr. Maria Polukarov}
\examiner   {Professor Sheng Chen}

\maketitle
\begin{abstract}
This project aim was to model and analyse the effects of competitive pricing behaviors of grocery retailers on the British market. 

This was achieved by creating a multi-agent model, containing retailer and consumer agents. The heterogeneous crowd of retailers employs either a uniform pricing strategy or a ‘local price flexing’ strategy. The actions of these retailers are chosen by predicting the profit of each action, using a perceptron. Following on from the consideration of different economic models, a discrete model was developed so that software agents have a discrete environment to operate within. Within the model, it has been observed how supermarkets with differing behaviors affect a heterogeneous crowd of consumer agents. The model was implemented in Java with Python used to evaluate the results. 

The simulation displays good acceptance with real grocery market behavior, i.e. captures the performance of British retailers thus can be used to determine the impact of changes in their behavior on their competitors and consumers.Furthermore it can be used to provide insight into sustainability of volatile pricing strategies, providing a useful insight in volatility of British supermarket retail industry. 
\end{abstract}
\acknowledgements{
I would like to express my sincere gratitude to Dr Maria Polukarov for her guidance and support which provided me the freedom to take this research in the direction of my interest.\\
\\
I would also like to thank my family and friends for their encouragement and support. To those who quietly listened to my software complaints. To those who worked throughout the nights with me. To those who helped me write what I couldn't say. I cannot thank you enough.
}

\declaration{
I, Stefan Collier, declare that this dissertation and the work presented in it are my own and has been generated by me as the result of my own original research.\\
I confirm that:\\
1. This work was done wholly or mainly while in candidature for a degree at this University;\\
2. Where any part of this dissertation has previously been submitted for any other qualification at this University or any other institution, this has been clearly stated;\\
3. Where I have consulted the published work of others, this is always clearly attributed;\\
4. Where I have quoted from the work of others, the source is always given. With the exception of such quotations, this dissertation is entirely my own work;\\
5. I have acknowledged all main sources of help;\\
6. Where the thesis is based on work done by myself jointly with others, I have made clear exactly what was done by others and what I have contributed myself;\\
7. Either none of this work has been published before submission, or parts of this work have been published by :\\
\\
Stefan Collier\\
April 2016
}
\tableofcontents
\listoffigures
\listoftables

\mainmatter
%% ----------------------------------------------------------------
%\include{Introduction}
%\include{Conclusions}
\include{chapters/1Project/main}
\include{chapters/2Lit/main}
\include{chapters/3Design/HighLevel}
\include{chapters/3Design/InDepth}
\include{chapters/4Impl/main}

\include{chapters/5Experiments/1/main}
\include{chapters/5Experiments/2/main}
\include{chapters/5Experiments/3/main}
\include{chapters/5Experiments/4/main}

\include{chapters/6Conclusion/main}

\appendix
\include{appendix/AppendixB}
\include{appendix/D/main}
\include{appendix/AppendixC}

\backmatter
\bibliographystyle{ecs}
\bibliography{ECS}
\end{document}
%% ----------------------------------------------------------------


 %% ----------------------------------------------------------------
%% Progress.tex
%% ---------------------------------------------------------------- 
\documentclass{ecsprogress}    % Use the progress Style
\graphicspath{{../figs/}}   % Location of your graphics files
    \usepackage{natbib}            % Use Natbib style for the refs.
\hypersetup{colorlinks=true}   % Set to false for black/white printing
\input{Definitions}            % Include your abbreviations



\usepackage{enumitem}% http://ctan.org/pkg/enumitem
\usepackage{multirow}
\usepackage{float}
\usepackage{amsmath}
\usepackage{multicol}
\usepackage{amssymb}
\usepackage[normalem]{ulem}
\useunder{\uline}{\ul}{}
\usepackage{wrapfig}


\usepackage[table,xcdraw]{xcolor}


%% ----------------------------------------------------------------
\begin{document}
\frontmatter
\title      {Heterogeneous Agent-based Model for Supermarket Competition}
\authors    {\texorpdfstring
             {\href{mailto:sc22g13@ecs.soton.ac.uk}{Stefan J. Collier}}
             {Stefan J. Collier}
            }
\addresses  {\groupname\\\deptname\\\univname}
\date       {\today}
\subject    {}
\keywords   {}
\supervisor {Dr. Maria Polukarov}
\examiner   {Professor Sheng Chen}

\maketitle
\begin{abstract}
This project aim was to model and analyse the effects of competitive pricing behaviors of grocery retailers on the British market. 

This was achieved by creating a multi-agent model, containing retailer and consumer agents. The heterogeneous crowd of retailers employs either a uniform pricing strategy or a ‘local price flexing’ strategy. The actions of these retailers are chosen by predicting the profit of each action, using a perceptron. Following on from the consideration of different economic models, a discrete model was developed so that software agents have a discrete environment to operate within. Within the model, it has been observed how supermarkets with differing behaviors affect a heterogeneous crowd of consumer agents. The model was implemented in Java with Python used to evaluate the results. 

The simulation displays good acceptance with real grocery market behavior, i.e. captures the performance of British retailers thus can be used to determine the impact of changes in their behavior on their competitors and consumers.Furthermore it can be used to provide insight into sustainability of volatile pricing strategies, providing a useful insight in volatility of British supermarket retail industry. 
\end{abstract}
\acknowledgements{
I would like to express my sincere gratitude to Dr Maria Polukarov for her guidance and support which provided me the freedom to take this research in the direction of my interest.\\
\\
I would also like to thank my family and friends for their encouragement and support. To those who quietly listened to my software complaints. To those who worked throughout the nights with me. To those who helped me write what I couldn't say. I cannot thank you enough.
}

\declaration{
I, Stefan Collier, declare that this dissertation and the work presented in it are my own and has been generated by me as the result of my own original research.\\
I confirm that:\\
1. This work was done wholly or mainly while in candidature for a degree at this University;\\
2. Where any part of this dissertation has previously been submitted for any other qualification at this University or any other institution, this has been clearly stated;\\
3. Where I have consulted the published work of others, this is always clearly attributed;\\
4. Where I have quoted from the work of others, the source is always given. With the exception of such quotations, this dissertation is entirely my own work;\\
5. I have acknowledged all main sources of help;\\
6. Where the thesis is based on work done by myself jointly with others, I have made clear exactly what was done by others and what I have contributed myself;\\
7. Either none of this work has been published before submission, or parts of this work have been published by :\\
\\
Stefan Collier\\
April 2016
}
\tableofcontents
\listoffigures
\listoftables

\mainmatter
%% ----------------------------------------------------------------
%\include{Introduction}
%\include{Conclusions}
\include{chapters/1Project/main}
\include{chapters/2Lit/main}
\include{chapters/3Design/HighLevel}
\include{chapters/3Design/InDepth}
\include{chapters/4Impl/main}

\include{chapters/5Experiments/1/main}
\include{chapters/5Experiments/2/main}
\include{chapters/5Experiments/3/main}
\include{chapters/5Experiments/4/main}

\include{chapters/6Conclusion/main}

\appendix
\include{appendix/AppendixB}
\include{appendix/D/main}
\include{appendix/AppendixC}

\backmatter
\bibliographystyle{ecs}
\bibliography{ECS}
\end{document}
%% ----------------------------------------------------------------

 %% ----------------------------------------------------------------
%% Progress.tex
%% ---------------------------------------------------------------- 
\documentclass{ecsprogress}    % Use the progress Style
\graphicspath{{../figs/}}   % Location of your graphics files
    \usepackage{natbib}            % Use Natbib style for the refs.
\hypersetup{colorlinks=true}   % Set to false for black/white printing
\input{Definitions}            % Include your abbreviations



\usepackage{enumitem}% http://ctan.org/pkg/enumitem
\usepackage{multirow}
\usepackage{float}
\usepackage{amsmath}
\usepackage{multicol}
\usepackage{amssymb}
\usepackage[normalem]{ulem}
\useunder{\uline}{\ul}{}
\usepackage{wrapfig}


\usepackage[table,xcdraw]{xcolor}


%% ----------------------------------------------------------------
\begin{document}
\frontmatter
\title      {Heterogeneous Agent-based Model for Supermarket Competition}
\authors    {\texorpdfstring
             {\href{mailto:sc22g13@ecs.soton.ac.uk}{Stefan J. Collier}}
             {Stefan J. Collier}
            }
\addresses  {\groupname\\\deptname\\\univname}
\date       {\today}
\subject    {}
\keywords   {}
\supervisor {Dr. Maria Polukarov}
\examiner   {Professor Sheng Chen}

\maketitle
\begin{abstract}
This project aim was to model and analyse the effects of competitive pricing behaviors of grocery retailers on the British market. 

This was achieved by creating a multi-agent model, containing retailer and consumer agents. The heterogeneous crowd of retailers employs either a uniform pricing strategy or a ‘local price flexing’ strategy. The actions of these retailers are chosen by predicting the profit of each action, using a perceptron. Following on from the consideration of different economic models, a discrete model was developed so that software agents have a discrete environment to operate within. Within the model, it has been observed how supermarkets with differing behaviors affect a heterogeneous crowd of consumer agents. The model was implemented in Java with Python used to evaluate the results. 

The simulation displays good acceptance with real grocery market behavior, i.e. captures the performance of British retailers thus can be used to determine the impact of changes in their behavior on their competitors and consumers.Furthermore it can be used to provide insight into sustainability of volatile pricing strategies, providing a useful insight in volatility of British supermarket retail industry. 
\end{abstract}
\acknowledgements{
I would like to express my sincere gratitude to Dr Maria Polukarov for her guidance and support which provided me the freedom to take this research in the direction of my interest.\\
\\
I would also like to thank my family and friends for their encouragement and support. To those who quietly listened to my software complaints. To those who worked throughout the nights with me. To those who helped me write what I couldn't say. I cannot thank you enough.
}

\declaration{
I, Stefan Collier, declare that this dissertation and the work presented in it are my own and has been generated by me as the result of my own original research.\\
I confirm that:\\
1. This work was done wholly or mainly while in candidature for a degree at this University;\\
2. Where any part of this dissertation has previously been submitted for any other qualification at this University or any other institution, this has been clearly stated;\\
3. Where I have consulted the published work of others, this is always clearly attributed;\\
4. Where I have quoted from the work of others, the source is always given. With the exception of such quotations, this dissertation is entirely my own work;\\
5. I have acknowledged all main sources of help;\\
6. Where the thesis is based on work done by myself jointly with others, I have made clear exactly what was done by others and what I have contributed myself;\\
7. Either none of this work has been published before submission, or parts of this work have been published by :\\
\\
Stefan Collier\\
April 2016
}
\tableofcontents
\listoffigures
\listoftables

\mainmatter
%% ----------------------------------------------------------------
%\include{Introduction}
%\include{Conclusions}
\include{chapters/1Project/main}
\include{chapters/2Lit/main}
\include{chapters/3Design/HighLevel}
\include{chapters/3Design/InDepth}
\include{chapters/4Impl/main}

\include{chapters/5Experiments/1/main}
\include{chapters/5Experiments/2/main}
\include{chapters/5Experiments/3/main}
\include{chapters/5Experiments/4/main}

\include{chapters/6Conclusion/main}

\appendix
\include{appendix/AppendixB}
\include{appendix/D/main}
\include{appendix/AppendixC}

\backmatter
\bibliographystyle{ecs}
\bibliography{ECS}
\end{document}
%% ----------------------------------------------------------------

 %% ----------------------------------------------------------------
%% Progress.tex
%% ---------------------------------------------------------------- 
\documentclass{ecsprogress}    % Use the progress Style
\graphicspath{{../figs/}}   % Location of your graphics files
    \usepackage{natbib}            % Use Natbib style for the refs.
\hypersetup{colorlinks=true}   % Set to false for black/white printing
\input{Definitions}            % Include your abbreviations



\usepackage{enumitem}% http://ctan.org/pkg/enumitem
\usepackage{multirow}
\usepackage{float}
\usepackage{amsmath}
\usepackage{multicol}
\usepackage{amssymb}
\usepackage[normalem]{ulem}
\useunder{\uline}{\ul}{}
\usepackage{wrapfig}


\usepackage[table,xcdraw]{xcolor}


%% ----------------------------------------------------------------
\begin{document}
\frontmatter
\title      {Heterogeneous Agent-based Model for Supermarket Competition}
\authors    {\texorpdfstring
             {\href{mailto:sc22g13@ecs.soton.ac.uk}{Stefan J. Collier}}
             {Stefan J. Collier}
            }
\addresses  {\groupname\\\deptname\\\univname}
\date       {\today}
\subject    {}
\keywords   {}
\supervisor {Dr. Maria Polukarov}
\examiner   {Professor Sheng Chen}

\maketitle
\begin{abstract}
This project aim was to model and analyse the effects of competitive pricing behaviors of grocery retailers on the British market. 

This was achieved by creating a multi-agent model, containing retailer and consumer agents. The heterogeneous crowd of retailers employs either a uniform pricing strategy or a ‘local price flexing’ strategy. The actions of these retailers are chosen by predicting the profit of each action, using a perceptron. Following on from the consideration of different economic models, a discrete model was developed so that software agents have a discrete environment to operate within. Within the model, it has been observed how supermarkets with differing behaviors affect a heterogeneous crowd of consumer agents. The model was implemented in Java with Python used to evaluate the results. 

The simulation displays good acceptance with real grocery market behavior, i.e. captures the performance of British retailers thus can be used to determine the impact of changes in their behavior on their competitors and consumers.Furthermore it can be used to provide insight into sustainability of volatile pricing strategies, providing a useful insight in volatility of British supermarket retail industry. 
\end{abstract}
\acknowledgements{
I would like to express my sincere gratitude to Dr Maria Polukarov for her guidance and support which provided me the freedom to take this research in the direction of my interest.\\
\\
I would also like to thank my family and friends for their encouragement and support. To those who quietly listened to my software complaints. To those who worked throughout the nights with me. To those who helped me write what I couldn't say. I cannot thank you enough.
}

\declaration{
I, Stefan Collier, declare that this dissertation and the work presented in it are my own and has been generated by me as the result of my own original research.\\
I confirm that:\\
1. This work was done wholly or mainly while in candidature for a degree at this University;\\
2. Where any part of this dissertation has previously been submitted for any other qualification at this University or any other institution, this has been clearly stated;\\
3. Where I have consulted the published work of others, this is always clearly attributed;\\
4. Where I have quoted from the work of others, the source is always given. With the exception of such quotations, this dissertation is entirely my own work;\\
5. I have acknowledged all main sources of help;\\
6. Where the thesis is based on work done by myself jointly with others, I have made clear exactly what was done by others and what I have contributed myself;\\
7. Either none of this work has been published before submission, or parts of this work have been published by :\\
\\
Stefan Collier\\
April 2016
}
\tableofcontents
\listoffigures
\listoftables

\mainmatter
%% ----------------------------------------------------------------
%\include{Introduction}
%\include{Conclusions}
\include{chapters/1Project/main}
\include{chapters/2Lit/main}
\include{chapters/3Design/HighLevel}
\include{chapters/3Design/InDepth}
\include{chapters/4Impl/main}

\include{chapters/5Experiments/1/main}
\include{chapters/5Experiments/2/main}
\include{chapters/5Experiments/3/main}
\include{chapters/5Experiments/4/main}

\include{chapters/6Conclusion/main}

\appendix
\include{appendix/AppendixB}
\include{appendix/D/main}
\include{appendix/AppendixC}

\backmatter
\bibliographystyle{ecs}
\bibliography{ECS}
\end{document}
%% ----------------------------------------------------------------

 %% ----------------------------------------------------------------
%% Progress.tex
%% ---------------------------------------------------------------- 
\documentclass{ecsprogress}    % Use the progress Style
\graphicspath{{../figs/}}   % Location of your graphics files
    \usepackage{natbib}            % Use Natbib style for the refs.
\hypersetup{colorlinks=true}   % Set to false for black/white printing
\input{Definitions}            % Include your abbreviations



\usepackage{enumitem}% http://ctan.org/pkg/enumitem
\usepackage{multirow}
\usepackage{float}
\usepackage{amsmath}
\usepackage{multicol}
\usepackage{amssymb}
\usepackage[normalem]{ulem}
\useunder{\uline}{\ul}{}
\usepackage{wrapfig}


\usepackage[table,xcdraw]{xcolor}


%% ----------------------------------------------------------------
\begin{document}
\frontmatter
\title      {Heterogeneous Agent-based Model for Supermarket Competition}
\authors    {\texorpdfstring
             {\href{mailto:sc22g13@ecs.soton.ac.uk}{Stefan J. Collier}}
             {Stefan J. Collier}
            }
\addresses  {\groupname\\\deptname\\\univname}
\date       {\today}
\subject    {}
\keywords   {}
\supervisor {Dr. Maria Polukarov}
\examiner   {Professor Sheng Chen}

\maketitle
\begin{abstract}
This project aim was to model and analyse the effects of competitive pricing behaviors of grocery retailers on the British market. 

This was achieved by creating a multi-agent model, containing retailer and consumer agents. The heterogeneous crowd of retailers employs either a uniform pricing strategy or a ‘local price flexing’ strategy. The actions of these retailers are chosen by predicting the profit of each action, using a perceptron. Following on from the consideration of different economic models, a discrete model was developed so that software agents have a discrete environment to operate within. Within the model, it has been observed how supermarkets with differing behaviors affect a heterogeneous crowd of consumer agents. The model was implemented in Java with Python used to evaluate the results. 

The simulation displays good acceptance with real grocery market behavior, i.e. captures the performance of British retailers thus can be used to determine the impact of changes in their behavior on their competitors and consumers.Furthermore it can be used to provide insight into sustainability of volatile pricing strategies, providing a useful insight in volatility of British supermarket retail industry. 
\end{abstract}
\acknowledgements{
I would like to express my sincere gratitude to Dr Maria Polukarov for her guidance and support which provided me the freedom to take this research in the direction of my interest.\\
\\
I would also like to thank my family and friends for their encouragement and support. To those who quietly listened to my software complaints. To those who worked throughout the nights with me. To those who helped me write what I couldn't say. I cannot thank you enough.
}

\declaration{
I, Stefan Collier, declare that this dissertation and the work presented in it are my own and has been generated by me as the result of my own original research.\\
I confirm that:\\
1. This work was done wholly or mainly while in candidature for a degree at this University;\\
2. Where any part of this dissertation has previously been submitted for any other qualification at this University or any other institution, this has been clearly stated;\\
3. Where I have consulted the published work of others, this is always clearly attributed;\\
4. Where I have quoted from the work of others, the source is always given. With the exception of such quotations, this dissertation is entirely my own work;\\
5. I have acknowledged all main sources of help;\\
6. Where the thesis is based on work done by myself jointly with others, I have made clear exactly what was done by others and what I have contributed myself;\\
7. Either none of this work has been published before submission, or parts of this work have been published by :\\
\\
Stefan Collier\\
April 2016
}
\tableofcontents
\listoffigures
\listoftables

\mainmatter
%% ----------------------------------------------------------------
%\include{Introduction}
%\include{Conclusions}
\include{chapters/1Project/main}
\include{chapters/2Lit/main}
\include{chapters/3Design/HighLevel}
\include{chapters/3Design/InDepth}
\include{chapters/4Impl/main}

\include{chapters/5Experiments/1/main}
\include{chapters/5Experiments/2/main}
\include{chapters/5Experiments/3/main}
\include{chapters/5Experiments/4/main}

\include{chapters/6Conclusion/main}

\appendix
\include{appendix/AppendixB}
\include{appendix/D/main}
\include{appendix/AppendixC}

\backmatter
\bibliographystyle{ecs}
\bibliography{ECS}
\end{document}
%% ----------------------------------------------------------------


 %% ----------------------------------------------------------------
%% Progress.tex
%% ---------------------------------------------------------------- 
\documentclass{ecsprogress}    % Use the progress Style
\graphicspath{{../figs/}}   % Location of your graphics files
    \usepackage{natbib}            % Use Natbib style for the refs.
\hypersetup{colorlinks=true}   % Set to false for black/white printing
\input{Definitions}            % Include your abbreviations



\usepackage{enumitem}% http://ctan.org/pkg/enumitem
\usepackage{multirow}
\usepackage{float}
\usepackage{amsmath}
\usepackage{multicol}
\usepackage{amssymb}
\usepackage[normalem]{ulem}
\useunder{\uline}{\ul}{}
\usepackage{wrapfig}


\usepackage[table,xcdraw]{xcolor}


%% ----------------------------------------------------------------
\begin{document}
\frontmatter
\title      {Heterogeneous Agent-based Model for Supermarket Competition}
\authors    {\texorpdfstring
             {\href{mailto:sc22g13@ecs.soton.ac.uk}{Stefan J. Collier}}
             {Stefan J. Collier}
            }
\addresses  {\groupname\\\deptname\\\univname}
\date       {\today}
\subject    {}
\keywords   {}
\supervisor {Dr. Maria Polukarov}
\examiner   {Professor Sheng Chen}

\maketitle
\begin{abstract}
This project aim was to model and analyse the effects of competitive pricing behaviors of grocery retailers on the British market. 

This was achieved by creating a multi-agent model, containing retailer and consumer agents. The heterogeneous crowd of retailers employs either a uniform pricing strategy or a ‘local price flexing’ strategy. The actions of these retailers are chosen by predicting the profit of each action, using a perceptron. Following on from the consideration of different economic models, a discrete model was developed so that software agents have a discrete environment to operate within. Within the model, it has been observed how supermarkets with differing behaviors affect a heterogeneous crowd of consumer agents. The model was implemented in Java with Python used to evaluate the results. 

The simulation displays good acceptance with real grocery market behavior, i.e. captures the performance of British retailers thus can be used to determine the impact of changes in their behavior on their competitors and consumers.Furthermore it can be used to provide insight into sustainability of volatile pricing strategies, providing a useful insight in volatility of British supermarket retail industry. 
\end{abstract}
\acknowledgements{
I would like to express my sincere gratitude to Dr Maria Polukarov for her guidance and support which provided me the freedom to take this research in the direction of my interest.\\
\\
I would also like to thank my family and friends for their encouragement and support. To those who quietly listened to my software complaints. To those who worked throughout the nights with me. To those who helped me write what I couldn't say. I cannot thank you enough.
}

\declaration{
I, Stefan Collier, declare that this dissertation and the work presented in it are my own and has been generated by me as the result of my own original research.\\
I confirm that:\\
1. This work was done wholly or mainly while in candidature for a degree at this University;\\
2. Where any part of this dissertation has previously been submitted for any other qualification at this University or any other institution, this has been clearly stated;\\
3. Where I have consulted the published work of others, this is always clearly attributed;\\
4. Where I have quoted from the work of others, the source is always given. With the exception of such quotations, this dissertation is entirely my own work;\\
5. I have acknowledged all main sources of help;\\
6. Where the thesis is based on work done by myself jointly with others, I have made clear exactly what was done by others and what I have contributed myself;\\
7. Either none of this work has been published before submission, or parts of this work have been published by :\\
\\
Stefan Collier\\
April 2016
}
\tableofcontents
\listoffigures
\listoftables

\mainmatter
%% ----------------------------------------------------------------
%\include{Introduction}
%\include{Conclusions}
\include{chapters/1Project/main}
\include{chapters/2Lit/main}
\include{chapters/3Design/HighLevel}
\include{chapters/3Design/InDepth}
\include{chapters/4Impl/main}

\include{chapters/5Experiments/1/main}
\include{chapters/5Experiments/2/main}
\include{chapters/5Experiments/3/main}
\include{chapters/5Experiments/4/main}

\include{chapters/6Conclusion/main}

\appendix
\include{appendix/AppendixB}
\include{appendix/D/main}
\include{appendix/AppendixC}

\backmatter
\bibliographystyle{ecs}
\bibliography{ECS}
\end{document}
%% ----------------------------------------------------------------


\appendix
\include{appendix/AppendixB}
 %% ----------------------------------------------------------------
%% Progress.tex
%% ---------------------------------------------------------------- 
\documentclass{ecsprogress}    % Use the progress Style
\graphicspath{{../figs/}}   % Location of your graphics files
    \usepackage{natbib}            % Use Natbib style for the refs.
\hypersetup{colorlinks=true}   % Set to false for black/white printing
\input{Definitions}            % Include your abbreviations



\usepackage{enumitem}% http://ctan.org/pkg/enumitem
\usepackage{multirow}
\usepackage{float}
\usepackage{amsmath}
\usepackage{multicol}
\usepackage{amssymb}
\usepackage[normalem]{ulem}
\useunder{\uline}{\ul}{}
\usepackage{wrapfig}


\usepackage[table,xcdraw]{xcolor}


%% ----------------------------------------------------------------
\begin{document}
\frontmatter
\title      {Heterogeneous Agent-based Model for Supermarket Competition}
\authors    {\texorpdfstring
             {\href{mailto:sc22g13@ecs.soton.ac.uk}{Stefan J. Collier}}
             {Stefan J. Collier}
            }
\addresses  {\groupname\\\deptname\\\univname}
\date       {\today}
\subject    {}
\keywords   {}
\supervisor {Dr. Maria Polukarov}
\examiner   {Professor Sheng Chen}

\maketitle
\begin{abstract}
This project aim was to model and analyse the effects of competitive pricing behaviors of grocery retailers on the British market. 

This was achieved by creating a multi-agent model, containing retailer and consumer agents. The heterogeneous crowd of retailers employs either a uniform pricing strategy or a ‘local price flexing’ strategy. The actions of these retailers are chosen by predicting the profit of each action, using a perceptron. Following on from the consideration of different economic models, a discrete model was developed so that software agents have a discrete environment to operate within. Within the model, it has been observed how supermarkets with differing behaviors affect a heterogeneous crowd of consumer agents. The model was implemented in Java with Python used to evaluate the results. 

The simulation displays good acceptance with real grocery market behavior, i.e. captures the performance of British retailers thus can be used to determine the impact of changes in their behavior on their competitors and consumers.Furthermore it can be used to provide insight into sustainability of volatile pricing strategies, providing a useful insight in volatility of British supermarket retail industry. 
\end{abstract}
\acknowledgements{
I would like to express my sincere gratitude to Dr Maria Polukarov for her guidance and support which provided me the freedom to take this research in the direction of my interest.\\
\\
I would also like to thank my family and friends for their encouragement and support. To those who quietly listened to my software complaints. To those who worked throughout the nights with me. To those who helped me write what I couldn't say. I cannot thank you enough.
}

\declaration{
I, Stefan Collier, declare that this dissertation and the work presented in it are my own and has been generated by me as the result of my own original research.\\
I confirm that:\\
1. This work was done wholly or mainly while in candidature for a degree at this University;\\
2. Where any part of this dissertation has previously been submitted for any other qualification at this University or any other institution, this has been clearly stated;\\
3. Where I have consulted the published work of others, this is always clearly attributed;\\
4. Where I have quoted from the work of others, the source is always given. With the exception of such quotations, this dissertation is entirely my own work;\\
5. I have acknowledged all main sources of help;\\
6. Where the thesis is based on work done by myself jointly with others, I have made clear exactly what was done by others and what I have contributed myself;\\
7. Either none of this work has been published before submission, or parts of this work have been published by :\\
\\
Stefan Collier\\
April 2016
}
\tableofcontents
\listoffigures
\listoftables

\mainmatter
%% ----------------------------------------------------------------
%\include{Introduction}
%\include{Conclusions}
\include{chapters/1Project/main}
\include{chapters/2Lit/main}
\include{chapters/3Design/HighLevel}
\include{chapters/3Design/InDepth}
\include{chapters/4Impl/main}

\include{chapters/5Experiments/1/main}
\include{chapters/5Experiments/2/main}
\include{chapters/5Experiments/3/main}
\include{chapters/5Experiments/4/main}

\include{chapters/6Conclusion/main}

\appendix
\include{appendix/AppendixB}
\include{appendix/D/main}
\include{appendix/AppendixC}

\backmatter
\bibliographystyle{ecs}
\bibliography{ECS}
\end{document}
%% ----------------------------------------------------------------

\include{appendix/AppendixC}

\backmatter
\bibliographystyle{ecs}
\bibliography{ECS}
\end{document}
%% ----------------------------------------------------------------

\include{chapters/3Design/HighLevel}
\include{chapters/3Design/InDepth}
 %% ----------------------------------------------------------------
%% Progress.tex
%% ---------------------------------------------------------------- 
\documentclass{ecsprogress}    % Use the progress Style
\graphicspath{{../figs/}}   % Location of your graphics files
    \usepackage{natbib}            % Use Natbib style for the refs.
\hypersetup{colorlinks=true}   % Set to false for black/white printing
\input{Definitions}            % Include your abbreviations



\usepackage{enumitem}% http://ctan.org/pkg/enumitem
\usepackage{multirow}
\usepackage{float}
\usepackage{amsmath}
\usepackage{multicol}
\usepackage{amssymb}
\usepackage[normalem]{ulem}
\useunder{\uline}{\ul}{}
\usepackage{wrapfig}


\usepackage[table,xcdraw]{xcolor}


%% ----------------------------------------------------------------
\begin{document}
\frontmatter
\title      {Heterogeneous Agent-based Model for Supermarket Competition}
\authors    {\texorpdfstring
             {\href{mailto:sc22g13@ecs.soton.ac.uk}{Stefan J. Collier}}
             {Stefan J. Collier}
            }
\addresses  {\groupname\\\deptname\\\univname}
\date       {\today}
\subject    {}
\keywords   {}
\supervisor {Dr. Maria Polukarov}
\examiner   {Professor Sheng Chen}

\maketitle
\begin{abstract}
This project aim was to model and analyse the effects of competitive pricing behaviors of grocery retailers on the British market. 

This was achieved by creating a multi-agent model, containing retailer and consumer agents. The heterogeneous crowd of retailers employs either a uniform pricing strategy or a ‘local price flexing’ strategy. The actions of these retailers are chosen by predicting the profit of each action, using a perceptron. Following on from the consideration of different economic models, a discrete model was developed so that software agents have a discrete environment to operate within. Within the model, it has been observed how supermarkets with differing behaviors affect a heterogeneous crowd of consumer agents. The model was implemented in Java with Python used to evaluate the results. 

The simulation displays good acceptance with real grocery market behavior, i.e. captures the performance of British retailers thus can be used to determine the impact of changes in their behavior on their competitors and consumers.Furthermore it can be used to provide insight into sustainability of volatile pricing strategies, providing a useful insight in volatility of British supermarket retail industry. 
\end{abstract}
\acknowledgements{
I would like to express my sincere gratitude to Dr Maria Polukarov for her guidance and support which provided me the freedom to take this research in the direction of my interest.\\
\\
I would also like to thank my family and friends for their encouragement and support. To those who quietly listened to my software complaints. To those who worked throughout the nights with me. To those who helped me write what I couldn't say. I cannot thank you enough.
}

\declaration{
I, Stefan Collier, declare that this dissertation and the work presented in it are my own and has been generated by me as the result of my own original research.\\
I confirm that:\\
1. This work was done wholly or mainly while in candidature for a degree at this University;\\
2. Where any part of this dissertation has previously been submitted for any other qualification at this University or any other institution, this has been clearly stated;\\
3. Where I have consulted the published work of others, this is always clearly attributed;\\
4. Where I have quoted from the work of others, the source is always given. With the exception of such quotations, this dissertation is entirely my own work;\\
5. I have acknowledged all main sources of help;\\
6. Where the thesis is based on work done by myself jointly with others, I have made clear exactly what was done by others and what I have contributed myself;\\
7. Either none of this work has been published before submission, or parts of this work have been published by :\\
\\
Stefan Collier\\
April 2016
}
\tableofcontents
\listoffigures
\listoftables

\mainmatter
%% ----------------------------------------------------------------
%\include{Introduction}
%\include{Conclusions}
 %% ----------------------------------------------------------------
%% Progress.tex
%% ---------------------------------------------------------------- 
\documentclass{ecsprogress}    % Use the progress Style
\graphicspath{{../figs/}}   % Location of your graphics files
    \usepackage{natbib}            % Use Natbib style for the refs.
\hypersetup{colorlinks=true}   % Set to false for black/white printing
\input{Definitions}            % Include your abbreviations



\usepackage{enumitem}% http://ctan.org/pkg/enumitem
\usepackage{multirow}
\usepackage{float}
\usepackage{amsmath}
\usepackage{multicol}
\usepackage{amssymb}
\usepackage[normalem]{ulem}
\useunder{\uline}{\ul}{}
\usepackage{wrapfig}


\usepackage[table,xcdraw]{xcolor}


%% ----------------------------------------------------------------
\begin{document}
\frontmatter
\title      {Heterogeneous Agent-based Model for Supermarket Competition}
\authors    {\texorpdfstring
             {\href{mailto:sc22g13@ecs.soton.ac.uk}{Stefan J. Collier}}
             {Stefan J. Collier}
            }
\addresses  {\groupname\\\deptname\\\univname}
\date       {\today}
\subject    {}
\keywords   {}
\supervisor {Dr. Maria Polukarov}
\examiner   {Professor Sheng Chen}

\maketitle
\begin{abstract}
This project aim was to model and analyse the effects of competitive pricing behaviors of grocery retailers on the British market. 

This was achieved by creating a multi-agent model, containing retailer and consumer agents. The heterogeneous crowd of retailers employs either a uniform pricing strategy or a ‘local price flexing’ strategy. The actions of these retailers are chosen by predicting the profit of each action, using a perceptron. Following on from the consideration of different economic models, a discrete model was developed so that software agents have a discrete environment to operate within. Within the model, it has been observed how supermarkets with differing behaviors affect a heterogeneous crowd of consumer agents. The model was implemented in Java with Python used to evaluate the results. 

The simulation displays good acceptance with real grocery market behavior, i.e. captures the performance of British retailers thus can be used to determine the impact of changes in their behavior on their competitors and consumers.Furthermore it can be used to provide insight into sustainability of volatile pricing strategies, providing a useful insight in volatility of British supermarket retail industry. 
\end{abstract}
\acknowledgements{
I would like to express my sincere gratitude to Dr Maria Polukarov for her guidance and support which provided me the freedom to take this research in the direction of my interest.\\
\\
I would also like to thank my family and friends for their encouragement and support. To those who quietly listened to my software complaints. To those who worked throughout the nights with me. To those who helped me write what I couldn't say. I cannot thank you enough.
}

\declaration{
I, Stefan Collier, declare that this dissertation and the work presented in it are my own and has been generated by me as the result of my own original research.\\
I confirm that:\\
1. This work was done wholly or mainly while in candidature for a degree at this University;\\
2. Where any part of this dissertation has previously been submitted for any other qualification at this University or any other institution, this has been clearly stated;\\
3. Where I have consulted the published work of others, this is always clearly attributed;\\
4. Where I have quoted from the work of others, the source is always given. With the exception of such quotations, this dissertation is entirely my own work;\\
5. I have acknowledged all main sources of help;\\
6. Where the thesis is based on work done by myself jointly with others, I have made clear exactly what was done by others and what I have contributed myself;\\
7. Either none of this work has been published before submission, or parts of this work have been published by :\\
\\
Stefan Collier\\
April 2016
}
\tableofcontents
\listoffigures
\listoftables

\mainmatter
%% ----------------------------------------------------------------
%\include{Introduction}
%\include{Conclusions}
\include{chapters/1Project/main}
\include{chapters/2Lit/main}
\include{chapters/3Design/HighLevel}
\include{chapters/3Design/InDepth}
\include{chapters/4Impl/main}

\include{chapters/5Experiments/1/main}
\include{chapters/5Experiments/2/main}
\include{chapters/5Experiments/3/main}
\include{chapters/5Experiments/4/main}

\include{chapters/6Conclusion/main}

\appendix
\include{appendix/AppendixB}
\include{appendix/D/main}
\include{appendix/AppendixC}

\backmatter
\bibliographystyle{ecs}
\bibliography{ECS}
\end{document}
%% ----------------------------------------------------------------

 %% ----------------------------------------------------------------
%% Progress.tex
%% ---------------------------------------------------------------- 
\documentclass{ecsprogress}    % Use the progress Style
\graphicspath{{../figs/}}   % Location of your graphics files
    \usepackage{natbib}            % Use Natbib style for the refs.
\hypersetup{colorlinks=true}   % Set to false for black/white printing
\input{Definitions}            % Include your abbreviations



\usepackage{enumitem}% http://ctan.org/pkg/enumitem
\usepackage{multirow}
\usepackage{float}
\usepackage{amsmath}
\usepackage{multicol}
\usepackage{amssymb}
\usepackage[normalem]{ulem}
\useunder{\uline}{\ul}{}
\usepackage{wrapfig}


\usepackage[table,xcdraw]{xcolor}


%% ----------------------------------------------------------------
\begin{document}
\frontmatter
\title      {Heterogeneous Agent-based Model for Supermarket Competition}
\authors    {\texorpdfstring
             {\href{mailto:sc22g13@ecs.soton.ac.uk}{Stefan J. Collier}}
             {Stefan J. Collier}
            }
\addresses  {\groupname\\\deptname\\\univname}
\date       {\today}
\subject    {}
\keywords   {}
\supervisor {Dr. Maria Polukarov}
\examiner   {Professor Sheng Chen}

\maketitle
\begin{abstract}
This project aim was to model and analyse the effects of competitive pricing behaviors of grocery retailers on the British market. 

This was achieved by creating a multi-agent model, containing retailer and consumer agents. The heterogeneous crowd of retailers employs either a uniform pricing strategy or a ‘local price flexing’ strategy. The actions of these retailers are chosen by predicting the profit of each action, using a perceptron. Following on from the consideration of different economic models, a discrete model was developed so that software agents have a discrete environment to operate within. Within the model, it has been observed how supermarkets with differing behaviors affect a heterogeneous crowd of consumer agents. The model was implemented in Java with Python used to evaluate the results. 

The simulation displays good acceptance with real grocery market behavior, i.e. captures the performance of British retailers thus can be used to determine the impact of changes in their behavior on their competitors and consumers.Furthermore it can be used to provide insight into sustainability of volatile pricing strategies, providing a useful insight in volatility of British supermarket retail industry. 
\end{abstract}
\acknowledgements{
I would like to express my sincere gratitude to Dr Maria Polukarov for her guidance and support which provided me the freedom to take this research in the direction of my interest.\\
\\
I would also like to thank my family and friends for their encouragement and support. To those who quietly listened to my software complaints. To those who worked throughout the nights with me. To those who helped me write what I couldn't say. I cannot thank you enough.
}

\declaration{
I, Stefan Collier, declare that this dissertation and the work presented in it are my own and has been generated by me as the result of my own original research.\\
I confirm that:\\
1. This work was done wholly or mainly while in candidature for a degree at this University;\\
2. Where any part of this dissertation has previously been submitted for any other qualification at this University or any other institution, this has been clearly stated;\\
3. Where I have consulted the published work of others, this is always clearly attributed;\\
4. Where I have quoted from the work of others, the source is always given. With the exception of such quotations, this dissertation is entirely my own work;\\
5. I have acknowledged all main sources of help;\\
6. Where the thesis is based on work done by myself jointly with others, I have made clear exactly what was done by others and what I have contributed myself;\\
7. Either none of this work has been published before submission, or parts of this work have been published by :\\
\\
Stefan Collier\\
April 2016
}
\tableofcontents
\listoffigures
\listoftables

\mainmatter
%% ----------------------------------------------------------------
%\include{Introduction}
%\include{Conclusions}
\include{chapters/1Project/main}
\include{chapters/2Lit/main}
\include{chapters/3Design/HighLevel}
\include{chapters/3Design/InDepth}
\include{chapters/4Impl/main}

\include{chapters/5Experiments/1/main}
\include{chapters/5Experiments/2/main}
\include{chapters/5Experiments/3/main}
\include{chapters/5Experiments/4/main}

\include{chapters/6Conclusion/main}

\appendix
\include{appendix/AppendixB}
\include{appendix/D/main}
\include{appendix/AppendixC}

\backmatter
\bibliographystyle{ecs}
\bibliography{ECS}
\end{document}
%% ----------------------------------------------------------------

\include{chapters/3Design/HighLevel}
\include{chapters/3Design/InDepth}
 %% ----------------------------------------------------------------
%% Progress.tex
%% ---------------------------------------------------------------- 
\documentclass{ecsprogress}    % Use the progress Style
\graphicspath{{../figs/}}   % Location of your graphics files
    \usepackage{natbib}            % Use Natbib style for the refs.
\hypersetup{colorlinks=true}   % Set to false for black/white printing
\input{Definitions}            % Include your abbreviations



\usepackage{enumitem}% http://ctan.org/pkg/enumitem
\usepackage{multirow}
\usepackage{float}
\usepackage{amsmath}
\usepackage{multicol}
\usepackage{amssymb}
\usepackage[normalem]{ulem}
\useunder{\uline}{\ul}{}
\usepackage{wrapfig}


\usepackage[table,xcdraw]{xcolor}


%% ----------------------------------------------------------------
\begin{document}
\frontmatter
\title      {Heterogeneous Agent-based Model for Supermarket Competition}
\authors    {\texorpdfstring
             {\href{mailto:sc22g13@ecs.soton.ac.uk}{Stefan J. Collier}}
             {Stefan J. Collier}
            }
\addresses  {\groupname\\\deptname\\\univname}
\date       {\today}
\subject    {}
\keywords   {}
\supervisor {Dr. Maria Polukarov}
\examiner   {Professor Sheng Chen}

\maketitle
\begin{abstract}
This project aim was to model and analyse the effects of competitive pricing behaviors of grocery retailers on the British market. 

This was achieved by creating a multi-agent model, containing retailer and consumer agents. The heterogeneous crowd of retailers employs either a uniform pricing strategy or a ‘local price flexing’ strategy. The actions of these retailers are chosen by predicting the profit of each action, using a perceptron. Following on from the consideration of different economic models, a discrete model was developed so that software agents have a discrete environment to operate within. Within the model, it has been observed how supermarkets with differing behaviors affect a heterogeneous crowd of consumer agents. The model was implemented in Java with Python used to evaluate the results. 

The simulation displays good acceptance with real grocery market behavior, i.e. captures the performance of British retailers thus can be used to determine the impact of changes in their behavior on their competitors and consumers.Furthermore it can be used to provide insight into sustainability of volatile pricing strategies, providing a useful insight in volatility of British supermarket retail industry. 
\end{abstract}
\acknowledgements{
I would like to express my sincere gratitude to Dr Maria Polukarov for her guidance and support which provided me the freedom to take this research in the direction of my interest.\\
\\
I would also like to thank my family and friends for their encouragement and support. To those who quietly listened to my software complaints. To those who worked throughout the nights with me. To those who helped me write what I couldn't say. I cannot thank you enough.
}

\declaration{
I, Stefan Collier, declare that this dissertation and the work presented in it are my own and has been generated by me as the result of my own original research.\\
I confirm that:\\
1. This work was done wholly or mainly while in candidature for a degree at this University;\\
2. Where any part of this dissertation has previously been submitted for any other qualification at this University or any other institution, this has been clearly stated;\\
3. Where I have consulted the published work of others, this is always clearly attributed;\\
4. Where I have quoted from the work of others, the source is always given. With the exception of such quotations, this dissertation is entirely my own work;\\
5. I have acknowledged all main sources of help;\\
6. Where the thesis is based on work done by myself jointly with others, I have made clear exactly what was done by others and what I have contributed myself;\\
7. Either none of this work has been published before submission, or parts of this work have been published by :\\
\\
Stefan Collier\\
April 2016
}
\tableofcontents
\listoffigures
\listoftables

\mainmatter
%% ----------------------------------------------------------------
%\include{Introduction}
%\include{Conclusions}
\include{chapters/1Project/main}
\include{chapters/2Lit/main}
\include{chapters/3Design/HighLevel}
\include{chapters/3Design/InDepth}
\include{chapters/4Impl/main}

\include{chapters/5Experiments/1/main}
\include{chapters/5Experiments/2/main}
\include{chapters/5Experiments/3/main}
\include{chapters/5Experiments/4/main}

\include{chapters/6Conclusion/main}

\appendix
\include{appendix/AppendixB}
\include{appendix/D/main}
\include{appendix/AppendixC}

\backmatter
\bibliographystyle{ecs}
\bibliography{ECS}
\end{document}
%% ----------------------------------------------------------------


 %% ----------------------------------------------------------------
%% Progress.tex
%% ---------------------------------------------------------------- 
\documentclass{ecsprogress}    % Use the progress Style
\graphicspath{{../figs/}}   % Location of your graphics files
    \usepackage{natbib}            % Use Natbib style for the refs.
\hypersetup{colorlinks=true}   % Set to false for black/white printing
\input{Definitions}            % Include your abbreviations



\usepackage{enumitem}% http://ctan.org/pkg/enumitem
\usepackage{multirow}
\usepackage{float}
\usepackage{amsmath}
\usepackage{multicol}
\usepackage{amssymb}
\usepackage[normalem]{ulem}
\useunder{\uline}{\ul}{}
\usepackage{wrapfig}


\usepackage[table,xcdraw]{xcolor}


%% ----------------------------------------------------------------
\begin{document}
\frontmatter
\title      {Heterogeneous Agent-based Model for Supermarket Competition}
\authors    {\texorpdfstring
             {\href{mailto:sc22g13@ecs.soton.ac.uk}{Stefan J. Collier}}
             {Stefan J. Collier}
            }
\addresses  {\groupname\\\deptname\\\univname}
\date       {\today}
\subject    {}
\keywords   {}
\supervisor {Dr. Maria Polukarov}
\examiner   {Professor Sheng Chen}

\maketitle
\begin{abstract}
This project aim was to model and analyse the effects of competitive pricing behaviors of grocery retailers on the British market. 

This was achieved by creating a multi-agent model, containing retailer and consumer agents. The heterogeneous crowd of retailers employs either a uniform pricing strategy or a ‘local price flexing’ strategy. The actions of these retailers are chosen by predicting the profit of each action, using a perceptron. Following on from the consideration of different economic models, a discrete model was developed so that software agents have a discrete environment to operate within. Within the model, it has been observed how supermarkets with differing behaviors affect a heterogeneous crowd of consumer agents. The model was implemented in Java with Python used to evaluate the results. 

The simulation displays good acceptance with real grocery market behavior, i.e. captures the performance of British retailers thus can be used to determine the impact of changes in their behavior on their competitors and consumers.Furthermore it can be used to provide insight into sustainability of volatile pricing strategies, providing a useful insight in volatility of British supermarket retail industry. 
\end{abstract}
\acknowledgements{
I would like to express my sincere gratitude to Dr Maria Polukarov for her guidance and support which provided me the freedom to take this research in the direction of my interest.\\
\\
I would also like to thank my family and friends for their encouragement and support. To those who quietly listened to my software complaints. To those who worked throughout the nights with me. To those who helped me write what I couldn't say. I cannot thank you enough.
}

\declaration{
I, Stefan Collier, declare that this dissertation and the work presented in it are my own and has been generated by me as the result of my own original research.\\
I confirm that:\\
1. This work was done wholly or mainly while in candidature for a degree at this University;\\
2. Where any part of this dissertation has previously been submitted for any other qualification at this University or any other institution, this has been clearly stated;\\
3. Where I have consulted the published work of others, this is always clearly attributed;\\
4. Where I have quoted from the work of others, the source is always given. With the exception of such quotations, this dissertation is entirely my own work;\\
5. I have acknowledged all main sources of help;\\
6. Where the thesis is based on work done by myself jointly with others, I have made clear exactly what was done by others and what I have contributed myself;\\
7. Either none of this work has been published before submission, or parts of this work have been published by :\\
\\
Stefan Collier\\
April 2016
}
\tableofcontents
\listoffigures
\listoftables

\mainmatter
%% ----------------------------------------------------------------
%\include{Introduction}
%\include{Conclusions}
\include{chapters/1Project/main}
\include{chapters/2Lit/main}
\include{chapters/3Design/HighLevel}
\include{chapters/3Design/InDepth}
\include{chapters/4Impl/main}

\include{chapters/5Experiments/1/main}
\include{chapters/5Experiments/2/main}
\include{chapters/5Experiments/3/main}
\include{chapters/5Experiments/4/main}

\include{chapters/6Conclusion/main}

\appendix
\include{appendix/AppendixB}
\include{appendix/D/main}
\include{appendix/AppendixC}

\backmatter
\bibliographystyle{ecs}
\bibliography{ECS}
\end{document}
%% ----------------------------------------------------------------

 %% ----------------------------------------------------------------
%% Progress.tex
%% ---------------------------------------------------------------- 
\documentclass{ecsprogress}    % Use the progress Style
\graphicspath{{../figs/}}   % Location of your graphics files
    \usepackage{natbib}            % Use Natbib style for the refs.
\hypersetup{colorlinks=true}   % Set to false for black/white printing
\input{Definitions}            % Include your abbreviations



\usepackage{enumitem}% http://ctan.org/pkg/enumitem
\usepackage{multirow}
\usepackage{float}
\usepackage{amsmath}
\usepackage{multicol}
\usepackage{amssymb}
\usepackage[normalem]{ulem}
\useunder{\uline}{\ul}{}
\usepackage{wrapfig}


\usepackage[table,xcdraw]{xcolor}


%% ----------------------------------------------------------------
\begin{document}
\frontmatter
\title      {Heterogeneous Agent-based Model for Supermarket Competition}
\authors    {\texorpdfstring
             {\href{mailto:sc22g13@ecs.soton.ac.uk}{Stefan J. Collier}}
             {Stefan J. Collier}
            }
\addresses  {\groupname\\\deptname\\\univname}
\date       {\today}
\subject    {}
\keywords   {}
\supervisor {Dr. Maria Polukarov}
\examiner   {Professor Sheng Chen}

\maketitle
\begin{abstract}
This project aim was to model and analyse the effects of competitive pricing behaviors of grocery retailers on the British market. 

This was achieved by creating a multi-agent model, containing retailer and consumer agents. The heterogeneous crowd of retailers employs either a uniform pricing strategy or a ‘local price flexing’ strategy. The actions of these retailers are chosen by predicting the profit of each action, using a perceptron. Following on from the consideration of different economic models, a discrete model was developed so that software agents have a discrete environment to operate within. Within the model, it has been observed how supermarkets with differing behaviors affect a heterogeneous crowd of consumer agents. The model was implemented in Java with Python used to evaluate the results. 

The simulation displays good acceptance with real grocery market behavior, i.e. captures the performance of British retailers thus can be used to determine the impact of changes in their behavior on their competitors and consumers.Furthermore it can be used to provide insight into sustainability of volatile pricing strategies, providing a useful insight in volatility of British supermarket retail industry. 
\end{abstract}
\acknowledgements{
I would like to express my sincere gratitude to Dr Maria Polukarov for her guidance and support which provided me the freedom to take this research in the direction of my interest.\\
\\
I would also like to thank my family and friends for their encouragement and support. To those who quietly listened to my software complaints. To those who worked throughout the nights with me. To those who helped me write what I couldn't say. I cannot thank you enough.
}

\declaration{
I, Stefan Collier, declare that this dissertation and the work presented in it are my own and has been generated by me as the result of my own original research.\\
I confirm that:\\
1. This work was done wholly or mainly while in candidature for a degree at this University;\\
2. Where any part of this dissertation has previously been submitted for any other qualification at this University or any other institution, this has been clearly stated;\\
3. Where I have consulted the published work of others, this is always clearly attributed;\\
4. Where I have quoted from the work of others, the source is always given. With the exception of such quotations, this dissertation is entirely my own work;\\
5. I have acknowledged all main sources of help;\\
6. Where the thesis is based on work done by myself jointly with others, I have made clear exactly what was done by others and what I have contributed myself;\\
7. Either none of this work has been published before submission, or parts of this work have been published by :\\
\\
Stefan Collier\\
April 2016
}
\tableofcontents
\listoffigures
\listoftables

\mainmatter
%% ----------------------------------------------------------------
%\include{Introduction}
%\include{Conclusions}
\include{chapters/1Project/main}
\include{chapters/2Lit/main}
\include{chapters/3Design/HighLevel}
\include{chapters/3Design/InDepth}
\include{chapters/4Impl/main}

\include{chapters/5Experiments/1/main}
\include{chapters/5Experiments/2/main}
\include{chapters/5Experiments/3/main}
\include{chapters/5Experiments/4/main}

\include{chapters/6Conclusion/main}

\appendix
\include{appendix/AppendixB}
\include{appendix/D/main}
\include{appendix/AppendixC}

\backmatter
\bibliographystyle{ecs}
\bibliography{ECS}
\end{document}
%% ----------------------------------------------------------------

 %% ----------------------------------------------------------------
%% Progress.tex
%% ---------------------------------------------------------------- 
\documentclass{ecsprogress}    % Use the progress Style
\graphicspath{{../figs/}}   % Location of your graphics files
    \usepackage{natbib}            % Use Natbib style for the refs.
\hypersetup{colorlinks=true}   % Set to false for black/white printing
\input{Definitions}            % Include your abbreviations



\usepackage{enumitem}% http://ctan.org/pkg/enumitem
\usepackage{multirow}
\usepackage{float}
\usepackage{amsmath}
\usepackage{multicol}
\usepackage{amssymb}
\usepackage[normalem]{ulem}
\useunder{\uline}{\ul}{}
\usepackage{wrapfig}


\usepackage[table,xcdraw]{xcolor}


%% ----------------------------------------------------------------
\begin{document}
\frontmatter
\title      {Heterogeneous Agent-based Model for Supermarket Competition}
\authors    {\texorpdfstring
             {\href{mailto:sc22g13@ecs.soton.ac.uk}{Stefan J. Collier}}
             {Stefan J. Collier}
            }
\addresses  {\groupname\\\deptname\\\univname}
\date       {\today}
\subject    {}
\keywords   {}
\supervisor {Dr. Maria Polukarov}
\examiner   {Professor Sheng Chen}

\maketitle
\begin{abstract}
This project aim was to model and analyse the effects of competitive pricing behaviors of grocery retailers on the British market. 

This was achieved by creating a multi-agent model, containing retailer and consumer agents. The heterogeneous crowd of retailers employs either a uniform pricing strategy or a ‘local price flexing’ strategy. The actions of these retailers are chosen by predicting the profit of each action, using a perceptron. Following on from the consideration of different economic models, a discrete model was developed so that software agents have a discrete environment to operate within. Within the model, it has been observed how supermarkets with differing behaviors affect a heterogeneous crowd of consumer agents. The model was implemented in Java with Python used to evaluate the results. 

The simulation displays good acceptance with real grocery market behavior, i.e. captures the performance of British retailers thus can be used to determine the impact of changes in their behavior on their competitors and consumers.Furthermore it can be used to provide insight into sustainability of volatile pricing strategies, providing a useful insight in volatility of British supermarket retail industry. 
\end{abstract}
\acknowledgements{
I would like to express my sincere gratitude to Dr Maria Polukarov for her guidance and support which provided me the freedom to take this research in the direction of my interest.\\
\\
I would also like to thank my family and friends for their encouragement and support. To those who quietly listened to my software complaints. To those who worked throughout the nights with me. To those who helped me write what I couldn't say. I cannot thank you enough.
}

\declaration{
I, Stefan Collier, declare that this dissertation and the work presented in it are my own and has been generated by me as the result of my own original research.\\
I confirm that:\\
1. This work was done wholly or mainly while in candidature for a degree at this University;\\
2. Where any part of this dissertation has previously been submitted for any other qualification at this University or any other institution, this has been clearly stated;\\
3. Where I have consulted the published work of others, this is always clearly attributed;\\
4. Where I have quoted from the work of others, the source is always given. With the exception of such quotations, this dissertation is entirely my own work;\\
5. I have acknowledged all main sources of help;\\
6. Where the thesis is based on work done by myself jointly with others, I have made clear exactly what was done by others and what I have contributed myself;\\
7. Either none of this work has been published before submission, or parts of this work have been published by :\\
\\
Stefan Collier\\
April 2016
}
\tableofcontents
\listoffigures
\listoftables

\mainmatter
%% ----------------------------------------------------------------
%\include{Introduction}
%\include{Conclusions}
\include{chapters/1Project/main}
\include{chapters/2Lit/main}
\include{chapters/3Design/HighLevel}
\include{chapters/3Design/InDepth}
\include{chapters/4Impl/main}

\include{chapters/5Experiments/1/main}
\include{chapters/5Experiments/2/main}
\include{chapters/5Experiments/3/main}
\include{chapters/5Experiments/4/main}

\include{chapters/6Conclusion/main}

\appendix
\include{appendix/AppendixB}
\include{appendix/D/main}
\include{appendix/AppendixC}

\backmatter
\bibliographystyle{ecs}
\bibliography{ECS}
\end{document}
%% ----------------------------------------------------------------

 %% ----------------------------------------------------------------
%% Progress.tex
%% ---------------------------------------------------------------- 
\documentclass{ecsprogress}    % Use the progress Style
\graphicspath{{../figs/}}   % Location of your graphics files
    \usepackage{natbib}            % Use Natbib style for the refs.
\hypersetup{colorlinks=true}   % Set to false for black/white printing
\input{Definitions}            % Include your abbreviations



\usepackage{enumitem}% http://ctan.org/pkg/enumitem
\usepackage{multirow}
\usepackage{float}
\usepackage{amsmath}
\usepackage{multicol}
\usepackage{amssymb}
\usepackage[normalem]{ulem}
\useunder{\uline}{\ul}{}
\usepackage{wrapfig}


\usepackage[table,xcdraw]{xcolor}


%% ----------------------------------------------------------------
\begin{document}
\frontmatter
\title      {Heterogeneous Agent-based Model for Supermarket Competition}
\authors    {\texorpdfstring
             {\href{mailto:sc22g13@ecs.soton.ac.uk}{Stefan J. Collier}}
             {Stefan J. Collier}
            }
\addresses  {\groupname\\\deptname\\\univname}
\date       {\today}
\subject    {}
\keywords   {}
\supervisor {Dr. Maria Polukarov}
\examiner   {Professor Sheng Chen}

\maketitle
\begin{abstract}
This project aim was to model and analyse the effects of competitive pricing behaviors of grocery retailers on the British market. 

This was achieved by creating a multi-agent model, containing retailer and consumer agents. The heterogeneous crowd of retailers employs either a uniform pricing strategy or a ‘local price flexing’ strategy. The actions of these retailers are chosen by predicting the profit of each action, using a perceptron. Following on from the consideration of different economic models, a discrete model was developed so that software agents have a discrete environment to operate within. Within the model, it has been observed how supermarkets with differing behaviors affect a heterogeneous crowd of consumer agents. The model was implemented in Java with Python used to evaluate the results. 

The simulation displays good acceptance with real grocery market behavior, i.e. captures the performance of British retailers thus can be used to determine the impact of changes in their behavior on their competitors and consumers.Furthermore it can be used to provide insight into sustainability of volatile pricing strategies, providing a useful insight in volatility of British supermarket retail industry. 
\end{abstract}
\acknowledgements{
I would like to express my sincere gratitude to Dr Maria Polukarov for her guidance and support which provided me the freedom to take this research in the direction of my interest.\\
\\
I would also like to thank my family and friends for their encouragement and support. To those who quietly listened to my software complaints. To those who worked throughout the nights with me. To those who helped me write what I couldn't say. I cannot thank you enough.
}

\declaration{
I, Stefan Collier, declare that this dissertation and the work presented in it are my own and has been generated by me as the result of my own original research.\\
I confirm that:\\
1. This work was done wholly or mainly while in candidature for a degree at this University;\\
2. Where any part of this dissertation has previously been submitted for any other qualification at this University or any other institution, this has been clearly stated;\\
3. Where I have consulted the published work of others, this is always clearly attributed;\\
4. Where I have quoted from the work of others, the source is always given. With the exception of such quotations, this dissertation is entirely my own work;\\
5. I have acknowledged all main sources of help;\\
6. Where the thesis is based on work done by myself jointly with others, I have made clear exactly what was done by others and what I have contributed myself;\\
7. Either none of this work has been published before submission, or parts of this work have been published by :\\
\\
Stefan Collier\\
April 2016
}
\tableofcontents
\listoffigures
\listoftables

\mainmatter
%% ----------------------------------------------------------------
%\include{Introduction}
%\include{Conclusions}
\include{chapters/1Project/main}
\include{chapters/2Lit/main}
\include{chapters/3Design/HighLevel}
\include{chapters/3Design/InDepth}
\include{chapters/4Impl/main}

\include{chapters/5Experiments/1/main}
\include{chapters/5Experiments/2/main}
\include{chapters/5Experiments/3/main}
\include{chapters/5Experiments/4/main}

\include{chapters/6Conclusion/main}

\appendix
\include{appendix/AppendixB}
\include{appendix/D/main}
\include{appendix/AppendixC}

\backmatter
\bibliographystyle{ecs}
\bibliography{ECS}
\end{document}
%% ----------------------------------------------------------------


 %% ----------------------------------------------------------------
%% Progress.tex
%% ---------------------------------------------------------------- 
\documentclass{ecsprogress}    % Use the progress Style
\graphicspath{{../figs/}}   % Location of your graphics files
    \usepackage{natbib}            % Use Natbib style for the refs.
\hypersetup{colorlinks=true}   % Set to false for black/white printing
\input{Definitions}            % Include your abbreviations



\usepackage{enumitem}% http://ctan.org/pkg/enumitem
\usepackage{multirow}
\usepackage{float}
\usepackage{amsmath}
\usepackage{multicol}
\usepackage{amssymb}
\usepackage[normalem]{ulem}
\useunder{\uline}{\ul}{}
\usepackage{wrapfig}


\usepackage[table,xcdraw]{xcolor}


%% ----------------------------------------------------------------
\begin{document}
\frontmatter
\title      {Heterogeneous Agent-based Model for Supermarket Competition}
\authors    {\texorpdfstring
             {\href{mailto:sc22g13@ecs.soton.ac.uk}{Stefan J. Collier}}
             {Stefan J. Collier}
            }
\addresses  {\groupname\\\deptname\\\univname}
\date       {\today}
\subject    {}
\keywords   {}
\supervisor {Dr. Maria Polukarov}
\examiner   {Professor Sheng Chen}

\maketitle
\begin{abstract}
This project aim was to model and analyse the effects of competitive pricing behaviors of grocery retailers on the British market. 

This was achieved by creating a multi-agent model, containing retailer and consumer agents. The heterogeneous crowd of retailers employs either a uniform pricing strategy or a ‘local price flexing’ strategy. The actions of these retailers are chosen by predicting the profit of each action, using a perceptron. Following on from the consideration of different economic models, a discrete model was developed so that software agents have a discrete environment to operate within. Within the model, it has been observed how supermarkets with differing behaviors affect a heterogeneous crowd of consumer agents. The model was implemented in Java with Python used to evaluate the results. 

The simulation displays good acceptance with real grocery market behavior, i.e. captures the performance of British retailers thus can be used to determine the impact of changes in their behavior on their competitors and consumers.Furthermore it can be used to provide insight into sustainability of volatile pricing strategies, providing a useful insight in volatility of British supermarket retail industry. 
\end{abstract}
\acknowledgements{
I would like to express my sincere gratitude to Dr Maria Polukarov for her guidance and support which provided me the freedom to take this research in the direction of my interest.\\
\\
I would also like to thank my family and friends for their encouragement and support. To those who quietly listened to my software complaints. To those who worked throughout the nights with me. To those who helped me write what I couldn't say. I cannot thank you enough.
}

\declaration{
I, Stefan Collier, declare that this dissertation and the work presented in it are my own and has been generated by me as the result of my own original research.\\
I confirm that:\\
1. This work was done wholly or mainly while in candidature for a degree at this University;\\
2. Where any part of this dissertation has previously been submitted for any other qualification at this University or any other institution, this has been clearly stated;\\
3. Where I have consulted the published work of others, this is always clearly attributed;\\
4. Where I have quoted from the work of others, the source is always given. With the exception of such quotations, this dissertation is entirely my own work;\\
5. I have acknowledged all main sources of help;\\
6. Where the thesis is based on work done by myself jointly with others, I have made clear exactly what was done by others and what I have contributed myself;\\
7. Either none of this work has been published before submission, or parts of this work have been published by :\\
\\
Stefan Collier\\
April 2016
}
\tableofcontents
\listoffigures
\listoftables

\mainmatter
%% ----------------------------------------------------------------
%\include{Introduction}
%\include{Conclusions}
\include{chapters/1Project/main}
\include{chapters/2Lit/main}
\include{chapters/3Design/HighLevel}
\include{chapters/3Design/InDepth}
\include{chapters/4Impl/main}

\include{chapters/5Experiments/1/main}
\include{chapters/5Experiments/2/main}
\include{chapters/5Experiments/3/main}
\include{chapters/5Experiments/4/main}

\include{chapters/6Conclusion/main}

\appendix
\include{appendix/AppendixB}
\include{appendix/D/main}
\include{appendix/AppendixC}

\backmatter
\bibliographystyle{ecs}
\bibliography{ECS}
\end{document}
%% ----------------------------------------------------------------


\appendix
\include{appendix/AppendixB}
 %% ----------------------------------------------------------------
%% Progress.tex
%% ---------------------------------------------------------------- 
\documentclass{ecsprogress}    % Use the progress Style
\graphicspath{{../figs/}}   % Location of your graphics files
    \usepackage{natbib}            % Use Natbib style for the refs.
\hypersetup{colorlinks=true}   % Set to false for black/white printing
\input{Definitions}            % Include your abbreviations



\usepackage{enumitem}% http://ctan.org/pkg/enumitem
\usepackage{multirow}
\usepackage{float}
\usepackage{amsmath}
\usepackage{multicol}
\usepackage{amssymb}
\usepackage[normalem]{ulem}
\useunder{\uline}{\ul}{}
\usepackage{wrapfig}


\usepackage[table,xcdraw]{xcolor}


%% ----------------------------------------------------------------
\begin{document}
\frontmatter
\title      {Heterogeneous Agent-based Model for Supermarket Competition}
\authors    {\texorpdfstring
             {\href{mailto:sc22g13@ecs.soton.ac.uk}{Stefan J. Collier}}
             {Stefan J. Collier}
            }
\addresses  {\groupname\\\deptname\\\univname}
\date       {\today}
\subject    {}
\keywords   {}
\supervisor {Dr. Maria Polukarov}
\examiner   {Professor Sheng Chen}

\maketitle
\begin{abstract}
This project aim was to model and analyse the effects of competitive pricing behaviors of grocery retailers on the British market. 

This was achieved by creating a multi-agent model, containing retailer and consumer agents. The heterogeneous crowd of retailers employs either a uniform pricing strategy or a ‘local price flexing’ strategy. The actions of these retailers are chosen by predicting the profit of each action, using a perceptron. Following on from the consideration of different economic models, a discrete model was developed so that software agents have a discrete environment to operate within. Within the model, it has been observed how supermarkets with differing behaviors affect a heterogeneous crowd of consumer agents. The model was implemented in Java with Python used to evaluate the results. 

The simulation displays good acceptance with real grocery market behavior, i.e. captures the performance of British retailers thus can be used to determine the impact of changes in their behavior on their competitors and consumers.Furthermore it can be used to provide insight into sustainability of volatile pricing strategies, providing a useful insight in volatility of British supermarket retail industry. 
\end{abstract}
\acknowledgements{
I would like to express my sincere gratitude to Dr Maria Polukarov for her guidance and support which provided me the freedom to take this research in the direction of my interest.\\
\\
I would also like to thank my family and friends for their encouragement and support. To those who quietly listened to my software complaints. To those who worked throughout the nights with me. To those who helped me write what I couldn't say. I cannot thank you enough.
}

\declaration{
I, Stefan Collier, declare that this dissertation and the work presented in it are my own and has been generated by me as the result of my own original research.\\
I confirm that:\\
1. This work was done wholly or mainly while in candidature for a degree at this University;\\
2. Where any part of this dissertation has previously been submitted for any other qualification at this University or any other institution, this has been clearly stated;\\
3. Where I have consulted the published work of others, this is always clearly attributed;\\
4. Where I have quoted from the work of others, the source is always given. With the exception of such quotations, this dissertation is entirely my own work;\\
5. I have acknowledged all main sources of help;\\
6. Where the thesis is based on work done by myself jointly with others, I have made clear exactly what was done by others and what I have contributed myself;\\
7. Either none of this work has been published before submission, or parts of this work have been published by :\\
\\
Stefan Collier\\
April 2016
}
\tableofcontents
\listoffigures
\listoftables

\mainmatter
%% ----------------------------------------------------------------
%\include{Introduction}
%\include{Conclusions}
\include{chapters/1Project/main}
\include{chapters/2Lit/main}
\include{chapters/3Design/HighLevel}
\include{chapters/3Design/InDepth}
\include{chapters/4Impl/main}

\include{chapters/5Experiments/1/main}
\include{chapters/5Experiments/2/main}
\include{chapters/5Experiments/3/main}
\include{chapters/5Experiments/4/main}

\include{chapters/6Conclusion/main}

\appendix
\include{appendix/AppendixB}
\include{appendix/D/main}
\include{appendix/AppendixC}

\backmatter
\bibliographystyle{ecs}
\bibliography{ECS}
\end{document}
%% ----------------------------------------------------------------

\include{appendix/AppendixC}

\backmatter
\bibliographystyle{ecs}
\bibliography{ECS}
\end{document}
%% ----------------------------------------------------------------


 %% ----------------------------------------------------------------
%% Progress.tex
%% ---------------------------------------------------------------- 
\documentclass{ecsprogress}    % Use the progress Style
\graphicspath{{../figs/}}   % Location of your graphics files
    \usepackage{natbib}            % Use Natbib style for the refs.
\hypersetup{colorlinks=true}   % Set to false for black/white printing
\input{Definitions}            % Include your abbreviations



\usepackage{enumitem}% http://ctan.org/pkg/enumitem
\usepackage{multirow}
\usepackage{float}
\usepackage{amsmath}
\usepackage{multicol}
\usepackage{amssymb}
\usepackage[normalem]{ulem}
\useunder{\uline}{\ul}{}
\usepackage{wrapfig}


\usepackage[table,xcdraw]{xcolor}


%% ----------------------------------------------------------------
\begin{document}
\frontmatter
\title      {Heterogeneous Agent-based Model for Supermarket Competition}
\authors    {\texorpdfstring
             {\href{mailto:sc22g13@ecs.soton.ac.uk}{Stefan J. Collier}}
             {Stefan J. Collier}
            }
\addresses  {\groupname\\\deptname\\\univname}
\date       {\today}
\subject    {}
\keywords   {}
\supervisor {Dr. Maria Polukarov}
\examiner   {Professor Sheng Chen}

\maketitle
\begin{abstract}
This project aim was to model and analyse the effects of competitive pricing behaviors of grocery retailers on the British market. 

This was achieved by creating a multi-agent model, containing retailer and consumer agents. The heterogeneous crowd of retailers employs either a uniform pricing strategy or a ‘local price flexing’ strategy. The actions of these retailers are chosen by predicting the profit of each action, using a perceptron. Following on from the consideration of different economic models, a discrete model was developed so that software agents have a discrete environment to operate within. Within the model, it has been observed how supermarkets with differing behaviors affect a heterogeneous crowd of consumer agents. The model was implemented in Java with Python used to evaluate the results. 

The simulation displays good acceptance with real grocery market behavior, i.e. captures the performance of British retailers thus can be used to determine the impact of changes in their behavior on their competitors and consumers.Furthermore it can be used to provide insight into sustainability of volatile pricing strategies, providing a useful insight in volatility of British supermarket retail industry. 
\end{abstract}
\acknowledgements{
I would like to express my sincere gratitude to Dr Maria Polukarov for her guidance and support which provided me the freedom to take this research in the direction of my interest.\\
\\
I would also like to thank my family and friends for their encouragement and support. To those who quietly listened to my software complaints. To those who worked throughout the nights with me. To those who helped me write what I couldn't say. I cannot thank you enough.
}

\declaration{
I, Stefan Collier, declare that this dissertation and the work presented in it are my own and has been generated by me as the result of my own original research.\\
I confirm that:\\
1. This work was done wholly or mainly while in candidature for a degree at this University;\\
2. Where any part of this dissertation has previously been submitted for any other qualification at this University or any other institution, this has been clearly stated;\\
3. Where I have consulted the published work of others, this is always clearly attributed;\\
4. Where I have quoted from the work of others, the source is always given. With the exception of such quotations, this dissertation is entirely my own work;\\
5. I have acknowledged all main sources of help;\\
6. Where the thesis is based on work done by myself jointly with others, I have made clear exactly what was done by others and what I have contributed myself;\\
7. Either none of this work has been published before submission, or parts of this work have been published by :\\
\\
Stefan Collier\\
April 2016
}
\tableofcontents
\listoffigures
\listoftables

\mainmatter
%% ----------------------------------------------------------------
%\include{Introduction}
%\include{Conclusions}
 %% ----------------------------------------------------------------
%% Progress.tex
%% ---------------------------------------------------------------- 
\documentclass{ecsprogress}    % Use the progress Style
\graphicspath{{../figs/}}   % Location of your graphics files
    \usepackage{natbib}            % Use Natbib style for the refs.
\hypersetup{colorlinks=true}   % Set to false for black/white printing
\input{Definitions}            % Include your abbreviations



\usepackage{enumitem}% http://ctan.org/pkg/enumitem
\usepackage{multirow}
\usepackage{float}
\usepackage{amsmath}
\usepackage{multicol}
\usepackage{amssymb}
\usepackage[normalem]{ulem}
\useunder{\uline}{\ul}{}
\usepackage{wrapfig}


\usepackage[table,xcdraw]{xcolor}


%% ----------------------------------------------------------------
\begin{document}
\frontmatter
\title      {Heterogeneous Agent-based Model for Supermarket Competition}
\authors    {\texorpdfstring
             {\href{mailto:sc22g13@ecs.soton.ac.uk}{Stefan J. Collier}}
             {Stefan J. Collier}
            }
\addresses  {\groupname\\\deptname\\\univname}
\date       {\today}
\subject    {}
\keywords   {}
\supervisor {Dr. Maria Polukarov}
\examiner   {Professor Sheng Chen}

\maketitle
\begin{abstract}
This project aim was to model and analyse the effects of competitive pricing behaviors of grocery retailers on the British market. 

This was achieved by creating a multi-agent model, containing retailer and consumer agents. The heterogeneous crowd of retailers employs either a uniform pricing strategy or a ‘local price flexing’ strategy. The actions of these retailers are chosen by predicting the profit of each action, using a perceptron. Following on from the consideration of different economic models, a discrete model was developed so that software agents have a discrete environment to operate within. Within the model, it has been observed how supermarkets with differing behaviors affect a heterogeneous crowd of consumer agents. The model was implemented in Java with Python used to evaluate the results. 

The simulation displays good acceptance with real grocery market behavior, i.e. captures the performance of British retailers thus can be used to determine the impact of changes in their behavior on their competitors and consumers.Furthermore it can be used to provide insight into sustainability of volatile pricing strategies, providing a useful insight in volatility of British supermarket retail industry. 
\end{abstract}
\acknowledgements{
I would like to express my sincere gratitude to Dr Maria Polukarov for her guidance and support which provided me the freedom to take this research in the direction of my interest.\\
\\
I would also like to thank my family and friends for their encouragement and support. To those who quietly listened to my software complaints. To those who worked throughout the nights with me. To those who helped me write what I couldn't say. I cannot thank you enough.
}

\declaration{
I, Stefan Collier, declare that this dissertation and the work presented in it are my own and has been generated by me as the result of my own original research.\\
I confirm that:\\
1. This work was done wholly or mainly while in candidature for a degree at this University;\\
2. Where any part of this dissertation has previously been submitted for any other qualification at this University or any other institution, this has been clearly stated;\\
3. Where I have consulted the published work of others, this is always clearly attributed;\\
4. Where I have quoted from the work of others, the source is always given. With the exception of such quotations, this dissertation is entirely my own work;\\
5. I have acknowledged all main sources of help;\\
6. Where the thesis is based on work done by myself jointly with others, I have made clear exactly what was done by others and what I have contributed myself;\\
7. Either none of this work has been published before submission, or parts of this work have been published by :\\
\\
Stefan Collier\\
April 2016
}
\tableofcontents
\listoffigures
\listoftables

\mainmatter
%% ----------------------------------------------------------------
%\include{Introduction}
%\include{Conclusions}
\include{chapters/1Project/main}
\include{chapters/2Lit/main}
\include{chapters/3Design/HighLevel}
\include{chapters/3Design/InDepth}
\include{chapters/4Impl/main}

\include{chapters/5Experiments/1/main}
\include{chapters/5Experiments/2/main}
\include{chapters/5Experiments/3/main}
\include{chapters/5Experiments/4/main}

\include{chapters/6Conclusion/main}

\appendix
\include{appendix/AppendixB}
\include{appendix/D/main}
\include{appendix/AppendixC}

\backmatter
\bibliographystyle{ecs}
\bibliography{ECS}
\end{document}
%% ----------------------------------------------------------------

 %% ----------------------------------------------------------------
%% Progress.tex
%% ---------------------------------------------------------------- 
\documentclass{ecsprogress}    % Use the progress Style
\graphicspath{{../figs/}}   % Location of your graphics files
    \usepackage{natbib}            % Use Natbib style for the refs.
\hypersetup{colorlinks=true}   % Set to false for black/white printing
\input{Definitions}            % Include your abbreviations



\usepackage{enumitem}% http://ctan.org/pkg/enumitem
\usepackage{multirow}
\usepackage{float}
\usepackage{amsmath}
\usepackage{multicol}
\usepackage{amssymb}
\usepackage[normalem]{ulem}
\useunder{\uline}{\ul}{}
\usepackage{wrapfig}


\usepackage[table,xcdraw]{xcolor}


%% ----------------------------------------------------------------
\begin{document}
\frontmatter
\title      {Heterogeneous Agent-based Model for Supermarket Competition}
\authors    {\texorpdfstring
             {\href{mailto:sc22g13@ecs.soton.ac.uk}{Stefan J. Collier}}
             {Stefan J. Collier}
            }
\addresses  {\groupname\\\deptname\\\univname}
\date       {\today}
\subject    {}
\keywords   {}
\supervisor {Dr. Maria Polukarov}
\examiner   {Professor Sheng Chen}

\maketitle
\begin{abstract}
This project aim was to model and analyse the effects of competitive pricing behaviors of grocery retailers on the British market. 

This was achieved by creating a multi-agent model, containing retailer and consumer agents. The heterogeneous crowd of retailers employs either a uniform pricing strategy or a ‘local price flexing’ strategy. The actions of these retailers are chosen by predicting the profit of each action, using a perceptron. Following on from the consideration of different economic models, a discrete model was developed so that software agents have a discrete environment to operate within. Within the model, it has been observed how supermarkets with differing behaviors affect a heterogeneous crowd of consumer agents. The model was implemented in Java with Python used to evaluate the results. 

The simulation displays good acceptance with real grocery market behavior, i.e. captures the performance of British retailers thus can be used to determine the impact of changes in their behavior on their competitors and consumers.Furthermore it can be used to provide insight into sustainability of volatile pricing strategies, providing a useful insight in volatility of British supermarket retail industry. 
\end{abstract}
\acknowledgements{
I would like to express my sincere gratitude to Dr Maria Polukarov for her guidance and support which provided me the freedom to take this research in the direction of my interest.\\
\\
I would also like to thank my family and friends for their encouragement and support. To those who quietly listened to my software complaints. To those who worked throughout the nights with me. To those who helped me write what I couldn't say. I cannot thank you enough.
}

\declaration{
I, Stefan Collier, declare that this dissertation and the work presented in it are my own and has been generated by me as the result of my own original research.\\
I confirm that:\\
1. This work was done wholly or mainly while in candidature for a degree at this University;\\
2. Where any part of this dissertation has previously been submitted for any other qualification at this University or any other institution, this has been clearly stated;\\
3. Where I have consulted the published work of others, this is always clearly attributed;\\
4. Where I have quoted from the work of others, the source is always given. With the exception of such quotations, this dissertation is entirely my own work;\\
5. I have acknowledged all main sources of help;\\
6. Where the thesis is based on work done by myself jointly with others, I have made clear exactly what was done by others and what I have contributed myself;\\
7. Either none of this work has been published before submission, or parts of this work have been published by :\\
\\
Stefan Collier\\
April 2016
}
\tableofcontents
\listoffigures
\listoftables

\mainmatter
%% ----------------------------------------------------------------
%\include{Introduction}
%\include{Conclusions}
\include{chapters/1Project/main}
\include{chapters/2Lit/main}
\include{chapters/3Design/HighLevel}
\include{chapters/3Design/InDepth}
\include{chapters/4Impl/main}

\include{chapters/5Experiments/1/main}
\include{chapters/5Experiments/2/main}
\include{chapters/5Experiments/3/main}
\include{chapters/5Experiments/4/main}

\include{chapters/6Conclusion/main}

\appendix
\include{appendix/AppendixB}
\include{appendix/D/main}
\include{appendix/AppendixC}

\backmatter
\bibliographystyle{ecs}
\bibliography{ECS}
\end{document}
%% ----------------------------------------------------------------

\include{chapters/3Design/HighLevel}
\include{chapters/3Design/InDepth}
 %% ----------------------------------------------------------------
%% Progress.tex
%% ---------------------------------------------------------------- 
\documentclass{ecsprogress}    % Use the progress Style
\graphicspath{{../figs/}}   % Location of your graphics files
    \usepackage{natbib}            % Use Natbib style for the refs.
\hypersetup{colorlinks=true}   % Set to false for black/white printing
\input{Definitions}            % Include your abbreviations



\usepackage{enumitem}% http://ctan.org/pkg/enumitem
\usepackage{multirow}
\usepackage{float}
\usepackage{amsmath}
\usepackage{multicol}
\usepackage{amssymb}
\usepackage[normalem]{ulem}
\useunder{\uline}{\ul}{}
\usepackage{wrapfig}


\usepackage[table,xcdraw]{xcolor}


%% ----------------------------------------------------------------
\begin{document}
\frontmatter
\title      {Heterogeneous Agent-based Model for Supermarket Competition}
\authors    {\texorpdfstring
             {\href{mailto:sc22g13@ecs.soton.ac.uk}{Stefan J. Collier}}
             {Stefan J. Collier}
            }
\addresses  {\groupname\\\deptname\\\univname}
\date       {\today}
\subject    {}
\keywords   {}
\supervisor {Dr. Maria Polukarov}
\examiner   {Professor Sheng Chen}

\maketitle
\begin{abstract}
This project aim was to model and analyse the effects of competitive pricing behaviors of grocery retailers on the British market. 

This was achieved by creating a multi-agent model, containing retailer and consumer agents. The heterogeneous crowd of retailers employs either a uniform pricing strategy or a ‘local price flexing’ strategy. The actions of these retailers are chosen by predicting the profit of each action, using a perceptron. Following on from the consideration of different economic models, a discrete model was developed so that software agents have a discrete environment to operate within. Within the model, it has been observed how supermarkets with differing behaviors affect a heterogeneous crowd of consumer agents. The model was implemented in Java with Python used to evaluate the results. 

The simulation displays good acceptance with real grocery market behavior, i.e. captures the performance of British retailers thus can be used to determine the impact of changes in their behavior on their competitors and consumers.Furthermore it can be used to provide insight into sustainability of volatile pricing strategies, providing a useful insight in volatility of British supermarket retail industry. 
\end{abstract}
\acknowledgements{
I would like to express my sincere gratitude to Dr Maria Polukarov for her guidance and support which provided me the freedom to take this research in the direction of my interest.\\
\\
I would also like to thank my family and friends for their encouragement and support. To those who quietly listened to my software complaints. To those who worked throughout the nights with me. To those who helped me write what I couldn't say. I cannot thank you enough.
}

\declaration{
I, Stefan Collier, declare that this dissertation and the work presented in it are my own and has been generated by me as the result of my own original research.\\
I confirm that:\\
1. This work was done wholly or mainly while in candidature for a degree at this University;\\
2. Where any part of this dissertation has previously been submitted for any other qualification at this University or any other institution, this has been clearly stated;\\
3. Where I have consulted the published work of others, this is always clearly attributed;\\
4. Where I have quoted from the work of others, the source is always given. With the exception of such quotations, this dissertation is entirely my own work;\\
5. I have acknowledged all main sources of help;\\
6. Where the thesis is based on work done by myself jointly with others, I have made clear exactly what was done by others and what I have contributed myself;\\
7. Either none of this work has been published before submission, or parts of this work have been published by :\\
\\
Stefan Collier\\
April 2016
}
\tableofcontents
\listoffigures
\listoftables

\mainmatter
%% ----------------------------------------------------------------
%\include{Introduction}
%\include{Conclusions}
\include{chapters/1Project/main}
\include{chapters/2Lit/main}
\include{chapters/3Design/HighLevel}
\include{chapters/3Design/InDepth}
\include{chapters/4Impl/main}

\include{chapters/5Experiments/1/main}
\include{chapters/5Experiments/2/main}
\include{chapters/5Experiments/3/main}
\include{chapters/5Experiments/4/main}

\include{chapters/6Conclusion/main}

\appendix
\include{appendix/AppendixB}
\include{appendix/D/main}
\include{appendix/AppendixC}

\backmatter
\bibliographystyle{ecs}
\bibliography{ECS}
\end{document}
%% ----------------------------------------------------------------


 %% ----------------------------------------------------------------
%% Progress.tex
%% ---------------------------------------------------------------- 
\documentclass{ecsprogress}    % Use the progress Style
\graphicspath{{../figs/}}   % Location of your graphics files
    \usepackage{natbib}            % Use Natbib style for the refs.
\hypersetup{colorlinks=true}   % Set to false for black/white printing
\input{Definitions}            % Include your abbreviations



\usepackage{enumitem}% http://ctan.org/pkg/enumitem
\usepackage{multirow}
\usepackage{float}
\usepackage{amsmath}
\usepackage{multicol}
\usepackage{amssymb}
\usepackage[normalem]{ulem}
\useunder{\uline}{\ul}{}
\usepackage{wrapfig}


\usepackage[table,xcdraw]{xcolor}


%% ----------------------------------------------------------------
\begin{document}
\frontmatter
\title      {Heterogeneous Agent-based Model for Supermarket Competition}
\authors    {\texorpdfstring
             {\href{mailto:sc22g13@ecs.soton.ac.uk}{Stefan J. Collier}}
             {Stefan J. Collier}
            }
\addresses  {\groupname\\\deptname\\\univname}
\date       {\today}
\subject    {}
\keywords   {}
\supervisor {Dr. Maria Polukarov}
\examiner   {Professor Sheng Chen}

\maketitle
\begin{abstract}
This project aim was to model and analyse the effects of competitive pricing behaviors of grocery retailers on the British market. 

This was achieved by creating a multi-agent model, containing retailer and consumer agents. The heterogeneous crowd of retailers employs either a uniform pricing strategy or a ‘local price flexing’ strategy. The actions of these retailers are chosen by predicting the profit of each action, using a perceptron. Following on from the consideration of different economic models, a discrete model was developed so that software agents have a discrete environment to operate within. Within the model, it has been observed how supermarkets with differing behaviors affect a heterogeneous crowd of consumer agents. The model was implemented in Java with Python used to evaluate the results. 

The simulation displays good acceptance with real grocery market behavior, i.e. captures the performance of British retailers thus can be used to determine the impact of changes in their behavior on their competitors and consumers.Furthermore it can be used to provide insight into sustainability of volatile pricing strategies, providing a useful insight in volatility of British supermarket retail industry. 
\end{abstract}
\acknowledgements{
I would like to express my sincere gratitude to Dr Maria Polukarov for her guidance and support which provided me the freedom to take this research in the direction of my interest.\\
\\
I would also like to thank my family and friends for their encouragement and support. To those who quietly listened to my software complaints. To those who worked throughout the nights with me. To those who helped me write what I couldn't say. I cannot thank you enough.
}

\declaration{
I, Stefan Collier, declare that this dissertation and the work presented in it are my own and has been generated by me as the result of my own original research.\\
I confirm that:\\
1. This work was done wholly or mainly while in candidature for a degree at this University;\\
2. Where any part of this dissertation has previously been submitted for any other qualification at this University or any other institution, this has been clearly stated;\\
3. Where I have consulted the published work of others, this is always clearly attributed;\\
4. Where I have quoted from the work of others, the source is always given. With the exception of such quotations, this dissertation is entirely my own work;\\
5. I have acknowledged all main sources of help;\\
6. Where the thesis is based on work done by myself jointly with others, I have made clear exactly what was done by others and what I have contributed myself;\\
7. Either none of this work has been published before submission, or parts of this work have been published by :\\
\\
Stefan Collier\\
April 2016
}
\tableofcontents
\listoffigures
\listoftables

\mainmatter
%% ----------------------------------------------------------------
%\include{Introduction}
%\include{Conclusions}
\include{chapters/1Project/main}
\include{chapters/2Lit/main}
\include{chapters/3Design/HighLevel}
\include{chapters/3Design/InDepth}
\include{chapters/4Impl/main}

\include{chapters/5Experiments/1/main}
\include{chapters/5Experiments/2/main}
\include{chapters/5Experiments/3/main}
\include{chapters/5Experiments/4/main}

\include{chapters/6Conclusion/main}

\appendix
\include{appendix/AppendixB}
\include{appendix/D/main}
\include{appendix/AppendixC}

\backmatter
\bibliographystyle{ecs}
\bibliography{ECS}
\end{document}
%% ----------------------------------------------------------------

 %% ----------------------------------------------------------------
%% Progress.tex
%% ---------------------------------------------------------------- 
\documentclass{ecsprogress}    % Use the progress Style
\graphicspath{{../figs/}}   % Location of your graphics files
    \usepackage{natbib}            % Use Natbib style for the refs.
\hypersetup{colorlinks=true}   % Set to false for black/white printing
\input{Definitions}            % Include your abbreviations



\usepackage{enumitem}% http://ctan.org/pkg/enumitem
\usepackage{multirow}
\usepackage{float}
\usepackage{amsmath}
\usepackage{multicol}
\usepackage{amssymb}
\usepackage[normalem]{ulem}
\useunder{\uline}{\ul}{}
\usepackage{wrapfig}


\usepackage[table,xcdraw]{xcolor}


%% ----------------------------------------------------------------
\begin{document}
\frontmatter
\title      {Heterogeneous Agent-based Model for Supermarket Competition}
\authors    {\texorpdfstring
             {\href{mailto:sc22g13@ecs.soton.ac.uk}{Stefan J. Collier}}
             {Stefan J. Collier}
            }
\addresses  {\groupname\\\deptname\\\univname}
\date       {\today}
\subject    {}
\keywords   {}
\supervisor {Dr. Maria Polukarov}
\examiner   {Professor Sheng Chen}

\maketitle
\begin{abstract}
This project aim was to model and analyse the effects of competitive pricing behaviors of grocery retailers on the British market. 

This was achieved by creating a multi-agent model, containing retailer and consumer agents. The heterogeneous crowd of retailers employs either a uniform pricing strategy or a ‘local price flexing’ strategy. The actions of these retailers are chosen by predicting the profit of each action, using a perceptron. Following on from the consideration of different economic models, a discrete model was developed so that software agents have a discrete environment to operate within. Within the model, it has been observed how supermarkets with differing behaviors affect a heterogeneous crowd of consumer agents. The model was implemented in Java with Python used to evaluate the results. 

The simulation displays good acceptance with real grocery market behavior, i.e. captures the performance of British retailers thus can be used to determine the impact of changes in their behavior on their competitors and consumers.Furthermore it can be used to provide insight into sustainability of volatile pricing strategies, providing a useful insight in volatility of British supermarket retail industry. 
\end{abstract}
\acknowledgements{
I would like to express my sincere gratitude to Dr Maria Polukarov for her guidance and support which provided me the freedom to take this research in the direction of my interest.\\
\\
I would also like to thank my family and friends for their encouragement and support. To those who quietly listened to my software complaints. To those who worked throughout the nights with me. To those who helped me write what I couldn't say. I cannot thank you enough.
}

\declaration{
I, Stefan Collier, declare that this dissertation and the work presented in it are my own and has been generated by me as the result of my own original research.\\
I confirm that:\\
1. This work was done wholly or mainly while in candidature for a degree at this University;\\
2. Where any part of this dissertation has previously been submitted for any other qualification at this University or any other institution, this has been clearly stated;\\
3. Where I have consulted the published work of others, this is always clearly attributed;\\
4. Where I have quoted from the work of others, the source is always given. With the exception of such quotations, this dissertation is entirely my own work;\\
5. I have acknowledged all main sources of help;\\
6. Where the thesis is based on work done by myself jointly with others, I have made clear exactly what was done by others and what I have contributed myself;\\
7. Either none of this work has been published before submission, or parts of this work have been published by :\\
\\
Stefan Collier\\
April 2016
}
\tableofcontents
\listoffigures
\listoftables

\mainmatter
%% ----------------------------------------------------------------
%\include{Introduction}
%\include{Conclusions}
\include{chapters/1Project/main}
\include{chapters/2Lit/main}
\include{chapters/3Design/HighLevel}
\include{chapters/3Design/InDepth}
\include{chapters/4Impl/main}

\include{chapters/5Experiments/1/main}
\include{chapters/5Experiments/2/main}
\include{chapters/5Experiments/3/main}
\include{chapters/5Experiments/4/main}

\include{chapters/6Conclusion/main}

\appendix
\include{appendix/AppendixB}
\include{appendix/D/main}
\include{appendix/AppendixC}

\backmatter
\bibliographystyle{ecs}
\bibliography{ECS}
\end{document}
%% ----------------------------------------------------------------

 %% ----------------------------------------------------------------
%% Progress.tex
%% ---------------------------------------------------------------- 
\documentclass{ecsprogress}    % Use the progress Style
\graphicspath{{../figs/}}   % Location of your graphics files
    \usepackage{natbib}            % Use Natbib style for the refs.
\hypersetup{colorlinks=true}   % Set to false for black/white printing
\input{Definitions}            % Include your abbreviations



\usepackage{enumitem}% http://ctan.org/pkg/enumitem
\usepackage{multirow}
\usepackage{float}
\usepackage{amsmath}
\usepackage{multicol}
\usepackage{amssymb}
\usepackage[normalem]{ulem}
\useunder{\uline}{\ul}{}
\usepackage{wrapfig}


\usepackage[table,xcdraw]{xcolor}


%% ----------------------------------------------------------------
\begin{document}
\frontmatter
\title      {Heterogeneous Agent-based Model for Supermarket Competition}
\authors    {\texorpdfstring
             {\href{mailto:sc22g13@ecs.soton.ac.uk}{Stefan J. Collier}}
             {Stefan J. Collier}
            }
\addresses  {\groupname\\\deptname\\\univname}
\date       {\today}
\subject    {}
\keywords   {}
\supervisor {Dr. Maria Polukarov}
\examiner   {Professor Sheng Chen}

\maketitle
\begin{abstract}
This project aim was to model and analyse the effects of competitive pricing behaviors of grocery retailers on the British market. 

This was achieved by creating a multi-agent model, containing retailer and consumer agents. The heterogeneous crowd of retailers employs either a uniform pricing strategy or a ‘local price flexing’ strategy. The actions of these retailers are chosen by predicting the profit of each action, using a perceptron. Following on from the consideration of different economic models, a discrete model was developed so that software agents have a discrete environment to operate within. Within the model, it has been observed how supermarkets with differing behaviors affect a heterogeneous crowd of consumer agents. The model was implemented in Java with Python used to evaluate the results. 

The simulation displays good acceptance with real grocery market behavior, i.e. captures the performance of British retailers thus can be used to determine the impact of changes in their behavior on their competitors and consumers.Furthermore it can be used to provide insight into sustainability of volatile pricing strategies, providing a useful insight in volatility of British supermarket retail industry. 
\end{abstract}
\acknowledgements{
I would like to express my sincere gratitude to Dr Maria Polukarov for her guidance and support which provided me the freedom to take this research in the direction of my interest.\\
\\
I would also like to thank my family and friends for their encouragement and support. To those who quietly listened to my software complaints. To those who worked throughout the nights with me. To those who helped me write what I couldn't say. I cannot thank you enough.
}

\declaration{
I, Stefan Collier, declare that this dissertation and the work presented in it are my own and has been generated by me as the result of my own original research.\\
I confirm that:\\
1. This work was done wholly or mainly while in candidature for a degree at this University;\\
2. Where any part of this dissertation has previously been submitted for any other qualification at this University or any other institution, this has been clearly stated;\\
3. Where I have consulted the published work of others, this is always clearly attributed;\\
4. Where I have quoted from the work of others, the source is always given. With the exception of such quotations, this dissertation is entirely my own work;\\
5. I have acknowledged all main sources of help;\\
6. Where the thesis is based on work done by myself jointly with others, I have made clear exactly what was done by others and what I have contributed myself;\\
7. Either none of this work has been published before submission, or parts of this work have been published by :\\
\\
Stefan Collier\\
April 2016
}
\tableofcontents
\listoffigures
\listoftables

\mainmatter
%% ----------------------------------------------------------------
%\include{Introduction}
%\include{Conclusions}
\include{chapters/1Project/main}
\include{chapters/2Lit/main}
\include{chapters/3Design/HighLevel}
\include{chapters/3Design/InDepth}
\include{chapters/4Impl/main}

\include{chapters/5Experiments/1/main}
\include{chapters/5Experiments/2/main}
\include{chapters/5Experiments/3/main}
\include{chapters/5Experiments/4/main}

\include{chapters/6Conclusion/main}

\appendix
\include{appendix/AppendixB}
\include{appendix/D/main}
\include{appendix/AppendixC}

\backmatter
\bibliographystyle{ecs}
\bibliography{ECS}
\end{document}
%% ----------------------------------------------------------------

 %% ----------------------------------------------------------------
%% Progress.tex
%% ---------------------------------------------------------------- 
\documentclass{ecsprogress}    % Use the progress Style
\graphicspath{{../figs/}}   % Location of your graphics files
    \usepackage{natbib}            % Use Natbib style for the refs.
\hypersetup{colorlinks=true}   % Set to false for black/white printing
\input{Definitions}            % Include your abbreviations



\usepackage{enumitem}% http://ctan.org/pkg/enumitem
\usepackage{multirow}
\usepackage{float}
\usepackage{amsmath}
\usepackage{multicol}
\usepackage{amssymb}
\usepackage[normalem]{ulem}
\useunder{\uline}{\ul}{}
\usepackage{wrapfig}


\usepackage[table,xcdraw]{xcolor}


%% ----------------------------------------------------------------
\begin{document}
\frontmatter
\title      {Heterogeneous Agent-based Model for Supermarket Competition}
\authors    {\texorpdfstring
             {\href{mailto:sc22g13@ecs.soton.ac.uk}{Stefan J. Collier}}
             {Stefan J. Collier}
            }
\addresses  {\groupname\\\deptname\\\univname}
\date       {\today}
\subject    {}
\keywords   {}
\supervisor {Dr. Maria Polukarov}
\examiner   {Professor Sheng Chen}

\maketitle
\begin{abstract}
This project aim was to model and analyse the effects of competitive pricing behaviors of grocery retailers on the British market. 

This was achieved by creating a multi-agent model, containing retailer and consumer agents. The heterogeneous crowd of retailers employs either a uniform pricing strategy or a ‘local price flexing’ strategy. The actions of these retailers are chosen by predicting the profit of each action, using a perceptron. Following on from the consideration of different economic models, a discrete model was developed so that software agents have a discrete environment to operate within. Within the model, it has been observed how supermarkets with differing behaviors affect a heterogeneous crowd of consumer agents. The model was implemented in Java with Python used to evaluate the results. 

The simulation displays good acceptance with real grocery market behavior, i.e. captures the performance of British retailers thus can be used to determine the impact of changes in their behavior on their competitors and consumers.Furthermore it can be used to provide insight into sustainability of volatile pricing strategies, providing a useful insight in volatility of British supermarket retail industry. 
\end{abstract}
\acknowledgements{
I would like to express my sincere gratitude to Dr Maria Polukarov for her guidance and support which provided me the freedom to take this research in the direction of my interest.\\
\\
I would also like to thank my family and friends for their encouragement and support. To those who quietly listened to my software complaints. To those who worked throughout the nights with me. To those who helped me write what I couldn't say. I cannot thank you enough.
}

\declaration{
I, Stefan Collier, declare that this dissertation and the work presented in it are my own and has been generated by me as the result of my own original research.\\
I confirm that:\\
1. This work was done wholly or mainly while in candidature for a degree at this University;\\
2. Where any part of this dissertation has previously been submitted for any other qualification at this University or any other institution, this has been clearly stated;\\
3. Where I have consulted the published work of others, this is always clearly attributed;\\
4. Where I have quoted from the work of others, the source is always given. With the exception of such quotations, this dissertation is entirely my own work;\\
5. I have acknowledged all main sources of help;\\
6. Where the thesis is based on work done by myself jointly with others, I have made clear exactly what was done by others and what I have contributed myself;\\
7. Either none of this work has been published before submission, or parts of this work have been published by :\\
\\
Stefan Collier\\
April 2016
}
\tableofcontents
\listoffigures
\listoftables

\mainmatter
%% ----------------------------------------------------------------
%\include{Introduction}
%\include{Conclusions}
\include{chapters/1Project/main}
\include{chapters/2Lit/main}
\include{chapters/3Design/HighLevel}
\include{chapters/3Design/InDepth}
\include{chapters/4Impl/main}

\include{chapters/5Experiments/1/main}
\include{chapters/5Experiments/2/main}
\include{chapters/5Experiments/3/main}
\include{chapters/5Experiments/4/main}

\include{chapters/6Conclusion/main}

\appendix
\include{appendix/AppendixB}
\include{appendix/D/main}
\include{appendix/AppendixC}

\backmatter
\bibliographystyle{ecs}
\bibliography{ECS}
\end{document}
%% ----------------------------------------------------------------


 %% ----------------------------------------------------------------
%% Progress.tex
%% ---------------------------------------------------------------- 
\documentclass{ecsprogress}    % Use the progress Style
\graphicspath{{../figs/}}   % Location of your graphics files
    \usepackage{natbib}            % Use Natbib style for the refs.
\hypersetup{colorlinks=true}   % Set to false for black/white printing
\input{Definitions}            % Include your abbreviations



\usepackage{enumitem}% http://ctan.org/pkg/enumitem
\usepackage{multirow}
\usepackage{float}
\usepackage{amsmath}
\usepackage{multicol}
\usepackage{amssymb}
\usepackage[normalem]{ulem}
\useunder{\uline}{\ul}{}
\usepackage{wrapfig}


\usepackage[table,xcdraw]{xcolor}


%% ----------------------------------------------------------------
\begin{document}
\frontmatter
\title      {Heterogeneous Agent-based Model for Supermarket Competition}
\authors    {\texorpdfstring
             {\href{mailto:sc22g13@ecs.soton.ac.uk}{Stefan J. Collier}}
             {Stefan J. Collier}
            }
\addresses  {\groupname\\\deptname\\\univname}
\date       {\today}
\subject    {}
\keywords   {}
\supervisor {Dr. Maria Polukarov}
\examiner   {Professor Sheng Chen}

\maketitle
\begin{abstract}
This project aim was to model and analyse the effects of competitive pricing behaviors of grocery retailers on the British market. 

This was achieved by creating a multi-agent model, containing retailer and consumer agents. The heterogeneous crowd of retailers employs either a uniform pricing strategy or a ‘local price flexing’ strategy. The actions of these retailers are chosen by predicting the profit of each action, using a perceptron. Following on from the consideration of different economic models, a discrete model was developed so that software agents have a discrete environment to operate within. Within the model, it has been observed how supermarkets with differing behaviors affect a heterogeneous crowd of consumer agents. The model was implemented in Java with Python used to evaluate the results. 

The simulation displays good acceptance with real grocery market behavior, i.e. captures the performance of British retailers thus can be used to determine the impact of changes in their behavior on their competitors and consumers.Furthermore it can be used to provide insight into sustainability of volatile pricing strategies, providing a useful insight in volatility of British supermarket retail industry. 
\end{abstract}
\acknowledgements{
I would like to express my sincere gratitude to Dr Maria Polukarov for her guidance and support which provided me the freedom to take this research in the direction of my interest.\\
\\
I would also like to thank my family and friends for their encouragement and support. To those who quietly listened to my software complaints. To those who worked throughout the nights with me. To those who helped me write what I couldn't say. I cannot thank you enough.
}

\declaration{
I, Stefan Collier, declare that this dissertation and the work presented in it are my own and has been generated by me as the result of my own original research.\\
I confirm that:\\
1. This work was done wholly or mainly while in candidature for a degree at this University;\\
2. Where any part of this dissertation has previously been submitted for any other qualification at this University or any other institution, this has been clearly stated;\\
3. Where I have consulted the published work of others, this is always clearly attributed;\\
4. Where I have quoted from the work of others, the source is always given. With the exception of such quotations, this dissertation is entirely my own work;\\
5. I have acknowledged all main sources of help;\\
6. Where the thesis is based on work done by myself jointly with others, I have made clear exactly what was done by others and what I have contributed myself;\\
7. Either none of this work has been published before submission, or parts of this work have been published by :\\
\\
Stefan Collier\\
April 2016
}
\tableofcontents
\listoffigures
\listoftables

\mainmatter
%% ----------------------------------------------------------------
%\include{Introduction}
%\include{Conclusions}
\include{chapters/1Project/main}
\include{chapters/2Lit/main}
\include{chapters/3Design/HighLevel}
\include{chapters/3Design/InDepth}
\include{chapters/4Impl/main}

\include{chapters/5Experiments/1/main}
\include{chapters/5Experiments/2/main}
\include{chapters/5Experiments/3/main}
\include{chapters/5Experiments/4/main}

\include{chapters/6Conclusion/main}

\appendix
\include{appendix/AppendixB}
\include{appendix/D/main}
\include{appendix/AppendixC}

\backmatter
\bibliographystyle{ecs}
\bibliography{ECS}
\end{document}
%% ----------------------------------------------------------------


\appendix
\include{appendix/AppendixB}
 %% ----------------------------------------------------------------
%% Progress.tex
%% ---------------------------------------------------------------- 
\documentclass{ecsprogress}    % Use the progress Style
\graphicspath{{../figs/}}   % Location of your graphics files
    \usepackage{natbib}            % Use Natbib style for the refs.
\hypersetup{colorlinks=true}   % Set to false for black/white printing
\input{Definitions}            % Include your abbreviations



\usepackage{enumitem}% http://ctan.org/pkg/enumitem
\usepackage{multirow}
\usepackage{float}
\usepackage{amsmath}
\usepackage{multicol}
\usepackage{amssymb}
\usepackage[normalem]{ulem}
\useunder{\uline}{\ul}{}
\usepackage{wrapfig}


\usepackage[table,xcdraw]{xcolor}


%% ----------------------------------------------------------------
\begin{document}
\frontmatter
\title      {Heterogeneous Agent-based Model for Supermarket Competition}
\authors    {\texorpdfstring
             {\href{mailto:sc22g13@ecs.soton.ac.uk}{Stefan J. Collier}}
             {Stefan J. Collier}
            }
\addresses  {\groupname\\\deptname\\\univname}
\date       {\today}
\subject    {}
\keywords   {}
\supervisor {Dr. Maria Polukarov}
\examiner   {Professor Sheng Chen}

\maketitle
\begin{abstract}
This project aim was to model and analyse the effects of competitive pricing behaviors of grocery retailers on the British market. 

This was achieved by creating a multi-agent model, containing retailer and consumer agents. The heterogeneous crowd of retailers employs either a uniform pricing strategy or a ‘local price flexing’ strategy. The actions of these retailers are chosen by predicting the profit of each action, using a perceptron. Following on from the consideration of different economic models, a discrete model was developed so that software agents have a discrete environment to operate within. Within the model, it has been observed how supermarkets with differing behaviors affect a heterogeneous crowd of consumer agents. The model was implemented in Java with Python used to evaluate the results. 

The simulation displays good acceptance with real grocery market behavior, i.e. captures the performance of British retailers thus can be used to determine the impact of changes in their behavior on their competitors and consumers.Furthermore it can be used to provide insight into sustainability of volatile pricing strategies, providing a useful insight in volatility of British supermarket retail industry. 
\end{abstract}
\acknowledgements{
I would like to express my sincere gratitude to Dr Maria Polukarov for her guidance and support which provided me the freedom to take this research in the direction of my interest.\\
\\
I would also like to thank my family and friends for their encouragement and support. To those who quietly listened to my software complaints. To those who worked throughout the nights with me. To those who helped me write what I couldn't say. I cannot thank you enough.
}

\declaration{
I, Stefan Collier, declare that this dissertation and the work presented in it are my own and has been generated by me as the result of my own original research.\\
I confirm that:\\
1. This work was done wholly or mainly while in candidature for a degree at this University;\\
2. Where any part of this dissertation has previously been submitted for any other qualification at this University or any other institution, this has been clearly stated;\\
3. Where I have consulted the published work of others, this is always clearly attributed;\\
4. Where I have quoted from the work of others, the source is always given. With the exception of such quotations, this dissertation is entirely my own work;\\
5. I have acknowledged all main sources of help;\\
6. Where the thesis is based on work done by myself jointly with others, I have made clear exactly what was done by others and what I have contributed myself;\\
7. Either none of this work has been published before submission, or parts of this work have been published by :\\
\\
Stefan Collier\\
April 2016
}
\tableofcontents
\listoffigures
\listoftables

\mainmatter
%% ----------------------------------------------------------------
%\include{Introduction}
%\include{Conclusions}
\include{chapters/1Project/main}
\include{chapters/2Lit/main}
\include{chapters/3Design/HighLevel}
\include{chapters/3Design/InDepth}
\include{chapters/4Impl/main}

\include{chapters/5Experiments/1/main}
\include{chapters/5Experiments/2/main}
\include{chapters/5Experiments/3/main}
\include{chapters/5Experiments/4/main}

\include{chapters/6Conclusion/main}

\appendix
\include{appendix/AppendixB}
\include{appendix/D/main}
\include{appendix/AppendixC}

\backmatter
\bibliographystyle{ecs}
\bibliography{ECS}
\end{document}
%% ----------------------------------------------------------------

\include{appendix/AppendixC}

\backmatter
\bibliographystyle{ecs}
\bibliography{ECS}
\end{document}
%% ----------------------------------------------------------------

 %% ----------------------------------------------------------------
%% Progress.tex
%% ---------------------------------------------------------------- 
\documentclass{ecsprogress}    % Use the progress Style
\graphicspath{{../figs/}}   % Location of your graphics files
    \usepackage{natbib}            % Use Natbib style for the refs.
\hypersetup{colorlinks=true}   % Set to false for black/white printing
\input{Definitions}            % Include your abbreviations



\usepackage{enumitem}% http://ctan.org/pkg/enumitem
\usepackage{multirow}
\usepackage{float}
\usepackage{amsmath}
\usepackage{multicol}
\usepackage{amssymb}
\usepackage[normalem]{ulem}
\useunder{\uline}{\ul}{}
\usepackage{wrapfig}


\usepackage[table,xcdraw]{xcolor}


%% ----------------------------------------------------------------
\begin{document}
\frontmatter
\title      {Heterogeneous Agent-based Model for Supermarket Competition}
\authors    {\texorpdfstring
             {\href{mailto:sc22g13@ecs.soton.ac.uk}{Stefan J. Collier}}
             {Stefan J. Collier}
            }
\addresses  {\groupname\\\deptname\\\univname}
\date       {\today}
\subject    {}
\keywords   {}
\supervisor {Dr. Maria Polukarov}
\examiner   {Professor Sheng Chen}

\maketitle
\begin{abstract}
This project aim was to model and analyse the effects of competitive pricing behaviors of grocery retailers on the British market. 

This was achieved by creating a multi-agent model, containing retailer and consumer agents. The heterogeneous crowd of retailers employs either a uniform pricing strategy or a ‘local price flexing’ strategy. The actions of these retailers are chosen by predicting the profit of each action, using a perceptron. Following on from the consideration of different economic models, a discrete model was developed so that software agents have a discrete environment to operate within. Within the model, it has been observed how supermarkets with differing behaviors affect a heterogeneous crowd of consumer agents. The model was implemented in Java with Python used to evaluate the results. 

The simulation displays good acceptance with real grocery market behavior, i.e. captures the performance of British retailers thus can be used to determine the impact of changes in their behavior on their competitors and consumers.Furthermore it can be used to provide insight into sustainability of volatile pricing strategies, providing a useful insight in volatility of British supermarket retail industry. 
\end{abstract}
\acknowledgements{
I would like to express my sincere gratitude to Dr Maria Polukarov for her guidance and support which provided me the freedom to take this research in the direction of my interest.\\
\\
I would also like to thank my family and friends for their encouragement and support. To those who quietly listened to my software complaints. To those who worked throughout the nights with me. To those who helped me write what I couldn't say. I cannot thank you enough.
}

\declaration{
I, Stefan Collier, declare that this dissertation and the work presented in it are my own and has been generated by me as the result of my own original research.\\
I confirm that:\\
1. This work was done wholly or mainly while in candidature for a degree at this University;\\
2. Where any part of this dissertation has previously been submitted for any other qualification at this University or any other institution, this has been clearly stated;\\
3. Where I have consulted the published work of others, this is always clearly attributed;\\
4. Where I have quoted from the work of others, the source is always given. With the exception of such quotations, this dissertation is entirely my own work;\\
5. I have acknowledged all main sources of help;\\
6. Where the thesis is based on work done by myself jointly with others, I have made clear exactly what was done by others and what I have contributed myself;\\
7. Either none of this work has been published before submission, or parts of this work have been published by :\\
\\
Stefan Collier\\
April 2016
}
\tableofcontents
\listoffigures
\listoftables

\mainmatter
%% ----------------------------------------------------------------
%\include{Introduction}
%\include{Conclusions}
 %% ----------------------------------------------------------------
%% Progress.tex
%% ---------------------------------------------------------------- 
\documentclass{ecsprogress}    % Use the progress Style
\graphicspath{{../figs/}}   % Location of your graphics files
    \usepackage{natbib}            % Use Natbib style for the refs.
\hypersetup{colorlinks=true}   % Set to false for black/white printing
\input{Definitions}            % Include your abbreviations



\usepackage{enumitem}% http://ctan.org/pkg/enumitem
\usepackage{multirow}
\usepackage{float}
\usepackage{amsmath}
\usepackage{multicol}
\usepackage{amssymb}
\usepackage[normalem]{ulem}
\useunder{\uline}{\ul}{}
\usepackage{wrapfig}


\usepackage[table,xcdraw]{xcolor}


%% ----------------------------------------------------------------
\begin{document}
\frontmatter
\title      {Heterogeneous Agent-based Model for Supermarket Competition}
\authors    {\texorpdfstring
             {\href{mailto:sc22g13@ecs.soton.ac.uk}{Stefan J. Collier}}
             {Stefan J. Collier}
            }
\addresses  {\groupname\\\deptname\\\univname}
\date       {\today}
\subject    {}
\keywords   {}
\supervisor {Dr. Maria Polukarov}
\examiner   {Professor Sheng Chen}

\maketitle
\begin{abstract}
This project aim was to model and analyse the effects of competitive pricing behaviors of grocery retailers on the British market. 

This was achieved by creating a multi-agent model, containing retailer and consumer agents. The heterogeneous crowd of retailers employs either a uniform pricing strategy or a ‘local price flexing’ strategy. The actions of these retailers are chosen by predicting the profit of each action, using a perceptron. Following on from the consideration of different economic models, a discrete model was developed so that software agents have a discrete environment to operate within. Within the model, it has been observed how supermarkets with differing behaviors affect a heterogeneous crowd of consumer agents. The model was implemented in Java with Python used to evaluate the results. 

The simulation displays good acceptance with real grocery market behavior, i.e. captures the performance of British retailers thus can be used to determine the impact of changes in their behavior on their competitors and consumers.Furthermore it can be used to provide insight into sustainability of volatile pricing strategies, providing a useful insight in volatility of British supermarket retail industry. 
\end{abstract}
\acknowledgements{
I would like to express my sincere gratitude to Dr Maria Polukarov for her guidance and support which provided me the freedom to take this research in the direction of my interest.\\
\\
I would also like to thank my family and friends for their encouragement and support. To those who quietly listened to my software complaints. To those who worked throughout the nights with me. To those who helped me write what I couldn't say. I cannot thank you enough.
}

\declaration{
I, Stefan Collier, declare that this dissertation and the work presented in it are my own and has been generated by me as the result of my own original research.\\
I confirm that:\\
1. This work was done wholly or mainly while in candidature for a degree at this University;\\
2. Where any part of this dissertation has previously been submitted for any other qualification at this University or any other institution, this has been clearly stated;\\
3. Where I have consulted the published work of others, this is always clearly attributed;\\
4. Where I have quoted from the work of others, the source is always given. With the exception of such quotations, this dissertation is entirely my own work;\\
5. I have acknowledged all main sources of help;\\
6. Where the thesis is based on work done by myself jointly with others, I have made clear exactly what was done by others and what I have contributed myself;\\
7. Either none of this work has been published before submission, or parts of this work have been published by :\\
\\
Stefan Collier\\
April 2016
}
\tableofcontents
\listoffigures
\listoftables

\mainmatter
%% ----------------------------------------------------------------
%\include{Introduction}
%\include{Conclusions}
\include{chapters/1Project/main}
\include{chapters/2Lit/main}
\include{chapters/3Design/HighLevel}
\include{chapters/3Design/InDepth}
\include{chapters/4Impl/main}

\include{chapters/5Experiments/1/main}
\include{chapters/5Experiments/2/main}
\include{chapters/5Experiments/3/main}
\include{chapters/5Experiments/4/main}

\include{chapters/6Conclusion/main}

\appendix
\include{appendix/AppendixB}
\include{appendix/D/main}
\include{appendix/AppendixC}

\backmatter
\bibliographystyle{ecs}
\bibliography{ECS}
\end{document}
%% ----------------------------------------------------------------

 %% ----------------------------------------------------------------
%% Progress.tex
%% ---------------------------------------------------------------- 
\documentclass{ecsprogress}    % Use the progress Style
\graphicspath{{../figs/}}   % Location of your graphics files
    \usepackage{natbib}            % Use Natbib style for the refs.
\hypersetup{colorlinks=true}   % Set to false for black/white printing
\input{Definitions}            % Include your abbreviations



\usepackage{enumitem}% http://ctan.org/pkg/enumitem
\usepackage{multirow}
\usepackage{float}
\usepackage{amsmath}
\usepackage{multicol}
\usepackage{amssymb}
\usepackage[normalem]{ulem}
\useunder{\uline}{\ul}{}
\usepackage{wrapfig}


\usepackage[table,xcdraw]{xcolor}


%% ----------------------------------------------------------------
\begin{document}
\frontmatter
\title      {Heterogeneous Agent-based Model for Supermarket Competition}
\authors    {\texorpdfstring
             {\href{mailto:sc22g13@ecs.soton.ac.uk}{Stefan J. Collier}}
             {Stefan J. Collier}
            }
\addresses  {\groupname\\\deptname\\\univname}
\date       {\today}
\subject    {}
\keywords   {}
\supervisor {Dr. Maria Polukarov}
\examiner   {Professor Sheng Chen}

\maketitle
\begin{abstract}
This project aim was to model and analyse the effects of competitive pricing behaviors of grocery retailers on the British market. 

This was achieved by creating a multi-agent model, containing retailer and consumer agents. The heterogeneous crowd of retailers employs either a uniform pricing strategy or a ‘local price flexing’ strategy. The actions of these retailers are chosen by predicting the profit of each action, using a perceptron. Following on from the consideration of different economic models, a discrete model was developed so that software agents have a discrete environment to operate within. Within the model, it has been observed how supermarkets with differing behaviors affect a heterogeneous crowd of consumer agents. The model was implemented in Java with Python used to evaluate the results. 

The simulation displays good acceptance with real grocery market behavior, i.e. captures the performance of British retailers thus can be used to determine the impact of changes in their behavior on their competitors and consumers.Furthermore it can be used to provide insight into sustainability of volatile pricing strategies, providing a useful insight in volatility of British supermarket retail industry. 
\end{abstract}
\acknowledgements{
I would like to express my sincere gratitude to Dr Maria Polukarov for her guidance and support which provided me the freedom to take this research in the direction of my interest.\\
\\
I would also like to thank my family and friends for their encouragement and support. To those who quietly listened to my software complaints. To those who worked throughout the nights with me. To those who helped me write what I couldn't say. I cannot thank you enough.
}

\declaration{
I, Stefan Collier, declare that this dissertation and the work presented in it are my own and has been generated by me as the result of my own original research.\\
I confirm that:\\
1. This work was done wholly or mainly while in candidature for a degree at this University;\\
2. Where any part of this dissertation has previously been submitted for any other qualification at this University or any other institution, this has been clearly stated;\\
3. Where I have consulted the published work of others, this is always clearly attributed;\\
4. Where I have quoted from the work of others, the source is always given. With the exception of such quotations, this dissertation is entirely my own work;\\
5. I have acknowledged all main sources of help;\\
6. Where the thesis is based on work done by myself jointly with others, I have made clear exactly what was done by others and what I have contributed myself;\\
7. Either none of this work has been published before submission, or parts of this work have been published by :\\
\\
Stefan Collier\\
April 2016
}
\tableofcontents
\listoffigures
\listoftables

\mainmatter
%% ----------------------------------------------------------------
%\include{Introduction}
%\include{Conclusions}
\include{chapters/1Project/main}
\include{chapters/2Lit/main}
\include{chapters/3Design/HighLevel}
\include{chapters/3Design/InDepth}
\include{chapters/4Impl/main}

\include{chapters/5Experiments/1/main}
\include{chapters/5Experiments/2/main}
\include{chapters/5Experiments/3/main}
\include{chapters/5Experiments/4/main}

\include{chapters/6Conclusion/main}

\appendix
\include{appendix/AppendixB}
\include{appendix/D/main}
\include{appendix/AppendixC}

\backmatter
\bibliographystyle{ecs}
\bibliography{ECS}
\end{document}
%% ----------------------------------------------------------------

\include{chapters/3Design/HighLevel}
\include{chapters/3Design/InDepth}
 %% ----------------------------------------------------------------
%% Progress.tex
%% ---------------------------------------------------------------- 
\documentclass{ecsprogress}    % Use the progress Style
\graphicspath{{../figs/}}   % Location of your graphics files
    \usepackage{natbib}            % Use Natbib style for the refs.
\hypersetup{colorlinks=true}   % Set to false for black/white printing
\input{Definitions}            % Include your abbreviations



\usepackage{enumitem}% http://ctan.org/pkg/enumitem
\usepackage{multirow}
\usepackage{float}
\usepackage{amsmath}
\usepackage{multicol}
\usepackage{amssymb}
\usepackage[normalem]{ulem}
\useunder{\uline}{\ul}{}
\usepackage{wrapfig}


\usepackage[table,xcdraw]{xcolor}


%% ----------------------------------------------------------------
\begin{document}
\frontmatter
\title      {Heterogeneous Agent-based Model for Supermarket Competition}
\authors    {\texorpdfstring
             {\href{mailto:sc22g13@ecs.soton.ac.uk}{Stefan J. Collier}}
             {Stefan J. Collier}
            }
\addresses  {\groupname\\\deptname\\\univname}
\date       {\today}
\subject    {}
\keywords   {}
\supervisor {Dr. Maria Polukarov}
\examiner   {Professor Sheng Chen}

\maketitle
\begin{abstract}
This project aim was to model and analyse the effects of competitive pricing behaviors of grocery retailers on the British market. 

This was achieved by creating a multi-agent model, containing retailer and consumer agents. The heterogeneous crowd of retailers employs either a uniform pricing strategy or a ‘local price flexing’ strategy. The actions of these retailers are chosen by predicting the profit of each action, using a perceptron. Following on from the consideration of different economic models, a discrete model was developed so that software agents have a discrete environment to operate within. Within the model, it has been observed how supermarkets with differing behaviors affect a heterogeneous crowd of consumer agents. The model was implemented in Java with Python used to evaluate the results. 

The simulation displays good acceptance with real grocery market behavior, i.e. captures the performance of British retailers thus can be used to determine the impact of changes in their behavior on their competitors and consumers.Furthermore it can be used to provide insight into sustainability of volatile pricing strategies, providing a useful insight in volatility of British supermarket retail industry. 
\end{abstract}
\acknowledgements{
I would like to express my sincere gratitude to Dr Maria Polukarov for her guidance and support which provided me the freedom to take this research in the direction of my interest.\\
\\
I would also like to thank my family and friends for their encouragement and support. To those who quietly listened to my software complaints. To those who worked throughout the nights with me. To those who helped me write what I couldn't say. I cannot thank you enough.
}

\declaration{
I, Stefan Collier, declare that this dissertation and the work presented in it are my own and has been generated by me as the result of my own original research.\\
I confirm that:\\
1. This work was done wholly or mainly while in candidature for a degree at this University;\\
2. Where any part of this dissertation has previously been submitted for any other qualification at this University or any other institution, this has been clearly stated;\\
3. Where I have consulted the published work of others, this is always clearly attributed;\\
4. Where I have quoted from the work of others, the source is always given. With the exception of such quotations, this dissertation is entirely my own work;\\
5. I have acknowledged all main sources of help;\\
6. Where the thesis is based on work done by myself jointly with others, I have made clear exactly what was done by others and what I have contributed myself;\\
7. Either none of this work has been published before submission, or parts of this work have been published by :\\
\\
Stefan Collier\\
April 2016
}
\tableofcontents
\listoffigures
\listoftables

\mainmatter
%% ----------------------------------------------------------------
%\include{Introduction}
%\include{Conclusions}
\include{chapters/1Project/main}
\include{chapters/2Lit/main}
\include{chapters/3Design/HighLevel}
\include{chapters/3Design/InDepth}
\include{chapters/4Impl/main}

\include{chapters/5Experiments/1/main}
\include{chapters/5Experiments/2/main}
\include{chapters/5Experiments/3/main}
\include{chapters/5Experiments/4/main}

\include{chapters/6Conclusion/main}

\appendix
\include{appendix/AppendixB}
\include{appendix/D/main}
\include{appendix/AppendixC}

\backmatter
\bibliographystyle{ecs}
\bibliography{ECS}
\end{document}
%% ----------------------------------------------------------------


 %% ----------------------------------------------------------------
%% Progress.tex
%% ---------------------------------------------------------------- 
\documentclass{ecsprogress}    % Use the progress Style
\graphicspath{{../figs/}}   % Location of your graphics files
    \usepackage{natbib}            % Use Natbib style for the refs.
\hypersetup{colorlinks=true}   % Set to false for black/white printing
\input{Definitions}            % Include your abbreviations



\usepackage{enumitem}% http://ctan.org/pkg/enumitem
\usepackage{multirow}
\usepackage{float}
\usepackage{amsmath}
\usepackage{multicol}
\usepackage{amssymb}
\usepackage[normalem]{ulem}
\useunder{\uline}{\ul}{}
\usepackage{wrapfig}


\usepackage[table,xcdraw]{xcolor}


%% ----------------------------------------------------------------
\begin{document}
\frontmatter
\title      {Heterogeneous Agent-based Model for Supermarket Competition}
\authors    {\texorpdfstring
             {\href{mailto:sc22g13@ecs.soton.ac.uk}{Stefan J. Collier}}
             {Stefan J. Collier}
            }
\addresses  {\groupname\\\deptname\\\univname}
\date       {\today}
\subject    {}
\keywords   {}
\supervisor {Dr. Maria Polukarov}
\examiner   {Professor Sheng Chen}

\maketitle
\begin{abstract}
This project aim was to model and analyse the effects of competitive pricing behaviors of grocery retailers on the British market. 

This was achieved by creating a multi-agent model, containing retailer and consumer agents. The heterogeneous crowd of retailers employs either a uniform pricing strategy or a ‘local price flexing’ strategy. The actions of these retailers are chosen by predicting the profit of each action, using a perceptron. Following on from the consideration of different economic models, a discrete model was developed so that software agents have a discrete environment to operate within. Within the model, it has been observed how supermarkets with differing behaviors affect a heterogeneous crowd of consumer agents. The model was implemented in Java with Python used to evaluate the results. 

The simulation displays good acceptance with real grocery market behavior, i.e. captures the performance of British retailers thus can be used to determine the impact of changes in their behavior on their competitors and consumers.Furthermore it can be used to provide insight into sustainability of volatile pricing strategies, providing a useful insight in volatility of British supermarket retail industry. 
\end{abstract}
\acknowledgements{
I would like to express my sincere gratitude to Dr Maria Polukarov for her guidance and support which provided me the freedom to take this research in the direction of my interest.\\
\\
I would also like to thank my family and friends for their encouragement and support. To those who quietly listened to my software complaints. To those who worked throughout the nights with me. To those who helped me write what I couldn't say. I cannot thank you enough.
}

\declaration{
I, Stefan Collier, declare that this dissertation and the work presented in it are my own and has been generated by me as the result of my own original research.\\
I confirm that:\\
1. This work was done wholly or mainly while in candidature for a degree at this University;\\
2. Where any part of this dissertation has previously been submitted for any other qualification at this University or any other institution, this has been clearly stated;\\
3. Where I have consulted the published work of others, this is always clearly attributed;\\
4. Where I have quoted from the work of others, the source is always given. With the exception of such quotations, this dissertation is entirely my own work;\\
5. I have acknowledged all main sources of help;\\
6. Where the thesis is based on work done by myself jointly with others, I have made clear exactly what was done by others and what I have contributed myself;\\
7. Either none of this work has been published before submission, or parts of this work have been published by :\\
\\
Stefan Collier\\
April 2016
}
\tableofcontents
\listoffigures
\listoftables

\mainmatter
%% ----------------------------------------------------------------
%\include{Introduction}
%\include{Conclusions}
\include{chapters/1Project/main}
\include{chapters/2Lit/main}
\include{chapters/3Design/HighLevel}
\include{chapters/3Design/InDepth}
\include{chapters/4Impl/main}

\include{chapters/5Experiments/1/main}
\include{chapters/5Experiments/2/main}
\include{chapters/5Experiments/3/main}
\include{chapters/5Experiments/4/main}

\include{chapters/6Conclusion/main}

\appendix
\include{appendix/AppendixB}
\include{appendix/D/main}
\include{appendix/AppendixC}

\backmatter
\bibliographystyle{ecs}
\bibliography{ECS}
\end{document}
%% ----------------------------------------------------------------

 %% ----------------------------------------------------------------
%% Progress.tex
%% ---------------------------------------------------------------- 
\documentclass{ecsprogress}    % Use the progress Style
\graphicspath{{../figs/}}   % Location of your graphics files
    \usepackage{natbib}            % Use Natbib style for the refs.
\hypersetup{colorlinks=true}   % Set to false for black/white printing
\input{Definitions}            % Include your abbreviations



\usepackage{enumitem}% http://ctan.org/pkg/enumitem
\usepackage{multirow}
\usepackage{float}
\usepackage{amsmath}
\usepackage{multicol}
\usepackage{amssymb}
\usepackage[normalem]{ulem}
\useunder{\uline}{\ul}{}
\usepackage{wrapfig}


\usepackage[table,xcdraw]{xcolor}


%% ----------------------------------------------------------------
\begin{document}
\frontmatter
\title      {Heterogeneous Agent-based Model for Supermarket Competition}
\authors    {\texorpdfstring
             {\href{mailto:sc22g13@ecs.soton.ac.uk}{Stefan J. Collier}}
             {Stefan J. Collier}
            }
\addresses  {\groupname\\\deptname\\\univname}
\date       {\today}
\subject    {}
\keywords   {}
\supervisor {Dr. Maria Polukarov}
\examiner   {Professor Sheng Chen}

\maketitle
\begin{abstract}
This project aim was to model and analyse the effects of competitive pricing behaviors of grocery retailers on the British market. 

This was achieved by creating a multi-agent model, containing retailer and consumer agents. The heterogeneous crowd of retailers employs either a uniform pricing strategy or a ‘local price flexing’ strategy. The actions of these retailers are chosen by predicting the profit of each action, using a perceptron. Following on from the consideration of different economic models, a discrete model was developed so that software agents have a discrete environment to operate within. Within the model, it has been observed how supermarkets with differing behaviors affect a heterogeneous crowd of consumer agents. The model was implemented in Java with Python used to evaluate the results. 

The simulation displays good acceptance with real grocery market behavior, i.e. captures the performance of British retailers thus can be used to determine the impact of changes in their behavior on their competitors and consumers.Furthermore it can be used to provide insight into sustainability of volatile pricing strategies, providing a useful insight in volatility of British supermarket retail industry. 
\end{abstract}
\acknowledgements{
I would like to express my sincere gratitude to Dr Maria Polukarov for her guidance and support which provided me the freedom to take this research in the direction of my interest.\\
\\
I would also like to thank my family and friends for their encouragement and support. To those who quietly listened to my software complaints. To those who worked throughout the nights with me. To those who helped me write what I couldn't say. I cannot thank you enough.
}

\declaration{
I, Stefan Collier, declare that this dissertation and the work presented in it are my own and has been generated by me as the result of my own original research.\\
I confirm that:\\
1. This work was done wholly or mainly while in candidature for a degree at this University;\\
2. Where any part of this dissertation has previously been submitted for any other qualification at this University or any other institution, this has been clearly stated;\\
3. Where I have consulted the published work of others, this is always clearly attributed;\\
4. Where I have quoted from the work of others, the source is always given. With the exception of such quotations, this dissertation is entirely my own work;\\
5. I have acknowledged all main sources of help;\\
6. Where the thesis is based on work done by myself jointly with others, I have made clear exactly what was done by others and what I have contributed myself;\\
7. Either none of this work has been published before submission, or parts of this work have been published by :\\
\\
Stefan Collier\\
April 2016
}
\tableofcontents
\listoffigures
\listoftables

\mainmatter
%% ----------------------------------------------------------------
%\include{Introduction}
%\include{Conclusions}
\include{chapters/1Project/main}
\include{chapters/2Lit/main}
\include{chapters/3Design/HighLevel}
\include{chapters/3Design/InDepth}
\include{chapters/4Impl/main}

\include{chapters/5Experiments/1/main}
\include{chapters/5Experiments/2/main}
\include{chapters/5Experiments/3/main}
\include{chapters/5Experiments/4/main}

\include{chapters/6Conclusion/main}

\appendix
\include{appendix/AppendixB}
\include{appendix/D/main}
\include{appendix/AppendixC}

\backmatter
\bibliographystyle{ecs}
\bibliography{ECS}
\end{document}
%% ----------------------------------------------------------------

 %% ----------------------------------------------------------------
%% Progress.tex
%% ---------------------------------------------------------------- 
\documentclass{ecsprogress}    % Use the progress Style
\graphicspath{{../figs/}}   % Location of your graphics files
    \usepackage{natbib}            % Use Natbib style for the refs.
\hypersetup{colorlinks=true}   % Set to false for black/white printing
\input{Definitions}            % Include your abbreviations



\usepackage{enumitem}% http://ctan.org/pkg/enumitem
\usepackage{multirow}
\usepackage{float}
\usepackage{amsmath}
\usepackage{multicol}
\usepackage{amssymb}
\usepackage[normalem]{ulem}
\useunder{\uline}{\ul}{}
\usepackage{wrapfig}


\usepackage[table,xcdraw]{xcolor}


%% ----------------------------------------------------------------
\begin{document}
\frontmatter
\title      {Heterogeneous Agent-based Model for Supermarket Competition}
\authors    {\texorpdfstring
             {\href{mailto:sc22g13@ecs.soton.ac.uk}{Stefan J. Collier}}
             {Stefan J. Collier}
            }
\addresses  {\groupname\\\deptname\\\univname}
\date       {\today}
\subject    {}
\keywords   {}
\supervisor {Dr. Maria Polukarov}
\examiner   {Professor Sheng Chen}

\maketitle
\begin{abstract}
This project aim was to model and analyse the effects of competitive pricing behaviors of grocery retailers on the British market. 

This was achieved by creating a multi-agent model, containing retailer and consumer agents. The heterogeneous crowd of retailers employs either a uniform pricing strategy or a ‘local price flexing’ strategy. The actions of these retailers are chosen by predicting the profit of each action, using a perceptron. Following on from the consideration of different economic models, a discrete model was developed so that software agents have a discrete environment to operate within. Within the model, it has been observed how supermarkets with differing behaviors affect a heterogeneous crowd of consumer agents. The model was implemented in Java with Python used to evaluate the results. 

The simulation displays good acceptance with real grocery market behavior, i.e. captures the performance of British retailers thus can be used to determine the impact of changes in their behavior on their competitors and consumers.Furthermore it can be used to provide insight into sustainability of volatile pricing strategies, providing a useful insight in volatility of British supermarket retail industry. 
\end{abstract}
\acknowledgements{
I would like to express my sincere gratitude to Dr Maria Polukarov for her guidance and support which provided me the freedom to take this research in the direction of my interest.\\
\\
I would also like to thank my family and friends for their encouragement and support. To those who quietly listened to my software complaints. To those who worked throughout the nights with me. To those who helped me write what I couldn't say. I cannot thank you enough.
}

\declaration{
I, Stefan Collier, declare that this dissertation and the work presented in it are my own and has been generated by me as the result of my own original research.\\
I confirm that:\\
1. This work was done wholly or mainly while in candidature for a degree at this University;\\
2. Where any part of this dissertation has previously been submitted for any other qualification at this University or any other institution, this has been clearly stated;\\
3. Where I have consulted the published work of others, this is always clearly attributed;\\
4. Where I have quoted from the work of others, the source is always given. With the exception of such quotations, this dissertation is entirely my own work;\\
5. I have acknowledged all main sources of help;\\
6. Where the thesis is based on work done by myself jointly with others, I have made clear exactly what was done by others and what I have contributed myself;\\
7. Either none of this work has been published before submission, or parts of this work have been published by :\\
\\
Stefan Collier\\
April 2016
}
\tableofcontents
\listoffigures
\listoftables

\mainmatter
%% ----------------------------------------------------------------
%\include{Introduction}
%\include{Conclusions}
\include{chapters/1Project/main}
\include{chapters/2Lit/main}
\include{chapters/3Design/HighLevel}
\include{chapters/3Design/InDepth}
\include{chapters/4Impl/main}

\include{chapters/5Experiments/1/main}
\include{chapters/5Experiments/2/main}
\include{chapters/5Experiments/3/main}
\include{chapters/5Experiments/4/main}

\include{chapters/6Conclusion/main}

\appendix
\include{appendix/AppendixB}
\include{appendix/D/main}
\include{appendix/AppendixC}

\backmatter
\bibliographystyle{ecs}
\bibliography{ECS}
\end{document}
%% ----------------------------------------------------------------

 %% ----------------------------------------------------------------
%% Progress.tex
%% ---------------------------------------------------------------- 
\documentclass{ecsprogress}    % Use the progress Style
\graphicspath{{../figs/}}   % Location of your graphics files
    \usepackage{natbib}            % Use Natbib style for the refs.
\hypersetup{colorlinks=true}   % Set to false for black/white printing
\input{Definitions}            % Include your abbreviations



\usepackage{enumitem}% http://ctan.org/pkg/enumitem
\usepackage{multirow}
\usepackage{float}
\usepackage{amsmath}
\usepackage{multicol}
\usepackage{amssymb}
\usepackage[normalem]{ulem}
\useunder{\uline}{\ul}{}
\usepackage{wrapfig}


\usepackage[table,xcdraw]{xcolor}


%% ----------------------------------------------------------------
\begin{document}
\frontmatter
\title      {Heterogeneous Agent-based Model for Supermarket Competition}
\authors    {\texorpdfstring
             {\href{mailto:sc22g13@ecs.soton.ac.uk}{Stefan J. Collier}}
             {Stefan J. Collier}
            }
\addresses  {\groupname\\\deptname\\\univname}
\date       {\today}
\subject    {}
\keywords   {}
\supervisor {Dr. Maria Polukarov}
\examiner   {Professor Sheng Chen}

\maketitle
\begin{abstract}
This project aim was to model and analyse the effects of competitive pricing behaviors of grocery retailers on the British market. 

This was achieved by creating a multi-agent model, containing retailer and consumer agents. The heterogeneous crowd of retailers employs either a uniform pricing strategy or a ‘local price flexing’ strategy. The actions of these retailers are chosen by predicting the profit of each action, using a perceptron. Following on from the consideration of different economic models, a discrete model was developed so that software agents have a discrete environment to operate within. Within the model, it has been observed how supermarkets with differing behaviors affect a heterogeneous crowd of consumer agents. The model was implemented in Java with Python used to evaluate the results. 

The simulation displays good acceptance with real grocery market behavior, i.e. captures the performance of British retailers thus can be used to determine the impact of changes in their behavior on their competitors and consumers.Furthermore it can be used to provide insight into sustainability of volatile pricing strategies, providing a useful insight in volatility of British supermarket retail industry. 
\end{abstract}
\acknowledgements{
I would like to express my sincere gratitude to Dr Maria Polukarov for her guidance and support which provided me the freedom to take this research in the direction of my interest.\\
\\
I would also like to thank my family and friends for their encouragement and support. To those who quietly listened to my software complaints. To those who worked throughout the nights with me. To those who helped me write what I couldn't say. I cannot thank you enough.
}

\declaration{
I, Stefan Collier, declare that this dissertation and the work presented in it are my own and has been generated by me as the result of my own original research.\\
I confirm that:\\
1. This work was done wholly or mainly while in candidature for a degree at this University;\\
2. Where any part of this dissertation has previously been submitted for any other qualification at this University or any other institution, this has been clearly stated;\\
3. Where I have consulted the published work of others, this is always clearly attributed;\\
4. Where I have quoted from the work of others, the source is always given. With the exception of such quotations, this dissertation is entirely my own work;\\
5. I have acknowledged all main sources of help;\\
6. Where the thesis is based on work done by myself jointly with others, I have made clear exactly what was done by others and what I have contributed myself;\\
7. Either none of this work has been published before submission, or parts of this work have been published by :\\
\\
Stefan Collier\\
April 2016
}
\tableofcontents
\listoffigures
\listoftables

\mainmatter
%% ----------------------------------------------------------------
%\include{Introduction}
%\include{Conclusions}
\include{chapters/1Project/main}
\include{chapters/2Lit/main}
\include{chapters/3Design/HighLevel}
\include{chapters/3Design/InDepth}
\include{chapters/4Impl/main}

\include{chapters/5Experiments/1/main}
\include{chapters/5Experiments/2/main}
\include{chapters/5Experiments/3/main}
\include{chapters/5Experiments/4/main}

\include{chapters/6Conclusion/main}

\appendix
\include{appendix/AppendixB}
\include{appendix/D/main}
\include{appendix/AppendixC}

\backmatter
\bibliographystyle{ecs}
\bibliography{ECS}
\end{document}
%% ----------------------------------------------------------------


 %% ----------------------------------------------------------------
%% Progress.tex
%% ---------------------------------------------------------------- 
\documentclass{ecsprogress}    % Use the progress Style
\graphicspath{{../figs/}}   % Location of your graphics files
    \usepackage{natbib}            % Use Natbib style for the refs.
\hypersetup{colorlinks=true}   % Set to false for black/white printing
\input{Definitions}            % Include your abbreviations



\usepackage{enumitem}% http://ctan.org/pkg/enumitem
\usepackage{multirow}
\usepackage{float}
\usepackage{amsmath}
\usepackage{multicol}
\usepackage{amssymb}
\usepackage[normalem]{ulem}
\useunder{\uline}{\ul}{}
\usepackage{wrapfig}


\usepackage[table,xcdraw]{xcolor}


%% ----------------------------------------------------------------
\begin{document}
\frontmatter
\title      {Heterogeneous Agent-based Model for Supermarket Competition}
\authors    {\texorpdfstring
             {\href{mailto:sc22g13@ecs.soton.ac.uk}{Stefan J. Collier}}
             {Stefan J. Collier}
            }
\addresses  {\groupname\\\deptname\\\univname}
\date       {\today}
\subject    {}
\keywords   {}
\supervisor {Dr. Maria Polukarov}
\examiner   {Professor Sheng Chen}

\maketitle
\begin{abstract}
This project aim was to model and analyse the effects of competitive pricing behaviors of grocery retailers on the British market. 

This was achieved by creating a multi-agent model, containing retailer and consumer agents. The heterogeneous crowd of retailers employs either a uniform pricing strategy or a ‘local price flexing’ strategy. The actions of these retailers are chosen by predicting the profit of each action, using a perceptron. Following on from the consideration of different economic models, a discrete model was developed so that software agents have a discrete environment to operate within. Within the model, it has been observed how supermarkets with differing behaviors affect a heterogeneous crowd of consumer agents. The model was implemented in Java with Python used to evaluate the results. 

The simulation displays good acceptance with real grocery market behavior, i.e. captures the performance of British retailers thus can be used to determine the impact of changes in their behavior on their competitors and consumers.Furthermore it can be used to provide insight into sustainability of volatile pricing strategies, providing a useful insight in volatility of British supermarket retail industry. 
\end{abstract}
\acknowledgements{
I would like to express my sincere gratitude to Dr Maria Polukarov for her guidance and support which provided me the freedom to take this research in the direction of my interest.\\
\\
I would also like to thank my family and friends for their encouragement and support. To those who quietly listened to my software complaints. To those who worked throughout the nights with me. To those who helped me write what I couldn't say. I cannot thank you enough.
}

\declaration{
I, Stefan Collier, declare that this dissertation and the work presented in it are my own and has been generated by me as the result of my own original research.\\
I confirm that:\\
1. This work was done wholly or mainly while in candidature for a degree at this University;\\
2. Where any part of this dissertation has previously been submitted for any other qualification at this University or any other institution, this has been clearly stated;\\
3. Where I have consulted the published work of others, this is always clearly attributed;\\
4. Where I have quoted from the work of others, the source is always given. With the exception of such quotations, this dissertation is entirely my own work;\\
5. I have acknowledged all main sources of help;\\
6. Where the thesis is based on work done by myself jointly with others, I have made clear exactly what was done by others and what I have contributed myself;\\
7. Either none of this work has been published before submission, or parts of this work have been published by :\\
\\
Stefan Collier\\
April 2016
}
\tableofcontents
\listoffigures
\listoftables

\mainmatter
%% ----------------------------------------------------------------
%\include{Introduction}
%\include{Conclusions}
\include{chapters/1Project/main}
\include{chapters/2Lit/main}
\include{chapters/3Design/HighLevel}
\include{chapters/3Design/InDepth}
\include{chapters/4Impl/main}

\include{chapters/5Experiments/1/main}
\include{chapters/5Experiments/2/main}
\include{chapters/5Experiments/3/main}
\include{chapters/5Experiments/4/main}

\include{chapters/6Conclusion/main}

\appendix
\include{appendix/AppendixB}
\include{appendix/D/main}
\include{appendix/AppendixC}

\backmatter
\bibliographystyle{ecs}
\bibliography{ECS}
\end{document}
%% ----------------------------------------------------------------


\appendix
\include{appendix/AppendixB}
 %% ----------------------------------------------------------------
%% Progress.tex
%% ---------------------------------------------------------------- 
\documentclass{ecsprogress}    % Use the progress Style
\graphicspath{{../figs/}}   % Location of your graphics files
    \usepackage{natbib}            % Use Natbib style for the refs.
\hypersetup{colorlinks=true}   % Set to false for black/white printing
\input{Definitions}            % Include your abbreviations



\usepackage{enumitem}% http://ctan.org/pkg/enumitem
\usepackage{multirow}
\usepackage{float}
\usepackage{amsmath}
\usepackage{multicol}
\usepackage{amssymb}
\usepackage[normalem]{ulem}
\useunder{\uline}{\ul}{}
\usepackage{wrapfig}


\usepackage[table,xcdraw]{xcolor}


%% ----------------------------------------------------------------
\begin{document}
\frontmatter
\title      {Heterogeneous Agent-based Model for Supermarket Competition}
\authors    {\texorpdfstring
             {\href{mailto:sc22g13@ecs.soton.ac.uk}{Stefan J. Collier}}
             {Stefan J. Collier}
            }
\addresses  {\groupname\\\deptname\\\univname}
\date       {\today}
\subject    {}
\keywords   {}
\supervisor {Dr. Maria Polukarov}
\examiner   {Professor Sheng Chen}

\maketitle
\begin{abstract}
This project aim was to model and analyse the effects of competitive pricing behaviors of grocery retailers on the British market. 

This was achieved by creating a multi-agent model, containing retailer and consumer agents. The heterogeneous crowd of retailers employs either a uniform pricing strategy or a ‘local price flexing’ strategy. The actions of these retailers are chosen by predicting the profit of each action, using a perceptron. Following on from the consideration of different economic models, a discrete model was developed so that software agents have a discrete environment to operate within. Within the model, it has been observed how supermarkets with differing behaviors affect a heterogeneous crowd of consumer agents. The model was implemented in Java with Python used to evaluate the results. 

The simulation displays good acceptance with real grocery market behavior, i.e. captures the performance of British retailers thus can be used to determine the impact of changes in their behavior on their competitors and consumers.Furthermore it can be used to provide insight into sustainability of volatile pricing strategies, providing a useful insight in volatility of British supermarket retail industry. 
\end{abstract}
\acknowledgements{
I would like to express my sincere gratitude to Dr Maria Polukarov for her guidance and support which provided me the freedom to take this research in the direction of my interest.\\
\\
I would also like to thank my family and friends for their encouragement and support. To those who quietly listened to my software complaints. To those who worked throughout the nights with me. To those who helped me write what I couldn't say. I cannot thank you enough.
}

\declaration{
I, Stefan Collier, declare that this dissertation and the work presented in it are my own and has been generated by me as the result of my own original research.\\
I confirm that:\\
1. This work was done wholly or mainly while in candidature for a degree at this University;\\
2. Where any part of this dissertation has previously been submitted for any other qualification at this University or any other institution, this has been clearly stated;\\
3. Where I have consulted the published work of others, this is always clearly attributed;\\
4. Where I have quoted from the work of others, the source is always given. With the exception of such quotations, this dissertation is entirely my own work;\\
5. I have acknowledged all main sources of help;\\
6. Where the thesis is based on work done by myself jointly with others, I have made clear exactly what was done by others and what I have contributed myself;\\
7. Either none of this work has been published before submission, or parts of this work have been published by :\\
\\
Stefan Collier\\
April 2016
}
\tableofcontents
\listoffigures
\listoftables

\mainmatter
%% ----------------------------------------------------------------
%\include{Introduction}
%\include{Conclusions}
\include{chapters/1Project/main}
\include{chapters/2Lit/main}
\include{chapters/3Design/HighLevel}
\include{chapters/3Design/InDepth}
\include{chapters/4Impl/main}

\include{chapters/5Experiments/1/main}
\include{chapters/5Experiments/2/main}
\include{chapters/5Experiments/3/main}
\include{chapters/5Experiments/4/main}

\include{chapters/6Conclusion/main}

\appendix
\include{appendix/AppendixB}
\include{appendix/D/main}
\include{appendix/AppendixC}

\backmatter
\bibliographystyle{ecs}
\bibliography{ECS}
\end{document}
%% ----------------------------------------------------------------

\include{appendix/AppendixC}

\backmatter
\bibliographystyle{ecs}
\bibliography{ECS}
\end{document}
%% ----------------------------------------------------------------

 %% ----------------------------------------------------------------
%% Progress.tex
%% ---------------------------------------------------------------- 
\documentclass{ecsprogress}    % Use the progress Style
\graphicspath{{../figs/}}   % Location of your graphics files
    \usepackage{natbib}            % Use Natbib style for the refs.
\hypersetup{colorlinks=true}   % Set to false for black/white printing
\input{Definitions}            % Include your abbreviations



\usepackage{enumitem}% http://ctan.org/pkg/enumitem
\usepackage{multirow}
\usepackage{float}
\usepackage{amsmath}
\usepackage{multicol}
\usepackage{amssymb}
\usepackage[normalem]{ulem}
\useunder{\uline}{\ul}{}
\usepackage{wrapfig}


\usepackage[table,xcdraw]{xcolor}


%% ----------------------------------------------------------------
\begin{document}
\frontmatter
\title      {Heterogeneous Agent-based Model for Supermarket Competition}
\authors    {\texorpdfstring
             {\href{mailto:sc22g13@ecs.soton.ac.uk}{Stefan J. Collier}}
             {Stefan J. Collier}
            }
\addresses  {\groupname\\\deptname\\\univname}
\date       {\today}
\subject    {}
\keywords   {}
\supervisor {Dr. Maria Polukarov}
\examiner   {Professor Sheng Chen}

\maketitle
\begin{abstract}
This project aim was to model and analyse the effects of competitive pricing behaviors of grocery retailers on the British market. 

This was achieved by creating a multi-agent model, containing retailer and consumer agents. The heterogeneous crowd of retailers employs either a uniform pricing strategy or a ‘local price flexing’ strategy. The actions of these retailers are chosen by predicting the profit of each action, using a perceptron. Following on from the consideration of different economic models, a discrete model was developed so that software agents have a discrete environment to operate within. Within the model, it has been observed how supermarkets with differing behaviors affect a heterogeneous crowd of consumer agents. The model was implemented in Java with Python used to evaluate the results. 

The simulation displays good acceptance with real grocery market behavior, i.e. captures the performance of British retailers thus can be used to determine the impact of changes in their behavior on their competitors and consumers.Furthermore it can be used to provide insight into sustainability of volatile pricing strategies, providing a useful insight in volatility of British supermarket retail industry. 
\end{abstract}
\acknowledgements{
I would like to express my sincere gratitude to Dr Maria Polukarov for her guidance and support which provided me the freedom to take this research in the direction of my interest.\\
\\
I would also like to thank my family and friends for their encouragement and support. To those who quietly listened to my software complaints. To those who worked throughout the nights with me. To those who helped me write what I couldn't say. I cannot thank you enough.
}

\declaration{
I, Stefan Collier, declare that this dissertation and the work presented in it are my own and has been generated by me as the result of my own original research.\\
I confirm that:\\
1. This work was done wholly or mainly while in candidature for a degree at this University;\\
2. Where any part of this dissertation has previously been submitted for any other qualification at this University or any other institution, this has been clearly stated;\\
3. Where I have consulted the published work of others, this is always clearly attributed;\\
4. Where I have quoted from the work of others, the source is always given. With the exception of such quotations, this dissertation is entirely my own work;\\
5. I have acknowledged all main sources of help;\\
6. Where the thesis is based on work done by myself jointly with others, I have made clear exactly what was done by others and what I have contributed myself;\\
7. Either none of this work has been published before submission, or parts of this work have been published by :\\
\\
Stefan Collier\\
April 2016
}
\tableofcontents
\listoffigures
\listoftables

\mainmatter
%% ----------------------------------------------------------------
%\include{Introduction}
%\include{Conclusions}
 %% ----------------------------------------------------------------
%% Progress.tex
%% ---------------------------------------------------------------- 
\documentclass{ecsprogress}    % Use the progress Style
\graphicspath{{../figs/}}   % Location of your graphics files
    \usepackage{natbib}            % Use Natbib style for the refs.
\hypersetup{colorlinks=true}   % Set to false for black/white printing
\input{Definitions}            % Include your abbreviations



\usepackage{enumitem}% http://ctan.org/pkg/enumitem
\usepackage{multirow}
\usepackage{float}
\usepackage{amsmath}
\usepackage{multicol}
\usepackage{amssymb}
\usepackage[normalem]{ulem}
\useunder{\uline}{\ul}{}
\usepackage{wrapfig}


\usepackage[table,xcdraw]{xcolor}


%% ----------------------------------------------------------------
\begin{document}
\frontmatter
\title      {Heterogeneous Agent-based Model for Supermarket Competition}
\authors    {\texorpdfstring
             {\href{mailto:sc22g13@ecs.soton.ac.uk}{Stefan J. Collier}}
             {Stefan J. Collier}
            }
\addresses  {\groupname\\\deptname\\\univname}
\date       {\today}
\subject    {}
\keywords   {}
\supervisor {Dr. Maria Polukarov}
\examiner   {Professor Sheng Chen}

\maketitle
\begin{abstract}
This project aim was to model and analyse the effects of competitive pricing behaviors of grocery retailers on the British market. 

This was achieved by creating a multi-agent model, containing retailer and consumer agents. The heterogeneous crowd of retailers employs either a uniform pricing strategy or a ‘local price flexing’ strategy. The actions of these retailers are chosen by predicting the profit of each action, using a perceptron. Following on from the consideration of different economic models, a discrete model was developed so that software agents have a discrete environment to operate within. Within the model, it has been observed how supermarkets with differing behaviors affect a heterogeneous crowd of consumer agents. The model was implemented in Java with Python used to evaluate the results. 

The simulation displays good acceptance with real grocery market behavior, i.e. captures the performance of British retailers thus can be used to determine the impact of changes in their behavior on their competitors and consumers.Furthermore it can be used to provide insight into sustainability of volatile pricing strategies, providing a useful insight in volatility of British supermarket retail industry. 
\end{abstract}
\acknowledgements{
I would like to express my sincere gratitude to Dr Maria Polukarov for her guidance and support which provided me the freedom to take this research in the direction of my interest.\\
\\
I would also like to thank my family and friends for their encouragement and support. To those who quietly listened to my software complaints. To those who worked throughout the nights with me. To those who helped me write what I couldn't say. I cannot thank you enough.
}

\declaration{
I, Stefan Collier, declare that this dissertation and the work presented in it are my own and has been generated by me as the result of my own original research.\\
I confirm that:\\
1. This work was done wholly or mainly while in candidature for a degree at this University;\\
2. Where any part of this dissertation has previously been submitted for any other qualification at this University or any other institution, this has been clearly stated;\\
3. Where I have consulted the published work of others, this is always clearly attributed;\\
4. Where I have quoted from the work of others, the source is always given. With the exception of such quotations, this dissertation is entirely my own work;\\
5. I have acknowledged all main sources of help;\\
6. Where the thesis is based on work done by myself jointly with others, I have made clear exactly what was done by others and what I have contributed myself;\\
7. Either none of this work has been published before submission, or parts of this work have been published by :\\
\\
Stefan Collier\\
April 2016
}
\tableofcontents
\listoffigures
\listoftables

\mainmatter
%% ----------------------------------------------------------------
%\include{Introduction}
%\include{Conclusions}
\include{chapters/1Project/main}
\include{chapters/2Lit/main}
\include{chapters/3Design/HighLevel}
\include{chapters/3Design/InDepth}
\include{chapters/4Impl/main}

\include{chapters/5Experiments/1/main}
\include{chapters/5Experiments/2/main}
\include{chapters/5Experiments/3/main}
\include{chapters/5Experiments/4/main}

\include{chapters/6Conclusion/main}

\appendix
\include{appendix/AppendixB}
\include{appendix/D/main}
\include{appendix/AppendixC}

\backmatter
\bibliographystyle{ecs}
\bibliography{ECS}
\end{document}
%% ----------------------------------------------------------------

 %% ----------------------------------------------------------------
%% Progress.tex
%% ---------------------------------------------------------------- 
\documentclass{ecsprogress}    % Use the progress Style
\graphicspath{{../figs/}}   % Location of your graphics files
    \usepackage{natbib}            % Use Natbib style for the refs.
\hypersetup{colorlinks=true}   % Set to false for black/white printing
\input{Definitions}            % Include your abbreviations



\usepackage{enumitem}% http://ctan.org/pkg/enumitem
\usepackage{multirow}
\usepackage{float}
\usepackage{amsmath}
\usepackage{multicol}
\usepackage{amssymb}
\usepackage[normalem]{ulem}
\useunder{\uline}{\ul}{}
\usepackage{wrapfig}


\usepackage[table,xcdraw]{xcolor}


%% ----------------------------------------------------------------
\begin{document}
\frontmatter
\title      {Heterogeneous Agent-based Model for Supermarket Competition}
\authors    {\texorpdfstring
             {\href{mailto:sc22g13@ecs.soton.ac.uk}{Stefan J. Collier}}
             {Stefan J. Collier}
            }
\addresses  {\groupname\\\deptname\\\univname}
\date       {\today}
\subject    {}
\keywords   {}
\supervisor {Dr. Maria Polukarov}
\examiner   {Professor Sheng Chen}

\maketitle
\begin{abstract}
This project aim was to model and analyse the effects of competitive pricing behaviors of grocery retailers on the British market. 

This was achieved by creating a multi-agent model, containing retailer and consumer agents. The heterogeneous crowd of retailers employs either a uniform pricing strategy or a ‘local price flexing’ strategy. The actions of these retailers are chosen by predicting the profit of each action, using a perceptron. Following on from the consideration of different economic models, a discrete model was developed so that software agents have a discrete environment to operate within. Within the model, it has been observed how supermarkets with differing behaviors affect a heterogeneous crowd of consumer agents. The model was implemented in Java with Python used to evaluate the results. 

The simulation displays good acceptance with real grocery market behavior, i.e. captures the performance of British retailers thus can be used to determine the impact of changes in their behavior on their competitors and consumers.Furthermore it can be used to provide insight into sustainability of volatile pricing strategies, providing a useful insight in volatility of British supermarket retail industry. 
\end{abstract}
\acknowledgements{
I would like to express my sincere gratitude to Dr Maria Polukarov for her guidance and support which provided me the freedom to take this research in the direction of my interest.\\
\\
I would also like to thank my family and friends for their encouragement and support. To those who quietly listened to my software complaints. To those who worked throughout the nights with me. To those who helped me write what I couldn't say. I cannot thank you enough.
}

\declaration{
I, Stefan Collier, declare that this dissertation and the work presented in it are my own and has been generated by me as the result of my own original research.\\
I confirm that:\\
1. This work was done wholly or mainly while in candidature for a degree at this University;\\
2. Where any part of this dissertation has previously been submitted for any other qualification at this University or any other institution, this has been clearly stated;\\
3. Where I have consulted the published work of others, this is always clearly attributed;\\
4. Where I have quoted from the work of others, the source is always given. With the exception of such quotations, this dissertation is entirely my own work;\\
5. I have acknowledged all main sources of help;\\
6. Where the thesis is based on work done by myself jointly with others, I have made clear exactly what was done by others and what I have contributed myself;\\
7. Either none of this work has been published before submission, or parts of this work have been published by :\\
\\
Stefan Collier\\
April 2016
}
\tableofcontents
\listoffigures
\listoftables

\mainmatter
%% ----------------------------------------------------------------
%\include{Introduction}
%\include{Conclusions}
\include{chapters/1Project/main}
\include{chapters/2Lit/main}
\include{chapters/3Design/HighLevel}
\include{chapters/3Design/InDepth}
\include{chapters/4Impl/main}

\include{chapters/5Experiments/1/main}
\include{chapters/5Experiments/2/main}
\include{chapters/5Experiments/3/main}
\include{chapters/5Experiments/4/main}

\include{chapters/6Conclusion/main}

\appendix
\include{appendix/AppendixB}
\include{appendix/D/main}
\include{appendix/AppendixC}

\backmatter
\bibliographystyle{ecs}
\bibliography{ECS}
\end{document}
%% ----------------------------------------------------------------

\include{chapters/3Design/HighLevel}
\include{chapters/3Design/InDepth}
 %% ----------------------------------------------------------------
%% Progress.tex
%% ---------------------------------------------------------------- 
\documentclass{ecsprogress}    % Use the progress Style
\graphicspath{{../figs/}}   % Location of your graphics files
    \usepackage{natbib}            % Use Natbib style for the refs.
\hypersetup{colorlinks=true}   % Set to false for black/white printing
\input{Definitions}            % Include your abbreviations



\usepackage{enumitem}% http://ctan.org/pkg/enumitem
\usepackage{multirow}
\usepackage{float}
\usepackage{amsmath}
\usepackage{multicol}
\usepackage{amssymb}
\usepackage[normalem]{ulem}
\useunder{\uline}{\ul}{}
\usepackage{wrapfig}


\usepackage[table,xcdraw]{xcolor}


%% ----------------------------------------------------------------
\begin{document}
\frontmatter
\title      {Heterogeneous Agent-based Model for Supermarket Competition}
\authors    {\texorpdfstring
             {\href{mailto:sc22g13@ecs.soton.ac.uk}{Stefan J. Collier}}
             {Stefan J. Collier}
            }
\addresses  {\groupname\\\deptname\\\univname}
\date       {\today}
\subject    {}
\keywords   {}
\supervisor {Dr. Maria Polukarov}
\examiner   {Professor Sheng Chen}

\maketitle
\begin{abstract}
This project aim was to model and analyse the effects of competitive pricing behaviors of grocery retailers on the British market. 

This was achieved by creating a multi-agent model, containing retailer and consumer agents. The heterogeneous crowd of retailers employs either a uniform pricing strategy or a ‘local price flexing’ strategy. The actions of these retailers are chosen by predicting the profit of each action, using a perceptron. Following on from the consideration of different economic models, a discrete model was developed so that software agents have a discrete environment to operate within. Within the model, it has been observed how supermarkets with differing behaviors affect a heterogeneous crowd of consumer agents. The model was implemented in Java with Python used to evaluate the results. 

The simulation displays good acceptance with real grocery market behavior, i.e. captures the performance of British retailers thus can be used to determine the impact of changes in their behavior on their competitors and consumers.Furthermore it can be used to provide insight into sustainability of volatile pricing strategies, providing a useful insight in volatility of British supermarket retail industry. 
\end{abstract}
\acknowledgements{
I would like to express my sincere gratitude to Dr Maria Polukarov for her guidance and support which provided me the freedom to take this research in the direction of my interest.\\
\\
I would also like to thank my family and friends for their encouragement and support. To those who quietly listened to my software complaints. To those who worked throughout the nights with me. To those who helped me write what I couldn't say. I cannot thank you enough.
}

\declaration{
I, Stefan Collier, declare that this dissertation and the work presented in it are my own and has been generated by me as the result of my own original research.\\
I confirm that:\\
1. This work was done wholly or mainly while in candidature for a degree at this University;\\
2. Where any part of this dissertation has previously been submitted for any other qualification at this University or any other institution, this has been clearly stated;\\
3. Where I have consulted the published work of others, this is always clearly attributed;\\
4. Where I have quoted from the work of others, the source is always given. With the exception of such quotations, this dissertation is entirely my own work;\\
5. I have acknowledged all main sources of help;\\
6. Where the thesis is based on work done by myself jointly with others, I have made clear exactly what was done by others and what I have contributed myself;\\
7. Either none of this work has been published before submission, or parts of this work have been published by :\\
\\
Stefan Collier\\
April 2016
}
\tableofcontents
\listoffigures
\listoftables

\mainmatter
%% ----------------------------------------------------------------
%\include{Introduction}
%\include{Conclusions}
\include{chapters/1Project/main}
\include{chapters/2Lit/main}
\include{chapters/3Design/HighLevel}
\include{chapters/3Design/InDepth}
\include{chapters/4Impl/main}

\include{chapters/5Experiments/1/main}
\include{chapters/5Experiments/2/main}
\include{chapters/5Experiments/3/main}
\include{chapters/5Experiments/4/main}

\include{chapters/6Conclusion/main}

\appendix
\include{appendix/AppendixB}
\include{appendix/D/main}
\include{appendix/AppendixC}

\backmatter
\bibliographystyle{ecs}
\bibliography{ECS}
\end{document}
%% ----------------------------------------------------------------


 %% ----------------------------------------------------------------
%% Progress.tex
%% ---------------------------------------------------------------- 
\documentclass{ecsprogress}    % Use the progress Style
\graphicspath{{../figs/}}   % Location of your graphics files
    \usepackage{natbib}            % Use Natbib style for the refs.
\hypersetup{colorlinks=true}   % Set to false for black/white printing
\input{Definitions}            % Include your abbreviations



\usepackage{enumitem}% http://ctan.org/pkg/enumitem
\usepackage{multirow}
\usepackage{float}
\usepackage{amsmath}
\usepackage{multicol}
\usepackage{amssymb}
\usepackage[normalem]{ulem}
\useunder{\uline}{\ul}{}
\usepackage{wrapfig}


\usepackage[table,xcdraw]{xcolor}


%% ----------------------------------------------------------------
\begin{document}
\frontmatter
\title      {Heterogeneous Agent-based Model for Supermarket Competition}
\authors    {\texorpdfstring
             {\href{mailto:sc22g13@ecs.soton.ac.uk}{Stefan J. Collier}}
             {Stefan J. Collier}
            }
\addresses  {\groupname\\\deptname\\\univname}
\date       {\today}
\subject    {}
\keywords   {}
\supervisor {Dr. Maria Polukarov}
\examiner   {Professor Sheng Chen}

\maketitle
\begin{abstract}
This project aim was to model and analyse the effects of competitive pricing behaviors of grocery retailers on the British market. 

This was achieved by creating a multi-agent model, containing retailer and consumer agents. The heterogeneous crowd of retailers employs either a uniform pricing strategy or a ‘local price flexing’ strategy. The actions of these retailers are chosen by predicting the profit of each action, using a perceptron. Following on from the consideration of different economic models, a discrete model was developed so that software agents have a discrete environment to operate within. Within the model, it has been observed how supermarkets with differing behaviors affect a heterogeneous crowd of consumer agents. The model was implemented in Java with Python used to evaluate the results. 

The simulation displays good acceptance with real grocery market behavior, i.e. captures the performance of British retailers thus can be used to determine the impact of changes in their behavior on their competitors and consumers.Furthermore it can be used to provide insight into sustainability of volatile pricing strategies, providing a useful insight in volatility of British supermarket retail industry. 
\end{abstract}
\acknowledgements{
I would like to express my sincere gratitude to Dr Maria Polukarov for her guidance and support which provided me the freedom to take this research in the direction of my interest.\\
\\
I would also like to thank my family and friends for their encouragement and support. To those who quietly listened to my software complaints. To those who worked throughout the nights with me. To those who helped me write what I couldn't say. I cannot thank you enough.
}

\declaration{
I, Stefan Collier, declare that this dissertation and the work presented in it are my own and has been generated by me as the result of my own original research.\\
I confirm that:\\
1. This work was done wholly or mainly while in candidature for a degree at this University;\\
2. Where any part of this dissertation has previously been submitted for any other qualification at this University or any other institution, this has been clearly stated;\\
3. Where I have consulted the published work of others, this is always clearly attributed;\\
4. Where I have quoted from the work of others, the source is always given. With the exception of such quotations, this dissertation is entirely my own work;\\
5. I have acknowledged all main sources of help;\\
6. Where the thesis is based on work done by myself jointly with others, I have made clear exactly what was done by others and what I have contributed myself;\\
7. Either none of this work has been published before submission, or parts of this work have been published by :\\
\\
Stefan Collier\\
April 2016
}
\tableofcontents
\listoffigures
\listoftables

\mainmatter
%% ----------------------------------------------------------------
%\include{Introduction}
%\include{Conclusions}
\include{chapters/1Project/main}
\include{chapters/2Lit/main}
\include{chapters/3Design/HighLevel}
\include{chapters/3Design/InDepth}
\include{chapters/4Impl/main}

\include{chapters/5Experiments/1/main}
\include{chapters/5Experiments/2/main}
\include{chapters/5Experiments/3/main}
\include{chapters/5Experiments/4/main}

\include{chapters/6Conclusion/main}

\appendix
\include{appendix/AppendixB}
\include{appendix/D/main}
\include{appendix/AppendixC}

\backmatter
\bibliographystyle{ecs}
\bibliography{ECS}
\end{document}
%% ----------------------------------------------------------------

 %% ----------------------------------------------------------------
%% Progress.tex
%% ---------------------------------------------------------------- 
\documentclass{ecsprogress}    % Use the progress Style
\graphicspath{{../figs/}}   % Location of your graphics files
    \usepackage{natbib}            % Use Natbib style for the refs.
\hypersetup{colorlinks=true}   % Set to false for black/white printing
\input{Definitions}            % Include your abbreviations



\usepackage{enumitem}% http://ctan.org/pkg/enumitem
\usepackage{multirow}
\usepackage{float}
\usepackage{amsmath}
\usepackage{multicol}
\usepackage{amssymb}
\usepackage[normalem]{ulem}
\useunder{\uline}{\ul}{}
\usepackage{wrapfig}


\usepackage[table,xcdraw]{xcolor}


%% ----------------------------------------------------------------
\begin{document}
\frontmatter
\title      {Heterogeneous Agent-based Model for Supermarket Competition}
\authors    {\texorpdfstring
             {\href{mailto:sc22g13@ecs.soton.ac.uk}{Stefan J. Collier}}
             {Stefan J. Collier}
            }
\addresses  {\groupname\\\deptname\\\univname}
\date       {\today}
\subject    {}
\keywords   {}
\supervisor {Dr. Maria Polukarov}
\examiner   {Professor Sheng Chen}

\maketitle
\begin{abstract}
This project aim was to model and analyse the effects of competitive pricing behaviors of grocery retailers on the British market. 

This was achieved by creating a multi-agent model, containing retailer and consumer agents. The heterogeneous crowd of retailers employs either a uniform pricing strategy or a ‘local price flexing’ strategy. The actions of these retailers are chosen by predicting the profit of each action, using a perceptron. Following on from the consideration of different economic models, a discrete model was developed so that software agents have a discrete environment to operate within. Within the model, it has been observed how supermarkets with differing behaviors affect a heterogeneous crowd of consumer agents. The model was implemented in Java with Python used to evaluate the results. 

The simulation displays good acceptance with real grocery market behavior, i.e. captures the performance of British retailers thus can be used to determine the impact of changes in their behavior on their competitors and consumers.Furthermore it can be used to provide insight into sustainability of volatile pricing strategies, providing a useful insight in volatility of British supermarket retail industry. 
\end{abstract}
\acknowledgements{
I would like to express my sincere gratitude to Dr Maria Polukarov for her guidance and support which provided me the freedom to take this research in the direction of my interest.\\
\\
I would also like to thank my family and friends for their encouragement and support. To those who quietly listened to my software complaints. To those who worked throughout the nights with me. To those who helped me write what I couldn't say. I cannot thank you enough.
}

\declaration{
I, Stefan Collier, declare that this dissertation and the work presented in it are my own and has been generated by me as the result of my own original research.\\
I confirm that:\\
1. This work was done wholly or mainly while in candidature for a degree at this University;\\
2. Where any part of this dissertation has previously been submitted for any other qualification at this University or any other institution, this has been clearly stated;\\
3. Where I have consulted the published work of others, this is always clearly attributed;\\
4. Where I have quoted from the work of others, the source is always given. With the exception of such quotations, this dissertation is entirely my own work;\\
5. I have acknowledged all main sources of help;\\
6. Where the thesis is based on work done by myself jointly with others, I have made clear exactly what was done by others and what I have contributed myself;\\
7. Either none of this work has been published before submission, or parts of this work have been published by :\\
\\
Stefan Collier\\
April 2016
}
\tableofcontents
\listoffigures
\listoftables

\mainmatter
%% ----------------------------------------------------------------
%\include{Introduction}
%\include{Conclusions}
\include{chapters/1Project/main}
\include{chapters/2Lit/main}
\include{chapters/3Design/HighLevel}
\include{chapters/3Design/InDepth}
\include{chapters/4Impl/main}

\include{chapters/5Experiments/1/main}
\include{chapters/5Experiments/2/main}
\include{chapters/5Experiments/3/main}
\include{chapters/5Experiments/4/main}

\include{chapters/6Conclusion/main}

\appendix
\include{appendix/AppendixB}
\include{appendix/D/main}
\include{appendix/AppendixC}

\backmatter
\bibliographystyle{ecs}
\bibliography{ECS}
\end{document}
%% ----------------------------------------------------------------

 %% ----------------------------------------------------------------
%% Progress.tex
%% ---------------------------------------------------------------- 
\documentclass{ecsprogress}    % Use the progress Style
\graphicspath{{../figs/}}   % Location of your graphics files
    \usepackage{natbib}            % Use Natbib style for the refs.
\hypersetup{colorlinks=true}   % Set to false for black/white printing
\input{Definitions}            % Include your abbreviations



\usepackage{enumitem}% http://ctan.org/pkg/enumitem
\usepackage{multirow}
\usepackage{float}
\usepackage{amsmath}
\usepackage{multicol}
\usepackage{amssymb}
\usepackage[normalem]{ulem}
\useunder{\uline}{\ul}{}
\usepackage{wrapfig}


\usepackage[table,xcdraw]{xcolor}


%% ----------------------------------------------------------------
\begin{document}
\frontmatter
\title      {Heterogeneous Agent-based Model for Supermarket Competition}
\authors    {\texorpdfstring
             {\href{mailto:sc22g13@ecs.soton.ac.uk}{Stefan J. Collier}}
             {Stefan J. Collier}
            }
\addresses  {\groupname\\\deptname\\\univname}
\date       {\today}
\subject    {}
\keywords   {}
\supervisor {Dr. Maria Polukarov}
\examiner   {Professor Sheng Chen}

\maketitle
\begin{abstract}
This project aim was to model and analyse the effects of competitive pricing behaviors of grocery retailers on the British market. 

This was achieved by creating a multi-agent model, containing retailer and consumer agents. The heterogeneous crowd of retailers employs either a uniform pricing strategy or a ‘local price flexing’ strategy. The actions of these retailers are chosen by predicting the profit of each action, using a perceptron. Following on from the consideration of different economic models, a discrete model was developed so that software agents have a discrete environment to operate within. Within the model, it has been observed how supermarkets with differing behaviors affect a heterogeneous crowd of consumer agents. The model was implemented in Java with Python used to evaluate the results. 

The simulation displays good acceptance with real grocery market behavior, i.e. captures the performance of British retailers thus can be used to determine the impact of changes in their behavior on their competitors and consumers.Furthermore it can be used to provide insight into sustainability of volatile pricing strategies, providing a useful insight in volatility of British supermarket retail industry. 
\end{abstract}
\acknowledgements{
I would like to express my sincere gratitude to Dr Maria Polukarov for her guidance and support which provided me the freedom to take this research in the direction of my interest.\\
\\
I would also like to thank my family and friends for their encouragement and support. To those who quietly listened to my software complaints. To those who worked throughout the nights with me. To those who helped me write what I couldn't say. I cannot thank you enough.
}

\declaration{
I, Stefan Collier, declare that this dissertation and the work presented in it are my own and has been generated by me as the result of my own original research.\\
I confirm that:\\
1. This work was done wholly or mainly while in candidature for a degree at this University;\\
2. Where any part of this dissertation has previously been submitted for any other qualification at this University or any other institution, this has been clearly stated;\\
3. Where I have consulted the published work of others, this is always clearly attributed;\\
4. Where I have quoted from the work of others, the source is always given. With the exception of such quotations, this dissertation is entirely my own work;\\
5. I have acknowledged all main sources of help;\\
6. Where the thesis is based on work done by myself jointly with others, I have made clear exactly what was done by others and what I have contributed myself;\\
7. Either none of this work has been published before submission, or parts of this work have been published by :\\
\\
Stefan Collier\\
April 2016
}
\tableofcontents
\listoffigures
\listoftables

\mainmatter
%% ----------------------------------------------------------------
%\include{Introduction}
%\include{Conclusions}
\include{chapters/1Project/main}
\include{chapters/2Lit/main}
\include{chapters/3Design/HighLevel}
\include{chapters/3Design/InDepth}
\include{chapters/4Impl/main}

\include{chapters/5Experiments/1/main}
\include{chapters/5Experiments/2/main}
\include{chapters/5Experiments/3/main}
\include{chapters/5Experiments/4/main}

\include{chapters/6Conclusion/main}

\appendix
\include{appendix/AppendixB}
\include{appendix/D/main}
\include{appendix/AppendixC}

\backmatter
\bibliographystyle{ecs}
\bibliography{ECS}
\end{document}
%% ----------------------------------------------------------------

 %% ----------------------------------------------------------------
%% Progress.tex
%% ---------------------------------------------------------------- 
\documentclass{ecsprogress}    % Use the progress Style
\graphicspath{{../figs/}}   % Location of your graphics files
    \usepackage{natbib}            % Use Natbib style for the refs.
\hypersetup{colorlinks=true}   % Set to false for black/white printing
\input{Definitions}            % Include your abbreviations



\usepackage{enumitem}% http://ctan.org/pkg/enumitem
\usepackage{multirow}
\usepackage{float}
\usepackage{amsmath}
\usepackage{multicol}
\usepackage{amssymb}
\usepackage[normalem]{ulem}
\useunder{\uline}{\ul}{}
\usepackage{wrapfig}


\usepackage[table,xcdraw]{xcolor}


%% ----------------------------------------------------------------
\begin{document}
\frontmatter
\title      {Heterogeneous Agent-based Model for Supermarket Competition}
\authors    {\texorpdfstring
             {\href{mailto:sc22g13@ecs.soton.ac.uk}{Stefan J. Collier}}
             {Stefan J. Collier}
            }
\addresses  {\groupname\\\deptname\\\univname}
\date       {\today}
\subject    {}
\keywords   {}
\supervisor {Dr. Maria Polukarov}
\examiner   {Professor Sheng Chen}

\maketitle
\begin{abstract}
This project aim was to model and analyse the effects of competitive pricing behaviors of grocery retailers on the British market. 

This was achieved by creating a multi-agent model, containing retailer and consumer agents. The heterogeneous crowd of retailers employs either a uniform pricing strategy or a ‘local price flexing’ strategy. The actions of these retailers are chosen by predicting the profit of each action, using a perceptron. Following on from the consideration of different economic models, a discrete model was developed so that software agents have a discrete environment to operate within. Within the model, it has been observed how supermarkets with differing behaviors affect a heterogeneous crowd of consumer agents. The model was implemented in Java with Python used to evaluate the results. 

The simulation displays good acceptance with real grocery market behavior, i.e. captures the performance of British retailers thus can be used to determine the impact of changes in their behavior on their competitors and consumers.Furthermore it can be used to provide insight into sustainability of volatile pricing strategies, providing a useful insight in volatility of British supermarket retail industry. 
\end{abstract}
\acknowledgements{
I would like to express my sincere gratitude to Dr Maria Polukarov for her guidance and support which provided me the freedom to take this research in the direction of my interest.\\
\\
I would also like to thank my family and friends for their encouragement and support. To those who quietly listened to my software complaints. To those who worked throughout the nights with me. To those who helped me write what I couldn't say. I cannot thank you enough.
}

\declaration{
I, Stefan Collier, declare that this dissertation and the work presented in it are my own and has been generated by me as the result of my own original research.\\
I confirm that:\\
1. This work was done wholly or mainly while in candidature for a degree at this University;\\
2. Where any part of this dissertation has previously been submitted for any other qualification at this University or any other institution, this has been clearly stated;\\
3. Where I have consulted the published work of others, this is always clearly attributed;\\
4. Where I have quoted from the work of others, the source is always given. With the exception of such quotations, this dissertation is entirely my own work;\\
5. I have acknowledged all main sources of help;\\
6. Where the thesis is based on work done by myself jointly with others, I have made clear exactly what was done by others and what I have contributed myself;\\
7. Either none of this work has been published before submission, or parts of this work have been published by :\\
\\
Stefan Collier\\
April 2016
}
\tableofcontents
\listoffigures
\listoftables

\mainmatter
%% ----------------------------------------------------------------
%\include{Introduction}
%\include{Conclusions}
\include{chapters/1Project/main}
\include{chapters/2Lit/main}
\include{chapters/3Design/HighLevel}
\include{chapters/3Design/InDepth}
\include{chapters/4Impl/main}

\include{chapters/5Experiments/1/main}
\include{chapters/5Experiments/2/main}
\include{chapters/5Experiments/3/main}
\include{chapters/5Experiments/4/main}

\include{chapters/6Conclusion/main}

\appendix
\include{appendix/AppendixB}
\include{appendix/D/main}
\include{appendix/AppendixC}

\backmatter
\bibliographystyle{ecs}
\bibliography{ECS}
\end{document}
%% ----------------------------------------------------------------


 %% ----------------------------------------------------------------
%% Progress.tex
%% ---------------------------------------------------------------- 
\documentclass{ecsprogress}    % Use the progress Style
\graphicspath{{../figs/}}   % Location of your graphics files
    \usepackage{natbib}            % Use Natbib style for the refs.
\hypersetup{colorlinks=true}   % Set to false for black/white printing
\input{Definitions}            % Include your abbreviations



\usepackage{enumitem}% http://ctan.org/pkg/enumitem
\usepackage{multirow}
\usepackage{float}
\usepackage{amsmath}
\usepackage{multicol}
\usepackage{amssymb}
\usepackage[normalem]{ulem}
\useunder{\uline}{\ul}{}
\usepackage{wrapfig}


\usepackage[table,xcdraw]{xcolor}


%% ----------------------------------------------------------------
\begin{document}
\frontmatter
\title      {Heterogeneous Agent-based Model for Supermarket Competition}
\authors    {\texorpdfstring
             {\href{mailto:sc22g13@ecs.soton.ac.uk}{Stefan J. Collier}}
             {Stefan J. Collier}
            }
\addresses  {\groupname\\\deptname\\\univname}
\date       {\today}
\subject    {}
\keywords   {}
\supervisor {Dr. Maria Polukarov}
\examiner   {Professor Sheng Chen}

\maketitle
\begin{abstract}
This project aim was to model and analyse the effects of competitive pricing behaviors of grocery retailers on the British market. 

This was achieved by creating a multi-agent model, containing retailer and consumer agents. The heterogeneous crowd of retailers employs either a uniform pricing strategy or a ‘local price flexing’ strategy. The actions of these retailers are chosen by predicting the profit of each action, using a perceptron. Following on from the consideration of different economic models, a discrete model was developed so that software agents have a discrete environment to operate within. Within the model, it has been observed how supermarkets with differing behaviors affect a heterogeneous crowd of consumer agents. The model was implemented in Java with Python used to evaluate the results. 

The simulation displays good acceptance with real grocery market behavior, i.e. captures the performance of British retailers thus can be used to determine the impact of changes in their behavior on their competitors and consumers.Furthermore it can be used to provide insight into sustainability of volatile pricing strategies, providing a useful insight in volatility of British supermarket retail industry. 
\end{abstract}
\acknowledgements{
I would like to express my sincere gratitude to Dr Maria Polukarov for her guidance and support which provided me the freedom to take this research in the direction of my interest.\\
\\
I would also like to thank my family and friends for their encouragement and support. To those who quietly listened to my software complaints. To those who worked throughout the nights with me. To those who helped me write what I couldn't say. I cannot thank you enough.
}

\declaration{
I, Stefan Collier, declare that this dissertation and the work presented in it are my own and has been generated by me as the result of my own original research.\\
I confirm that:\\
1. This work was done wholly or mainly while in candidature for a degree at this University;\\
2. Where any part of this dissertation has previously been submitted for any other qualification at this University or any other institution, this has been clearly stated;\\
3. Where I have consulted the published work of others, this is always clearly attributed;\\
4. Where I have quoted from the work of others, the source is always given. With the exception of such quotations, this dissertation is entirely my own work;\\
5. I have acknowledged all main sources of help;\\
6. Where the thesis is based on work done by myself jointly with others, I have made clear exactly what was done by others and what I have contributed myself;\\
7. Either none of this work has been published before submission, or parts of this work have been published by :\\
\\
Stefan Collier\\
April 2016
}
\tableofcontents
\listoffigures
\listoftables

\mainmatter
%% ----------------------------------------------------------------
%\include{Introduction}
%\include{Conclusions}
\include{chapters/1Project/main}
\include{chapters/2Lit/main}
\include{chapters/3Design/HighLevel}
\include{chapters/3Design/InDepth}
\include{chapters/4Impl/main}

\include{chapters/5Experiments/1/main}
\include{chapters/5Experiments/2/main}
\include{chapters/5Experiments/3/main}
\include{chapters/5Experiments/4/main}

\include{chapters/6Conclusion/main}

\appendix
\include{appendix/AppendixB}
\include{appendix/D/main}
\include{appendix/AppendixC}

\backmatter
\bibliographystyle{ecs}
\bibliography{ECS}
\end{document}
%% ----------------------------------------------------------------


\appendix
\include{appendix/AppendixB}
 %% ----------------------------------------------------------------
%% Progress.tex
%% ---------------------------------------------------------------- 
\documentclass{ecsprogress}    % Use the progress Style
\graphicspath{{../figs/}}   % Location of your graphics files
    \usepackage{natbib}            % Use Natbib style for the refs.
\hypersetup{colorlinks=true}   % Set to false for black/white printing
\input{Definitions}            % Include your abbreviations



\usepackage{enumitem}% http://ctan.org/pkg/enumitem
\usepackage{multirow}
\usepackage{float}
\usepackage{amsmath}
\usepackage{multicol}
\usepackage{amssymb}
\usepackage[normalem]{ulem}
\useunder{\uline}{\ul}{}
\usepackage{wrapfig}


\usepackage[table,xcdraw]{xcolor}


%% ----------------------------------------------------------------
\begin{document}
\frontmatter
\title      {Heterogeneous Agent-based Model for Supermarket Competition}
\authors    {\texorpdfstring
             {\href{mailto:sc22g13@ecs.soton.ac.uk}{Stefan J. Collier}}
             {Stefan J. Collier}
            }
\addresses  {\groupname\\\deptname\\\univname}
\date       {\today}
\subject    {}
\keywords   {}
\supervisor {Dr. Maria Polukarov}
\examiner   {Professor Sheng Chen}

\maketitle
\begin{abstract}
This project aim was to model and analyse the effects of competitive pricing behaviors of grocery retailers on the British market. 

This was achieved by creating a multi-agent model, containing retailer and consumer agents. The heterogeneous crowd of retailers employs either a uniform pricing strategy or a ‘local price flexing’ strategy. The actions of these retailers are chosen by predicting the profit of each action, using a perceptron. Following on from the consideration of different economic models, a discrete model was developed so that software agents have a discrete environment to operate within. Within the model, it has been observed how supermarkets with differing behaviors affect a heterogeneous crowd of consumer agents. The model was implemented in Java with Python used to evaluate the results. 

The simulation displays good acceptance with real grocery market behavior, i.e. captures the performance of British retailers thus can be used to determine the impact of changes in their behavior on their competitors and consumers.Furthermore it can be used to provide insight into sustainability of volatile pricing strategies, providing a useful insight in volatility of British supermarket retail industry. 
\end{abstract}
\acknowledgements{
I would like to express my sincere gratitude to Dr Maria Polukarov for her guidance and support which provided me the freedom to take this research in the direction of my interest.\\
\\
I would also like to thank my family and friends for their encouragement and support. To those who quietly listened to my software complaints. To those who worked throughout the nights with me. To those who helped me write what I couldn't say. I cannot thank you enough.
}

\declaration{
I, Stefan Collier, declare that this dissertation and the work presented in it are my own and has been generated by me as the result of my own original research.\\
I confirm that:\\
1. This work was done wholly or mainly while in candidature for a degree at this University;\\
2. Where any part of this dissertation has previously been submitted for any other qualification at this University or any other institution, this has been clearly stated;\\
3. Where I have consulted the published work of others, this is always clearly attributed;\\
4. Where I have quoted from the work of others, the source is always given. With the exception of such quotations, this dissertation is entirely my own work;\\
5. I have acknowledged all main sources of help;\\
6. Where the thesis is based on work done by myself jointly with others, I have made clear exactly what was done by others and what I have contributed myself;\\
7. Either none of this work has been published before submission, or parts of this work have been published by :\\
\\
Stefan Collier\\
April 2016
}
\tableofcontents
\listoffigures
\listoftables

\mainmatter
%% ----------------------------------------------------------------
%\include{Introduction}
%\include{Conclusions}
\include{chapters/1Project/main}
\include{chapters/2Lit/main}
\include{chapters/3Design/HighLevel}
\include{chapters/3Design/InDepth}
\include{chapters/4Impl/main}

\include{chapters/5Experiments/1/main}
\include{chapters/5Experiments/2/main}
\include{chapters/5Experiments/3/main}
\include{chapters/5Experiments/4/main}

\include{chapters/6Conclusion/main}

\appendix
\include{appendix/AppendixB}
\include{appendix/D/main}
\include{appendix/AppendixC}

\backmatter
\bibliographystyle{ecs}
\bibliography{ECS}
\end{document}
%% ----------------------------------------------------------------

\include{appendix/AppendixC}

\backmatter
\bibliographystyle{ecs}
\bibliography{ECS}
\end{document}
%% ----------------------------------------------------------------

 %% ----------------------------------------------------------------
%% Progress.tex
%% ---------------------------------------------------------------- 
\documentclass{ecsprogress}    % Use the progress Style
\graphicspath{{../figs/}}   % Location of your graphics files
    \usepackage{natbib}            % Use Natbib style for the refs.
\hypersetup{colorlinks=true}   % Set to false for black/white printing
\input{Definitions}            % Include your abbreviations



\usepackage{enumitem}% http://ctan.org/pkg/enumitem
\usepackage{multirow}
\usepackage{float}
\usepackage{amsmath}
\usepackage{multicol}
\usepackage{amssymb}
\usepackage[normalem]{ulem}
\useunder{\uline}{\ul}{}
\usepackage{wrapfig}


\usepackage[table,xcdraw]{xcolor}


%% ----------------------------------------------------------------
\begin{document}
\frontmatter
\title      {Heterogeneous Agent-based Model for Supermarket Competition}
\authors    {\texorpdfstring
             {\href{mailto:sc22g13@ecs.soton.ac.uk}{Stefan J. Collier}}
             {Stefan J. Collier}
            }
\addresses  {\groupname\\\deptname\\\univname}
\date       {\today}
\subject    {}
\keywords   {}
\supervisor {Dr. Maria Polukarov}
\examiner   {Professor Sheng Chen}

\maketitle
\begin{abstract}
This project aim was to model and analyse the effects of competitive pricing behaviors of grocery retailers on the British market. 

This was achieved by creating a multi-agent model, containing retailer and consumer agents. The heterogeneous crowd of retailers employs either a uniform pricing strategy or a ‘local price flexing’ strategy. The actions of these retailers are chosen by predicting the profit of each action, using a perceptron. Following on from the consideration of different economic models, a discrete model was developed so that software agents have a discrete environment to operate within. Within the model, it has been observed how supermarkets with differing behaviors affect a heterogeneous crowd of consumer agents. The model was implemented in Java with Python used to evaluate the results. 

The simulation displays good acceptance with real grocery market behavior, i.e. captures the performance of British retailers thus can be used to determine the impact of changes in their behavior on their competitors and consumers.Furthermore it can be used to provide insight into sustainability of volatile pricing strategies, providing a useful insight in volatility of British supermarket retail industry. 
\end{abstract}
\acknowledgements{
I would like to express my sincere gratitude to Dr Maria Polukarov for her guidance and support which provided me the freedom to take this research in the direction of my interest.\\
\\
I would also like to thank my family and friends for their encouragement and support. To those who quietly listened to my software complaints. To those who worked throughout the nights with me. To those who helped me write what I couldn't say. I cannot thank you enough.
}

\declaration{
I, Stefan Collier, declare that this dissertation and the work presented in it are my own and has been generated by me as the result of my own original research.\\
I confirm that:\\
1. This work was done wholly or mainly while in candidature for a degree at this University;\\
2. Where any part of this dissertation has previously been submitted for any other qualification at this University or any other institution, this has been clearly stated;\\
3. Where I have consulted the published work of others, this is always clearly attributed;\\
4. Where I have quoted from the work of others, the source is always given. With the exception of such quotations, this dissertation is entirely my own work;\\
5. I have acknowledged all main sources of help;\\
6. Where the thesis is based on work done by myself jointly with others, I have made clear exactly what was done by others and what I have contributed myself;\\
7. Either none of this work has been published before submission, or parts of this work have been published by :\\
\\
Stefan Collier\\
April 2016
}
\tableofcontents
\listoffigures
\listoftables

\mainmatter
%% ----------------------------------------------------------------
%\include{Introduction}
%\include{Conclusions}
 %% ----------------------------------------------------------------
%% Progress.tex
%% ---------------------------------------------------------------- 
\documentclass{ecsprogress}    % Use the progress Style
\graphicspath{{../figs/}}   % Location of your graphics files
    \usepackage{natbib}            % Use Natbib style for the refs.
\hypersetup{colorlinks=true}   % Set to false for black/white printing
\input{Definitions}            % Include your abbreviations



\usepackage{enumitem}% http://ctan.org/pkg/enumitem
\usepackage{multirow}
\usepackage{float}
\usepackage{amsmath}
\usepackage{multicol}
\usepackage{amssymb}
\usepackage[normalem]{ulem}
\useunder{\uline}{\ul}{}
\usepackage{wrapfig}


\usepackage[table,xcdraw]{xcolor}


%% ----------------------------------------------------------------
\begin{document}
\frontmatter
\title      {Heterogeneous Agent-based Model for Supermarket Competition}
\authors    {\texorpdfstring
             {\href{mailto:sc22g13@ecs.soton.ac.uk}{Stefan J. Collier}}
             {Stefan J. Collier}
            }
\addresses  {\groupname\\\deptname\\\univname}
\date       {\today}
\subject    {}
\keywords   {}
\supervisor {Dr. Maria Polukarov}
\examiner   {Professor Sheng Chen}

\maketitle
\begin{abstract}
This project aim was to model and analyse the effects of competitive pricing behaviors of grocery retailers on the British market. 

This was achieved by creating a multi-agent model, containing retailer and consumer agents. The heterogeneous crowd of retailers employs either a uniform pricing strategy or a ‘local price flexing’ strategy. The actions of these retailers are chosen by predicting the profit of each action, using a perceptron. Following on from the consideration of different economic models, a discrete model was developed so that software agents have a discrete environment to operate within. Within the model, it has been observed how supermarkets with differing behaviors affect a heterogeneous crowd of consumer agents. The model was implemented in Java with Python used to evaluate the results. 

The simulation displays good acceptance with real grocery market behavior, i.e. captures the performance of British retailers thus can be used to determine the impact of changes in their behavior on their competitors and consumers.Furthermore it can be used to provide insight into sustainability of volatile pricing strategies, providing a useful insight in volatility of British supermarket retail industry. 
\end{abstract}
\acknowledgements{
I would like to express my sincere gratitude to Dr Maria Polukarov for her guidance and support which provided me the freedom to take this research in the direction of my interest.\\
\\
I would also like to thank my family and friends for their encouragement and support. To those who quietly listened to my software complaints. To those who worked throughout the nights with me. To those who helped me write what I couldn't say. I cannot thank you enough.
}

\declaration{
I, Stefan Collier, declare that this dissertation and the work presented in it are my own and has been generated by me as the result of my own original research.\\
I confirm that:\\
1. This work was done wholly or mainly while in candidature for a degree at this University;\\
2. Where any part of this dissertation has previously been submitted for any other qualification at this University or any other institution, this has been clearly stated;\\
3. Where I have consulted the published work of others, this is always clearly attributed;\\
4. Where I have quoted from the work of others, the source is always given. With the exception of such quotations, this dissertation is entirely my own work;\\
5. I have acknowledged all main sources of help;\\
6. Where the thesis is based on work done by myself jointly with others, I have made clear exactly what was done by others and what I have contributed myself;\\
7. Either none of this work has been published before submission, or parts of this work have been published by :\\
\\
Stefan Collier\\
April 2016
}
\tableofcontents
\listoffigures
\listoftables

\mainmatter
%% ----------------------------------------------------------------
%\include{Introduction}
%\include{Conclusions}
\include{chapters/1Project/main}
\include{chapters/2Lit/main}
\include{chapters/3Design/HighLevel}
\include{chapters/3Design/InDepth}
\include{chapters/4Impl/main}

\include{chapters/5Experiments/1/main}
\include{chapters/5Experiments/2/main}
\include{chapters/5Experiments/3/main}
\include{chapters/5Experiments/4/main}

\include{chapters/6Conclusion/main}

\appendix
\include{appendix/AppendixB}
\include{appendix/D/main}
\include{appendix/AppendixC}

\backmatter
\bibliographystyle{ecs}
\bibliography{ECS}
\end{document}
%% ----------------------------------------------------------------

 %% ----------------------------------------------------------------
%% Progress.tex
%% ---------------------------------------------------------------- 
\documentclass{ecsprogress}    % Use the progress Style
\graphicspath{{../figs/}}   % Location of your graphics files
    \usepackage{natbib}            % Use Natbib style for the refs.
\hypersetup{colorlinks=true}   % Set to false for black/white printing
\input{Definitions}            % Include your abbreviations



\usepackage{enumitem}% http://ctan.org/pkg/enumitem
\usepackage{multirow}
\usepackage{float}
\usepackage{amsmath}
\usepackage{multicol}
\usepackage{amssymb}
\usepackage[normalem]{ulem}
\useunder{\uline}{\ul}{}
\usepackage{wrapfig}


\usepackage[table,xcdraw]{xcolor}


%% ----------------------------------------------------------------
\begin{document}
\frontmatter
\title      {Heterogeneous Agent-based Model for Supermarket Competition}
\authors    {\texorpdfstring
             {\href{mailto:sc22g13@ecs.soton.ac.uk}{Stefan J. Collier}}
             {Stefan J. Collier}
            }
\addresses  {\groupname\\\deptname\\\univname}
\date       {\today}
\subject    {}
\keywords   {}
\supervisor {Dr. Maria Polukarov}
\examiner   {Professor Sheng Chen}

\maketitle
\begin{abstract}
This project aim was to model and analyse the effects of competitive pricing behaviors of grocery retailers on the British market. 

This was achieved by creating a multi-agent model, containing retailer and consumer agents. The heterogeneous crowd of retailers employs either a uniform pricing strategy or a ‘local price flexing’ strategy. The actions of these retailers are chosen by predicting the profit of each action, using a perceptron. Following on from the consideration of different economic models, a discrete model was developed so that software agents have a discrete environment to operate within. Within the model, it has been observed how supermarkets with differing behaviors affect a heterogeneous crowd of consumer agents. The model was implemented in Java with Python used to evaluate the results. 

The simulation displays good acceptance with real grocery market behavior, i.e. captures the performance of British retailers thus can be used to determine the impact of changes in their behavior on their competitors and consumers.Furthermore it can be used to provide insight into sustainability of volatile pricing strategies, providing a useful insight in volatility of British supermarket retail industry. 
\end{abstract}
\acknowledgements{
I would like to express my sincere gratitude to Dr Maria Polukarov for her guidance and support which provided me the freedom to take this research in the direction of my interest.\\
\\
I would also like to thank my family and friends for their encouragement and support. To those who quietly listened to my software complaints. To those who worked throughout the nights with me. To those who helped me write what I couldn't say. I cannot thank you enough.
}

\declaration{
I, Stefan Collier, declare that this dissertation and the work presented in it are my own and has been generated by me as the result of my own original research.\\
I confirm that:\\
1. This work was done wholly or mainly while in candidature for a degree at this University;\\
2. Where any part of this dissertation has previously been submitted for any other qualification at this University or any other institution, this has been clearly stated;\\
3. Where I have consulted the published work of others, this is always clearly attributed;\\
4. Where I have quoted from the work of others, the source is always given. With the exception of such quotations, this dissertation is entirely my own work;\\
5. I have acknowledged all main sources of help;\\
6. Where the thesis is based on work done by myself jointly with others, I have made clear exactly what was done by others and what I have contributed myself;\\
7. Either none of this work has been published before submission, or parts of this work have been published by :\\
\\
Stefan Collier\\
April 2016
}
\tableofcontents
\listoffigures
\listoftables

\mainmatter
%% ----------------------------------------------------------------
%\include{Introduction}
%\include{Conclusions}
\include{chapters/1Project/main}
\include{chapters/2Lit/main}
\include{chapters/3Design/HighLevel}
\include{chapters/3Design/InDepth}
\include{chapters/4Impl/main}

\include{chapters/5Experiments/1/main}
\include{chapters/5Experiments/2/main}
\include{chapters/5Experiments/3/main}
\include{chapters/5Experiments/4/main}

\include{chapters/6Conclusion/main}

\appendix
\include{appendix/AppendixB}
\include{appendix/D/main}
\include{appendix/AppendixC}

\backmatter
\bibliographystyle{ecs}
\bibliography{ECS}
\end{document}
%% ----------------------------------------------------------------

\include{chapters/3Design/HighLevel}
\include{chapters/3Design/InDepth}
 %% ----------------------------------------------------------------
%% Progress.tex
%% ---------------------------------------------------------------- 
\documentclass{ecsprogress}    % Use the progress Style
\graphicspath{{../figs/}}   % Location of your graphics files
    \usepackage{natbib}            % Use Natbib style for the refs.
\hypersetup{colorlinks=true}   % Set to false for black/white printing
\input{Definitions}            % Include your abbreviations



\usepackage{enumitem}% http://ctan.org/pkg/enumitem
\usepackage{multirow}
\usepackage{float}
\usepackage{amsmath}
\usepackage{multicol}
\usepackage{amssymb}
\usepackage[normalem]{ulem}
\useunder{\uline}{\ul}{}
\usepackage{wrapfig}


\usepackage[table,xcdraw]{xcolor}


%% ----------------------------------------------------------------
\begin{document}
\frontmatter
\title      {Heterogeneous Agent-based Model for Supermarket Competition}
\authors    {\texorpdfstring
             {\href{mailto:sc22g13@ecs.soton.ac.uk}{Stefan J. Collier}}
             {Stefan J. Collier}
            }
\addresses  {\groupname\\\deptname\\\univname}
\date       {\today}
\subject    {}
\keywords   {}
\supervisor {Dr. Maria Polukarov}
\examiner   {Professor Sheng Chen}

\maketitle
\begin{abstract}
This project aim was to model and analyse the effects of competitive pricing behaviors of grocery retailers on the British market. 

This was achieved by creating a multi-agent model, containing retailer and consumer agents. The heterogeneous crowd of retailers employs either a uniform pricing strategy or a ‘local price flexing’ strategy. The actions of these retailers are chosen by predicting the profit of each action, using a perceptron. Following on from the consideration of different economic models, a discrete model was developed so that software agents have a discrete environment to operate within. Within the model, it has been observed how supermarkets with differing behaviors affect a heterogeneous crowd of consumer agents. The model was implemented in Java with Python used to evaluate the results. 

The simulation displays good acceptance with real grocery market behavior, i.e. captures the performance of British retailers thus can be used to determine the impact of changes in their behavior on their competitors and consumers.Furthermore it can be used to provide insight into sustainability of volatile pricing strategies, providing a useful insight in volatility of British supermarket retail industry. 
\end{abstract}
\acknowledgements{
I would like to express my sincere gratitude to Dr Maria Polukarov for her guidance and support which provided me the freedom to take this research in the direction of my interest.\\
\\
I would also like to thank my family and friends for their encouragement and support. To those who quietly listened to my software complaints. To those who worked throughout the nights with me. To those who helped me write what I couldn't say. I cannot thank you enough.
}

\declaration{
I, Stefan Collier, declare that this dissertation and the work presented in it are my own and has been generated by me as the result of my own original research.\\
I confirm that:\\
1. This work was done wholly or mainly while in candidature for a degree at this University;\\
2. Where any part of this dissertation has previously been submitted for any other qualification at this University or any other institution, this has been clearly stated;\\
3. Where I have consulted the published work of others, this is always clearly attributed;\\
4. Where I have quoted from the work of others, the source is always given. With the exception of such quotations, this dissertation is entirely my own work;\\
5. I have acknowledged all main sources of help;\\
6. Where the thesis is based on work done by myself jointly with others, I have made clear exactly what was done by others and what I have contributed myself;\\
7. Either none of this work has been published before submission, or parts of this work have been published by :\\
\\
Stefan Collier\\
April 2016
}
\tableofcontents
\listoffigures
\listoftables

\mainmatter
%% ----------------------------------------------------------------
%\include{Introduction}
%\include{Conclusions}
\include{chapters/1Project/main}
\include{chapters/2Lit/main}
\include{chapters/3Design/HighLevel}
\include{chapters/3Design/InDepth}
\include{chapters/4Impl/main}

\include{chapters/5Experiments/1/main}
\include{chapters/5Experiments/2/main}
\include{chapters/5Experiments/3/main}
\include{chapters/5Experiments/4/main}

\include{chapters/6Conclusion/main}

\appendix
\include{appendix/AppendixB}
\include{appendix/D/main}
\include{appendix/AppendixC}

\backmatter
\bibliographystyle{ecs}
\bibliography{ECS}
\end{document}
%% ----------------------------------------------------------------


 %% ----------------------------------------------------------------
%% Progress.tex
%% ---------------------------------------------------------------- 
\documentclass{ecsprogress}    % Use the progress Style
\graphicspath{{../figs/}}   % Location of your graphics files
    \usepackage{natbib}            % Use Natbib style for the refs.
\hypersetup{colorlinks=true}   % Set to false for black/white printing
\input{Definitions}            % Include your abbreviations



\usepackage{enumitem}% http://ctan.org/pkg/enumitem
\usepackage{multirow}
\usepackage{float}
\usepackage{amsmath}
\usepackage{multicol}
\usepackage{amssymb}
\usepackage[normalem]{ulem}
\useunder{\uline}{\ul}{}
\usepackage{wrapfig}


\usepackage[table,xcdraw]{xcolor}


%% ----------------------------------------------------------------
\begin{document}
\frontmatter
\title      {Heterogeneous Agent-based Model for Supermarket Competition}
\authors    {\texorpdfstring
             {\href{mailto:sc22g13@ecs.soton.ac.uk}{Stefan J. Collier}}
             {Stefan J. Collier}
            }
\addresses  {\groupname\\\deptname\\\univname}
\date       {\today}
\subject    {}
\keywords   {}
\supervisor {Dr. Maria Polukarov}
\examiner   {Professor Sheng Chen}

\maketitle
\begin{abstract}
This project aim was to model and analyse the effects of competitive pricing behaviors of grocery retailers on the British market. 

This was achieved by creating a multi-agent model, containing retailer and consumer agents. The heterogeneous crowd of retailers employs either a uniform pricing strategy or a ‘local price flexing’ strategy. The actions of these retailers are chosen by predicting the profit of each action, using a perceptron. Following on from the consideration of different economic models, a discrete model was developed so that software agents have a discrete environment to operate within. Within the model, it has been observed how supermarkets with differing behaviors affect a heterogeneous crowd of consumer agents. The model was implemented in Java with Python used to evaluate the results. 

The simulation displays good acceptance with real grocery market behavior, i.e. captures the performance of British retailers thus can be used to determine the impact of changes in their behavior on their competitors and consumers.Furthermore it can be used to provide insight into sustainability of volatile pricing strategies, providing a useful insight in volatility of British supermarket retail industry. 
\end{abstract}
\acknowledgements{
I would like to express my sincere gratitude to Dr Maria Polukarov for her guidance and support which provided me the freedom to take this research in the direction of my interest.\\
\\
I would also like to thank my family and friends for their encouragement and support. To those who quietly listened to my software complaints. To those who worked throughout the nights with me. To those who helped me write what I couldn't say. I cannot thank you enough.
}

\declaration{
I, Stefan Collier, declare that this dissertation and the work presented in it are my own and has been generated by me as the result of my own original research.\\
I confirm that:\\
1. This work was done wholly or mainly while in candidature for a degree at this University;\\
2. Where any part of this dissertation has previously been submitted for any other qualification at this University or any other institution, this has been clearly stated;\\
3. Where I have consulted the published work of others, this is always clearly attributed;\\
4. Where I have quoted from the work of others, the source is always given. With the exception of such quotations, this dissertation is entirely my own work;\\
5. I have acknowledged all main sources of help;\\
6. Where the thesis is based on work done by myself jointly with others, I have made clear exactly what was done by others and what I have contributed myself;\\
7. Either none of this work has been published before submission, or parts of this work have been published by :\\
\\
Stefan Collier\\
April 2016
}
\tableofcontents
\listoffigures
\listoftables

\mainmatter
%% ----------------------------------------------------------------
%\include{Introduction}
%\include{Conclusions}
\include{chapters/1Project/main}
\include{chapters/2Lit/main}
\include{chapters/3Design/HighLevel}
\include{chapters/3Design/InDepth}
\include{chapters/4Impl/main}

\include{chapters/5Experiments/1/main}
\include{chapters/5Experiments/2/main}
\include{chapters/5Experiments/3/main}
\include{chapters/5Experiments/4/main}

\include{chapters/6Conclusion/main}

\appendix
\include{appendix/AppendixB}
\include{appendix/D/main}
\include{appendix/AppendixC}

\backmatter
\bibliographystyle{ecs}
\bibliography{ECS}
\end{document}
%% ----------------------------------------------------------------

 %% ----------------------------------------------------------------
%% Progress.tex
%% ---------------------------------------------------------------- 
\documentclass{ecsprogress}    % Use the progress Style
\graphicspath{{../figs/}}   % Location of your graphics files
    \usepackage{natbib}            % Use Natbib style for the refs.
\hypersetup{colorlinks=true}   % Set to false for black/white printing
\input{Definitions}            % Include your abbreviations



\usepackage{enumitem}% http://ctan.org/pkg/enumitem
\usepackage{multirow}
\usepackage{float}
\usepackage{amsmath}
\usepackage{multicol}
\usepackage{amssymb}
\usepackage[normalem]{ulem}
\useunder{\uline}{\ul}{}
\usepackage{wrapfig}


\usepackage[table,xcdraw]{xcolor}


%% ----------------------------------------------------------------
\begin{document}
\frontmatter
\title      {Heterogeneous Agent-based Model for Supermarket Competition}
\authors    {\texorpdfstring
             {\href{mailto:sc22g13@ecs.soton.ac.uk}{Stefan J. Collier}}
             {Stefan J. Collier}
            }
\addresses  {\groupname\\\deptname\\\univname}
\date       {\today}
\subject    {}
\keywords   {}
\supervisor {Dr. Maria Polukarov}
\examiner   {Professor Sheng Chen}

\maketitle
\begin{abstract}
This project aim was to model and analyse the effects of competitive pricing behaviors of grocery retailers on the British market. 

This was achieved by creating a multi-agent model, containing retailer and consumer agents. The heterogeneous crowd of retailers employs either a uniform pricing strategy or a ‘local price flexing’ strategy. The actions of these retailers are chosen by predicting the profit of each action, using a perceptron. Following on from the consideration of different economic models, a discrete model was developed so that software agents have a discrete environment to operate within. Within the model, it has been observed how supermarkets with differing behaviors affect a heterogeneous crowd of consumer agents. The model was implemented in Java with Python used to evaluate the results. 

The simulation displays good acceptance with real grocery market behavior, i.e. captures the performance of British retailers thus can be used to determine the impact of changes in their behavior on their competitors and consumers.Furthermore it can be used to provide insight into sustainability of volatile pricing strategies, providing a useful insight in volatility of British supermarket retail industry. 
\end{abstract}
\acknowledgements{
I would like to express my sincere gratitude to Dr Maria Polukarov for her guidance and support which provided me the freedom to take this research in the direction of my interest.\\
\\
I would also like to thank my family and friends for their encouragement and support. To those who quietly listened to my software complaints. To those who worked throughout the nights with me. To those who helped me write what I couldn't say. I cannot thank you enough.
}

\declaration{
I, Stefan Collier, declare that this dissertation and the work presented in it are my own and has been generated by me as the result of my own original research.\\
I confirm that:\\
1. This work was done wholly or mainly while in candidature for a degree at this University;\\
2. Where any part of this dissertation has previously been submitted for any other qualification at this University or any other institution, this has been clearly stated;\\
3. Where I have consulted the published work of others, this is always clearly attributed;\\
4. Where I have quoted from the work of others, the source is always given. With the exception of such quotations, this dissertation is entirely my own work;\\
5. I have acknowledged all main sources of help;\\
6. Where the thesis is based on work done by myself jointly with others, I have made clear exactly what was done by others and what I have contributed myself;\\
7. Either none of this work has been published before submission, or parts of this work have been published by :\\
\\
Stefan Collier\\
April 2016
}
\tableofcontents
\listoffigures
\listoftables

\mainmatter
%% ----------------------------------------------------------------
%\include{Introduction}
%\include{Conclusions}
\include{chapters/1Project/main}
\include{chapters/2Lit/main}
\include{chapters/3Design/HighLevel}
\include{chapters/3Design/InDepth}
\include{chapters/4Impl/main}

\include{chapters/5Experiments/1/main}
\include{chapters/5Experiments/2/main}
\include{chapters/5Experiments/3/main}
\include{chapters/5Experiments/4/main}

\include{chapters/6Conclusion/main}

\appendix
\include{appendix/AppendixB}
\include{appendix/D/main}
\include{appendix/AppendixC}

\backmatter
\bibliographystyle{ecs}
\bibliography{ECS}
\end{document}
%% ----------------------------------------------------------------

 %% ----------------------------------------------------------------
%% Progress.tex
%% ---------------------------------------------------------------- 
\documentclass{ecsprogress}    % Use the progress Style
\graphicspath{{../figs/}}   % Location of your graphics files
    \usepackage{natbib}            % Use Natbib style for the refs.
\hypersetup{colorlinks=true}   % Set to false for black/white printing
\input{Definitions}            % Include your abbreviations



\usepackage{enumitem}% http://ctan.org/pkg/enumitem
\usepackage{multirow}
\usepackage{float}
\usepackage{amsmath}
\usepackage{multicol}
\usepackage{amssymb}
\usepackage[normalem]{ulem}
\useunder{\uline}{\ul}{}
\usepackage{wrapfig}


\usepackage[table,xcdraw]{xcolor}


%% ----------------------------------------------------------------
\begin{document}
\frontmatter
\title      {Heterogeneous Agent-based Model for Supermarket Competition}
\authors    {\texorpdfstring
             {\href{mailto:sc22g13@ecs.soton.ac.uk}{Stefan J. Collier}}
             {Stefan J. Collier}
            }
\addresses  {\groupname\\\deptname\\\univname}
\date       {\today}
\subject    {}
\keywords   {}
\supervisor {Dr. Maria Polukarov}
\examiner   {Professor Sheng Chen}

\maketitle
\begin{abstract}
This project aim was to model and analyse the effects of competitive pricing behaviors of grocery retailers on the British market. 

This was achieved by creating a multi-agent model, containing retailer and consumer agents. The heterogeneous crowd of retailers employs either a uniform pricing strategy or a ‘local price flexing’ strategy. The actions of these retailers are chosen by predicting the profit of each action, using a perceptron. Following on from the consideration of different economic models, a discrete model was developed so that software agents have a discrete environment to operate within. Within the model, it has been observed how supermarkets with differing behaviors affect a heterogeneous crowd of consumer agents. The model was implemented in Java with Python used to evaluate the results. 

The simulation displays good acceptance with real grocery market behavior, i.e. captures the performance of British retailers thus can be used to determine the impact of changes in their behavior on their competitors and consumers.Furthermore it can be used to provide insight into sustainability of volatile pricing strategies, providing a useful insight in volatility of British supermarket retail industry. 
\end{abstract}
\acknowledgements{
I would like to express my sincere gratitude to Dr Maria Polukarov for her guidance and support which provided me the freedom to take this research in the direction of my interest.\\
\\
I would also like to thank my family and friends for their encouragement and support. To those who quietly listened to my software complaints. To those who worked throughout the nights with me. To those who helped me write what I couldn't say. I cannot thank you enough.
}

\declaration{
I, Stefan Collier, declare that this dissertation and the work presented in it are my own and has been generated by me as the result of my own original research.\\
I confirm that:\\
1. This work was done wholly or mainly while in candidature for a degree at this University;\\
2. Where any part of this dissertation has previously been submitted for any other qualification at this University or any other institution, this has been clearly stated;\\
3. Where I have consulted the published work of others, this is always clearly attributed;\\
4. Where I have quoted from the work of others, the source is always given. With the exception of such quotations, this dissertation is entirely my own work;\\
5. I have acknowledged all main sources of help;\\
6. Where the thesis is based on work done by myself jointly with others, I have made clear exactly what was done by others and what I have contributed myself;\\
7. Either none of this work has been published before submission, or parts of this work have been published by :\\
\\
Stefan Collier\\
April 2016
}
\tableofcontents
\listoffigures
\listoftables

\mainmatter
%% ----------------------------------------------------------------
%\include{Introduction}
%\include{Conclusions}
\include{chapters/1Project/main}
\include{chapters/2Lit/main}
\include{chapters/3Design/HighLevel}
\include{chapters/3Design/InDepth}
\include{chapters/4Impl/main}

\include{chapters/5Experiments/1/main}
\include{chapters/5Experiments/2/main}
\include{chapters/5Experiments/3/main}
\include{chapters/5Experiments/4/main}

\include{chapters/6Conclusion/main}

\appendix
\include{appendix/AppendixB}
\include{appendix/D/main}
\include{appendix/AppendixC}

\backmatter
\bibliographystyle{ecs}
\bibliography{ECS}
\end{document}
%% ----------------------------------------------------------------

 %% ----------------------------------------------------------------
%% Progress.tex
%% ---------------------------------------------------------------- 
\documentclass{ecsprogress}    % Use the progress Style
\graphicspath{{../figs/}}   % Location of your graphics files
    \usepackage{natbib}            % Use Natbib style for the refs.
\hypersetup{colorlinks=true}   % Set to false for black/white printing
\input{Definitions}            % Include your abbreviations



\usepackage{enumitem}% http://ctan.org/pkg/enumitem
\usepackage{multirow}
\usepackage{float}
\usepackage{amsmath}
\usepackage{multicol}
\usepackage{amssymb}
\usepackage[normalem]{ulem}
\useunder{\uline}{\ul}{}
\usepackage{wrapfig}


\usepackage[table,xcdraw]{xcolor}


%% ----------------------------------------------------------------
\begin{document}
\frontmatter
\title      {Heterogeneous Agent-based Model for Supermarket Competition}
\authors    {\texorpdfstring
             {\href{mailto:sc22g13@ecs.soton.ac.uk}{Stefan J. Collier}}
             {Stefan J. Collier}
            }
\addresses  {\groupname\\\deptname\\\univname}
\date       {\today}
\subject    {}
\keywords   {}
\supervisor {Dr. Maria Polukarov}
\examiner   {Professor Sheng Chen}

\maketitle
\begin{abstract}
This project aim was to model and analyse the effects of competitive pricing behaviors of grocery retailers on the British market. 

This was achieved by creating a multi-agent model, containing retailer and consumer agents. The heterogeneous crowd of retailers employs either a uniform pricing strategy or a ‘local price flexing’ strategy. The actions of these retailers are chosen by predicting the profit of each action, using a perceptron. Following on from the consideration of different economic models, a discrete model was developed so that software agents have a discrete environment to operate within. Within the model, it has been observed how supermarkets with differing behaviors affect a heterogeneous crowd of consumer agents. The model was implemented in Java with Python used to evaluate the results. 

The simulation displays good acceptance with real grocery market behavior, i.e. captures the performance of British retailers thus can be used to determine the impact of changes in their behavior on their competitors and consumers.Furthermore it can be used to provide insight into sustainability of volatile pricing strategies, providing a useful insight in volatility of British supermarket retail industry. 
\end{abstract}
\acknowledgements{
I would like to express my sincere gratitude to Dr Maria Polukarov for her guidance and support which provided me the freedom to take this research in the direction of my interest.\\
\\
I would also like to thank my family and friends for their encouragement and support. To those who quietly listened to my software complaints. To those who worked throughout the nights with me. To those who helped me write what I couldn't say. I cannot thank you enough.
}

\declaration{
I, Stefan Collier, declare that this dissertation and the work presented in it are my own and has been generated by me as the result of my own original research.\\
I confirm that:\\
1. This work was done wholly or mainly while in candidature for a degree at this University;\\
2. Where any part of this dissertation has previously been submitted for any other qualification at this University or any other institution, this has been clearly stated;\\
3. Where I have consulted the published work of others, this is always clearly attributed;\\
4. Where I have quoted from the work of others, the source is always given. With the exception of such quotations, this dissertation is entirely my own work;\\
5. I have acknowledged all main sources of help;\\
6. Where the thesis is based on work done by myself jointly with others, I have made clear exactly what was done by others and what I have contributed myself;\\
7. Either none of this work has been published before submission, or parts of this work have been published by :\\
\\
Stefan Collier\\
April 2016
}
\tableofcontents
\listoffigures
\listoftables

\mainmatter
%% ----------------------------------------------------------------
%\include{Introduction}
%\include{Conclusions}
\include{chapters/1Project/main}
\include{chapters/2Lit/main}
\include{chapters/3Design/HighLevel}
\include{chapters/3Design/InDepth}
\include{chapters/4Impl/main}

\include{chapters/5Experiments/1/main}
\include{chapters/5Experiments/2/main}
\include{chapters/5Experiments/3/main}
\include{chapters/5Experiments/4/main}

\include{chapters/6Conclusion/main}

\appendix
\include{appendix/AppendixB}
\include{appendix/D/main}
\include{appendix/AppendixC}

\backmatter
\bibliographystyle{ecs}
\bibliography{ECS}
\end{document}
%% ----------------------------------------------------------------


 %% ----------------------------------------------------------------
%% Progress.tex
%% ---------------------------------------------------------------- 
\documentclass{ecsprogress}    % Use the progress Style
\graphicspath{{../figs/}}   % Location of your graphics files
    \usepackage{natbib}            % Use Natbib style for the refs.
\hypersetup{colorlinks=true}   % Set to false for black/white printing
\input{Definitions}            % Include your abbreviations



\usepackage{enumitem}% http://ctan.org/pkg/enumitem
\usepackage{multirow}
\usepackage{float}
\usepackage{amsmath}
\usepackage{multicol}
\usepackage{amssymb}
\usepackage[normalem]{ulem}
\useunder{\uline}{\ul}{}
\usepackage{wrapfig}


\usepackage[table,xcdraw]{xcolor}


%% ----------------------------------------------------------------
\begin{document}
\frontmatter
\title      {Heterogeneous Agent-based Model for Supermarket Competition}
\authors    {\texorpdfstring
             {\href{mailto:sc22g13@ecs.soton.ac.uk}{Stefan J. Collier}}
             {Stefan J. Collier}
            }
\addresses  {\groupname\\\deptname\\\univname}
\date       {\today}
\subject    {}
\keywords   {}
\supervisor {Dr. Maria Polukarov}
\examiner   {Professor Sheng Chen}

\maketitle
\begin{abstract}
This project aim was to model and analyse the effects of competitive pricing behaviors of grocery retailers on the British market. 

This was achieved by creating a multi-agent model, containing retailer and consumer agents. The heterogeneous crowd of retailers employs either a uniform pricing strategy or a ‘local price flexing’ strategy. The actions of these retailers are chosen by predicting the profit of each action, using a perceptron. Following on from the consideration of different economic models, a discrete model was developed so that software agents have a discrete environment to operate within. Within the model, it has been observed how supermarkets with differing behaviors affect a heterogeneous crowd of consumer agents. The model was implemented in Java with Python used to evaluate the results. 

The simulation displays good acceptance with real grocery market behavior, i.e. captures the performance of British retailers thus can be used to determine the impact of changes in their behavior on their competitors and consumers.Furthermore it can be used to provide insight into sustainability of volatile pricing strategies, providing a useful insight in volatility of British supermarket retail industry. 
\end{abstract}
\acknowledgements{
I would like to express my sincere gratitude to Dr Maria Polukarov for her guidance and support which provided me the freedom to take this research in the direction of my interest.\\
\\
I would also like to thank my family and friends for their encouragement and support. To those who quietly listened to my software complaints. To those who worked throughout the nights with me. To those who helped me write what I couldn't say. I cannot thank you enough.
}

\declaration{
I, Stefan Collier, declare that this dissertation and the work presented in it are my own and has been generated by me as the result of my own original research.\\
I confirm that:\\
1. This work was done wholly or mainly while in candidature for a degree at this University;\\
2. Where any part of this dissertation has previously been submitted for any other qualification at this University or any other institution, this has been clearly stated;\\
3. Where I have consulted the published work of others, this is always clearly attributed;\\
4. Where I have quoted from the work of others, the source is always given. With the exception of such quotations, this dissertation is entirely my own work;\\
5. I have acknowledged all main sources of help;\\
6. Where the thesis is based on work done by myself jointly with others, I have made clear exactly what was done by others and what I have contributed myself;\\
7. Either none of this work has been published before submission, or parts of this work have been published by :\\
\\
Stefan Collier\\
April 2016
}
\tableofcontents
\listoffigures
\listoftables

\mainmatter
%% ----------------------------------------------------------------
%\include{Introduction}
%\include{Conclusions}
\include{chapters/1Project/main}
\include{chapters/2Lit/main}
\include{chapters/3Design/HighLevel}
\include{chapters/3Design/InDepth}
\include{chapters/4Impl/main}

\include{chapters/5Experiments/1/main}
\include{chapters/5Experiments/2/main}
\include{chapters/5Experiments/3/main}
\include{chapters/5Experiments/4/main}

\include{chapters/6Conclusion/main}

\appendix
\include{appendix/AppendixB}
\include{appendix/D/main}
\include{appendix/AppendixC}

\backmatter
\bibliographystyle{ecs}
\bibliography{ECS}
\end{document}
%% ----------------------------------------------------------------


\appendix
\include{appendix/AppendixB}
 %% ----------------------------------------------------------------
%% Progress.tex
%% ---------------------------------------------------------------- 
\documentclass{ecsprogress}    % Use the progress Style
\graphicspath{{../figs/}}   % Location of your graphics files
    \usepackage{natbib}            % Use Natbib style for the refs.
\hypersetup{colorlinks=true}   % Set to false for black/white printing
\input{Definitions}            % Include your abbreviations



\usepackage{enumitem}% http://ctan.org/pkg/enumitem
\usepackage{multirow}
\usepackage{float}
\usepackage{amsmath}
\usepackage{multicol}
\usepackage{amssymb}
\usepackage[normalem]{ulem}
\useunder{\uline}{\ul}{}
\usepackage{wrapfig}


\usepackage[table,xcdraw]{xcolor}


%% ----------------------------------------------------------------
\begin{document}
\frontmatter
\title      {Heterogeneous Agent-based Model for Supermarket Competition}
\authors    {\texorpdfstring
             {\href{mailto:sc22g13@ecs.soton.ac.uk}{Stefan J. Collier}}
             {Stefan J. Collier}
            }
\addresses  {\groupname\\\deptname\\\univname}
\date       {\today}
\subject    {}
\keywords   {}
\supervisor {Dr. Maria Polukarov}
\examiner   {Professor Sheng Chen}

\maketitle
\begin{abstract}
This project aim was to model and analyse the effects of competitive pricing behaviors of grocery retailers on the British market. 

This was achieved by creating a multi-agent model, containing retailer and consumer agents. The heterogeneous crowd of retailers employs either a uniform pricing strategy or a ‘local price flexing’ strategy. The actions of these retailers are chosen by predicting the profit of each action, using a perceptron. Following on from the consideration of different economic models, a discrete model was developed so that software agents have a discrete environment to operate within. Within the model, it has been observed how supermarkets with differing behaviors affect a heterogeneous crowd of consumer agents. The model was implemented in Java with Python used to evaluate the results. 

The simulation displays good acceptance with real grocery market behavior, i.e. captures the performance of British retailers thus can be used to determine the impact of changes in their behavior on their competitors and consumers.Furthermore it can be used to provide insight into sustainability of volatile pricing strategies, providing a useful insight in volatility of British supermarket retail industry. 
\end{abstract}
\acknowledgements{
I would like to express my sincere gratitude to Dr Maria Polukarov for her guidance and support which provided me the freedom to take this research in the direction of my interest.\\
\\
I would also like to thank my family and friends for their encouragement and support. To those who quietly listened to my software complaints. To those who worked throughout the nights with me. To those who helped me write what I couldn't say. I cannot thank you enough.
}

\declaration{
I, Stefan Collier, declare that this dissertation and the work presented in it are my own and has been generated by me as the result of my own original research.\\
I confirm that:\\
1. This work was done wholly or mainly while in candidature for a degree at this University;\\
2. Where any part of this dissertation has previously been submitted for any other qualification at this University or any other institution, this has been clearly stated;\\
3. Where I have consulted the published work of others, this is always clearly attributed;\\
4. Where I have quoted from the work of others, the source is always given. With the exception of such quotations, this dissertation is entirely my own work;\\
5. I have acknowledged all main sources of help;\\
6. Where the thesis is based on work done by myself jointly with others, I have made clear exactly what was done by others and what I have contributed myself;\\
7. Either none of this work has been published before submission, or parts of this work have been published by :\\
\\
Stefan Collier\\
April 2016
}
\tableofcontents
\listoffigures
\listoftables

\mainmatter
%% ----------------------------------------------------------------
%\include{Introduction}
%\include{Conclusions}
\include{chapters/1Project/main}
\include{chapters/2Lit/main}
\include{chapters/3Design/HighLevel}
\include{chapters/3Design/InDepth}
\include{chapters/4Impl/main}

\include{chapters/5Experiments/1/main}
\include{chapters/5Experiments/2/main}
\include{chapters/5Experiments/3/main}
\include{chapters/5Experiments/4/main}

\include{chapters/6Conclusion/main}

\appendix
\include{appendix/AppendixB}
\include{appendix/D/main}
\include{appendix/AppendixC}

\backmatter
\bibliographystyle{ecs}
\bibliography{ECS}
\end{document}
%% ----------------------------------------------------------------

\include{appendix/AppendixC}

\backmatter
\bibliographystyle{ecs}
\bibliography{ECS}
\end{document}
%% ----------------------------------------------------------------


 %% ----------------------------------------------------------------
%% Progress.tex
%% ---------------------------------------------------------------- 
\documentclass{ecsprogress}    % Use the progress Style
\graphicspath{{../figs/}}   % Location of your graphics files
    \usepackage{natbib}            % Use Natbib style for the refs.
\hypersetup{colorlinks=true}   % Set to false for black/white printing
\input{Definitions}            % Include your abbreviations



\usepackage{enumitem}% http://ctan.org/pkg/enumitem
\usepackage{multirow}
\usepackage{float}
\usepackage{amsmath}
\usepackage{multicol}
\usepackage{amssymb}
\usepackage[normalem]{ulem}
\useunder{\uline}{\ul}{}
\usepackage{wrapfig}


\usepackage[table,xcdraw]{xcolor}


%% ----------------------------------------------------------------
\begin{document}
\frontmatter
\title      {Heterogeneous Agent-based Model for Supermarket Competition}
\authors    {\texorpdfstring
             {\href{mailto:sc22g13@ecs.soton.ac.uk}{Stefan J. Collier}}
             {Stefan J. Collier}
            }
\addresses  {\groupname\\\deptname\\\univname}
\date       {\today}
\subject    {}
\keywords   {}
\supervisor {Dr. Maria Polukarov}
\examiner   {Professor Sheng Chen}

\maketitle
\begin{abstract}
This project aim was to model and analyse the effects of competitive pricing behaviors of grocery retailers on the British market. 

This was achieved by creating a multi-agent model, containing retailer and consumer agents. The heterogeneous crowd of retailers employs either a uniform pricing strategy or a ‘local price flexing’ strategy. The actions of these retailers are chosen by predicting the profit of each action, using a perceptron. Following on from the consideration of different economic models, a discrete model was developed so that software agents have a discrete environment to operate within. Within the model, it has been observed how supermarkets with differing behaviors affect a heterogeneous crowd of consumer agents. The model was implemented in Java with Python used to evaluate the results. 

The simulation displays good acceptance with real grocery market behavior, i.e. captures the performance of British retailers thus can be used to determine the impact of changes in their behavior on their competitors and consumers.Furthermore it can be used to provide insight into sustainability of volatile pricing strategies, providing a useful insight in volatility of British supermarket retail industry. 
\end{abstract}
\acknowledgements{
I would like to express my sincere gratitude to Dr Maria Polukarov for her guidance and support which provided me the freedom to take this research in the direction of my interest.\\
\\
I would also like to thank my family and friends for their encouragement and support. To those who quietly listened to my software complaints. To those who worked throughout the nights with me. To those who helped me write what I couldn't say. I cannot thank you enough.
}

\declaration{
I, Stefan Collier, declare that this dissertation and the work presented in it are my own and has been generated by me as the result of my own original research.\\
I confirm that:\\
1. This work was done wholly or mainly while in candidature for a degree at this University;\\
2. Where any part of this dissertation has previously been submitted for any other qualification at this University or any other institution, this has been clearly stated;\\
3. Where I have consulted the published work of others, this is always clearly attributed;\\
4. Where I have quoted from the work of others, the source is always given. With the exception of such quotations, this dissertation is entirely my own work;\\
5. I have acknowledged all main sources of help;\\
6. Where the thesis is based on work done by myself jointly with others, I have made clear exactly what was done by others and what I have contributed myself;\\
7. Either none of this work has been published before submission, or parts of this work have been published by :\\
\\
Stefan Collier\\
April 2016
}
\tableofcontents
\listoffigures
\listoftables

\mainmatter
%% ----------------------------------------------------------------
%\include{Introduction}
%\include{Conclusions}
 %% ----------------------------------------------------------------
%% Progress.tex
%% ---------------------------------------------------------------- 
\documentclass{ecsprogress}    % Use the progress Style
\graphicspath{{../figs/}}   % Location of your graphics files
    \usepackage{natbib}            % Use Natbib style for the refs.
\hypersetup{colorlinks=true}   % Set to false for black/white printing
\input{Definitions}            % Include your abbreviations



\usepackage{enumitem}% http://ctan.org/pkg/enumitem
\usepackage{multirow}
\usepackage{float}
\usepackage{amsmath}
\usepackage{multicol}
\usepackage{amssymb}
\usepackage[normalem]{ulem}
\useunder{\uline}{\ul}{}
\usepackage{wrapfig}


\usepackage[table,xcdraw]{xcolor}


%% ----------------------------------------------------------------
\begin{document}
\frontmatter
\title      {Heterogeneous Agent-based Model for Supermarket Competition}
\authors    {\texorpdfstring
             {\href{mailto:sc22g13@ecs.soton.ac.uk}{Stefan J. Collier}}
             {Stefan J. Collier}
            }
\addresses  {\groupname\\\deptname\\\univname}
\date       {\today}
\subject    {}
\keywords   {}
\supervisor {Dr. Maria Polukarov}
\examiner   {Professor Sheng Chen}

\maketitle
\begin{abstract}
This project aim was to model and analyse the effects of competitive pricing behaviors of grocery retailers on the British market. 

This was achieved by creating a multi-agent model, containing retailer and consumer agents. The heterogeneous crowd of retailers employs either a uniform pricing strategy or a ‘local price flexing’ strategy. The actions of these retailers are chosen by predicting the profit of each action, using a perceptron. Following on from the consideration of different economic models, a discrete model was developed so that software agents have a discrete environment to operate within. Within the model, it has been observed how supermarkets with differing behaviors affect a heterogeneous crowd of consumer agents. The model was implemented in Java with Python used to evaluate the results. 

The simulation displays good acceptance with real grocery market behavior, i.e. captures the performance of British retailers thus can be used to determine the impact of changes in their behavior on their competitors and consumers.Furthermore it can be used to provide insight into sustainability of volatile pricing strategies, providing a useful insight in volatility of British supermarket retail industry. 
\end{abstract}
\acknowledgements{
I would like to express my sincere gratitude to Dr Maria Polukarov for her guidance and support which provided me the freedom to take this research in the direction of my interest.\\
\\
I would also like to thank my family and friends for their encouragement and support. To those who quietly listened to my software complaints. To those who worked throughout the nights with me. To those who helped me write what I couldn't say. I cannot thank you enough.
}

\declaration{
I, Stefan Collier, declare that this dissertation and the work presented in it are my own and has been generated by me as the result of my own original research.\\
I confirm that:\\
1. This work was done wholly or mainly while in candidature for a degree at this University;\\
2. Where any part of this dissertation has previously been submitted for any other qualification at this University or any other institution, this has been clearly stated;\\
3. Where I have consulted the published work of others, this is always clearly attributed;\\
4. Where I have quoted from the work of others, the source is always given. With the exception of such quotations, this dissertation is entirely my own work;\\
5. I have acknowledged all main sources of help;\\
6. Where the thesis is based on work done by myself jointly with others, I have made clear exactly what was done by others and what I have contributed myself;\\
7. Either none of this work has been published before submission, or parts of this work have been published by :\\
\\
Stefan Collier\\
April 2016
}
\tableofcontents
\listoffigures
\listoftables

\mainmatter
%% ----------------------------------------------------------------
%\include{Introduction}
%\include{Conclusions}
\include{chapters/1Project/main}
\include{chapters/2Lit/main}
\include{chapters/3Design/HighLevel}
\include{chapters/3Design/InDepth}
\include{chapters/4Impl/main}

\include{chapters/5Experiments/1/main}
\include{chapters/5Experiments/2/main}
\include{chapters/5Experiments/3/main}
\include{chapters/5Experiments/4/main}

\include{chapters/6Conclusion/main}

\appendix
\include{appendix/AppendixB}
\include{appendix/D/main}
\include{appendix/AppendixC}

\backmatter
\bibliographystyle{ecs}
\bibliography{ECS}
\end{document}
%% ----------------------------------------------------------------

 %% ----------------------------------------------------------------
%% Progress.tex
%% ---------------------------------------------------------------- 
\documentclass{ecsprogress}    % Use the progress Style
\graphicspath{{../figs/}}   % Location of your graphics files
    \usepackage{natbib}            % Use Natbib style for the refs.
\hypersetup{colorlinks=true}   % Set to false for black/white printing
\input{Definitions}            % Include your abbreviations



\usepackage{enumitem}% http://ctan.org/pkg/enumitem
\usepackage{multirow}
\usepackage{float}
\usepackage{amsmath}
\usepackage{multicol}
\usepackage{amssymb}
\usepackage[normalem]{ulem}
\useunder{\uline}{\ul}{}
\usepackage{wrapfig}


\usepackage[table,xcdraw]{xcolor}


%% ----------------------------------------------------------------
\begin{document}
\frontmatter
\title      {Heterogeneous Agent-based Model for Supermarket Competition}
\authors    {\texorpdfstring
             {\href{mailto:sc22g13@ecs.soton.ac.uk}{Stefan J. Collier}}
             {Stefan J. Collier}
            }
\addresses  {\groupname\\\deptname\\\univname}
\date       {\today}
\subject    {}
\keywords   {}
\supervisor {Dr. Maria Polukarov}
\examiner   {Professor Sheng Chen}

\maketitle
\begin{abstract}
This project aim was to model and analyse the effects of competitive pricing behaviors of grocery retailers on the British market. 

This was achieved by creating a multi-agent model, containing retailer and consumer agents. The heterogeneous crowd of retailers employs either a uniform pricing strategy or a ‘local price flexing’ strategy. The actions of these retailers are chosen by predicting the profit of each action, using a perceptron. Following on from the consideration of different economic models, a discrete model was developed so that software agents have a discrete environment to operate within. Within the model, it has been observed how supermarkets with differing behaviors affect a heterogeneous crowd of consumer agents. The model was implemented in Java with Python used to evaluate the results. 

The simulation displays good acceptance with real grocery market behavior, i.e. captures the performance of British retailers thus can be used to determine the impact of changes in their behavior on their competitors and consumers.Furthermore it can be used to provide insight into sustainability of volatile pricing strategies, providing a useful insight in volatility of British supermarket retail industry. 
\end{abstract}
\acknowledgements{
I would like to express my sincere gratitude to Dr Maria Polukarov for her guidance and support which provided me the freedom to take this research in the direction of my interest.\\
\\
I would also like to thank my family and friends for their encouragement and support. To those who quietly listened to my software complaints. To those who worked throughout the nights with me. To those who helped me write what I couldn't say. I cannot thank you enough.
}

\declaration{
I, Stefan Collier, declare that this dissertation and the work presented in it are my own and has been generated by me as the result of my own original research.\\
I confirm that:\\
1. This work was done wholly or mainly while in candidature for a degree at this University;\\
2. Where any part of this dissertation has previously been submitted for any other qualification at this University or any other institution, this has been clearly stated;\\
3. Where I have consulted the published work of others, this is always clearly attributed;\\
4. Where I have quoted from the work of others, the source is always given. With the exception of such quotations, this dissertation is entirely my own work;\\
5. I have acknowledged all main sources of help;\\
6. Where the thesis is based on work done by myself jointly with others, I have made clear exactly what was done by others and what I have contributed myself;\\
7. Either none of this work has been published before submission, or parts of this work have been published by :\\
\\
Stefan Collier\\
April 2016
}
\tableofcontents
\listoffigures
\listoftables

\mainmatter
%% ----------------------------------------------------------------
%\include{Introduction}
%\include{Conclusions}
\include{chapters/1Project/main}
\include{chapters/2Lit/main}
\include{chapters/3Design/HighLevel}
\include{chapters/3Design/InDepth}
\include{chapters/4Impl/main}

\include{chapters/5Experiments/1/main}
\include{chapters/5Experiments/2/main}
\include{chapters/5Experiments/3/main}
\include{chapters/5Experiments/4/main}

\include{chapters/6Conclusion/main}

\appendix
\include{appendix/AppendixB}
\include{appendix/D/main}
\include{appendix/AppendixC}

\backmatter
\bibliographystyle{ecs}
\bibliography{ECS}
\end{document}
%% ----------------------------------------------------------------

\include{chapters/3Design/HighLevel}
\include{chapters/3Design/InDepth}
 %% ----------------------------------------------------------------
%% Progress.tex
%% ---------------------------------------------------------------- 
\documentclass{ecsprogress}    % Use the progress Style
\graphicspath{{../figs/}}   % Location of your graphics files
    \usepackage{natbib}            % Use Natbib style for the refs.
\hypersetup{colorlinks=true}   % Set to false for black/white printing
\input{Definitions}            % Include your abbreviations



\usepackage{enumitem}% http://ctan.org/pkg/enumitem
\usepackage{multirow}
\usepackage{float}
\usepackage{amsmath}
\usepackage{multicol}
\usepackage{amssymb}
\usepackage[normalem]{ulem}
\useunder{\uline}{\ul}{}
\usepackage{wrapfig}


\usepackage[table,xcdraw]{xcolor}


%% ----------------------------------------------------------------
\begin{document}
\frontmatter
\title      {Heterogeneous Agent-based Model for Supermarket Competition}
\authors    {\texorpdfstring
             {\href{mailto:sc22g13@ecs.soton.ac.uk}{Stefan J. Collier}}
             {Stefan J. Collier}
            }
\addresses  {\groupname\\\deptname\\\univname}
\date       {\today}
\subject    {}
\keywords   {}
\supervisor {Dr. Maria Polukarov}
\examiner   {Professor Sheng Chen}

\maketitle
\begin{abstract}
This project aim was to model and analyse the effects of competitive pricing behaviors of grocery retailers on the British market. 

This was achieved by creating a multi-agent model, containing retailer and consumer agents. The heterogeneous crowd of retailers employs either a uniform pricing strategy or a ‘local price flexing’ strategy. The actions of these retailers are chosen by predicting the profit of each action, using a perceptron. Following on from the consideration of different economic models, a discrete model was developed so that software agents have a discrete environment to operate within. Within the model, it has been observed how supermarkets with differing behaviors affect a heterogeneous crowd of consumer agents. The model was implemented in Java with Python used to evaluate the results. 

The simulation displays good acceptance with real grocery market behavior, i.e. captures the performance of British retailers thus can be used to determine the impact of changes in their behavior on their competitors and consumers.Furthermore it can be used to provide insight into sustainability of volatile pricing strategies, providing a useful insight in volatility of British supermarket retail industry. 
\end{abstract}
\acknowledgements{
I would like to express my sincere gratitude to Dr Maria Polukarov for her guidance and support which provided me the freedom to take this research in the direction of my interest.\\
\\
I would also like to thank my family and friends for their encouragement and support. To those who quietly listened to my software complaints. To those who worked throughout the nights with me. To those who helped me write what I couldn't say. I cannot thank you enough.
}

\declaration{
I, Stefan Collier, declare that this dissertation and the work presented in it are my own and has been generated by me as the result of my own original research.\\
I confirm that:\\
1. This work was done wholly or mainly while in candidature for a degree at this University;\\
2. Where any part of this dissertation has previously been submitted for any other qualification at this University or any other institution, this has been clearly stated;\\
3. Where I have consulted the published work of others, this is always clearly attributed;\\
4. Where I have quoted from the work of others, the source is always given. With the exception of such quotations, this dissertation is entirely my own work;\\
5. I have acknowledged all main sources of help;\\
6. Where the thesis is based on work done by myself jointly with others, I have made clear exactly what was done by others and what I have contributed myself;\\
7. Either none of this work has been published before submission, or parts of this work have been published by :\\
\\
Stefan Collier\\
April 2016
}
\tableofcontents
\listoffigures
\listoftables

\mainmatter
%% ----------------------------------------------------------------
%\include{Introduction}
%\include{Conclusions}
\include{chapters/1Project/main}
\include{chapters/2Lit/main}
\include{chapters/3Design/HighLevel}
\include{chapters/3Design/InDepth}
\include{chapters/4Impl/main}

\include{chapters/5Experiments/1/main}
\include{chapters/5Experiments/2/main}
\include{chapters/5Experiments/3/main}
\include{chapters/5Experiments/4/main}

\include{chapters/6Conclusion/main}

\appendix
\include{appendix/AppendixB}
\include{appendix/D/main}
\include{appendix/AppendixC}

\backmatter
\bibliographystyle{ecs}
\bibliography{ECS}
\end{document}
%% ----------------------------------------------------------------


 %% ----------------------------------------------------------------
%% Progress.tex
%% ---------------------------------------------------------------- 
\documentclass{ecsprogress}    % Use the progress Style
\graphicspath{{../figs/}}   % Location of your graphics files
    \usepackage{natbib}            % Use Natbib style for the refs.
\hypersetup{colorlinks=true}   % Set to false for black/white printing
\input{Definitions}            % Include your abbreviations



\usepackage{enumitem}% http://ctan.org/pkg/enumitem
\usepackage{multirow}
\usepackage{float}
\usepackage{amsmath}
\usepackage{multicol}
\usepackage{amssymb}
\usepackage[normalem]{ulem}
\useunder{\uline}{\ul}{}
\usepackage{wrapfig}


\usepackage[table,xcdraw]{xcolor}


%% ----------------------------------------------------------------
\begin{document}
\frontmatter
\title      {Heterogeneous Agent-based Model for Supermarket Competition}
\authors    {\texorpdfstring
             {\href{mailto:sc22g13@ecs.soton.ac.uk}{Stefan J. Collier}}
             {Stefan J. Collier}
            }
\addresses  {\groupname\\\deptname\\\univname}
\date       {\today}
\subject    {}
\keywords   {}
\supervisor {Dr. Maria Polukarov}
\examiner   {Professor Sheng Chen}

\maketitle
\begin{abstract}
This project aim was to model and analyse the effects of competitive pricing behaviors of grocery retailers on the British market. 

This was achieved by creating a multi-agent model, containing retailer and consumer agents. The heterogeneous crowd of retailers employs either a uniform pricing strategy or a ‘local price flexing’ strategy. The actions of these retailers are chosen by predicting the profit of each action, using a perceptron. Following on from the consideration of different economic models, a discrete model was developed so that software agents have a discrete environment to operate within. Within the model, it has been observed how supermarkets with differing behaviors affect a heterogeneous crowd of consumer agents. The model was implemented in Java with Python used to evaluate the results. 

The simulation displays good acceptance with real grocery market behavior, i.e. captures the performance of British retailers thus can be used to determine the impact of changes in their behavior on their competitors and consumers.Furthermore it can be used to provide insight into sustainability of volatile pricing strategies, providing a useful insight in volatility of British supermarket retail industry. 
\end{abstract}
\acknowledgements{
I would like to express my sincere gratitude to Dr Maria Polukarov for her guidance and support which provided me the freedom to take this research in the direction of my interest.\\
\\
I would also like to thank my family and friends for their encouragement and support. To those who quietly listened to my software complaints. To those who worked throughout the nights with me. To those who helped me write what I couldn't say. I cannot thank you enough.
}

\declaration{
I, Stefan Collier, declare that this dissertation and the work presented in it are my own and has been generated by me as the result of my own original research.\\
I confirm that:\\
1. This work was done wholly or mainly while in candidature for a degree at this University;\\
2. Where any part of this dissertation has previously been submitted for any other qualification at this University or any other institution, this has been clearly stated;\\
3. Where I have consulted the published work of others, this is always clearly attributed;\\
4. Where I have quoted from the work of others, the source is always given. With the exception of such quotations, this dissertation is entirely my own work;\\
5. I have acknowledged all main sources of help;\\
6. Where the thesis is based on work done by myself jointly with others, I have made clear exactly what was done by others and what I have contributed myself;\\
7. Either none of this work has been published before submission, or parts of this work have been published by :\\
\\
Stefan Collier\\
April 2016
}
\tableofcontents
\listoffigures
\listoftables

\mainmatter
%% ----------------------------------------------------------------
%\include{Introduction}
%\include{Conclusions}
\include{chapters/1Project/main}
\include{chapters/2Lit/main}
\include{chapters/3Design/HighLevel}
\include{chapters/3Design/InDepth}
\include{chapters/4Impl/main}

\include{chapters/5Experiments/1/main}
\include{chapters/5Experiments/2/main}
\include{chapters/5Experiments/3/main}
\include{chapters/5Experiments/4/main}

\include{chapters/6Conclusion/main}

\appendix
\include{appendix/AppendixB}
\include{appendix/D/main}
\include{appendix/AppendixC}

\backmatter
\bibliographystyle{ecs}
\bibliography{ECS}
\end{document}
%% ----------------------------------------------------------------

 %% ----------------------------------------------------------------
%% Progress.tex
%% ---------------------------------------------------------------- 
\documentclass{ecsprogress}    % Use the progress Style
\graphicspath{{../figs/}}   % Location of your graphics files
    \usepackage{natbib}            % Use Natbib style for the refs.
\hypersetup{colorlinks=true}   % Set to false for black/white printing
\input{Definitions}            % Include your abbreviations



\usepackage{enumitem}% http://ctan.org/pkg/enumitem
\usepackage{multirow}
\usepackage{float}
\usepackage{amsmath}
\usepackage{multicol}
\usepackage{amssymb}
\usepackage[normalem]{ulem}
\useunder{\uline}{\ul}{}
\usepackage{wrapfig}


\usepackage[table,xcdraw]{xcolor}


%% ----------------------------------------------------------------
\begin{document}
\frontmatter
\title      {Heterogeneous Agent-based Model for Supermarket Competition}
\authors    {\texorpdfstring
             {\href{mailto:sc22g13@ecs.soton.ac.uk}{Stefan J. Collier}}
             {Stefan J. Collier}
            }
\addresses  {\groupname\\\deptname\\\univname}
\date       {\today}
\subject    {}
\keywords   {}
\supervisor {Dr. Maria Polukarov}
\examiner   {Professor Sheng Chen}

\maketitle
\begin{abstract}
This project aim was to model and analyse the effects of competitive pricing behaviors of grocery retailers on the British market. 

This was achieved by creating a multi-agent model, containing retailer and consumer agents. The heterogeneous crowd of retailers employs either a uniform pricing strategy or a ‘local price flexing’ strategy. The actions of these retailers are chosen by predicting the profit of each action, using a perceptron. Following on from the consideration of different economic models, a discrete model was developed so that software agents have a discrete environment to operate within. Within the model, it has been observed how supermarkets with differing behaviors affect a heterogeneous crowd of consumer agents. The model was implemented in Java with Python used to evaluate the results. 

The simulation displays good acceptance with real grocery market behavior, i.e. captures the performance of British retailers thus can be used to determine the impact of changes in their behavior on their competitors and consumers.Furthermore it can be used to provide insight into sustainability of volatile pricing strategies, providing a useful insight in volatility of British supermarket retail industry. 
\end{abstract}
\acknowledgements{
I would like to express my sincere gratitude to Dr Maria Polukarov for her guidance and support which provided me the freedom to take this research in the direction of my interest.\\
\\
I would also like to thank my family and friends for their encouragement and support. To those who quietly listened to my software complaints. To those who worked throughout the nights with me. To those who helped me write what I couldn't say. I cannot thank you enough.
}

\declaration{
I, Stefan Collier, declare that this dissertation and the work presented in it are my own and has been generated by me as the result of my own original research.\\
I confirm that:\\
1. This work was done wholly or mainly while in candidature for a degree at this University;\\
2. Where any part of this dissertation has previously been submitted for any other qualification at this University or any other institution, this has been clearly stated;\\
3. Where I have consulted the published work of others, this is always clearly attributed;\\
4. Where I have quoted from the work of others, the source is always given. With the exception of such quotations, this dissertation is entirely my own work;\\
5. I have acknowledged all main sources of help;\\
6. Where the thesis is based on work done by myself jointly with others, I have made clear exactly what was done by others and what I have contributed myself;\\
7. Either none of this work has been published before submission, or parts of this work have been published by :\\
\\
Stefan Collier\\
April 2016
}
\tableofcontents
\listoffigures
\listoftables

\mainmatter
%% ----------------------------------------------------------------
%\include{Introduction}
%\include{Conclusions}
\include{chapters/1Project/main}
\include{chapters/2Lit/main}
\include{chapters/3Design/HighLevel}
\include{chapters/3Design/InDepth}
\include{chapters/4Impl/main}

\include{chapters/5Experiments/1/main}
\include{chapters/5Experiments/2/main}
\include{chapters/5Experiments/3/main}
\include{chapters/5Experiments/4/main}

\include{chapters/6Conclusion/main}

\appendix
\include{appendix/AppendixB}
\include{appendix/D/main}
\include{appendix/AppendixC}

\backmatter
\bibliographystyle{ecs}
\bibliography{ECS}
\end{document}
%% ----------------------------------------------------------------

 %% ----------------------------------------------------------------
%% Progress.tex
%% ---------------------------------------------------------------- 
\documentclass{ecsprogress}    % Use the progress Style
\graphicspath{{../figs/}}   % Location of your graphics files
    \usepackage{natbib}            % Use Natbib style for the refs.
\hypersetup{colorlinks=true}   % Set to false for black/white printing
\input{Definitions}            % Include your abbreviations



\usepackage{enumitem}% http://ctan.org/pkg/enumitem
\usepackage{multirow}
\usepackage{float}
\usepackage{amsmath}
\usepackage{multicol}
\usepackage{amssymb}
\usepackage[normalem]{ulem}
\useunder{\uline}{\ul}{}
\usepackage{wrapfig}


\usepackage[table,xcdraw]{xcolor}


%% ----------------------------------------------------------------
\begin{document}
\frontmatter
\title      {Heterogeneous Agent-based Model for Supermarket Competition}
\authors    {\texorpdfstring
             {\href{mailto:sc22g13@ecs.soton.ac.uk}{Stefan J. Collier}}
             {Stefan J. Collier}
            }
\addresses  {\groupname\\\deptname\\\univname}
\date       {\today}
\subject    {}
\keywords   {}
\supervisor {Dr. Maria Polukarov}
\examiner   {Professor Sheng Chen}

\maketitle
\begin{abstract}
This project aim was to model and analyse the effects of competitive pricing behaviors of grocery retailers on the British market. 

This was achieved by creating a multi-agent model, containing retailer and consumer agents. The heterogeneous crowd of retailers employs either a uniform pricing strategy or a ‘local price flexing’ strategy. The actions of these retailers are chosen by predicting the profit of each action, using a perceptron. Following on from the consideration of different economic models, a discrete model was developed so that software agents have a discrete environment to operate within. Within the model, it has been observed how supermarkets with differing behaviors affect a heterogeneous crowd of consumer agents. The model was implemented in Java with Python used to evaluate the results. 

The simulation displays good acceptance with real grocery market behavior, i.e. captures the performance of British retailers thus can be used to determine the impact of changes in their behavior on their competitors and consumers.Furthermore it can be used to provide insight into sustainability of volatile pricing strategies, providing a useful insight in volatility of British supermarket retail industry. 
\end{abstract}
\acknowledgements{
I would like to express my sincere gratitude to Dr Maria Polukarov for her guidance and support which provided me the freedom to take this research in the direction of my interest.\\
\\
I would also like to thank my family and friends for their encouragement and support. To those who quietly listened to my software complaints. To those who worked throughout the nights with me. To those who helped me write what I couldn't say. I cannot thank you enough.
}

\declaration{
I, Stefan Collier, declare that this dissertation and the work presented in it are my own and has been generated by me as the result of my own original research.\\
I confirm that:\\
1. This work was done wholly or mainly while in candidature for a degree at this University;\\
2. Where any part of this dissertation has previously been submitted for any other qualification at this University or any other institution, this has been clearly stated;\\
3. Where I have consulted the published work of others, this is always clearly attributed;\\
4. Where I have quoted from the work of others, the source is always given. With the exception of such quotations, this dissertation is entirely my own work;\\
5. I have acknowledged all main sources of help;\\
6. Where the thesis is based on work done by myself jointly with others, I have made clear exactly what was done by others and what I have contributed myself;\\
7. Either none of this work has been published before submission, or parts of this work have been published by :\\
\\
Stefan Collier\\
April 2016
}
\tableofcontents
\listoffigures
\listoftables

\mainmatter
%% ----------------------------------------------------------------
%\include{Introduction}
%\include{Conclusions}
\include{chapters/1Project/main}
\include{chapters/2Lit/main}
\include{chapters/3Design/HighLevel}
\include{chapters/3Design/InDepth}
\include{chapters/4Impl/main}

\include{chapters/5Experiments/1/main}
\include{chapters/5Experiments/2/main}
\include{chapters/5Experiments/3/main}
\include{chapters/5Experiments/4/main}

\include{chapters/6Conclusion/main}

\appendix
\include{appendix/AppendixB}
\include{appendix/D/main}
\include{appendix/AppendixC}

\backmatter
\bibliographystyle{ecs}
\bibliography{ECS}
\end{document}
%% ----------------------------------------------------------------

 %% ----------------------------------------------------------------
%% Progress.tex
%% ---------------------------------------------------------------- 
\documentclass{ecsprogress}    % Use the progress Style
\graphicspath{{../figs/}}   % Location of your graphics files
    \usepackage{natbib}            % Use Natbib style for the refs.
\hypersetup{colorlinks=true}   % Set to false for black/white printing
\input{Definitions}            % Include your abbreviations



\usepackage{enumitem}% http://ctan.org/pkg/enumitem
\usepackage{multirow}
\usepackage{float}
\usepackage{amsmath}
\usepackage{multicol}
\usepackage{amssymb}
\usepackage[normalem]{ulem}
\useunder{\uline}{\ul}{}
\usepackage{wrapfig}


\usepackage[table,xcdraw]{xcolor}


%% ----------------------------------------------------------------
\begin{document}
\frontmatter
\title      {Heterogeneous Agent-based Model for Supermarket Competition}
\authors    {\texorpdfstring
             {\href{mailto:sc22g13@ecs.soton.ac.uk}{Stefan J. Collier}}
             {Stefan J. Collier}
            }
\addresses  {\groupname\\\deptname\\\univname}
\date       {\today}
\subject    {}
\keywords   {}
\supervisor {Dr. Maria Polukarov}
\examiner   {Professor Sheng Chen}

\maketitle
\begin{abstract}
This project aim was to model and analyse the effects of competitive pricing behaviors of grocery retailers on the British market. 

This was achieved by creating a multi-agent model, containing retailer and consumer agents. The heterogeneous crowd of retailers employs either a uniform pricing strategy or a ‘local price flexing’ strategy. The actions of these retailers are chosen by predicting the profit of each action, using a perceptron. Following on from the consideration of different economic models, a discrete model was developed so that software agents have a discrete environment to operate within. Within the model, it has been observed how supermarkets with differing behaviors affect a heterogeneous crowd of consumer agents. The model was implemented in Java with Python used to evaluate the results. 

The simulation displays good acceptance with real grocery market behavior, i.e. captures the performance of British retailers thus can be used to determine the impact of changes in their behavior on their competitors and consumers.Furthermore it can be used to provide insight into sustainability of volatile pricing strategies, providing a useful insight in volatility of British supermarket retail industry. 
\end{abstract}
\acknowledgements{
I would like to express my sincere gratitude to Dr Maria Polukarov for her guidance and support which provided me the freedom to take this research in the direction of my interest.\\
\\
I would also like to thank my family and friends for their encouragement and support. To those who quietly listened to my software complaints. To those who worked throughout the nights with me. To those who helped me write what I couldn't say. I cannot thank you enough.
}

\declaration{
I, Stefan Collier, declare that this dissertation and the work presented in it are my own and has been generated by me as the result of my own original research.\\
I confirm that:\\
1. This work was done wholly or mainly while in candidature for a degree at this University;\\
2. Where any part of this dissertation has previously been submitted for any other qualification at this University or any other institution, this has been clearly stated;\\
3. Where I have consulted the published work of others, this is always clearly attributed;\\
4. Where I have quoted from the work of others, the source is always given. With the exception of such quotations, this dissertation is entirely my own work;\\
5. I have acknowledged all main sources of help;\\
6. Where the thesis is based on work done by myself jointly with others, I have made clear exactly what was done by others and what I have contributed myself;\\
7. Either none of this work has been published before submission, or parts of this work have been published by :\\
\\
Stefan Collier\\
April 2016
}
\tableofcontents
\listoffigures
\listoftables

\mainmatter
%% ----------------------------------------------------------------
%\include{Introduction}
%\include{Conclusions}
\include{chapters/1Project/main}
\include{chapters/2Lit/main}
\include{chapters/3Design/HighLevel}
\include{chapters/3Design/InDepth}
\include{chapters/4Impl/main}

\include{chapters/5Experiments/1/main}
\include{chapters/5Experiments/2/main}
\include{chapters/5Experiments/3/main}
\include{chapters/5Experiments/4/main}

\include{chapters/6Conclusion/main}

\appendix
\include{appendix/AppendixB}
\include{appendix/D/main}
\include{appendix/AppendixC}

\backmatter
\bibliographystyle{ecs}
\bibliography{ECS}
\end{document}
%% ----------------------------------------------------------------


 %% ----------------------------------------------------------------
%% Progress.tex
%% ---------------------------------------------------------------- 
\documentclass{ecsprogress}    % Use the progress Style
\graphicspath{{../figs/}}   % Location of your graphics files
    \usepackage{natbib}            % Use Natbib style for the refs.
\hypersetup{colorlinks=true}   % Set to false for black/white printing
\input{Definitions}            % Include your abbreviations



\usepackage{enumitem}% http://ctan.org/pkg/enumitem
\usepackage{multirow}
\usepackage{float}
\usepackage{amsmath}
\usepackage{multicol}
\usepackage{amssymb}
\usepackage[normalem]{ulem}
\useunder{\uline}{\ul}{}
\usepackage{wrapfig}


\usepackage[table,xcdraw]{xcolor}


%% ----------------------------------------------------------------
\begin{document}
\frontmatter
\title      {Heterogeneous Agent-based Model for Supermarket Competition}
\authors    {\texorpdfstring
             {\href{mailto:sc22g13@ecs.soton.ac.uk}{Stefan J. Collier}}
             {Stefan J. Collier}
            }
\addresses  {\groupname\\\deptname\\\univname}
\date       {\today}
\subject    {}
\keywords   {}
\supervisor {Dr. Maria Polukarov}
\examiner   {Professor Sheng Chen}

\maketitle
\begin{abstract}
This project aim was to model and analyse the effects of competitive pricing behaviors of grocery retailers on the British market. 

This was achieved by creating a multi-agent model, containing retailer and consumer agents. The heterogeneous crowd of retailers employs either a uniform pricing strategy or a ‘local price flexing’ strategy. The actions of these retailers are chosen by predicting the profit of each action, using a perceptron. Following on from the consideration of different economic models, a discrete model was developed so that software agents have a discrete environment to operate within. Within the model, it has been observed how supermarkets with differing behaviors affect a heterogeneous crowd of consumer agents. The model was implemented in Java with Python used to evaluate the results. 

The simulation displays good acceptance with real grocery market behavior, i.e. captures the performance of British retailers thus can be used to determine the impact of changes in their behavior on their competitors and consumers.Furthermore it can be used to provide insight into sustainability of volatile pricing strategies, providing a useful insight in volatility of British supermarket retail industry. 
\end{abstract}
\acknowledgements{
I would like to express my sincere gratitude to Dr Maria Polukarov for her guidance and support which provided me the freedom to take this research in the direction of my interest.\\
\\
I would also like to thank my family and friends for their encouragement and support. To those who quietly listened to my software complaints. To those who worked throughout the nights with me. To those who helped me write what I couldn't say. I cannot thank you enough.
}

\declaration{
I, Stefan Collier, declare that this dissertation and the work presented in it are my own and has been generated by me as the result of my own original research.\\
I confirm that:\\
1. This work was done wholly or mainly while in candidature for a degree at this University;\\
2. Where any part of this dissertation has previously been submitted for any other qualification at this University or any other institution, this has been clearly stated;\\
3. Where I have consulted the published work of others, this is always clearly attributed;\\
4. Where I have quoted from the work of others, the source is always given. With the exception of such quotations, this dissertation is entirely my own work;\\
5. I have acknowledged all main sources of help;\\
6. Where the thesis is based on work done by myself jointly with others, I have made clear exactly what was done by others and what I have contributed myself;\\
7. Either none of this work has been published before submission, or parts of this work have been published by :\\
\\
Stefan Collier\\
April 2016
}
\tableofcontents
\listoffigures
\listoftables

\mainmatter
%% ----------------------------------------------------------------
%\include{Introduction}
%\include{Conclusions}
\include{chapters/1Project/main}
\include{chapters/2Lit/main}
\include{chapters/3Design/HighLevel}
\include{chapters/3Design/InDepth}
\include{chapters/4Impl/main}

\include{chapters/5Experiments/1/main}
\include{chapters/5Experiments/2/main}
\include{chapters/5Experiments/3/main}
\include{chapters/5Experiments/4/main}

\include{chapters/6Conclusion/main}

\appendix
\include{appendix/AppendixB}
\include{appendix/D/main}
\include{appendix/AppendixC}

\backmatter
\bibliographystyle{ecs}
\bibliography{ECS}
\end{document}
%% ----------------------------------------------------------------


\appendix
\include{appendix/AppendixB}
 %% ----------------------------------------------------------------
%% Progress.tex
%% ---------------------------------------------------------------- 
\documentclass{ecsprogress}    % Use the progress Style
\graphicspath{{../figs/}}   % Location of your graphics files
    \usepackage{natbib}            % Use Natbib style for the refs.
\hypersetup{colorlinks=true}   % Set to false for black/white printing
\input{Definitions}            % Include your abbreviations



\usepackage{enumitem}% http://ctan.org/pkg/enumitem
\usepackage{multirow}
\usepackage{float}
\usepackage{amsmath}
\usepackage{multicol}
\usepackage{amssymb}
\usepackage[normalem]{ulem}
\useunder{\uline}{\ul}{}
\usepackage{wrapfig}


\usepackage[table,xcdraw]{xcolor}


%% ----------------------------------------------------------------
\begin{document}
\frontmatter
\title      {Heterogeneous Agent-based Model for Supermarket Competition}
\authors    {\texorpdfstring
             {\href{mailto:sc22g13@ecs.soton.ac.uk}{Stefan J. Collier}}
             {Stefan J. Collier}
            }
\addresses  {\groupname\\\deptname\\\univname}
\date       {\today}
\subject    {}
\keywords   {}
\supervisor {Dr. Maria Polukarov}
\examiner   {Professor Sheng Chen}

\maketitle
\begin{abstract}
This project aim was to model and analyse the effects of competitive pricing behaviors of grocery retailers on the British market. 

This was achieved by creating a multi-agent model, containing retailer and consumer agents. The heterogeneous crowd of retailers employs either a uniform pricing strategy or a ‘local price flexing’ strategy. The actions of these retailers are chosen by predicting the profit of each action, using a perceptron. Following on from the consideration of different economic models, a discrete model was developed so that software agents have a discrete environment to operate within. Within the model, it has been observed how supermarkets with differing behaviors affect a heterogeneous crowd of consumer agents. The model was implemented in Java with Python used to evaluate the results. 

The simulation displays good acceptance with real grocery market behavior, i.e. captures the performance of British retailers thus can be used to determine the impact of changes in their behavior on their competitors and consumers.Furthermore it can be used to provide insight into sustainability of volatile pricing strategies, providing a useful insight in volatility of British supermarket retail industry. 
\end{abstract}
\acknowledgements{
I would like to express my sincere gratitude to Dr Maria Polukarov for her guidance and support which provided me the freedom to take this research in the direction of my interest.\\
\\
I would also like to thank my family and friends for their encouragement and support. To those who quietly listened to my software complaints. To those who worked throughout the nights with me. To those who helped me write what I couldn't say. I cannot thank you enough.
}

\declaration{
I, Stefan Collier, declare that this dissertation and the work presented in it are my own and has been generated by me as the result of my own original research.\\
I confirm that:\\
1. This work was done wholly or mainly while in candidature for a degree at this University;\\
2. Where any part of this dissertation has previously been submitted for any other qualification at this University or any other institution, this has been clearly stated;\\
3. Where I have consulted the published work of others, this is always clearly attributed;\\
4. Where I have quoted from the work of others, the source is always given. With the exception of such quotations, this dissertation is entirely my own work;\\
5. I have acknowledged all main sources of help;\\
6. Where the thesis is based on work done by myself jointly with others, I have made clear exactly what was done by others and what I have contributed myself;\\
7. Either none of this work has been published before submission, or parts of this work have been published by :\\
\\
Stefan Collier\\
April 2016
}
\tableofcontents
\listoffigures
\listoftables

\mainmatter
%% ----------------------------------------------------------------
%\include{Introduction}
%\include{Conclusions}
\include{chapters/1Project/main}
\include{chapters/2Lit/main}
\include{chapters/3Design/HighLevel}
\include{chapters/3Design/InDepth}
\include{chapters/4Impl/main}

\include{chapters/5Experiments/1/main}
\include{chapters/5Experiments/2/main}
\include{chapters/5Experiments/3/main}
\include{chapters/5Experiments/4/main}

\include{chapters/6Conclusion/main}

\appendix
\include{appendix/AppendixB}
\include{appendix/D/main}
\include{appendix/AppendixC}

\backmatter
\bibliographystyle{ecs}
\bibliography{ECS}
\end{document}
%% ----------------------------------------------------------------

\include{appendix/AppendixC}

\backmatter
\bibliographystyle{ecs}
\bibliography{ECS}
\end{document}
%% ----------------------------------------------------------------


\appendix
\include{appendix/AppendixB}
 %% ----------------------------------------------------------------
%% Progress.tex
%% ---------------------------------------------------------------- 
\documentclass{ecsprogress}    % Use the progress Style
\graphicspath{{../figs/}}   % Location of your graphics files
    \usepackage{natbib}            % Use Natbib style for the refs.
\hypersetup{colorlinks=true}   % Set to false for black/white printing
\input{Definitions}            % Include your abbreviations



\usepackage{enumitem}% http://ctan.org/pkg/enumitem
\usepackage{multirow}
\usepackage{float}
\usepackage{amsmath}
\usepackage{multicol}
\usepackage{amssymb}
\usepackage[normalem]{ulem}
\useunder{\uline}{\ul}{}
\usepackage{wrapfig}


\usepackage[table,xcdraw]{xcolor}


%% ----------------------------------------------------------------
\begin{document}
\frontmatter
\title      {Heterogeneous Agent-based Model for Supermarket Competition}
\authors    {\texorpdfstring
             {\href{mailto:sc22g13@ecs.soton.ac.uk}{Stefan J. Collier}}
             {Stefan J. Collier}
            }
\addresses  {\groupname\\\deptname\\\univname}
\date       {\today}
\subject    {}
\keywords   {}
\supervisor {Dr. Maria Polukarov}
\examiner   {Professor Sheng Chen}

\maketitle
\begin{abstract}
This project aim was to model and analyse the effects of competitive pricing behaviors of grocery retailers on the British market. 

This was achieved by creating a multi-agent model, containing retailer and consumer agents. The heterogeneous crowd of retailers employs either a uniform pricing strategy or a ‘local price flexing’ strategy. The actions of these retailers are chosen by predicting the profit of each action, using a perceptron. Following on from the consideration of different economic models, a discrete model was developed so that software agents have a discrete environment to operate within. Within the model, it has been observed how supermarkets with differing behaviors affect a heterogeneous crowd of consumer agents. The model was implemented in Java with Python used to evaluate the results. 

The simulation displays good acceptance with real grocery market behavior, i.e. captures the performance of British retailers thus can be used to determine the impact of changes in their behavior on their competitors and consumers.Furthermore it can be used to provide insight into sustainability of volatile pricing strategies, providing a useful insight in volatility of British supermarket retail industry. 
\end{abstract}
\acknowledgements{
I would like to express my sincere gratitude to Dr Maria Polukarov for her guidance and support which provided me the freedom to take this research in the direction of my interest.\\
\\
I would also like to thank my family and friends for their encouragement and support. To those who quietly listened to my software complaints. To those who worked throughout the nights with me. To those who helped me write what I couldn't say. I cannot thank you enough.
}

\declaration{
I, Stefan Collier, declare that this dissertation and the work presented in it are my own and has been generated by me as the result of my own original research.\\
I confirm that:\\
1. This work was done wholly or mainly while in candidature for a degree at this University;\\
2. Where any part of this dissertation has previously been submitted for any other qualification at this University or any other institution, this has been clearly stated;\\
3. Where I have consulted the published work of others, this is always clearly attributed;\\
4. Where I have quoted from the work of others, the source is always given. With the exception of such quotations, this dissertation is entirely my own work;\\
5. I have acknowledged all main sources of help;\\
6. Where the thesis is based on work done by myself jointly with others, I have made clear exactly what was done by others and what I have contributed myself;\\
7. Either none of this work has been published before submission, or parts of this work have been published by :\\
\\
Stefan Collier\\
April 2016
}
\tableofcontents
\listoffigures
\listoftables

\mainmatter
%% ----------------------------------------------------------------
%\include{Introduction}
%\include{Conclusions}
 %% ----------------------------------------------------------------
%% Progress.tex
%% ---------------------------------------------------------------- 
\documentclass{ecsprogress}    % Use the progress Style
\graphicspath{{../figs/}}   % Location of your graphics files
    \usepackage{natbib}            % Use Natbib style for the refs.
\hypersetup{colorlinks=true}   % Set to false for black/white printing
\input{Definitions}            % Include your abbreviations



\usepackage{enumitem}% http://ctan.org/pkg/enumitem
\usepackage{multirow}
\usepackage{float}
\usepackage{amsmath}
\usepackage{multicol}
\usepackage{amssymb}
\usepackage[normalem]{ulem}
\useunder{\uline}{\ul}{}
\usepackage{wrapfig}


\usepackage[table,xcdraw]{xcolor}


%% ----------------------------------------------------------------
\begin{document}
\frontmatter
\title      {Heterogeneous Agent-based Model for Supermarket Competition}
\authors    {\texorpdfstring
             {\href{mailto:sc22g13@ecs.soton.ac.uk}{Stefan J. Collier}}
             {Stefan J. Collier}
            }
\addresses  {\groupname\\\deptname\\\univname}
\date       {\today}
\subject    {}
\keywords   {}
\supervisor {Dr. Maria Polukarov}
\examiner   {Professor Sheng Chen}

\maketitle
\begin{abstract}
This project aim was to model and analyse the effects of competitive pricing behaviors of grocery retailers on the British market. 

This was achieved by creating a multi-agent model, containing retailer and consumer agents. The heterogeneous crowd of retailers employs either a uniform pricing strategy or a ‘local price flexing’ strategy. The actions of these retailers are chosen by predicting the profit of each action, using a perceptron. Following on from the consideration of different economic models, a discrete model was developed so that software agents have a discrete environment to operate within. Within the model, it has been observed how supermarkets with differing behaviors affect a heterogeneous crowd of consumer agents. The model was implemented in Java with Python used to evaluate the results. 

The simulation displays good acceptance with real grocery market behavior, i.e. captures the performance of British retailers thus can be used to determine the impact of changes in their behavior on their competitors and consumers.Furthermore it can be used to provide insight into sustainability of volatile pricing strategies, providing a useful insight in volatility of British supermarket retail industry. 
\end{abstract}
\acknowledgements{
I would like to express my sincere gratitude to Dr Maria Polukarov for her guidance and support which provided me the freedom to take this research in the direction of my interest.\\
\\
I would also like to thank my family and friends for their encouragement and support. To those who quietly listened to my software complaints. To those who worked throughout the nights with me. To those who helped me write what I couldn't say. I cannot thank you enough.
}

\declaration{
I, Stefan Collier, declare that this dissertation and the work presented in it are my own and has been generated by me as the result of my own original research.\\
I confirm that:\\
1. This work was done wholly or mainly while in candidature for a degree at this University;\\
2. Where any part of this dissertation has previously been submitted for any other qualification at this University or any other institution, this has been clearly stated;\\
3. Where I have consulted the published work of others, this is always clearly attributed;\\
4. Where I have quoted from the work of others, the source is always given. With the exception of such quotations, this dissertation is entirely my own work;\\
5. I have acknowledged all main sources of help;\\
6. Where the thesis is based on work done by myself jointly with others, I have made clear exactly what was done by others and what I have contributed myself;\\
7. Either none of this work has been published before submission, or parts of this work have been published by :\\
\\
Stefan Collier\\
April 2016
}
\tableofcontents
\listoffigures
\listoftables

\mainmatter
%% ----------------------------------------------------------------
%\include{Introduction}
%\include{Conclusions}
\include{chapters/1Project/main}
\include{chapters/2Lit/main}
\include{chapters/3Design/HighLevel}
\include{chapters/3Design/InDepth}
\include{chapters/4Impl/main}

\include{chapters/5Experiments/1/main}
\include{chapters/5Experiments/2/main}
\include{chapters/5Experiments/3/main}
\include{chapters/5Experiments/4/main}

\include{chapters/6Conclusion/main}

\appendix
\include{appendix/AppendixB}
\include{appendix/D/main}
\include{appendix/AppendixC}

\backmatter
\bibliographystyle{ecs}
\bibliography{ECS}
\end{document}
%% ----------------------------------------------------------------

 %% ----------------------------------------------------------------
%% Progress.tex
%% ---------------------------------------------------------------- 
\documentclass{ecsprogress}    % Use the progress Style
\graphicspath{{../figs/}}   % Location of your graphics files
    \usepackage{natbib}            % Use Natbib style for the refs.
\hypersetup{colorlinks=true}   % Set to false for black/white printing
\input{Definitions}            % Include your abbreviations



\usepackage{enumitem}% http://ctan.org/pkg/enumitem
\usepackage{multirow}
\usepackage{float}
\usepackage{amsmath}
\usepackage{multicol}
\usepackage{amssymb}
\usepackage[normalem]{ulem}
\useunder{\uline}{\ul}{}
\usepackage{wrapfig}


\usepackage[table,xcdraw]{xcolor}


%% ----------------------------------------------------------------
\begin{document}
\frontmatter
\title      {Heterogeneous Agent-based Model for Supermarket Competition}
\authors    {\texorpdfstring
             {\href{mailto:sc22g13@ecs.soton.ac.uk}{Stefan J. Collier}}
             {Stefan J. Collier}
            }
\addresses  {\groupname\\\deptname\\\univname}
\date       {\today}
\subject    {}
\keywords   {}
\supervisor {Dr. Maria Polukarov}
\examiner   {Professor Sheng Chen}

\maketitle
\begin{abstract}
This project aim was to model and analyse the effects of competitive pricing behaviors of grocery retailers on the British market. 

This was achieved by creating a multi-agent model, containing retailer and consumer agents. The heterogeneous crowd of retailers employs either a uniform pricing strategy or a ‘local price flexing’ strategy. The actions of these retailers are chosen by predicting the profit of each action, using a perceptron. Following on from the consideration of different economic models, a discrete model was developed so that software agents have a discrete environment to operate within. Within the model, it has been observed how supermarkets with differing behaviors affect a heterogeneous crowd of consumer agents. The model was implemented in Java with Python used to evaluate the results. 

The simulation displays good acceptance with real grocery market behavior, i.e. captures the performance of British retailers thus can be used to determine the impact of changes in their behavior on their competitors and consumers.Furthermore it can be used to provide insight into sustainability of volatile pricing strategies, providing a useful insight in volatility of British supermarket retail industry. 
\end{abstract}
\acknowledgements{
I would like to express my sincere gratitude to Dr Maria Polukarov for her guidance and support which provided me the freedom to take this research in the direction of my interest.\\
\\
I would also like to thank my family and friends for their encouragement and support. To those who quietly listened to my software complaints. To those who worked throughout the nights with me. To those who helped me write what I couldn't say. I cannot thank you enough.
}

\declaration{
I, Stefan Collier, declare that this dissertation and the work presented in it are my own and has been generated by me as the result of my own original research.\\
I confirm that:\\
1. This work was done wholly or mainly while in candidature for a degree at this University;\\
2. Where any part of this dissertation has previously been submitted for any other qualification at this University or any other institution, this has been clearly stated;\\
3. Where I have consulted the published work of others, this is always clearly attributed;\\
4. Where I have quoted from the work of others, the source is always given. With the exception of such quotations, this dissertation is entirely my own work;\\
5. I have acknowledged all main sources of help;\\
6. Where the thesis is based on work done by myself jointly with others, I have made clear exactly what was done by others and what I have contributed myself;\\
7. Either none of this work has been published before submission, or parts of this work have been published by :\\
\\
Stefan Collier\\
April 2016
}
\tableofcontents
\listoffigures
\listoftables

\mainmatter
%% ----------------------------------------------------------------
%\include{Introduction}
%\include{Conclusions}
\include{chapters/1Project/main}
\include{chapters/2Lit/main}
\include{chapters/3Design/HighLevel}
\include{chapters/3Design/InDepth}
\include{chapters/4Impl/main}

\include{chapters/5Experiments/1/main}
\include{chapters/5Experiments/2/main}
\include{chapters/5Experiments/3/main}
\include{chapters/5Experiments/4/main}

\include{chapters/6Conclusion/main}

\appendix
\include{appendix/AppendixB}
\include{appendix/D/main}
\include{appendix/AppendixC}

\backmatter
\bibliographystyle{ecs}
\bibliography{ECS}
\end{document}
%% ----------------------------------------------------------------

\include{chapters/3Design/HighLevel}
\include{chapters/3Design/InDepth}
 %% ----------------------------------------------------------------
%% Progress.tex
%% ---------------------------------------------------------------- 
\documentclass{ecsprogress}    % Use the progress Style
\graphicspath{{../figs/}}   % Location of your graphics files
    \usepackage{natbib}            % Use Natbib style for the refs.
\hypersetup{colorlinks=true}   % Set to false for black/white printing
\input{Definitions}            % Include your abbreviations



\usepackage{enumitem}% http://ctan.org/pkg/enumitem
\usepackage{multirow}
\usepackage{float}
\usepackage{amsmath}
\usepackage{multicol}
\usepackage{amssymb}
\usepackage[normalem]{ulem}
\useunder{\uline}{\ul}{}
\usepackage{wrapfig}


\usepackage[table,xcdraw]{xcolor}


%% ----------------------------------------------------------------
\begin{document}
\frontmatter
\title      {Heterogeneous Agent-based Model for Supermarket Competition}
\authors    {\texorpdfstring
             {\href{mailto:sc22g13@ecs.soton.ac.uk}{Stefan J. Collier}}
             {Stefan J. Collier}
            }
\addresses  {\groupname\\\deptname\\\univname}
\date       {\today}
\subject    {}
\keywords   {}
\supervisor {Dr. Maria Polukarov}
\examiner   {Professor Sheng Chen}

\maketitle
\begin{abstract}
This project aim was to model and analyse the effects of competitive pricing behaviors of grocery retailers on the British market. 

This was achieved by creating a multi-agent model, containing retailer and consumer agents. The heterogeneous crowd of retailers employs either a uniform pricing strategy or a ‘local price flexing’ strategy. The actions of these retailers are chosen by predicting the profit of each action, using a perceptron. Following on from the consideration of different economic models, a discrete model was developed so that software agents have a discrete environment to operate within. Within the model, it has been observed how supermarkets with differing behaviors affect a heterogeneous crowd of consumer agents. The model was implemented in Java with Python used to evaluate the results. 

The simulation displays good acceptance with real grocery market behavior, i.e. captures the performance of British retailers thus can be used to determine the impact of changes in their behavior on their competitors and consumers.Furthermore it can be used to provide insight into sustainability of volatile pricing strategies, providing a useful insight in volatility of British supermarket retail industry. 
\end{abstract}
\acknowledgements{
I would like to express my sincere gratitude to Dr Maria Polukarov for her guidance and support which provided me the freedom to take this research in the direction of my interest.\\
\\
I would also like to thank my family and friends for their encouragement and support. To those who quietly listened to my software complaints. To those who worked throughout the nights with me. To those who helped me write what I couldn't say. I cannot thank you enough.
}

\declaration{
I, Stefan Collier, declare that this dissertation and the work presented in it are my own and has been generated by me as the result of my own original research.\\
I confirm that:\\
1. This work was done wholly or mainly while in candidature for a degree at this University;\\
2. Where any part of this dissertation has previously been submitted for any other qualification at this University or any other institution, this has been clearly stated;\\
3. Where I have consulted the published work of others, this is always clearly attributed;\\
4. Where I have quoted from the work of others, the source is always given. With the exception of such quotations, this dissertation is entirely my own work;\\
5. I have acknowledged all main sources of help;\\
6. Where the thesis is based on work done by myself jointly with others, I have made clear exactly what was done by others and what I have contributed myself;\\
7. Either none of this work has been published before submission, or parts of this work have been published by :\\
\\
Stefan Collier\\
April 2016
}
\tableofcontents
\listoffigures
\listoftables

\mainmatter
%% ----------------------------------------------------------------
%\include{Introduction}
%\include{Conclusions}
\include{chapters/1Project/main}
\include{chapters/2Lit/main}
\include{chapters/3Design/HighLevel}
\include{chapters/3Design/InDepth}
\include{chapters/4Impl/main}

\include{chapters/5Experiments/1/main}
\include{chapters/5Experiments/2/main}
\include{chapters/5Experiments/3/main}
\include{chapters/5Experiments/4/main}

\include{chapters/6Conclusion/main}

\appendix
\include{appendix/AppendixB}
\include{appendix/D/main}
\include{appendix/AppendixC}

\backmatter
\bibliographystyle{ecs}
\bibliography{ECS}
\end{document}
%% ----------------------------------------------------------------


 %% ----------------------------------------------------------------
%% Progress.tex
%% ---------------------------------------------------------------- 
\documentclass{ecsprogress}    % Use the progress Style
\graphicspath{{../figs/}}   % Location of your graphics files
    \usepackage{natbib}            % Use Natbib style for the refs.
\hypersetup{colorlinks=true}   % Set to false for black/white printing
\input{Definitions}            % Include your abbreviations



\usepackage{enumitem}% http://ctan.org/pkg/enumitem
\usepackage{multirow}
\usepackage{float}
\usepackage{amsmath}
\usepackage{multicol}
\usepackage{amssymb}
\usepackage[normalem]{ulem}
\useunder{\uline}{\ul}{}
\usepackage{wrapfig}


\usepackage[table,xcdraw]{xcolor}


%% ----------------------------------------------------------------
\begin{document}
\frontmatter
\title      {Heterogeneous Agent-based Model for Supermarket Competition}
\authors    {\texorpdfstring
             {\href{mailto:sc22g13@ecs.soton.ac.uk}{Stefan J. Collier}}
             {Stefan J. Collier}
            }
\addresses  {\groupname\\\deptname\\\univname}
\date       {\today}
\subject    {}
\keywords   {}
\supervisor {Dr. Maria Polukarov}
\examiner   {Professor Sheng Chen}

\maketitle
\begin{abstract}
This project aim was to model and analyse the effects of competitive pricing behaviors of grocery retailers on the British market. 

This was achieved by creating a multi-agent model, containing retailer and consumer agents. The heterogeneous crowd of retailers employs either a uniform pricing strategy or a ‘local price flexing’ strategy. The actions of these retailers are chosen by predicting the profit of each action, using a perceptron. Following on from the consideration of different economic models, a discrete model was developed so that software agents have a discrete environment to operate within. Within the model, it has been observed how supermarkets with differing behaviors affect a heterogeneous crowd of consumer agents. The model was implemented in Java with Python used to evaluate the results. 

The simulation displays good acceptance with real grocery market behavior, i.e. captures the performance of British retailers thus can be used to determine the impact of changes in their behavior on their competitors and consumers.Furthermore it can be used to provide insight into sustainability of volatile pricing strategies, providing a useful insight in volatility of British supermarket retail industry. 
\end{abstract}
\acknowledgements{
I would like to express my sincere gratitude to Dr Maria Polukarov for her guidance and support which provided me the freedom to take this research in the direction of my interest.\\
\\
I would also like to thank my family and friends for their encouragement and support. To those who quietly listened to my software complaints. To those who worked throughout the nights with me. To those who helped me write what I couldn't say. I cannot thank you enough.
}

\declaration{
I, Stefan Collier, declare that this dissertation and the work presented in it are my own and has been generated by me as the result of my own original research.\\
I confirm that:\\
1. This work was done wholly or mainly while in candidature for a degree at this University;\\
2. Where any part of this dissertation has previously been submitted for any other qualification at this University or any other institution, this has been clearly stated;\\
3. Where I have consulted the published work of others, this is always clearly attributed;\\
4. Where I have quoted from the work of others, the source is always given. With the exception of such quotations, this dissertation is entirely my own work;\\
5. I have acknowledged all main sources of help;\\
6. Where the thesis is based on work done by myself jointly with others, I have made clear exactly what was done by others and what I have contributed myself;\\
7. Either none of this work has been published before submission, or parts of this work have been published by :\\
\\
Stefan Collier\\
April 2016
}
\tableofcontents
\listoffigures
\listoftables

\mainmatter
%% ----------------------------------------------------------------
%\include{Introduction}
%\include{Conclusions}
\include{chapters/1Project/main}
\include{chapters/2Lit/main}
\include{chapters/3Design/HighLevel}
\include{chapters/3Design/InDepth}
\include{chapters/4Impl/main}

\include{chapters/5Experiments/1/main}
\include{chapters/5Experiments/2/main}
\include{chapters/5Experiments/3/main}
\include{chapters/5Experiments/4/main}

\include{chapters/6Conclusion/main}

\appendix
\include{appendix/AppendixB}
\include{appendix/D/main}
\include{appendix/AppendixC}

\backmatter
\bibliographystyle{ecs}
\bibliography{ECS}
\end{document}
%% ----------------------------------------------------------------

 %% ----------------------------------------------------------------
%% Progress.tex
%% ---------------------------------------------------------------- 
\documentclass{ecsprogress}    % Use the progress Style
\graphicspath{{../figs/}}   % Location of your graphics files
    \usepackage{natbib}            % Use Natbib style for the refs.
\hypersetup{colorlinks=true}   % Set to false for black/white printing
\input{Definitions}            % Include your abbreviations



\usepackage{enumitem}% http://ctan.org/pkg/enumitem
\usepackage{multirow}
\usepackage{float}
\usepackage{amsmath}
\usepackage{multicol}
\usepackage{amssymb}
\usepackage[normalem]{ulem}
\useunder{\uline}{\ul}{}
\usepackage{wrapfig}


\usepackage[table,xcdraw]{xcolor}


%% ----------------------------------------------------------------
\begin{document}
\frontmatter
\title      {Heterogeneous Agent-based Model for Supermarket Competition}
\authors    {\texorpdfstring
             {\href{mailto:sc22g13@ecs.soton.ac.uk}{Stefan J. Collier}}
             {Stefan J. Collier}
            }
\addresses  {\groupname\\\deptname\\\univname}
\date       {\today}
\subject    {}
\keywords   {}
\supervisor {Dr. Maria Polukarov}
\examiner   {Professor Sheng Chen}

\maketitle
\begin{abstract}
This project aim was to model and analyse the effects of competitive pricing behaviors of grocery retailers on the British market. 

This was achieved by creating a multi-agent model, containing retailer and consumer agents. The heterogeneous crowd of retailers employs either a uniform pricing strategy or a ‘local price flexing’ strategy. The actions of these retailers are chosen by predicting the profit of each action, using a perceptron. Following on from the consideration of different economic models, a discrete model was developed so that software agents have a discrete environment to operate within. Within the model, it has been observed how supermarkets with differing behaviors affect a heterogeneous crowd of consumer agents. The model was implemented in Java with Python used to evaluate the results. 

The simulation displays good acceptance with real grocery market behavior, i.e. captures the performance of British retailers thus can be used to determine the impact of changes in their behavior on their competitors and consumers.Furthermore it can be used to provide insight into sustainability of volatile pricing strategies, providing a useful insight in volatility of British supermarket retail industry. 
\end{abstract}
\acknowledgements{
I would like to express my sincere gratitude to Dr Maria Polukarov for her guidance and support which provided me the freedom to take this research in the direction of my interest.\\
\\
I would also like to thank my family and friends for their encouragement and support. To those who quietly listened to my software complaints. To those who worked throughout the nights with me. To those who helped me write what I couldn't say. I cannot thank you enough.
}

\declaration{
I, Stefan Collier, declare that this dissertation and the work presented in it are my own and has been generated by me as the result of my own original research.\\
I confirm that:\\
1. This work was done wholly or mainly while in candidature for a degree at this University;\\
2. Where any part of this dissertation has previously been submitted for any other qualification at this University or any other institution, this has been clearly stated;\\
3. Where I have consulted the published work of others, this is always clearly attributed;\\
4. Where I have quoted from the work of others, the source is always given. With the exception of such quotations, this dissertation is entirely my own work;\\
5. I have acknowledged all main sources of help;\\
6. Where the thesis is based on work done by myself jointly with others, I have made clear exactly what was done by others and what I have contributed myself;\\
7. Either none of this work has been published before submission, or parts of this work have been published by :\\
\\
Stefan Collier\\
April 2016
}
\tableofcontents
\listoffigures
\listoftables

\mainmatter
%% ----------------------------------------------------------------
%\include{Introduction}
%\include{Conclusions}
\include{chapters/1Project/main}
\include{chapters/2Lit/main}
\include{chapters/3Design/HighLevel}
\include{chapters/3Design/InDepth}
\include{chapters/4Impl/main}

\include{chapters/5Experiments/1/main}
\include{chapters/5Experiments/2/main}
\include{chapters/5Experiments/3/main}
\include{chapters/5Experiments/4/main}

\include{chapters/6Conclusion/main}

\appendix
\include{appendix/AppendixB}
\include{appendix/D/main}
\include{appendix/AppendixC}

\backmatter
\bibliographystyle{ecs}
\bibliography{ECS}
\end{document}
%% ----------------------------------------------------------------

 %% ----------------------------------------------------------------
%% Progress.tex
%% ---------------------------------------------------------------- 
\documentclass{ecsprogress}    % Use the progress Style
\graphicspath{{../figs/}}   % Location of your graphics files
    \usepackage{natbib}            % Use Natbib style for the refs.
\hypersetup{colorlinks=true}   % Set to false for black/white printing
\input{Definitions}            % Include your abbreviations



\usepackage{enumitem}% http://ctan.org/pkg/enumitem
\usepackage{multirow}
\usepackage{float}
\usepackage{amsmath}
\usepackage{multicol}
\usepackage{amssymb}
\usepackage[normalem]{ulem}
\useunder{\uline}{\ul}{}
\usepackage{wrapfig}


\usepackage[table,xcdraw]{xcolor}


%% ----------------------------------------------------------------
\begin{document}
\frontmatter
\title      {Heterogeneous Agent-based Model for Supermarket Competition}
\authors    {\texorpdfstring
             {\href{mailto:sc22g13@ecs.soton.ac.uk}{Stefan J. Collier}}
             {Stefan J. Collier}
            }
\addresses  {\groupname\\\deptname\\\univname}
\date       {\today}
\subject    {}
\keywords   {}
\supervisor {Dr. Maria Polukarov}
\examiner   {Professor Sheng Chen}

\maketitle
\begin{abstract}
This project aim was to model and analyse the effects of competitive pricing behaviors of grocery retailers on the British market. 

This was achieved by creating a multi-agent model, containing retailer and consumer agents. The heterogeneous crowd of retailers employs either a uniform pricing strategy or a ‘local price flexing’ strategy. The actions of these retailers are chosen by predicting the profit of each action, using a perceptron. Following on from the consideration of different economic models, a discrete model was developed so that software agents have a discrete environment to operate within. Within the model, it has been observed how supermarkets with differing behaviors affect a heterogeneous crowd of consumer agents. The model was implemented in Java with Python used to evaluate the results. 

The simulation displays good acceptance with real grocery market behavior, i.e. captures the performance of British retailers thus can be used to determine the impact of changes in their behavior on their competitors and consumers.Furthermore it can be used to provide insight into sustainability of volatile pricing strategies, providing a useful insight in volatility of British supermarket retail industry. 
\end{abstract}
\acknowledgements{
I would like to express my sincere gratitude to Dr Maria Polukarov for her guidance and support which provided me the freedom to take this research in the direction of my interest.\\
\\
I would also like to thank my family and friends for their encouragement and support. To those who quietly listened to my software complaints. To those who worked throughout the nights with me. To those who helped me write what I couldn't say. I cannot thank you enough.
}

\declaration{
I, Stefan Collier, declare that this dissertation and the work presented in it are my own and has been generated by me as the result of my own original research.\\
I confirm that:\\
1. This work was done wholly or mainly while in candidature for a degree at this University;\\
2. Where any part of this dissertation has previously been submitted for any other qualification at this University or any other institution, this has been clearly stated;\\
3. Where I have consulted the published work of others, this is always clearly attributed;\\
4. Where I have quoted from the work of others, the source is always given. With the exception of such quotations, this dissertation is entirely my own work;\\
5. I have acknowledged all main sources of help;\\
6. Where the thesis is based on work done by myself jointly with others, I have made clear exactly what was done by others and what I have contributed myself;\\
7. Either none of this work has been published before submission, or parts of this work have been published by :\\
\\
Stefan Collier\\
April 2016
}
\tableofcontents
\listoffigures
\listoftables

\mainmatter
%% ----------------------------------------------------------------
%\include{Introduction}
%\include{Conclusions}
\include{chapters/1Project/main}
\include{chapters/2Lit/main}
\include{chapters/3Design/HighLevel}
\include{chapters/3Design/InDepth}
\include{chapters/4Impl/main}

\include{chapters/5Experiments/1/main}
\include{chapters/5Experiments/2/main}
\include{chapters/5Experiments/3/main}
\include{chapters/5Experiments/4/main}

\include{chapters/6Conclusion/main}

\appendix
\include{appendix/AppendixB}
\include{appendix/D/main}
\include{appendix/AppendixC}

\backmatter
\bibliographystyle{ecs}
\bibliography{ECS}
\end{document}
%% ----------------------------------------------------------------

 %% ----------------------------------------------------------------
%% Progress.tex
%% ---------------------------------------------------------------- 
\documentclass{ecsprogress}    % Use the progress Style
\graphicspath{{../figs/}}   % Location of your graphics files
    \usepackage{natbib}            % Use Natbib style for the refs.
\hypersetup{colorlinks=true}   % Set to false for black/white printing
\input{Definitions}            % Include your abbreviations



\usepackage{enumitem}% http://ctan.org/pkg/enumitem
\usepackage{multirow}
\usepackage{float}
\usepackage{amsmath}
\usepackage{multicol}
\usepackage{amssymb}
\usepackage[normalem]{ulem}
\useunder{\uline}{\ul}{}
\usepackage{wrapfig}


\usepackage[table,xcdraw]{xcolor}


%% ----------------------------------------------------------------
\begin{document}
\frontmatter
\title      {Heterogeneous Agent-based Model for Supermarket Competition}
\authors    {\texorpdfstring
             {\href{mailto:sc22g13@ecs.soton.ac.uk}{Stefan J. Collier}}
             {Stefan J. Collier}
            }
\addresses  {\groupname\\\deptname\\\univname}
\date       {\today}
\subject    {}
\keywords   {}
\supervisor {Dr. Maria Polukarov}
\examiner   {Professor Sheng Chen}

\maketitle
\begin{abstract}
This project aim was to model and analyse the effects of competitive pricing behaviors of grocery retailers on the British market. 

This was achieved by creating a multi-agent model, containing retailer and consumer agents. The heterogeneous crowd of retailers employs either a uniform pricing strategy or a ‘local price flexing’ strategy. The actions of these retailers are chosen by predicting the profit of each action, using a perceptron. Following on from the consideration of different economic models, a discrete model was developed so that software agents have a discrete environment to operate within. Within the model, it has been observed how supermarkets with differing behaviors affect a heterogeneous crowd of consumer agents. The model was implemented in Java with Python used to evaluate the results. 

The simulation displays good acceptance with real grocery market behavior, i.e. captures the performance of British retailers thus can be used to determine the impact of changes in their behavior on their competitors and consumers.Furthermore it can be used to provide insight into sustainability of volatile pricing strategies, providing a useful insight in volatility of British supermarket retail industry. 
\end{abstract}
\acknowledgements{
I would like to express my sincere gratitude to Dr Maria Polukarov for her guidance and support which provided me the freedom to take this research in the direction of my interest.\\
\\
I would also like to thank my family and friends for their encouragement and support. To those who quietly listened to my software complaints. To those who worked throughout the nights with me. To those who helped me write what I couldn't say. I cannot thank you enough.
}

\declaration{
I, Stefan Collier, declare that this dissertation and the work presented in it are my own and has been generated by me as the result of my own original research.\\
I confirm that:\\
1. This work was done wholly or mainly while in candidature for a degree at this University;\\
2. Where any part of this dissertation has previously been submitted for any other qualification at this University or any other institution, this has been clearly stated;\\
3. Where I have consulted the published work of others, this is always clearly attributed;\\
4. Where I have quoted from the work of others, the source is always given. With the exception of such quotations, this dissertation is entirely my own work;\\
5. I have acknowledged all main sources of help;\\
6. Where the thesis is based on work done by myself jointly with others, I have made clear exactly what was done by others and what I have contributed myself;\\
7. Either none of this work has been published before submission, or parts of this work have been published by :\\
\\
Stefan Collier\\
April 2016
}
\tableofcontents
\listoffigures
\listoftables

\mainmatter
%% ----------------------------------------------------------------
%\include{Introduction}
%\include{Conclusions}
\include{chapters/1Project/main}
\include{chapters/2Lit/main}
\include{chapters/3Design/HighLevel}
\include{chapters/3Design/InDepth}
\include{chapters/4Impl/main}

\include{chapters/5Experiments/1/main}
\include{chapters/5Experiments/2/main}
\include{chapters/5Experiments/3/main}
\include{chapters/5Experiments/4/main}

\include{chapters/6Conclusion/main}

\appendix
\include{appendix/AppendixB}
\include{appendix/D/main}
\include{appendix/AppendixC}

\backmatter
\bibliographystyle{ecs}
\bibliography{ECS}
\end{document}
%% ----------------------------------------------------------------


 %% ----------------------------------------------------------------
%% Progress.tex
%% ---------------------------------------------------------------- 
\documentclass{ecsprogress}    % Use the progress Style
\graphicspath{{../figs/}}   % Location of your graphics files
    \usepackage{natbib}            % Use Natbib style for the refs.
\hypersetup{colorlinks=true}   % Set to false for black/white printing
\input{Definitions}            % Include your abbreviations



\usepackage{enumitem}% http://ctan.org/pkg/enumitem
\usepackage{multirow}
\usepackage{float}
\usepackage{amsmath}
\usepackage{multicol}
\usepackage{amssymb}
\usepackage[normalem]{ulem}
\useunder{\uline}{\ul}{}
\usepackage{wrapfig}


\usepackage[table,xcdraw]{xcolor}


%% ----------------------------------------------------------------
\begin{document}
\frontmatter
\title      {Heterogeneous Agent-based Model for Supermarket Competition}
\authors    {\texorpdfstring
             {\href{mailto:sc22g13@ecs.soton.ac.uk}{Stefan J. Collier}}
             {Stefan J. Collier}
            }
\addresses  {\groupname\\\deptname\\\univname}
\date       {\today}
\subject    {}
\keywords   {}
\supervisor {Dr. Maria Polukarov}
\examiner   {Professor Sheng Chen}

\maketitle
\begin{abstract}
This project aim was to model and analyse the effects of competitive pricing behaviors of grocery retailers on the British market. 

This was achieved by creating a multi-agent model, containing retailer and consumer agents. The heterogeneous crowd of retailers employs either a uniform pricing strategy or a ‘local price flexing’ strategy. The actions of these retailers are chosen by predicting the profit of each action, using a perceptron. Following on from the consideration of different economic models, a discrete model was developed so that software agents have a discrete environment to operate within. Within the model, it has been observed how supermarkets with differing behaviors affect a heterogeneous crowd of consumer agents. The model was implemented in Java with Python used to evaluate the results. 

The simulation displays good acceptance with real grocery market behavior, i.e. captures the performance of British retailers thus can be used to determine the impact of changes in their behavior on their competitors and consumers.Furthermore it can be used to provide insight into sustainability of volatile pricing strategies, providing a useful insight in volatility of British supermarket retail industry. 
\end{abstract}
\acknowledgements{
I would like to express my sincere gratitude to Dr Maria Polukarov for her guidance and support which provided me the freedom to take this research in the direction of my interest.\\
\\
I would also like to thank my family and friends for their encouragement and support. To those who quietly listened to my software complaints. To those who worked throughout the nights with me. To those who helped me write what I couldn't say. I cannot thank you enough.
}

\declaration{
I, Stefan Collier, declare that this dissertation and the work presented in it are my own and has been generated by me as the result of my own original research.\\
I confirm that:\\
1. This work was done wholly or mainly while in candidature for a degree at this University;\\
2. Where any part of this dissertation has previously been submitted for any other qualification at this University or any other institution, this has been clearly stated;\\
3. Where I have consulted the published work of others, this is always clearly attributed;\\
4. Where I have quoted from the work of others, the source is always given. With the exception of such quotations, this dissertation is entirely my own work;\\
5. I have acknowledged all main sources of help;\\
6. Where the thesis is based on work done by myself jointly with others, I have made clear exactly what was done by others and what I have contributed myself;\\
7. Either none of this work has been published before submission, or parts of this work have been published by :\\
\\
Stefan Collier\\
April 2016
}
\tableofcontents
\listoffigures
\listoftables

\mainmatter
%% ----------------------------------------------------------------
%\include{Introduction}
%\include{Conclusions}
\include{chapters/1Project/main}
\include{chapters/2Lit/main}
\include{chapters/3Design/HighLevel}
\include{chapters/3Design/InDepth}
\include{chapters/4Impl/main}

\include{chapters/5Experiments/1/main}
\include{chapters/5Experiments/2/main}
\include{chapters/5Experiments/3/main}
\include{chapters/5Experiments/4/main}

\include{chapters/6Conclusion/main}

\appendix
\include{appendix/AppendixB}
\include{appendix/D/main}
\include{appendix/AppendixC}

\backmatter
\bibliographystyle{ecs}
\bibliography{ECS}
\end{document}
%% ----------------------------------------------------------------


\appendix
\include{appendix/AppendixB}
 %% ----------------------------------------------------------------
%% Progress.tex
%% ---------------------------------------------------------------- 
\documentclass{ecsprogress}    % Use the progress Style
\graphicspath{{../figs/}}   % Location of your graphics files
    \usepackage{natbib}            % Use Natbib style for the refs.
\hypersetup{colorlinks=true}   % Set to false for black/white printing
\input{Definitions}            % Include your abbreviations



\usepackage{enumitem}% http://ctan.org/pkg/enumitem
\usepackage{multirow}
\usepackage{float}
\usepackage{amsmath}
\usepackage{multicol}
\usepackage{amssymb}
\usepackage[normalem]{ulem}
\useunder{\uline}{\ul}{}
\usepackage{wrapfig}


\usepackage[table,xcdraw]{xcolor}


%% ----------------------------------------------------------------
\begin{document}
\frontmatter
\title      {Heterogeneous Agent-based Model for Supermarket Competition}
\authors    {\texorpdfstring
             {\href{mailto:sc22g13@ecs.soton.ac.uk}{Stefan J. Collier}}
             {Stefan J. Collier}
            }
\addresses  {\groupname\\\deptname\\\univname}
\date       {\today}
\subject    {}
\keywords   {}
\supervisor {Dr. Maria Polukarov}
\examiner   {Professor Sheng Chen}

\maketitle
\begin{abstract}
This project aim was to model and analyse the effects of competitive pricing behaviors of grocery retailers on the British market. 

This was achieved by creating a multi-agent model, containing retailer and consumer agents. The heterogeneous crowd of retailers employs either a uniform pricing strategy or a ‘local price flexing’ strategy. The actions of these retailers are chosen by predicting the profit of each action, using a perceptron. Following on from the consideration of different economic models, a discrete model was developed so that software agents have a discrete environment to operate within. Within the model, it has been observed how supermarkets with differing behaviors affect a heterogeneous crowd of consumer agents. The model was implemented in Java with Python used to evaluate the results. 

The simulation displays good acceptance with real grocery market behavior, i.e. captures the performance of British retailers thus can be used to determine the impact of changes in their behavior on their competitors and consumers.Furthermore it can be used to provide insight into sustainability of volatile pricing strategies, providing a useful insight in volatility of British supermarket retail industry. 
\end{abstract}
\acknowledgements{
I would like to express my sincere gratitude to Dr Maria Polukarov for her guidance and support which provided me the freedom to take this research in the direction of my interest.\\
\\
I would also like to thank my family and friends for their encouragement and support. To those who quietly listened to my software complaints. To those who worked throughout the nights with me. To those who helped me write what I couldn't say. I cannot thank you enough.
}

\declaration{
I, Stefan Collier, declare that this dissertation and the work presented in it are my own and has been generated by me as the result of my own original research.\\
I confirm that:\\
1. This work was done wholly or mainly while in candidature for a degree at this University;\\
2. Where any part of this dissertation has previously been submitted for any other qualification at this University or any other institution, this has been clearly stated;\\
3. Where I have consulted the published work of others, this is always clearly attributed;\\
4. Where I have quoted from the work of others, the source is always given. With the exception of such quotations, this dissertation is entirely my own work;\\
5. I have acknowledged all main sources of help;\\
6. Where the thesis is based on work done by myself jointly with others, I have made clear exactly what was done by others and what I have contributed myself;\\
7. Either none of this work has been published before submission, or parts of this work have been published by :\\
\\
Stefan Collier\\
April 2016
}
\tableofcontents
\listoffigures
\listoftables

\mainmatter
%% ----------------------------------------------------------------
%\include{Introduction}
%\include{Conclusions}
\include{chapters/1Project/main}
\include{chapters/2Lit/main}
\include{chapters/3Design/HighLevel}
\include{chapters/3Design/InDepth}
\include{chapters/4Impl/main}

\include{chapters/5Experiments/1/main}
\include{chapters/5Experiments/2/main}
\include{chapters/5Experiments/3/main}
\include{chapters/5Experiments/4/main}

\include{chapters/6Conclusion/main}

\appendix
\include{appendix/AppendixB}
\include{appendix/D/main}
\include{appendix/AppendixC}

\backmatter
\bibliographystyle{ecs}
\bibliography{ECS}
\end{document}
%% ----------------------------------------------------------------

\include{appendix/AppendixC}

\backmatter
\bibliographystyle{ecs}
\bibliography{ECS}
\end{document}
%% ----------------------------------------------------------------

\include{appendix/AppendixC}

\backmatter
\bibliographystyle{ecs}
\bibliography{ECS}
\end{document}
%% ----------------------------------------------------------------

\include{chapters/3Design/HighLevel}
\include{chapters/3Design/InDepth}
 %% ----------------------------------------------------------------
%% Progress.tex
%% ---------------------------------------------------------------- 
\documentclass{ecsprogress}    % Use the progress Style
\graphicspath{{../figs/}}   % Location of your graphics files
    \usepackage{natbib}            % Use Natbib style for the refs.
\hypersetup{colorlinks=true}   % Set to false for black/white printing
\input{Definitions}            % Include your abbreviations



\usepackage{enumitem}% http://ctan.org/pkg/enumitem
\usepackage{multirow}
\usepackage{float}
\usepackage{amsmath}
\usepackage{multicol}
\usepackage{amssymb}
\usepackage[normalem]{ulem}
\useunder{\uline}{\ul}{}
\usepackage{wrapfig}


\usepackage[table,xcdraw]{xcolor}


%% ----------------------------------------------------------------
\begin{document}
\frontmatter
\title      {Heterogeneous Agent-based Model for Supermarket Competition}
\authors    {\texorpdfstring
             {\href{mailto:sc22g13@ecs.soton.ac.uk}{Stefan J. Collier}}
             {Stefan J. Collier}
            }
\addresses  {\groupname\\\deptname\\\univname}
\date       {\today}
\subject    {}
\keywords   {}
\supervisor {Dr. Maria Polukarov}
\examiner   {Professor Sheng Chen}

\maketitle
\begin{abstract}
This project aim was to model and analyse the effects of competitive pricing behaviors of grocery retailers on the British market. 

This was achieved by creating a multi-agent model, containing retailer and consumer agents. The heterogeneous crowd of retailers employs either a uniform pricing strategy or a ‘local price flexing’ strategy. The actions of these retailers are chosen by predicting the profit of each action, using a perceptron. Following on from the consideration of different economic models, a discrete model was developed so that software agents have a discrete environment to operate within. Within the model, it has been observed how supermarkets with differing behaviors affect a heterogeneous crowd of consumer agents. The model was implemented in Java with Python used to evaluate the results. 

The simulation displays good acceptance with real grocery market behavior, i.e. captures the performance of British retailers thus can be used to determine the impact of changes in their behavior on their competitors and consumers.Furthermore it can be used to provide insight into sustainability of volatile pricing strategies, providing a useful insight in volatility of British supermarket retail industry. 
\end{abstract}
\acknowledgements{
I would like to express my sincere gratitude to Dr Maria Polukarov for her guidance and support which provided me the freedom to take this research in the direction of my interest.\\
\\
I would also like to thank my family and friends for their encouragement and support. To those who quietly listened to my software complaints. To those who worked throughout the nights with me. To those who helped me write what I couldn't say. I cannot thank you enough.
}

\declaration{
I, Stefan Collier, declare that this dissertation and the work presented in it are my own and has been generated by me as the result of my own original research.\\
I confirm that:\\
1. This work was done wholly or mainly while in candidature for a degree at this University;\\
2. Where any part of this dissertation has previously been submitted for any other qualification at this University or any other institution, this has been clearly stated;\\
3. Where I have consulted the published work of others, this is always clearly attributed;\\
4. Where I have quoted from the work of others, the source is always given. With the exception of such quotations, this dissertation is entirely my own work;\\
5. I have acknowledged all main sources of help;\\
6. Where the thesis is based on work done by myself jointly with others, I have made clear exactly what was done by others and what I have contributed myself;\\
7. Either none of this work has been published before submission, or parts of this work have been published by :\\
\\
Stefan Collier\\
April 2016
}
\tableofcontents
\listoffigures
\listoftables

\mainmatter
%% ----------------------------------------------------------------
%\include{Introduction}
%\include{Conclusions}
 %% ----------------------------------------------------------------
%% Progress.tex
%% ---------------------------------------------------------------- 
\documentclass{ecsprogress}    % Use the progress Style
\graphicspath{{../figs/}}   % Location of your graphics files
    \usepackage{natbib}            % Use Natbib style for the refs.
\hypersetup{colorlinks=true}   % Set to false for black/white printing
\input{Definitions}            % Include your abbreviations



\usepackage{enumitem}% http://ctan.org/pkg/enumitem
\usepackage{multirow}
\usepackage{float}
\usepackage{amsmath}
\usepackage{multicol}
\usepackage{amssymb}
\usepackage[normalem]{ulem}
\useunder{\uline}{\ul}{}
\usepackage{wrapfig}


\usepackage[table,xcdraw]{xcolor}


%% ----------------------------------------------------------------
\begin{document}
\frontmatter
\title      {Heterogeneous Agent-based Model for Supermarket Competition}
\authors    {\texorpdfstring
             {\href{mailto:sc22g13@ecs.soton.ac.uk}{Stefan J. Collier}}
             {Stefan J. Collier}
            }
\addresses  {\groupname\\\deptname\\\univname}
\date       {\today}
\subject    {}
\keywords   {}
\supervisor {Dr. Maria Polukarov}
\examiner   {Professor Sheng Chen}

\maketitle
\begin{abstract}
This project aim was to model and analyse the effects of competitive pricing behaviors of grocery retailers on the British market. 

This was achieved by creating a multi-agent model, containing retailer and consumer agents. The heterogeneous crowd of retailers employs either a uniform pricing strategy or a ‘local price flexing’ strategy. The actions of these retailers are chosen by predicting the profit of each action, using a perceptron. Following on from the consideration of different economic models, a discrete model was developed so that software agents have a discrete environment to operate within. Within the model, it has been observed how supermarkets with differing behaviors affect a heterogeneous crowd of consumer agents. The model was implemented in Java with Python used to evaluate the results. 

The simulation displays good acceptance with real grocery market behavior, i.e. captures the performance of British retailers thus can be used to determine the impact of changes in their behavior on their competitors and consumers.Furthermore it can be used to provide insight into sustainability of volatile pricing strategies, providing a useful insight in volatility of British supermarket retail industry. 
\end{abstract}
\acknowledgements{
I would like to express my sincere gratitude to Dr Maria Polukarov for her guidance and support which provided me the freedom to take this research in the direction of my interest.\\
\\
I would also like to thank my family and friends for their encouragement and support. To those who quietly listened to my software complaints. To those who worked throughout the nights with me. To those who helped me write what I couldn't say. I cannot thank you enough.
}

\declaration{
I, Stefan Collier, declare that this dissertation and the work presented in it are my own and has been generated by me as the result of my own original research.\\
I confirm that:\\
1. This work was done wholly or mainly while in candidature for a degree at this University;\\
2. Where any part of this dissertation has previously been submitted for any other qualification at this University or any other institution, this has been clearly stated;\\
3. Where I have consulted the published work of others, this is always clearly attributed;\\
4. Where I have quoted from the work of others, the source is always given. With the exception of such quotations, this dissertation is entirely my own work;\\
5. I have acknowledged all main sources of help;\\
6. Where the thesis is based on work done by myself jointly with others, I have made clear exactly what was done by others and what I have contributed myself;\\
7. Either none of this work has been published before submission, or parts of this work have been published by :\\
\\
Stefan Collier\\
April 2016
}
\tableofcontents
\listoffigures
\listoftables

\mainmatter
%% ----------------------------------------------------------------
%\include{Introduction}
%\include{Conclusions}
 %% ----------------------------------------------------------------
%% Progress.tex
%% ---------------------------------------------------------------- 
\documentclass{ecsprogress}    % Use the progress Style
\graphicspath{{../figs/}}   % Location of your graphics files
    \usepackage{natbib}            % Use Natbib style for the refs.
\hypersetup{colorlinks=true}   % Set to false for black/white printing
\input{Definitions}            % Include your abbreviations



\usepackage{enumitem}% http://ctan.org/pkg/enumitem
\usepackage{multirow}
\usepackage{float}
\usepackage{amsmath}
\usepackage{multicol}
\usepackage{amssymb}
\usepackage[normalem]{ulem}
\useunder{\uline}{\ul}{}
\usepackage{wrapfig}


\usepackage[table,xcdraw]{xcolor}


%% ----------------------------------------------------------------
\begin{document}
\frontmatter
\title      {Heterogeneous Agent-based Model for Supermarket Competition}
\authors    {\texorpdfstring
             {\href{mailto:sc22g13@ecs.soton.ac.uk}{Stefan J. Collier}}
             {Stefan J. Collier}
            }
\addresses  {\groupname\\\deptname\\\univname}
\date       {\today}
\subject    {}
\keywords   {}
\supervisor {Dr. Maria Polukarov}
\examiner   {Professor Sheng Chen}

\maketitle
\begin{abstract}
This project aim was to model and analyse the effects of competitive pricing behaviors of grocery retailers on the British market. 

This was achieved by creating a multi-agent model, containing retailer and consumer agents. The heterogeneous crowd of retailers employs either a uniform pricing strategy or a ‘local price flexing’ strategy. The actions of these retailers are chosen by predicting the profit of each action, using a perceptron. Following on from the consideration of different economic models, a discrete model was developed so that software agents have a discrete environment to operate within. Within the model, it has been observed how supermarkets with differing behaviors affect a heterogeneous crowd of consumer agents. The model was implemented in Java with Python used to evaluate the results. 

The simulation displays good acceptance with real grocery market behavior, i.e. captures the performance of British retailers thus can be used to determine the impact of changes in their behavior on their competitors and consumers.Furthermore it can be used to provide insight into sustainability of volatile pricing strategies, providing a useful insight in volatility of British supermarket retail industry. 
\end{abstract}
\acknowledgements{
I would like to express my sincere gratitude to Dr Maria Polukarov for her guidance and support which provided me the freedom to take this research in the direction of my interest.\\
\\
I would also like to thank my family and friends for their encouragement and support. To those who quietly listened to my software complaints. To those who worked throughout the nights with me. To those who helped me write what I couldn't say. I cannot thank you enough.
}

\declaration{
I, Stefan Collier, declare that this dissertation and the work presented in it are my own and has been generated by me as the result of my own original research.\\
I confirm that:\\
1. This work was done wholly or mainly while in candidature for a degree at this University;\\
2. Where any part of this dissertation has previously been submitted for any other qualification at this University or any other institution, this has been clearly stated;\\
3. Where I have consulted the published work of others, this is always clearly attributed;\\
4. Where I have quoted from the work of others, the source is always given. With the exception of such quotations, this dissertation is entirely my own work;\\
5. I have acknowledged all main sources of help;\\
6. Where the thesis is based on work done by myself jointly with others, I have made clear exactly what was done by others and what I have contributed myself;\\
7. Either none of this work has been published before submission, or parts of this work have been published by :\\
\\
Stefan Collier\\
April 2016
}
\tableofcontents
\listoffigures
\listoftables

\mainmatter
%% ----------------------------------------------------------------
%\include{Introduction}
%\include{Conclusions}
\include{chapters/1Project/main}
\include{chapters/2Lit/main}
\include{chapters/3Design/HighLevel}
\include{chapters/3Design/InDepth}
\include{chapters/4Impl/main}

\include{chapters/5Experiments/1/main}
\include{chapters/5Experiments/2/main}
\include{chapters/5Experiments/3/main}
\include{chapters/5Experiments/4/main}

\include{chapters/6Conclusion/main}

\appendix
\include{appendix/AppendixB}
\include{appendix/D/main}
\include{appendix/AppendixC}

\backmatter
\bibliographystyle{ecs}
\bibliography{ECS}
\end{document}
%% ----------------------------------------------------------------

 %% ----------------------------------------------------------------
%% Progress.tex
%% ---------------------------------------------------------------- 
\documentclass{ecsprogress}    % Use the progress Style
\graphicspath{{../figs/}}   % Location of your graphics files
    \usepackage{natbib}            % Use Natbib style for the refs.
\hypersetup{colorlinks=true}   % Set to false for black/white printing
\input{Definitions}            % Include your abbreviations



\usepackage{enumitem}% http://ctan.org/pkg/enumitem
\usepackage{multirow}
\usepackage{float}
\usepackage{amsmath}
\usepackage{multicol}
\usepackage{amssymb}
\usepackage[normalem]{ulem}
\useunder{\uline}{\ul}{}
\usepackage{wrapfig}


\usepackage[table,xcdraw]{xcolor}


%% ----------------------------------------------------------------
\begin{document}
\frontmatter
\title      {Heterogeneous Agent-based Model for Supermarket Competition}
\authors    {\texorpdfstring
             {\href{mailto:sc22g13@ecs.soton.ac.uk}{Stefan J. Collier}}
             {Stefan J. Collier}
            }
\addresses  {\groupname\\\deptname\\\univname}
\date       {\today}
\subject    {}
\keywords   {}
\supervisor {Dr. Maria Polukarov}
\examiner   {Professor Sheng Chen}

\maketitle
\begin{abstract}
This project aim was to model and analyse the effects of competitive pricing behaviors of grocery retailers on the British market. 

This was achieved by creating a multi-agent model, containing retailer and consumer agents. The heterogeneous crowd of retailers employs either a uniform pricing strategy or a ‘local price flexing’ strategy. The actions of these retailers are chosen by predicting the profit of each action, using a perceptron. Following on from the consideration of different economic models, a discrete model was developed so that software agents have a discrete environment to operate within. Within the model, it has been observed how supermarkets with differing behaviors affect a heterogeneous crowd of consumer agents. The model was implemented in Java with Python used to evaluate the results. 

The simulation displays good acceptance with real grocery market behavior, i.e. captures the performance of British retailers thus can be used to determine the impact of changes in their behavior on their competitors and consumers.Furthermore it can be used to provide insight into sustainability of volatile pricing strategies, providing a useful insight in volatility of British supermarket retail industry. 
\end{abstract}
\acknowledgements{
I would like to express my sincere gratitude to Dr Maria Polukarov for her guidance and support which provided me the freedom to take this research in the direction of my interest.\\
\\
I would also like to thank my family and friends for their encouragement and support. To those who quietly listened to my software complaints. To those who worked throughout the nights with me. To those who helped me write what I couldn't say. I cannot thank you enough.
}

\declaration{
I, Stefan Collier, declare that this dissertation and the work presented in it are my own and has been generated by me as the result of my own original research.\\
I confirm that:\\
1. This work was done wholly or mainly while in candidature for a degree at this University;\\
2. Where any part of this dissertation has previously been submitted for any other qualification at this University or any other institution, this has been clearly stated;\\
3. Where I have consulted the published work of others, this is always clearly attributed;\\
4. Where I have quoted from the work of others, the source is always given. With the exception of such quotations, this dissertation is entirely my own work;\\
5. I have acknowledged all main sources of help;\\
6. Where the thesis is based on work done by myself jointly with others, I have made clear exactly what was done by others and what I have contributed myself;\\
7. Either none of this work has been published before submission, or parts of this work have been published by :\\
\\
Stefan Collier\\
April 2016
}
\tableofcontents
\listoffigures
\listoftables

\mainmatter
%% ----------------------------------------------------------------
%\include{Introduction}
%\include{Conclusions}
\include{chapters/1Project/main}
\include{chapters/2Lit/main}
\include{chapters/3Design/HighLevel}
\include{chapters/3Design/InDepth}
\include{chapters/4Impl/main}

\include{chapters/5Experiments/1/main}
\include{chapters/5Experiments/2/main}
\include{chapters/5Experiments/3/main}
\include{chapters/5Experiments/4/main}

\include{chapters/6Conclusion/main}

\appendix
\include{appendix/AppendixB}
\include{appendix/D/main}
\include{appendix/AppendixC}

\backmatter
\bibliographystyle{ecs}
\bibliography{ECS}
\end{document}
%% ----------------------------------------------------------------

\include{chapters/3Design/HighLevel}
\include{chapters/3Design/InDepth}
 %% ----------------------------------------------------------------
%% Progress.tex
%% ---------------------------------------------------------------- 
\documentclass{ecsprogress}    % Use the progress Style
\graphicspath{{../figs/}}   % Location of your graphics files
    \usepackage{natbib}            % Use Natbib style for the refs.
\hypersetup{colorlinks=true}   % Set to false for black/white printing
\input{Definitions}            % Include your abbreviations



\usepackage{enumitem}% http://ctan.org/pkg/enumitem
\usepackage{multirow}
\usepackage{float}
\usepackage{amsmath}
\usepackage{multicol}
\usepackage{amssymb}
\usepackage[normalem]{ulem}
\useunder{\uline}{\ul}{}
\usepackage{wrapfig}


\usepackage[table,xcdraw]{xcolor}


%% ----------------------------------------------------------------
\begin{document}
\frontmatter
\title      {Heterogeneous Agent-based Model for Supermarket Competition}
\authors    {\texorpdfstring
             {\href{mailto:sc22g13@ecs.soton.ac.uk}{Stefan J. Collier}}
             {Stefan J. Collier}
            }
\addresses  {\groupname\\\deptname\\\univname}
\date       {\today}
\subject    {}
\keywords   {}
\supervisor {Dr. Maria Polukarov}
\examiner   {Professor Sheng Chen}

\maketitle
\begin{abstract}
This project aim was to model and analyse the effects of competitive pricing behaviors of grocery retailers on the British market. 

This was achieved by creating a multi-agent model, containing retailer and consumer agents. The heterogeneous crowd of retailers employs either a uniform pricing strategy or a ‘local price flexing’ strategy. The actions of these retailers are chosen by predicting the profit of each action, using a perceptron. Following on from the consideration of different economic models, a discrete model was developed so that software agents have a discrete environment to operate within. Within the model, it has been observed how supermarkets with differing behaviors affect a heterogeneous crowd of consumer agents. The model was implemented in Java with Python used to evaluate the results. 

The simulation displays good acceptance with real grocery market behavior, i.e. captures the performance of British retailers thus can be used to determine the impact of changes in their behavior on their competitors and consumers.Furthermore it can be used to provide insight into sustainability of volatile pricing strategies, providing a useful insight in volatility of British supermarket retail industry. 
\end{abstract}
\acknowledgements{
I would like to express my sincere gratitude to Dr Maria Polukarov for her guidance and support which provided me the freedom to take this research in the direction of my interest.\\
\\
I would also like to thank my family and friends for their encouragement and support. To those who quietly listened to my software complaints. To those who worked throughout the nights with me. To those who helped me write what I couldn't say. I cannot thank you enough.
}

\declaration{
I, Stefan Collier, declare that this dissertation and the work presented in it are my own and has been generated by me as the result of my own original research.\\
I confirm that:\\
1. This work was done wholly or mainly while in candidature for a degree at this University;\\
2. Where any part of this dissertation has previously been submitted for any other qualification at this University or any other institution, this has been clearly stated;\\
3. Where I have consulted the published work of others, this is always clearly attributed;\\
4. Where I have quoted from the work of others, the source is always given. With the exception of such quotations, this dissertation is entirely my own work;\\
5. I have acknowledged all main sources of help;\\
6. Where the thesis is based on work done by myself jointly with others, I have made clear exactly what was done by others and what I have contributed myself;\\
7. Either none of this work has been published before submission, or parts of this work have been published by :\\
\\
Stefan Collier\\
April 2016
}
\tableofcontents
\listoffigures
\listoftables

\mainmatter
%% ----------------------------------------------------------------
%\include{Introduction}
%\include{Conclusions}
\include{chapters/1Project/main}
\include{chapters/2Lit/main}
\include{chapters/3Design/HighLevel}
\include{chapters/3Design/InDepth}
\include{chapters/4Impl/main}

\include{chapters/5Experiments/1/main}
\include{chapters/5Experiments/2/main}
\include{chapters/5Experiments/3/main}
\include{chapters/5Experiments/4/main}

\include{chapters/6Conclusion/main}

\appendix
\include{appendix/AppendixB}
\include{appendix/D/main}
\include{appendix/AppendixC}

\backmatter
\bibliographystyle{ecs}
\bibliography{ECS}
\end{document}
%% ----------------------------------------------------------------


 %% ----------------------------------------------------------------
%% Progress.tex
%% ---------------------------------------------------------------- 
\documentclass{ecsprogress}    % Use the progress Style
\graphicspath{{../figs/}}   % Location of your graphics files
    \usepackage{natbib}            % Use Natbib style for the refs.
\hypersetup{colorlinks=true}   % Set to false for black/white printing
\input{Definitions}            % Include your abbreviations



\usepackage{enumitem}% http://ctan.org/pkg/enumitem
\usepackage{multirow}
\usepackage{float}
\usepackage{amsmath}
\usepackage{multicol}
\usepackage{amssymb}
\usepackage[normalem]{ulem}
\useunder{\uline}{\ul}{}
\usepackage{wrapfig}


\usepackage[table,xcdraw]{xcolor}


%% ----------------------------------------------------------------
\begin{document}
\frontmatter
\title      {Heterogeneous Agent-based Model for Supermarket Competition}
\authors    {\texorpdfstring
             {\href{mailto:sc22g13@ecs.soton.ac.uk}{Stefan J. Collier}}
             {Stefan J. Collier}
            }
\addresses  {\groupname\\\deptname\\\univname}
\date       {\today}
\subject    {}
\keywords   {}
\supervisor {Dr. Maria Polukarov}
\examiner   {Professor Sheng Chen}

\maketitle
\begin{abstract}
This project aim was to model and analyse the effects of competitive pricing behaviors of grocery retailers on the British market. 

This was achieved by creating a multi-agent model, containing retailer and consumer agents. The heterogeneous crowd of retailers employs either a uniform pricing strategy or a ‘local price flexing’ strategy. The actions of these retailers are chosen by predicting the profit of each action, using a perceptron. Following on from the consideration of different economic models, a discrete model was developed so that software agents have a discrete environment to operate within. Within the model, it has been observed how supermarkets with differing behaviors affect a heterogeneous crowd of consumer agents. The model was implemented in Java with Python used to evaluate the results. 

The simulation displays good acceptance with real grocery market behavior, i.e. captures the performance of British retailers thus can be used to determine the impact of changes in their behavior on their competitors and consumers.Furthermore it can be used to provide insight into sustainability of volatile pricing strategies, providing a useful insight in volatility of British supermarket retail industry. 
\end{abstract}
\acknowledgements{
I would like to express my sincere gratitude to Dr Maria Polukarov for her guidance and support which provided me the freedom to take this research in the direction of my interest.\\
\\
I would also like to thank my family and friends for their encouragement and support. To those who quietly listened to my software complaints. To those who worked throughout the nights with me. To those who helped me write what I couldn't say. I cannot thank you enough.
}

\declaration{
I, Stefan Collier, declare that this dissertation and the work presented in it are my own and has been generated by me as the result of my own original research.\\
I confirm that:\\
1. This work was done wholly or mainly while in candidature for a degree at this University;\\
2. Where any part of this dissertation has previously been submitted for any other qualification at this University or any other institution, this has been clearly stated;\\
3. Where I have consulted the published work of others, this is always clearly attributed;\\
4. Where I have quoted from the work of others, the source is always given. With the exception of such quotations, this dissertation is entirely my own work;\\
5. I have acknowledged all main sources of help;\\
6. Where the thesis is based on work done by myself jointly with others, I have made clear exactly what was done by others and what I have contributed myself;\\
7. Either none of this work has been published before submission, or parts of this work have been published by :\\
\\
Stefan Collier\\
April 2016
}
\tableofcontents
\listoffigures
\listoftables

\mainmatter
%% ----------------------------------------------------------------
%\include{Introduction}
%\include{Conclusions}
\include{chapters/1Project/main}
\include{chapters/2Lit/main}
\include{chapters/3Design/HighLevel}
\include{chapters/3Design/InDepth}
\include{chapters/4Impl/main}

\include{chapters/5Experiments/1/main}
\include{chapters/5Experiments/2/main}
\include{chapters/5Experiments/3/main}
\include{chapters/5Experiments/4/main}

\include{chapters/6Conclusion/main}

\appendix
\include{appendix/AppendixB}
\include{appendix/D/main}
\include{appendix/AppendixC}

\backmatter
\bibliographystyle{ecs}
\bibliography{ECS}
\end{document}
%% ----------------------------------------------------------------

 %% ----------------------------------------------------------------
%% Progress.tex
%% ---------------------------------------------------------------- 
\documentclass{ecsprogress}    % Use the progress Style
\graphicspath{{../figs/}}   % Location of your graphics files
    \usepackage{natbib}            % Use Natbib style for the refs.
\hypersetup{colorlinks=true}   % Set to false for black/white printing
\input{Definitions}            % Include your abbreviations



\usepackage{enumitem}% http://ctan.org/pkg/enumitem
\usepackage{multirow}
\usepackage{float}
\usepackage{amsmath}
\usepackage{multicol}
\usepackage{amssymb}
\usepackage[normalem]{ulem}
\useunder{\uline}{\ul}{}
\usepackage{wrapfig}


\usepackage[table,xcdraw]{xcolor}


%% ----------------------------------------------------------------
\begin{document}
\frontmatter
\title      {Heterogeneous Agent-based Model for Supermarket Competition}
\authors    {\texorpdfstring
             {\href{mailto:sc22g13@ecs.soton.ac.uk}{Stefan J. Collier}}
             {Stefan J. Collier}
            }
\addresses  {\groupname\\\deptname\\\univname}
\date       {\today}
\subject    {}
\keywords   {}
\supervisor {Dr. Maria Polukarov}
\examiner   {Professor Sheng Chen}

\maketitle
\begin{abstract}
This project aim was to model and analyse the effects of competitive pricing behaviors of grocery retailers on the British market. 

This was achieved by creating a multi-agent model, containing retailer and consumer agents. The heterogeneous crowd of retailers employs either a uniform pricing strategy or a ‘local price flexing’ strategy. The actions of these retailers are chosen by predicting the profit of each action, using a perceptron. Following on from the consideration of different economic models, a discrete model was developed so that software agents have a discrete environment to operate within. Within the model, it has been observed how supermarkets with differing behaviors affect a heterogeneous crowd of consumer agents. The model was implemented in Java with Python used to evaluate the results. 

The simulation displays good acceptance with real grocery market behavior, i.e. captures the performance of British retailers thus can be used to determine the impact of changes in their behavior on their competitors and consumers.Furthermore it can be used to provide insight into sustainability of volatile pricing strategies, providing a useful insight in volatility of British supermarket retail industry. 
\end{abstract}
\acknowledgements{
I would like to express my sincere gratitude to Dr Maria Polukarov for her guidance and support which provided me the freedom to take this research in the direction of my interest.\\
\\
I would also like to thank my family and friends for their encouragement and support. To those who quietly listened to my software complaints. To those who worked throughout the nights with me. To those who helped me write what I couldn't say. I cannot thank you enough.
}

\declaration{
I, Stefan Collier, declare that this dissertation and the work presented in it are my own and has been generated by me as the result of my own original research.\\
I confirm that:\\
1. This work was done wholly or mainly while in candidature for a degree at this University;\\
2. Where any part of this dissertation has previously been submitted for any other qualification at this University or any other institution, this has been clearly stated;\\
3. Where I have consulted the published work of others, this is always clearly attributed;\\
4. Where I have quoted from the work of others, the source is always given. With the exception of such quotations, this dissertation is entirely my own work;\\
5. I have acknowledged all main sources of help;\\
6. Where the thesis is based on work done by myself jointly with others, I have made clear exactly what was done by others and what I have contributed myself;\\
7. Either none of this work has been published before submission, or parts of this work have been published by :\\
\\
Stefan Collier\\
April 2016
}
\tableofcontents
\listoffigures
\listoftables

\mainmatter
%% ----------------------------------------------------------------
%\include{Introduction}
%\include{Conclusions}
\include{chapters/1Project/main}
\include{chapters/2Lit/main}
\include{chapters/3Design/HighLevel}
\include{chapters/3Design/InDepth}
\include{chapters/4Impl/main}

\include{chapters/5Experiments/1/main}
\include{chapters/5Experiments/2/main}
\include{chapters/5Experiments/3/main}
\include{chapters/5Experiments/4/main}

\include{chapters/6Conclusion/main}

\appendix
\include{appendix/AppendixB}
\include{appendix/D/main}
\include{appendix/AppendixC}

\backmatter
\bibliographystyle{ecs}
\bibliography{ECS}
\end{document}
%% ----------------------------------------------------------------

 %% ----------------------------------------------------------------
%% Progress.tex
%% ---------------------------------------------------------------- 
\documentclass{ecsprogress}    % Use the progress Style
\graphicspath{{../figs/}}   % Location of your graphics files
    \usepackage{natbib}            % Use Natbib style for the refs.
\hypersetup{colorlinks=true}   % Set to false for black/white printing
\input{Definitions}            % Include your abbreviations



\usepackage{enumitem}% http://ctan.org/pkg/enumitem
\usepackage{multirow}
\usepackage{float}
\usepackage{amsmath}
\usepackage{multicol}
\usepackage{amssymb}
\usepackage[normalem]{ulem}
\useunder{\uline}{\ul}{}
\usepackage{wrapfig}


\usepackage[table,xcdraw]{xcolor}


%% ----------------------------------------------------------------
\begin{document}
\frontmatter
\title      {Heterogeneous Agent-based Model for Supermarket Competition}
\authors    {\texorpdfstring
             {\href{mailto:sc22g13@ecs.soton.ac.uk}{Stefan J. Collier}}
             {Stefan J. Collier}
            }
\addresses  {\groupname\\\deptname\\\univname}
\date       {\today}
\subject    {}
\keywords   {}
\supervisor {Dr. Maria Polukarov}
\examiner   {Professor Sheng Chen}

\maketitle
\begin{abstract}
This project aim was to model and analyse the effects of competitive pricing behaviors of grocery retailers on the British market. 

This was achieved by creating a multi-agent model, containing retailer and consumer agents. The heterogeneous crowd of retailers employs either a uniform pricing strategy or a ‘local price flexing’ strategy. The actions of these retailers are chosen by predicting the profit of each action, using a perceptron. Following on from the consideration of different economic models, a discrete model was developed so that software agents have a discrete environment to operate within. Within the model, it has been observed how supermarkets with differing behaviors affect a heterogeneous crowd of consumer agents. The model was implemented in Java with Python used to evaluate the results. 

The simulation displays good acceptance with real grocery market behavior, i.e. captures the performance of British retailers thus can be used to determine the impact of changes in their behavior on their competitors and consumers.Furthermore it can be used to provide insight into sustainability of volatile pricing strategies, providing a useful insight in volatility of British supermarket retail industry. 
\end{abstract}
\acknowledgements{
I would like to express my sincere gratitude to Dr Maria Polukarov for her guidance and support which provided me the freedom to take this research in the direction of my interest.\\
\\
I would also like to thank my family and friends for their encouragement and support. To those who quietly listened to my software complaints. To those who worked throughout the nights with me. To those who helped me write what I couldn't say. I cannot thank you enough.
}

\declaration{
I, Stefan Collier, declare that this dissertation and the work presented in it are my own and has been generated by me as the result of my own original research.\\
I confirm that:\\
1. This work was done wholly or mainly while in candidature for a degree at this University;\\
2. Where any part of this dissertation has previously been submitted for any other qualification at this University or any other institution, this has been clearly stated;\\
3. Where I have consulted the published work of others, this is always clearly attributed;\\
4. Where I have quoted from the work of others, the source is always given. With the exception of such quotations, this dissertation is entirely my own work;\\
5. I have acknowledged all main sources of help;\\
6. Where the thesis is based on work done by myself jointly with others, I have made clear exactly what was done by others and what I have contributed myself;\\
7. Either none of this work has been published before submission, or parts of this work have been published by :\\
\\
Stefan Collier\\
April 2016
}
\tableofcontents
\listoffigures
\listoftables

\mainmatter
%% ----------------------------------------------------------------
%\include{Introduction}
%\include{Conclusions}
\include{chapters/1Project/main}
\include{chapters/2Lit/main}
\include{chapters/3Design/HighLevel}
\include{chapters/3Design/InDepth}
\include{chapters/4Impl/main}

\include{chapters/5Experiments/1/main}
\include{chapters/5Experiments/2/main}
\include{chapters/5Experiments/3/main}
\include{chapters/5Experiments/4/main}

\include{chapters/6Conclusion/main}

\appendix
\include{appendix/AppendixB}
\include{appendix/D/main}
\include{appendix/AppendixC}

\backmatter
\bibliographystyle{ecs}
\bibliography{ECS}
\end{document}
%% ----------------------------------------------------------------

 %% ----------------------------------------------------------------
%% Progress.tex
%% ---------------------------------------------------------------- 
\documentclass{ecsprogress}    % Use the progress Style
\graphicspath{{../figs/}}   % Location of your graphics files
    \usepackage{natbib}            % Use Natbib style for the refs.
\hypersetup{colorlinks=true}   % Set to false for black/white printing
\input{Definitions}            % Include your abbreviations



\usepackage{enumitem}% http://ctan.org/pkg/enumitem
\usepackage{multirow}
\usepackage{float}
\usepackage{amsmath}
\usepackage{multicol}
\usepackage{amssymb}
\usepackage[normalem]{ulem}
\useunder{\uline}{\ul}{}
\usepackage{wrapfig}


\usepackage[table,xcdraw]{xcolor}


%% ----------------------------------------------------------------
\begin{document}
\frontmatter
\title      {Heterogeneous Agent-based Model for Supermarket Competition}
\authors    {\texorpdfstring
             {\href{mailto:sc22g13@ecs.soton.ac.uk}{Stefan J. Collier}}
             {Stefan J. Collier}
            }
\addresses  {\groupname\\\deptname\\\univname}
\date       {\today}
\subject    {}
\keywords   {}
\supervisor {Dr. Maria Polukarov}
\examiner   {Professor Sheng Chen}

\maketitle
\begin{abstract}
This project aim was to model and analyse the effects of competitive pricing behaviors of grocery retailers on the British market. 

This was achieved by creating a multi-agent model, containing retailer and consumer agents. The heterogeneous crowd of retailers employs either a uniform pricing strategy or a ‘local price flexing’ strategy. The actions of these retailers are chosen by predicting the profit of each action, using a perceptron. Following on from the consideration of different economic models, a discrete model was developed so that software agents have a discrete environment to operate within. Within the model, it has been observed how supermarkets with differing behaviors affect a heterogeneous crowd of consumer agents. The model was implemented in Java with Python used to evaluate the results. 

The simulation displays good acceptance with real grocery market behavior, i.e. captures the performance of British retailers thus can be used to determine the impact of changes in their behavior on their competitors and consumers.Furthermore it can be used to provide insight into sustainability of volatile pricing strategies, providing a useful insight in volatility of British supermarket retail industry. 
\end{abstract}
\acknowledgements{
I would like to express my sincere gratitude to Dr Maria Polukarov for her guidance and support which provided me the freedom to take this research in the direction of my interest.\\
\\
I would also like to thank my family and friends for their encouragement and support. To those who quietly listened to my software complaints. To those who worked throughout the nights with me. To those who helped me write what I couldn't say. I cannot thank you enough.
}

\declaration{
I, Stefan Collier, declare that this dissertation and the work presented in it are my own and has been generated by me as the result of my own original research.\\
I confirm that:\\
1. This work was done wholly or mainly while in candidature for a degree at this University;\\
2. Where any part of this dissertation has previously been submitted for any other qualification at this University or any other institution, this has been clearly stated;\\
3. Where I have consulted the published work of others, this is always clearly attributed;\\
4. Where I have quoted from the work of others, the source is always given. With the exception of such quotations, this dissertation is entirely my own work;\\
5. I have acknowledged all main sources of help;\\
6. Where the thesis is based on work done by myself jointly with others, I have made clear exactly what was done by others and what I have contributed myself;\\
7. Either none of this work has been published before submission, or parts of this work have been published by :\\
\\
Stefan Collier\\
April 2016
}
\tableofcontents
\listoffigures
\listoftables

\mainmatter
%% ----------------------------------------------------------------
%\include{Introduction}
%\include{Conclusions}
\include{chapters/1Project/main}
\include{chapters/2Lit/main}
\include{chapters/3Design/HighLevel}
\include{chapters/3Design/InDepth}
\include{chapters/4Impl/main}

\include{chapters/5Experiments/1/main}
\include{chapters/5Experiments/2/main}
\include{chapters/5Experiments/3/main}
\include{chapters/5Experiments/4/main}

\include{chapters/6Conclusion/main}

\appendix
\include{appendix/AppendixB}
\include{appendix/D/main}
\include{appendix/AppendixC}

\backmatter
\bibliographystyle{ecs}
\bibliography{ECS}
\end{document}
%% ----------------------------------------------------------------


 %% ----------------------------------------------------------------
%% Progress.tex
%% ---------------------------------------------------------------- 
\documentclass{ecsprogress}    % Use the progress Style
\graphicspath{{../figs/}}   % Location of your graphics files
    \usepackage{natbib}            % Use Natbib style for the refs.
\hypersetup{colorlinks=true}   % Set to false for black/white printing
\input{Definitions}            % Include your abbreviations



\usepackage{enumitem}% http://ctan.org/pkg/enumitem
\usepackage{multirow}
\usepackage{float}
\usepackage{amsmath}
\usepackage{multicol}
\usepackage{amssymb}
\usepackage[normalem]{ulem}
\useunder{\uline}{\ul}{}
\usepackage{wrapfig}


\usepackage[table,xcdraw]{xcolor}


%% ----------------------------------------------------------------
\begin{document}
\frontmatter
\title      {Heterogeneous Agent-based Model for Supermarket Competition}
\authors    {\texorpdfstring
             {\href{mailto:sc22g13@ecs.soton.ac.uk}{Stefan J. Collier}}
             {Stefan J. Collier}
            }
\addresses  {\groupname\\\deptname\\\univname}
\date       {\today}
\subject    {}
\keywords   {}
\supervisor {Dr. Maria Polukarov}
\examiner   {Professor Sheng Chen}

\maketitle
\begin{abstract}
This project aim was to model and analyse the effects of competitive pricing behaviors of grocery retailers on the British market. 

This was achieved by creating a multi-agent model, containing retailer and consumer agents. The heterogeneous crowd of retailers employs either a uniform pricing strategy or a ‘local price flexing’ strategy. The actions of these retailers are chosen by predicting the profit of each action, using a perceptron. Following on from the consideration of different economic models, a discrete model was developed so that software agents have a discrete environment to operate within. Within the model, it has been observed how supermarkets with differing behaviors affect a heterogeneous crowd of consumer agents. The model was implemented in Java with Python used to evaluate the results. 

The simulation displays good acceptance with real grocery market behavior, i.e. captures the performance of British retailers thus can be used to determine the impact of changes in their behavior on their competitors and consumers.Furthermore it can be used to provide insight into sustainability of volatile pricing strategies, providing a useful insight in volatility of British supermarket retail industry. 
\end{abstract}
\acknowledgements{
I would like to express my sincere gratitude to Dr Maria Polukarov for her guidance and support which provided me the freedom to take this research in the direction of my interest.\\
\\
I would also like to thank my family and friends for their encouragement and support. To those who quietly listened to my software complaints. To those who worked throughout the nights with me. To those who helped me write what I couldn't say. I cannot thank you enough.
}

\declaration{
I, Stefan Collier, declare that this dissertation and the work presented in it are my own and has been generated by me as the result of my own original research.\\
I confirm that:\\
1. This work was done wholly or mainly while in candidature for a degree at this University;\\
2. Where any part of this dissertation has previously been submitted for any other qualification at this University or any other institution, this has been clearly stated;\\
3. Where I have consulted the published work of others, this is always clearly attributed;\\
4. Where I have quoted from the work of others, the source is always given. With the exception of such quotations, this dissertation is entirely my own work;\\
5. I have acknowledged all main sources of help;\\
6. Where the thesis is based on work done by myself jointly with others, I have made clear exactly what was done by others and what I have contributed myself;\\
7. Either none of this work has been published before submission, or parts of this work have been published by :\\
\\
Stefan Collier\\
April 2016
}
\tableofcontents
\listoffigures
\listoftables

\mainmatter
%% ----------------------------------------------------------------
%\include{Introduction}
%\include{Conclusions}
\include{chapters/1Project/main}
\include{chapters/2Lit/main}
\include{chapters/3Design/HighLevel}
\include{chapters/3Design/InDepth}
\include{chapters/4Impl/main}

\include{chapters/5Experiments/1/main}
\include{chapters/5Experiments/2/main}
\include{chapters/5Experiments/3/main}
\include{chapters/5Experiments/4/main}

\include{chapters/6Conclusion/main}

\appendix
\include{appendix/AppendixB}
\include{appendix/D/main}
\include{appendix/AppendixC}

\backmatter
\bibliographystyle{ecs}
\bibliography{ECS}
\end{document}
%% ----------------------------------------------------------------


\appendix
\include{appendix/AppendixB}
 %% ----------------------------------------------------------------
%% Progress.tex
%% ---------------------------------------------------------------- 
\documentclass{ecsprogress}    % Use the progress Style
\graphicspath{{../figs/}}   % Location of your graphics files
    \usepackage{natbib}            % Use Natbib style for the refs.
\hypersetup{colorlinks=true}   % Set to false for black/white printing
\input{Definitions}            % Include your abbreviations



\usepackage{enumitem}% http://ctan.org/pkg/enumitem
\usepackage{multirow}
\usepackage{float}
\usepackage{amsmath}
\usepackage{multicol}
\usepackage{amssymb}
\usepackage[normalem]{ulem}
\useunder{\uline}{\ul}{}
\usepackage{wrapfig}


\usepackage[table,xcdraw]{xcolor}


%% ----------------------------------------------------------------
\begin{document}
\frontmatter
\title      {Heterogeneous Agent-based Model for Supermarket Competition}
\authors    {\texorpdfstring
             {\href{mailto:sc22g13@ecs.soton.ac.uk}{Stefan J. Collier}}
             {Stefan J. Collier}
            }
\addresses  {\groupname\\\deptname\\\univname}
\date       {\today}
\subject    {}
\keywords   {}
\supervisor {Dr. Maria Polukarov}
\examiner   {Professor Sheng Chen}

\maketitle
\begin{abstract}
This project aim was to model and analyse the effects of competitive pricing behaviors of grocery retailers on the British market. 

This was achieved by creating a multi-agent model, containing retailer and consumer agents. The heterogeneous crowd of retailers employs either a uniform pricing strategy or a ‘local price flexing’ strategy. The actions of these retailers are chosen by predicting the profit of each action, using a perceptron. Following on from the consideration of different economic models, a discrete model was developed so that software agents have a discrete environment to operate within. Within the model, it has been observed how supermarkets with differing behaviors affect a heterogeneous crowd of consumer agents. The model was implemented in Java with Python used to evaluate the results. 

The simulation displays good acceptance with real grocery market behavior, i.e. captures the performance of British retailers thus can be used to determine the impact of changes in their behavior on their competitors and consumers.Furthermore it can be used to provide insight into sustainability of volatile pricing strategies, providing a useful insight in volatility of British supermarket retail industry. 
\end{abstract}
\acknowledgements{
I would like to express my sincere gratitude to Dr Maria Polukarov for her guidance and support which provided me the freedom to take this research in the direction of my interest.\\
\\
I would also like to thank my family and friends for their encouragement and support. To those who quietly listened to my software complaints. To those who worked throughout the nights with me. To those who helped me write what I couldn't say. I cannot thank you enough.
}

\declaration{
I, Stefan Collier, declare that this dissertation and the work presented in it are my own and has been generated by me as the result of my own original research.\\
I confirm that:\\
1. This work was done wholly or mainly while in candidature for a degree at this University;\\
2. Where any part of this dissertation has previously been submitted for any other qualification at this University or any other institution, this has been clearly stated;\\
3. Where I have consulted the published work of others, this is always clearly attributed;\\
4. Where I have quoted from the work of others, the source is always given. With the exception of such quotations, this dissertation is entirely my own work;\\
5. I have acknowledged all main sources of help;\\
6. Where the thesis is based on work done by myself jointly with others, I have made clear exactly what was done by others and what I have contributed myself;\\
7. Either none of this work has been published before submission, or parts of this work have been published by :\\
\\
Stefan Collier\\
April 2016
}
\tableofcontents
\listoffigures
\listoftables

\mainmatter
%% ----------------------------------------------------------------
%\include{Introduction}
%\include{Conclusions}
\include{chapters/1Project/main}
\include{chapters/2Lit/main}
\include{chapters/3Design/HighLevel}
\include{chapters/3Design/InDepth}
\include{chapters/4Impl/main}

\include{chapters/5Experiments/1/main}
\include{chapters/5Experiments/2/main}
\include{chapters/5Experiments/3/main}
\include{chapters/5Experiments/4/main}

\include{chapters/6Conclusion/main}

\appendix
\include{appendix/AppendixB}
\include{appendix/D/main}
\include{appendix/AppendixC}

\backmatter
\bibliographystyle{ecs}
\bibliography{ECS}
\end{document}
%% ----------------------------------------------------------------

\include{appendix/AppendixC}

\backmatter
\bibliographystyle{ecs}
\bibliography{ECS}
\end{document}
%% ----------------------------------------------------------------

 %% ----------------------------------------------------------------
%% Progress.tex
%% ---------------------------------------------------------------- 
\documentclass{ecsprogress}    % Use the progress Style
\graphicspath{{../figs/}}   % Location of your graphics files
    \usepackage{natbib}            % Use Natbib style for the refs.
\hypersetup{colorlinks=true}   % Set to false for black/white printing
\input{Definitions}            % Include your abbreviations



\usepackage{enumitem}% http://ctan.org/pkg/enumitem
\usepackage{multirow}
\usepackage{float}
\usepackage{amsmath}
\usepackage{multicol}
\usepackage{amssymb}
\usepackage[normalem]{ulem}
\useunder{\uline}{\ul}{}
\usepackage{wrapfig}


\usepackage[table,xcdraw]{xcolor}


%% ----------------------------------------------------------------
\begin{document}
\frontmatter
\title      {Heterogeneous Agent-based Model for Supermarket Competition}
\authors    {\texorpdfstring
             {\href{mailto:sc22g13@ecs.soton.ac.uk}{Stefan J. Collier}}
             {Stefan J. Collier}
            }
\addresses  {\groupname\\\deptname\\\univname}
\date       {\today}
\subject    {}
\keywords   {}
\supervisor {Dr. Maria Polukarov}
\examiner   {Professor Sheng Chen}

\maketitle
\begin{abstract}
This project aim was to model and analyse the effects of competitive pricing behaviors of grocery retailers on the British market. 

This was achieved by creating a multi-agent model, containing retailer and consumer agents. The heterogeneous crowd of retailers employs either a uniform pricing strategy or a ‘local price flexing’ strategy. The actions of these retailers are chosen by predicting the profit of each action, using a perceptron. Following on from the consideration of different economic models, a discrete model was developed so that software agents have a discrete environment to operate within. Within the model, it has been observed how supermarkets with differing behaviors affect a heterogeneous crowd of consumer agents. The model was implemented in Java with Python used to evaluate the results. 

The simulation displays good acceptance with real grocery market behavior, i.e. captures the performance of British retailers thus can be used to determine the impact of changes in their behavior on their competitors and consumers.Furthermore it can be used to provide insight into sustainability of volatile pricing strategies, providing a useful insight in volatility of British supermarket retail industry. 
\end{abstract}
\acknowledgements{
I would like to express my sincere gratitude to Dr Maria Polukarov for her guidance and support which provided me the freedom to take this research in the direction of my interest.\\
\\
I would also like to thank my family and friends for their encouragement and support. To those who quietly listened to my software complaints. To those who worked throughout the nights with me. To those who helped me write what I couldn't say. I cannot thank you enough.
}

\declaration{
I, Stefan Collier, declare that this dissertation and the work presented in it are my own and has been generated by me as the result of my own original research.\\
I confirm that:\\
1. This work was done wholly or mainly while in candidature for a degree at this University;\\
2. Where any part of this dissertation has previously been submitted for any other qualification at this University or any other institution, this has been clearly stated;\\
3. Where I have consulted the published work of others, this is always clearly attributed;\\
4. Where I have quoted from the work of others, the source is always given. With the exception of such quotations, this dissertation is entirely my own work;\\
5. I have acknowledged all main sources of help;\\
6. Where the thesis is based on work done by myself jointly with others, I have made clear exactly what was done by others and what I have contributed myself;\\
7. Either none of this work has been published before submission, or parts of this work have been published by :\\
\\
Stefan Collier\\
April 2016
}
\tableofcontents
\listoffigures
\listoftables

\mainmatter
%% ----------------------------------------------------------------
%\include{Introduction}
%\include{Conclusions}
 %% ----------------------------------------------------------------
%% Progress.tex
%% ---------------------------------------------------------------- 
\documentclass{ecsprogress}    % Use the progress Style
\graphicspath{{../figs/}}   % Location of your graphics files
    \usepackage{natbib}            % Use Natbib style for the refs.
\hypersetup{colorlinks=true}   % Set to false for black/white printing
\input{Definitions}            % Include your abbreviations



\usepackage{enumitem}% http://ctan.org/pkg/enumitem
\usepackage{multirow}
\usepackage{float}
\usepackage{amsmath}
\usepackage{multicol}
\usepackage{amssymb}
\usepackage[normalem]{ulem}
\useunder{\uline}{\ul}{}
\usepackage{wrapfig}


\usepackage[table,xcdraw]{xcolor}


%% ----------------------------------------------------------------
\begin{document}
\frontmatter
\title      {Heterogeneous Agent-based Model for Supermarket Competition}
\authors    {\texorpdfstring
             {\href{mailto:sc22g13@ecs.soton.ac.uk}{Stefan J. Collier}}
             {Stefan J. Collier}
            }
\addresses  {\groupname\\\deptname\\\univname}
\date       {\today}
\subject    {}
\keywords   {}
\supervisor {Dr. Maria Polukarov}
\examiner   {Professor Sheng Chen}

\maketitle
\begin{abstract}
This project aim was to model and analyse the effects of competitive pricing behaviors of grocery retailers on the British market. 

This was achieved by creating a multi-agent model, containing retailer and consumer agents. The heterogeneous crowd of retailers employs either a uniform pricing strategy or a ‘local price flexing’ strategy. The actions of these retailers are chosen by predicting the profit of each action, using a perceptron. Following on from the consideration of different economic models, a discrete model was developed so that software agents have a discrete environment to operate within. Within the model, it has been observed how supermarkets with differing behaviors affect a heterogeneous crowd of consumer agents. The model was implemented in Java with Python used to evaluate the results. 

The simulation displays good acceptance with real grocery market behavior, i.e. captures the performance of British retailers thus can be used to determine the impact of changes in their behavior on their competitors and consumers.Furthermore it can be used to provide insight into sustainability of volatile pricing strategies, providing a useful insight in volatility of British supermarket retail industry. 
\end{abstract}
\acknowledgements{
I would like to express my sincere gratitude to Dr Maria Polukarov for her guidance and support which provided me the freedom to take this research in the direction of my interest.\\
\\
I would also like to thank my family and friends for their encouragement and support. To those who quietly listened to my software complaints. To those who worked throughout the nights with me. To those who helped me write what I couldn't say. I cannot thank you enough.
}

\declaration{
I, Stefan Collier, declare that this dissertation and the work presented in it are my own and has been generated by me as the result of my own original research.\\
I confirm that:\\
1. This work was done wholly or mainly while in candidature for a degree at this University;\\
2. Where any part of this dissertation has previously been submitted for any other qualification at this University or any other institution, this has been clearly stated;\\
3. Where I have consulted the published work of others, this is always clearly attributed;\\
4. Where I have quoted from the work of others, the source is always given. With the exception of such quotations, this dissertation is entirely my own work;\\
5. I have acknowledged all main sources of help;\\
6. Where the thesis is based on work done by myself jointly with others, I have made clear exactly what was done by others and what I have contributed myself;\\
7. Either none of this work has been published before submission, or parts of this work have been published by :\\
\\
Stefan Collier\\
April 2016
}
\tableofcontents
\listoffigures
\listoftables

\mainmatter
%% ----------------------------------------------------------------
%\include{Introduction}
%\include{Conclusions}
\include{chapters/1Project/main}
\include{chapters/2Lit/main}
\include{chapters/3Design/HighLevel}
\include{chapters/3Design/InDepth}
\include{chapters/4Impl/main}

\include{chapters/5Experiments/1/main}
\include{chapters/5Experiments/2/main}
\include{chapters/5Experiments/3/main}
\include{chapters/5Experiments/4/main}

\include{chapters/6Conclusion/main}

\appendix
\include{appendix/AppendixB}
\include{appendix/D/main}
\include{appendix/AppendixC}

\backmatter
\bibliographystyle{ecs}
\bibliography{ECS}
\end{document}
%% ----------------------------------------------------------------

 %% ----------------------------------------------------------------
%% Progress.tex
%% ---------------------------------------------------------------- 
\documentclass{ecsprogress}    % Use the progress Style
\graphicspath{{../figs/}}   % Location of your graphics files
    \usepackage{natbib}            % Use Natbib style for the refs.
\hypersetup{colorlinks=true}   % Set to false for black/white printing
\input{Definitions}            % Include your abbreviations



\usepackage{enumitem}% http://ctan.org/pkg/enumitem
\usepackage{multirow}
\usepackage{float}
\usepackage{amsmath}
\usepackage{multicol}
\usepackage{amssymb}
\usepackage[normalem]{ulem}
\useunder{\uline}{\ul}{}
\usepackage{wrapfig}


\usepackage[table,xcdraw]{xcolor}


%% ----------------------------------------------------------------
\begin{document}
\frontmatter
\title      {Heterogeneous Agent-based Model for Supermarket Competition}
\authors    {\texorpdfstring
             {\href{mailto:sc22g13@ecs.soton.ac.uk}{Stefan J. Collier}}
             {Stefan J. Collier}
            }
\addresses  {\groupname\\\deptname\\\univname}
\date       {\today}
\subject    {}
\keywords   {}
\supervisor {Dr. Maria Polukarov}
\examiner   {Professor Sheng Chen}

\maketitle
\begin{abstract}
This project aim was to model and analyse the effects of competitive pricing behaviors of grocery retailers on the British market. 

This was achieved by creating a multi-agent model, containing retailer and consumer agents. The heterogeneous crowd of retailers employs either a uniform pricing strategy or a ‘local price flexing’ strategy. The actions of these retailers are chosen by predicting the profit of each action, using a perceptron. Following on from the consideration of different economic models, a discrete model was developed so that software agents have a discrete environment to operate within. Within the model, it has been observed how supermarkets with differing behaviors affect a heterogeneous crowd of consumer agents. The model was implemented in Java with Python used to evaluate the results. 

The simulation displays good acceptance with real grocery market behavior, i.e. captures the performance of British retailers thus can be used to determine the impact of changes in their behavior on their competitors and consumers.Furthermore it can be used to provide insight into sustainability of volatile pricing strategies, providing a useful insight in volatility of British supermarket retail industry. 
\end{abstract}
\acknowledgements{
I would like to express my sincere gratitude to Dr Maria Polukarov for her guidance and support which provided me the freedom to take this research in the direction of my interest.\\
\\
I would also like to thank my family and friends for their encouragement and support. To those who quietly listened to my software complaints. To those who worked throughout the nights with me. To those who helped me write what I couldn't say. I cannot thank you enough.
}

\declaration{
I, Stefan Collier, declare that this dissertation and the work presented in it are my own and has been generated by me as the result of my own original research.\\
I confirm that:\\
1. This work was done wholly or mainly while in candidature for a degree at this University;\\
2. Where any part of this dissertation has previously been submitted for any other qualification at this University or any other institution, this has been clearly stated;\\
3. Where I have consulted the published work of others, this is always clearly attributed;\\
4. Where I have quoted from the work of others, the source is always given. With the exception of such quotations, this dissertation is entirely my own work;\\
5. I have acknowledged all main sources of help;\\
6. Where the thesis is based on work done by myself jointly with others, I have made clear exactly what was done by others and what I have contributed myself;\\
7. Either none of this work has been published before submission, or parts of this work have been published by :\\
\\
Stefan Collier\\
April 2016
}
\tableofcontents
\listoffigures
\listoftables

\mainmatter
%% ----------------------------------------------------------------
%\include{Introduction}
%\include{Conclusions}
\include{chapters/1Project/main}
\include{chapters/2Lit/main}
\include{chapters/3Design/HighLevel}
\include{chapters/3Design/InDepth}
\include{chapters/4Impl/main}

\include{chapters/5Experiments/1/main}
\include{chapters/5Experiments/2/main}
\include{chapters/5Experiments/3/main}
\include{chapters/5Experiments/4/main}

\include{chapters/6Conclusion/main}

\appendix
\include{appendix/AppendixB}
\include{appendix/D/main}
\include{appendix/AppendixC}

\backmatter
\bibliographystyle{ecs}
\bibliography{ECS}
\end{document}
%% ----------------------------------------------------------------

\include{chapters/3Design/HighLevel}
\include{chapters/3Design/InDepth}
 %% ----------------------------------------------------------------
%% Progress.tex
%% ---------------------------------------------------------------- 
\documentclass{ecsprogress}    % Use the progress Style
\graphicspath{{../figs/}}   % Location of your graphics files
    \usepackage{natbib}            % Use Natbib style for the refs.
\hypersetup{colorlinks=true}   % Set to false for black/white printing
\input{Definitions}            % Include your abbreviations



\usepackage{enumitem}% http://ctan.org/pkg/enumitem
\usepackage{multirow}
\usepackage{float}
\usepackage{amsmath}
\usepackage{multicol}
\usepackage{amssymb}
\usepackage[normalem]{ulem}
\useunder{\uline}{\ul}{}
\usepackage{wrapfig}


\usepackage[table,xcdraw]{xcolor}


%% ----------------------------------------------------------------
\begin{document}
\frontmatter
\title      {Heterogeneous Agent-based Model for Supermarket Competition}
\authors    {\texorpdfstring
             {\href{mailto:sc22g13@ecs.soton.ac.uk}{Stefan J. Collier}}
             {Stefan J. Collier}
            }
\addresses  {\groupname\\\deptname\\\univname}
\date       {\today}
\subject    {}
\keywords   {}
\supervisor {Dr. Maria Polukarov}
\examiner   {Professor Sheng Chen}

\maketitle
\begin{abstract}
This project aim was to model and analyse the effects of competitive pricing behaviors of grocery retailers on the British market. 

This was achieved by creating a multi-agent model, containing retailer and consumer agents. The heterogeneous crowd of retailers employs either a uniform pricing strategy or a ‘local price flexing’ strategy. The actions of these retailers are chosen by predicting the profit of each action, using a perceptron. Following on from the consideration of different economic models, a discrete model was developed so that software agents have a discrete environment to operate within. Within the model, it has been observed how supermarkets with differing behaviors affect a heterogeneous crowd of consumer agents. The model was implemented in Java with Python used to evaluate the results. 

The simulation displays good acceptance with real grocery market behavior, i.e. captures the performance of British retailers thus can be used to determine the impact of changes in their behavior on their competitors and consumers.Furthermore it can be used to provide insight into sustainability of volatile pricing strategies, providing a useful insight in volatility of British supermarket retail industry. 
\end{abstract}
\acknowledgements{
I would like to express my sincere gratitude to Dr Maria Polukarov for her guidance and support which provided me the freedom to take this research in the direction of my interest.\\
\\
I would also like to thank my family and friends for their encouragement and support. To those who quietly listened to my software complaints. To those who worked throughout the nights with me. To those who helped me write what I couldn't say. I cannot thank you enough.
}

\declaration{
I, Stefan Collier, declare that this dissertation and the work presented in it are my own and has been generated by me as the result of my own original research.\\
I confirm that:\\
1. This work was done wholly or mainly while in candidature for a degree at this University;\\
2. Where any part of this dissertation has previously been submitted for any other qualification at this University or any other institution, this has been clearly stated;\\
3. Where I have consulted the published work of others, this is always clearly attributed;\\
4. Where I have quoted from the work of others, the source is always given. With the exception of such quotations, this dissertation is entirely my own work;\\
5. I have acknowledged all main sources of help;\\
6. Where the thesis is based on work done by myself jointly with others, I have made clear exactly what was done by others and what I have contributed myself;\\
7. Either none of this work has been published before submission, or parts of this work have been published by :\\
\\
Stefan Collier\\
April 2016
}
\tableofcontents
\listoffigures
\listoftables

\mainmatter
%% ----------------------------------------------------------------
%\include{Introduction}
%\include{Conclusions}
\include{chapters/1Project/main}
\include{chapters/2Lit/main}
\include{chapters/3Design/HighLevel}
\include{chapters/3Design/InDepth}
\include{chapters/4Impl/main}

\include{chapters/5Experiments/1/main}
\include{chapters/5Experiments/2/main}
\include{chapters/5Experiments/3/main}
\include{chapters/5Experiments/4/main}

\include{chapters/6Conclusion/main}

\appendix
\include{appendix/AppendixB}
\include{appendix/D/main}
\include{appendix/AppendixC}

\backmatter
\bibliographystyle{ecs}
\bibliography{ECS}
\end{document}
%% ----------------------------------------------------------------


 %% ----------------------------------------------------------------
%% Progress.tex
%% ---------------------------------------------------------------- 
\documentclass{ecsprogress}    % Use the progress Style
\graphicspath{{../figs/}}   % Location of your graphics files
    \usepackage{natbib}            % Use Natbib style for the refs.
\hypersetup{colorlinks=true}   % Set to false for black/white printing
\input{Definitions}            % Include your abbreviations



\usepackage{enumitem}% http://ctan.org/pkg/enumitem
\usepackage{multirow}
\usepackage{float}
\usepackage{amsmath}
\usepackage{multicol}
\usepackage{amssymb}
\usepackage[normalem]{ulem}
\useunder{\uline}{\ul}{}
\usepackage{wrapfig}


\usepackage[table,xcdraw]{xcolor}


%% ----------------------------------------------------------------
\begin{document}
\frontmatter
\title      {Heterogeneous Agent-based Model for Supermarket Competition}
\authors    {\texorpdfstring
             {\href{mailto:sc22g13@ecs.soton.ac.uk}{Stefan J. Collier}}
             {Stefan J. Collier}
            }
\addresses  {\groupname\\\deptname\\\univname}
\date       {\today}
\subject    {}
\keywords   {}
\supervisor {Dr. Maria Polukarov}
\examiner   {Professor Sheng Chen}

\maketitle
\begin{abstract}
This project aim was to model and analyse the effects of competitive pricing behaviors of grocery retailers on the British market. 

This was achieved by creating a multi-agent model, containing retailer and consumer agents. The heterogeneous crowd of retailers employs either a uniform pricing strategy or a ‘local price flexing’ strategy. The actions of these retailers are chosen by predicting the profit of each action, using a perceptron. Following on from the consideration of different economic models, a discrete model was developed so that software agents have a discrete environment to operate within. Within the model, it has been observed how supermarkets with differing behaviors affect a heterogeneous crowd of consumer agents. The model was implemented in Java with Python used to evaluate the results. 

The simulation displays good acceptance with real grocery market behavior, i.e. captures the performance of British retailers thus can be used to determine the impact of changes in their behavior on their competitors and consumers.Furthermore it can be used to provide insight into sustainability of volatile pricing strategies, providing a useful insight in volatility of British supermarket retail industry. 
\end{abstract}
\acknowledgements{
I would like to express my sincere gratitude to Dr Maria Polukarov for her guidance and support which provided me the freedom to take this research in the direction of my interest.\\
\\
I would also like to thank my family and friends for their encouragement and support. To those who quietly listened to my software complaints. To those who worked throughout the nights with me. To those who helped me write what I couldn't say. I cannot thank you enough.
}

\declaration{
I, Stefan Collier, declare that this dissertation and the work presented in it are my own and has been generated by me as the result of my own original research.\\
I confirm that:\\
1. This work was done wholly or mainly while in candidature for a degree at this University;\\
2. Where any part of this dissertation has previously been submitted for any other qualification at this University or any other institution, this has been clearly stated;\\
3. Where I have consulted the published work of others, this is always clearly attributed;\\
4. Where I have quoted from the work of others, the source is always given. With the exception of such quotations, this dissertation is entirely my own work;\\
5. I have acknowledged all main sources of help;\\
6. Where the thesis is based on work done by myself jointly with others, I have made clear exactly what was done by others and what I have contributed myself;\\
7. Either none of this work has been published before submission, or parts of this work have been published by :\\
\\
Stefan Collier\\
April 2016
}
\tableofcontents
\listoffigures
\listoftables

\mainmatter
%% ----------------------------------------------------------------
%\include{Introduction}
%\include{Conclusions}
\include{chapters/1Project/main}
\include{chapters/2Lit/main}
\include{chapters/3Design/HighLevel}
\include{chapters/3Design/InDepth}
\include{chapters/4Impl/main}

\include{chapters/5Experiments/1/main}
\include{chapters/5Experiments/2/main}
\include{chapters/5Experiments/3/main}
\include{chapters/5Experiments/4/main}

\include{chapters/6Conclusion/main}

\appendix
\include{appendix/AppendixB}
\include{appendix/D/main}
\include{appendix/AppendixC}

\backmatter
\bibliographystyle{ecs}
\bibliography{ECS}
\end{document}
%% ----------------------------------------------------------------

 %% ----------------------------------------------------------------
%% Progress.tex
%% ---------------------------------------------------------------- 
\documentclass{ecsprogress}    % Use the progress Style
\graphicspath{{../figs/}}   % Location of your graphics files
    \usepackage{natbib}            % Use Natbib style for the refs.
\hypersetup{colorlinks=true}   % Set to false for black/white printing
\input{Definitions}            % Include your abbreviations



\usepackage{enumitem}% http://ctan.org/pkg/enumitem
\usepackage{multirow}
\usepackage{float}
\usepackage{amsmath}
\usepackage{multicol}
\usepackage{amssymb}
\usepackage[normalem]{ulem}
\useunder{\uline}{\ul}{}
\usepackage{wrapfig}


\usepackage[table,xcdraw]{xcolor}


%% ----------------------------------------------------------------
\begin{document}
\frontmatter
\title      {Heterogeneous Agent-based Model for Supermarket Competition}
\authors    {\texorpdfstring
             {\href{mailto:sc22g13@ecs.soton.ac.uk}{Stefan J. Collier}}
             {Stefan J. Collier}
            }
\addresses  {\groupname\\\deptname\\\univname}
\date       {\today}
\subject    {}
\keywords   {}
\supervisor {Dr. Maria Polukarov}
\examiner   {Professor Sheng Chen}

\maketitle
\begin{abstract}
This project aim was to model and analyse the effects of competitive pricing behaviors of grocery retailers on the British market. 

This was achieved by creating a multi-agent model, containing retailer and consumer agents. The heterogeneous crowd of retailers employs either a uniform pricing strategy or a ‘local price flexing’ strategy. The actions of these retailers are chosen by predicting the profit of each action, using a perceptron. Following on from the consideration of different economic models, a discrete model was developed so that software agents have a discrete environment to operate within. Within the model, it has been observed how supermarkets with differing behaviors affect a heterogeneous crowd of consumer agents. The model was implemented in Java with Python used to evaluate the results. 

The simulation displays good acceptance with real grocery market behavior, i.e. captures the performance of British retailers thus can be used to determine the impact of changes in their behavior on their competitors and consumers.Furthermore it can be used to provide insight into sustainability of volatile pricing strategies, providing a useful insight in volatility of British supermarket retail industry. 
\end{abstract}
\acknowledgements{
I would like to express my sincere gratitude to Dr Maria Polukarov for her guidance and support which provided me the freedom to take this research in the direction of my interest.\\
\\
I would also like to thank my family and friends for their encouragement and support. To those who quietly listened to my software complaints. To those who worked throughout the nights with me. To those who helped me write what I couldn't say. I cannot thank you enough.
}

\declaration{
I, Stefan Collier, declare that this dissertation and the work presented in it are my own and has been generated by me as the result of my own original research.\\
I confirm that:\\
1. This work was done wholly or mainly while in candidature for a degree at this University;\\
2. Where any part of this dissertation has previously been submitted for any other qualification at this University or any other institution, this has been clearly stated;\\
3. Where I have consulted the published work of others, this is always clearly attributed;\\
4. Where I have quoted from the work of others, the source is always given. With the exception of such quotations, this dissertation is entirely my own work;\\
5. I have acknowledged all main sources of help;\\
6. Where the thesis is based on work done by myself jointly with others, I have made clear exactly what was done by others and what I have contributed myself;\\
7. Either none of this work has been published before submission, or parts of this work have been published by :\\
\\
Stefan Collier\\
April 2016
}
\tableofcontents
\listoffigures
\listoftables

\mainmatter
%% ----------------------------------------------------------------
%\include{Introduction}
%\include{Conclusions}
\include{chapters/1Project/main}
\include{chapters/2Lit/main}
\include{chapters/3Design/HighLevel}
\include{chapters/3Design/InDepth}
\include{chapters/4Impl/main}

\include{chapters/5Experiments/1/main}
\include{chapters/5Experiments/2/main}
\include{chapters/5Experiments/3/main}
\include{chapters/5Experiments/4/main}

\include{chapters/6Conclusion/main}

\appendix
\include{appendix/AppendixB}
\include{appendix/D/main}
\include{appendix/AppendixC}

\backmatter
\bibliographystyle{ecs}
\bibliography{ECS}
\end{document}
%% ----------------------------------------------------------------

 %% ----------------------------------------------------------------
%% Progress.tex
%% ---------------------------------------------------------------- 
\documentclass{ecsprogress}    % Use the progress Style
\graphicspath{{../figs/}}   % Location of your graphics files
    \usepackage{natbib}            % Use Natbib style for the refs.
\hypersetup{colorlinks=true}   % Set to false for black/white printing
\input{Definitions}            % Include your abbreviations



\usepackage{enumitem}% http://ctan.org/pkg/enumitem
\usepackage{multirow}
\usepackage{float}
\usepackage{amsmath}
\usepackage{multicol}
\usepackage{amssymb}
\usepackage[normalem]{ulem}
\useunder{\uline}{\ul}{}
\usepackage{wrapfig}


\usepackage[table,xcdraw]{xcolor}


%% ----------------------------------------------------------------
\begin{document}
\frontmatter
\title      {Heterogeneous Agent-based Model for Supermarket Competition}
\authors    {\texorpdfstring
             {\href{mailto:sc22g13@ecs.soton.ac.uk}{Stefan J. Collier}}
             {Stefan J. Collier}
            }
\addresses  {\groupname\\\deptname\\\univname}
\date       {\today}
\subject    {}
\keywords   {}
\supervisor {Dr. Maria Polukarov}
\examiner   {Professor Sheng Chen}

\maketitle
\begin{abstract}
This project aim was to model and analyse the effects of competitive pricing behaviors of grocery retailers on the British market. 

This was achieved by creating a multi-agent model, containing retailer and consumer agents. The heterogeneous crowd of retailers employs either a uniform pricing strategy or a ‘local price flexing’ strategy. The actions of these retailers are chosen by predicting the profit of each action, using a perceptron. Following on from the consideration of different economic models, a discrete model was developed so that software agents have a discrete environment to operate within. Within the model, it has been observed how supermarkets with differing behaviors affect a heterogeneous crowd of consumer agents. The model was implemented in Java with Python used to evaluate the results. 

The simulation displays good acceptance with real grocery market behavior, i.e. captures the performance of British retailers thus can be used to determine the impact of changes in their behavior on their competitors and consumers.Furthermore it can be used to provide insight into sustainability of volatile pricing strategies, providing a useful insight in volatility of British supermarket retail industry. 
\end{abstract}
\acknowledgements{
I would like to express my sincere gratitude to Dr Maria Polukarov for her guidance and support which provided me the freedom to take this research in the direction of my interest.\\
\\
I would also like to thank my family and friends for their encouragement and support. To those who quietly listened to my software complaints. To those who worked throughout the nights with me. To those who helped me write what I couldn't say. I cannot thank you enough.
}

\declaration{
I, Stefan Collier, declare that this dissertation and the work presented in it are my own and has been generated by me as the result of my own original research.\\
I confirm that:\\
1. This work was done wholly or mainly while in candidature for a degree at this University;\\
2. Where any part of this dissertation has previously been submitted for any other qualification at this University or any other institution, this has been clearly stated;\\
3. Where I have consulted the published work of others, this is always clearly attributed;\\
4. Where I have quoted from the work of others, the source is always given. With the exception of such quotations, this dissertation is entirely my own work;\\
5. I have acknowledged all main sources of help;\\
6. Where the thesis is based on work done by myself jointly with others, I have made clear exactly what was done by others and what I have contributed myself;\\
7. Either none of this work has been published before submission, or parts of this work have been published by :\\
\\
Stefan Collier\\
April 2016
}
\tableofcontents
\listoffigures
\listoftables

\mainmatter
%% ----------------------------------------------------------------
%\include{Introduction}
%\include{Conclusions}
\include{chapters/1Project/main}
\include{chapters/2Lit/main}
\include{chapters/3Design/HighLevel}
\include{chapters/3Design/InDepth}
\include{chapters/4Impl/main}

\include{chapters/5Experiments/1/main}
\include{chapters/5Experiments/2/main}
\include{chapters/5Experiments/3/main}
\include{chapters/5Experiments/4/main}

\include{chapters/6Conclusion/main}

\appendix
\include{appendix/AppendixB}
\include{appendix/D/main}
\include{appendix/AppendixC}

\backmatter
\bibliographystyle{ecs}
\bibliography{ECS}
\end{document}
%% ----------------------------------------------------------------

 %% ----------------------------------------------------------------
%% Progress.tex
%% ---------------------------------------------------------------- 
\documentclass{ecsprogress}    % Use the progress Style
\graphicspath{{../figs/}}   % Location of your graphics files
    \usepackage{natbib}            % Use Natbib style for the refs.
\hypersetup{colorlinks=true}   % Set to false for black/white printing
\input{Definitions}            % Include your abbreviations



\usepackage{enumitem}% http://ctan.org/pkg/enumitem
\usepackage{multirow}
\usepackage{float}
\usepackage{amsmath}
\usepackage{multicol}
\usepackage{amssymb}
\usepackage[normalem]{ulem}
\useunder{\uline}{\ul}{}
\usepackage{wrapfig}


\usepackage[table,xcdraw]{xcolor}


%% ----------------------------------------------------------------
\begin{document}
\frontmatter
\title      {Heterogeneous Agent-based Model for Supermarket Competition}
\authors    {\texorpdfstring
             {\href{mailto:sc22g13@ecs.soton.ac.uk}{Stefan J. Collier}}
             {Stefan J. Collier}
            }
\addresses  {\groupname\\\deptname\\\univname}
\date       {\today}
\subject    {}
\keywords   {}
\supervisor {Dr. Maria Polukarov}
\examiner   {Professor Sheng Chen}

\maketitle
\begin{abstract}
This project aim was to model and analyse the effects of competitive pricing behaviors of grocery retailers on the British market. 

This was achieved by creating a multi-agent model, containing retailer and consumer agents. The heterogeneous crowd of retailers employs either a uniform pricing strategy or a ‘local price flexing’ strategy. The actions of these retailers are chosen by predicting the profit of each action, using a perceptron. Following on from the consideration of different economic models, a discrete model was developed so that software agents have a discrete environment to operate within. Within the model, it has been observed how supermarkets with differing behaviors affect a heterogeneous crowd of consumer agents. The model was implemented in Java with Python used to evaluate the results. 

The simulation displays good acceptance with real grocery market behavior, i.e. captures the performance of British retailers thus can be used to determine the impact of changes in their behavior on their competitors and consumers.Furthermore it can be used to provide insight into sustainability of volatile pricing strategies, providing a useful insight in volatility of British supermarket retail industry. 
\end{abstract}
\acknowledgements{
I would like to express my sincere gratitude to Dr Maria Polukarov for her guidance and support which provided me the freedom to take this research in the direction of my interest.\\
\\
I would also like to thank my family and friends for their encouragement and support. To those who quietly listened to my software complaints. To those who worked throughout the nights with me. To those who helped me write what I couldn't say. I cannot thank you enough.
}

\declaration{
I, Stefan Collier, declare that this dissertation and the work presented in it are my own and has been generated by me as the result of my own original research.\\
I confirm that:\\
1. This work was done wholly or mainly while in candidature for a degree at this University;\\
2. Where any part of this dissertation has previously been submitted for any other qualification at this University or any other institution, this has been clearly stated;\\
3. Where I have consulted the published work of others, this is always clearly attributed;\\
4. Where I have quoted from the work of others, the source is always given. With the exception of such quotations, this dissertation is entirely my own work;\\
5. I have acknowledged all main sources of help;\\
6. Where the thesis is based on work done by myself jointly with others, I have made clear exactly what was done by others and what I have contributed myself;\\
7. Either none of this work has been published before submission, or parts of this work have been published by :\\
\\
Stefan Collier\\
April 2016
}
\tableofcontents
\listoffigures
\listoftables

\mainmatter
%% ----------------------------------------------------------------
%\include{Introduction}
%\include{Conclusions}
\include{chapters/1Project/main}
\include{chapters/2Lit/main}
\include{chapters/3Design/HighLevel}
\include{chapters/3Design/InDepth}
\include{chapters/4Impl/main}

\include{chapters/5Experiments/1/main}
\include{chapters/5Experiments/2/main}
\include{chapters/5Experiments/3/main}
\include{chapters/5Experiments/4/main}

\include{chapters/6Conclusion/main}

\appendix
\include{appendix/AppendixB}
\include{appendix/D/main}
\include{appendix/AppendixC}

\backmatter
\bibliographystyle{ecs}
\bibliography{ECS}
\end{document}
%% ----------------------------------------------------------------


 %% ----------------------------------------------------------------
%% Progress.tex
%% ---------------------------------------------------------------- 
\documentclass{ecsprogress}    % Use the progress Style
\graphicspath{{../figs/}}   % Location of your graphics files
    \usepackage{natbib}            % Use Natbib style for the refs.
\hypersetup{colorlinks=true}   % Set to false for black/white printing
\input{Definitions}            % Include your abbreviations



\usepackage{enumitem}% http://ctan.org/pkg/enumitem
\usepackage{multirow}
\usepackage{float}
\usepackage{amsmath}
\usepackage{multicol}
\usepackage{amssymb}
\usepackage[normalem]{ulem}
\useunder{\uline}{\ul}{}
\usepackage{wrapfig}


\usepackage[table,xcdraw]{xcolor}


%% ----------------------------------------------------------------
\begin{document}
\frontmatter
\title      {Heterogeneous Agent-based Model for Supermarket Competition}
\authors    {\texorpdfstring
             {\href{mailto:sc22g13@ecs.soton.ac.uk}{Stefan J. Collier}}
             {Stefan J. Collier}
            }
\addresses  {\groupname\\\deptname\\\univname}
\date       {\today}
\subject    {}
\keywords   {}
\supervisor {Dr. Maria Polukarov}
\examiner   {Professor Sheng Chen}

\maketitle
\begin{abstract}
This project aim was to model and analyse the effects of competitive pricing behaviors of grocery retailers on the British market. 

This was achieved by creating a multi-agent model, containing retailer and consumer agents. The heterogeneous crowd of retailers employs either a uniform pricing strategy or a ‘local price flexing’ strategy. The actions of these retailers are chosen by predicting the profit of each action, using a perceptron. Following on from the consideration of different economic models, a discrete model was developed so that software agents have a discrete environment to operate within. Within the model, it has been observed how supermarkets with differing behaviors affect a heterogeneous crowd of consumer agents. The model was implemented in Java with Python used to evaluate the results. 

The simulation displays good acceptance with real grocery market behavior, i.e. captures the performance of British retailers thus can be used to determine the impact of changes in their behavior on their competitors and consumers.Furthermore it can be used to provide insight into sustainability of volatile pricing strategies, providing a useful insight in volatility of British supermarket retail industry. 
\end{abstract}
\acknowledgements{
I would like to express my sincere gratitude to Dr Maria Polukarov for her guidance and support which provided me the freedom to take this research in the direction of my interest.\\
\\
I would also like to thank my family and friends for their encouragement and support. To those who quietly listened to my software complaints. To those who worked throughout the nights with me. To those who helped me write what I couldn't say. I cannot thank you enough.
}

\declaration{
I, Stefan Collier, declare that this dissertation and the work presented in it are my own and has been generated by me as the result of my own original research.\\
I confirm that:\\
1. This work was done wholly or mainly while in candidature for a degree at this University;\\
2. Where any part of this dissertation has previously been submitted for any other qualification at this University or any other institution, this has been clearly stated;\\
3. Where I have consulted the published work of others, this is always clearly attributed;\\
4. Where I have quoted from the work of others, the source is always given. With the exception of such quotations, this dissertation is entirely my own work;\\
5. I have acknowledged all main sources of help;\\
6. Where the thesis is based on work done by myself jointly with others, I have made clear exactly what was done by others and what I have contributed myself;\\
7. Either none of this work has been published before submission, or parts of this work have been published by :\\
\\
Stefan Collier\\
April 2016
}
\tableofcontents
\listoffigures
\listoftables

\mainmatter
%% ----------------------------------------------------------------
%\include{Introduction}
%\include{Conclusions}
\include{chapters/1Project/main}
\include{chapters/2Lit/main}
\include{chapters/3Design/HighLevel}
\include{chapters/3Design/InDepth}
\include{chapters/4Impl/main}

\include{chapters/5Experiments/1/main}
\include{chapters/5Experiments/2/main}
\include{chapters/5Experiments/3/main}
\include{chapters/5Experiments/4/main}

\include{chapters/6Conclusion/main}

\appendix
\include{appendix/AppendixB}
\include{appendix/D/main}
\include{appendix/AppendixC}

\backmatter
\bibliographystyle{ecs}
\bibliography{ECS}
\end{document}
%% ----------------------------------------------------------------


\appendix
\include{appendix/AppendixB}
 %% ----------------------------------------------------------------
%% Progress.tex
%% ---------------------------------------------------------------- 
\documentclass{ecsprogress}    % Use the progress Style
\graphicspath{{../figs/}}   % Location of your graphics files
    \usepackage{natbib}            % Use Natbib style for the refs.
\hypersetup{colorlinks=true}   % Set to false for black/white printing
\input{Definitions}            % Include your abbreviations



\usepackage{enumitem}% http://ctan.org/pkg/enumitem
\usepackage{multirow}
\usepackage{float}
\usepackage{amsmath}
\usepackage{multicol}
\usepackage{amssymb}
\usepackage[normalem]{ulem}
\useunder{\uline}{\ul}{}
\usepackage{wrapfig}


\usepackage[table,xcdraw]{xcolor}


%% ----------------------------------------------------------------
\begin{document}
\frontmatter
\title      {Heterogeneous Agent-based Model for Supermarket Competition}
\authors    {\texorpdfstring
             {\href{mailto:sc22g13@ecs.soton.ac.uk}{Stefan J. Collier}}
             {Stefan J. Collier}
            }
\addresses  {\groupname\\\deptname\\\univname}
\date       {\today}
\subject    {}
\keywords   {}
\supervisor {Dr. Maria Polukarov}
\examiner   {Professor Sheng Chen}

\maketitle
\begin{abstract}
This project aim was to model and analyse the effects of competitive pricing behaviors of grocery retailers on the British market. 

This was achieved by creating a multi-agent model, containing retailer and consumer agents. The heterogeneous crowd of retailers employs either a uniform pricing strategy or a ‘local price flexing’ strategy. The actions of these retailers are chosen by predicting the profit of each action, using a perceptron. Following on from the consideration of different economic models, a discrete model was developed so that software agents have a discrete environment to operate within. Within the model, it has been observed how supermarkets with differing behaviors affect a heterogeneous crowd of consumer agents. The model was implemented in Java with Python used to evaluate the results. 

The simulation displays good acceptance with real grocery market behavior, i.e. captures the performance of British retailers thus can be used to determine the impact of changes in their behavior on their competitors and consumers.Furthermore it can be used to provide insight into sustainability of volatile pricing strategies, providing a useful insight in volatility of British supermarket retail industry. 
\end{abstract}
\acknowledgements{
I would like to express my sincere gratitude to Dr Maria Polukarov for her guidance and support which provided me the freedom to take this research in the direction of my interest.\\
\\
I would also like to thank my family and friends for their encouragement and support. To those who quietly listened to my software complaints. To those who worked throughout the nights with me. To those who helped me write what I couldn't say. I cannot thank you enough.
}

\declaration{
I, Stefan Collier, declare that this dissertation and the work presented in it are my own and has been generated by me as the result of my own original research.\\
I confirm that:\\
1. This work was done wholly or mainly while in candidature for a degree at this University;\\
2. Where any part of this dissertation has previously been submitted for any other qualification at this University or any other institution, this has been clearly stated;\\
3. Where I have consulted the published work of others, this is always clearly attributed;\\
4. Where I have quoted from the work of others, the source is always given. With the exception of such quotations, this dissertation is entirely my own work;\\
5. I have acknowledged all main sources of help;\\
6. Where the thesis is based on work done by myself jointly with others, I have made clear exactly what was done by others and what I have contributed myself;\\
7. Either none of this work has been published before submission, or parts of this work have been published by :\\
\\
Stefan Collier\\
April 2016
}
\tableofcontents
\listoffigures
\listoftables

\mainmatter
%% ----------------------------------------------------------------
%\include{Introduction}
%\include{Conclusions}
\include{chapters/1Project/main}
\include{chapters/2Lit/main}
\include{chapters/3Design/HighLevel}
\include{chapters/3Design/InDepth}
\include{chapters/4Impl/main}

\include{chapters/5Experiments/1/main}
\include{chapters/5Experiments/2/main}
\include{chapters/5Experiments/3/main}
\include{chapters/5Experiments/4/main}

\include{chapters/6Conclusion/main}

\appendix
\include{appendix/AppendixB}
\include{appendix/D/main}
\include{appendix/AppendixC}

\backmatter
\bibliographystyle{ecs}
\bibliography{ECS}
\end{document}
%% ----------------------------------------------------------------

\include{appendix/AppendixC}

\backmatter
\bibliographystyle{ecs}
\bibliography{ECS}
\end{document}
%% ----------------------------------------------------------------

\include{chapters/3Design/HighLevel}
\include{chapters/3Design/InDepth}
 %% ----------------------------------------------------------------
%% Progress.tex
%% ---------------------------------------------------------------- 
\documentclass{ecsprogress}    % Use the progress Style
\graphicspath{{../figs/}}   % Location of your graphics files
    \usepackage{natbib}            % Use Natbib style for the refs.
\hypersetup{colorlinks=true}   % Set to false for black/white printing
\input{Definitions}            % Include your abbreviations



\usepackage{enumitem}% http://ctan.org/pkg/enumitem
\usepackage{multirow}
\usepackage{float}
\usepackage{amsmath}
\usepackage{multicol}
\usepackage{amssymb}
\usepackage[normalem]{ulem}
\useunder{\uline}{\ul}{}
\usepackage{wrapfig}


\usepackage[table,xcdraw]{xcolor}


%% ----------------------------------------------------------------
\begin{document}
\frontmatter
\title      {Heterogeneous Agent-based Model for Supermarket Competition}
\authors    {\texorpdfstring
             {\href{mailto:sc22g13@ecs.soton.ac.uk}{Stefan J. Collier}}
             {Stefan J. Collier}
            }
\addresses  {\groupname\\\deptname\\\univname}
\date       {\today}
\subject    {}
\keywords   {}
\supervisor {Dr. Maria Polukarov}
\examiner   {Professor Sheng Chen}

\maketitle
\begin{abstract}
This project aim was to model and analyse the effects of competitive pricing behaviors of grocery retailers on the British market. 

This was achieved by creating a multi-agent model, containing retailer and consumer agents. The heterogeneous crowd of retailers employs either a uniform pricing strategy or a ‘local price flexing’ strategy. The actions of these retailers are chosen by predicting the profit of each action, using a perceptron. Following on from the consideration of different economic models, a discrete model was developed so that software agents have a discrete environment to operate within. Within the model, it has been observed how supermarkets with differing behaviors affect a heterogeneous crowd of consumer agents. The model was implemented in Java with Python used to evaluate the results. 

The simulation displays good acceptance with real grocery market behavior, i.e. captures the performance of British retailers thus can be used to determine the impact of changes in their behavior on their competitors and consumers.Furthermore it can be used to provide insight into sustainability of volatile pricing strategies, providing a useful insight in volatility of British supermarket retail industry. 
\end{abstract}
\acknowledgements{
I would like to express my sincere gratitude to Dr Maria Polukarov for her guidance and support which provided me the freedom to take this research in the direction of my interest.\\
\\
I would also like to thank my family and friends for their encouragement and support. To those who quietly listened to my software complaints. To those who worked throughout the nights with me. To those who helped me write what I couldn't say. I cannot thank you enough.
}

\declaration{
I, Stefan Collier, declare that this dissertation and the work presented in it are my own and has been generated by me as the result of my own original research.\\
I confirm that:\\
1. This work was done wholly or mainly while in candidature for a degree at this University;\\
2. Where any part of this dissertation has previously been submitted for any other qualification at this University or any other institution, this has been clearly stated;\\
3. Where I have consulted the published work of others, this is always clearly attributed;\\
4. Where I have quoted from the work of others, the source is always given. With the exception of such quotations, this dissertation is entirely my own work;\\
5. I have acknowledged all main sources of help;\\
6. Where the thesis is based on work done by myself jointly with others, I have made clear exactly what was done by others and what I have contributed myself;\\
7. Either none of this work has been published before submission, or parts of this work have been published by :\\
\\
Stefan Collier\\
April 2016
}
\tableofcontents
\listoffigures
\listoftables

\mainmatter
%% ----------------------------------------------------------------
%\include{Introduction}
%\include{Conclusions}
 %% ----------------------------------------------------------------
%% Progress.tex
%% ---------------------------------------------------------------- 
\documentclass{ecsprogress}    % Use the progress Style
\graphicspath{{../figs/}}   % Location of your graphics files
    \usepackage{natbib}            % Use Natbib style for the refs.
\hypersetup{colorlinks=true}   % Set to false for black/white printing
\input{Definitions}            % Include your abbreviations



\usepackage{enumitem}% http://ctan.org/pkg/enumitem
\usepackage{multirow}
\usepackage{float}
\usepackage{amsmath}
\usepackage{multicol}
\usepackage{amssymb}
\usepackage[normalem]{ulem}
\useunder{\uline}{\ul}{}
\usepackage{wrapfig}


\usepackage[table,xcdraw]{xcolor}


%% ----------------------------------------------------------------
\begin{document}
\frontmatter
\title      {Heterogeneous Agent-based Model for Supermarket Competition}
\authors    {\texorpdfstring
             {\href{mailto:sc22g13@ecs.soton.ac.uk}{Stefan J. Collier}}
             {Stefan J. Collier}
            }
\addresses  {\groupname\\\deptname\\\univname}
\date       {\today}
\subject    {}
\keywords   {}
\supervisor {Dr. Maria Polukarov}
\examiner   {Professor Sheng Chen}

\maketitle
\begin{abstract}
This project aim was to model and analyse the effects of competitive pricing behaviors of grocery retailers on the British market. 

This was achieved by creating a multi-agent model, containing retailer and consumer agents. The heterogeneous crowd of retailers employs either a uniform pricing strategy or a ‘local price flexing’ strategy. The actions of these retailers are chosen by predicting the profit of each action, using a perceptron. Following on from the consideration of different economic models, a discrete model was developed so that software agents have a discrete environment to operate within. Within the model, it has been observed how supermarkets with differing behaviors affect a heterogeneous crowd of consumer agents. The model was implemented in Java with Python used to evaluate the results. 

The simulation displays good acceptance with real grocery market behavior, i.e. captures the performance of British retailers thus can be used to determine the impact of changes in their behavior on their competitors and consumers.Furthermore it can be used to provide insight into sustainability of volatile pricing strategies, providing a useful insight in volatility of British supermarket retail industry. 
\end{abstract}
\acknowledgements{
I would like to express my sincere gratitude to Dr Maria Polukarov for her guidance and support which provided me the freedom to take this research in the direction of my interest.\\
\\
I would also like to thank my family and friends for their encouragement and support. To those who quietly listened to my software complaints. To those who worked throughout the nights with me. To those who helped me write what I couldn't say. I cannot thank you enough.
}

\declaration{
I, Stefan Collier, declare that this dissertation and the work presented in it are my own and has been generated by me as the result of my own original research.\\
I confirm that:\\
1. This work was done wholly or mainly while in candidature for a degree at this University;\\
2. Where any part of this dissertation has previously been submitted for any other qualification at this University or any other institution, this has been clearly stated;\\
3. Where I have consulted the published work of others, this is always clearly attributed;\\
4. Where I have quoted from the work of others, the source is always given. With the exception of such quotations, this dissertation is entirely my own work;\\
5. I have acknowledged all main sources of help;\\
6. Where the thesis is based on work done by myself jointly with others, I have made clear exactly what was done by others and what I have contributed myself;\\
7. Either none of this work has been published before submission, or parts of this work have been published by :\\
\\
Stefan Collier\\
April 2016
}
\tableofcontents
\listoffigures
\listoftables

\mainmatter
%% ----------------------------------------------------------------
%\include{Introduction}
%\include{Conclusions}
\include{chapters/1Project/main}
\include{chapters/2Lit/main}
\include{chapters/3Design/HighLevel}
\include{chapters/3Design/InDepth}
\include{chapters/4Impl/main}

\include{chapters/5Experiments/1/main}
\include{chapters/5Experiments/2/main}
\include{chapters/5Experiments/3/main}
\include{chapters/5Experiments/4/main}

\include{chapters/6Conclusion/main}

\appendix
\include{appendix/AppendixB}
\include{appendix/D/main}
\include{appendix/AppendixC}

\backmatter
\bibliographystyle{ecs}
\bibliography{ECS}
\end{document}
%% ----------------------------------------------------------------

 %% ----------------------------------------------------------------
%% Progress.tex
%% ---------------------------------------------------------------- 
\documentclass{ecsprogress}    % Use the progress Style
\graphicspath{{../figs/}}   % Location of your graphics files
    \usepackage{natbib}            % Use Natbib style for the refs.
\hypersetup{colorlinks=true}   % Set to false for black/white printing
\input{Definitions}            % Include your abbreviations



\usepackage{enumitem}% http://ctan.org/pkg/enumitem
\usepackage{multirow}
\usepackage{float}
\usepackage{amsmath}
\usepackage{multicol}
\usepackage{amssymb}
\usepackage[normalem]{ulem}
\useunder{\uline}{\ul}{}
\usepackage{wrapfig}


\usepackage[table,xcdraw]{xcolor}


%% ----------------------------------------------------------------
\begin{document}
\frontmatter
\title      {Heterogeneous Agent-based Model for Supermarket Competition}
\authors    {\texorpdfstring
             {\href{mailto:sc22g13@ecs.soton.ac.uk}{Stefan J. Collier}}
             {Stefan J. Collier}
            }
\addresses  {\groupname\\\deptname\\\univname}
\date       {\today}
\subject    {}
\keywords   {}
\supervisor {Dr. Maria Polukarov}
\examiner   {Professor Sheng Chen}

\maketitle
\begin{abstract}
This project aim was to model and analyse the effects of competitive pricing behaviors of grocery retailers on the British market. 

This was achieved by creating a multi-agent model, containing retailer and consumer agents. The heterogeneous crowd of retailers employs either a uniform pricing strategy or a ‘local price flexing’ strategy. The actions of these retailers are chosen by predicting the profit of each action, using a perceptron. Following on from the consideration of different economic models, a discrete model was developed so that software agents have a discrete environment to operate within. Within the model, it has been observed how supermarkets with differing behaviors affect a heterogeneous crowd of consumer agents. The model was implemented in Java with Python used to evaluate the results. 

The simulation displays good acceptance with real grocery market behavior, i.e. captures the performance of British retailers thus can be used to determine the impact of changes in their behavior on their competitors and consumers.Furthermore it can be used to provide insight into sustainability of volatile pricing strategies, providing a useful insight in volatility of British supermarket retail industry. 
\end{abstract}
\acknowledgements{
I would like to express my sincere gratitude to Dr Maria Polukarov for her guidance and support which provided me the freedom to take this research in the direction of my interest.\\
\\
I would also like to thank my family and friends for their encouragement and support. To those who quietly listened to my software complaints. To those who worked throughout the nights with me. To those who helped me write what I couldn't say. I cannot thank you enough.
}

\declaration{
I, Stefan Collier, declare that this dissertation and the work presented in it are my own and has been generated by me as the result of my own original research.\\
I confirm that:\\
1. This work was done wholly or mainly while in candidature for a degree at this University;\\
2. Where any part of this dissertation has previously been submitted for any other qualification at this University or any other institution, this has been clearly stated;\\
3. Where I have consulted the published work of others, this is always clearly attributed;\\
4. Where I have quoted from the work of others, the source is always given. With the exception of such quotations, this dissertation is entirely my own work;\\
5. I have acknowledged all main sources of help;\\
6. Where the thesis is based on work done by myself jointly with others, I have made clear exactly what was done by others and what I have contributed myself;\\
7. Either none of this work has been published before submission, or parts of this work have been published by :\\
\\
Stefan Collier\\
April 2016
}
\tableofcontents
\listoffigures
\listoftables

\mainmatter
%% ----------------------------------------------------------------
%\include{Introduction}
%\include{Conclusions}
\include{chapters/1Project/main}
\include{chapters/2Lit/main}
\include{chapters/3Design/HighLevel}
\include{chapters/3Design/InDepth}
\include{chapters/4Impl/main}

\include{chapters/5Experiments/1/main}
\include{chapters/5Experiments/2/main}
\include{chapters/5Experiments/3/main}
\include{chapters/5Experiments/4/main}

\include{chapters/6Conclusion/main}

\appendix
\include{appendix/AppendixB}
\include{appendix/D/main}
\include{appendix/AppendixC}

\backmatter
\bibliographystyle{ecs}
\bibliography{ECS}
\end{document}
%% ----------------------------------------------------------------

\include{chapters/3Design/HighLevel}
\include{chapters/3Design/InDepth}
 %% ----------------------------------------------------------------
%% Progress.tex
%% ---------------------------------------------------------------- 
\documentclass{ecsprogress}    % Use the progress Style
\graphicspath{{../figs/}}   % Location of your graphics files
    \usepackage{natbib}            % Use Natbib style for the refs.
\hypersetup{colorlinks=true}   % Set to false for black/white printing
\input{Definitions}            % Include your abbreviations



\usepackage{enumitem}% http://ctan.org/pkg/enumitem
\usepackage{multirow}
\usepackage{float}
\usepackage{amsmath}
\usepackage{multicol}
\usepackage{amssymb}
\usepackage[normalem]{ulem}
\useunder{\uline}{\ul}{}
\usepackage{wrapfig}


\usepackage[table,xcdraw]{xcolor}


%% ----------------------------------------------------------------
\begin{document}
\frontmatter
\title      {Heterogeneous Agent-based Model for Supermarket Competition}
\authors    {\texorpdfstring
             {\href{mailto:sc22g13@ecs.soton.ac.uk}{Stefan J. Collier}}
             {Stefan J. Collier}
            }
\addresses  {\groupname\\\deptname\\\univname}
\date       {\today}
\subject    {}
\keywords   {}
\supervisor {Dr. Maria Polukarov}
\examiner   {Professor Sheng Chen}

\maketitle
\begin{abstract}
This project aim was to model and analyse the effects of competitive pricing behaviors of grocery retailers on the British market. 

This was achieved by creating a multi-agent model, containing retailer and consumer agents. The heterogeneous crowd of retailers employs either a uniform pricing strategy or a ‘local price flexing’ strategy. The actions of these retailers are chosen by predicting the profit of each action, using a perceptron. Following on from the consideration of different economic models, a discrete model was developed so that software agents have a discrete environment to operate within. Within the model, it has been observed how supermarkets with differing behaviors affect a heterogeneous crowd of consumer agents. The model was implemented in Java with Python used to evaluate the results. 

The simulation displays good acceptance with real grocery market behavior, i.e. captures the performance of British retailers thus can be used to determine the impact of changes in their behavior on their competitors and consumers.Furthermore it can be used to provide insight into sustainability of volatile pricing strategies, providing a useful insight in volatility of British supermarket retail industry. 
\end{abstract}
\acknowledgements{
I would like to express my sincere gratitude to Dr Maria Polukarov for her guidance and support which provided me the freedom to take this research in the direction of my interest.\\
\\
I would also like to thank my family and friends for their encouragement and support. To those who quietly listened to my software complaints. To those who worked throughout the nights with me. To those who helped me write what I couldn't say. I cannot thank you enough.
}

\declaration{
I, Stefan Collier, declare that this dissertation and the work presented in it are my own and has been generated by me as the result of my own original research.\\
I confirm that:\\
1. This work was done wholly or mainly while in candidature for a degree at this University;\\
2. Where any part of this dissertation has previously been submitted for any other qualification at this University or any other institution, this has been clearly stated;\\
3. Where I have consulted the published work of others, this is always clearly attributed;\\
4. Where I have quoted from the work of others, the source is always given. With the exception of such quotations, this dissertation is entirely my own work;\\
5. I have acknowledged all main sources of help;\\
6. Where the thesis is based on work done by myself jointly with others, I have made clear exactly what was done by others and what I have contributed myself;\\
7. Either none of this work has been published before submission, or parts of this work have been published by :\\
\\
Stefan Collier\\
April 2016
}
\tableofcontents
\listoffigures
\listoftables

\mainmatter
%% ----------------------------------------------------------------
%\include{Introduction}
%\include{Conclusions}
\include{chapters/1Project/main}
\include{chapters/2Lit/main}
\include{chapters/3Design/HighLevel}
\include{chapters/3Design/InDepth}
\include{chapters/4Impl/main}

\include{chapters/5Experiments/1/main}
\include{chapters/5Experiments/2/main}
\include{chapters/5Experiments/3/main}
\include{chapters/5Experiments/4/main}

\include{chapters/6Conclusion/main}

\appendix
\include{appendix/AppendixB}
\include{appendix/D/main}
\include{appendix/AppendixC}

\backmatter
\bibliographystyle{ecs}
\bibliography{ECS}
\end{document}
%% ----------------------------------------------------------------


 %% ----------------------------------------------------------------
%% Progress.tex
%% ---------------------------------------------------------------- 
\documentclass{ecsprogress}    % Use the progress Style
\graphicspath{{../figs/}}   % Location of your graphics files
    \usepackage{natbib}            % Use Natbib style for the refs.
\hypersetup{colorlinks=true}   % Set to false for black/white printing
\input{Definitions}            % Include your abbreviations



\usepackage{enumitem}% http://ctan.org/pkg/enumitem
\usepackage{multirow}
\usepackage{float}
\usepackage{amsmath}
\usepackage{multicol}
\usepackage{amssymb}
\usepackage[normalem]{ulem}
\useunder{\uline}{\ul}{}
\usepackage{wrapfig}


\usepackage[table,xcdraw]{xcolor}


%% ----------------------------------------------------------------
\begin{document}
\frontmatter
\title      {Heterogeneous Agent-based Model for Supermarket Competition}
\authors    {\texorpdfstring
             {\href{mailto:sc22g13@ecs.soton.ac.uk}{Stefan J. Collier}}
             {Stefan J. Collier}
            }
\addresses  {\groupname\\\deptname\\\univname}
\date       {\today}
\subject    {}
\keywords   {}
\supervisor {Dr. Maria Polukarov}
\examiner   {Professor Sheng Chen}

\maketitle
\begin{abstract}
This project aim was to model and analyse the effects of competitive pricing behaviors of grocery retailers on the British market. 

This was achieved by creating a multi-agent model, containing retailer and consumer agents. The heterogeneous crowd of retailers employs either a uniform pricing strategy or a ‘local price flexing’ strategy. The actions of these retailers are chosen by predicting the profit of each action, using a perceptron. Following on from the consideration of different economic models, a discrete model was developed so that software agents have a discrete environment to operate within. Within the model, it has been observed how supermarkets with differing behaviors affect a heterogeneous crowd of consumer agents. The model was implemented in Java with Python used to evaluate the results. 

The simulation displays good acceptance with real grocery market behavior, i.e. captures the performance of British retailers thus can be used to determine the impact of changes in their behavior on their competitors and consumers.Furthermore it can be used to provide insight into sustainability of volatile pricing strategies, providing a useful insight in volatility of British supermarket retail industry. 
\end{abstract}
\acknowledgements{
I would like to express my sincere gratitude to Dr Maria Polukarov for her guidance and support which provided me the freedom to take this research in the direction of my interest.\\
\\
I would also like to thank my family and friends for their encouragement and support. To those who quietly listened to my software complaints. To those who worked throughout the nights with me. To those who helped me write what I couldn't say. I cannot thank you enough.
}

\declaration{
I, Stefan Collier, declare that this dissertation and the work presented in it are my own and has been generated by me as the result of my own original research.\\
I confirm that:\\
1. This work was done wholly or mainly while in candidature for a degree at this University;\\
2. Where any part of this dissertation has previously been submitted for any other qualification at this University or any other institution, this has been clearly stated;\\
3. Where I have consulted the published work of others, this is always clearly attributed;\\
4. Where I have quoted from the work of others, the source is always given. With the exception of such quotations, this dissertation is entirely my own work;\\
5. I have acknowledged all main sources of help;\\
6. Where the thesis is based on work done by myself jointly with others, I have made clear exactly what was done by others and what I have contributed myself;\\
7. Either none of this work has been published before submission, or parts of this work have been published by :\\
\\
Stefan Collier\\
April 2016
}
\tableofcontents
\listoffigures
\listoftables

\mainmatter
%% ----------------------------------------------------------------
%\include{Introduction}
%\include{Conclusions}
\include{chapters/1Project/main}
\include{chapters/2Lit/main}
\include{chapters/3Design/HighLevel}
\include{chapters/3Design/InDepth}
\include{chapters/4Impl/main}

\include{chapters/5Experiments/1/main}
\include{chapters/5Experiments/2/main}
\include{chapters/5Experiments/3/main}
\include{chapters/5Experiments/4/main}

\include{chapters/6Conclusion/main}

\appendix
\include{appendix/AppendixB}
\include{appendix/D/main}
\include{appendix/AppendixC}

\backmatter
\bibliographystyle{ecs}
\bibliography{ECS}
\end{document}
%% ----------------------------------------------------------------

 %% ----------------------------------------------------------------
%% Progress.tex
%% ---------------------------------------------------------------- 
\documentclass{ecsprogress}    % Use the progress Style
\graphicspath{{../figs/}}   % Location of your graphics files
    \usepackage{natbib}            % Use Natbib style for the refs.
\hypersetup{colorlinks=true}   % Set to false for black/white printing
\input{Definitions}            % Include your abbreviations



\usepackage{enumitem}% http://ctan.org/pkg/enumitem
\usepackage{multirow}
\usepackage{float}
\usepackage{amsmath}
\usepackage{multicol}
\usepackage{amssymb}
\usepackage[normalem]{ulem}
\useunder{\uline}{\ul}{}
\usepackage{wrapfig}


\usepackage[table,xcdraw]{xcolor}


%% ----------------------------------------------------------------
\begin{document}
\frontmatter
\title      {Heterogeneous Agent-based Model for Supermarket Competition}
\authors    {\texorpdfstring
             {\href{mailto:sc22g13@ecs.soton.ac.uk}{Stefan J. Collier}}
             {Stefan J. Collier}
            }
\addresses  {\groupname\\\deptname\\\univname}
\date       {\today}
\subject    {}
\keywords   {}
\supervisor {Dr. Maria Polukarov}
\examiner   {Professor Sheng Chen}

\maketitle
\begin{abstract}
This project aim was to model and analyse the effects of competitive pricing behaviors of grocery retailers on the British market. 

This was achieved by creating a multi-agent model, containing retailer and consumer agents. The heterogeneous crowd of retailers employs either a uniform pricing strategy or a ‘local price flexing’ strategy. The actions of these retailers are chosen by predicting the profit of each action, using a perceptron. Following on from the consideration of different economic models, a discrete model was developed so that software agents have a discrete environment to operate within. Within the model, it has been observed how supermarkets with differing behaviors affect a heterogeneous crowd of consumer agents. The model was implemented in Java with Python used to evaluate the results. 

The simulation displays good acceptance with real grocery market behavior, i.e. captures the performance of British retailers thus can be used to determine the impact of changes in their behavior on their competitors and consumers.Furthermore it can be used to provide insight into sustainability of volatile pricing strategies, providing a useful insight in volatility of British supermarket retail industry. 
\end{abstract}
\acknowledgements{
I would like to express my sincere gratitude to Dr Maria Polukarov for her guidance and support which provided me the freedom to take this research in the direction of my interest.\\
\\
I would also like to thank my family and friends for their encouragement and support. To those who quietly listened to my software complaints. To those who worked throughout the nights with me. To those who helped me write what I couldn't say. I cannot thank you enough.
}

\declaration{
I, Stefan Collier, declare that this dissertation and the work presented in it are my own and has been generated by me as the result of my own original research.\\
I confirm that:\\
1. This work was done wholly or mainly while in candidature for a degree at this University;\\
2. Where any part of this dissertation has previously been submitted for any other qualification at this University or any other institution, this has been clearly stated;\\
3. Where I have consulted the published work of others, this is always clearly attributed;\\
4. Where I have quoted from the work of others, the source is always given. With the exception of such quotations, this dissertation is entirely my own work;\\
5. I have acknowledged all main sources of help;\\
6. Where the thesis is based on work done by myself jointly with others, I have made clear exactly what was done by others and what I have contributed myself;\\
7. Either none of this work has been published before submission, or parts of this work have been published by :\\
\\
Stefan Collier\\
April 2016
}
\tableofcontents
\listoffigures
\listoftables

\mainmatter
%% ----------------------------------------------------------------
%\include{Introduction}
%\include{Conclusions}
\include{chapters/1Project/main}
\include{chapters/2Lit/main}
\include{chapters/3Design/HighLevel}
\include{chapters/3Design/InDepth}
\include{chapters/4Impl/main}

\include{chapters/5Experiments/1/main}
\include{chapters/5Experiments/2/main}
\include{chapters/5Experiments/3/main}
\include{chapters/5Experiments/4/main}

\include{chapters/6Conclusion/main}

\appendix
\include{appendix/AppendixB}
\include{appendix/D/main}
\include{appendix/AppendixC}

\backmatter
\bibliographystyle{ecs}
\bibliography{ECS}
\end{document}
%% ----------------------------------------------------------------

 %% ----------------------------------------------------------------
%% Progress.tex
%% ---------------------------------------------------------------- 
\documentclass{ecsprogress}    % Use the progress Style
\graphicspath{{../figs/}}   % Location of your graphics files
    \usepackage{natbib}            % Use Natbib style for the refs.
\hypersetup{colorlinks=true}   % Set to false for black/white printing
\input{Definitions}            % Include your abbreviations



\usepackage{enumitem}% http://ctan.org/pkg/enumitem
\usepackage{multirow}
\usepackage{float}
\usepackage{amsmath}
\usepackage{multicol}
\usepackage{amssymb}
\usepackage[normalem]{ulem}
\useunder{\uline}{\ul}{}
\usepackage{wrapfig}


\usepackage[table,xcdraw]{xcolor}


%% ----------------------------------------------------------------
\begin{document}
\frontmatter
\title      {Heterogeneous Agent-based Model for Supermarket Competition}
\authors    {\texorpdfstring
             {\href{mailto:sc22g13@ecs.soton.ac.uk}{Stefan J. Collier}}
             {Stefan J. Collier}
            }
\addresses  {\groupname\\\deptname\\\univname}
\date       {\today}
\subject    {}
\keywords   {}
\supervisor {Dr. Maria Polukarov}
\examiner   {Professor Sheng Chen}

\maketitle
\begin{abstract}
This project aim was to model and analyse the effects of competitive pricing behaviors of grocery retailers on the British market. 

This was achieved by creating a multi-agent model, containing retailer and consumer agents. The heterogeneous crowd of retailers employs either a uniform pricing strategy or a ‘local price flexing’ strategy. The actions of these retailers are chosen by predicting the profit of each action, using a perceptron. Following on from the consideration of different economic models, a discrete model was developed so that software agents have a discrete environment to operate within. Within the model, it has been observed how supermarkets with differing behaviors affect a heterogeneous crowd of consumer agents. The model was implemented in Java with Python used to evaluate the results. 

The simulation displays good acceptance with real grocery market behavior, i.e. captures the performance of British retailers thus can be used to determine the impact of changes in their behavior on their competitors and consumers.Furthermore it can be used to provide insight into sustainability of volatile pricing strategies, providing a useful insight in volatility of British supermarket retail industry. 
\end{abstract}
\acknowledgements{
I would like to express my sincere gratitude to Dr Maria Polukarov for her guidance and support which provided me the freedom to take this research in the direction of my interest.\\
\\
I would also like to thank my family and friends for their encouragement and support. To those who quietly listened to my software complaints. To those who worked throughout the nights with me. To those who helped me write what I couldn't say. I cannot thank you enough.
}

\declaration{
I, Stefan Collier, declare that this dissertation and the work presented in it are my own and has been generated by me as the result of my own original research.\\
I confirm that:\\
1. This work was done wholly or mainly while in candidature for a degree at this University;\\
2. Where any part of this dissertation has previously been submitted for any other qualification at this University or any other institution, this has been clearly stated;\\
3. Where I have consulted the published work of others, this is always clearly attributed;\\
4. Where I have quoted from the work of others, the source is always given. With the exception of such quotations, this dissertation is entirely my own work;\\
5. I have acknowledged all main sources of help;\\
6. Where the thesis is based on work done by myself jointly with others, I have made clear exactly what was done by others and what I have contributed myself;\\
7. Either none of this work has been published before submission, or parts of this work have been published by :\\
\\
Stefan Collier\\
April 2016
}
\tableofcontents
\listoffigures
\listoftables

\mainmatter
%% ----------------------------------------------------------------
%\include{Introduction}
%\include{Conclusions}
\include{chapters/1Project/main}
\include{chapters/2Lit/main}
\include{chapters/3Design/HighLevel}
\include{chapters/3Design/InDepth}
\include{chapters/4Impl/main}

\include{chapters/5Experiments/1/main}
\include{chapters/5Experiments/2/main}
\include{chapters/5Experiments/3/main}
\include{chapters/5Experiments/4/main}

\include{chapters/6Conclusion/main}

\appendix
\include{appendix/AppendixB}
\include{appendix/D/main}
\include{appendix/AppendixC}

\backmatter
\bibliographystyle{ecs}
\bibliography{ECS}
\end{document}
%% ----------------------------------------------------------------

 %% ----------------------------------------------------------------
%% Progress.tex
%% ---------------------------------------------------------------- 
\documentclass{ecsprogress}    % Use the progress Style
\graphicspath{{../figs/}}   % Location of your graphics files
    \usepackage{natbib}            % Use Natbib style for the refs.
\hypersetup{colorlinks=true}   % Set to false for black/white printing
\input{Definitions}            % Include your abbreviations



\usepackage{enumitem}% http://ctan.org/pkg/enumitem
\usepackage{multirow}
\usepackage{float}
\usepackage{amsmath}
\usepackage{multicol}
\usepackage{amssymb}
\usepackage[normalem]{ulem}
\useunder{\uline}{\ul}{}
\usepackage{wrapfig}


\usepackage[table,xcdraw]{xcolor}


%% ----------------------------------------------------------------
\begin{document}
\frontmatter
\title      {Heterogeneous Agent-based Model for Supermarket Competition}
\authors    {\texorpdfstring
             {\href{mailto:sc22g13@ecs.soton.ac.uk}{Stefan J. Collier}}
             {Stefan J. Collier}
            }
\addresses  {\groupname\\\deptname\\\univname}
\date       {\today}
\subject    {}
\keywords   {}
\supervisor {Dr. Maria Polukarov}
\examiner   {Professor Sheng Chen}

\maketitle
\begin{abstract}
This project aim was to model and analyse the effects of competitive pricing behaviors of grocery retailers on the British market. 

This was achieved by creating a multi-agent model, containing retailer and consumer agents. The heterogeneous crowd of retailers employs either a uniform pricing strategy or a ‘local price flexing’ strategy. The actions of these retailers are chosen by predicting the profit of each action, using a perceptron. Following on from the consideration of different economic models, a discrete model was developed so that software agents have a discrete environment to operate within. Within the model, it has been observed how supermarkets with differing behaviors affect a heterogeneous crowd of consumer agents. The model was implemented in Java with Python used to evaluate the results. 

The simulation displays good acceptance with real grocery market behavior, i.e. captures the performance of British retailers thus can be used to determine the impact of changes in their behavior on their competitors and consumers.Furthermore it can be used to provide insight into sustainability of volatile pricing strategies, providing a useful insight in volatility of British supermarket retail industry. 
\end{abstract}
\acknowledgements{
I would like to express my sincere gratitude to Dr Maria Polukarov for her guidance and support which provided me the freedom to take this research in the direction of my interest.\\
\\
I would also like to thank my family and friends for their encouragement and support. To those who quietly listened to my software complaints. To those who worked throughout the nights with me. To those who helped me write what I couldn't say. I cannot thank you enough.
}

\declaration{
I, Stefan Collier, declare that this dissertation and the work presented in it are my own and has been generated by me as the result of my own original research.\\
I confirm that:\\
1. This work was done wholly or mainly while in candidature for a degree at this University;\\
2. Where any part of this dissertation has previously been submitted for any other qualification at this University or any other institution, this has been clearly stated;\\
3. Where I have consulted the published work of others, this is always clearly attributed;\\
4. Where I have quoted from the work of others, the source is always given. With the exception of such quotations, this dissertation is entirely my own work;\\
5. I have acknowledged all main sources of help;\\
6. Where the thesis is based on work done by myself jointly with others, I have made clear exactly what was done by others and what I have contributed myself;\\
7. Either none of this work has been published before submission, or parts of this work have been published by :\\
\\
Stefan Collier\\
April 2016
}
\tableofcontents
\listoffigures
\listoftables

\mainmatter
%% ----------------------------------------------------------------
%\include{Introduction}
%\include{Conclusions}
\include{chapters/1Project/main}
\include{chapters/2Lit/main}
\include{chapters/3Design/HighLevel}
\include{chapters/3Design/InDepth}
\include{chapters/4Impl/main}

\include{chapters/5Experiments/1/main}
\include{chapters/5Experiments/2/main}
\include{chapters/5Experiments/3/main}
\include{chapters/5Experiments/4/main}

\include{chapters/6Conclusion/main}

\appendix
\include{appendix/AppendixB}
\include{appendix/D/main}
\include{appendix/AppendixC}

\backmatter
\bibliographystyle{ecs}
\bibliography{ECS}
\end{document}
%% ----------------------------------------------------------------


 %% ----------------------------------------------------------------
%% Progress.tex
%% ---------------------------------------------------------------- 
\documentclass{ecsprogress}    % Use the progress Style
\graphicspath{{../figs/}}   % Location of your graphics files
    \usepackage{natbib}            % Use Natbib style for the refs.
\hypersetup{colorlinks=true}   % Set to false for black/white printing
\input{Definitions}            % Include your abbreviations



\usepackage{enumitem}% http://ctan.org/pkg/enumitem
\usepackage{multirow}
\usepackage{float}
\usepackage{amsmath}
\usepackage{multicol}
\usepackage{amssymb}
\usepackage[normalem]{ulem}
\useunder{\uline}{\ul}{}
\usepackage{wrapfig}


\usepackage[table,xcdraw]{xcolor}


%% ----------------------------------------------------------------
\begin{document}
\frontmatter
\title      {Heterogeneous Agent-based Model for Supermarket Competition}
\authors    {\texorpdfstring
             {\href{mailto:sc22g13@ecs.soton.ac.uk}{Stefan J. Collier}}
             {Stefan J. Collier}
            }
\addresses  {\groupname\\\deptname\\\univname}
\date       {\today}
\subject    {}
\keywords   {}
\supervisor {Dr. Maria Polukarov}
\examiner   {Professor Sheng Chen}

\maketitle
\begin{abstract}
This project aim was to model and analyse the effects of competitive pricing behaviors of grocery retailers on the British market. 

This was achieved by creating a multi-agent model, containing retailer and consumer agents. The heterogeneous crowd of retailers employs either a uniform pricing strategy or a ‘local price flexing’ strategy. The actions of these retailers are chosen by predicting the profit of each action, using a perceptron. Following on from the consideration of different economic models, a discrete model was developed so that software agents have a discrete environment to operate within. Within the model, it has been observed how supermarkets with differing behaviors affect a heterogeneous crowd of consumer agents. The model was implemented in Java with Python used to evaluate the results. 

The simulation displays good acceptance with real grocery market behavior, i.e. captures the performance of British retailers thus can be used to determine the impact of changes in their behavior on their competitors and consumers.Furthermore it can be used to provide insight into sustainability of volatile pricing strategies, providing a useful insight in volatility of British supermarket retail industry. 
\end{abstract}
\acknowledgements{
I would like to express my sincere gratitude to Dr Maria Polukarov for her guidance and support which provided me the freedom to take this research in the direction of my interest.\\
\\
I would also like to thank my family and friends for their encouragement and support. To those who quietly listened to my software complaints. To those who worked throughout the nights with me. To those who helped me write what I couldn't say. I cannot thank you enough.
}

\declaration{
I, Stefan Collier, declare that this dissertation and the work presented in it are my own and has been generated by me as the result of my own original research.\\
I confirm that:\\
1. This work was done wholly or mainly while in candidature for a degree at this University;\\
2. Where any part of this dissertation has previously been submitted for any other qualification at this University or any other institution, this has been clearly stated;\\
3. Where I have consulted the published work of others, this is always clearly attributed;\\
4. Where I have quoted from the work of others, the source is always given. With the exception of such quotations, this dissertation is entirely my own work;\\
5. I have acknowledged all main sources of help;\\
6. Where the thesis is based on work done by myself jointly with others, I have made clear exactly what was done by others and what I have contributed myself;\\
7. Either none of this work has been published before submission, or parts of this work have been published by :\\
\\
Stefan Collier\\
April 2016
}
\tableofcontents
\listoffigures
\listoftables

\mainmatter
%% ----------------------------------------------------------------
%\include{Introduction}
%\include{Conclusions}
\include{chapters/1Project/main}
\include{chapters/2Lit/main}
\include{chapters/3Design/HighLevel}
\include{chapters/3Design/InDepth}
\include{chapters/4Impl/main}

\include{chapters/5Experiments/1/main}
\include{chapters/5Experiments/2/main}
\include{chapters/5Experiments/3/main}
\include{chapters/5Experiments/4/main}

\include{chapters/6Conclusion/main}

\appendix
\include{appendix/AppendixB}
\include{appendix/D/main}
\include{appendix/AppendixC}

\backmatter
\bibliographystyle{ecs}
\bibliography{ECS}
\end{document}
%% ----------------------------------------------------------------


\appendix
\include{appendix/AppendixB}
 %% ----------------------------------------------------------------
%% Progress.tex
%% ---------------------------------------------------------------- 
\documentclass{ecsprogress}    % Use the progress Style
\graphicspath{{../figs/}}   % Location of your graphics files
    \usepackage{natbib}            % Use Natbib style for the refs.
\hypersetup{colorlinks=true}   % Set to false for black/white printing
\input{Definitions}            % Include your abbreviations



\usepackage{enumitem}% http://ctan.org/pkg/enumitem
\usepackage{multirow}
\usepackage{float}
\usepackage{amsmath}
\usepackage{multicol}
\usepackage{amssymb}
\usepackage[normalem]{ulem}
\useunder{\uline}{\ul}{}
\usepackage{wrapfig}


\usepackage[table,xcdraw]{xcolor}


%% ----------------------------------------------------------------
\begin{document}
\frontmatter
\title      {Heterogeneous Agent-based Model for Supermarket Competition}
\authors    {\texorpdfstring
             {\href{mailto:sc22g13@ecs.soton.ac.uk}{Stefan J. Collier}}
             {Stefan J. Collier}
            }
\addresses  {\groupname\\\deptname\\\univname}
\date       {\today}
\subject    {}
\keywords   {}
\supervisor {Dr. Maria Polukarov}
\examiner   {Professor Sheng Chen}

\maketitle
\begin{abstract}
This project aim was to model and analyse the effects of competitive pricing behaviors of grocery retailers on the British market. 

This was achieved by creating a multi-agent model, containing retailer and consumer agents. The heterogeneous crowd of retailers employs either a uniform pricing strategy or a ‘local price flexing’ strategy. The actions of these retailers are chosen by predicting the profit of each action, using a perceptron. Following on from the consideration of different economic models, a discrete model was developed so that software agents have a discrete environment to operate within. Within the model, it has been observed how supermarkets with differing behaviors affect a heterogeneous crowd of consumer agents. The model was implemented in Java with Python used to evaluate the results. 

The simulation displays good acceptance with real grocery market behavior, i.e. captures the performance of British retailers thus can be used to determine the impact of changes in their behavior on their competitors and consumers.Furthermore it can be used to provide insight into sustainability of volatile pricing strategies, providing a useful insight in volatility of British supermarket retail industry. 
\end{abstract}
\acknowledgements{
I would like to express my sincere gratitude to Dr Maria Polukarov for her guidance and support which provided me the freedom to take this research in the direction of my interest.\\
\\
I would also like to thank my family and friends for their encouragement and support. To those who quietly listened to my software complaints. To those who worked throughout the nights with me. To those who helped me write what I couldn't say. I cannot thank you enough.
}

\declaration{
I, Stefan Collier, declare that this dissertation and the work presented in it are my own and has been generated by me as the result of my own original research.\\
I confirm that:\\
1. This work was done wholly or mainly while in candidature for a degree at this University;\\
2. Where any part of this dissertation has previously been submitted for any other qualification at this University or any other institution, this has been clearly stated;\\
3. Where I have consulted the published work of others, this is always clearly attributed;\\
4. Where I have quoted from the work of others, the source is always given. With the exception of such quotations, this dissertation is entirely my own work;\\
5. I have acknowledged all main sources of help;\\
6. Where the thesis is based on work done by myself jointly with others, I have made clear exactly what was done by others and what I have contributed myself;\\
7. Either none of this work has been published before submission, or parts of this work have been published by :\\
\\
Stefan Collier\\
April 2016
}
\tableofcontents
\listoffigures
\listoftables

\mainmatter
%% ----------------------------------------------------------------
%\include{Introduction}
%\include{Conclusions}
\include{chapters/1Project/main}
\include{chapters/2Lit/main}
\include{chapters/3Design/HighLevel}
\include{chapters/3Design/InDepth}
\include{chapters/4Impl/main}

\include{chapters/5Experiments/1/main}
\include{chapters/5Experiments/2/main}
\include{chapters/5Experiments/3/main}
\include{chapters/5Experiments/4/main}

\include{chapters/6Conclusion/main}

\appendix
\include{appendix/AppendixB}
\include{appendix/D/main}
\include{appendix/AppendixC}

\backmatter
\bibliographystyle{ecs}
\bibliography{ECS}
\end{document}
%% ----------------------------------------------------------------

\include{appendix/AppendixC}

\backmatter
\bibliographystyle{ecs}
\bibliography{ECS}
\end{document}
%% ----------------------------------------------------------------


 %% ----------------------------------------------------------------
%% Progress.tex
%% ---------------------------------------------------------------- 
\documentclass{ecsprogress}    % Use the progress Style
\graphicspath{{../figs/}}   % Location of your graphics files
    \usepackage{natbib}            % Use Natbib style for the refs.
\hypersetup{colorlinks=true}   % Set to false for black/white printing
\input{Definitions}            % Include your abbreviations



\usepackage{enumitem}% http://ctan.org/pkg/enumitem
\usepackage{multirow}
\usepackage{float}
\usepackage{amsmath}
\usepackage{multicol}
\usepackage{amssymb}
\usepackage[normalem]{ulem}
\useunder{\uline}{\ul}{}
\usepackage{wrapfig}


\usepackage[table,xcdraw]{xcolor}


%% ----------------------------------------------------------------
\begin{document}
\frontmatter
\title      {Heterogeneous Agent-based Model for Supermarket Competition}
\authors    {\texorpdfstring
             {\href{mailto:sc22g13@ecs.soton.ac.uk}{Stefan J. Collier}}
             {Stefan J. Collier}
            }
\addresses  {\groupname\\\deptname\\\univname}
\date       {\today}
\subject    {}
\keywords   {}
\supervisor {Dr. Maria Polukarov}
\examiner   {Professor Sheng Chen}

\maketitle
\begin{abstract}
This project aim was to model and analyse the effects of competitive pricing behaviors of grocery retailers on the British market. 

This was achieved by creating a multi-agent model, containing retailer and consumer agents. The heterogeneous crowd of retailers employs either a uniform pricing strategy or a ‘local price flexing’ strategy. The actions of these retailers are chosen by predicting the profit of each action, using a perceptron. Following on from the consideration of different economic models, a discrete model was developed so that software agents have a discrete environment to operate within. Within the model, it has been observed how supermarkets with differing behaviors affect a heterogeneous crowd of consumer agents. The model was implemented in Java with Python used to evaluate the results. 

The simulation displays good acceptance with real grocery market behavior, i.e. captures the performance of British retailers thus can be used to determine the impact of changes in their behavior on their competitors and consumers.Furthermore it can be used to provide insight into sustainability of volatile pricing strategies, providing a useful insight in volatility of British supermarket retail industry. 
\end{abstract}
\acknowledgements{
I would like to express my sincere gratitude to Dr Maria Polukarov for her guidance and support which provided me the freedom to take this research in the direction of my interest.\\
\\
I would also like to thank my family and friends for their encouragement and support. To those who quietly listened to my software complaints. To those who worked throughout the nights with me. To those who helped me write what I couldn't say. I cannot thank you enough.
}

\declaration{
I, Stefan Collier, declare that this dissertation and the work presented in it are my own and has been generated by me as the result of my own original research.\\
I confirm that:\\
1. This work was done wholly or mainly while in candidature for a degree at this University;\\
2. Where any part of this dissertation has previously been submitted for any other qualification at this University or any other institution, this has been clearly stated;\\
3. Where I have consulted the published work of others, this is always clearly attributed;\\
4. Where I have quoted from the work of others, the source is always given. With the exception of such quotations, this dissertation is entirely my own work;\\
5. I have acknowledged all main sources of help;\\
6. Where the thesis is based on work done by myself jointly with others, I have made clear exactly what was done by others and what I have contributed myself;\\
7. Either none of this work has been published before submission, or parts of this work have been published by :\\
\\
Stefan Collier\\
April 2016
}
\tableofcontents
\listoffigures
\listoftables

\mainmatter
%% ----------------------------------------------------------------
%\include{Introduction}
%\include{Conclusions}
 %% ----------------------------------------------------------------
%% Progress.tex
%% ---------------------------------------------------------------- 
\documentclass{ecsprogress}    % Use the progress Style
\graphicspath{{../figs/}}   % Location of your graphics files
    \usepackage{natbib}            % Use Natbib style for the refs.
\hypersetup{colorlinks=true}   % Set to false for black/white printing
\input{Definitions}            % Include your abbreviations



\usepackage{enumitem}% http://ctan.org/pkg/enumitem
\usepackage{multirow}
\usepackage{float}
\usepackage{amsmath}
\usepackage{multicol}
\usepackage{amssymb}
\usepackage[normalem]{ulem}
\useunder{\uline}{\ul}{}
\usepackage{wrapfig}


\usepackage[table,xcdraw]{xcolor}


%% ----------------------------------------------------------------
\begin{document}
\frontmatter
\title      {Heterogeneous Agent-based Model for Supermarket Competition}
\authors    {\texorpdfstring
             {\href{mailto:sc22g13@ecs.soton.ac.uk}{Stefan J. Collier}}
             {Stefan J. Collier}
            }
\addresses  {\groupname\\\deptname\\\univname}
\date       {\today}
\subject    {}
\keywords   {}
\supervisor {Dr. Maria Polukarov}
\examiner   {Professor Sheng Chen}

\maketitle
\begin{abstract}
This project aim was to model and analyse the effects of competitive pricing behaviors of grocery retailers on the British market. 

This was achieved by creating a multi-agent model, containing retailer and consumer agents. The heterogeneous crowd of retailers employs either a uniform pricing strategy or a ‘local price flexing’ strategy. The actions of these retailers are chosen by predicting the profit of each action, using a perceptron. Following on from the consideration of different economic models, a discrete model was developed so that software agents have a discrete environment to operate within. Within the model, it has been observed how supermarkets with differing behaviors affect a heterogeneous crowd of consumer agents. The model was implemented in Java with Python used to evaluate the results. 

The simulation displays good acceptance with real grocery market behavior, i.e. captures the performance of British retailers thus can be used to determine the impact of changes in their behavior on their competitors and consumers.Furthermore it can be used to provide insight into sustainability of volatile pricing strategies, providing a useful insight in volatility of British supermarket retail industry. 
\end{abstract}
\acknowledgements{
I would like to express my sincere gratitude to Dr Maria Polukarov for her guidance and support which provided me the freedom to take this research in the direction of my interest.\\
\\
I would also like to thank my family and friends for their encouragement and support. To those who quietly listened to my software complaints. To those who worked throughout the nights with me. To those who helped me write what I couldn't say. I cannot thank you enough.
}

\declaration{
I, Stefan Collier, declare that this dissertation and the work presented in it are my own and has been generated by me as the result of my own original research.\\
I confirm that:\\
1. This work was done wholly or mainly while in candidature for a degree at this University;\\
2. Where any part of this dissertation has previously been submitted for any other qualification at this University or any other institution, this has been clearly stated;\\
3. Where I have consulted the published work of others, this is always clearly attributed;\\
4. Where I have quoted from the work of others, the source is always given. With the exception of such quotations, this dissertation is entirely my own work;\\
5. I have acknowledged all main sources of help;\\
6. Where the thesis is based on work done by myself jointly with others, I have made clear exactly what was done by others and what I have contributed myself;\\
7. Either none of this work has been published before submission, or parts of this work have been published by :\\
\\
Stefan Collier\\
April 2016
}
\tableofcontents
\listoffigures
\listoftables

\mainmatter
%% ----------------------------------------------------------------
%\include{Introduction}
%\include{Conclusions}
\include{chapters/1Project/main}
\include{chapters/2Lit/main}
\include{chapters/3Design/HighLevel}
\include{chapters/3Design/InDepth}
\include{chapters/4Impl/main}

\include{chapters/5Experiments/1/main}
\include{chapters/5Experiments/2/main}
\include{chapters/5Experiments/3/main}
\include{chapters/5Experiments/4/main}

\include{chapters/6Conclusion/main}

\appendix
\include{appendix/AppendixB}
\include{appendix/D/main}
\include{appendix/AppendixC}

\backmatter
\bibliographystyle{ecs}
\bibliography{ECS}
\end{document}
%% ----------------------------------------------------------------

 %% ----------------------------------------------------------------
%% Progress.tex
%% ---------------------------------------------------------------- 
\documentclass{ecsprogress}    % Use the progress Style
\graphicspath{{../figs/}}   % Location of your graphics files
    \usepackage{natbib}            % Use Natbib style for the refs.
\hypersetup{colorlinks=true}   % Set to false for black/white printing
\input{Definitions}            % Include your abbreviations



\usepackage{enumitem}% http://ctan.org/pkg/enumitem
\usepackage{multirow}
\usepackage{float}
\usepackage{amsmath}
\usepackage{multicol}
\usepackage{amssymb}
\usepackage[normalem]{ulem}
\useunder{\uline}{\ul}{}
\usepackage{wrapfig}


\usepackage[table,xcdraw]{xcolor}


%% ----------------------------------------------------------------
\begin{document}
\frontmatter
\title      {Heterogeneous Agent-based Model for Supermarket Competition}
\authors    {\texorpdfstring
             {\href{mailto:sc22g13@ecs.soton.ac.uk}{Stefan J. Collier}}
             {Stefan J. Collier}
            }
\addresses  {\groupname\\\deptname\\\univname}
\date       {\today}
\subject    {}
\keywords   {}
\supervisor {Dr. Maria Polukarov}
\examiner   {Professor Sheng Chen}

\maketitle
\begin{abstract}
This project aim was to model and analyse the effects of competitive pricing behaviors of grocery retailers on the British market. 

This was achieved by creating a multi-agent model, containing retailer and consumer agents. The heterogeneous crowd of retailers employs either a uniform pricing strategy or a ‘local price flexing’ strategy. The actions of these retailers are chosen by predicting the profit of each action, using a perceptron. Following on from the consideration of different economic models, a discrete model was developed so that software agents have a discrete environment to operate within. Within the model, it has been observed how supermarkets with differing behaviors affect a heterogeneous crowd of consumer agents. The model was implemented in Java with Python used to evaluate the results. 

The simulation displays good acceptance with real grocery market behavior, i.e. captures the performance of British retailers thus can be used to determine the impact of changes in their behavior on their competitors and consumers.Furthermore it can be used to provide insight into sustainability of volatile pricing strategies, providing a useful insight in volatility of British supermarket retail industry. 
\end{abstract}
\acknowledgements{
I would like to express my sincere gratitude to Dr Maria Polukarov for her guidance and support which provided me the freedom to take this research in the direction of my interest.\\
\\
I would also like to thank my family and friends for their encouragement and support. To those who quietly listened to my software complaints. To those who worked throughout the nights with me. To those who helped me write what I couldn't say. I cannot thank you enough.
}

\declaration{
I, Stefan Collier, declare that this dissertation and the work presented in it are my own and has been generated by me as the result of my own original research.\\
I confirm that:\\
1. This work was done wholly or mainly while in candidature for a degree at this University;\\
2. Where any part of this dissertation has previously been submitted for any other qualification at this University or any other institution, this has been clearly stated;\\
3. Where I have consulted the published work of others, this is always clearly attributed;\\
4. Where I have quoted from the work of others, the source is always given. With the exception of such quotations, this dissertation is entirely my own work;\\
5. I have acknowledged all main sources of help;\\
6. Where the thesis is based on work done by myself jointly with others, I have made clear exactly what was done by others and what I have contributed myself;\\
7. Either none of this work has been published before submission, or parts of this work have been published by :\\
\\
Stefan Collier\\
April 2016
}
\tableofcontents
\listoffigures
\listoftables

\mainmatter
%% ----------------------------------------------------------------
%\include{Introduction}
%\include{Conclusions}
\include{chapters/1Project/main}
\include{chapters/2Lit/main}
\include{chapters/3Design/HighLevel}
\include{chapters/3Design/InDepth}
\include{chapters/4Impl/main}

\include{chapters/5Experiments/1/main}
\include{chapters/5Experiments/2/main}
\include{chapters/5Experiments/3/main}
\include{chapters/5Experiments/4/main}

\include{chapters/6Conclusion/main}

\appendix
\include{appendix/AppendixB}
\include{appendix/D/main}
\include{appendix/AppendixC}

\backmatter
\bibliographystyle{ecs}
\bibliography{ECS}
\end{document}
%% ----------------------------------------------------------------

\include{chapters/3Design/HighLevel}
\include{chapters/3Design/InDepth}
 %% ----------------------------------------------------------------
%% Progress.tex
%% ---------------------------------------------------------------- 
\documentclass{ecsprogress}    % Use the progress Style
\graphicspath{{../figs/}}   % Location of your graphics files
    \usepackage{natbib}            % Use Natbib style for the refs.
\hypersetup{colorlinks=true}   % Set to false for black/white printing
\input{Definitions}            % Include your abbreviations



\usepackage{enumitem}% http://ctan.org/pkg/enumitem
\usepackage{multirow}
\usepackage{float}
\usepackage{amsmath}
\usepackage{multicol}
\usepackage{amssymb}
\usepackage[normalem]{ulem}
\useunder{\uline}{\ul}{}
\usepackage{wrapfig}


\usepackage[table,xcdraw]{xcolor}


%% ----------------------------------------------------------------
\begin{document}
\frontmatter
\title      {Heterogeneous Agent-based Model for Supermarket Competition}
\authors    {\texorpdfstring
             {\href{mailto:sc22g13@ecs.soton.ac.uk}{Stefan J. Collier}}
             {Stefan J. Collier}
            }
\addresses  {\groupname\\\deptname\\\univname}
\date       {\today}
\subject    {}
\keywords   {}
\supervisor {Dr. Maria Polukarov}
\examiner   {Professor Sheng Chen}

\maketitle
\begin{abstract}
This project aim was to model and analyse the effects of competitive pricing behaviors of grocery retailers on the British market. 

This was achieved by creating a multi-agent model, containing retailer and consumer agents. The heterogeneous crowd of retailers employs either a uniform pricing strategy or a ‘local price flexing’ strategy. The actions of these retailers are chosen by predicting the profit of each action, using a perceptron. Following on from the consideration of different economic models, a discrete model was developed so that software agents have a discrete environment to operate within. Within the model, it has been observed how supermarkets with differing behaviors affect a heterogeneous crowd of consumer agents. The model was implemented in Java with Python used to evaluate the results. 

The simulation displays good acceptance with real grocery market behavior, i.e. captures the performance of British retailers thus can be used to determine the impact of changes in their behavior on their competitors and consumers.Furthermore it can be used to provide insight into sustainability of volatile pricing strategies, providing a useful insight in volatility of British supermarket retail industry. 
\end{abstract}
\acknowledgements{
I would like to express my sincere gratitude to Dr Maria Polukarov for her guidance and support which provided me the freedom to take this research in the direction of my interest.\\
\\
I would also like to thank my family and friends for their encouragement and support. To those who quietly listened to my software complaints. To those who worked throughout the nights with me. To those who helped me write what I couldn't say. I cannot thank you enough.
}

\declaration{
I, Stefan Collier, declare that this dissertation and the work presented in it are my own and has been generated by me as the result of my own original research.\\
I confirm that:\\
1. This work was done wholly or mainly while in candidature for a degree at this University;\\
2. Where any part of this dissertation has previously been submitted for any other qualification at this University or any other institution, this has been clearly stated;\\
3. Where I have consulted the published work of others, this is always clearly attributed;\\
4. Where I have quoted from the work of others, the source is always given. With the exception of such quotations, this dissertation is entirely my own work;\\
5. I have acknowledged all main sources of help;\\
6. Where the thesis is based on work done by myself jointly with others, I have made clear exactly what was done by others and what I have contributed myself;\\
7. Either none of this work has been published before submission, or parts of this work have been published by :\\
\\
Stefan Collier\\
April 2016
}
\tableofcontents
\listoffigures
\listoftables

\mainmatter
%% ----------------------------------------------------------------
%\include{Introduction}
%\include{Conclusions}
\include{chapters/1Project/main}
\include{chapters/2Lit/main}
\include{chapters/3Design/HighLevel}
\include{chapters/3Design/InDepth}
\include{chapters/4Impl/main}

\include{chapters/5Experiments/1/main}
\include{chapters/5Experiments/2/main}
\include{chapters/5Experiments/3/main}
\include{chapters/5Experiments/4/main}

\include{chapters/6Conclusion/main}

\appendix
\include{appendix/AppendixB}
\include{appendix/D/main}
\include{appendix/AppendixC}

\backmatter
\bibliographystyle{ecs}
\bibliography{ECS}
\end{document}
%% ----------------------------------------------------------------


 %% ----------------------------------------------------------------
%% Progress.tex
%% ---------------------------------------------------------------- 
\documentclass{ecsprogress}    % Use the progress Style
\graphicspath{{../figs/}}   % Location of your graphics files
    \usepackage{natbib}            % Use Natbib style for the refs.
\hypersetup{colorlinks=true}   % Set to false for black/white printing
\input{Definitions}            % Include your abbreviations



\usepackage{enumitem}% http://ctan.org/pkg/enumitem
\usepackage{multirow}
\usepackage{float}
\usepackage{amsmath}
\usepackage{multicol}
\usepackage{amssymb}
\usepackage[normalem]{ulem}
\useunder{\uline}{\ul}{}
\usepackage{wrapfig}


\usepackage[table,xcdraw]{xcolor}


%% ----------------------------------------------------------------
\begin{document}
\frontmatter
\title      {Heterogeneous Agent-based Model for Supermarket Competition}
\authors    {\texorpdfstring
             {\href{mailto:sc22g13@ecs.soton.ac.uk}{Stefan J. Collier}}
             {Stefan J. Collier}
            }
\addresses  {\groupname\\\deptname\\\univname}
\date       {\today}
\subject    {}
\keywords   {}
\supervisor {Dr. Maria Polukarov}
\examiner   {Professor Sheng Chen}

\maketitle
\begin{abstract}
This project aim was to model and analyse the effects of competitive pricing behaviors of grocery retailers on the British market. 

This was achieved by creating a multi-agent model, containing retailer and consumer agents. The heterogeneous crowd of retailers employs either a uniform pricing strategy or a ‘local price flexing’ strategy. The actions of these retailers are chosen by predicting the profit of each action, using a perceptron. Following on from the consideration of different economic models, a discrete model was developed so that software agents have a discrete environment to operate within. Within the model, it has been observed how supermarkets with differing behaviors affect a heterogeneous crowd of consumer agents. The model was implemented in Java with Python used to evaluate the results. 

The simulation displays good acceptance with real grocery market behavior, i.e. captures the performance of British retailers thus can be used to determine the impact of changes in their behavior on their competitors and consumers.Furthermore it can be used to provide insight into sustainability of volatile pricing strategies, providing a useful insight in volatility of British supermarket retail industry. 
\end{abstract}
\acknowledgements{
I would like to express my sincere gratitude to Dr Maria Polukarov for her guidance and support which provided me the freedom to take this research in the direction of my interest.\\
\\
I would also like to thank my family and friends for their encouragement and support. To those who quietly listened to my software complaints. To those who worked throughout the nights with me. To those who helped me write what I couldn't say. I cannot thank you enough.
}

\declaration{
I, Stefan Collier, declare that this dissertation and the work presented in it are my own and has been generated by me as the result of my own original research.\\
I confirm that:\\
1. This work was done wholly or mainly while in candidature for a degree at this University;\\
2. Where any part of this dissertation has previously been submitted for any other qualification at this University or any other institution, this has been clearly stated;\\
3. Where I have consulted the published work of others, this is always clearly attributed;\\
4. Where I have quoted from the work of others, the source is always given. With the exception of such quotations, this dissertation is entirely my own work;\\
5. I have acknowledged all main sources of help;\\
6. Where the thesis is based on work done by myself jointly with others, I have made clear exactly what was done by others and what I have contributed myself;\\
7. Either none of this work has been published before submission, or parts of this work have been published by :\\
\\
Stefan Collier\\
April 2016
}
\tableofcontents
\listoffigures
\listoftables

\mainmatter
%% ----------------------------------------------------------------
%\include{Introduction}
%\include{Conclusions}
\include{chapters/1Project/main}
\include{chapters/2Lit/main}
\include{chapters/3Design/HighLevel}
\include{chapters/3Design/InDepth}
\include{chapters/4Impl/main}

\include{chapters/5Experiments/1/main}
\include{chapters/5Experiments/2/main}
\include{chapters/5Experiments/3/main}
\include{chapters/5Experiments/4/main}

\include{chapters/6Conclusion/main}

\appendix
\include{appendix/AppendixB}
\include{appendix/D/main}
\include{appendix/AppendixC}

\backmatter
\bibliographystyle{ecs}
\bibliography{ECS}
\end{document}
%% ----------------------------------------------------------------

 %% ----------------------------------------------------------------
%% Progress.tex
%% ---------------------------------------------------------------- 
\documentclass{ecsprogress}    % Use the progress Style
\graphicspath{{../figs/}}   % Location of your graphics files
    \usepackage{natbib}            % Use Natbib style for the refs.
\hypersetup{colorlinks=true}   % Set to false for black/white printing
\input{Definitions}            % Include your abbreviations



\usepackage{enumitem}% http://ctan.org/pkg/enumitem
\usepackage{multirow}
\usepackage{float}
\usepackage{amsmath}
\usepackage{multicol}
\usepackage{amssymb}
\usepackage[normalem]{ulem}
\useunder{\uline}{\ul}{}
\usepackage{wrapfig}


\usepackage[table,xcdraw]{xcolor}


%% ----------------------------------------------------------------
\begin{document}
\frontmatter
\title      {Heterogeneous Agent-based Model for Supermarket Competition}
\authors    {\texorpdfstring
             {\href{mailto:sc22g13@ecs.soton.ac.uk}{Stefan J. Collier}}
             {Stefan J. Collier}
            }
\addresses  {\groupname\\\deptname\\\univname}
\date       {\today}
\subject    {}
\keywords   {}
\supervisor {Dr. Maria Polukarov}
\examiner   {Professor Sheng Chen}

\maketitle
\begin{abstract}
This project aim was to model and analyse the effects of competitive pricing behaviors of grocery retailers on the British market. 

This was achieved by creating a multi-agent model, containing retailer and consumer agents. The heterogeneous crowd of retailers employs either a uniform pricing strategy or a ‘local price flexing’ strategy. The actions of these retailers are chosen by predicting the profit of each action, using a perceptron. Following on from the consideration of different economic models, a discrete model was developed so that software agents have a discrete environment to operate within. Within the model, it has been observed how supermarkets with differing behaviors affect a heterogeneous crowd of consumer agents. The model was implemented in Java with Python used to evaluate the results. 

The simulation displays good acceptance with real grocery market behavior, i.e. captures the performance of British retailers thus can be used to determine the impact of changes in their behavior on their competitors and consumers.Furthermore it can be used to provide insight into sustainability of volatile pricing strategies, providing a useful insight in volatility of British supermarket retail industry. 
\end{abstract}
\acknowledgements{
I would like to express my sincere gratitude to Dr Maria Polukarov for her guidance and support which provided me the freedom to take this research in the direction of my interest.\\
\\
I would also like to thank my family and friends for their encouragement and support. To those who quietly listened to my software complaints. To those who worked throughout the nights with me. To those who helped me write what I couldn't say. I cannot thank you enough.
}

\declaration{
I, Stefan Collier, declare that this dissertation and the work presented in it are my own and has been generated by me as the result of my own original research.\\
I confirm that:\\
1. This work was done wholly or mainly while in candidature for a degree at this University;\\
2. Where any part of this dissertation has previously been submitted for any other qualification at this University or any other institution, this has been clearly stated;\\
3. Where I have consulted the published work of others, this is always clearly attributed;\\
4. Where I have quoted from the work of others, the source is always given. With the exception of such quotations, this dissertation is entirely my own work;\\
5. I have acknowledged all main sources of help;\\
6. Where the thesis is based on work done by myself jointly with others, I have made clear exactly what was done by others and what I have contributed myself;\\
7. Either none of this work has been published before submission, or parts of this work have been published by :\\
\\
Stefan Collier\\
April 2016
}
\tableofcontents
\listoffigures
\listoftables

\mainmatter
%% ----------------------------------------------------------------
%\include{Introduction}
%\include{Conclusions}
\include{chapters/1Project/main}
\include{chapters/2Lit/main}
\include{chapters/3Design/HighLevel}
\include{chapters/3Design/InDepth}
\include{chapters/4Impl/main}

\include{chapters/5Experiments/1/main}
\include{chapters/5Experiments/2/main}
\include{chapters/5Experiments/3/main}
\include{chapters/5Experiments/4/main}

\include{chapters/6Conclusion/main}

\appendix
\include{appendix/AppendixB}
\include{appendix/D/main}
\include{appendix/AppendixC}

\backmatter
\bibliographystyle{ecs}
\bibliography{ECS}
\end{document}
%% ----------------------------------------------------------------

 %% ----------------------------------------------------------------
%% Progress.tex
%% ---------------------------------------------------------------- 
\documentclass{ecsprogress}    % Use the progress Style
\graphicspath{{../figs/}}   % Location of your graphics files
    \usepackage{natbib}            % Use Natbib style for the refs.
\hypersetup{colorlinks=true}   % Set to false for black/white printing
\input{Definitions}            % Include your abbreviations



\usepackage{enumitem}% http://ctan.org/pkg/enumitem
\usepackage{multirow}
\usepackage{float}
\usepackage{amsmath}
\usepackage{multicol}
\usepackage{amssymb}
\usepackage[normalem]{ulem}
\useunder{\uline}{\ul}{}
\usepackage{wrapfig}


\usepackage[table,xcdraw]{xcolor}


%% ----------------------------------------------------------------
\begin{document}
\frontmatter
\title      {Heterogeneous Agent-based Model for Supermarket Competition}
\authors    {\texorpdfstring
             {\href{mailto:sc22g13@ecs.soton.ac.uk}{Stefan J. Collier}}
             {Stefan J. Collier}
            }
\addresses  {\groupname\\\deptname\\\univname}
\date       {\today}
\subject    {}
\keywords   {}
\supervisor {Dr. Maria Polukarov}
\examiner   {Professor Sheng Chen}

\maketitle
\begin{abstract}
This project aim was to model and analyse the effects of competitive pricing behaviors of grocery retailers on the British market. 

This was achieved by creating a multi-agent model, containing retailer and consumer agents. The heterogeneous crowd of retailers employs either a uniform pricing strategy or a ‘local price flexing’ strategy. The actions of these retailers are chosen by predicting the profit of each action, using a perceptron. Following on from the consideration of different economic models, a discrete model was developed so that software agents have a discrete environment to operate within. Within the model, it has been observed how supermarkets with differing behaviors affect a heterogeneous crowd of consumer agents. The model was implemented in Java with Python used to evaluate the results. 

The simulation displays good acceptance with real grocery market behavior, i.e. captures the performance of British retailers thus can be used to determine the impact of changes in their behavior on their competitors and consumers.Furthermore it can be used to provide insight into sustainability of volatile pricing strategies, providing a useful insight in volatility of British supermarket retail industry. 
\end{abstract}
\acknowledgements{
I would like to express my sincere gratitude to Dr Maria Polukarov for her guidance and support which provided me the freedom to take this research in the direction of my interest.\\
\\
I would also like to thank my family and friends for their encouragement and support. To those who quietly listened to my software complaints. To those who worked throughout the nights with me. To those who helped me write what I couldn't say. I cannot thank you enough.
}

\declaration{
I, Stefan Collier, declare that this dissertation and the work presented in it are my own and has been generated by me as the result of my own original research.\\
I confirm that:\\
1. This work was done wholly or mainly while in candidature for a degree at this University;\\
2. Where any part of this dissertation has previously been submitted for any other qualification at this University or any other institution, this has been clearly stated;\\
3. Where I have consulted the published work of others, this is always clearly attributed;\\
4. Where I have quoted from the work of others, the source is always given. With the exception of such quotations, this dissertation is entirely my own work;\\
5. I have acknowledged all main sources of help;\\
6. Where the thesis is based on work done by myself jointly with others, I have made clear exactly what was done by others and what I have contributed myself;\\
7. Either none of this work has been published before submission, or parts of this work have been published by :\\
\\
Stefan Collier\\
April 2016
}
\tableofcontents
\listoffigures
\listoftables

\mainmatter
%% ----------------------------------------------------------------
%\include{Introduction}
%\include{Conclusions}
\include{chapters/1Project/main}
\include{chapters/2Lit/main}
\include{chapters/3Design/HighLevel}
\include{chapters/3Design/InDepth}
\include{chapters/4Impl/main}

\include{chapters/5Experiments/1/main}
\include{chapters/5Experiments/2/main}
\include{chapters/5Experiments/3/main}
\include{chapters/5Experiments/4/main}

\include{chapters/6Conclusion/main}

\appendix
\include{appendix/AppendixB}
\include{appendix/D/main}
\include{appendix/AppendixC}

\backmatter
\bibliographystyle{ecs}
\bibliography{ECS}
\end{document}
%% ----------------------------------------------------------------

 %% ----------------------------------------------------------------
%% Progress.tex
%% ---------------------------------------------------------------- 
\documentclass{ecsprogress}    % Use the progress Style
\graphicspath{{../figs/}}   % Location of your graphics files
    \usepackage{natbib}            % Use Natbib style for the refs.
\hypersetup{colorlinks=true}   % Set to false for black/white printing
\input{Definitions}            % Include your abbreviations



\usepackage{enumitem}% http://ctan.org/pkg/enumitem
\usepackage{multirow}
\usepackage{float}
\usepackage{amsmath}
\usepackage{multicol}
\usepackage{amssymb}
\usepackage[normalem]{ulem}
\useunder{\uline}{\ul}{}
\usepackage{wrapfig}


\usepackage[table,xcdraw]{xcolor}


%% ----------------------------------------------------------------
\begin{document}
\frontmatter
\title      {Heterogeneous Agent-based Model for Supermarket Competition}
\authors    {\texorpdfstring
             {\href{mailto:sc22g13@ecs.soton.ac.uk}{Stefan J. Collier}}
             {Stefan J. Collier}
            }
\addresses  {\groupname\\\deptname\\\univname}
\date       {\today}
\subject    {}
\keywords   {}
\supervisor {Dr. Maria Polukarov}
\examiner   {Professor Sheng Chen}

\maketitle
\begin{abstract}
This project aim was to model and analyse the effects of competitive pricing behaviors of grocery retailers on the British market. 

This was achieved by creating a multi-agent model, containing retailer and consumer agents. The heterogeneous crowd of retailers employs either a uniform pricing strategy or a ‘local price flexing’ strategy. The actions of these retailers are chosen by predicting the profit of each action, using a perceptron. Following on from the consideration of different economic models, a discrete model was developed so that software agents have a discrete environment to operate within. Within the model, it has been observed how supermarkets with differing behaviors affect a heterogeneous crowd of consumer agents. The model was implemented in Java with Python used to evaluate the results. 

The simulation displays good acceptance with real grocery market behavior, i.e. captures the performance of British retailers thus can be used to determine the impact of changes in their behavior on their competitors and consumers.Furthermore it can be used to provide insight into sustainability of volatile pricing strategies, providing a useful insight in volatility of British supermarket retail industry. 
\end{abstract}
\acknowledgements{
I would like to express my sincere gratitude to Dr Maria Polukarov for her guidance and support which provided me the freedom to take this research in the direction of my interest.\\
\\
I would also like to thank my family and friends for their encouragement and support. To those who quietly listened to my software complaints. To those who worked throughout the nights with me. To those who helped me write what I couldn't say. I cannot thank you enough.
}

\declaration{
I, Stefan Collier, declare that this dissertation and the work presented in it are my own and has been generated by me as the result of my own original research.\\
I confirm that:\\
1. This work was done wholly or mainly while in candidature for a degree at this University;\\
2. Where any part of this dissertation has previously been submitted for any other qualification at this University or any other institution, this has been clearly stated;\\
3. Where I have consulted the published work of others, this is always clearly attributed;\\
4. Where I have quoted from the work of others, the source is always given. With the exception of such quotations, this dissertation is entirely my own work;\\
5. I have acknowledged all main sources of help;\\
6. Where the thesis is based on work done by myself jointly with others, I have made clear exactly what was done by others and what I have contributed myself;\\
7. Either none of this work has been published before submission, or parts of this work have been published by :\\
\\
Stefan Collier\\
April 2016
}
\tableofcontents
\listoffigures
\listoftables

\mainmatter
%% ----------------------------------------------------------------
%\include{Introduction}
%\include{Conclusions}
\include{chapters/1Project/main}
\include{chapters/2Lit/main}
\include{chapters/3Design/HighLevel}
\include{chapters/3Design/InDepth}
\include{chapters/4Impl/main}

\include{chapters/5Experiments/1/main}
\include{chapters/5Experiments/2/main}
\include{chapters/5Experiments/3/main}
\include{chapters/5Experiments/4/main}

\include{chapters/6Conclusion/main}

\appendix
\include{appendix/AppendixB}
\include{appendix/D/main}
\include{appendix/AppendixC}

\backmatter
\bibliographystyle{ecs}
\bibliography{ECS}
\end{document}
%% ----------------------------------------------------------------


 %% ----------------------------------------------------------------
%% Progress.tex
%% ---------------------------------------------------------------- 
\documentclass{ecsprogress}    % Use the progress Style
\graphicspath{{../figs/}}   % Location of your graphics files
    \usepackage{natbib}            % Use Natbib style for the refs.
\hypersetup{colorlinks=true}   % Set to false for black/white printing
\input{Definitions}            % Include your abbreviations



\usepackage{enumitem}% http://ctan.org/pkg/enumitem
\usepackage{multirow}
\usepackage{float}
\usepackage{amsmath}
\usepackage{multicol}
\usepackage{amssymb}
\usepackage[normalem]{ulem}
\useunder{\uline}{\ul}{}
\usepackage{wrapfig}


\usepackage[table,xcdraw]{xcolor}


%% ----------------------------------------------------------------
\begin{document}
\frontmatter
\title      {Heterogeneous Agent-based Model for Supermarket Competition}
\authors    {\texorpdfstring
             {\href{mailto:sc22g13@ecs.soton.ac.uk}{Stefan J. Collier}}
             {Stefan J. Collier}
            }
\addresses  {\groupname\\\deptname\\\univname}
\date       {\today}
\subject    {}
\keywords   {}
\supervisor {Dr. Maria Polukarov}
\examiner   {Professor Sheng Chen}

\maketitle
\begin{abstract}
This project aim was to model and analyse the effects of competitive pricing behaviors of grocery retailers on the British market. 

This was achieved by creating a multi-agent model, containing retailer and consumer agents. The heterogeneous crowd of retailers employs either a uniform pricing strategy or a ‘local price flexing’ strategy. The actions of these retailers are chosen by predicting the profit of each action, using a perceptron. Following on from the consideration of different economic models, a discrete model was developed so that software agents have a discrete environment to operate within. Within the model, it has been observed how supermarkets with differing behaviors affect a heterogeneous crowd of consumer agents. The model was implemented in Java with Python used to evaluate the results. 

The simulation displays good acceptance with real grocery market behavior, i.e. captures the performance of British retailers thus can be used to determine the impact of changes in their behavior on their competitors and consumers.Furthermore it can be used to provide insight into sustainability of volatile pricing strategies, providing a useful insight in volatility of British supermarket retail industry. 
\end{abstract}
\acknowledgements{
I would like to express my sincere gratitude to Dr Maria Polukarov for her guidance and support which provided me the freedom to take this research in the direction of my interest.\\
\\
I would also like to thank my family and friends for their encouragement and support. To those who quietly listened to my software complaints. To those who worked throughout the nights with me. To those who helped me write what I couldn't say. I cannot thank you enough.
}

\declaration{
I, Stefan Collier, declare that this dissertation and the work presented in it are my own and has been generated by me as the result of my own original research.\\
I confirm that:\\
1. This work was done wholly or mainly while in candidature for a degree at this University;\\
2. Where any part of this dissertation has previously been submitted for any other qualification at this University or any other institution, this has been clearly stated;\\
3. Where I have consulted the published work of others, this is always clearly attributed;\\
4. Where I have quoted from the work of others, the source is always given. With the exception of such quotations, this dissertation is entirely my own work;\\
5. I have acknowledged all main sources of help;\\
6. Where the thesis is based on work done by myself jointly with others, I have made clear exactly what was done by others and what I have contributed myself;\\
7. Either none of this work has been published before submission, or parts of this work have been published by :\\
\\
Stefan Collier\\
April 2016
}
\tableofcontents
\listoffigures
\listoftables

\mainmatter
%% ----------------------------------------------------------------
%\include{Introduction}
%\include{Conclusions}
\include{chapters/1Project/main}
\include{chapters/2Lit/main}
\include{chapters/3Design/HighLevel}
\include{chapters/3Design/InDepth}
\include{chapters/4Impl/main}

\include{chapters/5Experiments/1/main}
\include{chapters/5Experiments/2/main}
\include{chapters/5Experiments/3/main}
\include{chapters/5Experiments/4/main}

\include{chapters/6Conclusion/main}

\appendix
\include{appendix/AppendixB}
\include{appendix/D/main}
\include{appendix/AppendixC}

\backmatter
\bibliographystyle{ecs}
\bibliography{ECS}
\end{document}
%% ----------------------------------------------------------------


\appendix
\include{appendix/AppendixB}
 %% ----------------------------------------------------------------
%% Progress.tex
%% ---------------------------------------------------------------- 
\documentclass{ecsprogress}    % Use the progress Style
\graphicspath{{../figs/}}   % Location of your graphics files
    \usepackage{natbib}            % Use Natbib style for the refs.
\hypersetup{colorlinks=true}   % Set to false for black/white printing
\input{Definitions}            % Include your abbreviations



\usepackage{enumitem}% http://ctan.org/pkg/enumitem
\usepackage{multirow}
\usepackage{float}
\usepackage{amsmath}
\usepackage{multicol}
\usepackage{amssymb}
\usepackage[normalem]{ulem}
\useunder{\uline}{\ul}{}
\usepackage{wrapfig}


\usepackage[table,xcdraw]{xcolor}


%% ----------------------------------------------------------------
\begin{document}
\frontmatter
\title      {Heterogeneous Agent-based Model for Supermarket Competition}
\authors    {\texorpdfstring
             {\href{mailto:sc22g13@ecs.soton.ac.uk}{Stefan J. Collier}}
             {Stefan J. Collier}
            }
\addresses  {\groupname\\\deptname\\\univname}
\date       {\today}
\subject    {}
\keywords   {}
\supervisor {Dr. Maria Polukarov}
\examiner   {Professor Sheng Chen}

\maketitle
\begin{abstract}
This project aim was to model and analyse the effects of competitive pricing behaviors of grocery retailers on the British market. 

This was achieved by creating a multi-agent model, containing retailer and consumer agents. The heterogeneous crowd of retailers employs either a uniform pricing strategy or a ‘local price flexing’ strategy. The actions of these retailers are chosen by predicting the profit of each action, using a perceptron. Following on from the consideration of different economic models, a discrete model was developed so that software agents have a discrete environment to operate within. Within the model, it has been observed how supermarkets with differing behaviors affect a heterogeneous crowd of consumer agents. The model was implemented in Java with Python used to evaluate the results. 

The simulation displays good acceptance with real grocery market behavior, i.e. captures the performance of British retailers thus can be used to determine the impact of changes in their behavior on their competitors and consumers.Furthermore it can be used to provide insight into sustainability of volatile pricing strategies, providing a useful insight in volatility of British supermarket retail industry. 
\end{abstract}
\acknowledgements{
I would like to express my sincere gratitude to Dr Maria Polukarov for her guidance and support which provided me the freedom to take this research in the direction of my interest.\\
\\
I would also like to thank my family and friends for their encouragement and support. To those who quietly listened to my software complaints. To those who worked throughout the nights with me. To those who helped me write what I couldn't say. I cannot thank you enough.
}

\declaration{
I, Stefan Collier, declare that this dissertation and the work presented in it are my own and has been generated by me as the result of my own original research.\\
I confirm that:\\
1. This work was done wholly or mainly while in candidature for a degree at this University;\\
2. Where any part of this dissertation has previously been submitted for any other qualification at this University or any other institution, this has been clearly stated;\\
3. Where I have consulted the published work of others, this is always clearly attributed;\\
4. Where I have quoted from the work of others, the source is always given. With the exception of such quotations, this dissertation is entirely my own work;\\
5. I have acknowledged all main sources of help;\\
6. Where the thesis is based on work done by myself jointly with others, I have made clear exactly what was done by others and what I have contributed myself;\\
7. Either none of this work has been published before submission, or parts of this work have been published by :\\
\\
Stefan Collier\\
April 2016
}
\tableofcontents
\listoffigures
\listoftables

\mainmatter
%% ----------------------------------------------------------------
%\include{Introduction}
%\include{Conclusions}
\include{chapters/1Project/main}
\include{chapters/2Lit/main}
\include{chapters/3Design/HighLevel}
\include{chapters/3Design/InDepth}
\include{chapters/4Impl/main}

\include{chapters/5Experiments/1/main}
\include{chapters/5Experiments/2/main}
\include{chapters/5Experiments/3/main}
\include{chapters/5Experiments/4/main}

\include{chapters/6Conclusion/main}

\appendix
\include{appendix/AppendixB}
\include{appendix/D/main}
\include{appendix/AppendixC}

\backmatter
\bibliographystyle{ecs}
\bibliography{ECS}
\end{document}
%% ----------------------------------------------------------------

\include{appendix/AppendixC}

\backmatter
\bibliographystyle{ecs}
\bibliography{ECS}
\end{document}
%% ----------------------------------------------------------------

 %% ----------------------------------------------------------------
%% Progress.tex
%% ---------------------------------------------------------------- 
\documentclass{ecsprogress}    % Use the progress Style
\graphicspath{{../figs/}}   % Location of your graphics files
    \usepackage{natbib}            % Use Natbib style for the refs.
\hypersetup{colorlinks=true}   % Set to false for black/white printing
\input{Definitions}            % Include your abbreviations



\usepackage{enumitem}% http://ctan.org/pkg/enumitem
\usepackage{multirow}
\usepackage{float}
\usepackage{amsmath}
\usepackage{multicol}
\usepackage{amssymb}
\usepackage[normalem]{ulem}
\useunder{\uline}{\ul}{}
\usepackage{wrapfig}


\usepackage[table,xcdraw]{xcolor}


%% ----------------------------------------------------------------
\begin{document}
\frontmatter
\title      {Heterogeneous Agent-based Model for Supermarket Competition}
\authors    {\texorpdfstring
             {\href{mailto:sc22g13@ecs.soton.ac.uk}{Stefan J. Collier}}
             {Stefan J. Collier}
            }
\addresses  {\groupname\\\deptname\\\univname}
\date       {\today}
\subject    {}
\keywords   {}
\supervisor {Dr. Maria Polukarov}
\examiner   {Professor Sheng Chen}

\maketitle
\begin{abstract}
This project aim was to model and analyse the effects of competitive pricing behaviors of grocery retailers on the British market. 

This was achieved by creating a multi-agent model, containing retailer and consumer agents. The heterogeneous crowd of retailers employs either a uniform pricing strategy or a ‘local price flexing’ strategy. The actions of these retailers are chosen by predicting the profit of each action, using a perceptron. Following on from the consideration of different economic models, a discrete model was developed so that software agents have a discrete environment to operate within. Within the model, it has been observed how supermarkets with differing behaviors affect a heterogeneous crowd of consumer agents. The model was implemented in Java with Python used to evaluate the results. 

The simulation displays good acceptance with real grocery market behavior, i.e. captures the performance of British retailers thus can be used to determine the impact of changes in their behavior on their competitors and consumers.Furthermore it can be used to provide insight into sustainability of volatile pricing strategies, providing a useful insight in volatility of British supermarket retail industry. 
\end{abstract}
\acknowledgements{
I would like to express my sincere gratitude to Dr Maria Polukarov for her guidance and support which provided me the freedom to take this research in the direction of my interest.\\
\\
I would also like to thank my family and friends for their encouragement and support. To those who quietly listened to my software complaints. To those who worked throughout the nights with me. To those who helped me write what I couldn't say. I cannot thank you enough.
}

\declaration{
I, Stefan Collier, declare that this dissertation and the work presented in it are my own and has been generated by me as the result of my own original research.\\
I confirm that:\\
1. This work was done wholly or mainly while in candidature for a degree at this University;\\
2. Where any part of this dissertation has previously been submitted for any other qualification at this University or any other institution, this has been clearly stated;\\
3. Where I have consulted the published work of others, this is always clearly attributed;\\
4. Where I have quoted from the work of others, the source is always given. With the exception of such quotations, this dissertation is entirely my own work;\\
5. I have acknowledged all main sources of help;\\
6. Where the thesis is based on work done by myself jointly with others, I have made clear exactly what was done by others and what I have contributed myself;\\
7. Either none of this work has been published before submission, or parts of this work have been published by :\\
\\
Stefan Collier\\
April 2016
}
\tableofcontents
\listoffigures
\listoftables

\mainmatter
%% ----------------------------------------------------------------
%\include{Introduction}
%\include{Conclusions}
 %% ----------------------------------------------------------------
%% Progress.tex
%% ---------------------------------------------------------------- 
\documentclass{ecsprogress}    % Use the progress Style
\graphicspath{{../figs/}}   % Location of your graphics files
    \usepackage{natbib}            % Use Natbib style for the refs.
\hypersetup{colorlinks=true}   % Set to false for black/white printing
\input{Definitions}            % Include your abbreviations



\usepackage{enumitem}% http://ctan.org/pkg/enumitem
\usepackage{multirow}
\usepackage{float}
\usepackage{amsmath}
\usepackage{multicol}
\usepackage{amssymb}
\usepackage[normalem]{ulem}
\useunder{\uline}{\ul}{}
\usepackage{wrapfig}


\usepackage[table,xcdraw]{xcolor}


%% ----------------------------------------------------------------
\begin{document}
\frontmatter
\title      {Heterogeneous Agent-based Model for Supermarket Competition}
\authors    {\texorpdfstring
             {\href{mailto:sc22g13@ecs.soton.ac.uk}{Stefan J. Collier}}
             {Stefan J. Collier}
            }
\addresses  {\groupname\\\deptname\\\univname}
\date       {\today}
\subject    {}
\keywords   {}
\supervisor {Dr. Maria Polukarov}
\examiner   {Professor Sheng Chen}

\maketitle
\begin{abstract}
This project aim was to model and analyse the effects of competitive pricing behaviors of grocery retailers on the British market. 

This was achieved by creating a multi-agent model, containing retailer and consumer agents. The heterogeneous crowd of retailers employs either a uniform pricing strategy or a ‘local price flexing’ strategy. The actions of these retailers are chosen by predicting the profit of each action, using a perceptron. Following on from the consideration of different economic models, a discrete model was developed so that software agents have a discrete environment to operate within. Within the model, it has been observed how supermarkets with differing behaviors affect a heterogeneous crowd of consumer agents. The model was implemented in Java with Python used to evaluate the results. 

The simulation displays good acceptance with real grocery market behavior, i.e. captures the performance of British retailers thus can be used to determine the impact of changes in their behavior on their competitors and consumers.Furthermore it can be used to provide insight into sustainability of volatile pricing strategies, providing a useful insight in volatility of British supermarket retail industry. 
\end{abstract}
\acknowledgements{
I would like to express my sincere gratitude to Dr Maria Polukarov for her guidance and support which provided me the freedom to take this research in the direction of my interest.\\
\\
I would also like to thank my family and friends for their encouragement and support. To those who quietly listened to my software complaints. To those who worked throughout the nights with me. To those who helped me write what I couldn't say. I cannot thank you enough.
}

\declaration{
I, Stefan Collier, declare that this dissertation and the work presented in it are my own and has been generated by me as the result of my own original research.\\
I confirm that:\\
1. This work was done wholly or mainly while in candidature for a degree at this University;\\
2. Where any part of this dissertation has previously been submitted for any other qualification at this University or any other institution, this has been clearly stated;\\
3. Where I have consulted the published work of others, this is always clearly attributed;\\
4. Where I have quoted from the work of others, the source is always given. With the exception of such quotations, this dissertation is entirely my own work;\\
5. I have acknowledged all main sources of help;\\
6. Where the thesis is based on work done by myself jointly with others, I have made clear exactly what was done by others and what I have contributed myself;\\
7. Either none of this work has been published before submission, or parts of this work have been published by :\\
\\
Stefan Collier\\
April 2016
}
\tableofcontents
\listoffigures
\listoftables

\mainmatter
%% ----------------------------------------------------------------
%\include{Introduction}
%\include{Conclusions}
\include{chapters/1Project/main}
\include{chapters/2Lit/main}
\include{chapters/3Design/HighLevel}
\include{chapters/3Design/InDepth}
\include{chapters/4Impl/main}

\include{chapters/5Experiments/1/main}
\include{chapters/5Experiments/2/main}
\include{chapters/5Experiments/3/main}
\include{chapters/5Experiments/4/main}

\include{chapters/6Conclusion/main}

\appendix
\include{appendix/AppendixB}
\include{appendix/D/main}
\include{appendix/AppendixC}

\backmatter
\bibliographystyle{ecs}
\bibliography{ECS}
\end{document}
%% ----------------------------------------------------------------

 %% ----------------------------------------------------------------
%% Progress.tex
%% ---------------------------------------------------------------- 
\documentclass{ecsprogress}    % Use the progress Style
\graphicspath{{../figs/}}   % Location of your graphics files
    \usepackage{natbib}            % Use Natbib style for the refs.
\hypersetup{colorlinks=true}   % Set to false for black/white printing
\input{Definitions}            % Include your abbreviations



\usepackage{enumitem}% http://ctan.org/pkg/enumitem
\usepackage{multirow}
\usepackage{float}
\usepackage{amsmath}
\usepackage{multicol}
\usepackage{amssymb}
\usepackage[normalem]{ulem}
\useunder{\uline}{\ul}{}
\usepackage{wrapfig}


\usepackage[table,xcdraw]{xcolor}


%% ----------------------------------------------------------------
\begin{document}
\frontmatter
\title      {Heterogeneous Agent-based Model for Supermarket Competition}
\authors    {\texorpdfstring
             {\href{mailto:sc22g13@ecs.soton.ac.uk}{Stefan J. Collier}}
             {Stefan J. Collier}
            }
\addresses  {\groupname\\\deptname\\\univname}
\date       {\today}
\subject    {}
\keywords   {}
\supervisor {Dr. Maria Polukarov}
\examiner   {Professor Sheng Chen}

\maketitle
\begin{abstract}
This project aim was to model and analyse the effects of competitive pricing behaviors of grocery retailers on the British market. 

This was achieved by creating a multi-agent model, containing retailer and consumer agents. The heterogeneous crowd of retailers employs either a uniform pricing strategy or a ‘local price flexing’ strategy. The actions of these retailers are chosen by predicting the profit of each action, using a perceptron. Following on from the consideration of different economic models, a discrete model was developed so that software agents have a discrete environment to operate within. Within the model, it has been observed how supermarkets with differing behaviors affect a heterogeneous crowd of consumer agents. The model was implemented in Java with Python used to evaluate the results. 

The simulation displays good acceptance with real grocery market behavior, i.e. captures the performance of British retailers thus can be used to determine the impact of changes in their behavior on their competitors and consumers.Furthermore it can be used to provide insight into sustainability of volatile pricing strategies, providing a useful insight in volatility of British supermarket retail industry. 
\end{abstract}
\acknowledgements{
I would like to express my sincere gratitude to Dr Maria Polukarov for her guidance and support which provided me the freedom to take this research in the direction of my interest.\\
\\
I would also like to thank my family and friends for their encouragement and support. To those who quietly listened to my software complaints. To those who worked throughout the nights with me. To those who helped me write what I couldn't say. I cannot thank you enough.
}

\declaration{
I, Stefan Collier, declare that this dissertation and the work presented in it are my own and has been generated by me as the result of my own original research.\\
I confirm that:\\
1. This work was done wholly or mainly while in candidature for a degree at this University;\\
2. Where any part of this dissertation has previously been submitted for any other qualification at this University or any other institution, this has been clearly stated;\\
3. Where I have consulted the published work of others, this is always clearly attributed;\\
4. Where I have quoted from the work of others, the source is always given. With the exception of such quotations, this dissertation is entirely my own work;\\
5. I have acknowledged all main sources of help;\\
6. Where the thesis is based on work done by myself jointly with others, I have made clear exactly what was done by others and what I have contributed myself;\\
7. Either none of this work has been published before submission, or parts of this work have been published by :\\
\\
Stefan Collier\\
April 2016
}
\tableofcontents
\listoffigures
\listoftables

\mainmatter
%% ----------------------------------------------------------------
%\include{Introduction}
%\include{Conclusions}
\include{chapters/1Project/main}
\include{chapters/2Lit/main}
\include{chapters/3Design/HighLevel}
\include{chapters/3Design/InDepth}
\include{chapters/4Impl/main}

\include{chapters/5Experiments/1/main}
\include{chapters/5Experiments/2/main}
\include{chapters/5Experiments/3/main}
\include{chapters/5Experiments/4/main}

\include{chapters/6Conclusion/main}

\appendix
\include{appendix/AppendixB}
\include{appendix/D/main}
\include{appendix/AppendixC}

\backmatter
\bibliographystyle{ecs}
\bibliography{ECS}
\end{document}
%% ----------------------------------------------------------------

\include{chapters/3Design/HighLevel}
\include{chapters/3Design/InDepth}
 %% ----------------------------------------------------------------
%% Progress.tex
%% ---------------------------------------------------------------- 
\documentclass{ecsprogress}    % Use the progress Style
\graphicspath{{../figs/}}   % Location of your graphics files
    \usepackage{natbib}            % Use Natbib style for the refs.
\hypersetup{colorlinks=true}   % Set to false for black/white printing
\input{Definitions}            % Include your abbreviations



\usepackage{enumitem}% http://ctan.org/pkg/enumitem
\usepackage{multirow}
\usepackage{float}
\usepackage{amsmath}
\usepackage{multicol}
\usepackage{amssymb}
\usepackage[normalem]{ulem}
\useunder{\uline}{\ul}{}
\usepackage{wrapfig}


\usepackage[table,xcdraw]{xcolor}


%% ----------------------------------------------------------------
\begin{document}
\frontmatter
\title      {Heterogeneous Agent-based Model for Supermarket Competition}
\authors    {\texorpdfstring
             {\href{mailto:sc22g13@ecs.soton.ac.uk}{Stefan J. Collier}}
             {Stefan J. Collier}
            }
\addresses  {\groupname\\\deptname\\\univname}
\date       {\today}
\subject    {}
\keywords   {}
\supervisor {Dr. Maria Polukarov}
\examiner   {Professor Sheng Chen}

\maketitle
\begin{abstract}
This project aim was to model and analyse the effects of competitive pricing behaviors of grocery retailers on the British market. 

This was achieved by creating a multi-agent model, containing retailer and consumer agents. The heterogeneous crowd of retailers employs either a uniform pricing strategy or a ‘local price flexing’ strategy. The actions of these retailers are chosen by predicting the profit of each action, using a perceptron. Following on from the consideration of different economic models, a discrete model was developed so that software agents have a discrete environment to operate within. Within the model, it has been observed how supermarkets with differing behaviors affect a heterogeneous crowd of consumer agents. The model was implemented in Java with Python used to evaluate the results. 

The simulation displays good acceptance with real grocery market behavior, i.e. captures the performance of British retailers thus can be used to determine the impact of changes in their behavior on their competitors and consumers.Furthermore it can be used to provide insight into sustainability of volatile pricing strategies, providing a useful insight in volatility of British supermarket retail industry. 
\end{abstract}
\acknowledgements{
I would like to express my sincere gratitude to Dr Maria Polukarov for her guidance and support which provided me the freedom to take this research in the direction of my interest.\\
\\
I would also like to thank my family and friends for their encouragement and support. To those who quietly listened to my software complaints. To those who worked throughout the nights with me. To those who helped me write what I couldn't say. I cannot thank you enough.
}

\declaration{
I, Stefan Collier, declare that this dissertation and the work presented in it are my own and has been generated by me as the result of my own original research.\\
I confirm that:\\
1. This work was done wholly or mainly while in candidature for a degree at this University;\\
2. Where any part of this dissertation has previously been submitted for any other qualification at this University or any other institution, this has been clearly stated;\\
3. Where I have consulted the published work of others, this is always clearly attributed;\\
4. Where I have quoted from the work of others, the source is always given. With the exception of such quotations, this dissertation is entirely my own work;\\
5. I have acknowledged all main sources of help;\\
6. Where the thesis is based on work done by myself jointly with others, I have made clear exactly what was done by others and what I have contributed myself;\\
7. Either none of this work has been published before submission, or parts of this work have been published by :\\
\\
Stefan Collier\\
April 2016
}
\tableofcontents
\listoffigures
\listoftables

\mainmatter
%% ----------------------------------------------------------------
%\include{Introduction}
%\include{Conclusions}
\include{chapters/1Project/main}
\include{chapters/2Lit/main}
\include{chapters/3Design/HighLevel}
\include{chapters/3Design/InDepth}
\include{chapters/4Impl/main}

\include{chapters/5Experiments/1/main}
\include{chapters/5Experiments/2/main}
\include{chapters/5Experiments/3/main}
\include{chapters/5Experiments/4/main}

\include{chapters/6Conclusion/main}

\appendix
\include{appendix/AppendixB}
\include{appendix/D/main}
\include{appendix/AppendixC}

\backmatter
\bibliographystyle{ecs}
\bibliography{ECS}
\end{document}
%% ----------------------------------------------------------------


 %% ----------------------------------------------------------------
%% Progress.tex
%% ---------------------------------------------------------------- 
\documentclass{ecsprogress}    % Use the progress Style
\graphicspath{{../figs/}}   % Location of your graphics files
    \usepackage{natbib}            % Use Natbib style for the refs.
\hypersetup{colorlinks=true}   % Set to false for black/white printing
\input{Definitions}            % Include your abbreviations



\usepackage{enumitem}% http://ctan.org/pkg/enumitem
\usepackage{multirow}
\usepackage{float}
\usepackage{amsmath}
\usepackage{multicol}
\usepackage{amssymb}
\usepackage[normalem]{ulem}
\useunder{\uline}{\ul}{}
\usepackage{wrapfig}


\usepackage[table,xcdraw]{xcolor}


%% ----------------------------------------------------------------
\begin{document}
\frontmatter
\title      {Heterogeneous Agent-based Model for Supermarket Competition}
\authors    {\texorpdfstring
             {\href{mailto:sc22g13@ecs.soton.ac.uk}{Stefan J. Collier}}
             {Stefan J. Collier}
            }
\addresses  {\groupname\\\deptname\\\univname}
\date       {\today}
\subject    {}
\keywords   {}
\supervisor {Dr. Maria Polukarov}
\examiner   {Professor Sheng Chen}

\maketitle
\begin{abstract}
This project aim was to model and analyse the effects of competitive pricing behaviors of grocery retailers on the British market. 

This was achieved by creating a multi-agent model, containing retailer and consumer agents. The heterogeneous crowd of retailers employs either a uniform pricing strategy or a ‘local price flexing’ strategy. The actions of these retailers are chosen by predicting the profit of each action, using a perceptron. Following on from the consideration of different economic models, a discrete model was developed so that software agents have a discrete environment to operate within. Within the model, it has been observed how supermarkets with differing behaviors affect a heterogeneous crowd of consumer agents. The model was implemented in Java with Python used to evaluate the results. 

The simulation displays good acceptance with real grocery market behavior, i.e. captures the performance of British retailers thus can be used to determine the impact of changes in their behavior on their competitors and consumers.Furthermore it can be used to provide insight into sustainability of volatile pricing strategies, providing a useful insight in volatility of British supermarket retail industry. 
\end{abstract}
\acknowledgements{
I would like to express my sincere gratitude to Dr Maria Polukarov for her guidance and support which provided me the freedom to take this research in the direction of my interest.\\
\\
I would also like to thank my family and friends for their encouragement and support. To those who quietly listened to my software complaints. To those who worked throughout the nights with me. To those who helped me write what I couldn't say. I cannot thank you enough.
}

\declaration{
I, Stefan Collier, declare that this dissertation and the work presented in it are my own and has been generated by me as the result of my own original research.\\
I confirm that:\\
1. This work was done wholly or mainly while in candidature for a degree at this University;\\
2. Where any part of this dissertation has previously been submitted for any other qualification at this University or any other institution, this has been clearly stated;\\
3. Where I have consulted the published work of others, this is always clearly attributed;\\
4. Where I have quoted from the work of others, the source is always given. With the exception of such quotations, this dissertation is entirely my own work;\\
5. I have acknowledged all main sources of help;\\
6. Where the thesis is based on work done by myself jointly with others, I have made clear exactly what was done by others and what I have contributed myself;\\
7. Either none of this work has been published before submission, or parts of this work have been published by :\\
\\
Stefan Collier\\
April 2016
}
\tableofcontents
\listoffigures
\listoftables

\mainmatter
%% ----------------------------------------------------------------
%\include{Introduction}
%\include{Conclusions}
\include{chapters/1Project/main}
\include{chapters/2Lit/main}
\include{chapters/3Design/HighLevel}
\include{chapters/3Design/InDepth}
\include{chapters/4Impl/main}

\include{chapters/5Experiments/1/main}
\include{chapters/5Experiments/2/main}
\include{chapters/5Experiments/3/main}
\include{chapters/5Experiments/4/main}

\include{chapters/6Conclusion/main}

\appendix
\include{appendix/AppendixB}
\include{appendix/D/main}
\include{appendix/AppendixC}

\backmatter
\bibliographystyle{ecs}
\bibliography{ECS}
\end{document}
%% ----------------------------------------------------------------

 %% ----------------------------------------------------------------
%% Progress.tex
%% ---------------------------------------------------------------- 
\documentclass{ecsprogress}    % Use the progress Style
\graphicspath{{../figs/}}   % Location of your graphics files
    \usepackage{natbib}            % Use Natbib style for the refs.
\hypersetup{colorlinks=true}   % Set to false for black/white printing
\input{Definitions}            % Include your abbreviations



\usepackage{enumitem}% http://ctan.org/pkg/enumitem
\usepackage{multirow}
\usepackage{float}
\usepackage{amsmath}
\usepackage{multicol}
\usepackage{amssymb}
\usepackage[normalem]{ulem}
\useunder{\uline}{\ul}{}
\usepackage{wrapfig}


\usepackage[table,xcdraw]{xcolor}


%% ----------------------------------------------------------------
\begin{document}
\frontmatter
\title      {Heterogeneous Agent-based Model for Supermarket Competition}
\authors    {\texorpdfstring
             {\href{mailto:sc22g13@ecs.soton.ac.uk}{Stefan J. Collier}}
             {Stefan J. Collier}
            }
\addresses  {\groupname\\\deptname\\\univname}
\date       {\today}
\subject    {}
\keywords   {}
\supervisor {Dr. Maria Polukarov}
\examiner   {Professor Sheng Chen}

\maketitle
\begin{abstract}
This project aim was to model and analyse the effects of competitive pricing behaviors of grocery retailers on the British market. 

This was achieved by creating a multi-agent model, containing retailer and consumer agents. The heterogeneous crowd of retailers employs either a uniform pricing strategy or a ‘local price flexing’ strategy. The actions of these retailers are chosen by predicting the profit of each action, using a perceptron. Following on from the consideration of different economic models, a discrete model was developed so that software agents have a discrete environment to operate within. Within the model, it has been observed how supermarkets with differing behaviors affect a heterogeneous crowd of consumer agents. The model was implemented in Java with Python used to evaluate the results. 

The simulation displays good acceptance with real grocery market behavior, i.e. captures the performance of British retailers thus can be used to determine the impact of changes in their behavior on their competitors and consumers.Furthermore it can be used to provide insight into sustainability of volatile pricing strategies, providing a useful insight in volatility of British supermarket retail industry. 
\end{abstract}
\acknowledgements{
I would like to express my sincere gratitude to Dr Maria Polukarov for her guidance and support which provided me the freedom to take this research in the direction of my interest.\\
\\
I would also like to thank my family and friends for their encouragement and support. To those who quietly listened to my software complaints. To those who worked throughout the nights with me. To those who helped me write what I couldn't say. I cannot thank you enough.
}

\declaration{
I, Stefan Collier, declare that this dissertation and the work presented in it are my own and has been generated by me as the result of my own original research.\\
I confirm that:\\
1. This work was done wholly or mainly while in candidature for a degree at this University;\\
2. Where any part of this dissertation has previously been submitted for any other qualification at this University or any other institution, this has been clearly stated;\\
3. Where I have consulted the published work of others, this is always clearly attributed;\\
4. Where I have quoted from the work of others, the source is always given. With the exception of such quotations, this dissertation is entirely my own work;\\
5. I have acknowledged all main sources of help;\\
6. Where the thesis is based on work done by myself jointly with others, I have made clear exactly what was done by others and what I have contributed myself;\\
7. Either none of this work has been published before submission, or parts of this work have been published by :\\
\\
Stefan Collier\\
April 2016
}
\tableofcontents
\listoffigures
\listoftables

\mainmatter
%% ----------------------------------------------------------------
%\include{Introduction}
%\include{Conclusions}
\include{chapters/1Project/main}
\include{chapters/2Lit/main}
\include{chapters/3Design/HighLevel}
\include{chapters/3Design/InDepth}
\include{chapters/4Impl/main}

\include{chapters/5Experiments/1/main}
\include{chapters/5Experiments/2/main}
\include{chapters/5Experiments/3/main}
\include{chapters/5Experiments/4/main}

\include{chapters/6Conclusion/main}

\appendix
\include{appendix/AppendixB}
\include{appendix/D/main}
\include{appendix/AppendixC}

\backmatter
\bibliographystyle{ecs}
\bibliography{ECS}
\end{document}
%% ----------------------------------------------------------------

 %% ----------------------------------------------------------------
%% Progress.tex
%% ---------------------------------------------------------------- 
\documentclass{ecsprogress}    % Use the progress Style
\graphicspath{{../figs/}}   % Location of your graphics files
    \usepackage{natbib}            % Use Natbib style for the refs.
\hypersetup{colorlinks=true}   % Set to false for black/white printing
\input{Definitions}            % Include your abbreviations



\usepackage{enumitem}% http://ctan.org/pkg/enumitem
\usepackage{multirow}
\usepackage{float}
\usepackage{amsmath}
\usepackage{multicol}
\usepackage{amssymb}
\usepackage[normalem]{ulem}
\useunder{\uline}{\ul}{}
\usepackage{wrapfig}


\usepackage[table,xcdraw]{xcolor}


%% ----------------------------------------------------------------
\begin{document}
\frontmatter
\title      {Heterogeneous Agent-based Model for Supermarket Competition}
\authors    {\texorpdfstring
             {\href{mailto:sc22g13@ecs.soton.ac.uk}{Stefan J. Collier}}
             {Stefan J. Collier}
            }
\addresses  {\groupname\\\deptname\\\univname}
\date       {\today}
\subject    {}
\keywords   {}
\supervisor {Dr. Maria Polukarov}
\examiner   {Professor Sheng Chen}

\maketitle
\begin{abstract}
This project aim was to model and analyse the effects of competitive pricing behaviors of grocery retailers on the British market. 

This was achieved by creating a multi-agent model, containing retailer and consumer agents. The heterogeneous crowd of retailers employs either a uniform pricing strategy or a ‘local price flexing’ strategy. The actions of these retailers are chosen by predicting the profit of each action, using a perceptron. Following on from the consideration of different economic models, a discrete model was developed so that software agents have a discrete environment to operate within. Within the model, it has been observed how supermarkets with differing behaviors affect a heterogeneous crowd of consumer agents. The model was implemented in Java with Python used to evaluate the results. 

The simulation displays good acceptance with real grocery market behavior, i.e. captures the performance of British retailers thus can be used to determine the impact of changes in their behavior on their competitors and consumers.Furthermore it can be used to provide insight into sustainability of volatile pricing strategies, providing a useful insight in volatility of British supermarket retail industry. 
\end{abstract}
\acknowledgements{
I would like to express my sincere gratitude to Dr Maria Polukarov for her guidance and support which provided me the freedom to take this research in the direction of my interest.\\
\\
I would also like to thank my family and friends for their encouragement and support. To those who quietly listened to my software complaints. To those who worked throughout the nights with me. To those who helped me write what I couldn't say. I cannot thank you enough.
}

\declaration{
I, Stefan Collier, declare that this dissertation and the work presented in it are my own and has been generated by me as the result of my own original research.\\
I confirm that:\\
1. This work was done wholly or mainly while in candidature for a degree at this University;\\
2. Where any part of this dissertation has previously been submitted for any other qualification at this University or any other institution, this has been clearly stated;\\
3. Where I have consulted the published work of others, this is always clearly attributed;\\
4. Where I have quoted from the work of others, the source is always given. With the exception of such quotations, this dissertation is entirely my own work;\\
5. I have acknowledged all main sources of help;\\
6. Where the thesis is based on work done by myself jointly with others, I have made clear exactly what was done by others and what I have contributed myself;\\
7. Either none of this work has been published before submission, or parts of this work have been published by :\\
\\
Stefan Collier\\
April 2016
}
\tableofcontents
\listoffigures
\listoftables

\mainmatter
%% ----------------------------------------------------------------
%\include{Introduction}
%\include{Conclusions}
\include{chapters/1Project/main}
\include{chapters/2Lit/main}
\include{chapters/3Design/HighLevel}
\include{chapters/3Design/InDepth}
\include{chapters/4Impl/main}

\include{chapters/5Experiments/1/main}
\include{chapters/5Experiments/2/main}
\include{chapters/5Experiments/3/main}
\include{chapters/5Experiments/4/main}

\include{chapters/6Conclusion/main}

\appendix
\include{appendix/AppendixB}
\include{appendix/D/main}
\include{appendix/AppendixC}

\backmatter
\bibliographystyle{ecs}
\bibliography{ECS}
\end{document}
%% ----------------------------------------------------------------

 %% ----------------------------------------------------------------
%% Progress.tex
%% ---------------------------------------------------------------- 
\documentclass{ecsprogress}    % Use the progress Style
\graphicspath{{../figs/}}   % Location of your graphics files
    \usepackage{natbib}            % Use Natbib style for the refs.
\hypersetup{colorlinks=true}   % Set to false for black/white printing
\input{Definitions}            % Include your abbreviations



\usepackage{enumitem}% http://ctan.org/pkg/enumitem
\usepackage{multirow}
\usepackage{float}
\usepackage{amsmath}
\usepackage{multicol}
\usepackage{amssymb}
\usepackage[normalem]{ulem}
\useunder{\uline}{\ul}{}
\usepackage{wrapfig}


\usepackage[table,xcdraw]{xcolor}


%% ----------------------------------------------------------------
\begin{document}
\frontmatter
\title      {Heterogeneous Agent-based Model for Supermarket Competition}
\authors    {\texorpdfstring
             {\href{mailto:sc22g13@ecs.soton.ac.uk}{Stefan J. Collier}}
             {Stefan J. Collier}
            }
\addresses  {\groupname\\\deptname\\\univname}
\date       {\today}
\subject    {}
\keywords   {}
\supervisor {Dr. Maria Polukarov}
\examiner   {Professor Sheng Chen}

\maketitle
\begin{abstract}
This project aim was to model and analyse the effects of competitive pricing behaviors of grocery retailers on the British market. 

This was achieved by creating a multi-agent model, containing retailer and consumer agents. The heterogeneous crowd of retailers employs either a uniform pricing strategy or a ‘local price flexing’ strategy. The actions of these retailers are chosen by predicting the profit of each action, using a perceptron. Following on from the consideration of different economic models, a discrete model was developed so that software agents have a discrete environment to operate within. Within the model, it has been observed how supermarkets with differing behaviors affect a heterogeneous crowd of consumer agents. The model was implemented in Java with Python used to evaluate the results. 

The simulation displays good acceptance with real grocery market behavior, i.e. captures the performance of British retailers thus can be used to determine the impact of changes in their behavior on their competitors and consumers.Furthermore it can be used to provide insight into sustainability of volatile pricing strategies, providing a useful insight in volatility of British supermarket retail industry. 
\end{abstract}
\acknowledgements{
I would like to express my sincere gratitude to Dr Maria Polukarov for her guidance and support which provided me the freedom to take this research in the direction of my interest.\\
\\
I would also like to thank my family and friends for their encouragement and support. To those who quietly listened to my software complaints. To those who worked throughout the nights with me. To those who helped me write what I couldn't say. I cannot thank you enough.
}

\declaration{
I, Stefan Collier, declare that this dissertation and the work presented in it are my own and has been generated by me as the result of my own original research.\\
I confirm that:\\
1. This work was done wholly or mainly while in candidature for a degree at this University;\\
2. Where any part of this dissertation has previously been submitted for any other qualification at this University or any other institution, this has been clearly stated;\\
3. Where I have consulted the published work of others, this is always clearly attributed;\\
4. Where I have quoted from the work of others, the source is always given. With the exception of such quotations, this dissertation is entirely my own work;\\
5. I have acknowledged all main sources of help;\\
6. Where the thesis is based on work done by myself jointly with others, I have made clear exactly what was done by others and what I have contributed myself;\\
7. Either none of this work has been published before submission, or parts of this work have been published by :\\
\\
Stefan Collier\\
April 2016
}
\tableofcontents
\listoffigures
\listoftables

\mainmatter
%% ----------------------------------------------------------------
%\include{Introduction}
%\include{Conclusions}
\include{chapters/1Project/main}
\include{chapters/2Lit/main}
\include{chapters/3Design/HighLevel}
\include{chapters/3Design/InDepth}
\include{chapters/4Impl/main}

\include{chapters/5Experiments/1/main}
\include{chapters/5Experiments/2/main}
\include{chapters/5Experiments/3/main}
\include{chapters/5Experiments/4/main}

\include{chapters/6Conclusion/main}

\appendix
\include{appendix/AppendixB}
\include{appendix/D/main}
\include{appendix/AppendixC}

\backmatter
\bibliographystyle{ecs}
\bibliography{ECS}
\end{document}
%% ----------------------------------------------------------------


 %% ----------------------------------------------------------------
%% Progress.tex
%% ---------------------------------------------------------------- 
\documentclass{ecsprogress}    % Use the progress Style
\graphicspath{{../figs/}}   % Location of your graphics files
    \usepackage{natbib}            % Use Natbib style for the refs.
\hypersetup{colorlinks=true}   % Set to false for black/white printing
\input{Definitions}            % Include your abbreviations



\usepackage{enumitem}% http://ctan.org/pkg/enumitem
\usepackage{multirow}
\usepackage{float}
\usepackage{amsmath}
\usepackage{multicol}
\usepackage{amssymb}
\usepackage[normalem]{ulem}
\useunder{\uline}{\ul}{}
\usepackage{wrapfig}


\usepackage[table,xcdraw]{xcolor}


%% ----------------------------------------------------------------
\begin{document}
\frontmatter
\title      {Heterogeneous Agent-based Model for Supermarket Competition}
\authors    {\texorpdfstring
             {\href{mailto:sc22g13@ecs.soton.ac.uk}{Stefan J. Collier}}
             {Stefan J. Collier}
            }
\addresses  {\groupname\\\deptname\\\univname}
\date       {\today}
\subject    {}
\keywords   {}
\supervisor {Dr. Maria Polukarov}
\examiner   {Professor Sheng Chen}

\maketitle
\begin{abstract}
This project aim was to model and analyse the effects of competitive pricing behaviors of grocery retailers on the British market. 

This was achieved by creating a multi-agent model, containing retailer and consumer agents. The heterogeneous crowd of retailers employs either a uniform pricing strategy or a ‘local price flexing’ strategy. The actions of these retailers are chosen by predicting the profit of each action, using a perceptron. Following on from the consideration of different economic models, a discrete model was developed so that software agents have a discrete environment to operate within. Within the model, it has been observed how supermarkets with differing behaviors affect a heterogeneous crowd of consumer agents. The model was implemented in Java with Python used to evaluate the results. 

The simulation displays good acceptance with real grocery market behavior, i.e. captures the performance of British retailers thus can be used to determine the impact of changes in their behavior on their competitors and consumers.Furthermore it can be used to provide insight into sustainability of volatile pricing strategies, providing a useful insight in volatility of British supermarket retail industry. 
\end{abstract}
\acknowledgements{
I would like to express my sincere gratitude to Dr Maria Polukarov for her guidance and support which provided me the freedom to take this research in the direction of my interest.\\
\\
I would also like to thank my family and friends for their encouragement and support. To those who quietly listened to my software complaints. To those who worked throughout the nights with me. To those who helped me write what I couldn't say. I cannot thank you enough.
}

\declaration{
I, Stefan Collier, declare that this dissertation and the work presented in it are my own and has been generated by me as the result of my own original research.\\
I confirm that:\\
1. This work was done wholly or mainly while in candidature for a degree at this University;\\
2. Where any part of this dissertation has previously been submitted for any other qualification at this University or any other institution, this has been clearly stated;\\
3. Where I have consulted the published work of others, this is always clearly attributed;\\
4. Where I have quoted from the work of others, the source is always given. With the exception of such quotations, this dissertation is entirely my own work;\\
5. I have acknowledged all main sources of help;\\
6. Where the thesis is based on work done by myself jointly with others, I have made clear exactly what was done by others and what I have contributed myself;\\
7. Either none of this work has been published before submission, or parts of this work have been published by :\\
\\
Stefan Collier\\
April 2016
}
\tableofcontents
\listoffigures
\listoftables

\mainmatter
%% ----------------------------------------------------------------
%\include{Introduction}
%\include{Conclusions}
\include{chapters/1Project/main}
\include{chapters/2Lit/main}
\include{chapters/3Design/HighLevel}
\include{chapters/3Design/InDepth}
\include{chapters/4Impl/main}

\include{chapters/5Experiments/1/main}
\include{chapters/5Experiments/2/main}
\include{chapters/5Experiments/3/main}
\include{chapters/5Experiments/4/main}

\include{chapters/6Conclusion/main}

\appendix
\include{appendix/AppendixB}
\include{appendix/D/main}
\include{appendix/AppendixC}

\backmatter
\bibliographystyle{ecs}
\bibliography{ECS}
\end{document}
%% ----------------------------------------------------------------


\appendix
\include{appendix/AppendixB}
 %% ----------------------------------------------------------------
%% Progress.tex
%% ---------------------------------------------------------------- 
\documentclass{ecsprogress}    % Use the progress Style
\graphicspath{{../figs/}}   % Location of your graphics files
    \usepackage{natbib}            % Use Natbib style for the refs.
\hypersetup{colorlinks=true}   % Set to false for black/white printing
\input{Definitions}            % Include your abbreviations



\usepackage{enumitem}% http://ctan.org/pkg/enumitem
\usepackage{multirow}
\usepackage{float}
\usepackage{amsmath}
\usepackage{multicol}
\usepackage{amssymb}
\usepackage[normalem]{ulem}
\useunder{\uline}{\ul}{}
\usepackage{wrapfig}


\usepackage[table,xcdraw]{xcolor}


%% ----------------------------------------------------------------
\begin{document}
\frontmatter
\title      {Heterogeneous Agent-based Model for Supermarket Competition}
\authors    {\texorpdfstring
             {\href{mailto:sc22g13@ecs.soton.ac.uk}{Stefan J. Collier}}
             {Stefan J. Collier}
            }
\addresses  {\groupname\\\deptname\\\univname}
\date       {\today}
\subject    {}
\keywords   {}
\supervisor {Dr. Maria Polukarov}
\examiner   {Professor Sheng Chen}

\maketitle
\begin{abstract}
This project aim was to model and analyse the effects of competitive pricing behaviors of grocery retailers on the British market. 

This was achieved by creating a multi-agent model, containing retailer and consumer agents. The heterogeneous crowd of retailers employs either a uniform pricing strategy or a ‘local price flexing’ strategy. The actions of these retailers are chosen by predicting the profit of each action, using a perceptron. Following on from the consideration of different economic models, a discrete model was developed so that software agents have a discrete environment to operate within. Within the model, it has been observed how supermarkets with differing behaviors affect a heterogeneous crowd of consumer agents. The model was implemented in Java with Python used to evaluate the results. 

The simulation displays good acceptance with real grocery market behavior, i.e. captures the performance of British retailers thus can be used to determine the impact of changes in their behavior on their competitors and consumers.Furthermore it can be used to provide insight into sustainability of volatile pricing strategies, providing a useful insight in volatility of British supermarket retail industry. 
\end{abstract}
\acknowledgements{
I would like to express my sincere gratitude to Dr Maria Polukarov for her guidance and support which provided me the freedom to take this research in the direction of my interest.\\
\\
I would also like to thank my family and friends for their encouragement and support. To those who quietly listened to my software complaints. To those who worked throughout the nights with me. To those who helped me write what I couldn't say. I cannot thank you enough.
}

\declaration{
I, Stefan Collier, declare that this dissertation and the work presented in it are my own and has been generated by me as the result of my own original research.\\
I confirm that:\\
1. This work was done wholly or mainly while in candidature for a degree at this University;\\
2. Where any part of this dissertation has previously been submitted for any other qualification at this University or any other institution, this has been clearly stated;\\
3. Where I have consulted the published work of others, this is always clearly attributed;\\
4. Where I have quoted from the work of others, the source is always given. With the exception of such quotations, this dissertation is entirely my own work;\\
5. I have acknowledged all main sources of help;\\
6. Where the thesis is based on work done by myself jointly with others, I have made clear exactly what was done by others and what I have contributed myself;\\
7. Either none of this work has been published before submission, or parts of this work have been published by :\\
\\
Stefan Collier\\
April 2016
}
\tableofcontents
\listoffigures
\listoftables

\mainmatter
%% ----------------------------------------------------------------
%\include{Introduction}
%\include{Conclusions}
\include{chapters/1Project/main}
\include{chapters/2Lit/main}
\include{chapters/3Design/HighLevel}
\include{chapters/3Design/InDepth}
\include{chapters/4Impl/main}

\include{chapters/5Experiments/1/main}
\include{chapters/5Experiments/2/main}
\include{chapters/5Experiments/3/main}
\include{chapters/5Experiments/4/main}

\include{chapters/6Conclusion/main}

\appendix
\include{appendix/AppendixB}
\include{appendix/D/main}
\include{appendix/AppendixC}

\backmatter
\bibliographystyle{ecs}
\bibliography{ECS}
\end{document}
%% ----------------------------------------------------------------

\include{appendix/AppendixC}

\backmatter
\bibliographystyle{ecs}
\bibliography{ECS}
\end{document}
%% ----------------------------------------------------------------

 %% ----------------------------------------------------------------
%% Progress.tex
%% ---------------------------------------------------------------- 
\documentclass{ecsprogress}    % Use the progress Style
\graphicspath{{../figs/}}   % Location of your graphics files
    \usepackage{natbib}            % Use Natbib style for the refs.
\hypersetup{colorlinks=true}   % Set to false for black/white printing
\input{Definitions}            % Include your abbreviations



\usepackage{enumitem}% http://ctan.org/pkg/enumitem
\usepackage{multirow}
\usepackage{float}
\usepackage{amsmath}
\usepackage{multicol}
\usepackage{amssymb}
\usepackage[normalem]{ulem}
\useunder{\uline}{\ul}{}
\usepackage{wrapfig}


\usepackage[table,xcdraw]{xcolor}


%% ----------------------------------------------------------------
\begin{document}
\frontmatter
\title      {Heterogeneous Agent-based Model for Supermarket Competition}
\authors    {\texorpdfstring
             {\href{mailto:sc22g13@ecs.soton.ac.uk}{Stefan J. Collier}}
             {Stefan J. Collier}
            }
\addresses  {\groupname\\\deptname\\\univname}
\date       {\today}
\subject    {}
\keywords   {}
\supervisor {Dr. Maria Polukarov}
\examiner   {Professor Sheng Chen}

\maketitle
\begin{abstract}
This project aim was to model and analyse the effects of competitive pricing behaviors of grocery retailers on the British market. 

This was achieved by creating a multi-agent model, containing retailer and consumer agents. The heterogeneous crowd of retailers employs either a uniform pricing strategy or a ‘local price flexing’ strategy. The actions of these retailers are chosen by predicting the profit of each action, using a perceptron. Following on from the consideration of different economic models, a discrete model was developed so that software agents have a discrete environment to operate within. Within the model, it has been observed how supermarkets with differing behaviors affect a heterogeneous crowd of consumer agents. The model was implemented in Java with Python used to evaluate the results. 

The simulation displays good acceptance with real grocery market behavior, i.e. captures the performance of British retailers thus can be used to determine the impact of changes in their behavior on their competitors and consumers.Furthermore it can be used to provide insight into sustainability of volatile pricing strategies, providing a useful insight in volatility of British supermarket retail industry. 
\end{abstract}
\acknowledgements{
I would like to express my sincere gratitude to Dr Maria Polukarov for her guidance and support which provided me the freedom to take this research in the direction of my interest.\\
\\
I would also like to thank my family and friends for their encouragement and support. To those who quietly listened to my software complaints. To those who worked throughout the nights with me. To those who helped me write what I couldn't say. I cannot thank you enough.
}

\declaration{
I, Stefan Collier, declare that this dissertation and the work presented in it are my own and has been generated by me as the result of my own original research.\\
I confirm that:\\
1. This work was done wholly or mainly while in candidature for a degree at this University;\\
2. Where any part of this dissertation has previously been submitted for any other qualification at this University or any other institution, this has been clearly stated;\\
3. Where I have consulted the published work of others, this is always clearly attributed;\\
4. Where I have quoted from the work of others, the source is always given. With the exception of such quotations, this dissertation is entirely my own work;\\
5. I have acknowledged all main sources of help;\\
6. Where the thesis is based on work done by myself jointly with others, I have made clear exactly what was done by others and what I have contributed myself;\\
7. Either none of this work has been published before submission, or parts of this work have been published by :\\
\\
Stefan Collier\\
April 2016
}
\tableofcontents
\listoffigures
\listoftables

\mainmatter
%% ----------------------------------------------------------------
%\include{Introduction}
%\include{Conclusions}
 %% ----------------------------------------------------------------
%% Progress.tex
%% ---------------------------------------------------------------- 
\documentclass{ecsprogress}    % Use the progress Style
\graphicspath{{../figs/}}   % Location of your graphics files
    \usepackage{natbib}            % Use Natbib style for the refs.
\hypersetup{colorlinks=true}   % Set to false for black/white printing
\input{Definitions}            % Include your abbreviations



\usepackage{enumitem}% http://ctan.org/pkg/enumitem
\usepackage{multirow}
\usepackage{float}
\usepackage{amsmath}
\usepackage{multicol}
\usepackage{amssymb}
\usepackage[normalem]{ulem}
\useunder{\uline}{\ul}{}
\usepackage{wrapfig}


\usepackage[table,xcdraw]{xcolor}


%% ----------------------------------------------------------------
\begin{document}
\frontmatter
\title      {Heterogeneous Agent-based Model for Supermarket Competition}
\authors    {\texorpdfstring
             {\href{mailto:sc22g13@ecs.soton.ac.uk}{Stefan J. Collier}}
             {Stefan J. Collier}
            }
\addresses  {\groupname\\\deptname\\\univname}
\date       {\today}
\subject    {}
\keywords   {}
\supervisor {Dr. Maria Polukarov}
\examiner   {Professor Sheng Chen}

\maketitle
\begin{abstract}
This project aim was to model and analyse the effects of competitive pricing behaviors of grocery retailers on the British market. 

This was achieved by creating a multi-agent model, containing retailer and consumer agents. The heterogeneous crowd of retailers employs either a uniform pricing strategy or a ‘local price flexing’ strategy. The actions of these retailers are chosen by predicting the profit of each action, using a perceptron. Following on from the consideration of different economic models, a discrete model was developed so that software agents have a discrete environment to operate within. Within the model, it has been observed how supermarkets with differing behaviors affect a heterogeneous crowd of consumer agents. The model was implemented in Java with Python used to evaluate the results. 

The simulation displays good acceptance with real grocery market behavior, i.e. captures the performance of British retailers thus can be used to determine the impact of changes in their behavior on their competitors and consumers.Furthermore it can be used to provide insight into sustainability of volatile pricing strategies, providing a useful insight in volatility of British supermarket retail industry. 
\end{abstract}
\acknowledgements{
I would like to express my sincere gratitude to Dr Maria Polukarov for her guidance and support which provided me the freedom to take this research in the direction of my interest.\\
\\
I would also like to thank my family and friends for their encouragement and support. To those who quietly listened to my software complaints. To those who worked throughout the nights with me. To those who helped me write what I couldn't say. I cannot thank you enough.
}

\declaration{
I, Stefan Collier, declare that this dissertation and the work presented in it are my own and has been generated by me as the result of my own original research.\\
I confirm that:\\
1. This work was done wholly or mainly while in candidature for a degree at this University;\\
2. Where any part of this dissertation has previously been submitted for any other qualification at this University or any other institution, this has been clearly stated;\\
3. Where I have consulted the published work of others, this is always clearly attributed;\\
4. Where I have quoted from the work of others, the source is always given. With the exception of such quotations, this dissertation is entirely my own work;\\
5. I have acknowledged all main sources of help;\\
6. Where the thesis is based on work done by myself jointly with others, I have made clear exactly what was done by others and what I have contributed myself;\\
7. Either none of this work has been published before submission, or parts of this work have been published by :\\
\\
Stefan Collier\\
April 2016
}
\tableofcontents
\listoffigures
\listoftables

\mainmatter
%% ----------------------------------------------------------------
%\include{Introduction}
%\include{Conclusions}
\include{chapters/1Project/main}
\include{chapters/2Lit/main}
\include{chapters/3Design/HighLevel}
\include{chapters/3Design/InDepth}
\include{chapters/4Impl/main}

\include{chapters/5Experiments/1/main}
\include{chapters/5Experiments/2/main}
\include{chapters/5Experiments/3/main}
\include{chapters/5Experiments/4/main}

\include{chapters/6Conclusion/main}

\appendix
\include{appendix/AppendixB}
\include{appendix/D/main}
\include{appendix/AppendixC}

\backmatter
\bibliographystyle{ecs}
\bibliography{ECS}
\end{document}
%% ----------------------------------------------------------------

 %% ----------------------------------------------------------------
%% Progress.tex
%% ---------------------------------------------------------------- 
\documentclass{ecsprogress}    % Use the progress Style
\graphicspath{{../figs/}}   % Location of your graphics files
    \usepackage{natbib}            % Use Natbib style for the refs.
\hypersetup{colorlinks=true}   % Set to false for black/white printing
\input{Definitions}            % Include your abbreviations



\usepackage{enumitem}% http://ctan.org/pkg/enumitem
\usepackage{multirow}
\usepackage{float}
\usepackage{amsmath}
\usepackage{multicol}
\usepackage{amssymb}
\usepackage[normalem]{ulem}
\useunder{\uline}{\ul}{}
\usepackage{wrapfig}


\usepackage[table,xcdraw]{xcolor}


%% ----------------------------------------------------------------
\begin{document}
\frontmatter
\title      {Heterogeneous Agent-based Model for Supermarket Competition}
\authors    {\texorpdfstring
             {\href{mailto:sc22g13@ecs.soton.ac.uk}{Stefan J. Collier}}
             {Stefan J. Collier}
            }
\addresses  {\groupname\\\deptname\\\univname}
\date       {\today}
\subject    {}
\keywords   {}
\supervisor {Dr. Maria Polukarov}
\examiner   {Professor Sheng Chen}

\maketitle
\begin{abstract}
This project aim was to model and analyse the effects of competitive pricing behaviors of grocery retailers on the British market. 

This was achieved by creating a multi-agent model, containing retailer and consumer agents. The heterogeneous crowd of retailers employs either a uniform pricing strategy or a ‘local price flexing’ strategy. The actions of these retailers are chosen by predicting the profit of each action, using a perceptron. Following on from the consideration of different economic models, a discrete model was developed so that software agents have a discrete environment to operate within. Within the model, it has been observed how supermarkets with differing behaviors affect a heterogeneous crowd of consumer agents. The model was implemented in Java with Python used to evaluate the results. 

The simulation displays good acceptance with real grocery market behavior, i.e. captures the performance of British retailers thus can be used to determine the impact of changes in their behavior on their competitors and consumers.Furthermore it can be used to provide insight into sustainability of volatile pricing strategies, providing a useful insight in volatility of British supermarket retail industry. 
\end{abstract}
\acknowledgements{
I would like to express my sincere gratitude to Dr Maria Polukarov for her guidance and support which provided me the freedom to take this research in the direction of my interest.\\
\\
I would also like to thank my family and friends for their encouragement and support. To those who quietly listened to my software complaints. To those who worked throughout the nights with me. To those who helped me write what I couldn't say. I cannot thank you enough.
}

\declaration{
I, Stefan Collier, declare that this dissertation and the work presented in it are my own and has been generated by me as the result of my own original research.\\
I confirm that:\\
1. This work was done wholly or mainly while in candidature for a degree at this University;\\
2. Where any part of this dissertation has previously been submitted for any other qualification at this University or any other institution, this has been clearly stated;\\
3. Where I have consulted the published work of others, this is always clearly attributed;\\
4. Where I have quoted from the work of others, the source is always given. With the exception of such quotations, this dissertation is entirely my own work;\\
5. I have acknowledged all main sources of help;\\
6. Where the thesis is based on work done by myself jointly with others, I have made clear exactly what was done by others and what I have contributed myself;\\
7. Either none of this work has been published before submission, or parts of this work have been published by :\\
\\
Stefan Collier\\
April 2016
}
\tableofcontents
\listoffigures
\listoftables

\mainmatter
%% ----------------------------------------------------------------
%\include{Introduction}
%\include{Conclusions}
\include{chapters/1Project/main}
\include{chapters/2Lit/main}
\include{chapters/3Design/HighLevel}
\include{chapters/3Design/InDepth}
\include{chapters/4Impl/main}

\include{chapters/5Experiments/1/main}
\include{chapters/5Experiments/2/main}
\include{chapters/5Experiments/3/main}
\include{chapters/5Experiments/4/main}

\include{chapters/6Conclusion/main}

\appendix
\include{appendix/AppendixB}
\include{appendix/D/main}
\include{appendix/AppendixC}

\backmatter
\bibliographystyle{ecs}
\bibliography{ECS}
\end{document}
%% ----------------------------------------------------------------

\include{chapters/3Design/HighLevel}
\include{chapters/3Design/InDepth}
 %% ----------------------------------------------------------------
%% Progress.tex
%% ---------------------------------------------------------------- 
\documentclass{ecsprogress}    % Use the progress Style
\graphicspath{{../figs/}}   % Location of your graphics files
    \usepackage{natbib}            % Use Natbib style for the refs.
\hypersetup{colorlinks=true}   % Set to false for black/white printing
\input{Definitions}            % Include your abbreviations



\usepackage{enumitem}% http://ctan.org/pkg/enumitem
\usepackage{multirow}
\usepackage{float}
\usepackage{amsmath}
\usepackage{multicol}
\usepackage{amssymb}
\usepackage[normalem]{ulem}
\useunder{\uline}{\ul}{}
\usepackage{wrapfig}


\usepackage[table,xcdraw]{xcolor}


%% ----------------------------------------------------------------
\begin{document}
\frontmatter
\title      {Heterogeneous Agent-based Model for Supermarket Competition}
\authors    {\texorpdfstring
             {\href{mailto:sc22g13@ecs.soton.ac.uk}{Stefan J. Collier}}
             {Stefan J. Collier}
            }
\addresses  {\groupname\\\deptname\\\univname}
\date       {\today}
\subject    {}
\keywords   {}
\supervisor {Dr. Maria Polukarov}
\examiner   {Professor Sheng Chen}

\maketitle
\begin{abstract}
This project aim was to model and analyse the effects of competitive pricing behaviors of grocery retailers on the British market. 

This was achieved by creating a multi-agent model, containing retailer and consumer agents. The heterogeneous crowd of retailers employs either a uniform pricing strategy or a ‘local price flexing’ strategy. The actions of these retailers are chosen by predicting the profit of each action, using a perceptron. Following on from the consideration of different economic models, a discrete model was developed so that software agents have a discrete environment to operate within. Within the model, it has been observed how supermarkets with differing behaviors affect a heterogeneous crowd of consumer agents. The model was implemented in Java with Python used to evaluate the results. 

The simulation displays good acceptance with real grocery market behavior, i.e. captures the performance of British retailers thus can be used to determine the impact of changes in their behavior on their competitors and consumers.Furthermore it can be used to provide insight into sustainability of volatile pricing strategies, providing a useful insight in volatility of British supermarket retail industry. 
\end{abstract}
\acknowledgements{
I would like to express my sincere gratitude to Dr Maria Polukarov for her guidance and support which provided me the freedom to take this research in the direction of my interest.\\
\\
I would also like to thank my family and friends for their encouragement and support. To those who quietly listened to my software complaints. To those who worked throughout the nights with me. To those who helped me write what I couldn't say. I cannot thank you enough.
}

\declaration{
I, Stefan Collier, declare that this dissertation and the work presented in it are my own and has been generated by me as the result of my own original research.\\
I confirm that:\\
1. This work was done wholly or mainly while in candidature for a degree at this University;\\
2. Where any part of this dissertation has previously been submitted for any other qualification at this University or any other institution, this has been clearly stated;\\
3. Where I have consulted the published work of others, this is always clearly attributed;\\
4. Where I have quoted from the work of others, the source is always given. With the exception of such quotations, this dissertation is entirely my own work;\\
5. I have acknowledged all main sources of help;\\
6. Where the thesis is based on work done by myself jointly with others, I have made clear exactly what was done by others and what I have contributed myself;\\
7. Either none of this work has been published before submission, or parts of this work have been published by :\\
\\
Stefan Collier\\
April 2016
}
\tableofcontents
\listoffigures
\listoftables

\mainmatter
%% ----------------------------------------------------------------
%\include{Introduction}
%\include{Conclusions}
\include{chapters/1Project/main}
\include{chapters/2Lit/main}
\include{chapters/3Design/HighLevel}
\include{chapters/3Design/InDepth}
\include{chapters/4Impl/main}

\include{chapters/5Experiments/1/main}
\include{chapters/5Experiments/2/main}
\include{chapters/5Experiments/3/main}
\include{chapters/5Experiments/4/main}

\include{chapters/6Conclusion/main}

\appendix
\include{appendix/AppendixB}
\include{appendix/D/main}
\include{appendix/AppendixC}

\backmatter
\bibliographystyle{ecs}
\bibliography{ECS}
\end{document}
%% ----------------------------------------------------------------


 %% ----------------------------------------------------------------
%% Progress.tex
%% ---------------------------------------------------------------- 
\documentclass{ecsprogress}    % Use the progress Style
\graphicspath{{../figs/}}   % Location of your graphics files
    \usepackage{natbib}            % Use Natbib style for the refs.
\hypersetup{colorlinks=true}   % Set to false for black/white printing
\input{Definitions}            % Include your abbreviations



\usepackage{enumitem}% http://ctan.org/pkg/enumitem
\usepackage{multirow}
\usepackage{float}
\usepackage{amsmath}
\usepackage{multicol}
\usepackage{amssymb}
\usepackage[normalem]{ulem}
\useunder{\uline}{\ul}{}
\usepackage{wrapfig}


\usepackage[table,xcdraw]{xcolor}


%% ----------------------------------------------------------------
\begin{document}
\frontmatter
\title      {Heterogeneous Agent-based Model for Supermarket Competition}
\authors    {\texorpdfstring
             {\href{mailto:sc22g13@ecs.soton.ac.uk}{Stefan J. Collier}}
             {Stefan J. Collier}
            }
\addresses  {\groupname\\\deptname\\\univname}
\date       {\today}
\subject    {}
\keywords   {}
\supervisor {Dr. Maria Polukarov}
\examiner   {Professor Sheng Chen}

\maketitle
\begin{abstract}
This project aim was to model and analyse the effects of competitive pricing behaviors of grocery retailers on the British market. 

This was achieved by creating a multi-agent model, containing retailer and consumer agents. The heterogeneous crowd of retailers employs either a uniform pricing strategy or a ‘local price flexing’ strategy. The actions of these retailers are chosen by predicting the profit of each action, using a perceptron. Following on from the consideration of different economic models, a discrete model was developed so that software agents have a discrete environment to operate within. Within the model, it has been observed how supermarkets with differing behaviors affect a heterogeneous crowd of consumer agents. The model was implemented in Java with Python used to evaluate the results. 

The simulation displays good acceptance with real grocery market behavior, i.e. captures the performance of British retailers thus can be used to determine the impact of changes in their behavior on their competitors and consumers.Furthermore it can be used to provide insight into sustainability of volatile pricing strategies, providing a useful insight in volatility of British supermarket retail industry. 
\end{abstract}
\acknowledgements{
I would like to express my sincere gratitude to Dr Maria Polukarov for her guidance and support which provided me the freedom to take this research in the direction of my interest.\\
\\
I would also like to thank my family and friends for their encouragement and support. To those who quietly listened to my software complaints. To those who worked throughout the nights with me. To those who helped me write what I couldn't say. I cannot thank you enough.
}

\declaration{
I, Stefan Collier, declare that this dissertation and the work presented in it are my own and has been generated by me as the result of my own original research.\\
I confirm that:\\
1. This work was done wholly or mainly while in candidature for a degree at this University;\\
2. Where any part of this dissertation has previously been submitted for any other qualification at this University or any other institution, this has been clearly stated;\\
3. Where I have consulted the published work of others, this is always clearly attributed;\\
4. Where I have quoted from the work of others, the source is always given. With the exception of such quotations, this dissertation is entirely my own work;\\
5. I have acknowledged all main sources of help;\\
6. Where the thesis is based on work done by myself jointly with others, I have made clear exactly what was done by others and what I have contributed myself;\\
7. Either none of this work has been published before submission, or parts of this work have been published by :\\
\\
Stefan Collier\\
April 2016
}
\tableofcontents
\listoffigures
\listoftables

\mainmatter
%% ----------------------------------------------------------------
%\include{Introduction}
%\include{Conclusions}
\include{chapters/1Project/main}
\include{chapters/2Lit/main}
\include{chapters/3Design/HighLevel}
\include{chapters/3Design/InDepth}
\include{chapters/4Impl/main}

\include{chapters/5Experiments/1/main}
\include{chapters/5Experiments/2/main}
\include{chapters/5Experiments/3/main}
\include{chapters/5Experiments/4/main}

\include{chapters/6Conclusion/main}

\appendix
\include{appendix/AppendixB}
\include{appendix/D/main}
\include{appendix/AppendixC}

\backmatter
\bibliographystyle{ecs}
\bibliography{ECS}
\end{document}
%% ----------------------------------------------------------------

 %% ----------------------------------------------------------------
%% Progress.tex
%% ---------------------------------------------------------------- 
\documentclass{ecsprogress}    % Use the progress Style
\graphicspath{{../figs/}}   % Location of your graphics files
    \usepackage{natbib}            % Use Natbib style for the refs.
\hypersetup{colorlinks=true}   % Set to false for black/white printing
\input{Definitions}            % Include your abbreviations



\usepackage{enumitem}% http://ctan.org/pkg/enumitem
\usepackage{multirow}
\usepackage{float}
\usepackage{amsmath}
\usepackage{multicol}
\usepackage{amssymb}
\usepackage[normalem]{ulem}
\useunder{\uline}{\ul}{}
\usepackage{wrapfig}


\usepackage[table,xcdraw]{xcolor}


%% ----------------------------------------------------------------
\begin{document}
\frontmatter
\title      {Heterogeneous Agent-based Model for Supermarket Competition}
\authors    {\texorpdfstring
             {\href{mailto:sc22g13@ecs.soton.ac.uk}{Stefan J. Collier}}
             {Stefan J. Collier}
            }
\addresses  {\groupname\\\deptname\\\univname}
\date       {\today}
\subject    {}
\keywords   {}
\supervisor {Dr. Maria Polukarov}
\examiner   {Professor Sheng Chen}

\maketitle
\begin{abstract}
This project aim was to model and analyse the effects of competitive pricing behaviors of grocery retailers on the British market. 

This was achieved by creating a multi-agent model, containing retailer and consumer agents. The heterogeneous crowd of retailers employs either a uniform pricing strategy or a ‘local price flexing’ strategy. The actions of these retailers are chosen by predicting the profit of each action, using a perceptron. Following on from the consideration of different economic models, a discrete model was developed so that software agents have a discrete environment to operate within. Within the model, it has been observed how supermarkets with differing behaviors affect a heterogeneous crowd of consumer agents. The model was implemented in Java with Python used to evaluate the results. 

The simulation displays good acceptance with real grocery market behavior, i.e. captures the performance of British retailers thus can be used to determine the impact of changes in their behavior on their competitors and consumers.Furthermore it can be used to provide insight into sustainability of volatile pricing strategies, providing a useful insight in volatility of British supermarket retail industry. 
\end{abstract}
\acknowledgements{
I would like to express my sincere gratitude to Dr Maria Polukarov for her guidance and support which provided me the freedom to take this research in the direction of my interest.\\
\\
I would also like to thank my family and friends for their encouragement and support. To those who quietly listened to my software complaints. To those who worked throughout the nights with me. To those who helped me write what I couldn't say. I cannot thank you enough.
}

\declaration{
I, Stefan Collier, declare that this dissertation and the work presented in it are my own and has been generated by me as the result of my own original research.\\
I confirm that:\\
1. This work was done wholly or mainly while in candidature for a degree at this University;\\
2. Where any part of this dissertation has previously been submitted for any other qualification at this University or any other institution, this has been clearly stated;\\
3. Where I have consulted the published work of others, this is always clearly attributed;\\
4. Where I have quoted from the work of others, the source is always given. With the exception of such quotations, this dissertation is entirely my own work;\\
5. I have acknowledged all main sources of help;\\
6. Where the thesis is based on work done by myself jointly with others, I have made clear exactly what was done by others and what I have contributed myself;\\
7. Either none of this work has been published before submission, or parts of this work have been published by :\\
\\
Stefan Collier\\
April 2016
}
\tableofcontents
\listoffigures
\listoftables

\mainmatter
%% ----------------------------------------------------------------
%\include{Introduction}
%\include{Conclusions}
\include{chapters/1Project/main}
\include{chapters/2Lit/main}
\include{chapters/3Design/HighLevel}
\include{chapters/3Design/InDepth}
\include{chapters/4Impl/main}

\include{chapters/5Experiments/1/main}
\include{chapters/5Experiments/2/main}
\include{chapters/5Experiments/3/main}
\include{chapters/5Experiments/4/main}

\include{chapters/6Conclusion/main}

\appendix
\include{appendix/AppendixB}
\include{appendix/D/main}
\include{appendix/AppendixC}

\backmatter
\bibliographystyle{ecs}
\bibliography{ECS}
\end{document}
%% ----------------------------------------------------------------

 %% ----------------------------------------------------------------
%% Progress.tex
%% ---------------------------------------------------------------- 
\documentclass{ecsprogress}    % Use the progress Style
\graphicspath{{../figs/}}   % Location of your graphics files
    \usepackage{natbib}            % Use Natbib style for the refs.
\hypersetup{colorlinks=true}   % Set to false for black/white printing
\input{Definitions}            % Include your abbreviations



\usepackage{enumitem}% http://ctan.org/pkg/enumitem
\usepackage{multirow}
\usepackage{float}
\usepackage{amsmath}
\usepackage{multicol}
\usepackage{amssymb}
\usepackage[normalem]{ulem}
\useunder{\uline}{\ul}{}
\usepackage{wrapfig}


\usepackage[table,xcdraw]{xcolor}


%% ----------------------------------------------------------------
\begin{document}
\frontmatter
\title      {Heterogeneous Agent-based Model for Supermarket Competition}
\authors    {\texorpdfstring
             {\href{mailto:sc22g13@ecs.soton.ac.uk}{Stefan J. Collier}}
             {Stefan J. Collier}
            }
\addresses  {\groupname\\\deptname\\\univname}
\date       {\today}
\subject    {}
\keywords   {}
\supervisor {Dr. Maria Polukarov}
\examiner   {Professor Sheng Chen}

\maketitle
\begin{abstract}
This project aim was to model and analyse the effects of competitive pricing behaviors of grocery retailers on the British market. 

This was achieved by creating a multi-agent model, containing retailer and consumer agents. The heterogeneous crowd of retailers employs either a uniform pricing strategy or a ‘local price flexing’ strategy. The actions of these retailers are chosen by predicting the profit of each action, using a perceptron. Following on from the consideration of different economic models, a discrete model was developed so that software agents have a discrete environment to operate within. Within the model, it has been observed how supermarkets with differing behaviors affect a heterogeneous crowd of consumer agents. The model was implemented in Java with Python used to evaluate the results. 

The simulation displays good acceptance with real grocery market behavior, i.e. captures the performance of British retailers thus can be used to determine the impact of changes in their behavior on their competitors and consumers.Furthermore it can be used to provide insight into sustainability of volatile pricing strategies, providing a useful insight in volatility of British supermarket retail industry. 
\end{abstract}
\acknowledgements{
I would like to express my sincere gratitude to Dr Maria Polukarov for her guidance and support which provided me the freedom to take this research in the direction of my interest.\\
\\
I would also like to thank my family and friends for their encouragement and support. To those who quietly listened to my software complaints. To those who worked throughout the nights with me. To those who helped me write what I couldn't say. I cannot thank you enough.
}

\declaration{
I, Stefan Collier, declare that this dissertation and the work presented in it are my own and has been generated by me as the result of my own original research.\\
I confirm that:\\
1. This work was done wholly or mainly while in candidature for a degree at this University;\\
2. Where any part of this dissertation has previously been submitted for any other qualification at this University or any other institution, this has been clearly stated;\\
3. Where I have consulted the published work of others, this is always clearly attributed;\\
4. Where I have quoted from the work of others, the source is always given. With the exception of such quotations, this dissertation is entirely my own work;\\
5. I have acknowledged all main sources of help;\\
6. Where the thesis is based on work done by myself jointly with others, I have made clear exactly what was done by others and what I have contributed myself;\\
7. Either none of this work has been published before submission, or parts of this work have been published by :\\
\\
Stefan Collier\\
April 2016
}
\tableofcontents
\listoffigures
\listoftables

\mainmatter
%% ----------------------------------------------------------------
%\include{Introduction}
%\include{Conclusions}
\include{chapters/1Project/main}
\include{chapters/2Lit/main}
\include{chapters/3Design/HighLevel}
\include{chapters/3Design/InDepth}
\include{chapters/4Impl/main}

\include{chapters/5Experiments/1/main}
\include{chapters/5Experiments/2/main}
\include{chapters/5Experiments/3/main}
\include{chapters/5Experiments/4/main}

\include{chapters/6Conclusion/main}

\appendix
\include{appendix/AppendixB}
\include{appendix/D/main}
\include{appendix/AppendixC}

\backmatter
\bibliographystyle{ecs}
\bibliography{ECS}
\end{document}
%% ----------------------------------------------------------------

 %% ----------------------------------------------------------------
%% Progress.tex
%% ---------------------------------------------------------------- 
\documentclass{ecsprogress}    % Use the progress Style
\graphicspath{{../figs/}}   % Location of your graphics files
    \usepackage{natbib}            % Use Natbib style for the refs.
\hypersetup{colorlinks=true}   % Set to false for black/white printing
\input{Definitions}            % Include your abbreviations



\usepackage{enumitem}% http://ctan.org/pkg/enumitem
\usepackage{multirow}
\usepackage{float}
\usepackage{amsmath}
\usepackage{multicol}
\usepackage{amssymb}
\usepackage[normalem]{ulem}
\useunder{\uline}{\ul}{}
\usepackage{wrapfig}


\usepackage[table,xcdraw]{xcolor}


%% ----------------------------------------------------------------
\begin{document}
\frontmatter
\title      {Heterogeneous Agent-based Model for Supermarket Competition}
\authors    {\texorpdfstring
             {\href{mailto:sc22g13@ecs.soton.ac.uk}{Stefan J. Collier}}
             {Stefan J. Collier}
            }
\addresses  {\groupname\\\deptname\\\univname}
\date       {\today}
\subject    {}
\keywords   {}
\supervisor {Dr. Maria Polukarov}
\examiner   {Professor Sheng Chen}

\maketitle
\begin{abstract}
This project aim was to model and analyse the effects of competitive pricing behaviors of grocery retailers on the British market. 

This was achieved by creating a multi-agent model, containing retailer and consumer agents. The heterogeneous crowd of retailers employs either a uniform pricing strategy or a ‘local price flexing’ strategy. The actions of these retailers are chosen by predicting the profit of each action, using a perceptron. Following on from the consideration of different economic models, a discrete model was developed so that software agents have a discrete environment to operate within. Within the model, it has been observed how supermarkets with differing behaviors affect a heterogeneous crowd of consumer agents. The model was implemented in Java with Python used to evaluate the results. 

The simulation displays good acceptance with real grocery market behavior, i.e. captures the performance of British retailers thus can be used to determine the impact of changes in their behavior on their competitors and consumers.Furthermore it can be used to provide insight into sustainability of volatile pricing strategies, providing a useful insight in volatility of British supermarket retail industry. 
\end{abstract}
\acknowledgements{
I would like to express my sincere gratitude to Dr Maria Polukarov for her guidance and support which provided me the freedom to take this research in the direction of my interest.\\
\\
I would also like to thank my family and friends for their encouragement and support. To those who quietly listened to my software complaints. To those who worked throughout the nights with me. To those who helped me write what I couldn't say. I cannot thank you enough.
}

\declaration{
I, Stefan Collier, declare that this dissertation and the work presented in it are my own and has been generated by me as the result of my own original research.\\
I confirm that:\\
1. This work was done wholly or mainly while in candidature for a degree at this University;\\
2. Where any part of this dissertation has previously been submitted for any other qualification at this University or any other institution, this has been clearly stated;\\
3. Where I have consulted the published work of others, this is always clearly attributed;\\
4. Where I have quoted from the work of others, the source is always given. With the exception of such quotations, this dissertation is entirely my own work;\\
5. I have acknowledged all main sources of help;\\
6. Where the thesis is based on work done by myself jointly with others, I have made clear exactly what was done by others and what I have contributed myself;\\
7. Either none of this work has been published before submission, or parts of this work have been published by :\\
\\
Stefan Collier\\
April 2016
}
\tableofcontents
\listoffigures
\listoftables

\mainmatter
%% ----------------------------------------------------------------
%\include{Introduction}
%\include{Conclusions}
\include{chapters/1Project/main}
\include{chapters/2Lit/main}
\include{chapters/3Design/HighLevel}
\include{chapters/3Design/InDepth}
\include{chapters/4Impl/main}

\include{chapters/5Experiments/1/main}
\include{chapters/5Experiments/2/main}
\include{chapters/5Experiments/3/main}
\include{chapters/5Experiments/4/main}

\include{chapters/6Conclusion/main}

\appendix
\include{appendix/AppendixB}
\include{appendix/D/main}
\include{appendix/AppendixC}

\backmatter
\bibliographystyle{ecs}
\bibliography{ECS}
\end{document}
%% ----------------------------------------------------------------


 %% ----------------------------------------------------------------
%% Progress.tex
%% ---------------------------------------------------------------- 
\documentclass{ecsprogress}    % Use the progress Style
\graphicspath{{../figs/}}   % Location of your graphics files
    \usepackage{natbib}            % Use Natbib style for the refs.
\hypersetup{colorlinks=true}   % Set to false for black/white printing
\input{Definitions}            % Include your abbreviations



\usepackage{enumitem}% http://ctan.org/pkg/enumitem
\usepackage{multirow}
\usepackage{float}
\usepackage{amsmath}
\usepackage{multicol}
\usepackage{amssymb}
\usepackage[normalem]{ulem}
\useunder{\uline}{\ul}{}
\usepackage{wrapfig}


\usepackage[table,xcdraw]{xcolor}


%% ----------------------------------------------------------------
\begin{document}
\frontmatter
\title      {Heterogeneous Agent-based Model for Supermarket Competition}
\authors    {\texorpdfstring
             {\href{mailto:sc22g13@ecs.soton.ac.uk}{Stefan J. Collier}}
             {Stefan J. Collier}
            }
\addresses  {\groupname\\\deptname\\\univname}
\date       {\today}
\subject    {}
\keywords   {}
\supervisor {Dr. Maria Polukarov}
\examiner   {Professor Sheng Chen}

\maketitle
\begin{abstract}
This project aim was to model and analyse the effects of competitive pricing behaviors of grocery retailers on the British market. 

This was achieved by creating a multi-agent model, containing retailer and consumer agents. The heterogeneous crowd of retailers employs either a uniform pricing strategy or a ‘local price flexing’ strategy. The actions of these retailers are chosen by predicting the profit of each action, using a perceptron. Following on from the consideration of different economic models, a discrete model was developed so that software agents have a discrete environment to operate within. Within the model, it has been observed how supermarkets with differing behaviors affect a heterogeneous crowd of consumer agents. The model was implemented in Java with Python used to evaluate the results. 

The simulation displays good acceptance with real grocery market behavior, i.e. captures the performance of British retailers thus can be used to determine the impact of changes in their behavior on their competitors and consumers.Furthermore it can be used to provide insight into sustainability of volatile pricing strategies, providing a useful insight in volatility of British supermarket retail industry. 
\end{abstract}
\acknowledgements{
I would like to express my sincere gratitude to Dr Maria Polukarov for her guidance and support which provided me the freedom to take this research in the direction of my interest.\\
\\
I would also like to thank my family and friends for their encouragement and support. To those who quietly listened to my software complaints. To those who worked throughout the nights with me. To those who helped me write what I couldn't say. I cannot thank you enough.
}

\declaration{
I, Stefan Collier, declare that this dissertation and the work presented in it are my own and has been generated by me as the result of my own original research.\\
I confirm that:\\
1. This work was done wholly or mainly while in candidature for a degree at this University;\\
2. Where any part of this dissertation has previously been submitted for any other qualification at this University or any other institution, this has been clearly stated;\\
3. Where I have consulted the published work of others, this is always clearly attributed;\\
4. Where I have quoted from the work of others, the source is always given. With the exception of such quotations, this dissertation is entirely my own work;\\
5. I have acknowledged all main sources of help;\\
6. Where the thesis is based on work done by myself jointly with others, I have made clear exactly what was done by others and what I have contributed myself;\\
7. Either none of this work has been published before submission, or parts of this work have been published by :\\
\\
Stefan Collier\\
April 2016
}
\tableofcontents
\listoffigures
\listoftables

\mainmatter
%% ----------------------------------------------------------------
%\include{Introduction}
%\include{Conclusions}
\include{chapters/1Project/main}
\include{chapters/2Lit/main}
\include{chapters/3Design/HighLevel}
\include{chapters/3Design/InDepth}
\include{chapters/4Impl/main}

\include{chapters/5Experiments/1/main}
\include{chapters/5Experiments/2/main}
\include{chapters/5Experiments/3/main}
\include{chapters/5Experiments/4/main}

\include{chapters/6Conclusion/main}

\appendix
\include{appendix/AppendixB}
\include{appendix/D/main}
\include{appendix/AppendixC}

\backmatter
\bibliographystyle{ecs}
\bibliography{ECS}
\end{document}
%% ----------------------------------------------------------------


\appendix
\include{appendix/AppendixB}
 %% ----------------------------------------------------------------
%% Progress.tex
%% ---------------------------------------------------------------- 
\documentclass{ecsprogress}    % Use the progress Style
\graphicspath{{../figs/}}   % Location of your graphics files
    \usepackage{natbib}            % Use Natbib style for the refs.
\hypersetup{colorlinks=true}   % Set to false for black/white printing
\input{Definitions}            % Include your abbreviations



\usepackage{enumitem}% http://ctan.org/pkg/enumitem
\usepackage{multirow}
\usepackage{float}
\usepackage{amsmath}
\usepackage{multicol}
\usepackage{amssymb}
\usepackage[normalem]{ulem}
\useunder{\uline}{\ul}{}
\usepackage{wrapfig}


\usepackage[table,xcdraw]{xcolor}


%% ----------------------------------------------------------------
\begin{document}
\frontmatter
\title      {Heterogeneous Agent-based Model for Supermarket Competition}
\authors    {\texorpdfstring
             {\href{mailto:sc22g13@ecs.soton.ac.uk}{Stefan J. Collier}}
             {Stefan J. Collier}
            }
\addresses  {\groupname\\\deptname\\\univname}
\date       {\today}
\subject    {}
\keywords   {}
\supervisor {Dr. Maria Polukarov}
\examiner   {Professor Sheng Chen}

\maketitle
\begin{abstract}
This project aim was to model and analyse the effects of competitive pricing behaviors of grocery retailers on the British market. 

This was achieved by creating a multi-agent model, containing retailer and consumer agents. The heterogeneous crowd of retailers employs either a uniform pricing strategy or a ‘local price flexing’ strategy. The actions of these retailers are chosen by predicting the profit of each action, using a perceptron. Following on from the consideration of different economic models, a discrete model was developed so that software agents have a discrete environment to operate within. Within the model, it has been observed how supermarkets with differing behaviors affect a heterogeneous crowd of consumer agents. The model was implemented in Java with Python used to evaluate the results. 

The simulation displays good acceptance with real grocery market behavior, i.e. captures the performance of British retailers thus can be used to determine the impact of changes in their behavior on their competitors and consumers.Furthermore it can be used to provide insight into sustainability of volatile pricing strategies, providing a useful insight in volatility of British supermarket retail industry. 
\end{abstract}
\acknowledgements{
I would like to express my sincere gratitude to Dr Maria Polukarov for her guidance and support which provided me the freedom to take this research in the direction of my interest.\\
\\
I would also like to thank my family and friends for their encouragement and support. To those who quietly listened to my software complaints. To those who worked throughout the nights with me. To those who helped me write what I couldn't say. I cannot thank you enough.
}

\declaration{
I, Stefan Collier, declare that this dissertation and the work presented in it are my own and has been generated by me as the result of my own original research.\\
I confirm that:\\
1. This work was done wholly or mainly while in candidature for a degree at this University;\\
2. Where any part of this dissertation has previously been submitted for any other qualification at this University or any other institution, this has been clearly stated;\\
3. Where I have consulted the published work of others, this is always clearly attributed;\\
4. Where I have quoted from the work of others, the source is always given. With the exception of such quotations, this dissertation is entirely my own work;\\
5. I have acknowledged all main sources of help;\\
6. Where the thesis is based on work done by myself jointly with others, I have made clear exactly what was done by others and what I have contributed myself;\\
7. Either none of this work has been published before submission, or parts of this work have been published by :\\
\\
Stefan Collier\\
April 2016
}
\tableofcontents
\listoffigures
\listoftables

\mainmatter
%% ----------------------------------------------------------------
%\include{Introduction}
%\include{Conclusions}
\include{chapters/1Project/main}
\include{chapters/2Lit/main}
\include{chapters/3Design/HighLevel}
\include{chapters/3Design/InDepth}
\include{chapters/4Impl/main}

\include{chapters/5Experiments/1/main}
\include{chapters/5Experiments/2/main}
\include{chapters/5Experiments/3/main}
\include{chapters/5Experiments/4/main}

\include{chapters/6Conclusion/main}

\appendix
\include{appendix/AppendixB}
\include{appendix/D/main}
\include{appendix/AppendixC}

\backmatter
\bibliographystyle{ecs}
\bibliography{ECS}
\end{document}
%% ----------------------------------------------------------------

\include{appendix/AppendixC}

\backmatter
\bibliographystyle{ecs}
\bibliography{ECS}
\end{document}
%% ----------------------------------------------------------------

 %% ----------------------------------------------------------------
%% Progress.tex
%% ---------------------------------------------------------------- 
\documentclass{ecsprogress}    % Use the progress Style
\graphicspath{{../figs/}}   % Location of your graphics files
    \usepackage{natbib}            % Use Natbib style for the refs.
\hypersetup{colorlinks=true}   % Set to false for black/white printing
\input{Definitions}            % Include your abbreviations



\usepackage{enumitem}% http://ctan.org/pkg/enumitem
\usepackage{multirow}
\usepackage{float}
\usepackage{amsmath}
\usepackage{multicol}
\usepackage{amssymb}
\usepackage[normalem]{ulem}
\useunder{\uline}{\ul}{}
\usepackage{wrapfig}


\usepackage[table,xcdraw]{xcolor}


%% ----------------------------------------------------------------
\begin{document}
\frontmatter
\title      {Heterogeneous Agent-based Model for Supermarket Competition}
\authors    {\texorpdfstring
             {\href{mailto:sc22g13@ecs.soton.ac.uk}{Stefan J. Collier}}
             {Stefan J. Collier}
            }
\addresses  {\groupname\\\deptname\\\univname}
\date       {\today}
\subject    {}
\keywords   {}
\supervisor {Dr. Maria Polukarov}
\examiner   {Professor Sheng Chen}

\maketitle
\begin{abstract}
This project aim was to model and analyse the effects of competitive pricing behaviors of grocery retailers on the British market. 

This was achieved by creating a multi-agent model, containing retailer and consumer agents. The heterogeneous crowd of retailers employs either a uniform pricing strategy or a ‘local price flexing’ strategy. The actions of these retailers are chosen by predicting the profit of each action, using a perceptron. Following on from the consideration of different economic models, a discrete model was developed so that software agents have a discrete environment to operate within. Within the model, it has been observed how supermarkets with differing behaviors affect a heterogeneous crowd of consumer agents. The model was implemented in Java with Python used to evaluate the results. 

The simulation displays good acceptance with real grocery market behavior, i.e. captures the performance of British retailers thus can be used to determine the impact of changes in their behavior on their competitors and consumers.Furthermore it can be used to provide insight into sustainability of volatile pricing strategies, providing a useful insight in volatility of British supermarket retail industry. 
\end{abstract}
\acknowledgements{
I would like to express my sincere gratitude to Dr Maria Polukarov for her guidance and support which provided me the freedom to take this research in the direction of my interest.\\
\\
I would also like to thank my family and friends for their encouragement and support. To those who quietly listened to my software complaints. To those who worked throughout the nights with me. To those who helped me write what I couldn't say. I cannot thank you enough.
}

\declaration{
I, Stefan Collier, declare that this dissertation and the work presented in it are my own and has been generated by me as the result of my own original research.\\
I confirm that:\\
1. This work was done wholly or mainly while in candidature for a degree at this University;\\
2. Where any part of this dissertation has previously been submitted for any other qualification at this University or any other institution, this has been clearly stated;\\
3. Where I have consulted the published work of others, this is always clearly attributed;\\
4. Where I have quoted from the work of others, the source is always given. With the exception of such quotations, this dissertation is entirely my own work;\\
5. I have acknowledged all main sources of help;\\
6. Where the thesis is based on work done by myself jointly with others, I have made clear exactly what was done by others and what I have contributed myself;\\
7. Either none of this work has been published before submission, or parts of this work have been published by :\\
\\
Stefan Collier\\
April 2016
}
\tableofcontents
\listoffigures
\listoftables

\mainmatter
%% ----------------------------------------------------------------
%\include{Introduction}
%\include{Conclusions}
 %% ----------------------------------------------------------------
%% Progress.tex
%% ---------------------------------------------------------------- 
\documentclass{ecsprogress}    % Use the progress Style
\graphicspath{{../figs/}}   % Location of your graphics files
    \usepackage{natbib}            % Use Natbib style for the refs.
\hypersetup{colorlinks=true}   % Set to false for black/white printing
\input{Definitions}            % Include your abbreviations



\usepackage{enumitem}% http://ctan.org/pkg/enumitem
\usepackage{multirow}
\usepackage{float}
\usepackage{amsmath}
\usepackage{multicol}
\usepackage{amssymb}
\usepackage[normalem]{ulem}
\useunder{\uline}{\ul}{}
\usepackage{wrapfig}


\usepackage[table,xcdraw]{xcolor}


%% ----------------------------------------------------------------
\begin{document}
\frontmatter
\title      {Heterogeneous Agent-based Model for Supermarket Competition}
\authors    {\texorpdfstring
             {\href{mailto:sc22g13@ecs.soton.ac.uk}{Stefan J. Collier}}
             {Stefan J. Collier}
            }
\addresses  {\groupname\\\deptname\\\univname}
\date       {\today}
\subject    {}
\keywords   {}
\supervisor {Dr. Maria Polukarov}
\examiner   {Professor Sheng Chen}

\maketitle
\begin{abstract}
This project aim was to model and analyse the effects of competitive pricing behaviors of grocery retailers on the British market. 

This was achieved by creating a multi-agent model, containing retailer and consumer agents. The heterogeneous crowd of retailers employs either a uniform pricing strategy or a ‘local price flexing’ strategy. The actions of these retailers are chosen by predicting the profit of each action, using a perceptron. Following on from the consideration of different economic models, a discrete model was developed so that software agents have a discrete environment to operate within. Within the model, it has been observed how supermarkets with differing behaviors affect a heterogeneous crowd of consumer agents. The model was implemented in Java with Python used to evaluate the results. 

The simulation displays good acceptance with real grocery market behavior, i.e. captures the performance of British retailers thus can be used to determine the impact of changes in their behavior on their competitors and consumers.Furthermore it can be used to provide insight into sustainability of volatile pricing strategies, providing a useful insight in volatility of British supermarket retail industry. 
\end{abstract}
\acknowledgements{
I would like to express my sincere gratitude to Dr Maria Polukarov for her guidance and support which provided me the freedom to take this research in the direction of my interest.\\
\\
I would also like to thank my family and friends for their encouragement and support. To those who quietly listened to my software complaints. To those who worked throughout the nights with me. To those who helped me write what I couldn't say. I cannot thank you enough.
}

\declaration{
I, Stefan Collier, declare that this dissertation and the work presented in it are my own and has been generated by me as the result of my own original research.\\
I confirm that:\\
1. This work was done wholly or mainly while in candidature for a degree at this University;\\
2. Where any part of this dissertation has previously been submitted for any other qualification at this University or any other institution, this has been clearly stated;\\
3. Where I have consulted the published work of others, this is always clearly attributed;\\
4. Where I have quoted from the work of others, the source is always given. With the exception of such quotations, this dissertation is entirely my own work;\\
5. I have acknowledged all main sources of help;\\
6. Where the thesis is based on work done by myself jointly with others, I have made clear exactly what was done by others and what I have contributed myself;\\
7. Either none of this work has been published before submission, or parts of this work have been published by :\\
\\
Stefan Collier\\
April 2016
}
\tableofcontents
\listoffigures
\listoftables

\mainmatter
%% ----------------------------------------------------------------
%\include{Introduction}
%\include{Conclusions}
\include{chapters/1Project/main}
\include{chapters/2Lit/main}
\include{chapters/3Design/HighLevel}
\include{chapters/3Design/InDepth}
\include{chapters/4Impl/main}

\include{chapters/5Experiments/1/main}
\include{chapters/5Experiments/2/main}
\include{chapters/5Experiments/3/main}
\include{chapters/5Experiments/4/main}

\include{chapters/6Conclusion/main}

\appendix
\include{appendix/AppendixB}
\include{appendix/D/main}
\include{appendix/AppendixC}

\backmatter
\bibliographystyle{ecs}
\bibliography{ECS}
\end{document}
%% ----------------------------------------------------------------

 %% ----------------------------------------------------------------
%% Progress.tex
%% ---------------------------------------------------------------- 
\documentclass{ecsprogress}    % Use the progress Style
\graphicspath{{../figs/}}   % Location of your graphics files
    \usepackage{natbib}            % Use Natbib style for the refs.
\hypersetup{colorlinks=true}   % Set to false for black/white printing
\input{Definitions}            % Include your abbreviations



\usepackage{enumitem}% http://ctan.org/pkg/enumitem
\usepackage{multirow}
\usepackage{float}
\usepackage{amsmath}
\usepackage{multicol}
\usepackage{amssymb}
\usepackage[normalem]{ulem}
\useunder{\uline}{\ul}{}
\usepackage{wrapfig}


\usepackage[table,xcdraw]{xcolor}


%% ----------------------------------------------------------------
\begin{document}
\frontmatter
\title      {Heterogeneous Agent-based Model for Supermarket Competition}
\authors    {\texorpdfstring
             {\href{mailto:sc22g13@ecs.soton.ac.uk}{Stefan J. Collier}}
             {Stefan J. Collier}
            }
\addresses  {\groupname\\\deptname\\\univname}
\date       {\today}
\subject    {}
\keywords   {}
\supervisor {Dr. Maria Polukarov}
\examiner   {Professor Sheng Chen}

\maketitle
\begin{abstract}
This project aim was to model and analyse the effects of competitive pricing behaviors of grocery retailers on the British market. 

This was achieved by creating a multi-agent model, containing retailer and consumer agents. The heterogeneous crowd of retailers employs either a uniform pricing strategy or a ‘local price flexing’ strategy. The actions of these retailers are chosen by predicting the profit of each action, using a perceptron. Following on from the consideration of different economic models, a discrete model was developed so that software agents have a discrete environment to operate within. Within the model, it has been observed how supermarkets with differing behaviors affect a heterogeneous crowd of consumer agents. The model was implemented in Java with Python used to evaluate the results. 

The simulation displays good acceptance with real grocery market behavior, i.e. captures the performance of British retailers thus can be used to determine the impact of changes in their behavior on their competitors and consumers.Furthermore it can be used to provide insight into sustainability of volatile pricing strategies, providing a useful insight in volatility of British supermarket retail industry. 
\end{abstract}
\acknowledgements{
I would like to express my sincere gratitude to Dr Maria Polukarov for her guidance and support which provided me the freedom to take this research in the direction of my interest.\\
\\
I would also like to thank my family and friends for their encouragement and support. To those who quietly listened to my software complaints. To those who worked throughout the nights with me. To those who helped me write what I couldn't say. I cannot thank you enough.
}

\declaration{
I, Stefan Collier, declare that this dissertation and the work presented in it are my own and has been generated by me as the result of my own original research.\\
I confirm that:\\
1. This work was done wholly or mainly while in candidature for a degree at this University;\\
2. Where any part of this dissertation has previously been submitted for any other qualification at this University or any other institution, this has been clearly stated;\\
3. Where I have consulted the published work of others, this is always clearly attributed;\\
4. Where I have quoted from the work of others, the source is always given. With the exception of such quotations, this dissertation is entirely my own work;\\
5. I have acknowledged all main sources of help;\\
6. Where the thesis is based on work done by myself jointly with others, I have made clear exactly what was done by others and what I have contributed myself;\\
7. Either none of this work has been published before submission, or parts of this work have been published by :\\
\\
Stefan Collier\\
April 2016
}
\tableofcontents
\listoffigures
\listoftables

\mainmatter
%% ----------------------------------------------------------------
%\include{Introduction}
%\include{Conclusions}
\include{chapters/1Project/main}
\include{chapters/2Lit/main}
\include{chapters/3Design/HighLevel}
\include{chapters/3Design/InDepth}
\include{chapters/4Impl/main}

\include{chapters/5Experiments/1/main}
\include{chapters/5Experiments/2/main}
\include{chapters/5Experiments/3/main}
\include{chapters/5Experiments/4/main}

\include{chapters/6Conclusion/main}

\appendix
\include{appendix/AppendixB}
\include{appendix/D/main}
\include{appendix/AppendixC}

\backmatter
\bibliographystyle{ecs}
\bibliography{ECS}
\end{document}
%% ----------------------------------------------------------------

\include{chapters/3Design/HighLevel}
\include{chapters/3Design/InDepth}
 %% ----------------------------------------------------------------
%% Progress.tex
%% ---------------------------------------------------------------- 
\documentclass{ecsprogress}    % Use the progress Style
\graphicspath{{../figs/}}   % Location of your graphics files
    \usepackage{natbib}            % Use Natbib style for the refs.
\hypersetup{colorlinks=true}   % Set to false for black/white printing
\input{Definitions}            % Include your abbreviations



\usepackage{enumitem}% http://ctan.org/pkg/enumitem
\usepackage{multirow}
\usepackage{float}
\usepackage{amsmath}
\usepackage{multicol}
\usepackage{amssymb}
\usepackage[normalem]{ulem}
\useunder{\uline}{\ul}{}
\usepackage{wrapfig}


\usepackage[table,xcdraw]{xcolor}


%% ----------------------------------------------------------------
\begin{document}
\frontmatter
\title      {Heterogeneous Agent-based Model for Supermarket Competition}
\authors    {\texorpdfstring
             {\href{mailto:sc22g13@ecs.soton.ac.uk}{Stefan J. Collier}}
             {Stefan J. Collier}
            }
\addresses  {\groupname\\\deptname\\\univname}
\date       {\today}
\subject    {}
\keywords   {}
\supervisor {Dr. Maria Polukarov}
\examiner   {Professor Sheng Chen}

\maketitle
\begin{abstract}
This project aim was to model and analyse the effects of competitive pricing behaviors of grocery retailers on the British market. 

This was achieved by creating a multi-agent model, containing retailer and consumer agents. The heterogeneous crowd of retailers employs either a uniform pricing strategy or a ‘local price flexing’ strategy. The actions of these retailers are chosen by predicting the profit of each action, using a perceptron. Following on from the consideration of different economic models, a discrete model was developed so that software agents have a discrete environment to operate within. Within the model, it has been observed how supermarkets with differing behaviors affect a heterogeneous crowd of consumer agents. The model was implemented in Java with Python used to evaluate the results. 

The simulation displays good acceptance with real grocery market behavior, i.e. captures the performance of British retailers thus can be used to determine the impact of changes in their behavior on their competitors and consumers.Furthermore it can be used to provide insight into sustainability of volatile pricing strategies, providing a useful insight in volatility of British supermarket retail industry. 
\end{abstract}
\acknowledgements{
I would like to express my sincere gratitude to Dr Maria Polukarov for her guidance and support which provided me the freedom to take this research in the direction of my interest.\\
\\
I would also like to thank my family and friends for their encouragement and support. To those who quietly listened to my software complaints. To those who worked throughout the nights with me. To those who helped me write what I couldn't say. I cannot thank you enough.
}

\declaration{
I, Stefan Collier, declare that this dissertation and the work presented in it are my own and has been generated by me as the result of my own original research.\\
I confirm that:\\
1. This work was done wholly or mainly while in candidature for a degree at this University;\\
2. Where any part of this dissertation has previously been submitted for any other qualification at this University or any other institution, this has been clearly stated;\\
3. Where I have consulted the published work of others, this is always clearly attributed;\\
4. Where I have quoted from the work of others, the source is always given. With the exception of such quotations, this dissertation is entirely my own work;\\
5. I have acknowledged all main sources of help;\\
6. Where the thesis is based on work done by myself jointly with others, I have made clear exactly what was done by others and what I have contributed myself;\\
7. Either none of this work has been published before submission, or parts of this work have been published by :\\
\\
Stefan Collier\\
April 2016
}
\tableofcontents
\listoffigures
\listoftables

\mainmatter
%% ----------------------------------------------------------------
%\include{Introduction}
%\include{Conclusions}
\include{chapters/1Project/main}
\include{chapters/2Lit/main}
\include{chapters/3Design/HighLevel}
\include{chapters/3Design/InDepth}
\include{chapters/4Impl/main}

\include{chapters/5Experiments/1/main}
\include{chapters/5Experiments/2/main}
\include{chapters/5Experiments/3/main}
\include{chapters/5Experiments/4/main}

\include{chapters/6Conclusion/main}

\appendix
\include{appendix/AppendixB}
\include{appendix/D/main}
\include{appendix/AppendixC}

\backmatter
\bibliographystyle{ecs}
\bibliography{ECS}
\end{document}
%% ----------------------------------------------------------------


 %% ----------------------------------------------------------------
%% Progress.tex
%% ---------------------------------------------------------------- 
\documentclass{ecsprogress}    % Use the progress Style
\graphicspath{{../figs/}}   % Location of your graphics files
    \usepackage{natbib}            % Use Natbib style for the refs.
\hypersetup{colorlinks=true}   % Set to false for black/white printing
\input{Definitions}            % Include your abbreviations



\usepackage{enumitem}% http://ctan.org/pkg/enumitem
\usepackage{multirow}
\usepackage{float}
\usepackage{amsmath}
\usepackage{multicol}
\usepackage{amssymb}
\usepackage[normalem]{ulem}
\useunder{\uline}{\ul}{}
\usepackage{wrapfig}


\usepackage[table,xcdraw]{xcolor}


%% ----------------------------------------------------------------
\begin{document}
\frontmatter
\title      {Heterogeneous Agent-based Model for Supermarket Competition}
\authors    {\texorpdfstring
             {\href{mailto:sc22g13@ecs.soton.ac.uk}{Stefan J. Collier}}
             {Stefan J. Collier}
            }
\addresses  {\groupname\\\deptname\\\univname}
\date       {\today}
\subject    {}
\keywords   {}
\supervisor {Dr. Maria Polukarov}
\examiner   {Professor Sheng Chen}

\maketitle
\begin{abstract}
This project aim was to model and analyse the effects of competitive pricing behaviors of grocery retailers on the British market. 

This was achieved by creating a multi-agent model, containing retailer and consumer agents. The heterogeneous crowd of retailers employs either a uniform pricing strategy or a ‘local price flexing’ strategy. The actions of these retailers are chosen by predicting the profit of each action, using a perceptron. Following on from the consideration of different economic models, a discrete model was developed so that software agents have a discrete environment to operate within. Within the model, it has been observed how supermarkets with differing behaviors affect a heterogeneous crowd of consumer agents. The model was implemented in Java with Python used to evaluate the results. 

The simulation displays good acceptance with real grocery market behavior, i.e. captures the performance of British retailers thus can be used to determine the impact of changes in their behavior on their competitors and consumers.Furthermore it can be used to provide insight into sustainability of volatile pricing strategies, providing a useful insight in volatility of British supermarket retail industry. 
\end{abstract}
\acknowledgements{
I would like to express my sincere gratitude to Dr Maria Polukarov for her guidance and support which provided me the freedom to take this research in the direction of my interest.\\
\\
I would also like to thank my family and friends for their encouragement and support. To those who quietly listened to my software complaints. To those who worked throughout the nights with me. To those who helped me write what I couldn't say. I cannot thank you enough.
}

\declaration{
I, Stefan Collier, declare that this dissertation and the work presented in it are my own and has been generated by me as the result of my own original research.\\
I confirm that:\\
1. This work was done wholly or mainly while in candidature for a degree at this University;\\
2. Where any part of this dissertation has previously been submitted for any other qualification at this University or any other institution, this has been clearly stated;\\
3. Where I have consulted the published work of others, this is always clearly attributed;\\
4. Where I have quoted from the work of others, the source is always given. With the exception of such quotations, this dissertation is entirely my own work;\\
5. I have acknowledged all main sources of help;\\
6. Where the thesis is based on work done by myself jointly with others, I have made clear exactly what was done by others and what I have contributed myself;\\
7. Either none of this work has been published before submission, or parts of this work have been published by :\\
\\
Stefan Collier\\
April 2016
}
\tableofcontents
\listoffigures
\listoftables

\mainmatter
%% ----------------------------------------------------------------
%\include{Introduction}
%\include{Conclusions}
\include{chapters/1Project/main}
\include{chapters/2Lit/main}
\include{chapters/3Design/HighLevel}
\include{chapters/3Design/InDepth}
\include{chapters/4Impl/main}

\include{chapters/5Experiments/1/main}
\include{chapters/5Experiments/2/main}
\include{chapters/5Experiments/3/main}
\include{chapters/5Experiments/4/main}

\include{chapters/6Conclusion/main}

\appendix
\include{appendix/AppendixB}
\include{appendix/D/main}
\include{appendix/AppendixC}

\backmatter
\bibliographystyle{ecs}
\bibliography{ECS}
\end{document}
%% ----------------------------------------------------------------

 %% ----------------------------------------------------------------
%% Progress.tex
%% ---------------------------------------------------------------- 
\documentclass{ecsprogress}    % Use the progress Style
\graphicspath{{../figs/}}   % Location of your graphics files
    \usepackage{natbib}            % Use Natbib style for the refs.
\hypersetup{colorlinks=true}   % Set to false for black/white printing
\input{Definitions}            % Include your abbreviations



\usepackage{enumitem}% http://ctan.org/pkg/enumitem
\usepackage{multirow}
\usepackage{float}
\usepackage{amsmath}
\usepackage{multicol}
\usepackage{amssymb}
\usepackage[normalem]{ulem}
\useunder{\uline}{\ul}{}
\usepackage{wrapfig}


\usepackage[table,xcdraw]{xcolor}


%% ----------------------------------------------------------------
\begin{document}
\frontmatter
\title      {Heterogeneous Agent-based Model for Supermarket Competition}
\authors    {\texorpdfstring
             {\href{mailto:sc22g13@ecs.soton.ac.uk}{Stefan J. Collier}}
             {Stefan J. Collier}
            }
\addresses  {\groupname\\\deptname\\\univname}
\date       {\today}
\subject    {}
\keywords   {}
\supervisor {Dr. Maria Polukarov}
\examiner   {Professor Sheng Chen}

\maketitle
\begin{abstract}
This project aim was to model and analyse the effects of competitive pricing behaviors of grocery retailers on the British market. 

This was achieved by creating a multi-agent model, containing retailer and consumer agents. The heterogeneous crowd of retailers employs either a uniform pricing strategy or a ‘local price flexing’ strategy. The actions of these retailers are chosen by predicting the profit of each action, using a perceptron. Following on from the consideration of different economic models, a discrete model was developed so that software agents have a discrete environment to operate within. Within the model, it has been observed how supermarkets with differing behaviors affect a heterogeneous crowd of consumer agents. The model was implemented in Java with Python used to evaluate the results. 

The simulation displays good acceptance with real grocery market behavior, i.e. captures the performance of British retailers thus can be used to determine the impact of changes in their behavior on their competitors and consumers.Furthermore it can be used to provide insight into sustainability of volatile pricing strategies, providing a useful insight in volatility of British supermarket retail industry. 
\end{abstract}
\acknowledgements{
I would like to express my sincere gratitude to Dr Maria Polukarov for her guidance and support which provided me the freedom to take this research in the direction of my interest.\\
\\
I would also like to thank my family and friends for their encouragement and support. To those who quietly listened to my software complaints. To those who worked throughout the nights with me. To those who helped me write what I couldn't say. I cannot thank you enough.
}

\declaration{
I, Stefan Collier, declare that this dissertation and the work presented in it are my own and has been generated by me as the result of my own original research.\\
I confirm that:\\
1. This work was done wholly or mainly while in candidature for a degree at this University;\\
2. Where any part of this dissertation has previously been submitted for any other qualification at this University or any other institution, this has been clearly stated;\\
3. Where I have consulted the published work of others, this is always clearly attributed;\\
4. Where I have quoted from the work of others, the source is always given. With the exception of such quotations, this dissertation is entirely my own work;\\
5. I have acknowledged all main sources of help;\\
6. Where the thesis is based on work done by myself jointly with others, I have made clear exactly what was done by others and what I have contributed myself;\\
7. Either none of this work has been published before submission, or parts of this work have been published by :\\
\\
Stefan Collier\\
April 2016
}
\tableofcontents
\listoffigures
\listoftables

\mainmatter
%% ----------------------------------------------------------------
%\include{Introduction}
%\include{Conclusions}
\include{chapters/1Project/main}
\include{chapters/2Lit/main}
\include{chapters/3Design/HighLevel}
\include{chapters/3Design/InDepth}
\include{chapters/4Impl/main}

\include{chapters/5Experiments/1/main}
\include{chapters/5Experiments/2/main}
\include{chapters/5Experiments/3/main}
\include{chapters/5Experiments/4/main}

\include{chapters/6Conclusion/main}

\appendix
\include{appendix/AppendixB}
\include{appendix/D/main}
\include{appendix/AppendixC}

\backmatter
\bibliographystyle{ecs}
\bibliography{ECS}
\end{document}
%% ----------------------------------------------------------------

 %% ----------------------------------------------------------------
%% Progress.tex
%% ---------------------------------------------------------------- 
\documentclass{ecsprogress}    % Use the progress Style
\graphicspath{{../figs/}}   % Location of your graphics files
    \usepackage{natbib}            % Use Natbib style for the refs.
\hypersetup{colorlinks=true}   % Set to false for black/white printing
\input{Definitions}            % Include your abbreviations



\usepackage{enumitem}% http://ctan.org/pkg/enumitem
\usepackage{multirow}
\usepackage{float}
\usepackage{amsmath}
\usepackage{multicol}
\usepackage{amssymb}
\usepackage[normalem]{ulem}
\useunder{\uline}{\ul}{}
\usepackage{wrapfig}


\usepackage[table,xcdraw]{xcolor}


%% ----------------------------------------------------------------
\begin{document}
\frontmatter
\title      {Heterogeneous Agent-based Model for Supermarket Competition}
\authors    {\texorpdfstring
             {\href{mailto:sc22g13@ecs.soton.ac.uk}{Stefan J. Collier}}
             {Stefan J. Collier}
            }
\addresses  {\groupname\\\deptname\\\univname}
\date       {\today}
\subject    {}
\keywords   {}
\supervisor {Dr. Maria Polukarov}
\examiner   {Professor Sheng Chen}

\maketitle
\begin{abstract}
This project aim was to model and analyse the effects of competitive pricing behaviors of grocery retailers on the British market. 

This was achieved by creating a multi-agent model, containing retailer and consumer agents. The heterogeneous crowd of retailers employs either a uniform pricing strategy or a ‘local price flexing’ strategy. The actions of these retailers are chosen by predicting the profit of each action, using a perceptron. Following on from the consideration of different economic models, a discrete model was developed so that software agents have a discrete environment to operate within. Within the model, it has been observed how supermarkets with differing behaviors affect a heterogeneous crowd of consumer agents. The model was implemented in Java with Python used to evaluate the results. 

The simulation displays good acceptance with real grocery market behavior, i.e. captures the performance of British retailers thus can be used to determine the impact of changes in their behavior on their competitors and consumers.Furthermore it can be used to provide insight into sustainability of volatile pricing strategies, providing a useful insight in volatility of British supermarket retail industry. 
\end{abstract}
\acknowledgements{
I would like to express my sincere gratitude to Dr Maria Polukarov for her guidance and support which provided me the freedom to take this research in the direction of my interest.\\
\\
I would also like to thank my family and friends for their encouragement and support. To those who quietly listened to my software complaints. To those who worked throughout the nights with me. To those who helped me write what I couldn't say. I cannot thank you enough.
}

\declaration{
I, Stefan Collier, declare that this dissertation and the work presented in it are my own and has been generated by me as the result of my own original research.\\
I confirm that:\\
1. This work was done wholly or mainly while in candidature for a degree at this University;\\
2. Where any part of this dissertation has previously been submitted for any other qualification at this University or any other institution, this has been clearly stated;\\
3. Where I have consulted the published work of others, this is always clearly attributed;\\
4. Where I have quoted from the work of others, the source is always given. With the exception of such quotations, this dissertation is entirely my own work;\\
5. I have acknowledged all main sources of help;\\
6. Where the thesis is based on work done by myself jointly with others, I have made clear exactly what was done by others and what I have contributed myself;\\
7. Either none of this work has been published before submission, or parts of this work have been published by :\\
\\
Stefan Collier\\
April 2016
}
\tableofcontents
\listoffigures
\listoftables

\mainmatter
%% ----------------------------------------------------------------
%\include{Introduction}
%\include{Conclusions}
\include{chapters/1Project/main}
\include{chapters/2Lit/main}
\include{chapters/3Design/HighLevel}
\include{chapters/3Design/InDepth}
\include{chapters/4Impl/main}

\include{chapters/5Experiments/1/main}
\include{chapters/5Experiments/2/main}
\include{chapters/5Experiments/3/main}
\include{chapters/5Experiments/4/main}

\include{chapters/6Conclusion/main}

\appendix
\include{appendix/AppendixB}
\include{appendix/D/main}
\include{appendix/AppendixC}

\backmatter
\bibliographystyle{ecs}
\bibliography{ECS}
\end{document}
%% ----------------------------------------------------------------

 %% ----------------------------------------------------------------
%% Progress.tex
%% ---------------------------------------------------------------- 
\documentclass{ecsprogress}    % Use the progress Style
\graphicspath{{../figs/}}   % Location of your graphics files
    \usepackage{natbib}            % Use Natbib style for the refs.
\hypersetup{colorlinks=true}   % Set to false for black/white printing
\input{Definitions}            % Include your abbreviations



\usepackage{enumitem}% http://ctan.org/pkg/enumitem
\usepackage{multirow}
\usepackage{float}
\usepackage{amsmath}
\usepackage{multicol}
\usepackage{amssymb}
\usepackage[normalem]{ulem}
\useunder{\uline}{\ul}{}
\usepackage{wrapfig}


\usepackage[table,xcdraw]{xcolor}


%% ----------------------------------------------------------------
\begin{document}
\frontmatter
\title      {Heterogeneous Agent-based Model for Supermarket Competition}
\authors    {\texorpdfstring
             {\href{mailto:sc22g13@ecs.soton.ac.uk}{Stefan J. Collier}}
             {Stefan J. Collier}
            }
\addresses  {\groupname\\\deptname\\\univname}
\date       {\today}
\subject    {}
\keywords   {}
\supervisor {Dr. Maria Polukarov}
\examiner   {Professor Sheng Chen}

\maketitle
\begin{abstract}
This project aim was to model and analyse the effects of competitive pricing behaviors of grocery retailers on the British market. 

This was achieved by creating a multi-agent model, containing retailer and consumer agents. The heterogeneous crowd of retailers employs either a uniform pricing strategy or a ‘local price flexing’ strategy. The actions of these retailers are chosen by predicting the profit of each action, using a perceptron. Following on from the consideration of different economic models, a discrete model was developed so that software agents have a discrete environment to operate within. Within the model, it has been observed how supermarkets with differing behaviors affect a heterogeneous crowd of consumer agents. The model was implemented in Java with Python used to evaluate the results. 

The simulation displays good acceptance with real grocery market behavior, i.e. captures the performance of British retailers thus can be used to determine the impact of changes in their behavior on their competitors and consumers.Furthermore it can be used to provide insight into sustainability of volatile pricing strategies, providing a useful insight in volatility of British supermarket retail industry. 
\end{abstract}
\acknowledgements{
I would like to express my sincere gratitude to Dr Maria Polukarov for her guidance and support which provided me the freedom to take this research in the direction of my interest.\\
\\
I would also like to thank my family and friends for their encouragement and support. To those who quietly listened to my software complaints. To those who worked throughout the nights with me. To those who helped me write what I couldn't say. I cannot thank you enough.
}

\declaration{
I, Stefan Collier, declare that this dissertation and the work presented in it are my own and has been generated by me as the result of my own original research.\\
I confirm that:\\
1. This work was done wholly or mainly while in candidature for a degree at this University;\\
2. Where any part of this dissertation has previously been submitted for any other qualification at this University or any other institution, this has been clearly stated;\\
3. Where I have consulted the published work of others, this is always clearly attributed;\\
4. Where I have quoted from the work of others, the source is always given. With the exception of such quotations, this dissertation is entirely my own work;\\
5. I have acknowledged all main sources of help;\\
6. Where the thesis is based on work done by myself jointly with others, I have made clear exactly what was done by others and what I have contributed myself;\\
7. Either none of this work has been published before submission, or parts of this work have been published by :\\
\\
Stefan Collier\\
April 2016
}
\tableofcontents
\listoffigures
\listoftables

\mainmatter
%% ----------------------------------------------------------------
%\include{Introduction}
%\include{Conclusions}
\include{chapters/1Project/main}
\include{chapters/2Lit/main}
\include{chapters/3Design/HighLevel}
\include{chapters/3Design/InDepth}
\include{chapters/4Impl/main}

\include{chapters/5Experiments/1/main}
\include{chapters/5Experiments/2/main}
\include{chapters/5Experiments/3/main}
\include{chapters/5Experiments/4/main}

\include{chapters/6Conclusion/main}

\appendix
\include{appendix/AppendixB}
\include{appendix/D/main}
\include{appendix/AppendixC}

\backmatter
\bibliographystyle{ecs}
\bibliography{ECS}
\end{document}
%% ----------------------------------------------------------------


 %% ----------------------------------------------------------------
%% Progress.tex
%% ---------------------------------------------------------------- 
\documentclass{ecsprogress}    % Use the progress Style
\graphicspath{{../figs/}}   % Location of your graphics files
    \usepackage{natbib}            % Use Natbib style for the refs.
\hypersetup{colorlinks=true}   % Set to false for black/white printing
\input{Definitions}            % Include your abbreviations



\usepackage{enumitem}% http://ctan.org/pkg/enumitem
\usepackage{multirow}
\usepackage{float}
\usepackage{amsmath}
\usepackage{multicol}
\usepackage{amssymb}
\usepackage[normalem]{ulem}
\useunder{\uline}{\ul}{}
\usepackage{wrapfig}


\usepackage[table,xcdraw]{xcolor}


%% ----------------------------------------------------------------
\begin{document}
\frontmatter
\title      {Heterogeneous Agent-based Model for Supermarket Competition}
\authors    {\texorpdfstring
             {\href{mailto:sc22g13@ecs.soton.ac.uk}{Stefan J. Collier}}
             {Stefan J. Collier}
            }
\addresses  {\groupname\\\deptname\\\univname}
\date       {\today}
\subject    {}
\keywords   {}
\supervisor {Dr. Maria Polukarov}
\examiner   {Professor Sheng Chen}

\maketitle
\begin{abstract}
This project aim was to model and analyse the effects of competitive pricing behaviors of grocery retailers on the British market. 

This was achieved by creating a multi-agent model, containing retailer and consumer agents. The heterogeneous crowd of retailers employs either a uniform pricing strategy or a ‘local price flexing’ strategy. The actions of these retailers are chosen by predicting the profit of each action, using a perceptron. Following on from the consideration of different economic models, a discrete model was developed so that software agents have a discrete environment to operate within. Within the model, it has been observed how supermarkets with differing behaviors affect a heterogeneous crowd of consumer agents. The model was implemented in Java with Python used to evaluate the results. 

The simulation displays good acceptance with real grocery market behavior, i.e. captures the performance of British retailers thus can be used to determine the impact of changes in their behavior on their competitors and consumers.Furthermore it can be used to provide insight into sustainability of volatile pricing strategies, providing a useful insight in volatility of British supermarket retail industry. 
\end{abstract}
\acknowledgements{
I would like to express my sincere gratitude to Dr Maria Polukarov for her guidance and support which provided me the freedom to take this research in the direction of my interest.\\
\\
I would also like to thank my family and friends for their encouragement and support. To those who quietly listened to my software complaints. To those who worked throughout the nights with me. To those who helped me write what I couldn't say. I cannot thank you enough.
}

\declaration{
I, Stefan Collier, declare that this dissertation and the work presented in it are my own and has been generated by me as the result of my own original research.\\
I confirm that:\\
1. This work was done wholly or mainly while in candidature for a degree at this University;\\
2. Where any part of this dissertation has previously been submitted for any other qualification at this University or any other institution, this has been clearly stated;\\
3. Where I have consulted the published work of others, this is always clearly attributed;\\
4. Where I have quoted from the work of others, the source is always given. With the exception of such quotations, this dissertation is entirely my own work;\\
5. I have acknowledged all main sources of help;\\
6. Where the thesis is based on work done by myself jointly with others, I have made clear exactly what was done by others and what I have contributed myself;\\
7. Either none of this work has been published before submission, or parts of this work have been published by :\\
\\
Stefan Collier\\
April 2016
}
\tableofcontents
\listoffigures
\listoftables

\mainmatter
%% ----------------------------------------------------------------
%\include{Introduction}
%\include{Conclusions}
\include{chapters/1Project/main}
\include{chapters/2Lit/main}
\include{chapters/3Design/HighLevel}
\include{chapters/3Design/InDepth}
\include{chapters/4Impl/main}

\include{chapters/5Experiments/1/main}
\include{chapters/5Experiments/2/main}
\include{chapters/5Experiments/3/main}
\include{chapters/5Experiments/4/main}

\include{chapters/6Conclusion/main}

\appendix
\include{appendix/AppendixB}
\include{appendix/D/main}
\include{appendix/AppendixC}

\backmatter
\bibliographystyle{ecs}
\bibliography{ECS}
\end{document}
%% ----------------------------------------------------------------


\appendix
\include{appendix/AppendixB}
 %% ----------------------------------------------------------------
%% Progress.tex
%% ---------------------------------------------------------------- 
\documentclass{ecsprogress}    % Use the progress Style
\graphicspath{{../figs/}}   % Location of your graphics files
    \usepackage{natbib}            % Use Natbib style for the refs.
\hypersetup{colorlinks=true}   % Set to false for black/white printing
\input{Definitions}            % Include your abbreviations



\usepackage{enumitem}% http://ctan.org/pkg/enumitem
\usepackage{multirow}
\usepackage{float}
\usepackage{amsmath}
\usepackage{multicol}
\usepackage{amssymb}
\usepackage[normalem]{ulem}
\useunder{\uline}{\ul}{}
\usepackage{wrapfig}


\usepackage[table,xcdraw]{xcolor}


%% ----------------------------------------------------------------
\begin{document}
\frontmatter
\title      {Heterogeneous Agent-based Model for Supermarket Competition}
\authors    {\texorpdfstring
             {\href{mailto:sc22g13@ecs.soton.ac.uk}{Stefan J. Collier}}
             {Stefan J. Collier}
            }
\addresses  {\groupname\\\deptname\\\univname}
\date       {\today}
\subject    {}
\keywords   {}
\supervisor {Dr. Maria Polukarov}
\examiner   {Professor Sheng Chen}

\maketitle
\begin{abstract}
This project aim was to model and analyse the effects of competitive pricing behaviors of grocery retailers on the British market. 

This was achieved by creating a multi-agent model, containing retailer and consumer agents. The heterogeneous crowd of retailers employs either a uniform pricing strategy or a ‘local price flexing’ strategy. The actions of these retailers are chosen by predicting the profit of each action, using a perceptron. Following on from the consideration of different economic models, a discrete model was developed so that software agents have a discrete environment to operate within. Within the model, it has been observed how supermarkets with differing behaviors affect a heterogeneous crowd of consumer agents. The model was implemented in Java with Python used to evaluate the results. 

The simulation displays good acceptance with real grocery market behavior, i.e. captures the performance of British retailers thus can be used to determine the impact of changes in their behavior on their competitors and consumers.Furthermore it can be used to provide insight into sustainability of volatile pricing strategies, providing a useful insight in volatility of British supermarket retail industry. 
\end{abstract}
\acknowledgements{
I would like to express my sincere gratitude to Dr Maria Polukarov for her guidance and support which provided me the freedom to take this research in the direction of my interest.\\
\\
I would also like to thank my family and friends for their encouragement and support. To those who quietly listened to my software complaints. To those who worked throughout the nights with me. To those who helped me write what I couldn't say. I cannot thank you enough.
}

\declaration{
I, Stefan Collier, declare that this dissertation and the work presented in it are my own and has been generated by me as the result of my own original research.\\
I confirm that:\\
1. This work was done wholly or mainly while in candidature for a degree at this University;\\
2. Where any part of this dissertation has previously been submitted for any other qualification at this University or any other institution, this has been clearly stated;\\
3. Where I have consulted the published work of others, this is always clearly attributed;\\
4. Where I have quoted from the work of others, the source is always given. With the exception of such quotations, this dissertation is entirely my own work;\\
5. I have acknowledged all main sources of help;\\
6. Where the thesis is based on work done by myself jointly with others, I have made clear exactly what was done by others and what I have contributed myself;\\
7. Either none of this work has been published before submission, or parts of this work have been published by :\\
\\
Stefan Collier\\
April 2016
}
\tableofcontents
\listoffigures
\listoftables

\mainmatter
%% ----------------------------------------------------------------
%\include{Introduction}
%\include{Conclusions}
\include{chapters/1Project/main}
\include{chapters/2Lit/main}
\include{chapters/3Design/HighLevel}
\include{chapters/3Design/InDepth}
\include{chapters/4Impl/main}

\include{chapters/5Experiments/1/main}
\include{chapters/5Experiments/2/main}
\include{chapters/5Experiments/3/main}
\include{chapters/5Experiments/4/main}

\include{chapters/6Conclusion/main}

\appendix
\include{appendix/AppendixB}
\include{appendix/D/main}
\include{appendix/AppendixC}

\backmatter
\bibliographystyle{ecs}
\bibliography{ECS}
\end{document}
%% ----------------------------------------------------------------

\include{appendix/AppendixC}

\backmatter
\bibliographystyle{ecs}
\bibliography{ECS}
\end{document}
%% ----------------------------------------------------------------


 %% ----------------------------------------------------------------
%% Progress.tex
%% ---------------------------------------------------------------- 
\documentclass{ecsprogress}    % Use the progress Style
\graphicspath{{../figs/}}   % Location of your graphics files
    \usepackage{natbib}            % Use Natbib style for the refs.
\hypersetup{colorlinks=true}   % Set to false for black/white printing
\input{Definitions}            % Include your abbreviations



\usepackage{enumitem}% http://ctan.org/pkg/enumitem
\usepackage{multirow}
\usepackage{float}
\usepackage{amsmath}
\usepackage{multicol}
\usepackage{amssymb}
\usepackage[normalem]{ulem}
\useunder{\uline}{\ul}{}
\usepackage{wrapfig}


\usepackage[table,xcdraw]{xcolor}


%% ----------------------------------------------------------------
\begin{document}
\frontmatter
\title      {Heterogeneous Agent-based Model for Supermarket Competition}
\authors    {\texorpdfstring
             {\href{mailto:sc22g13@ecs.soton.ac.uk}{Stefan J. Collier}}
             {Stefan J. Collier}
            }
\addresses  {\groupname\\\deptname\\\univname}
\date       {\today}
\subject    {}
\keywords   {}
\supervisor {Dr. Maria Polukarov}
\examiner   {Professor Sheng Chen}

\maketitle
\begin{abstract}
This project aim was to model and analyse the effects of competitive pricing behaviors of grocery retailers on the British market. 

This was achieved by creating a multi-agent model, containing retailer and consumer agents. The heterogeneous crowd of retailers employs either a uniform pricing strategy or a ‘local price flexing’ strategy. The actions of these retailers are chosen by predicting the profit of each action, using a perceptron. Following on from the consideration of different economic models, a discrete model was developed so that software agents have a discrete environment to operate within. Within the model, it has been observed how supermarkets with differing behaviors affect a heterogeneous crowd of consumer agents. The model was implemented in Java with Python used to evaluate the results. 

The simulation displays good acceptance with real grocery market behavior, i.e. captures the performance of British retailers thus can be used to determine the impact of changes in their behavior on their competitors and consumers.Furthermore it can be used to provide insight into sustainability of volatile pricing strategies, providing a useful insight in volatility of British supermarket retail industry. 
\end{abstract}
\acknowledgements{
I would like to express my sincere gratitude to Dr Maria Polukarov for her guidance and support which provided me the freedom to take this research in the direction of my interest.\\
\\
I would also like to thank my family and friends for their encouragement and support. To those who quietly listened to my software complaints. To those who worked throughout the nights with me. To those who helped me write what I couldn't say. I cannot thank you enough.
}

\declaration{
I, Stefan Collier, declare that this dissertation and the work presented in it are my own and has been generated by me as the result of my own original research.\\
I confirm that:\\
1. This work was done wholly or mainly while in candidature for a degree at this University;\\
2. Where any part of this dissertation has previously been submitted for any other qualification at this University or any other institution, this has been clearly stated;\\
3. Where I have consulted the published work of others, this is always clearly attributed;\\
4. Where I have quoted from the work of others, the source is always given. With the exception of such quotations, this dissertation is entirely my own work;\\
5. I have acknowledged all main sources of help;\\
6. Where the thesis is based on work done by myself jointly with others, I have made clear exactly what was done by others and what I have contributed myself;\\
7. Either none of this work has been published before submission, or parts of this work have been published by :\\
\\
Stefan Collier\\
April 2016
}
\tableofcontents
\listoffigures
\listoftables

\mainmatter
%% ----------------------------------------------------------------
%\include{Introduction}
%\include{Conclusions}
 %% ----------------------------------------------------------------
%% Progress.tex
%% ---------------------------------------------------------------- 
\documentclass{ecsprogress}    % Use the progress Style
\graphicspath{{../figs/}}   % Location of your graphics files
    \usepackage{natbib}            % Use Natbib style for the refs.
\hypersetup{colorlinks=true}   % Set to false for black/white printing
\input{Definitions}            % Include your abbreviations



\usepackage{enumitem}% http://ctan.org/pkg/enumitem
\usepackage{multirow}
\usepackage{float}
\usepackage{amsmath}
\usepackage{multicol}
\usepackage{amssymb}
\usepackage[normalem]{ulem}
\useunder{\uline}{\ul}{}
\usepackage{wrapfig}


\usepackage[table,xcdraw]{xcolor}


%% ----------------------------------------------------------------
\begin{document}
\frontmatter
\title      {Heterogeneous Agent-based Model for Supermarket Competition}
\authors    {\texorpdfstring
             {\href{mailto:sc22g13@ecs.soton.ac.uk}{Stefan J. Collier}}
             {Stefan J. Collier}
            }
\addresses  {\groupname\\\deptname\\\univname}
\date       {\today}
\subject    {}
\keywords   {}
\supervisor {Dr. Maria Polukarov}
\examiner   {Professor Sheng Chen}

\maketitle
\begin{abstract}
This project aim was to model and analyse the effects of competitive pricing behaviors of grocery retailers on the British market. 

This was achieved by creating a multi-agent model, containing retailer and consumer agents. The heterogeneous crowd of retailers employs either a uniform pricing strategy or a ‘local price flexing’ strategy. The actions of these retailers are chosen by predicting the profit of each action, using a perceptron. Following on from the consideration of different economic models, a discrete model was developed so that software agents have a discrete environment to operate within. Within the model, it has been observed how supermarkets with differing behaviors affect a heterogeneous crowd of consumer agents. The model was implemented in Java with Python used to evaluate the results. 

The simulation displays good acceptance with real grocery market behavior, i.e. captures the performance of British retailers thus can be used to determine the impact of changes in their behavior on their competitors and consumers.Furthermore it can be used to provide insight into sustainability of volatile pricing strategies, providing a useful insight in volatility of British supermarket retail industry. 
\end{abstract}
\acknowledgements{
I would like to express my sincere gratitude to Dr Maria Polukarov for her guidance and support which provided me the freedom to take this research in the direction of my interest.\\
\\
I would also like to thank my family and friends for their encouragement and support. To those who quietly listened to my software complaints. To those who worked throughout the nights with me. To those who helped me write what I couldn't say. I cannot thank you enough.
}

\declaration{
I, Stefan Collier, declare that this dissertation and the work presented in it are my own and has been generated by me as the result of my own original research.\\
I confirm that:\\
1. This work was done wholly or mainly while in candidature for a degree at this University;\\
2. Where any part of this dissertation has previously been submitted for any other qualification at this University or any other institution, this has been clearly stated;\\
3. Where I have consulted the published work of others, this is always clearly attributed;\\
4. Where I have quoted from the work of others, the source is always given. With the exception of such quotations, this dissertation is entirely my own work;\\
5. I have acknowledged all main sources of help;\\
6. Where the thesis is based on work done by myself jointly with others, I have made clear exactly what was done by others and what I have contributed myself;\\
7. Either none of this work has been published before submission, or parts of this work have been published by :\\
\\
Stefan Collier\\
April 2016
}
\tableofcontents
\listoffigures
\listoftables

\mainmatter
%% ----------------------------------------------------------------
%\include{Introduction}
%\include{Conclusions}
\include{chapters/1Project/main}
\include{chapters/2Lit/main}
\include{chapters/3Design/HighLevel}
\include{chapters/3Design/InDepth}
\include{chapters/4Impl/main}

\include{chapters/5Experiments/1/main}
\include{chapters/5Experiments/2/main}
\include{chapters/5Experiments/3/main}
\include{chapters/5Experiments/4/main}

\include{chapters/6Conclusion/main}

\appendix
\include{appendix/AppendixB}
\include{appendix/D/main}
\include{appendix/AppendixC}

\backmatter
\bibliographystyle{ecs}
\bibliography{ECS}
\end{document}
%% ----------------------------------------------------------------

 %% ----------------------------------------------------------------
%% Progress.tex
%% ---------------------------------------------------------------- 
\documentclass{ecsprogress}    % Use the progress Style
\graphicspath{{../figs/}}   % Location of your graphics files
    \usepackage{natbib}            % Use Natbib style for the refs.
\hypersetup{colorlinks=true}   % Set to false for black/white printing
\input{Definitions}            % Include your abbreviations



\usepackage{enumitem}% http://ctan.org/pkg/enumitem
\usepackage{multirow}
\usepackage{float}
\usepackage{amsmath}
\usepackage{multicol}
\usepackage{amssymb}
\usepackage[normalem]{ulem}
\useunder{\uline}{\ul}{}
\usepackage{wrapfig}


\usepackage[table,xcdraw]{xcolor}


%% ----------------------------------------------------------------
\begin{document}
\frontmatter
\title      {Heterogeneous Agent-based Model for Supermarket Competition}
\authors    {\texorpdfstring
             {\href{mailto:sc22g13@ecs.soton.ac.uk}{Stefan J. Collier}}
             {Stefan J. Collier}
            }
\addresses  {\groupname\\\deptname\\\univname}
\date       {\today}
\subject    {}
\keywords   {}
\supervisor {Dr. Maria Polukarov}
\examiner   {Professor Sheng Chen}

\maketitle
\begin{abstract}
This project aim was to model and analyse the effects of competitive pricing behaviors of grocery retailers on the British market. 

This was achieved by creating a multi-agent model, containing retailer and consumer agents. The heterogeneous crowd of retailers employs either a uniform pricing strategy or a ‘local price flexing’ strategy. The actions of these retailers are chosen by predicting the profit of each action, using a perceptron. Following on from the consideration of different economic models, a discrete model was developed so that software agents have a discrete environment to operate within. Within the model, it has been observed how supermarkets with differing behaviors affect a heterogeneous crowd of consumer agents. The model was implemented in Java with Python used to evaluate the results. 

The simulation displays good acceptance with real grocery market behavior, i.e. captures the performance of British retailers thus can be used to determine the impact of changes in their behavior on their competitors and consumers.Furthermore it can be used to provide insight into sustainability of volatile pricing strategies, providing a useful insight in volatility of British supermarket retail industry. 
\end{abstract}
\acknowledgements{
I would like to express my sincere gratitude to Dr Maria Polukarov for her guidance and support which provided me the freedom to take this research in the direction of my interest.\\
\\
I would also like to thank my family and friends for their encouragement and support. To those who quietly listened to my software complaints. To those who worked throughout the nights with me. To those who helped me write what I couldn't say. I cannot thank you enough.
}

\declaration{
I, Stefan Collier, declare that this dissertation and the work presented in it are my own and has been generated by me as the result of my own original research.\\
I confirm that:\\
1. This work was done wholly or mainly while in candidature for a degree at this University;\\
2. Where any part of this dissertation has previously been submitted for any other qualification at this University or any other institution, this has been clearly stated;\\
3. Where I have consulted the published work of others, this is always clearly attributed;\\
4. Where I have quoted from the work of others, the source is always given. With the exception of such quotations, this dissertation is entirely my own work;\\
5. I have acknowledged all main sources of help;\\
6. Where the thesis is based on work done by myself jointly with others, I have made clear exactly what was done by others and what I have contributed myself;\\
7. Either none of this work has been published before submission, or parts of this work have been published by :\\
\\
Stefan Collier\\
April 2016
}
\tableofcontents
\listoffigures
\listoftables

\mainmatter
%% ----------------------------------------------------------------
%\include{Introduction}
%\include{Conclusions}
\include{chapters/1Project/main}
\include{chapters/2Lit/main}
\include{chapters/3Design/HighLevel}
\include{chapters/3Design/InDepth}
\include{chapters/4Impl/main}

\include{chapters/5Experiments/1/main}
\include{chapters/5Experiments/2/main}
\include{chapters/5Experiments/3/main}
\include{chapters/5Experiments/4/main}

\include{chapters/6Conclusion/main}

\appendix
\include{appendix/AppendixB}
\include{appendix/D/main}
\include{appendix/AppendixC}

\backmatter
\bibliographystyle{ecs}
\bibliography{ECS}
\end{document}
%% ----------------------------------------------------------------

\include{chapters/3Design/HighLevel}
\include{chapters/3Design/InDepth}
 %% ----------------------------------------------------------------
%% Progress.tex
%% ---------------------------------------------------------------- 
\documentclass{ecsprogress}    % Use the progress Style
\graphicspath{{../figs/}}   % Location of your graphics files
    \usepackage{natbib}            % Use Natbib style for the refs.
\hypersetup{colorlinks=true}   % Set to false for black/white printing
\input{Definitions}            % Include your abbreviations



\usepackage{enumitem}% http://ctan.org/pkg/enumitem
\usepackage{multirow}
\usepackage{float}
\usepackage{amsmath}
\usepackage{multicol}
\usepackage{amssymb}
\usepackage[normalem]{ulem}
\useunder{\uline}{\ul}{}
\usepackage{wrapfig}


\usepackage[table,xcdraw]{xcolor}


%% ----------------------------------------------------------------
\begin{document}
\frontmatter
\title      {Heterogeneous Agent-based Model for Supermarket Competition}
\authors    {\texorpdfstring
             {\href{mailto:sc22g13@ecs.soton.ac.uk}{Stefan J. Collier}}
             {Stefan J. Collier}
            }
\addresses  {\groupname\\\deptname\\\univname}
\date       {\today}
\subject    {}
\keywords   {}
\supervisor {Dr. Maria Polukarov}
\examiner   {Professor Sheng Chen}

\maketitle
\begin{abstract}
This project aim was to model and analyse the effects of competitive pricing behaviors of grocery retailers on the British market. 

This was achieved by creating a multi-agent model, containing retailer and consumer agents. The heterogeneous crowd of retailers employs either a uniform pricing strategy or a ‘local price flexing’ strategy. The actions of these retailers are chosen by predicting the profit of each action, using a perceptron. Following on from the consideration of different economic models, a discrete model was developed so that software agents have a discrete environment to operate within. Within the model, it has been observed how supermarkets with differing behaviors affect a heterogeneous crowd of consumer agents. The model was implemented in Java with Python used to evaluate the results. 

The simulation displays good acceptance with real grocery market behavior, i.e. captures the performance of British retailers thus can be used to determine the impact of changes in their behavior on their competitors and consumers.Furthermore it can be used to provide insight into sustainability of volatile pricing strategies, providing a useful insight in volatility of British supermarket retail industry. 
\end{abstract}
\acknowledgements{
I would like to express my sincere gratitude to Dr Maria Polukarov for her guidance and support which provided me the freedom to take this research in the direction of my interest.\\
\\
I would also like to thank my family and friends for their encouragement and support. To those who quietly listened to my software complaints. To those who worked throughout the nights with me. To those who helped me write what I couldn't say. I cannot thank you enough.
}

\declaration{
I, Stefan Collier, declare that this dissertation and the work presented in it are my own and has been generated by me as the result of my own original research.\\
I confirm that:\\
1. This work was done wholly or mainly while in candidature for a degree at this University;\\
2. Where any part of this dissertation has previously been submitted for any other qualification at this University or any other institution, this has been clearly stated;\\
3. Where I have consulted the published work of others, this is always clearly attributed;\\
4. Where I have quoted from the work of others, the source is always given. With the exception of such quotations, this dissertation is entirely my own work;\\
5. I have acknowledged all main sources of help;\\
6. Where the thesis is based on work done by myself jointly with others, I have made clear exactly what was done by others and what I have contributed myself;\\
7. Either none of this work has been published before submission, or parts of this work have been published by :\\
\\
Stefan Collier\\
April 2016
}
\tableofcontents
\listoffigures
\listoftables

\mainmatter
%% ----------------------------------------------------------------
%\include{Introduction}
%\include{Conclusions}
\include{chapters/1Project/main}
\include{chapters/2Lit/main}
\include{chapters/3Design/HighLevel}
\include{chapters/3Design/InDepth}
\include{chapters/4Impl/main}

\include{chapters/5Experiments/1/main}
\include{chapters/5Experiments/2/main}
\include{chapters/5Experiments/3/main}
\include{chapters/5Experiments/4/main}

\include{chapters/6Conclusion/main}

\appendix
\include{appendix/AppendixB}
\include{appendix/D/main}
\include{appendix/AppendixC}

\backmatter
\bibliographystyle{ecs}
\bibliography{ECS}
\end{document}
%% ----------------------------------------------------------------


 %% ----------------------------------------------------------------
%% Progress.tex
%% ---------------------------------------------------------------- 
\documentclass{ecsprogress}    % Use the progress Style
\graphicspath{{../figs/}}   % Location of your graphics files
    \usepackage{natbib}            % Use Natbib style for the refs.
\hypersetup{colorlinks=true}   % Set to false for black/white printing
\input{Definitions}            % Include your abbreviations



\usepackage{enumitem}% http://ctan.org/pkg/enumitem
\usepackage{multirow}
\usepackage{float}
\usepackage{amsmath}
\usepackage{multicol}
\usepackage{amssymb}
\usepackage[normalem]{ulem}
\useunder{\uline}{\ul}{}
\usepackage{wrapfig}


\usepackage[table,xcdraw]{xcolor}


%% ----------------------------------------------------------------
\begin{document}
\frontmatter
\title      {Heterogeneous Agent-based Model for Supermarket Competition}
\authors    {\texorpdfstring
             {\href{mailto:sc22g13@ecs.soton.ac.uk}{Stefan J. Collier}}
             {Stefan J. Collier}
            }
\addresses  {\groupname\\\deptname\\\univname}
\date       {\today}
\subject    {}
\keywords   {}
\supervisor {Dr. Maria Polukarov}
\examiner   {Professor Sheng Chen}

\maketitle
\begin{abstract}
This project aim was to model and analyse the effects of competitive pricing behaviors of grocery retailers on the British market. 

This was achieved by creating a multi-agent model, containing retailer and consumer agents. The heterogeneous crowd of retailers employs either a uniform pricing strategy or a ‘local price flexing’ strategy. The actions of these retailers are chosen by predicting the profit of each action, using a perceptron. Following on from the consideration of different economic models, a discrete model was developed so that software agents have a discrete environment to operate within. Within the model, it has been observed how supermarkets with differing behaviors affect a heterogeneous crowd of consumer agents. The model was implemented in Java with Python used to evaluate the results. 

The simulation displays good acceptance with real grocery market behavior, i.e. captures the performance of British retailers thus can be used to determine the impact of changes in their behavior on their competitors and consumers.Furthermore it can be used to provide insight into sustainability of volatile pricing strategies, providing a useful insight in volatility of British supermarket retail industry. 
\end{abstract}
\acknowledgements{
I would like to express my sincere gratitude to Dr Maria Polukarov for her guidance and support which provided me the freedom to take this research in the direction of my interest.\\
\\
I would also like to thank my family and friends for their encouragement and support. To those who quietly listened to my software complaints. To those who worked throughout the nights with me. To those who helped me write what I couldn't say. I cannot thank you enough.
}

\declaration{
I, Stefan Collier, declare that this dissertation and the work presented in it are my own and has been generated by me as the result of my own original research.\\
I confirm that:\\
1. This work was done wholly or mainly while in candidature for a degree at this University;\\
2. Where any part of this dissertation has previously been submitted for any other qualification at this University or any other institution, this has been clearly stated;\\
3. Where I have consulted the published work of others, this is always clearly attributed;\\
4. Where I have quoted from the work of others, the source is always given. With the exception of such quotations, this dissertation is entirely my own work;\\
5. I have acknowledged all main sources of help;\\
6. Where the thesis is based on work done by myself jointly with others, I have made clear exactly what was done by others and what I have contributed myself;\\
7. Either none of this work has been published before submission, or parts of this work have been published by :\\
\\
Stefan Collier\\
April 2016
}
\tableofcontents
\listoffigures
\listoftables

\mainmatter
%% ----------------------------------------------------------------
%\include{Introduction}
%\include{Conclusions}
\include{chapters/1Project/main}
\include{chapters/2Lit/main}
\include{chapters/3Design/HighLevel}
\include{chapters/3Design/InDepth}
\include{chapters/4Impl/main}

\include{chapters/5Experiments/1/main}
\include{chapters/5Experiments/2/main}
\include{chapters/5Experiments/3/main}
\include{chapters/5Experiments/4/main}

\include{chapters/6Conclusion/main}

\appendix
\include{appendix/AppendixB}
\include{appendix/D/main}
\include{appendix/AppendixC}

\backmatter
\bibliographystyle{ecs}
\bibliography{ECS}
\end{document}
%% ----------------------------------------------------------------

 %% ----------------------------------------------------------------
%% Progress.tex
%% ---------------------------------------------------------------- 
\documentclass{ecsprogress}    % Use the progress Style
\graphicspath{{../figs/}}   % Location of your graphics files
    \usepackage{natbib}            % Use Natbib style for the refs.
\hypersetup{colorlinks=true}   % Set to false for black/white printing
\input{Definitions}            % Include your abbreviations



\usepackage{enumitem}% http://ctan.org/pkg/enumitem
\usepackage{multirow}
\usepackage{float}
\usepackage{amsmath}
\usepackage{multicol}
\usepackage{amssymb}
\usepackage[normalem]{ulem}
\useunder{\uline}{\ul}{}
\usepackage{wrapfig}


\usepackage[table,xcdraw]{xcolor}


%% ----------------------------------------------------------------
\begin{document}
\frontmatter
\title      {Heterogeneous Agent-based Model for Supermarket Competition}
\authors    {\texorpdfstring
             {\href{mailto:sc22g13@ecs.soton.ac.uk}{Stefan J. Collier}}
             {Stefan J. Collier}
            }
\addresses  {\groupname\\\deptname\\\univname}
\date       {\today}
\subject    {}
\keywords   {}
\supervisor {Dr. Maria Polukarov}
\examiner   {Professor Sheng Chen}

\maketitle
\begin{abstract}
This project aim was to model and analyse the effects of competitive pricing behaviors of grocery retailers on the British market. 

This was achieved by creating a multi-agent model, containing retailer and consumer agents. The heterogeneous crowd of retailers employs either a uniform pricing strategy or a ‘local price flexing’ strategy. The actions of these retailers are chosen by predicting the profit of each action, using a perceptron. Following on from the consideration of different economic models, a discrete model was developed so that software agents have a discrete environment to operate within. Within the model, it has been observed how supermarkets with differing behaviors affect a heterogeneous crowd of consumer agents. The model was implemented in Java with Python used to evaluate the results. 

The simulation displays good acceptance with real grocery market behavior, i.e. captures the performance of British retailers thus can be used to determine the impact of changes in their behavior on their competitors and consumers.Furthermore it can be used to provide insight into sustainability of volatile pricing strategies, providing a useful insight in volatility of British supermarket retail industry. 
\end{abstract}
\acknowledgements{
I would like to express my sincere gratitude to Dr Maria Polukarov for her guidance and support which provided me the freedom to take this research in the direction of my interest.\\
\\
I would also like to thank my family and friends for their encouragement and support. To those who quietly listened to my software complaints. To those who worked throughout the nights with me. To those who helped me write what I couldn't say. I cannot thank you enough.
}

\declaration{
I, Stefan Collier, declare that this dissertation and the work presented in it are my own and has been generated by me as the result of my own original research.\\
I confirm that:\\
1. This work was done wholly or mainly while in candidature for a degree at this University;\\
2. Where any part of this dissertation has previously been submitted for any other qualification at this University or any other institution, this has been clearly stated;\\
3. Where I have consulted the published work of others, this is always clearly attributed;\\
4. Where I have quoted from the work of others, the source is always given. With the exception of such quotations, this dissertation is entirely my own work;\\
5. I have acknowledged all main sources of help;\\
6. Where the thesis is based on work done by myself jointly with others, I have made clear exactly what was done by others and what I have contributed myself;\\
7. Either none of this work has been published before submission, or parts of this work have been published by :\\
\\
Stefan Collier\\
April 2016
}
\tableofcontents
\listoffigures
\listoftables

\mainmatter
%% ----------------------------------------------------------------
%\include{Introduction}
%\include{Conclusions}
\include{chapters/1Project/main}
\include{chapters/2Lit/main}
\include{chapters/3Design/HighLevel}
\include{chapters/3Design/InDepth}
\include{chapters/4Impl/main}

\include{chapters/5Experiments/1/main}
\include{chapters/5Experiments/2/main}
\include{chapters/5Experiments/3/main}
\include{chapters/5Experiments/4/main}

\include{chapters/6Conclusion/main}

\appendix
\include{appendix/AppendixB}
\include{appendix/D/main}
\include{appendix/AppendixC}

\backmatter
\bibliographystyle{ecs}
\bibliography{ECS}
\end{document}
%% ----------------------------------------------------------------

 %% ----------------------------------------------------------------
%% Progress.tex
%% ---------------------------------------------------------------- 
\documentclass{ecsprogress}    % Use the progress Style
\graphicspath{{../figs/}}   % Location of your graphics files
    \usepackage{natbib}            % Use Natbib style for the refs.
\hypersetup{colorlinks=true}   % Set to false for black/white printing
\input{Definitions}            % Include your abbreviations



\usepackage{enumitem}% http://ctan.org/pkg/enumitem
\usepackage{multirow}
\usepackage{float}
\usepackage{amsmath}
\usepackage{multicol}
\usepackage{amssymb}
\usepackage[normalem]{ulem}
\useunder{\uline}{\ul}{}
\usepackage{wrapfig}


\usepackage[table,xcdraw]{xcolor}


%% ----------------------------------------------------------------
\begin{document}
\frontmatter
\title      {Heterogeneous Agent-based Model for Supermarket Competition}
\authors    {\texorpdfstring
             {\href{mailto:sc22g13@ecs.soton.ac.uk}{Stefan J. Collier}}
             {Stefan J. Collier}
            }
\addresses  {\groupname\\\deptname\\\univname}
\date       {\today}
\subject    {}
\keywords   {}
\supervisor {Dr. Maria Polukarov}
\examiner   {Professor Sheng Chen}

\maketitle
\begin{abstract}
This project aim was to model and analyse the effects of competitive pricing behaviors of grocery retailers on the British market. 

This was achieved by creating a multi-agent model, containing retailer and consumer agents. The heterogeneous crowd of retailers employs either a uniform pricing strategy or a ‘local price flexing’ strategy. The actions of these retailers are chosen by predicting the profit of each action, using a perceptron. Following on from the consideration of different economic models, a discrete model was developed so that software agents have a discrete environment to operate within. Within the model, it has been observed how supermarkets with differing behaviors affect a heterogeneous crowd of consumer agents. The model was implemented in Java with Python used to evaluate the results. 

The simulation displays good acceptance with real grocery market behavior, i.e. captures the performance of British retailers thus can be used to determine the impact of changes in their behavior on their competitors and consumers.Furthermore it can be used to provide insight into sustainability of volatile pricing strategies, providing a useful insight in volatility of British supermarket retail industry. 
\end{abstract}
\acknowledgements{
I would like to express my sincere gratitude to Dr Maria Polukarov for her guidance and support which provided me the freedom to take this research in the direction of my interest.\\
\\
I would also like to thank my family and friends for their encouragement and support. To those who quietly listened to my software complaints. To those who worked throughout the nights with me. To those who helped me write what I couldn't say. I cannot thank you enough.
}

\declaration{
I, Stefan Collier, declare that this dissertation and the work presented in it are my own and has been generated by me as the result of my own original research.\\
I confirm that:\\
1. This work was done wholly or mainly while in candidature for a degree at this University;\\
2. Where any part of this dissertation has previously been submitted for any other qualification at this University or any other institution, this has been clearly stated;\\
3. Where I have consulted the published work of others, this is always clearly attributed;\\
4. Where I have quoted from the work of others, the source is always given. With the exception of such quotations, this dissertation is entirely my own work;\\
5. I have acknowledged all main sources of help;\\
6. Where the thesis is based on work done by myself jointly with others, I have made clear exactly what was done by others and what I have contributed myself;\\
7. Either none of this work has been published before submission, or parts of this work have been published by :\\
\\
Stefan Collier\\
April 2016
}
\tableofcontents
\listoffigures
\listoftables

\mainmatter
%% ----------------------------------------------------------------
%\include{Introduction}
%\include{Conclusions}
\include{chapters/1Project/main}
\include{chapters/2Lit/main}
\include{chapters/3Design/HighLevel}
\include{chapters/3Design/InDepth}
\include{chapters/4Impl/main}

\include{chapters/5Experiments/1/main}
\include{chapters/5Experiments/2/main}
\include{chapters/5Experiments/3/main}
\include{chapters/5Experiments/4/main}

\include{chapters/6Conclusion/main}

\appendix
\include{appendix/AppendixB}
\include{appendix/D/main}
\include{appendix/AppendixC}

\backmatter
\bibliographystyle{ecs}
\bibliography{ECS}
\end{document}
%% ----------------------------------------------------------------

 %% ----------------------------------------------------------------
%% Progress.tex
%% ---------------------------------------------------------------- 
\documentclass{ecsprogress}    % Use the progress Style
\graphicspath{{../figs/}}   % Location of your graphics files
    \usepackage{natbib}            % Use Natbib style for the refs.
\hypersetup{colorlinks=true}   % Set to false for black/white printing
\input{Definitions}            % Include your abbreviations



\usepackage{enumitem}% http://ctan.org/pkg/enumitem
\usepackage{multirow}
\usepackage{float}
\usepackage{amsmath}
\usepackage{multicol}
\usepackage{amssymb}
\usepackage[normalem]{ulem}
\useunder{\uline}{\ul}{}
\usepackage{wrapfig}


\usepackage[table,xcdraw]{xcolor}


%% ----------------------------------------------------------------
\begin{document}
\frontmatter
\title      {Heterogeneous Agent-based Model for Supermarket Competition}
\authors    {\texorpdfstring
             {\href{mailto:sc22g13@ecs.soton.ac.uk}{Stefan J. Collier}}
             {Stefan J. Collier}
            }
\addresses  {\groupname\\\deptname\\\univname}
\date       {\today}
\subject    {}
\keywords   {}
\supervisor {Dr. Maria Polukarov}
\examiner   {Professor Sheng Chen}

\maketitle
\begin{abstract}
This project aim was to model and analyse the effects of competitive pricing behaviors of grocery retailers on the British market. 

This was achieved by creating a multi-agent model, containing retailer and consumer agents. The heterogeneous crowd of retailers employs either a uniform pricing strategy or a ‘local price flexing’ strategy. The actions of these retailers are chosen by predicting the profit of each action, using a perceptron. Following on from the consideration of different economic models, a discrete model was developed so that software agents have a discrete environment to operate within. Within the model, it has been observed how supermarkets with differing behaviors affect a heterogeneous crowd of consumer agents. The model was implemented in Java with Python used to evaluate the results. 

The simulation displays good acceptance with real grocery market behavior, i.e. captures the performance of British retailers thus can be used to determine the impact of changes in their behavior on their competitors and consumers.Furthermore it can be used to provide insight into sustainability of volatile pricing strategies, providing a useful insight in volatility of British supermarket retail industry. 
\end{abstract}
\acknowledgements{
I would like to express my sincere gratitude to Dr Maria Polukarov for her guidance and support which provided me the freedom to take this research in the direction of my interest.\\
\\
I would also like to thank my family and friends for their encouragement and support. To those who quietly listened to my software complaints. To those who worked throughout the nights with me. To those who helped me write what I couldn't say. I cannot thank you enough.
}

\declaration{
I, Stefan Collier, declare that this dissertation and the work presented in it are my own and has been generated by me as the result of my own original research.\\
I confirm that:\\
1. This work was done wholly or mainly while in candidature for a degree at this University;\\
2. Where any part of this dissertation has previously been submitted for any other qualification at this University or any other institution, this has been clearly stated;\\
3. Where I have consulted the published work of others, this is always clearly attributed;\\
4. Where I have quoted from the work of others, the source is always given. With the exception of such quotations, this dissertation is entirely my own work;\\
5. I have acknowledged all main sources of help;\\
6. Where the thesis is based on work done by myself jointly with others, I have made clear exactly what was done by others and what I have contributed myself;\\
7. Either none of this work has been published before submission, or parts of this work have been published by :\\
\\
Stefan Collier\\
April 2016
}
\tableofcontents
\listoffigures
\listoftables

\mainmatter
%% ----------------------------------------------------------------
%\include{Introduction}
%\include{Conclusions}
\include{chapters/1Project/main}
\include{chapters/2Lit/main}
\include{chapters/3Design/HighLevel}
\include{chapters/3Design/InDepth}
\include{chapters/4Impl/main}

\include{chapters/5Experiments/1/main}
\include{chapters/5Experiments/2/main}
\include{chapters/5Experiments/3/main}
\include{chapters/5Experiments/4/main}

\include{chapters/6Conclusion/main}

\appendix
\include{appendix/AppendixB}
\include{appendix/D/main}
\include{appendix/AppendixC}

\backmatter
\bibliographystyle{ecs}
\bibliography{ECS}
\end{document}
%% ----------------------------------------------------------------


 %% ----------------------------------------------------------------
%% Progress.tex
%% ---------------------------------------------------------------- 
\documentclass{ecsprogress}    % Use the progress Style
\graphicspath{{../figs/}}   % Location of your graphics files
    \usepackage{natbib}            % Use Natbib style for the refs.
\hypersetup{colorlinks=true}   % Set to false for black/white printing
\input{Definitions}            % Include your abbreviations



\usepackage{enumitem}% http://ctan.org/pkg/enumitem
\usepackage{multirow}
\usepackage{float}
\usepackage{amsmath}
\usepackage{multicol}
\usepackage{amssymb}
\usepackage[normalem]{ulem}
\useunder{\uline}{\ul}{}
\usepackage{wrapfig}


\usepackage[table,xcdraw]{xcolor}


%% ----------------------------------------------------------------
\begin{document}
\frontmatter
\title      {Heterogeneous Agent-based Model for Supermarket Competition}
\authors    {\texorpdfstring
             {\href{mailto:sc22g13@ecs.soton.ac.uk}{Stefan J. Collier}}
             {Stefan J. Collier}
            }
\addresses  {\groupname\\\deptname\\\univname}
\date       {\today}
\subject    {}
\keywords   {}
\supervisor {Dr. Maria Polukarov}
\examiner   {Professor Sheng Chen}

\maketitle
\begin{abstract}
This project aim was to model and analyse the effects of competitive pricing behaviors of grocery retailers on the British market. 

This was achieved by creating a multi-agent model, containing retailer and consumer agents. The heterogeneous crowd of retailers employs either a uniform pricing strategy or a ‘local price flexing’ strategy. The actions of these retailers are chosen by predicting the profit of each action, using a perceptron. Following on from the consideration of different economic models, a discrete model was developed so that software agents have a discrete environment to operate within. Within the model, it has been observed how supermarkets with differing behaviors affect a heterogeneous crowd of consumer agents. The model was implemented in Java with Python used to evaluate the results. 

The simulation displays good acceptance with real grocery market behavior, i.e. captures the performance of British retailers thus can be used to determine the impact of changes in their behavior on their competitors and consumers.Furthermore it can be used to provide insight into sustainability of volatile pricing strategies, providing a useful insight in volatility of British supermarket retail industry. 
\end{abstract}
\acknowledgements{
I would like to express my sincere gratitude to Dr Maria Polukarov for her guidance and support which provided me the freedom to take this research in the direction of my interest.\\
\\
I would also like to thank my family and friends for their encouragement and support. To those who quietly listened to my software complaints. To those who worked throughout the nights with me. To those who helped me write what I couldn't say. I cannot thank you enough.
}

\declaration{
I, Stefan Collier, declare that this dissertation and the work presented in it are my own and has been generated by me as the result of my own original research.\\
I confirm that:\\
1. This work was done wholly or mainly while in candidature for a degree at this University;\\
2. Where any part of this dissertation has previously been submitted for any other qualification at this University or any other institution, this has been clearly stated;\\
3. Where I have consulted the published work of others, this is always clearly attributed;\\
4. Where I have quoted from the work of others, the source is always given. With the exception of such quotations, this dissertation is entirely my own work;\\
5. I have acknowledged all main sources of help;\\
6. Where the thesis is based on work done by myself jointly with others, I have made clear exactly what was done by others and what I have contributed myself;\\
7. Either none of this work has been published before submission, or parts of this work have been published by :\\
\\
Stefan Collier\\
April 2016
}
\tableofcontents
\listoffigures
\listoftables

\mainmatter
%% ----------------------------------------------------------------
%\include{Introduction}
%\include{Conclusions}
\include{chapters/1Project/main}
\include{chapters/2Lit/main}
\include{chapters/3Design/HighLevel}
\include{chapters/3Design/InDepth}
\include{chapters/4Impl/main}

\include{chapters/5Experiments/1/main}
\include{chapters/5Experiments/2/main}
\include{chapters/5Experiments/3/main}
\include{chapters/5Experiments/4/main}

\include{chapters/6Conclusion/main}

\appendix
\include{appendix/AppendixB}
\include{appendix/D/main}
\include{appendix/AppendixC}

\backmatter
\bibliographystyle{ecs}
\bibliography{ECS}
\end{document}
%% ----------------------------------------------------------------


\appendix
\include{appendix/AppendixB}
 %% ----------------------------------------------------------------
%% Progress.tex
%% ---------------------------------------------------------------- 
\documentclass{ecsprogress}    % Use the progress Style
\graphicspath{{../figs/}}   % Location of your graphics files
    \usepackage{natbib}            % Use Natbib style for the refs.
\hypersetup{colorlinks=true}   % Set to false for black/white printing
\input{Definitions}            % Include your abbreviations



\usepackage{enumitem}% http://ctan.org/pkg/enumitem
\usepackage{multirow}
\usepackage{float}
\usepackage{amsmath}
\usepackage{multicol}
\usepackage{amssymb}
\usepackage[normalem]{ulem}
\useunder{\uline}{\ul}{}
\usepackage{wrapfig}


\usepackage[table,xcdraw]{xcolor}


%% ----------------------------------------------------------------
\begin{document}
\frontmatter
\title      {Heterogeneous Agent-based Model for Supermarket Competition}
\authors    {\texorpdfstring
             {\href{mailto:sc22g13@ecs.soton.ac.uk}{Stefan J. Collier}}
             {Stefan J. Collier}
            }
\addresses  {\groupname\\\deptname\\\univname}
\date       {\today}
\subject    {}
\keywords   {}
\supervisor {Dr. Maria Polukarov}
\examiner   {Professor Sheng Chen}

\maketitle
\begin{abstract}
This project aim was to model and analyse the effects of competitive pricing behaviors of grocery retailers on the British market. 

This was achieved by creating a multi-agent model, containing retailer and consumer agents. The heterogeneous crowd of retailers employs either a uniform pricing strategy or a ‘local price flexing’ strategy. The actions of these retailers are chosen by predicting the profit of each action, using a perceptron. Following on from the consideration of different economic models, a discrete model was developed so that software agents have a discrete environment to operate within. Within the model, it has been observed how supermarkets with differing behaviors affect a heterogeneous crowd of consumer agents. The model was implemented in Java with Python used to evaluate the results. 

The simulation displays good acceptance with real grocery market behavior, i.e. captures the performance of British retailers thus can be used to determine the impact of changes in their behavior on their competitors and consumers.Furthermore it can be used to provide insight into sustainability of volatile pricing strategies, providing a useful insight in volatility of British supermarket retail industry. 
\end{abstract}
\acknowledgements{
I would like to express my sincere gratitude to Dr Maria Polukarov for her guidance and support which provided me the freedom to take this research in the direction of my interest.\\
\\
I would also like to thank my family and friends for their encouragement and support. To those who quietly listened to my software complaints. To those who worked throughout the nights with me. To those who helped me write what I couldn't say. I cannot thank you enough.
}

\declaration{
I, Stefan Collier, declare that this dissertation and the work presented in it are my own and has been generated by me as the result of my own original research.\\
I confirm that:\\
1. This work was done wholly or mainly while in candidature for a degree at this University;\\
2. Where any part of this dissertation has previously been submitted for any other qualification at this University or any other institution, this has been clearly stated;\\
3. Where I have consulted the published work of others, this is always clearly attributed;\\
4. Where I have quoted from the work of others, the source is always given. With the exception of such quotations, this dissertation is entirely my own work;\\
5. I have acknowledged all main sources of help;\\
6. Where the thesis is based on work done by myself jointly with others, I have made clear exactly what was done by others and what I have contributed myself;\\
7. Either none of this work has been published before submission, or parts of this work have been published by :\\
\\
Stefan Collier\\
April 2016
}
\tableofcontents
\listoffigures
\listoftables

\mainmatter
%% ----------------------------------------------------------------
%\include{Introduction}
%\include{Conclusions}
\include{chapters/1Project/main}
\include{chapters/2Lit/main}
\include{chapters/3Design/HighLevel}
\include{chapters/3Design/InDepth}
\include{chapters/4Impl/main}

\include{chapters/5Experiments/1/main}
\include{chapters/5Experiments/2/main}
\include{chapters/5Experiments/3/main}
\include{chapters/5Experiments/4/main}

\include{chapters/6Conclusion/main}

\appendix
\include{appendix/AppendixB}
\include{appendix/D/main}
\include{appendix/AppendixC}

\backmatter
\bibliographystyle{ecs}
\bibliography{ECS}
\end{document}
%% ----------------------------------------------------------------

\include{appendix/AppendixC}

\backmatter
\bibliographystyle{ecs}
\bibliography{ECS}
\end{document}
%% ----------------------------------------------------------------


\appendix
\include{appendix/AppendixB}
 %% ----------------------------------------------------------------
%% Progress.tex
%% ---------------------------------------------------------------- 
\documentclass{ecsprogress}    % Use the progress Style
\graphicspath{{../figs/}}   % Location of your graphics files
    \usepackage{natbib}            % Use Natbib style for the refs.
\hypersetup{colorlinks=true}   % Set to false for black/white printing
\input{Definitions}            % Include your abbreviations



\usepackage{enumitem}% http://ctan.org/pkg/enumitem
\usepackage{multirow}
\usepackage{float}
\usepackage{amsmath}
\usepackage{multicol}
\usepackage{amssymb}
\usepackage[normalem]{ulem}
\useunder{\uline}{\ul}{}
\usepackage{wrapfig}


\usepackage[table,xcdraw]{xcolor}


%% ----------------------------------------------------------------
\begin{document}
\frontmatter
\title      {Heterogeneous Agent-based Model for Supermarket Competition}
\authors    {\texorpdfstring
             {\href{mailto:sc22g13@ecs.soton.ac.uk}{Stefan J. Collier}}
             {Stefan J. Collier}
            }
\addresses  {\groupname\\\deptname\\\univname}
\date       {\today}
\subject    {}
\keywords   {}
\supervisor {Dr. Maria Polukarov}
\examiner   {Professor Sheng Chen}

\maketitle
\begin{abstract}
This project aim was to model and analyse the effects of competitive pricing behaviors of grocery retailers on the British market. 

This was achieved by creating a multi-agent model, containing retailer and consumer agents. The heterogeneous crowd of retailers employs either a uniform pricing strategy or a ‘local price flexing’ strategy. The actions of these retailers are chosen by predicting the profit of each action, using a perceptron. Following on from the consideration of different economic models, a discrete model was developed so that software agents have a discrete environment to operate within. Within the model, it has been observed how supermarkets with differing behaviors affect a heterogeneous crowd of consumer agents. The model was implemented in Java with Python used to evaluate the results. 

The simulation displays good acceptance with real grocery market behavior, i.e. captures the performance of British retailers thus can be used to determine the impact of changes in their behavior on their competitors and consumers.Furthermore it can be used to provide insight into sustainability of volatile pricing strategies, providing a useful insight in volatility of British supermarket retail industry. 
\end{abstract}
\acknowledgements{
I would like to express my sincere gratitude to Dr Maria Polukarov for her guidance and support which provided me the freedom to take this research in the direction of my interest.\\
\\
I would also like to thank my family and friends for their encouragement and support. To those who quietly listened to my software complaints. To those who worked throughout the nights with me. To those who helped me write what I couldn't say. I cannot thank you enough.
}

\declaration{
I, Stefan Collier, declare that this dissertation and the work presented in it are my own and has been generated by me as the result of my own original research.\\
I confirm that:\\
1. This work was done wholly or mainly while in candidature for a degree at this University;\\
2. Where any part of this dissertation has previously been submitted for any other qualification at this University or any other institution, this has been clearly stated;\\
3. Where I have consulted the published work of others, this is always clearly attributed;\\
4. Where I have quoted from the work of others, the source is always given. With the exception of such quotations, this dissertation is entirely my own work;\\
5. I have acknowledged all main sources of help;\\
6. Where the thesis is based on work done by myself jointly with others, I have made clear exactly what was done by others and what I have contributed myself;\\
7. Either none of this work has been published before submission, or parts of this work have been published by :\\
\\
Stefan Collier\\
April 2016
}
\tableofcontents
\listoffigures
\listoftables

\mainmatter
%% ----------------------------------------------------------------
%\include{Introduction}
%\include{Conclusions}
 %% ----------------------------------------------------------------
%% Progress.tex
%% ---------------------------------------------------------------- 
\documentclass{ecsprogress}    % Use the progress Style
\graphicspath{{../figs/}}   % Location of your graphics files
    \usepackage{natbib}            % Use Natbib style for the refs.
\hypersetup{colorlinks=true}   % Set to false for black/white printing
\input{Definitions}            % Include your abbreviations



\usepackage{enumitem}% http://ctan.org/pkg/enumitem
\usepackage{multirow}
\usepackage{float}
\usepackage{amsmath}
\usepackage{multicol}
\usepackage{amssymb}
\usepackage[normalem]{ulem}
\useunder{\uline}{\ul}{}
\usepackage{wrapfig}


\usepackage[table,xcdraw]{xcolor}


%% ----------------------------------------------------------------
\begin{document}
\frontmatter
\title      {Heterogeneous Agent-based Model for Supermarket Competition}
\authors    {\texorpdfstring
             {\href{mailto:sc22g13@ecs.soton.ac.uk}{Stefan J. Collier}}
             {Stefan J. Collier}
            }
\addresses  {\groupname\\\deptname\\\univname}
\date       {\today}
\subject    {}
\keywords   {}
\supervisor {Dr. Maria Polukarov}
\examiner   {Professor Sheng Chen}

\maketitle
\begin{abstract}
This project aim was to model and analyse the effects of competitive pricing behaviors of grocery retailers on the British market. 

This was achieved by creating a multi-agent model, containing retailer and consumer agents. The heterogeneous crowd of retailers employs either a uniform pricing strategy or a ‘local price flexing’ strategy. The actions of these retailers are chosen by predicting the profit of each action, using a perceptron. Following on from the consideration of different economic models, a discrete model was developed so that software agents have a discrete environment to operate within. Within the model, it has been observed how supermarkets with differing behaviors affect a heterogeneous crowd of consumer agents. The model was implemented in Java with Python used to evaluate the results. 

The simulation displays good acceptance with real grocery market behavior, i.e. captures the performance of British retailers thus can be used to determine the impact of changes in their behavior on their competitors and consumers.Furthermore it can be used to provide insight into sustainability of volatile pricing strategies, providing a useful insight in volatility of British supermarket retail industry. 
\end{abstract}
\acknowledgements{
I would like to express my sincere gratitude to Dr Maria Polukarov for her guidance and support which provided me the freedom to take this research in the direction of my interest.\\
\\
I would also like to thank my family and friends for their encouragement and support. To those who quietly listened to my software complaints. To those who worked throughout the nights with me. To those who helped me write what I couldn't say. I cannot thank you enough.
}

\declaration{
I, Stefan Collier, declare that this dissertation and the work presented in it are my own and has been generated by me as the result of my own original research.\\
I confirm that:\\
1. This work was done wholly or mainly while in candidature for a degree at this University;\\
2. Where any part of this dissertation has previously been submitted for any other qualification at this University or any other institution, this has been clearly stated;\\
3. Where I have consulted the published work of others, this is always clearly attributed;\\
4. Where I have quoted from the work of others, the source is always given. With the exception of such quotations, this dissertation is entirely my own work;\\
5. I have acknowledged all main sources of help;\\
6. Where the thesis is based on work done by myself jointly with others, I have made clear exactly what was done by others and what I have contributed myself;\\
7. Either none of this work has been published before submission, or parts of this work have been published by :\\
\\
Stefan Collier\\
April 2016
}
\tableofcontents
\listoffigures
\listoftables

\mainmatter
%% ----------------------------------------------------------------
%\include{Introduction}
%\include{Conclusions}
\include{chapters/1Project/main}
\include{chapters/2Lit/main}
\include{chapters/3Design/HighLevel}
\include{chapters/3Design/InDepth}
\include{chapters/4Impl/main}

\include{chapters/5Experiments/1/main}
\include{chapters/5Experiments/2/main}
\include{chapters/5Experiments/3/main}
\include{chapters/5Experiments/4/main}

\include{chapters/6Conclusion/main}

\appendix
\include{appendix/AppendixB}
\include{appendix/D/main}
\include{appendix/AppendixC}

\backmatter
\bibliographystyle{ecs}
\bibliography{ECS}
\end{document}
%% ----------------------------------------------------------------

 %% ----------------------------------------------------------------
%% Progress.tex
%% ---------------------------------------------------------------- 
\documentclass{ecsprogress}    % Use the progress Style
\graphicspath{{../figs/}}   % Location of your graphics files
    \usepackage{natbib}            % Use Natbib style for the refs.
\hypersetup{colorlinks=true}   % Set to false for black/white printing
\input{Definitions}            % Include your abbreviations



\usepackage{enumitem}% http://ctan.org/pkg/enumitem
\usepackage{multirow}
\usepackage{float}
\usepackage{amsmath}
\usepackage{multicol}
\usepackage{amssymb}
\usepackage[normalem]{ulem}
\useunder{\uline}{\ul}{}
\usepackage{wrapfig}


\usepackage[table,xcdraw]{xcolor}


%% ----------------------------------------------------------------
\begin{document}
\frontmatter
\title      {Heterogeneous Agent-based Model for Supermarket Competition}
\authors    {\texorpdfstring
             {\href{mailto:sc22g13@ecs.soton.ac.uk}{Stefan J. Collier}}
             {Stefan J. Collier}
            }
\addresses  {\groupname\\\deptname\\\univname}
\date       {\today}
\subject    {}
\keywords   {}
\supervisor {Dr. Maria Polukarov}
\examiner   {Professor Sheng Chen}

\maketitle
\begin{abstract}
This project aim was to model and analyse the effects of competitive pricing behaviors of grocery retailers on the British market. 

This was achieved by creating a multi-agent model, containing retailer and consumer agents. The heterogeneous crowd of retailers employs either a uniform pricing strategy or a ‘local price flexing’ strategy. The actions of these retailers are chosen by predicting the profit of each action, using a perceptron. Following on from the consideration of different economic models, a discrete model was developed so that software agents have a discrete environment to operate within. Within the model, it has been observed how supermarkets with differing behaviors affect a heterogeneous crowd of consumer agents. The model was implemented in Java with Python used to evaluate the results. 

The simulation displays good acceptance with real grocery market behavior, i.e. captures the performance of British retailers thus can be used to determine the impact of changes in their behavior on their competitors and consumers.Furthermore it can be used to provide insight into sustainability of volatile pricing strategies, providing a useful insight in volatility of British supermarket retail industry. 
\end{abstract}
\acknowledgements{
I would like to express my sincere gratitude to Dr Maria Polukarov for her guidance and support which provided me the freedom to take this research in the direction of my interest.\\
\\
I would also like to thank my family and friends for their encouragement and support. To those who quietly listened to my software complaints. To those who worked throughout the nights with me. To those who helped me write what I couldn't say. I cannot thank you enough.
}

\declaration{
I, Stefan Collier, declare that this dissertation and the work presented in it are my own and has been generated by me as the result of my own original research.\\
I confirm that:\\
1. This work was done wholly or mainly while in candidature for a degree at this University;\\
2. Where any part of this dissertation has previously been submitted for any other qualification at this University or any other institution, this has been clearly stated;\\
3. Where I have consulted the published work of others, this is always clearly attributed;\\
4. Where I have quoted from the work of others, the source is always given. With the exception of such quotations, this dissertation is entirely my own work;\\
5. I have acknowledged all main sources of help;\\
6. Where the thesis is based on work done by myself jointly with others, I have made clear exactly what was done by others and what I have contributed myself;\\
7. Either none of this work has been published before submission, or parts of this work have been published by :\\
\\
Stefan Collier\\
April 2016
}
\tableofcontents
\listoffigures
\listoftables

\mainmatter
%% ----------------------------------------------------------------
%\include{Introduction}
%\include{Conclusions}
\include{chapters/1Project/main}
\include{chapters/2Lit/main}
\include{chapters/3Design/HighLevel}
\include{chapters/3Design/InDepth}
\include{chapters/4Impl/main}

\include{chapters/5Experiments/1/main}
\include{chapters/5Experiments/2/main}
\include{chapters/5Experiments/3/main}
\include{chapters/5Experiments/4/main}

\include{chapters/6Conclusion/main}

\appendix
\include{appendix/AppendixB}
\include{appendix/D/main}
\include{appendix/AppendixC}

\backmatter
\bibliographystyle{ecs}
\bibliography{ECS}
\end{document}
%% ----------------------------------------------------------------

\include{chapters/3Design/HighLevel}
\include{chapters/3Design/InDepth}
 %% ----------------------------------------------------------------
%% Progress.tex
%% ---------------------------------------------------------------- 
\documentclass{ecsprogress}    % Use the progress Style
\graphicspath{{../figs/}}   % Location of your graphics files
    \usepackage{natbib}            % Use Natbib style for the refs.
\hypersetup{colorlinks=true}   % Set to false for black/white printing
\input{Definitions}            % Include your abbreviations



\usepackage{enumitem}% http://ctan.org/pkg/enumitem
\usepackage{multirow}
\usepackage{float}
\usepackage{amsmath}
\usepackage{multicol}
\usepackage{amssymb}
\usepackage[normalem]{ulem}
\useunder{\uline}{\ul}{}
\usepackage{wrapfig}


\usepackage[table,xcdraw]{xcolor}


%% ----------------------------------------------------------------
\begin{document}
\frontmatter
\title      {Heterogeneous Agent-based Model for Supermarket Competition}
\authors    {\texorpdfstring
             {\href{mailto:sc22g13@ecs.soton.ac.uk}{Stefan J. Collier}}
             {Stefan J. Collier}
            }
\addresses  {\groupname\\\deptname\\\univname}
\date       {\today}
\subject    {}
\keywords   {}
\supervisor {Dr. Maria Polukarov}
\examiner   {Professor Sheng Chen}

\maketitle
\begin{abstract}
This project aim was to model and analyse the effects of competitive pricing behaviors of grocery retailers on the British market. 

This was achieved by creating a multi-agent model, containing retailer and consumer agents. The heterogeneous crowd of retailers employs either a uniform pricing strategy or a ‘local price flexing’ strategy. The actions of these retailers are chosen by predicting the profit of each action, using a perceptron. Following on from the consideration of different economic models, a discrete model was developed so that software agents have a discrete environment to operate within. Within the model, it has been observed how supermarkets with differing behaviors affect a heterogeneous crowd of consumer agents. The model was implemented in Java with Python used to evaluate the results. 

The simulation displays good acceptance with real grocery market behavior, i.e. captures the performance of British retailers thus can be used to determine the impact of changes in their behavior on their competitors and consumers.Furthermore it can be used to provide insight into sustainability of volatile pricing strategies, providing a useful insight in volatility of British supermarket retail industry. 
\end{abstract}
\acknowledgements{
I would like to express my sincere gratitude to Dr Maria Polukarov for her guidance and support which provided me the freedom to take this research in the direction of my interest.\\
\\
I would also like to thank my family and friends for their encouragement and support. To those who quietly listened to my software complaints. To those who worked throughout the nights with me. To those who helped me write what I couldn't say. I cannot thank you enough.
}

\declaration{
I, Stefan Collier, declare that this dissertation and the work presented in it are my own and has been generated by me as the result of my own original research.\\
I confirm that:\\
1. This work was done wholly or mainly while in candidature for a degree at this University;\\
2. Where any part of this dissertation has previously been submitted for any other qualification at this University or any other institution, this has been clearly stated;\\
3. Where I have consulted the published work of others, this is always clearly attributed;\\
4. Where I have quoted from the work of others, the source is always given. With the exception of such quotations, this dissertation is entirely my own work;\\
5. I have acknowledged all main sources of help;\\
6. Where the thesis is based on work done by myself jointly with others, I have made clear exactly what was done by others and what I have contributed myself;\\
7. Either none of this work has been published before submission, or parts of this work have been published by :\\
\\
Stefan Collier\\
April 2016
}
\tableofcontents
\listoffigures
\listoftables

\mainmatter
%% ----------------------------------------------------------------
%\include{Introduction}
%\include{Conclusions}
\include{chapters/1Project/main}
\include{chapters/2Lit/main}
\include{chapters/3Design/HighLevel}
\include{chapters/3Design/InDepth}
\include{chapters/4Impl/main}

\include{chapters/5Experiments/1/main}
\include{chapters/5Experiments/2/main}
\include{chapters/5Experiments/3/main}
\include{chapters/5Experiments/4/main}

\include{chapters/6Conclusion/main}

\appendix
\include{appendix/AppendixB}
\include{appendix/D/main}
\include{appendix/AppendixC}

\backmatter
\bibliographystyle{ecs}
\bibliography{ECS}
\end{document}
%% ----------------------------------------------------------------


 %% ----------------------------------------------------------------
%% Progress.tex
%% ---------------------------------------------------------------- 
\documentclass{ecsprogress}    % Use the progress Style
\graphicspath{{../figs/}}   % Location of your graphics files
    \usepackage{natbib}            % Use Natbib style for the refs.
\hypersetup{colorlinks=true}   % Set to false for black/white printing
\input{Definitions}            % Include your abbreviations



\usepackage{enumitem}% http://ctan.org/pkg/enumitem
\usepackage{multirow}
\usepackage{float}
\usepackage{amsmath}
\usepackage{multicol}
\usepackage{amssymb}
\usepackage[normalem]{ulem}
\useunder{\uline}{\ul}{}
\usepackage{wrapfig}


\usepackage[table,xcdraw]{xcolor}


%% ----------------------------------------------------------------
\begin{document}
\frontmatter
\title      {Heterogeneous Agent-based Model for Supermarket Competition}
\authors    {\texorpdfstring
             {\href{mailto:sc22g13@ecs.soton.ac.uk}{Stefan J. Collier}}
             {Stefan J. Collier}
            }
\addresses  {\groupname\\\deptname\\\univname}
\date       {\today}
\subject    {}
\keywords   {}
\supervisor {Dr. Maria Polukarov}
\examiner   {Professor Sheng Chen}

\maketitle
\begin{abstract}
This project aim was to model and analyse the effects of competitive pricing behaviors of grocery retailers on the British market. 

This was achieved by creating a multi-agent model, containing retailer and consumer agents. The heterogeneous crowd of retailers employs either a uniform pricing strategy or a ‘local price flexing’ strategy. The actions of these retailers are chosen by predicting the profit of each action, using a perceptron. Following on from the consideration of different economic models, a discrete model was developed so that software agents have a discrete environment to operate within. Within the model, it has been observed how supermarkets with differing behaviors affect a heterogeneous crowd of consumer agents. The model was implemented in Java with Python used to evaluate the results. 

The simulation displays good acceptance with real grocery market behavior, i.e. captures the performance of British retailers thus can be used to determine the impact of changes in their behavior on their competitors and consumers.Furthermore it can be used to provide insight into sustainability of volatile pricing strategies, providing a useful insight in volatility of British supermarket retail industry. 
\end{abstract}
\acknowledgements{
I would like to express my sincere gratitude to Dr Maria Polukarov for her guidance and support which provided me the freedom to take this research in the direction of my interest.\\
\\
I would also like to thank my family and friends for their encouragement and support. To those who quietly listened to my software complaints. To those who worked throughout the nights with me. To those who helped me write what I couldn't say. I cannot thank you enough.
}

\declaration{
I, Stefan Collier, declare that this dissertation and the work presented in it are my own and has been generated by me as the result of my own original research.\\
I confirm that:\\
1. This work was done wholly or mainly while in candidature for a degree at this University;\\
2. Where any part of this dissertation has previously been submitted for any other qualification at this University or any other institution, this has been clearly stated;\\
3. Where I have consulted the published work of others, this is always clearly attributed;\\
4. Where I have quoted from the work of others, the source is always given. With the exception of such quotations, this dissertation is entirely my own work;\\
5. I have acknowledged all main sources of help;\\
6. Where the thesis is based on work done by myself jointly with others, I have made clear exactly what was done by others and what I have contributed myself;\\
7. Either none of this work has been published before submission, or parts of this work have been published by :\\
\\
Stefan Collier\\
April 2016
}
\tableofcontents
\listoffigures
\listoftables

\mainmatter
%% ----------------------------------------------------------------
%\include{Introduction}
%\include{Conclusions}
\include{chapters/1Project/main}
\include{chapters/2Lit/main}
\include{chapters/3Design/HighLevel}
\include{chapters/3Design/InDepth}
\include{chapters/4Impl/main}

\include{chapters/5Experiments/1/main}
\include{chapters/5Experiments/2/main}
\include{chapters/5Experiments/3/main}
\include{chapters/5Experiments/4/main}

\include{chapters/6Conclusion/main}

\appendix
\include{appendix/AppendixB}
\include{appendix/D/main}
\include{appendix/AppendixC}

\backmatter
\bibliographystyle{ecs}
\bibliography{ECS}
\end{document}
%% ----------------------------------------------------------------

 %% ----------------------------------------------------------------
%% Progress.tex
%% ---------------------------------------------------------------- 
\documentclass{ecsprogress}    % Use the progress Style
\graphicspath{{../figs/}}   % Location of your graphics files
    \usepackage{natbib}            % Use Natbib style for the refs.
\hypersetup{colorlinks=true}   % Set to false for black/white printing
\input{Definitions}            % Include your abbreviations



\usepackage{enumitem}% http://ctan.org/pkg/enumitem
\usepackage{multirow}
\usepackage{float}
\usepackage{amsmath}
\usepackage{multicol}
\usepackage{amssymb}
\usepackage[normalem]{ulem}
\useunder{\uline}{\ul}{}
\usepackage{wrapfig}


\usepackage[table,xcdraw]{xcolor}


%% ----------------------------------------------------------------
\begin{document}
\frontmatter
\title      {Heterogeneous Agent-based Model for Supermarket Competition}
\authors    {\texorpdfstring
             {\href{mailto:sc22g13@ecs.soton.ac.uk}{Stefan J. Collier}}
             {Stefan J. Collier}
            }
\addresses  {\groupname\\\deptname\\\univname}
\date       {\today}
\subject    {}
\keywords   {}
\supervisor {Dr. Maria Polukarov}
\examiner   {Professor Sheng Chen}

\maketitle
\begin{abstract}
This project aim was to model and analyse the effects of competitive pricing behaviors of grocery retailers on the British market. 

This was achieved by creating a multi-agent model, containing retailer and consumer agents. The heterogeneous crowd of retailers employs either a uniform pricing strategy or a ‘local price flexing’ strategy. The actions of these retailers are chosen by predicting the profit of each action, using a perceptron. Following on from the consideration of different economic models, a discrete model was developed so that software agents have a discrete environment to operate within. Within the model, it has been observed how supermarkets with differing behaviors affect a heterogeneous crowd of consumer agents. The model was implemented in Java with Python used to evaluate the results. 

The simulation displays good acceptance with real grocery market behavior, i.e. captures the performance of British retailers thus can be used to determine the impact of changes in their behavior on their competitors and consumers.Furthermore it can be used to provide insight into sustainability of volatile pricing strategies, providing a useful insight in volatility of British supermarket retail industry. 
\end{abstract}
\acknowledgements{
I would like to express my sincere gratitude to Dr Maria Polukarov for her guidance and support which provided me the freedom to take this research in the direction of my interest.\\
\\
I would also like to thank my family and friends for their encouragement and support. To those who quietly listened to my software complaints. To those who worked throughout the nights with me. To those who helped me write what I couldn't say. I cannot thank you enough.
}

\declaration{
I, Stefan Collier, declare that this dissertation and the work presented in it are my own and has been generated by me as the result of my own original research.\\
I confirm that:\\
1. This work was done wholly or mainly while in candidature for a degree at this University;\\
2. Where any part of this dissertation has previously been submitted for any other qualification at this University or any other institution, this has been clearly stated;\\
3. Where I have consulted the published work of others, this is always clearly attributed;\\
4. Where I have quoted from the work of others, the source is always given. With the exception of such quotations, this dissertation is entirely my own work;\\
5. I have acknowledged all main sources of help;\\
6. Where the thesis is based on work done by myself jointly with others, I have made clear exactly what was done by others and what I have contributed myself;\\
7. Either none of this work has been published before submission, or parts of this work have been published by :\\
\\
Stefan Collier\\
April 2016
}
\tableofcontents
\listoffigures
\listoftables

\mainmatter
%% ----------------------------------------------------------------
%\include{Introduction}
%\include{Conclusions}
\include{chapters/1Project/main}
\include{chapters/2Lit/main}
\include{chapters/3Design/HighLevel}
\include{chapters/3Design/InDepth}
\include{chapters/4Impl/main}

\include{chapters/5Experiments/1/main}
\include{chapters/5Experiments/2/main}
\include{chapters/5Experiments/3/main}
\include{chapters/5Experiments/4/main}

\include{chapters/6Conclusion/main}

\appendix
\include{appendix/AppendixB}
\include{appendix/D/main}
\include{appendix/AppendixC}

\backmatter
\bibliographystyle{ecs}
\bibliography{ECS}
\end{document}
%% ----------------------------------------------------------------

 %% ----------------------------------------------------------------
%% Progress.tex
%% ---------------------------------------------------------------- 
\documentclass{ecsprogress}    % Use the progress Style
\graphicspath{{../figs/}}   % Location of your graphics files
    \usepackage{natbib}            % Use Natbib style for the refs.
\hypersetup{colorlinks=true}   % Set to false for black/white printing
\input{Definitions}            % Include your abbreviations



\usepackage{enumitem}% http://ctan.org/pkg/enumitem
\usepackage{multirow}
\usepackage{float}
\usepackage{amsmath}
\usepackage{multicol}
\usepackage{amssymb}
\usepackage[normalem]{ulem}
\useunder{\uline}{\ul}{}
\usepackage{wrapfig}


\usepackage[table,xcdraw]{xcolor}


%% ----------------------------------------------------------------
\begin{document}
\frontmatter
\title      {Heterogeneous Agent-based Model for Supermarket Competition}
\authors    {\texorpdfstring
             {\href{mailto:sc22g13@ecs.soton.ac.uk}{Stefan J. Collier}}
             {Stefan J. Collier}
            }
\addresses  {\groupname\\\deptname\\\univname}
\date       {\today}
\subject    {}
\keywords   {}
\supervisor {Dr. Maria Polukarov}
\examiner   {Professor Sheng Chen}

\maketitle
\begin{abstract}
This project aim was to model and analyse the effects of competitive pricing behaviors of grocery retailers on the British market. 

This was achieved by creating a multi-agent model, containing retailer and consumer agents. The heterogeneous crowd of retailers employs either a uniform pricing strategy or a ‘local price flexing’ strategy. The actions of these retailers are chosen by predicting the profit of each action, using a perceptron. Following on from the consideration of different economic models, a discrete model was developed so that software agents have a discrete environment to operate within. Within the model, it has been observed how supermarkets with differing behaviors affect a heterogeneous crowd of consumer agents. The model was implemented in Java with Python used to evaluate the results. 

The simulation displays good acceptance with real grocery market behavior, i.e. captures the performance of British retailers thus can be used to determine the impact of changes in their behavior on their competitors and consumers.Furthermore it can be used to provide insight into sustainability of volatile pricing strategies, providing a useful insight in volatility of British supermarket retail industry. 
\end{abstract}
\acknowledgements{
I would like to express my sincere gratitude to Dr Maria Polukarov for her guidance and support which provided me the freedom to take this research in the direction of my interest.\\
\\
I would also like to thank my family and friends for their encouragement and support. To those who quietly listened to my software complaints. To those who worked throughout the nights with me. To those who helped me write what I couldn't say. I cannot thank you enough.
}

\declaration{
I, Stefan Collier, declare that this dissertation and the work presented in it are my own and has been generated by me as the result of my own original research.\\
I confirm that:\\
1. This work was done wholly or mainly while in candidature for a degree at this University;\\
2. Where any part of this dissertation has previously been submitted for any other qualification at this University or any other institution, this has been clearly stated;\\
3. Where I have consulted the published work of others, this is always clearly attributed;\\
4. Where I have quoted from the work of others, the source is always given. With the exception of such quotations, this dissertation is entirely my own work;\\
5. I have acknowledged all main sources of help;\\
6. Where the thesis is based on work done by myself jointly with others, I have made clear exactly what was done by others and what I have contributed myself;\\
7. Either none of this work has been published before submission, or parts of this work have been published by :\\
\\
Stefan Collier\\
April 2016
}
\tableofcontents
\listoffigures
\listoftables

\mainmatter
%% ----------------------------------------------------------------
%\include{Introduction}
%\include{Conclusions}
\include{chapters/1Project/main}
\include{chapters/2Lit/main}
\include{chapters/3Design/HighLevel}
\include{chapters/3Design/InDepth}
\include{chapters/4Impl/main}

\include{chapters/5Experiments/1/main}
\include{chapters/5Experiments/2/main}
\include{chapters/5Experiments/3/main}
\include{chapters/5Experiments/4/main}

\include{chapters/6Conclusion/main}

\appendix
\include{appendix/AppendixB}
\include{appendix/D/main}
\include{appendix/AppendixC}

\backmatter
\bibliographystyle{ecs}
\bibliography{ECS}
\end{document}
%% ----------------------------------------------------------------

 %% ----------------------------------------------------------------
%% Progress.tex
%% ---------------------------------------------------------------- 
\documentclass{ecsprogress}    % Use the progress Style
\graphicspath{{../figs/}}   % Location of your graphics files
    \usepackage{natbib}            % Use Natbib style for the refs.
\hypersetup{colorlinks=true}   % Set to false for black/white printing
\input{Definitions}            % Include your abbreviations



\usepackage{enumitem}% http://ctan.org/pkg/enumitem
\usepackage{multirow}
\usepackage{float}
\usepackage{amsmath}
\usepackage{multicol}
\usepackage{amssymb}
\usepackage[normalem]{ulem}
\useunder{\uline}{\ul}{}
\usepackage{wrapfig}


\usepackage[table,xcdraw]{xcolor}


%% ----------------------------------------------------------------
\begin{document}
\frontmatter
\title      {Heterogeneous Agent-based Model for Supermarket Competition}
\authors    {\texorpdfstring
             {\href{mailto:sc22g13@ecs.soton.ac.uk}{Stefan J. Collier}}
             {Stefan J. Collier}
            }
\addresses  {\groupname\\\deptname\\\univname}
\date       {\today}
\subject    {}
\keywords   {}
\supervisor {Dr. Maria Polukarov}
\examiner   {Professor Sheng Chen}

\maketitle
\begin{abstract}
This project aim was to model and analyse the effects of competitive pricing behaviors of grocery retailers on the British market. 

This was achieved by creating a multi-agent model, containing retailer and consumer agents. The heterogeneous crowd of retailers employs either a uniform pricing strategy or a ‘local price flexing’ strategy. The actions of these retailers are chosen by predicting the profit of each action, using a perceptron. Following on from the consideration of different economic models, a discrete model was developed so that software agents have a discrete environment to operate within. Within the model, it has been observed how supermarkets with differing behaviors affect a heterogeneous crowd of consumer agents. The model was implemented in Java with Python used to evaluate the results. 

The simulation displays good acceptance with real grocery market behavior, i.e. captures the performance of British retailers thus can be used to determine the impact of changes in their behavior on their competitors and consumers.Furthermore it can be used to provide insight into sustainability of volatile pricing strategies, providing a useful insight in volatility of British supermarket retail industry. 
\end{abstract}
\acknowledgements{
I would like to express my sincere gratitude to Dr Maria Polukarov for her guidance and support which provided me the freedom to take this research in the direction of my interest.\\
\\
I would also like to thank my family and friends for their encouragement and support. To those who quietly listened to my software complaints. To those who worked throughout the nights with me. To those who helped me write what I couldn't say. I cannot thank you enough.
}

\declaration{
I, Stefan Collier, declare that this dissertation and the work presented in it are my own and has been generated by me as the result of my own original research.\\
I confirm that:\\
1. This work was done wholly or mainly while in candidature for a degree at this University;\\
2. Where any part of this dissertation has previously been submitted for any other qualification at this University or any other institution, this has been clearly stated;\\
3. Where I have consulted the published work of others, this is always clearly attributed;\\
4. Where I have quoted from the work of others, the source is always given. With the exception of such quotations, this dissertation is entirely my own work;\\
5. I have acknowledged all main sources of help;\\
6. Where the thesis is based on work done by myself jointly with others, I have made clear exactly what was done by others and what I have contributed myself;\\
7. Either none of this work has been published before submission, or parts of this work have been published by :\\
\\
Stefan Collier\\
April 2016
}
\tableofcontents
\listoffigures
\listoftables

\mainmatter
%% ----------------------------------------------------------------
%\include{Introduction}
%\include{Conclusions}
\include{chapters/1Project/main}
\include{chapters/2Lit/main}
\include{chapters/3Design/HighLevel}
\include{chapters/3Design/InDepth}
\include{chapters/4Impl/main}

\include{chapters/5Experiments/1/main}
\include{chapters/5Experiments/2/main}
\include{chapters/5Experiments/3/main}
\include{chapters/5Experiments/4/main}

\include{chapters/6Conclusion/main}

\appendix
\include{appendix/AppendixB}
\include{appendix/D/main}
\include{appendix/AppendixC}

\backmatter
\bibliographystyle{ecs}
\bibliography{ECS}
\end{document}
%% ----------------------------------------------------------------


 %% ----------------------------------------------------------------
%% Progress.tex
%% ---------------------------------------------------------------- 
\documentclass{ecsprogress}    % Use the progress Style
\graphicspath{{../figs/}}   % Location of your graphics files
    \usepackage{natbib}            % Use Natbib style for the refs.
\hypersetup{colorlinks=true}   % Set to false for black/white printing
\input{Definitions}            % Include your abbreviations



\usepackage{enumitem}% http://ctan.org/pkg/enumitem
\usepackage{multirow}
\usepackage{float}
\usepackage{amsmath}
\usepackage{multicol}
\usepackage{amssymb}
\usepackage[normalem]{ulem}
\useunder{\uline}{\ul}{}
\usepackage{wrapfig}


\usepackage[table,xcdraw]{xcolor}


%% ----------------------------------------------------------------
\begin{document}
\frontmatter
\title      {Heterogeneous Agent-based Model for Supermarket Competition}
\authors    {\texorpdfstring
             {\href{mailto:sc22g13@ecs.soton.ac.uk}{Stefan J. Collier}}
             {Stefan J. Collier}
            }
\addresses  {\groupname\\\deptname\\\univname}
\date       {\today}
\subject    {}
\keywords   {}
\supervisor {Dr. Maria Polukarov}
\examiner   {Professor Sheng Chen}

\maketitle
\begin{abstract}
This project aim was to model and analyse the effects of competitive pricing behaviors of grocery retailers on the British market. 

This was achieved by creating a multi-agent model, containing retailer and consumer agents. The heterogeneous crowd of retailers employs either a uniform pricing strategy or a ‘local price flexing’ strategy. The actions of these retailers are chosen by predicting the profit of each action, using a perceptron. Following on from the consideration of different economic models, a discrete model was developed so that software agents have a discrete environment to operate within. Within the model, it has been observed how supermarkets with differing behaviors affect a heterogeneous crowd of consumer agents. The model was implemented in Java with Python used to evaluate the results. 

The simulation displays good acceptance with real grocery market behavior, i.e. captures the performance of British retailers thus can be used to determine the impact of changes in their behavior on their competitors and consumers.Furthermore it can be used to provide insight into sustainability of volatile pricing strategies, providing a useful insight in volatility of British supermarket retail industry. 
\end{abstract}
\acknowledgements{
I would like to express my sincere gratitude to Dr Maria Polukarov for her guidance and support which provided me the freedom to take this research in the direction of my interest.\\
\\
I would also like to thank my family and friends for their encouragement and support. To those who quietly listened to my software complaints. To those who worked throughout the nights with me. To those who helped me write what I couldn't say. I cannot thank you enough.
}

\declaration{
I, Stefan Collier, declare that this dissertation and the work presented in it are my own and has been generated by me as the result of my own original research.\\
I confirm that:\\
1. This work was done wholly or mainly while in candidature for a degree at this University;\\
2. Where any part of this dissertation has previously been submitted for any other qualification at this University or any other institution, this has been clearly stated;\\
3. Where I have consulted the published work of others, this is always clearly attributed;\\
4. Where I have quoted from the work of others, the source is always given. With the exception of such quotations, this dissertation is entirely my own work;\\
5. I have acknowledged all main sources of help;\\
6. Where the thesis is based on work done by myself jointly with others, I have made clear exactly what was done by others and what I have contributed myself;\\
7. Either none of this work has been published before submission, or parts of this work have been published by :\\
\\
Stefan Collier\\
April 2016
}
\tableofcontents
\listoffigures
\listoftables

\mainmatter
%% ----------------------------------------------------------------
%\include{Introduction}
%\include{Conclusions}
\include{chapters/1Project/main}
\include{chapters/2Lit/main}
\include{chapters/3Design/HighLevel}
\include{chapters/3Design/InDepth}
\include{chapters/4Impl/main}

\include{chapters/5Experiments/1/main}
\include{chapters/5Experiments/2/main}
\include{chapters/5Experiments/3/main}
\include{chapters/5Experiments/4/main}

\include{chapters/6Conclusion/main}

\appendix
\include{appendix/AppendixB}
\include{appendix/D/main}
\include{appendix/AppendixC}

\backmatter
\bibliographystyle{ecs}
\bibliography{ECS}
\end{document}
%% ----------------------------------------------------------------


\appendix
\include{appendix/AppendixB}
 %% ----------------------------------------------------------------
%% Progress.tex
%% ---------------------------------------------------------------- 
\documentclass{ecsprogress}    % Use the progress Style
\graphicspath{{../figs/}}   % Location of your graphics files
    \usepackage{natbib}            % Use Natbib style for the refs.
\hypersetup{colorlinks=true}   % Set to false for black/white printing
\input{Definitions}            % Include your abbreviations



\usepackage{enumitem}% http://ctan.org/pkg/enumitem
\usepackage{multirow}
\usepackage{float}
\usepackage{amsmath}
\usepackage{multicol}
\usepackage{amssymb}
\usepackage[normalem]{ulem}
\useunder{\uline}{\ul}{}
\usepackage{wrapfig}


\usepackage[table,xcdraw]{xcolor}


%% ----------------------------------------------------------------
\begin{document}
\frontmatter
\title      {Heterogeneous Agent-based Model for Supermarket Competition}
\authors    {\texorpdfstring
             {\href{mailto:sc22g13@ecs.soton.ac.uk}{Stefan J. Collier}}
             {Stefan J. Collier}
            }
\addresses  {\groupname\\\deptname\\\univname}
\date       {\today}
\subject    {}
\keywords   {}
\supervisor {Dr. Maria Polukarov}
\examiner   {Professor Sheng Chen}

\maketitle
\begin{abstract}
This project aim was to model and analyse the effects of competitive pricing behaviors of grocery retailers on the British market. 

This was achieved by creating a multi-agent model, containing retailer and consumer agents. The heterogeneous crowd of retailers employs either a uniform pricing strategy or a ‘local price flexing’ strategy. The actions of these retailers are chosen by predicting the profit of each action, using a perceptron. Following on from the consideration of different economic models, a discrete model was developed so that software agents have a discrete environment to operate within. Within the model, it has been observed how supermarkets with differing behaviors affect a heterogeneous crowd of consumer agents. The model was implemented in Java with Python used to evaluate the results. 

The simulation displays good acceptance with real grocery market behavior, i.e. captures the performance of British retailers thus can be used to determine the impact of changes in their behavior on their competitors and consumers.Furthermore it can be used to provide insight into sustainability of volatile pricing strategies, providing a useful insight in volatility of British supermarket retail industry. 
\end{abstract}
\acknowledgements{
I would like to express my sincere gratitude to Dr Maria Polukarov for her guidance and support which provided me the freedom to take this research in the direction of my interest.\\
\\
I would also like to thank my family and friends for their encouragement and support. To those who quietly listened to my software complaints. To those who worked throughout the nights with me. To those who helped me write what I couldn't say. I cannot thank you enough.
}

\declaration{
I, Stefan Collier, declare that this dissertation and the work presented in it are my own and has been generated by me as the result of my own original research.\\
I confirm that:\\
1. This work was done wholly or mainly while in candidature for a degree at this University;\\
2. Where any part of this dissertation has previously been submitted for any other qualification at this University or any other institution, this has been clearly stated;\\
3. Where I have consulted the published work of others, this is always clearly attributed;\\
4. Where I have quoted from the work of others, the source is always given. With the exception of such quotations, this dissertation is entirely my own work;\\
5. I have acknowledged all main sources of help;\\
6. Where the thesis is based on work done by myself jointly with others, I have made clear exactly what was done by others and what I have contributed myself;\\
7. Either none of this work has been published before submission, or parts of this work have been published by :\\
\\
Stefan Collier\\
April 2016
}
\tableofcontents
\listoffigures
\listoftables

\mainmatter
%% ----------------------------------------------------------------
%\include{Introduction}
%\include{Conclusions}
\include{chapters/1Project/main}
\include{chapters/2Lit/main}
\include{chapters/3Design/HighLevel}
\include{chapters/3Design/InDepth}
\include{chapters/4Impl/main}

\include{chapters/5Experiments/1/main}
\include{chapters/5Experiments/2/main}
\include{chapters/5Experiments/3/main}
\include{chapters/5Experiments/4/main}

\include{chapters/6Conclusion/main}

\appendix
\include{appendix/AppendixB}
\include{appendix/D/main}
\include{appendix/AppendixC}

\backmatter
\bibliographystyle{ecs}
\bibliography{ECS}
\end{document}
%% ----------------------------------------------------------------

\include{appendix/AppendixC}

\backmatter
\bibliographystyle{ecs}
\bibliography{ECS}
\end{document}
%% ----------------------------------------------------------------

\include{appendix/AppendixC}

\backmatter
\bibliographystyle{ecs}
\bibliography{ECS}
\end{document}
%% ----------------------------------------------------------------


 %% ----------------------------------------------------------------
%% Progress.tex
%% ---------------------------------------------------------------- 
\documentclass{ecsprogress}    % Use the progress Style
\graphicspath{{../figs/}}   % Location of your graphics files
    \usepackage{natbib}            % Use Natbib style for the refs.
\hypersetup{colorlinks=true}   % Set to false for black/white printing
\input{Definitions}            % Include your abbreviations



\usepackage{enumitem}% http://ctan.org/pkg/enumitem
\usepackage{multirow}
\usepackage{float}
\usepackage{amsmath}
\usepackage{multicol}
\usepackage{amssymb}
\usepackage[normalem]{ulem}
\useunder{\uline}{\ul}{}
\usepackage{wrapfig}


\usepackage[table,xcdraw]{xcolor}


%% ----------------------------------------------------------------
\begin{document}
\frontmatter
\title      {Heterogeneous Agent-based Model for Supermarket Competition}
\authors    {\texorpdfstring
             {\href{mailto:sc22g13@ecs.soton.ac.uk}{Stefan J. Collier}}
             {Stefan J. Collier}
            }
\addresses  {\groupname\\\deptname\\\univname}
\date       {\today}
\subject    {}
\keywords   {}
\supervisor {Dr. Maria Polukarov}
\examiner   {Professor Sheng Chen}

\maketitle
\begin{abstract}
This project aim was to model and analyse the effects of competitive pricing behaviors of grocery retailers on the British market. 

This was achieved by creating a multi-agent model, containing retailer and consumer agents. The heterogeneous crowd of retailers employs either a uniform pricing strategy or a ‘local price flexing’ strategy. The actions of these retailers are chosen by predicting the profit of each action, using a perceptron. Following on from the consideration of different economic models, a discrete model was developed so that software agents have a discrete environment to operate within. Within the model, it has been observed how supermarkets with differing behaviors affect a heterogeneous crowd of consumer agents. The model was implemented in Java with Python used to evaluate the results. 

The simulation displays good acceptance with real grocery market behavior, i.e. captures the performance of British retailers thus can be used to determine the impact of changes in their behavior on their competitors and consumers.Furthermore it can be used to provide insight into sustainability of volatile pricing strategies, providing a useful insight in volatility of British supermarket retail industry. 
\end{abstract}
\acknowledgements{
I would like to express my sincere gratitude to Dr Maria Polukarov for her guidance and support which provided me the freedom to take this research in the direction of my interest.\\
\\
I would also like to thank my family and friends for their encouragement and support. To those who quietly listened to my software complaints. To those who worked throughout the nights with me. To those who helped me write what I couldn't say. I cannot thank you enough.
}

\declaration{
I, Stefan Collier, declare that this dissertation and the work presented in it are my own and has been generated by me as the result of my own original research.\\
I confirm that:\\
1. This work was done wholly or mainly while in candidature for a degree at this University;\\
2. Where any part of this dissertation has previously been submitted for any other qualification at this University or any other institution, this has been clearly stated;\\
3. Where I have consulted the published work of others, this is always clearly attributed;\\
4. Where I have quoted from the work of others, the source is always given. With the exception of such quotations, this dissertation is entirely my own work;\\
5. I have acknowledged all main sources of help;\\
6. Where the thesis is based on work done by myself jointly with others, I have made clear exactly what was done by others and what I have contributed myself;\\
7. Either none of this work has been published before submission, or parts of this work have been published by :\\
\\
Stefan Collier\\
April 2016
}
\tableofcontents
\listoffigures
\listoftables

\mainmatter
%% ----------------------------------------------------------------
%\include{Introduction}
%\include{Conclusions}
 %% ----------------------------------------------------------------
%% Progress.tex
%% ---------------------------------------------------------------- 
\documentclass{ecsprogress}    % Use the progress Style
\graphicspath{{../figs/}}   % Location of your graphics files
    \usepackage{natbib}            % Use Natbib style for the refs.
\hypersetup{colorlinks=true}   % Set to false for black/white printing
\input{Definitions}            % Include your abbreviations



\usepackage{enumitem}% http://ctan.org/pkg/enumitem
\usepackage{multirow}
\usepackage{float}
\usepackage{amsmath}
\usepackage{multicol}
\usepackage{amssymb}
\usepackage[normalem]{ulem}
\useunder{\uline}{\ul}{}
\usepackage{wrapfig}


\usepackage[table,xcdraw]{xcolor}


%% ----------------------------------------------------------------
\begin{document}
\frontmatter
\title      {Heterogeneous Agent-based Model for Supermarket Competition}
\authors    {\texorpdfstring
             {\href{mailto:sc22g13@ecs.soton.ac.uk}{Stefan J. Collier}}
             {Stefan J. Collier}
            }
\addresses  {\groupname\\\deptname\\\univname}
\date       {\today}
\subject    {}
\keywords   {}
\supervisor {Dr. Maria Polukarov}
\examiner   {Professor Sheng Chen}

\maketitle
\begin{abstract}
This project aim was to model and analyse the effects of competitive pricing behaviors of grocery retailers on the British market. 

This was achieved by creating a multi-agent model, containing retailer and consumer agents. The heterogeneous crowd of retailers employs either a uniform pricing strategy or a ‘local price flexing’ strategy. The actions of these retailers are chosen by predicting the profit of each action, using a perceptron. Following on from the consideration of different economic models, a discrete model was developed so that software agents have a discrete environment to operate within. Within the model, it has been observed how supermarkets with differing behaviors affect a heterogeneous crowd of consumer agents. The model was implemented in Java with Python used to evaluate the results. 

The simulation displays good acceptance with real grocery market behavior, i.e. captures the performance of British retailers thus can be used to determine the impact of changes in their behavior on their competitors and consumers.Furthermore it can be used to provide insight into sustainability of volatile pricing strategies, providing a useful insight in volatility of British supermarket retail industry. 
\end{abstract}
\acknowledgements{
I would like to express my sincere gratitude to Dr Maria Polukarov for her guidance and support which provided me the freedom to take this research in the direction of my interest.\\
\\
I would also like to thank my family and friends for their encouragement and support. To those who quietly listened to my software complaints. To those who worked throughout the nights with me. To those who helped me write what I couldn't say. I cannot thank you enough.
}

\declaration{
I, Stefan Collier, declare that this dissertation and the work presented in it are my own and has been generated by me as the result of my own original research.\\
I confirm that:\\
1. This work was done wholly or mainly while in candidature for a degree at this University;\\
2. Where any part of this dissertation has previously been submitted for any other qualification at this University or any other institution, this has been clearly stated;\\
3. Where I have consulted the published work of others, this is always clearly attributed;\\
4. Where I have quoted from the work of others, the source is always given. With the exception of such quotations, this dissertation is entirely my own work;\\
5. I have acknowledged all main sources of help;\\
6. Where the thesis is based on work done by myself jointly with others, I have made clear exactly what was done by others and what I have contributed myself;\\
7. Either none of this work has been published before submission, or parts of this work have been published by :\\
\\
Stefan Collier\\
April 2016
}
\tableofcontents
\listoffigures
\listoftables

\mainmatter
%% ----------------------------------------------------------------
%\include{Introduction}
%\include{Conclusions}
 %% ----------------------------------------------------------------
%% Progress.tex
%% ---------------------------------------------------------------- 
\documentclass{ecsprogress}    % Use the progress Style
\graphicspath{{../figs/}}   % Location of your graphics files
    \usepackage{natbib}            % Use Natbib style for the refs.
\hypersetup{colorlinks=true}   % Set to false for black/white printing
\input{Definitions}            % Include your abbreviations



\usepackage{enumitem}% http://ctan.org/pkg/enumitem
\usepackage{multirow}
\usepackage{float}
\usepackage{amsmath}
\usepackage{multicol}
\usepackage{amssymb}
\usepackage[normalem]{ulem}
\useunder{\uline}{\ul}{}
\usepackage{wrapfig}


\usepackage[table,xcdraw]{xcolor}


%% ----------------------------------------------------------------
\begin{document}
\frontmatter
\title      {Heterogeneous Agent-based Model for Supermarket Competition}
\authors    {\texorpdfstring
             {\href{mailto:sc22g13@ecs.soton.ac.uk}{Stefan J. Collier}}
             {Stefan J. Collier}
            }
\addresses  {\groupname\\\deptname\\\univname}
\date       {\today}
\subject    {}
\keywords   {}
\supervisor {Dr. Maria Polukarov}
\examiner   {Professor Sheng Chen}

\maketitle
\begin{abstract}
This project aim was to model and analyse the effects of competitive pricing behaviors of grocery retailers on the British market. 

This was achieved by creating a multi-agent model, containing retailer and consumer agents. The heterogeneous crowd of retailers employs either a uniform pricing strategy or a ‘local price flexing’ strategy. The actions of these retailers are chosen by predicting the profit of each action, using a perceptron. Following on from the consideration of different economic models, a discrete model was developed so that software agents have a discrete environment to operate within. Within the model, it has been observed how supermarkets with differing behaviors affect a heterogeneous crowd of consumer agents. The model was implemented in Java with Python used to evaluate the results. 

The simulation displays good acceptance with real grocery market behavior, i.e. captures the performance of British retailers thus can be used to determine the impact of changes in their behavior on their competitors and consumers.Furthermore it can be used to provide insight into sustainability of volatile pricing strategies, providing a useful insight in volatility of British supermarket retail industry. 
\end{abstract}
\acknowledgements{
I would like to express my sincere gratitude to Dr Maria Polukarov for her guidance and support which provided me the freedom to take this research in the direction of my interest.\\
\\
I would also like to thank my family and friends for their encouragement and support. To those who quietly listened to my software complaints. To those who worked throughout the nights with me. To those who helped me write what I couldn't say. I cannot thank you enough.
}

\declaration{
I, Stefan Collier, declare that this dissertation and the work presented in it are my own and has been generated by me as the result of my own original research.\\
I confirm that:\\
1. This work was done wholly or mainly while in candidature for a degree at this University;\\
2. Where any part of this dissertation has previously been submitted for any other qualification at this University or any other institution, this has been clearly stated;\\
3. Where I have consulted the published work of others, this is always clearly attributed;\\
4. Where I have quoted from the work of others, the source is always given. With the exception of such quotations, this dissertation is entirely my own work;\\
5. I have acknowledged all main sources of help;\\
6. Where the thesis is based on work done by myself jointly with others, I have made clear exactly what was done by others and what I have contributed myself;\\
7. Either none of this work has been published before submission, or parts of this work have been published by :\\
\\
Stefan Collier\\
April 2016
}
\tableofcontents
\listoffigures
\listoftables

\mainmatter
%% ----------------------------------------------------------------
%\include{Introduction}
%\include{Conclusions}
\include{chapters/1Project/main}
\include{chapters/2Lit/main}
\include{chapters/3Design/HighLevel}
\include{chapters/3Design/InDepth}
\include{chapters/4Impl/main}

\include{chapters/5Experiments/1/main}
\include{chapters/5Experiments/2/main}
\include{chapters/5Experiments/3/main}
\include{chapters/5Experiments/4/main}

\include{chapters/6Conclusion/main}

\appendix
\include{appendix/AppendixB}
\include{appendix/D/main}
\include{appendix/AppendixC}

\backmatter
\bibliographystyle{ecs}
\bibliography{ECS}
\end{document}
%% ----------------------------------------------------------------

 %% ----------------------------------------------------------------
%% Progress.tex
%% ---------------------------------------------------------------- 
\documentclass{ecsprogress}    % Use the progress Style
\graphicspath{{../figs/}}   % Location of your graphics files
    \usepackage{natbib}            % Use Natbib style for the refs.
\hypersetup{colorlinks=true}   % Set to false for black/white printing
\input{Definitions}            % Include your abbreviations



\usepackage{enumitem}% http://ctan.org/pkg/enumitem
\usepackage{multirow}
\usepackage{float}
\usepackage{amsmath}
\usepackage{multicol}
\usepackage{amssymb}
\usepackage[normalem]{ulem}
\useunder{\uline}{\ul}{}
\usepackage{wrapfig}


\usepackage[table,xcdraw]{xcolor}


%% ----------------------------------------------------------------
\begin{document}
\frontmatter
\title      {Heterogeneous Agent-based Model for Supermarket Competition}
\authors    {\texorpdfstring
             {\href{mailto:sc22g13@ecs.soton.ac.uk}{Stefan J. Collier}}
             {Stefan J. Collier}
            }
\addresses  {\groupname\\\deptname\\\univname}
\date       {\today}
\subject    {}
\keywords   {}
\supervisor {Dr. Maria Polukarov}
\examiner   {Professor Sheng Chen}

\maketitle
\begin{abstract}
This project aim was to model and analyse the effects of competitive pricing behaviors of grocery retailers on the British market. 

This was achieved by creating a multi-agent model, containing retailer and consumer agents. The heterogeneous crowd of retailers employs either a uniform pricing strategy or a ‘local price flexing’ strategy. The actions of these retailers are chosen by predicting the profit of each action, using a perceptron. Following on from the consideration of different economic models, a discrete model was developed so that software agents have a discrete environment to operate within. Within the model, it has been observed how supermarkets with differing behaviors affect a heterogeneous crowd of consumer agents. The model was implemented in Java with Python used to evaluate the results. 

The simulation displays good acceptance with real grocery market behavior, i.e. captures the performance of British retailers thus can be used to determine the impact of changes in their behavior on their competitors and consumers.Furthermore it can be used to provide insight into sustainability of volatile pricing strategies, providing a useful insight in volatility of British supermarket retail industry. 
\end{abstract}
\acknowledgements{
I would like to express my sincere gratitude to Dr Maria Polukarov for her guidance and support which provided me the freedom to take this research in the direction of my interest.\\
\\
I would also like to thank my family and friends for their encouragement and support. To those who quietly listened to my software complaints. To those who worked throughout the nights with me. To those who helped me write what I couldn't say. I cannot thank you enough.
}

\declaration{
I, Stefan Collier, declare that this dissertation and the work presented in it are my own and has been generated by me as the result of my own original research.\\
I confirm that:\\
1. This work was done wholly or mainly while in candidature for a degree at this University;\\
2. Where any part of this dissertation has previously been submitted for any other qualification at this University or any other institution, this has been clearly stated;\\
3. Where I have consulted the published work of others, this is always clearly attributed;\\
4. Where I have quoted from the work of others, the source is always given. With the exception of such quotations, this dissertation is entirely my own work;\\
5. I have acknowledged all main sources of help;\\
6. Where the thesis is based on work done by myself jointly with others, I have made clear exactly what was done by others and what I have contributed myself;\\
7. Either none of this work has been published before submission, or parts of this work have been published by :\\
\\
Stefan Collier\\
April 2016
}
\tableofcontents
\listoffigures
\listoftables

\mainmatter
%% ----------------------------------------------------------------
%\include{Introduction}
%\include{Conclusions}
\include{chapters/1Project/main}
\include{chapters/2Lit/main}
\include{chapters/3Design/HighLevel}
\include{chapters/3Design/InDepth}
\include{chapters/4Impl/main}

\include{chapters/5Experiments/1/main}
\include{chapters/5Experiments/2/main}
\include{chapters/5Experiments/3/main}
\include{chapters/5Experiments/4/main}

\include{chapters/6Conclusion/main}

\appendix
\include{appendix/AppendixB}
\include{appendix/D/main}
\include{appendix/AppendixC}

\backmatter
\bibliographystyle{ecs}
\bibliography{ECS}
\end{document}
%% ----------------------------------------------------------------

\include{chapters/3Design/HighLevel}
\include{chapters/3Design/InDepth}
 %% ----------------------------------------------------------------
%% Progress.tex
%% ---------------------------------------------------------------- 
\documentclass{ecsprogress}    % Use the progress Style
\graphicspath{{../figs/}}   % Location of your graphics files
    \usepackage{natbib}            % Use Natbib style for the refs.
\hypersetup{colorlinks=true}   % Set to false for black/white printing
\input{Definitions}            % Include your abbreviations



\usepackage{enumitem}% http://ctan.org/pkg/enumitem
\usepackage{multirow}
\usepackage{float}
\usepackage{amsmath}
\usepackage{multicol}
\usepackage{amssymb}
\usepackage[normalem]{ulem}
\useunder{\uline}{\ul}{}
\usepackage{wrapfig}


\usepackage[table,xcdraw]{xcolor}


%% ----------------------------------------------------------------
\begin{document}
\frontmatter
\title      {Heterogeneous Agent-based Model for Supermarket Competition}
\authors    {\texorpdfstring
             {\href{mailto:sc22g13@ecs.soton.ac.uk}{Stefan J. Collier}}
             {Stefan J. Collier}
            }
\addresses  {\groupname\\\deptname\\\univname}
\date       {\today}
\subject    {}
\keywords   {}
\supervisor {Dr. Maria Polukarov}
\examiner   {Professor Sheng Chen}

\maketitle
\begin{abstract}
This project aim was to model and analyse the effects of competitive pricing behaviors of grocery retailers on the British market. 

This was achieved by creating a multi-agent model, containing retailer and consumer agents. The heterogeneous crowd of retailers employs either a uniform pricing strategy or a ‘local price flexing’ strategy. The actions of these retailers are chosen by predicting the profit of each action, using a perceptron. Following on from the consideration of different economic models, a discrete model was developed so that software agents have a discrete environment to operate within. Within the model, it has been observed how supermarkets with differing behaviors affect a heterogeneous crowd of consumer agents. The model was implemented in Java with Python used to evaluate the results. 

The simulation displays good acceptance with real grocery market behavior, i.e. captures the performance of British retailers thus can be used to determine the impact of changes in their behavior on their competitors and consumers.Furthermore it can be used to provide insight into sustainability of volatile pricing strategies, providing a useful insight in volatility of British supermarket retail industry. 
\end{abstract}
\acknowledgements{
I would like to express my sincere gratitude to Dr Maria Polukarov for her guidance and support which provided me the freedom to take this research in the direction of my interest.\\
\\
I would also like to thank my family and friends for their encouragement and support. To those who quietly listened to my software complaints. To those who worked throughout the nights with me. To those who helped me write what I couldn't say. I cannot thank you enough.
}

\declaration{
I, Stefan Collier, declare that this dissertation and the work presented in it are my own and has been generated by me as the result of my own original research.\\
I confirm that:\\
1. This work was done wholly or mainly while in candidature for a degree at this University;\\
2. Where any part of this dissertation has previously been submitted for any other qualification at this University or any other institution, this has been clearly stated;\\
3. Where I have consulted the published work of others, this is always clearly attributed;\\
4. Where I have quoted from the work of others, the source is always given. With the exception of such quotations, this dissertation is entirely my own work;\\
5. I have acknowledged all main sources of help;\\
6. Where the thesis is based on work done by myself jointly with others, I have made clear exactly what was done by others and what I have contributed myself;\\
7. Either none of this work has been published before submission, or parts of this work have been published by :\\
\\
Stefan Collier\\
April 2016
}
\tableofcontents
\listoffigures
\listoftables

\mainmatter
%% ----------------------------------------------------------------
%\include{Introduction}
%\include{Conclusions}
\include{chapters/1Project/main}
\include{chapters/2Lit/main}
\include{chapters/3Design/HighLevel}
\include{chapters/3Design/InDepth}
\include{chapters/4Impl/main}

\include{chapters/5Experiments/1/main}
\include{chapters/5Experiments/2/main}
\include{chapters/5Experiments/3/main}
\include{chapters/5Experiments/4/main}

\include{chapters/6Conclusion/main}

\appendix
\include{appendix/AppendixB}
\include{appendix/D/main}
\include{appendix/AppendixC}

\backmatter
\bibliographystyle{ecs}
\bibliography{ECS}
\end{document}
%% ----------------------------------------------------------------


 %% ----------------------------------------------------------------
%% Progress.tex
%% ---------------------------------------------------------------- 
\documentclass{ecsprogress}    % Use the progress Style
\graphicspath{{../figs/}}   % Location of your graphics files
    \usepackage{natbib}            % Use Natbib style for the refs.
\hypersetup{colorlinks=true}   % Set to false for black/white printing
\input{Definitions}            % Include your abbreviations



\usepackage{enumitem}% http://ctan.org/pkg/enumitem
\usepackage{multirow}
\usepackage{float}
\usepackage{amsmath}
\usepackage{multicol}
\usepackage{amssymb}
\usepackage[normalem]{ulem}
\useunder{\uline}{\ul}{}
\usepackage{wrapfig}


\usepackage[table,xcdraw]{xcolor}


%% ----------------------------------------------------------------
\begin{document}
\frontmatter
\title      {Heterogeneous Agent-based Model for Supermarket Competition}
\authors    {\texorpdfstring
             {\href{mailto:sc22g13@ecs.soton.ac.uk}{Stefan J. Collier}}
             {Stefan J. Collier}
            }
\addresses  {\groupname\\\deptname\\\univname}
\date       {\today}
\subject    {}
\keywords   {}
\supervisor {Dr. Maria Polukarov}
\examiner   {Professor Sheng Chen}

\maketitle
\begin{abstract}
This project aim was to model and analyse the effects of competitive pricing behaviors of grocery retailers on the British market. 

This was achieved by creating a multi-agent model, containing retailer and consumer agents. The heterogeneous crowd of retailers employs either a uniform pricing strategy or a ‘local price flexing’ strategy. The actions of these retailers are chosen by predicting the profit of each action, using a perceptron. Following on from the consideration of different economic models, a discrete model was developed so that software agents have a discrete environment to operate within. Within the model, it has been observed how supermarkets with differing behaviors affect a heterogeneous crowd of consumer agents. The model was implemented in Java with Python used to evaluate the results. 

The simulation displays good acceptance with real grocery market behavior, i.e. captures the performance of British retailers thus can be used to determine the impact of changes in their behavior on their competitors and consumers.Furthermore it can be used to provide insight into sustainability of volatile pricing strategies, providing a useful insight in volatility of British supermarket retail industry. 
\end{abstract}
\acknowledgements{
I would like to express my sincere gratitude to Dr Maria Polukarov for her guidance and support which provided me the freedom to take this research in the direction of my interest.\\
\\
I would also like to thank my family and friends for their encouragement and support. To those who quietly listened to my software complaints. To those who worked throughout the nights with me. To those who helped me write what I couldn't say. I cannot thank you enough.
}

\declaration{
I, Stefan Collier, declare that this dissertation and the work presented in it are my own and has been generated by me as the result of my own original research.\\
I confirm that:\\
1. This work was done wholly or mainly while in candidature for a degree at this University;\\
2. Where any part of this dissertation has previously been submitted for any other qualification at this University or any other institution, this has been clearly stated;\\
3. Where I have consulted the published work of others, this is always clearly attributed;\\
4. Where I have quoted from the work of others, the source is always given. With the exception of such quotations, this dissertation is entirely my own work;\\
5. I have acknowledged all main sources of help;\\
6. Where the thesis is based on work done by myself jointly with others, I have made clear exactly what was done by others and what I have contributed myself;\\
7. Either none of this work has been published before submission, or parts of this work have been published by :\\
\\
Stefan Collier\\
April 2016
}
\tableofcontents
\listoffigures
\listoftables

\mainmatter
%% ----------------------------------------------------------------
%\include{Introduction}
%\include{Conclusions}
\include{chapters/1Project/main}
\include{chapters/2Lit/main}
\include{chapters/3Design/HighLevel}
\include{chapters/3Design/InDepth}
\include{chapters/4Impl/main}

\include{chapters/5Experiments/1/main}
\include{chapters/5Experiments/2/main}
\include{chapters/5Experiments/3/main}
\include{chapters/5Experiments/4/main}

\include{chapters/6Conclusion/main}

\appendix
\include{appendix/AppendixB}
\include{appendix/D/main}
\include{appendix/AppendixC}

\backmatter
\bibliographystyle{ecs}
\bibliography{ECS}
\end{document}
%% ----------------------------------------------------------------

 %% ----------------------------------------------------------------
%% Progress.tex
%% ---------------------------------------------------------------- 
\documentclass{ecsprogress}    % Use the progress Style
\graphicspath{{../figs/}}   % Location of your graphics files
    \usepackage{natbib}            % Use Natbib style for the refs.
\hypersetup{colorlinks=true}   % Set to false for black/white printing
\input{Definitions}            % Include your abbreviations



\usepackage{enumitem}% http://ctan.org/pkg/enumitem
\usepackage{multirow}
\usepackage{float}
\usepackage{amsmath}
\usepackage{multicol}
\usepackage{amssymb}
\usepackage[normalem]{ulem}
\useunder{\uline}{\ul}{}
\usepackage{wrapfig}


\usepackage[table,xcdraw]{xcolor}


%% ----------------------------------------------------------------
\begin{document}
\frontmatter
\title      {Heterogeneous Agent-based Model for Supermarket Competition}
\authors    {\texorpdfstring
             {\href{mailto:sc22g13@ecs.soton.ac.uk}{Stefan J. Collier}}
             {Stefan J. Collier}
            }
\addresses  {\groupname\\\deptname\\\univname}
\date       {\today}
\subject    {}
\keywords   {}
\supervisor {Dr. Maria Polukarov}
\examiner   {Professor Sheng Chen}

\maketitle
\begin{abstract}
This project aim was to model and analyse the effects of competitive pricing behaviors of grocery retailers on the British market. 

This was achieved by creating a multi-agent model, containing retailer and consumer agents. The heterogeneous crowd of retailers employs either a uniform pricing strategy or a ‘local price flexing’ strategy. The actions of these retailers are chosen by predicting the profit of each action, using a perceptron. Following on from the consideration of different economic models, a discrete model was developed so that software agents have a discrete environment to operate within. Within the model, it has been observed how supermarkets with differing behaviors affect a heterogeneous crowd of consumer agents. The model was implemented in Java with Python used to evaluate the results. 

The simulation displays good acceptance with real grocery market behavior, i.e. captures the performance of British retailers thus can be used to determine the impact of changes in their behavior on their competitors and consumers.Furthermore it can be used to provide insight into sustainability of volatile pricing strategies, providing a useful insight in volatility of British supermarket retail industry. 
\end{abstract}
\acknowledgements{
I would like to express my sincere gratitude to Dr Maria Polukarov for her guidance and support which provided me the freedom to take this research in the direction of my interest.\\
\\
I would also like to thank my family and friends for their encouragement and support. To those who quietly listened to my software complaints. To those who worked throughout the nights with me. To those who helped me write what I couldn't say. I cannot thank you enough.
}

\declaration{
I, Stefan Collier, declare that this dissertation and the work presented in it are my own and has been generated by me as the result of my own original research.\\
I confirm that:\\
1. This work was done wholly or mainly while in candidature for a degree at this University;\\
2. Where any part of this dissertation has previously been submitted for any other qualification at this University or any other institution, this has been clearly stated;\\
3. Where I have consulted the published work of others, this is always clearly attributed;\\
4. Where I have quoted from the work of others, the source is always given. With the exception of such quotations, this dissertation is entirely my own work;\\
5. I have acknowledged all main sources of help;\\
6. Where the thesis is based on work done by myself jointly with others, I have made clear exactly what was done by others and what I have contributed myself;\\
7. Either none of this work has been published before submission, or parts of this work have been published by :\\
\\
Stefan Collier\\
April 2016
}
\tableofcontents
\listoffigures
\listoftables

\mainmatter
%% ----------------------------------------------------------------
%\include{Introduction}
%\include{Conclusions}
\include{chapters/1Project/main}
\include{chapters/2Lit/main}
\include{chapters/3Design/HighLevel}
\include{chapters/3Design/InDepth}
\include{chapters/4Impl/main}

\include{chapters/5Experiments/1/main}
\include{chapters/5Experiments/2/main}
\include{chapters/5Experiments/3/main}
\include{chapters/5Experiments/4/main}

\include{chapters/6Conclusion/main}

\appendix
\include{appendix/AppendixB}
\include{appendix/D/main}
\include{appendix/AppendixC}

\backmatter
\bibliographystyle{ecs}
\bibliography{ECS}
\end{document}
%% ----------------------------------------------------------------

 %% ----------------------------------------------------------------
%% Progress.tex
%% ---------------------------------------------------------------- 
\documentclass{ecsprogress}    % Use the progress Style
\graphicspath{{../figs/}}   % Location of your graphics files
    \usepackage{natbib}            % Use Natbib style for the refs.
\hypersetup{colorlinks=true}   % Set to false for black/white printing
\input{Definitions}            % Include your abbreviations



\usepackage{enumitem}% http://ctan.org/pkg/enumitem
\usepackage{multirow}
\usepackage{float}
\usepackage{amsmath}
\usepackage{multicol}
\usepackage{amssymb}
\usepackage[normalem]{ulem}
\useunder{\uline}{\ul}{}
\usepackage{wrapfig}


\usepackage[table,xcdraw]{xcolor}


%% ----------------------------------------------------------------
\begin{document}
\frontmatter
\title      {Heterogeneous Agent-based Model for Supermarket Competition}
\authors    {\texorpdfstring
             {\href{mailto:sc22g13@ecs.soton.ac.uk}{Stefan J. Collier}}
             {Stefan J. Collier}
            }
\addresses  {\groupname\\\deptname\\\univname}
\date       {\today}
\subject    {}
\keywords   {}
\supervisor {Dr. Maria Polukarov}
\examiner   {Professor Sheng Chen}

\maketitle
\begin{abstract}
This project aim was to model and analyse the effects of competitive pricing behaviors of grocery retailers on the British market. 

This was achieved by creating a multi-agent model, containing retailer and consumer agents. The heterogeneous crowd of retailers employs either a uniform pricing strategy or a ‘local price flexing’ strategy. The actions of these retailers are chosen by predicting the profit of each action, using a perceptron. Following on from the consideration of different economic models, a discrete model was developed so that software agents have a discrete environment to operate within. Within the model, it has been observed how supermarkets with differing behaviors affect a heterogeneous crowd of consumer agents. The model was implemented in Java with Python used to evaluate the results. 

The simulation displays good acceptance with real grocery market behavior, i.e. captures the performance of British retailers thus can be used to determine the impact of changes in their behavior on their competitors and consumers.Furthermore it can be used to provide insight into sustainability of volatile pricing strategies, providing a useful insight in volatility of British supermarket retail industry. 
\end{abstract}
\acknowledgements{
I would like to express my sincere gratitude to Dr Maria Polukarov for her guidance and support which provided me the freedom to take this research in the direction of my interest.\\
\\
I would also like to thank my family and friends for their encouragement and support. To those who quietly listened to my software complaints. To those who worked throughout the nights with me. To those who helped me write what I couldn't say. I cannot thank you enough.
}

\declaration{
I, Stefan Collier, declare that this dissertation and the work presented in it are my own and has been generated by me as the result of my own original research.\\
I confirm that:\\
1. This work was done wholly or mainly while in candidature for a degree at this University;\\
2. Where any part of this dissertation has previously been submitted for any other qualification at this University or any other institution, this has been clearly stated;\\
3. Where I have consulted the published work of others, this is always clearly attributed;\\
4. Where I have quoted from the work of others, the source is always given. With the exception of such quotations, this dissertation is entirely my own work;\\
5. I have acknowledged all main sources of help;\\
6. Where the thesis is based on work done by myself jointly with others, I have made clear exactly what was done by others and what I have contributed myself;\\
7. Either none of this work has been published before submission, or parts of this work have been published by :\\
\\
Stefan Collier\\
April 2016
}
\tableofcontents
\listoffigures
\listoftables

\mainmatter
%% ----------------------------------------------------------------
%\include{Introduction}
%\include{Conclusions}
\include{chapters/1Project/main}
\include{chapters/2Lit/main}
\include{chapters/3Design/HighLevel}
\include{chapters/3Design/InDepth}
\include{chapters/4Impl/main}

\include{chapters/5Experiments/1/main}
\include{chapters/5Experiments/2/main}
\include{chapters/5Experiments/3/main}
\include{chapters/5Experiments/4/main}

\include{chapters/6Conclusion/main}

\appendix
\include{appendix/AppendixB}
\include{appendix/D/main}
\include{appendix/AppendixC}

\backmatter
\bibliographystyle{ecs}
\bibliography{ECS}
\end{document}
%% ----------------------------------------------------------------

 %% ----------------------------------------------------------------
%% Progress.tex
%% ---------------------------------------------------------------- 
\documentclass{ecsprogress}    % Use the progress Style
\graphicspath{{../figs/}}   % Location of your graphics files
    \usepackage{natbib}            % Use Natbib style for the refs.
\hypersetup{colorlinks=true}   % Set to false for black/white printing
\input{Definitions}            % Include your abbreviations



\usepackage{enumitem}% http://ctan.org/pkg/enumitem
\usepackage{multirow}
\usepackage{float}
\usepackage{amsmath}
\usepackage{multicol}
\usepackage{amssymb}
\usepackage[normalem]{ulem}
\useunder{\uline}{\ul}{}
\usepackage{wrapfig}


\usepackage[table,xcdraw]{xcolor}


%% ----------------------------------------------------------------
\begin{document}
\frontmatter
\title      {Heterogeneous Agent-based Model for Supermarket Competition}
\authors    {\texorpdfstring
             {\href{mailto:sc22g13@ecs.soton.ac.uk}{Stefan J. Collier}}
             {Stefan J. Collier}
            }
\addresses  {\groupname\\\deptname\\\univname}
\date       {\today}
\subject    {}
\keywords   {}
\supervisor {Dr. Maria Polukarov}
\examiner   {Professor Sheng Chen}

\maketitle
\begin{abstract}
This project aim was to model and analyse the effects of competitive pricing behaviors of grocery retailers on the British market. 

This was achieved by creating a multi-agent model, containing retailer and consumer agents. The heterogeneous crowd of retailers employs either a uniform pricing strategy or a ‘local price flexing’ strategy. The actions of these retailers are chosen by predicting the profit of each action, using a perceptron. Following on from the consideration of different economic models, a discrete model was developed so that software agents have a discrete environment to operate within. Within the model, it has been observed how supermarkets with differing behaviors affect a heterogeneous crowd of consumer agents. The model was implemented in Java with Python used to evaluate the results. 

The simulation displays good acceptance with real grocery market behavior, i.e. captures the performance of British retailers thus can be used to determine the impact of changes in their behavior on their competitors and consumers.Furthermore it can be used to provide insight into sustainability of volatile pricing strategies, providing a useful insight in volatility of British supermarket retail industry. 
\end{abstract}
\acknowledgements{
I would like to express my sincere gratitude to Dr Maria Polukarov for her guidance and support which provided me the freedom to take this research in the direction of my interest.\\
\\
I would also like to thank my family and friends for their encouragement and support. To those who quietly listened to my software complaints. To those who worked throughout the nights with me. To those who helped me write what I couldn't say. I cannot thank you enough.
}

\declaration{
I, Stefan Collier, declare that this dissertation and the work presented in it are my own and has been generated by me as the result of my own original research.\\
I confirm that:\\
1. This work was done wholly or mainly while in candidature for a degree at this University;\\
2. Where any part of this dissertation has previously been submitted for any other qualification at this University or any other institution, this has been clearly stated;\\
3. Where I have consulted the published work of others, this is always clearly attributed;\\
4. Where I have quoted from the work of others, the source is always given. With the exception of such quotations, this dissertation is entirely my own work;\\
5. I have acknowledged all main sources of help;\\
6. Where the thesis is based on work done by myself jointly with others, I have made clear exactly what was done by others and what I have contributed myself;\\
7. Either none of this work has been published before submission, or parts of this work have been published by :\\
\\
Stefan Collier\\
April 2016
}
\tableofcontents
\listoffigures
\listoftables

\mainmatter
%% ----------------------------------------------------------------
%\include{Introduction}
%\include{Conclusions}
\include{chapters/1Project/main}
\include{chapters/2Lit/main}
\include{chapters/3Design/HighLevel}
\include{chapters/3Design/InDepth}
\include{chapters/4Impl/main}

\include{chapters/5Experiments/1/main}
\include{chapters/5Experiments/2/main}
\include{chapters/5Experiments/3/main}
\include{chapters/5Experiments/4/main}

\include{chapters/6Conclusion/main}

\appendix
\include{appendix/AppendixB}
\include{appendix/D/main}
\include{appendix/AppendixC}

\backmatter
\bibliographystyle{ecs}
\bibliography{ECS}
\end{document}
%% ----------------------------------------------------------------


 %% ----------------------------------------------------------------
%% Progress.tex
%% ---------------------------------------------------------------- 
\documentclass{ecsprogress}    % Use the progress Style
\graphicspath{{../figs/}}   % Location of your graphics files
    \usepackage{natbib}            % Use Natbib style for the refs.
\hypersetup{colorlinks=true}   % Set to false for black/white printing
\input{Definitions}            % Include your abbreviations



\usepackage{enumitem}% http://ctan.org/pkg/enumitem
\usepackage{multirow}
\usepackage{float}
\usepackage{amsmath}
\usepackage{multicol}
\usepackage{amssymb}
\usepackage[normalem]{ulem}
\useunder{\uline}{\ul}{}
\usepackage{wrapfig}


\usepackage[table,xcdraw]{xcolor}


%% ----------------------------------------------------------------
\begin{document}
\frontmatter
\title      {Heterogeneous Agent-based Model for Supermarket Competition}
\authors    {\texorpdfstring
             {\href{mailto:sc22g13@ecs.soton.ac.uk}{Stefan J. Collier}}
             {Stefan J. Collier}
            }
\addresses  {\groupname\\\deptname\\\univname}
\date       {\today}
\subject    {}
\keywords   {}
\supervisor {Dr. Maria Polukarov}
\examiner   {Professor Sheng Chen}

\maketitle
\begin{abstract}
This project aim was to model and analyse the effects of competitive pricing behaviors of grocery retailers on the British market. 

This was achieved by creating a multi-agent model, containing retailer and consumer agents. The heterogeneous crowd of retailers employs either a uniform pricing strategy or a ‘local price flexing’ strategy. The actions of these retailers are chosen by predicting the profit of each action, using a perceptron. Following on from the consideration of different economic models, a discrete model was developed so that software agents have a discrete environment to operate within. Within the model, it has been observed how supermarkets with differing behaviors affect a heterogeneous crowd of consumer agents. The model was implemented in Java with Python used to evaluate the results. 

The simulation displays good acceptance with real grocery market behavior, i.e. captures the performance of British retailers thus can be used to determine the impact of changes in their behavior on their competitors and consumers.Furthermore it can be used to provide insight into sustainability of volatile pricing strategies, providing a useful insight in volatility of British supermarket retail industry. 
\end{abstract}
\acknowledgements{
I would like to express my sincere gratitude to Dr Maria Polukarov for her guidance and support which provided me the freedom to take this research in the direction of my interest.\\
\\
I would also like to thank my family and friends for their encouragement and support. To those who quietly listened to my software complaints. To those who worked throughout the nights with me. To those who helped me write what I couldn't say. I cannot thank you enough.
}

\declaration{
I, Stefan Collier, declare that this dissertation and the work presented in it are my own and has been generated by me as the result of my own original research.\\
I confirm that:\\
1. This work was done wholly or mainly while in candidature for a degree at this University;\\
2. Where any part of this dissertation has previously been submitted for any other qualification at this University or any other institution, this has been clearly stated;\\
3. Where I have consulted the published work of others, this is always clearly attributed;\\
4. Where I have quoted from the work of others, the source is always given. With the exception of such quotations, this dissertation is entirely my own work;\\
5. I have acknowledged all main sources of help;\\
6. Where the thesis is based on work done by myself jointly with others, I have made clear exactly what was done by others and what I have contributed myself;\\
7. Either none of this work has been published before submission, or parts of this work have been published by :\\
\\
Stefan Collier\\
April 2016
}
\tableofcontents
\listoffigures
\listoftables

\mainmatter
%% ----------------------------------------------------------------
%\include{Introduction}
%\include{Conclusions}
\include{chapters/1Project/main}
\include{chapters/2Lit/main}
\include{chapters/3Design/HighLevel}
\include{chapters/3Design/InDepth}
\include{chapters/4Impl/main}

\include{chapters/5Experiments/1/main}
\include{chapters/5Experiments/2/main}
\include{chapters/5Experiments/3/main}
\include{chapters/5Experiments/4/main}

\include{chapters/6Conclusion/main}

\appendix
\include{appendix/AppendixB}
\include{appendix/D/main}
\include{appendix/AppendixC}

\backmatter
\bibliographystyle{ecs}
\bibliography{ECS}
\end{document}
%% ----------------------------------------------------------------


\appendix
\include{appendix/AppendixB}
 %% ----------------------------------------------------------------
%% Progress.tex
%% ---------------------------------------------------------------- 
\documentclass{ecsprogress}    % Use the progress Style
\graphicspath{{../figs/}}   % Location of your graphics files
    \usepackage{natbib}            % Use Natbib style for the refs.
\hypersetup{colorlinks=true}   % Set to false for black/white printing
\input{Definitions}            % Include your abbreviations



\usepackage{enumitem}% http://ctan.org/pkg/enumitem
\usepackage{multirow}
\usepackage{float}
\usepackage{amsmath}
\usepackage{multicol}
\usepackage{amssymb}
\usepackage[normalem]{ulem}
\useunder{\uline}{\ul}{}
\usepackage{wrapfig}


\usepackage[table,xcdraw]{xcolor}


%% ----------------------------------------------------------------
\begin{document}
\frontmatter
\title      {Heterogeneous Agent-based Model for Supermarket Competition}
\authors    {\texorpdfstring
             {\href{mailto:sc22g13@ecs.soton.ac.uk}{Stefan J. Collier}}
             {Stefan J. Collier}
            }
\addresses  {\groupname\\\deptname\\\univname}
\date       {\today}
\subject    {}
\keywords   {}
\supervisor {Dr. Maria Polukarov}
\examiner   {Professor Sheng Chen}

\maketitle
\begin{abstract}
This project aim was to model and analyse the effects of competitive pricing behaviors of grocery retailers on the British market. 

This was achieved by creating a multi-agent model, containing retailer and consumer agents. The heterogeneous crowd of retailers employs either a uniform pricing strategy or a ‘local price flexing’ strategy. The actions of these retailers are chosen by predicting the profit of each action, using a perceptron. Following on from the consideration of different economic models, a discrete model was developed so that software agents have a discrete environment to operate within. Within the model, it has been observed how supermarkets with differing behaviors affect a heterogeneous crowd of consumer agents. The model was implemented in Java with Python used to evaluate the results. 

The simulation displays good acceptance with real grocery market behavior, i.e. captures the performance of British retailers thus can be used to determine the impact of changes in their behavior on their competitors and consumers.Furthermore it can be used to provide insight into sustainability of volatile pricing strategies, providing a useful insight in volatility of British supermarket retail industry. 
\end{abstract}
\acknowledgements{
I would like to express my sincere gratitude to Dr Maria Polukarov for her guidance and support which provided me the freedom to take this research in the direction of my interest.\\
\\
I would also like to thank my family and friends for their encouragement and support. To those who quietly listened to my software complaints. To those who worked throughout the nights with me. To those who helped me write what I couldn't say. I cannot thank you enough.
}

\declaration{
I, Stefan Collier, declare that this dissertation and the work presented in it are my own and has been generated by me as the result of my own original research.\\
I confirm that:\\
1. This work was done wholly or mainly while in candidature for a degree at this University;\\
2. Where any part of this dissertation has previously been submitted for any other qualification at this University or any other institution, this has been clearly stated;\\
3. Where I have consulted the published work of others, this is always clearly attributed;\\
4. Where I have quoted from the work of others, the source is always given. With the exception of such quotations, this dissertation is entirely my own work;\\
5. I have acknowledged all main sources of help;\\
6. Where the thesis is based on work done by myself jointly with others, I have made clear exactly what was done by others and what I have contributed myself;\\
7. Either none of this work has been published before submission, or parts of this work have been published by :\\
\\
Stefan Collier\\
April 2016
}
\tableofcontents
\listoffigures
\listoftables

\mainmatter
%% ----------------------------------------------------------------
%\include{Introduction}
%\include{Conclusions}
\include{chapters/1Project/main}
\include{chapters/2Lit/main}
\include{chapters/3Design/HighLevel}
\include{chapters/3Design/InDepth}
\include{chapters/4Impl/main}

\include{chapters/5Experiments/1/main}
\include{chapters/5Experiments/2/main}
\include{chapters/5Experiments/3/main}
\include{chapters/5Experiments/4/main}

\include{chapters/6Conclusion/main}

\appendix
\include{appendix/AppendixB}
\include{appendix/D/main}
\include{appendix/AppendixC}

\backmatter
\bibliographystyle{ecs}
\bibliography{ECS}
\end{document}
%% ----------------------------------------------------------------

\include{appendix/AppendixC}

\backmatter
\bibliographystyle{ecs}
\bibliography{ECS}
\end{document}
%% ----------------------------------------------------------------

 %% ----------------------------------------------------------------
%% Progress.tex
%% ---------------------------------------------------------------- 
\documentclass{ecsprogress}    % Use the progress Style
\graphicspath{{../figs/}}   % Location of your graphics files
    \usepackage{natbib}            % Use Natbib style for the refs.
\hypersetup{colorlinks=true}   % Set to false for black/white printing
\input{Definitions}            % Include your abbreviations



\usepackage{enumitem}% http://ctan.org/pkg/enumitem
\usepackage{multirow}
\usepackage{float}
\usepackage{amsmath}
\usepackage{multicol}
\usepackage{amssymb}
\usepackage[normalem]{ulem}
\useunder{\uline}{\ul}{}
\usepackage{wrapfig}


\usepackage[table,xcdraw]{xcolor}


%% ----------------------------------------------------------------
\begin{document}
\frontmatter
\title      {Heterogeneous Agent-based Model for Supermarket Competition}
\authors    {\texorpdfstring
             {\href{mailto:sc22g13@ecs.soton.ac.uk}{Stefan J. Collier}}
             {Stefan J. Collier}
            }
\addresses  {\groupname\\\deptname\\\univname}
\date       {\today}
\subject    {}
\keywords   {}
\supervisor {Dr. Maria Polukarov}
\examiner   {Professor Sheng Chen}

\maketitle
\begin{abstract}
This project aim was to model and analyse the effects of competitive pricing behaviors of grocery retailers on the British market. 

This was achieved by creating a multi-agent model, containing retailer and consumer agents. The heterogeneous crowd of retailers employs either a uniform pricing strategy or a ‘local price flexing’ strategy. The actions of these retailers are chosen by predicting the profit of each action, using a perceptron. Following on from the consideration of different economic models, a discrete model was developed so that software agents have a discrete environment to operate within. Within the model, it has been observed how supermarkets with differing behaviors affect a heterogeneous crowd of consumer agents. The model was implemented in Java with Python used to evaluate the results. 

The simulation displays good acceptance with real grocery market behavior, i.e. captures the performance of British retailers thus can be used to determine the impact of changes in their behavior on their competitors and consumers.Furthermore it can be used to provide insight into sustainability of volatile pricing strategies, providing a useful insight in volatility of British supermarket retail industry. 
\end{abstract}
\acknowledgements{
I would like to express my sincere gratitude to Dr Maria Polukarov for her guidance and support which provided me the freedom to take this research in the direction of my interest.\\
\\
I would also like to thank my family and friends for their encouragement and support. To those who quietly listened to my software complaints. To those who worked throughout the nights with me. To those who helped me write what I couldn't say. I cannot thank you enough.
}

\declaration{
I, Stefan Collier, declare that this dissertation and the work presented in it are my own and has been generated by me as the result of my own original research.\\
I confirm that:\\
1. This work was done wholly or mainly while in candidature for a degree at this University;\\
2. Where any part of this dissertation has previously been submitted for any other qualification at this University or any other institution, this has been clearly stated;\\
3. Where I have consulted the published work of others, this is always clearly attributed;\\
4. Where I have quoted from the work of others, the source is always given. With the exception of such quotations, this dissertation is entirely my own work;\\
5. I have acknowledged all main sources of help;\\
6. Where the thesis is based on work done by myself jointly with others, I have made clear exactly what was done by others and what I have contributed myself;\\
7. Either none of this work has been published before submission, or parts of this work have been published by :\\
\\
Stefan Collier\\
April 2016
}
\tableofcontents
\listoffigures
\listoftables

\mainmatter
%% ----------------------------------------------------------------
%\include{Introduction}
%\include{Conclusions}
 %% ----------------------------------------------------------------
%% Progress.tex
%% ---------------------------------------------------------------- 
\documentclass{ecsprogress}    % Use the progress Style
\graphicspath{{../figs/}}   % Location of your graphics files
    \usepackage{natbib}            % Use Natbib style for the refs.
\hypersetup{colorlinks=true}   % Set to false for black/white printing
\input{Definitions}            % Include your abbreviations



\usepackage{enumitem}% http://ctan.org/pkg/enumitem
\usepackage{multirow}
\usepackage{float}
\usepackage{amsmath}
\usepackage{multicol}
\usepackage{amssymb}
\usepackage[normalem]{ulem}
\useunder{\uline}{\ul}{}
\usepackage{wrapfig}


\usepackage[table,xcdraw]{xcolor}


%% ----------------------------------------------------------------
\begin{document}
\frontmatter
\title      {Heterogeneous Agent-based Model for Supermarket Competition}
\authors    {\texorpdfstring
             {\href{mailto:sc22g13@ecs.soton.ac.uk}{Stefan J. Collier}}
             {Stefan J. Collier}
            }
\addresses  {\groupname\\\deptname\\\univname}
\date       {\today}
\subject    {}
\keywords   {}
\supervisor {Dr. Maria Polukarov}
\examiner   {Professor Sheng Chen}

\maketitle
\begin{abstract}
This project aim was to model and analyse the effects of competitive pricing behaviors of grocery retailers on the British market. 

This was achieved by creating a multi-agent model, containing retailer and consumer agents. The heterogeneous crowd of retailers employs either a uniform pricing strategy or a ‘local price flexing’ strategy. The actions of these retailers are chosen by predicting the profit of each action, using a perceptron. Following on from the consideration of different economic models, a discrete model was developed so that software agents have a discrete environment to operate within. Within the model, it has been observed how supermarkets with differing behaviors affect a heterogeneous crowd of consumer agents. The model was implemented in Java with Python used to evaluate the results. 

The simulation displays good acceptance with real grocery market behavior, i.e. captures the performance of British retailers thus can be used to determine the impact of changes in their behavior on their competitors and consumers.Furthermore it can be used to provide insight into sustainability of volatile pricing strategies, providing a useful insight in volatility of British supermarket retail industry. 
\end{abstract}
\acknowledgements{
I would like to express my sincere gratitude to Dr Maria Polukarov for her guidance and support which provided me the freedom to take this research in the direction of my interest.\\
\\
I would also like to thank my family and friends for their encouragement and support. To those who quietly listened to my software complaints. To those who worked throughout the nights with me. To those who helped me write what I couldn't say. I cannot thank you enough.
}

\declaration{
I, Stefan Collier, declare that this dissertation and the work presented in it are my own and has been generated by me as the result of my own original research.\\
I confirm that:\\
1. This work was done wholly or mainly while in candidature for a degree at this University;\\
2. Where any part of this dissertation has previously been submitted for any other qualification at this University or any other institution, this has been clearly stated;\\
3. Where I have consulted the published work of others, this is always clearly attributed;\\
4. Where I have quoted from the work of others, the source is always given. With the exception of such quotations, this dissertation is entirely my own work;\\
5. I have acknowledged all main sources of help;\\
6. Where the thesis is based on work done by myself jointly with others, I have made clear exactly what was done by others and what I have contributed myself;\\
7. Either none of this work has been published before submission, or parts of this work have been published by :\\
\\
Stefan Collier\\
April 2016
}
\tableofcontents
\listoffigures
\listoftables

\mainmatter
%% ----------------------------------------------------------------
%\include{Introduction}
%\include{Conclusions}
\include{chapters/1Project/main}
\include{chapters/2Lit/main}
\include{chapters/3Design/HighLevel}
\include{chapters/3Design/InDepth}
\include{chapters/4Impl/main}

\include{chapters/5Experiments/1/main}
\include{chapters/5Experiments/2/main}
\include{chapters/5Experiments/3/main}
\include{chapters/5Experiments/4/main}

\include{chapters/6Conclusion/main}

\appendix
\include{appendix/AppendixB}
\include{appendix/D/main}
\include{appendix/AppendixC}

\backmatter
\bibliographystyle{ecs}
\bibliography{ECS}
\end{document}
%% ----------------------------------------------------------------

 %% ----------------------------------------------------------------
%% Progress.tex
%% ---------------------------------------------------------------- 
\documentclass{ecsprogress}    % Use the progress Style
\graphicspath{{../figs/}}   % Location of your graphics files
    \usepackage{natbib}            % Use Natbib style for the refs.
\hypersetup{colorlinks=true}   % Set to false for black/white printing
\input{Definitions}            % Include your abbreviations



\usepackage{enumitem}% http://ctan.org/pkg/enumitem
\usepackage{multirow}
\usepackage{float}
\usepackage{amsmath}
\usepackage{multicol}
\usepackage{amssymb}
\usepackage[normalem]{ulem}
\useunder{\uline}{\ul}{}
\usepackage{wrapfig}


\usepackage[table,xcdraw]{xcolor}


%% ----------------------------------------------------------------
\begin{document}
\frontmatter
\title      {Heterogeneous Agent-based Model for Supermarket Competition}
\authors    {\texorpdfstring
             {\href{mailto:sc22g13@ecs.soton.ac.uk}{Stefan J. Collier}}
             {Stefan J. Collier}
            }
\addresses  {\groupname\\\deptname\\\univname}
\date       {\today}
\subject    {}
\keywords   {}
\supervisor {Dr. Maria Polukarov}
\examiner   {Professor Sheng Chen}

\maketitle
\begin{abstract}
This project aim was to model and analyse the effects of competitive pricing behaviors of grocery retailers on the British market. 

This was achieved by creating a multi-agent model, containing retailer and consumer agents. The heterogeneous crowd of retailers employs either a uniform pricing strategy or a ‘local price flexing’ strategy. The actions of these retailers are chosen by predicting the profit of each action, using a perceptron. Following on from the consideration of different economic models, a discrete model was developed so that software agents have a discrete environment to operate within. Within the model, it has been observed how supermarkets with differing behaviors affect a heterogeneous crowd of consumer agents. The model was implemented in Java with Python used to evaluate the results. 

The simulation displays good acceptance with real grocery market behavior, i.e. captures the performance of British retailers thus can be used to determine the impact of changes in their behavior on their competitors and consumers.Furthermore it can be used to provide insight into sustainability of volatile pricing strategies, providing a useful insight in volatility of British supermarket retail industry. 
\end{abstract}
\acknowledgements{
I would like to express my sincere gratitude to Dr Maria Polukarov for her guidance and support which provided me the freedom to take this research in the direction of my interest.\\
\\
I would also like to thank my family and friends for their encouragement and support. To those who quietly listened to my software complaints. To those who worked throughout the nights with me. To those who helped me write what I couldn't say. I cannot thank you enough.
}

\declaration{
I, Stefan Collier, declare that this dissertation and the work presented in it are my own and has been generated by me as the result of my own original research.\\
I confirm that:\\
1. This work was done wholly or mainly while in candidature for a degree at this University;\\
2. Where any part of this dissertation has previously been submitted for any other qualification at this University or any other institution, this has been clearly stated;\\
3. Where I have consulted the published work of others, this is always clearly attributed;\\
4. Where I have quoted from the work of others, the source is always given. With the exception of such quotations, this dissertation is entirely my own work;\\
5. I have acknowledged all main sources of help;\\
6. Where the thesis is based on work done by myself jointly with others, I have made clear exactly what was done by others and what I have contributed myself;\\
7. Either none of this work has been published before submission, or parts of this work have been published by :\\
\\
Stefan Collier\\
April 2016
}
\tableofcontents
\listoffigures
\listoftables

\mainmatter
%% ----------------------------------------------------------------
%\include{Introduction}
%\include{Conclusions}
\include{chapters/1Project/main}
\include{chapters/2Lit/main}
\include{chapters/3Design/HighLevel}
\include{chapters/3Design/InDepth}
\include{chapters/4Impl/main}

\include{chapters/5Experiments/1/main}
\include{chapters/5Experiments/2/main}
\include{chapters/5Experiments/3/main}
\include{chapters/5Experiments/4/main}

\include{chapters/6Conclusion/main}

\appendix
\include{appendix/AppendixB}
\include{appendix/D/main}
\include{appendix/AppendixC}

\backmatter
\bibliographystyle{ecs}
\bibliography{ECS}
\end{document}
%% ----------------------------------------------------------------

\include{chapters/3Design/HighLevel}
\include{chapters/3Design/InDepth}
 %% ----------------------------------------------------------------
%% Progress.tex
%% ---------------------------------------------------------------- 
\documentclass{ecsprogress}    % Use the progress Style
\graphicspath{{../figs/}}   % Location of your graphics files
    \usepackage{natbib}            % Use Natbib style for the refs.
\hypersetup{colorlinks=true}   % Set to false for black/white printing
\input{Definitions}            % Include your abbreviations



\usepackage{enumitem}% http://ctan.org/pkg/enumitem
\usepackage{multirow}
\usepackage{float}
\usepackage{amsmath}
\usepackage{multicol}
\usepackage{amssymb}
\usepackage[normalem]{ulem}
\useunder{\uline}{\ul}{}
\usepackage{wrapfig}


\usepackage[table,xcdraw]{xcolor}


%% ----------------------------------------------------------------
\begin{document}
\frontmatter
\title      {Heterogeneous Agent-based Model for Supermarket Competition}
\authors    {\texorpdfstring
             {\href{mailto:sc22g13@ecs.soton.ac.uk}{Stefan J. Collier}}
             {Stefan J. Collier}
            }
\addresses  {\groupname\\\deptname\\\univname}
\date       {\today}
\subject    {}
\keywords   {}
\supervisor {Dr. Maria Polukarov}
\examiner   {Professor Sheng Chen}

\maketitle
\begin{abstract}
This project aim was to model and analyse the effects of competitive pricing behaviors of grocery retailers on the British market. 

This was achieved by creating a multi-agent model, containing retailer and consumer agents. The heterogeneous crowd of retailers employs either a uniform pricing strategy or a ‘local price flexing’ strategy. The actions of these retailers are chosen by predicting the profit of each action, using a perceptron. Following on from the consideration of different economic models, a discrete model was developed so that software agents have a discrete environment to operate within. Within the model, it has been observed how supermarkets with differing behaviors affect a heterogeneous crowd of consumer agents. The model was implemented in Java with Python used to evaluate the results. 

The simulation displays good acceptance with real grocery market behavior, i.e. captures the performance of British retailers thus can be used to determine the impact of changes in their behavior on their competitors and consumers.Furthermore it can be used to provide insight into sustainability of volatile pricing strategies, providing a useful insight in volatility of British supermarket retail industry. 
\end{abstract}
\acknowledgements{
I would like to express my sincere gratitude to Dr Maria Polukarov for her guidance and support which provided me the freedom to take this research in the direction of my interest.\\
\\
I would also like to thank my family and friends for their encouragement and support. To those who quietly listened to my software complaints. To those who worked throughout the nights with me. To those who helped me write what I couldn't say. I cannot thank you enough.
}

\declaration{
I, Stefan Collier, declare that this dissertation and the work presented in it are my own and has been generated by me as the result of my own original research.\\
I confirm that:\\
1. This work was done wholly or mainly while in candidature for a degree at this University;\\
2. Where any part of this dissertation has previously been submitted for any other qualification at this University or any other institution, this has been clearly stated;\\
3. Where I have consulted the published work of others, this is always clearly attributed;\\
4. Where I have quoted from the work of others, the source is always given. With the exception of such quotations, this dissertation is entirely my own work;\\
5. I have acknowledged all main sources of help;\\
6. Where the thesis is based on work done by myself jointly with others, I have made clear exactly what was done by others and what I have contributed myself;\\
7. Either none of this work has been published before submission, or parts of this work have been published by :\\
\\
Stefan Collier\\
April 2016
}
\tableofcontents
\listoffigures
\listoftables

\mainmatter
%% ----------------------------------------------------------------
%\include{Introduction}
%\include{Conclusions}
\include{chapters/1Project/main}
\include{chapters/2Lit/main}
\include{chapters/3Design/HighLevel}
\include{chapters/3Design/InDepth}
\include{chapters/4Impl/main}

\include{chapters/5Experiments/1/main}
\include{chapters/5Experiments/2/main}
\include{chapters/5Experiments/3/main}
\include{chapters/5Experiments/4/main}

\include{chapters/6Conclusion/main}

\appendix
\include{appendix/AppendixB}
\include{appendix/D/main}
\include{appendix/AppendixC}

\backmatter
\bibliographystyle{ecs}
\bibliography{ECS}
\end{document}
%% ----------------------------------------------------------------


 %% ----------------------------------------------------------------
%% Progress.tex
%% ---------------------------------------------------------------- 
\documentclass{ecsprogress}    % Use the progress Style
\graphicspath{{../figs/}}   % Location of your graphics files
    \usepackage{natbib}            % Use Natbib style for the refs.
\hypersetup{colorlinks=true}   % Set to false for black/white printing
\input{Definitions}            % Include your abbreviations



\usepackage{enumitem}% http://ctan.org/pkg/enumitem
\usepackage{multirow}
\usepackage{float}
\usepackage{amsmath}
\usepackage{multicol}
\usepackage{amssymb}
\usepackage[normalem]{ulem}
\useunder{\uline}{\ul}{}
\usepackage{wrapfig}


\usepackage[table,xcdraw]{xcolor}


%% ----------------------------------------------------------------
\begin{document}
\frontmatter
\title      {Heterogeneous Agent-based Model for Supermarket Competition}
\authors    {\texorpdfstring
             {\href{mailto:sc22g13@ecs.soton.ac.uk}{Stefan J. Collier}}
             {Stefan J. Collier}
            }
\addresses  {\groupname\\\deptname\\\univname}
\date       {\today}
\subject    {}
\keywords   {}
\supervisor {Dr. Maria Polukarov}
\examiner   {Professor Sheng Chen}

\maketitle
\begin{abstract}
This project aim was to model and analyse the effects of competitive pricing behaviors of grocery retailers on the British market. 

This was achieved by creating a multi-agent model, containing retailer and consumer agents. The heterogeneous crowd of retailers employs either a uniform pricing strategy or a ‘local price flexing’ strategy. The actions of these retailers are chosen by predicting the profit of each action, using a perceptron. Following on from the consideration of different economic models, a discrete model was developed so that software agents have a discrete environment to operate within. Within the model, it has been observed how supermarkets with differing behaviors affect a heterogeneous crowd of consumer agents. The model was implemented in Java with Python used to evaluate the results. 

The simulation displays good acceptance with real grocery market behavior, i.e. captures the performance of British retailers thus can be used to determine the impact of changes in their behavior on their competitors and consumers.Furthermore it can be used to provide insight into sustainability of volatile pricing strategies, providing a useful insight in volatility of British supermarket retail industry. 
\end{abstract}
\acknowledgements{
I would like to express my sincere gratitude to Dr Maria Polukarov for her guidance and support which provided me the freedom to take this research in the direction of my interest.\\
\\
I would also like to thank my family and friends for their encouragement and support. To those who quietly listened to my software complaints. To those who worked throughout the nights with me. To those who helped me write what I couldn't say. I cannot thank you enough.
}

\declaration{
I, Stefan Collier, declare that this dissertation and the work presented in it are my own and has been generated by me as the result of my own original research.\\
I confirm that:\\
1. This work was done wholly or mainly while in candidature for a degree at this University;\\
2. Where any part of this dissertation has previously been submitted for any other qualification at this University or any other institution, this has been clearly stated;\\
3. Where I have consulted the published work of others, this is always clearly attributed;\\
4. Where I have quoted from the work of others, the source is always given. With the exception of such quotations, this dissertation is entirely my own work;\\
5. I have acknowledged all main sources of help;\\
6. Where the thesis is based on work done by myself jointly with others, I have made clear exactly what was done by others and what I have contributed myself;\\
7. Either none of this work has been published before submission, or parts of this work have been published by :\\
\\
Stefan Collier\\
April 2016
}
\tableofcontents
\listoffigures
\listoftables

\mainmatter
%% ----------------------------------------------------------------
%\include{Introduction}
%\include{Conclusions}
\include{chapters/1Project/main}
\include{chapters/2Lit/main}
\include{chapters/3Design/HighLevel}
\include{chapters/3Design/InDepth}
\include{chapters/4Impl/main}

\include{chapters/5Experiments/1/main}
\include{chapters/5Experiments/2/main}
\include{chapters/5Experiments/3/main}
\include{chapters/5Experiments/4/main}

\include{chapters/6Conclusion/main}

\appendix
\include{appendix/AppendixB}
\include{appendix/D/main}
\include{appendix/AppendixC}

\backmatter
\bibliographystyle{ecs}
\bibliography{ECS}
\end{document}
%% ----------------------------------------------------------------

 %% ----------------------------------------------------------------
%% Progress.tex
%% ---------------------------------------------------------------- 
\documentclass{ecsprogress}    % Use the progress Style
\graphicspath{{../figs/}}   % Location of your graphics files
    \usepackage{natbib}            % Use Natbib style for the refs.
\hypersetup{colorlinks=true}   % Set to false for black/white printing
\input{Definitions}            % Include your abbreviations



\usepackage{enumitem}% http://ctan.org/pkg/enumitem
\usepackage{multirow}
\usepackage{float}
\usepackage{amsmath}
\usepackage{multicol}
\usepackage{amssymb}
\usepackage[normalem]{ulem}
\useunder{\uline}{\ul}{}
\usepackage{wrapfig}


\usepackage[table,xcdraw]{xcolor}


%% ----------------------------------------------------------------
\begin{document}
\frontmatter
\title      {Heterogeneous Agent-based Model for Supermarket Competition}
\authors    {\texorpdfstring
             {\href{mailto:sc22g13@ecs.soton.ac.uk}{Stefan J. Collier}}
             {Stefan J. Collier}
            }
\addresses  {\groupname\\\deptname\\\univname}
\date       {\today}
\subject    {}
\keywords   {}
\supervisor {Dr. Maria Polukarov}
\examiner   {Professor Sheng Chen}

\maketitle
\begin{abstract}
This project aim was to model and analyse the effects of competitive pricing behaviors of grocery retailers on the British market. 

This was achieved by creating a multi-agent model, containing retailer and consumer agents. The heterogeneous crowd of retailers employs either a uniform pricing strategy or a ‘local price flexing’ strategy. The actions of these retailers are chosen by predicting the profit of each action, using a perceptron. Following on from the consideration of different economic models, a discrete model was developed so that software agents have a discrete environment to operate within. Within the model, it has been observed how supermarkets with differing behaviors affect a heterogeneous crowd of consumer agents. The model was implemented in Java with Python used to evaluate the results. 

The simulation displays good acceptance with real grocery market behavior, i.e. captures the performance of British retailers thus can be used to determine the impact of changes in their behavior on their competitors and consumers.Furthermore it can be used to provide insight into sustainability of volatile pricing strategies, providing a useful insight in volatility of British supermarket retail industry. 
\end{abstract}
\acknowledgements{
I would like to express my sincere gratitude to Dr Maria Polukarov for her guidance and support which provided me the freedom to take this research in the direction of my interest.\\
\\
I would also like to thank my family and friends for their encouragement and support. To those who quietly listened to my software complaints. To those who worked throughout the nights with me. To those who helped me write what I couldn't say. I cannot thank you enough.
}

\declaration{
I, Stefan Collier, declare that this dissertation and the work presented in it are my own and has been generated by me as the result of my own original research.\\
I confirm that:\\
1. This work was done wholly or mainly while in candidature for a degree at this University;\\
2. Where any part of this dissertation has previously been submitted for any other qualification at this University or any other institution, this has been clearly stated;\\
3. Where I have consulted the published work of others, this is always clearly attributed;\\
4. Where I have quoted from the work of others, the source is always given. With the exception of such quotations, this dissertation is entirely my own work;\\
5. I have acknowledged all main sources of help;\\
6. Where the thesis is based on work done by myself jointly with others, I have made clear exactly what was done by others and what I have contributed myself;\\
7. Either none of this work has been published before submission, or parts of this work have been published by :\\
\\
Stefan Collier\\
April 2016
}
\tableofcontents
\listoffigures
\listoftables

\mainmatter
%% ----------------------------------------------------------------
%\include{Introduction}
%\include{Conclusions}
\include{chapters/1Project/main}
\include{chapters/2Lit/main}
\include{chapters/3Design/HighLevel}
\include{chapters/3Design/InDepth}
\include{chapters/4Impl/main}

\include{chapters/5Experiments/1/main}
\include{chapters/5Experiments/2/main}
\include{chapters/5Experiments/3/main}
\include{chapters/5Experiments/4/main}

\include{chapters/6Conclusion/main}

\appendix
\include{appendix/AppendixB}
\include{appendix/D/main}
\include{appendix/AppendixC}

\backmatter
\bibliographystyle{ecs}
\bibliography{ECS}
\end{document}
%% ----------------------------------------------------------------

 %% ----------------------------------------------------------------
%% Progress.tex
%% ---------------------------------------------------------------- 
\documentclass{ecsprogress}    % Use the progress Style
\graphicspath{{../figs/}}   % Location of your graphics files
    \usepackage{natbib}            % Use Natbib style for the refs.
\hypersetup{colorlinks=true}   % Set to false for black/white printing
\input{Definitions}            % Include your abbreviations



\usepackage{enumitem}% http://ctan.org/pkg/enumitem
\usepackage{multirow}
\usepackage{float}
\usepackage{amsmath}
\usepackage{multicol}
\usepackage{amssymb}
\usepackage[normalem]{ulem}
\useunder{\uline}{\ul}{}
\usepackage{wrapfig}


\usepackage[table,xcdraw]{xcolor}


%% ----------------------------------------------------------------
\begin{document}
\frontmatter
\title      {Heterogeneous Agent-based Model for Supermarket Competition}
\authors    {\texorpdfstring
             {\href{mailto:sc22g13@ecs.soton.ac.uk}{Stefan J. Collier}}
             {Stefan J. Collier}
            }
\addresses  {\groupname\\\deptname\\\univname}
\date       {\today}
\subject    {}
\keywords   {}
\supervisor {Dr. Maria Polukarov}
\examiner   {Professor Sheng Chen}

\maketitle
\begin{abstract}
This project aim was to model and analyse the effects of competitive pricing behaviors of grocery retailers on the British market. 

This was achieved by creating a multi-agent model, containing retailer and consumer agents. The heterogeneous crowd of retailers employs either a uniform pricing strategy or a ‘local price flexing’ strategy. The actions of these retailers are chosen by predicting the profit of each action, using a perceptron. Following on from the consideration of different economic models, a discrete model was developed so that software agents have a discrete environment to operate within. Within the model, it has been observed how supermarkets with differing behaviors affect a heterogeneous crowd of consumer agents. The model was implemented in Java with Python used to evaluate the results. 

The simulation displays good acceptance with real grocery market behavior, i.e. captures the performance of British retailers thus can be used to determine the impact of changes in their behavior on their competitors and consumers.Furthermore it can be used to provide insight into sustainability of volatile pricing strategies, providing a useful insight in volatility of British supermarket retail industry. 
\end{abstract}
\acknowledgements{
I would like to express my sincere gratitude to Dr Maria Polukarov for her guidance and support which provided me the freedom to take this research in the direction of my interest.\\
\\
I would also like to thank my family and friends for their encouragement and support. To those who quietly listened to my software complaints. To those who worked throughout the nights with me. To those who helped me write what I couldn't say. I cannot thank you enough.
}

\declaration{
I, Stefan Collier, declare that this dissertation and the work presented in it are my own and has been generated by me as the result of my own original research.\\
I confirm that:\\
1. This work was done wholly or mainly while in candidature for a degree at this University;\\
2. Where any part of this dissertation has previously been submitted for any other qualification at this University or any other institution, this has been clearly stated;\\
3. Where I have consulted the published work of others, this is always clearly attributed;\\
4. Where I have quoted from the work of others, the source is always given. With the exception of such quotations, this dissertation is entirely my own work;\\
5. I have acknowledged all main sources of help;\\
6. Where the thesis is based on work done by myself jointly with others, I have made clear exactly what was done by others and what I have contributed myself;\\
7. Either none of this work has been published before submission, or parts of this work have been published by :\\
\\
Stefan Collier\\
April 2016
}
\tableofcontents
\listoffigures
\listoftables

\mainmatter
%% ----------------------------------------------------------------
%\include{Introduction}
%\include{Conclusions}
\include{chapters/1Project/main}
\include{chapters/2Lit/main}
\include{chapters/3Design/HighLevel}
\include{chapters/3Design/InDepth}
\include{chapters/4Impl/main}

\include{chapters/5Experiments/1/main}
\include{chapters/5Experiments/2/main}
\include{chapters/5Experiments/3/main}
\include{chapters/5Experiments/4/main}

\include{chapters/6Conclusion/main}

\appendix
\include{appendix/AppendixB}
\include{appendix/D/main}
\include{appendix/AppendixC}

\backmatter
\bibliographystyle{ecs}
\bibliography{ECS}
\end{document}
%% ----------------------------------------------------------------

 %% ----------------------------------------------------------------
%% Progress.tex
%% ---------------------------------------------------------------- 
\documentclass{ecsprogress}    % Use the progress Style
\graphicspath{{../figs/}}   % Location of your graphics files
    \usepackage{natbib}            % Use Natbib style for the refs.
\hypersetup{colorlinks=true}   % Set to false for black/white printing
\input{Definitions}            % Include your abbreviations



\usepackage{enumitem}% http://ctan.org/pkg/enumitem
\usepackage{multirow}
\usepackage{float}
\usepackage{amsmath}
\usepackage{multicol}
\usepackage{amssymb}
\usepackage[normalem]{ulem}
\useunder{\uline}{\ul}{}
\usepackage{wrapfig}


\usepackage[table,xcdraw]{xcolor}


%% ----------------------------------------------------------------
\begin{document}
\frontmatter
\title      {Heterogeneous Agent-based Model for Supermarket Competition}
\authors    {\texorpdfstring
             {\href{mailto:sc22g13@ecs.soton.ac.uk}{Stefan J. Collier}}
             {Stefan J. Collier}
            }
\addresses  {\groupname\\\deptname\\\univname}
\date       {\today}
\subject    {}
\keywords   {}
\supervisor {Dr. Maria Polukarov}
\examiner   {Professor Sheng Chen}

\maketitle
\begin{abstract}
This project aim was to model and analyse the effects of competitive pricing behaviors of grocery retailers on the British market. 

This was achieved by creating a multi-agent model, containing retailer and consumer agents. The heterogeneous crowd of retailers employs either a uniform pricing strategy or a ‘local price flexing’ strategy. The actions of these retailers are chosen by predicting the profit of each action, using a perceptron. Following on from the consideration of different economic models, a discrete model was developed so that software agents have a discrete environment to operate within. Within the model, it has been observed how supermarkets with differing behaviors affect a heterogeneous crowd of consumer agents. The model was implemented in Java with Python used to evaluate the results. 

The simulation displays good acceptance with real grocery market behavior, i.e. captures the performance of British retailers thus can be used to determine the impact of changes in their behavior on their competitors and consumers.Furthermore it can be used to provide insight into sustainability of volatile pricing strategies, providing a useful insight in volatility of British supermarket retail industry. 
\end{abstract}
\acknowledgements{
I would like to express my sincere gratitude to Dr Maria Polukarov for her guidance and support which provided me the freedom to take this research in the direction of my interest.\\
\\
I would also like to thank my family and friends for their encouragement and support. To those who quietly listened to my software complaints. To those who worked throughout the nights with me. To those who helped me write what I couldn't say. I cannot thank you enough.
}

\declaration{
I, Stefan Collier, declare that this dissertation and the work presented in it are my own and has been generated by me as the result of my own original research.\\
I confirm that:\\
1. This work was done wholly or mainly while in candidature for a degree at this University;\\
2. Where any part of this dissertation has previously been submitted for any other qualification at this University or any other institution, this has been clearly stated;\\
3. Where I have consulted the published work of others, this is always clearly attributed;\\
4. Where I have quoted from the work of others, the source is always given. With the exception of such quotations, this dissertation is entirely my own work;\\
5. I have acknowledged all main sources of help;\\
6. Where the thesis is based on work done by myself jointly with others, I have made clear exactly what was done by others and what I have contributed myself;\\
7. Either none of this work has been published before submission, or parts of this work have been published by :\\
\\
Stefan Collier\\
April 2016
}
\tableofcontents
\listoffigures
\listoftables

\mainmatter
%% ----------------------------------------------------------------
%\include{Introduction}
%\include{Conclusions}
\include{chapters/1Project/main}
\include{chapters/2Lit/main}
\include{chapters/3Design/HighLevel}
\include{chapters/3Design/InDepth}
\include{chapters/4Impl/main}

\include{chapters/5Experiments/1/main}
\include{chapters/5Experiments/2/main}
\include{chapters/5Experiments/3/main}
\include{chapters/5Experiments/4/main}

\include{chapters/6Conclusion/main}

\appendix
\include{appendix/AppendixB}
\include{appendix/D/main}
\include{appendix/AppendixC}

\backmatter
\bibliographystyle{ecs}
\bibliography{ECS}
\end{document}
%% ----------------------------------------------------------------


 %% ----------------------------------------------------------------
%% Progress.tex
%% ---------------------------------------------------------------- 
\documentclass{ecsprogress}    % Use the progress Style
\graphicspath{{../figs/}}   % Location of your graphics files
    \usepackage{natbib}            % Use Natbib style for the refs.
\hypersetup{colorlinks=true}   % Set to false for black/white printing
\input{Definitions}            % Include your abbreviations



\usepackage{enumitem}% http://ctan.org/pkg/enumitem
\usepackage{multirow}
\usepackage{float}
\usepackage{amsmath}
\usepackage{multicol}
\usepackage{amssymb}
\usepackage[normalem]{ulem}
\useunder{\uline}{\ul}{}
\usepackage{wrapfig}


\usepackage[table,xcdraw]{xcolor}


%% ----------------------------------------------------------------
\begin{document}
\frontmatter
\title      {Heterogeneous Agent-based Model for Supermarket Competition}
\authors    {\texorpdfstring
             {\href{mailto:sc22g13@ecs.soton.ac.uk}{Stefan J. Collier}}
             {Stefan J. Collier}
            }
\addresses  {\groupname\\\deptname\\\univname}
\date       {\today}
\subject    {}
\keywords   {}
\supervisor {Dr. Maria Polukarov}
\examiner   {Professor Sheng Chen}

\maketitle
\begin{abstract}
This project aim was to model and analyse the effects of competitive pricing behaviors of grocery retailers on the British market. 

This was achieved by creating a multi-agent model, containing retailer and consumer agents. The heterogeneous crowd of retailers employs either a uniform pricing strategy or a ‘local price flexing’ strategy. The actions of these retailers are chosen by predicting the profit of each action, using a perceptron. Following on from the consideration of different economic models, a discrete model was developed so that software agents have a discrete environment to operate within. Within the model, it has been observed how supermarkets with differing behaviors affect a heterogeneous crowd of consumer agents. The model was implemented in Java with Python used to evaluate the results. 

The simulation displays good acceptance with real grocery market behavior, i.e. captures the performance of British retailers thus can be used to determine the impact of changes in their behavior on their competitors and consumers.Furthermore it can be used to provide insight into sustainability of volatile pricing strategies, providing a useful insight in volatility of British supermarket retail industry. 
\end{abstract}
\acknowledgements{
I would like to express my sincere gratitude to Dr Maria Polukarov for her guidance and support which provided me the freedom to take this research in the direction of my interest.\\
\\
I would also like to thank my family and friends for their encouragement and support. To those who quietly listened to my software complaints. To those who worked throughout the nights with me. To those who helped me write what I couldn't say. I cannot thank you enough.
}

\declaration{
I, Stefan Collier, declare that this dissertation and the work presented in it are my own and has been generated by me as the result of my own original research.\\
I confirm that:\\
1. This work was done wholly or mainly while in candidature for a degree at this University;\\
2. Where any part of this dissertation has previously been submitted for any other qualification at this University or any other institution, this has been clearly stated;\\
3. Where I have consulted the published work of others, this is always clearly attributed;\\
4. Where I have quoted from the work of others, the source is always given. With the exception of such quotations, this dissertation is entirely my own work;\\
5. I have acknowledged all main sources of help;\\
6. Where the thesis is based on work done by myself jointly with others, I have made clear exactly what was done by others and what I have contributed myself;\\
7. Either none of this work has been published before submission, or parts of this work have been published by :\\
\\
Stefan Collier\\
April 2016
}
\tableofcontents
\listoffigures
\listoftables

\mainmatter
%% ----------------------------------------------------------------
%\include{Introduction}
%\include{Conclusions}
\include{chapters/1Project/main}
\include{chapters/2Lit/main}
\include{chapters/3Design/HighLevel}
\include{chapters/3Design/InDepth}
\include{chapters/4Impl/main}

\include{chapters/5Experiments/1/main}
\include{chapters/5Experiments/2/main}
\include{chapters/5Experiments/3/main}
\include{chapters/5Experiments/4/main}

\include{chapters/6Conclusion/main}

\appendix
\include{appendix/AppendixB}
\include{appendix/D/main}
\include{appendix/AppendixC}

\backmatter
\bibliographystyle{ecs}
\bibliography{ECS}
\end{document}
%% ----------------------------------------------------------------


\appendix
\include{appendix/AppendixB}
 %% ----------------------------------------------------------------
%% Progress.tex
%% ---------------------------------------------------------------- 
\documentclass{ecsprogress}    % Use the progress Style
\graphicspath{{../figs/}}   % Location of your graphics files
    \usepackage{natbib}            % Use Natbib style for the refs.
\hypersetup{colorlinks=true}   % Set to false for black/white printing
\input{Definitions}            % Include your abbreviations



\usepackage{enumitem}% http://ctan.org/pkg/enumitem
\usepackage{multirow}
\usepackage{float}
\usepackage{amsmath}
\usepackage{multicol}
\usepackage{amssymb}
\usepackage[normalem]{ulem}
\useunder{\uline}{\ul}{}
\usepackage{wrapfig}


\usepackage[table,xcdraw]{xcolor}


%% ----------------------------------------------------------------
\begin{document}
\frontmatter
\title      {Heterogeneous Agent-based Model for Supermarket Competition}
\authors    {\texorpdfstring
             {\href{mailto:sc22g13@ecs.soton.ac.uk}{Stefan J. Collier}}
             {Stefan J. Collier}
            }
\addresses  {\groupname\\\deptname\\\univname}
\date       {\today}
\subject    {}
\keywords   {}
\supervisor {Dr. Maria Polukarov}
\examiner   {Professor Sheng Chen}

\maketitle
\begin{abstract}
This project aim was to model and analyse the effects of competitive pricing behaviors of grocery retailers on the British market. 

This was achieved by creating a multi-agent model, containing retailer and consumer agents. The heterogeneous crowd of retailers employs either a uniform pricing strategy or a ‘local price flexing’ strategy. The actions of these retailers are chosen by predicting the profit of each action, using a perceptron. Following on from the consideration of different economic models, a discrete model was developed so that software agents have a discrete environment to operate within. Within the model, it has been observed how supermarkets with differing behaviors affect a heterogeneous crowd of consumer agents. The model was implemented in Java with Python used to evaluate the results. 

The simulation displays good acceptance with real grocery market behavior, i.e. captures the performance of British retailers thus can be used to determine the impact of changes in their behavior on their competitors and consumers.Furthermore it can be used to provide insight into sustainability of volatile pricing strategies, providing a useful insight in volatility of British supermarket retail industry. 
\end{abstract}
\acknowledgements{
I would like to express my sincere gratitude to Dr Maria Polukarov for her guidance and support which provided me the freedom to take this research in the direction of my interest.\\
\\
I would also like to thank my family and friends for their encouragement and support. To those who quietly listened to my software complaints. To those who worked throughout the nights with me. To those who helped me write what I couldn't say. I cannot thank you enough.
}

\declaration{
I, Stefan Collier, declare that this dissertation and the work presented in it are my own and has been generated by me as the result of my own original research.\\
I confirm that:\\
1. This work was done wholly or mainly while in candidature for a degree at this University;\\
2. Where any part of this dissertation has previously been submitted for any other qualification at this University or any other institution, this has been clearly stated;\\
3. Where I have consulted the published work of others, this is always clearly attributed;\\
4. Where I have quoted from the work of others, the source is always given. With the exception of such quotations, this dissertation is entirely my own work;\\
5. I have acknowledged all main sources of help;\\
6. Where the thesis is based on work done by myself jointly with others, I have made clear exactly what was done by others and what I have contributed myself;\\
7. Either none of this work has been published before submission, or parts of this work have been published by :\\
\\
Stefan Collier\\
April 2016
}
\tableofcontents
\listoffigures
\listoftables

\mainmatter
%% ----------------------------------------------------------------
%\include{Introduction}
%\include{Conclusions}
\include{chapters/1Project/main}
\include{chapters/2Lit/main}
\include{chapters/3Design/HighLevel}
\include{chapters/3Design/InDepth}
\include{chapters/4Impl/main}

\include{chapters/5Experiments/1/main}
\include{chapters/5Experiments/2/main}
\include{chapters/5Experiments/3/main}
\include{chapters/5Experiments/4/main}

\include{chapters/6Conclusion/main}

\appendix
\include{appendix/AppendixB}
\include{appendix/D/main}
\include{appendix/AppendixC}

\backmatter
\bibliographystyle{ecs}
\bibliography{ECS}
\end{document}
%% ----------------------------------------------------------------

\include{appendix/AppendixC}

\backmatter
\bibliographystyle{ecs}
\bibliography{ECS}
\end{document}
%% ----------------------------------------------------------------

\include{chapters/3Design/HighLevel}
\include{chapters/3Design/InDepth}
 %% ----------------------------------------------------------------
%% Progress.tex
%% ---------------------------------------------------------------- 
\documentclass{ecsprogress}    % Use the progress Style
\graphicspath{{../figs/}}   % Location of your graphics files
    \usepackage{natbib}            % Use Natbib style for the refs.
\hypersetup{colorlinks=true}   % Set to false for black/white printing
\input{Definitions}            % Include your abbreviations



\usepackage{enumitem}% http://ctan.org/pkg/enumitem
\usepackage{multirow}
\usepackage{float}
\usepackage{amsmath}
\usepackage{multicol}
\usepackage{amssymb}
\usepackage[normalem]{ulem}
\useunder{\uline}{\ul}{}
\usepackage{wrapfig}


\usepackage[table,xcdraw]{xcolor}


%% ----------------------------------------------------------------
\begin{document}
\frontmatter
\title      {Heterogeneous Agent-based Model for Supermarket Competition}
\authors    {\texorpdfstring
             {\href{mailto:sc22g13@ecs.soton.ac.uk}{Stefan J. Collier}}
             {Stefan J. Collier}
            }
\addresses  {\groupname\\\deptname\\\univname}
\date       {\today}
\subject    {}
\keywords   {}
\supervisor {Dr. Maria Polukarov}
\examiner   {Professor Sheng Chen}

\maketitle
\begin{abstract}
This project aim was to model and analyse the effects of competitive pricing behaviors of grocery retailers on the British market. 

This was achieved by creating a multi-agent model, containing retailer and consumer agents. The heterogeneous crowd of retailers employs either a uniform pricing strategy or a ‘local price flexing’ strategy. The actions of these retailers are chosen by predicting the profit of each action, using a perceptron. Following on from the consideration of different economic models, a discrete model was developed so that software agents have a discrete environment to operate within. Within the model, it has been observed how supermarkets with differing behaviors affect a heterogeneous crowd of consumer agents. The model was implemented in Java with Python used to evaluate the results. 

The simulation displays good acceptance with real grocery market behavior, i.e. captures the performance of British retailers thus can be used to determine the impact of changes in their behavior on their competitors and consumers.Furthermore it can be used to provide insight into sustainability of volatile pricing strategies, providing a useful insight in volatility of British supermarket retail industry. 
\end{abstract}
\acknowledgements{
I would like to express my sincere gratitude to Dr Maria Polukarov for her guidance and support which provided me the freedom to take this research in the direction of my interest.\\
\\
I would also like to thank my family and friends for their encouragement and support. To those who quietly listened to my software complaints. To those who worked throughout the nights with me. To those who helped me write what I couldn't say. I cannot thank you enough.
}

\declaration{
I, Stefan Collier, declare that this dissertation and the work presented in it are my own and has been generated by me as the result of my own original research.\\
I confirm that:\\
1. This work was done wholly or mainly while in candidature for a degree at this University;\\
2. Where any part of this dissertation has previously been submitted for any other qualification at this University or any other institution, this has been clearly stated;\\
3. Where I have consulted the published work of others, this is always clearly attributed;\\
4. Where I have quoted from the work of others, the source is always given. With the exception of such quotations, this dissertation is entirely my own work;\\
5. I have acknowledged all main sources of help;\\
6. Where the thesis is based on work done by myself jointly with others, I have made clear exactly what was done by others and what I have contributed myself;\\
7. Either none of this work has been published before submission, or parts of this work have been published by :\\
\\
Stefan Collier\\
April 2016
}
\tableofcontents
\listoffigures
\listoftables

\mainmatter
%% ----------------------------------------------------------------
%\include{Introduction}
%\include{Conclusions}
 %% ----------------------------------------------------------------
%% Progress.tex
%% ---------------------------------------------------------------- 
\documentclass{ecsprogress}    % Use the progress Style
\graphicspath{{../figs/}}   % Location of your graphics files
    \usepackage{natbib}            % Use Natbib style for the refs.
\hypersetup{colorlinks=true}   % Set to false for black/white printing
\input{Definitions}            % Include your abbreviations



\usepackage{enumitem}% http://ctan.org/pkg/enumitem
\usepackage{multirow}
\usepackage{float}
\usepackage{amsmath}
\usepackage{multicol}
\usepackage{amssymb}
\usepackage[normalem]{ulem}
\useunder{\uline}{\ul}{}
\usepackage{wrapfig}


\usepackage[table,xcdraw]{xcolor}


%% ----------------------------------------------------------------
\begin{document}
\frontmatter
\title      {Heterogeneous Agent-based Model for Supermarket Competition}
\authors    {\texorpdfstring
             {\href{mailto:sc22g13@ecs.soton.ac.uk}{Stefan J. Collier}}
             {Stefan J. Collier}
            }
\addresses  {\groupname\\\deptname\\\univname}
\date       {\today}
\subject    {}
\keywords   {}
\supervisor {Dr. Maria Polukarov}
\examiner   {Professor Sheng Chen}

\maketitle
\begin{abstract}
This project aim was to model and analyse the effects of competitive pricing behaviors of grocery retailers on the British market. 

This was achieved by creating a multi-agent model, containing retailer and consumer agents. The heterogeneous crowd of retailers employs either a uniform pricing strategy or a ‘local price flexing’ strategy. The actions of these retailers are chosen by predicting the profit of each action, using a perceptron. Following on from the consideration of different economic models, a discrete model was developed so that software agents have a discrete environment to operate within. Within the model, it has been observed how supermarkets with differing behaviors affect a heterogeneous crowd of consumer agents. The model was implemented in Java with Python used to evaluate the results. 

The simulation displays good acceptance with real grocery market behavior, i.e. captures the performance of British retailers thus can be used to determine the impact of changes in their behavior on their competitors and consumers.Furthermore it can be used to provide insight into sustainability of volatile pricing strategies, providing a useful insight in volatility of British supermarket retail industry. 
\end{abstract}
\acknowledgements{
I would like to express my sincere gratitude to Dr Maria Polukarov for her guidance and support which provided me the freedom to take this research in the direction of my interest.\\
\\
I would also like to thank my family and friends for their encouragement and support. To those who quietly listened to my software complaints. To those who worked throughout the nights with me. To those who helped me write what I couldn't say. I cannot thank you enough.
}

\declaration{
I, Stefan Collier, declare that this dissertation and the work presented in it are my own and has been generated by me as the result of my own original research.\\
I confirm that:\\
1. This work was done wholly or mainly while in candidature for a degree at this University;\\
2. Where any part of this dissertation has previously been submitted for any other qualification at this University or any other institution, this has been clearly stated;\\
3. Where I have consulted the published work of others, this is always clearly attributed;\\
4. Where I have quoted from the work of others, the source is always given. With the exception of such quotations, this dissertation is entirely my own work;\\
5. I have acknowledged all main sources of help;\\
6. Where the thesis is based on work done by myself jointly with others, I have made clear exactly what was done by others and what I have contributed myself;\\
7. Either none of this work has been published before submission, or parts of this work have been published by :\\
\\
Stefan Collier\\
April 2016
}
\tableofcontents
\listoffigures
\listoftables

\mainmatter
%% ----------------------------------------------------------------
%\include{Introduction}
%\include{Conclusions}
\include{chapters/1Project/main}
\include{chapters/2Lit/main}
\include{chapters/3Design/HighLevel}
\include{chapters/3Design/InDepth}
\include{chapters/4Impl/main}

\include{chapters/5Experiments/1/main}
\include{chapters/5Experiments/2/main}
\include{chapters/5Experiments/3/main}
\include{chapters/5Experiments/4/main}

\include{chapters/6Conclusion/main}

\appendix
\include{appendix/AppendixB}
\include{appendix/D/main}
\include{appendix/AppendixC}

\backmatter
\bibliographystyle{ecs}
\bibliography{ECS}
\end{document}
%% ----------------------------------------------------------------

 %% ----------------------------------------------------------------
%% Progress.tex
%% ---------------------------------------------------------------- 
\documentclass{ecsprogress}    % Use the progress Style
\graphicspath{{../figs/}}   % Location of your graphics files
    \usepackage{natbib}            % Use Natbib style for the refs.
\hypersetup{colorlinks=true}   % Set to false for black/white printing
\input{Definitions}            % Include your abbreviations



\usepackage{enumitem}% http://ctan.org/pkg/enumitem
\usepackage{multirow}
\usepackage{float}
\usepackage{amsmath}
\usepackage{multicol}
\usepackage{amssymb}
\usepackage[normalem]{ulem}
\useunder{\uline}{\ul}{}
\usepackage{wrapfig}


\usepackage[table,xcdraw]{xcolor}


%% ----------------------------------------------------------------
\begin{document}
\frontmatter
\title      {Heterogeneous Agent-based Model for Supermarket Competition}
\authors    {\texorpdfstring
             {\href{mailto:sc22g13@ecs.soton.ac.uk}{Stefan J. Collier}}
             {Stefan J. Collier}
            }
\addresses  {\groupname\\\deptname\\\univname}
\date       {\today}
\subject    {}
\keywords   {}
\supervisor {Dr. Maria Polukarov}
\examiner   {Professor Sheng Chen}

\maketitle
\begin{abstract}
This project aim was to model and analyse the effects of competitive pricing behaviors of grocery retailers on the British market. 

This was achieved by creating a multi-agent model, containing retailer and consumer agents. The heterogeneous crowd of retailers employs either a uniform pricing strategy or a ‘local price flexing’ strategy. The actions of these retailers are chosen by predicting the profit of each action, using a perceptron. Following on from the consideration of different economic models, a discrete model was developed so that software agents have a discrete environment to operate within. Within the model, it has been observed how supermarkets with differing behaviors affect a heterogeneous crowd of consumer agents. The model was implemented in Java with Python used to evaluate the results. 

The simulation displays good acceptance with real grocery market behavior, i.e. captures the performance of British retailers thus can be used to determine the impact of changes in their behavior on their competitors and consumers.Furthermore it can be used to provide insight into sustainability of volatile pricing strategies, providing a useful insight in volatility of British supermarket retail industry. 
\end{abstract}
\acknowledgements{
I would like to express my sincere gratitude to Dr Maria Polukarov for her guidance and support which provided me the freedom to take this research in the direction of my interest.\\
\\
I would also like to thank my family and friends for their encouragement and support. To those who quietly listened to my software complaints. To those who worked throughout the nights with me. To those who helped me write what I couldn't say. I cannot thank you enough.
}

\declaration{
I, Stefan Collier, declare that this dissertation and the work presented in it are my own and has been generated by me as the result of my own original research.\\
I confirm that:\\
1. This work was done wholly or mainly while in candidature for a degree at this University;\\
2. Where any part of this dissertation has previously been submitted for any other qualification at this University or any other institution, this has been clearly stated;\\
3. Where I have consulted the published work of others, this is always clearly attributed;\\
4. Where I have quoted from the work of others, the source is always given. With the exception of such quotations, this dissertation is entirely my own work;\\
5. I have acknowledged all main sources of help;\\
6. Where the thesis is based on work done by myself jointly with others, I have made clear exactly what was done by others and what I have contributed myself;\\
7. Either none of this work has been published before submission, or parts of this work have been published by :\\
\\
Stefan Collier\\
April 2016
}
\tableofcontents
\listoffigures
\listoftables

\mainmatter
%% ----------------------------------------------------------------
%\include{Introduction}
%\include{Conclusions}
\include{chapters/1Project/main}
\include{chapters/2Lit/main}
\include{chapters/3Design/HighLevel}
\include{chapters/3Design/InDepth}
\include{chapters/4Impl/main}

\include{chapters/5Experiments/1/main}
\include{chapters/5Experiments/2/main}
\include{chapters/5Experiments/3/main}
\include{chapters/5Experiments/4/main}

\include{chapters/6Conclusion/main}

\appendix
\include{appendix/AppendixB}
\include{appendix/D/main}
\include{appendix/AppendixC}

\backmatter
\bibliographystyle{ecs}
\bibliography{ECS}
\end{document}
%% ----------------------------------------------------------------

\include{chapters/3Design/HighLevel}
\include{chapters/3Design/InDepth}
 %% ----------------------------------------------------------------
%% Progress.tex
%% ---------------------------------------------------------------- 
\documentclass{ecsprogress}    % Use the progress Style
\graphicspath{{../figs/}}   % Location of your graphics files
    \usepackage{natbib}            % Use Natbib style for the refs.
\hypersetup{colorlinks=true}   % Set to false for black/white printing
\input{Definitions}            % Include your abbreviations



\usepackage{enumitem}% http://ctan.org/pkg/enumitem
\usepackage{multirow}
\usepackage{float}
\usepackage{amsmath}
\usepackage{multicol}
\usepackage{amssymb}
\usepackage[normalem]{ulem}
\useunder{\uline}{\ul}{}
\usepackage{wrapfig}


\usepackage[table,xcdraw]{xcolor}


%% ----------------------------------------------------------------
\begin{document}
\frontmatter
\title      {Heterogeneous Agent-based Model for Supermarket Competition}
\authors    {\texorpdfstring
             {\href{mailto:sc22g13@ecs.soton.ac.uk}{Stefan J. Collier}}
             {Stefan J. Collier}
            }
\addresses  {\groupname\\\deptname\\\univname}
\date       {\today}
\subject    {}
\keywords   {}
\supervisor {Dr. Maria Polukarov}
\examiner   {Professor Sheng Chen}

\maketitle
\begin{abstract}
This project aim was to model and analyse the effects of competitive pricing behaviors of grocery retailers on the British market. 

This was achieved by creating a multi-agent model, containing retailer and consumer agents. The heterogeneous crowd of retailers employs either a uniform pricing strategy or a ‘local price flexing’ strategy. The actions of these retailers are chosen by predicting the profit of each action, using a perceptron. Following on from the consideration of different economic models, a discrete model was developed so that software agents have a discrete environment to operate within. Within the model, it has been observed how supermarkets with differing behaviors affect a heterogeneous crowd of consumer agents. The model was implemented in Java with Python used to evaluate the results. 

The simulation displays good acceptance with real grocery market behavior, i.e. captures the performance of British retailers thus can be used to determine the impact of changes in their behavior on their competitors and consumers.Furthermore it can be used to provide insight into sustainability of volatile pricing strategies, providing a useful insight in volatility of British supermarket retail industry. 
\end{abstract}
\acknowledgements{
I would like to express my sincere gratitude to Dr Maria Polukarov for her guidance and support which provided me the freedom to take this research in the direction of my interest.\\
\\
I would also like to thank my family and friends for their encouragement and support. To those who quietly listened to my software complaints. To those who worked throughout the nights with me. To those who helped me write what I couldn't say. I cannot thank you enough.
}

\declaration{
I, Stefan Collier, declare that this dissertation and the work presented in it are my own and has been generated by me as the result of my own original research.\\
I confirm that:\\
1. This work was done wholly or mainly while in candidature for a degree at this University;\\
2. Where any part of this dissertation has previously been submitted for any other qualification at this University or any other institution, this has been clearly stated;\\
3. Where I have consulted the published work of others, this is always clearly attributed;\\
4. Where I have quoted from the work of others, the source is always given. With the exception of such quotations, this dissertation is entirely my own work;\\
5. I have acknowledged all main sources of help;\\
6. Where the thesis is based on work done by myself jointly with others, I have made clear exactly what was done by others and what I have contributed myself;\\
7. Either none of this work has been published before submission, or parts of this work have been published by :\\
\\
Stefan Collier\\
April 2016
}
\tableofcontents
\listoffigures
\listoftables

\mainmatter
%% ----------------------------------------------------------------
%\include{Introduction}
%\include{Conclusions}
\include{chapters/1Project/main}
\include{chapters/2Lit/main}
\include{chapters/3Design/HighLevel}
\include{chapters/3Design/InDepth}
\include{chapters/4Impl/main}

\include{chapters/5Experiments/1/main}
\include{chapters/5Experiments/2/main}
\include{chapters/5Experiments/3/main}
\include{chapters/5Experiments/4/main}

\include{chapters/6Conclusion/main}

\appendix
\include{appendix/AppendixB}
\include{appendix/D/main}
\include{appendix/AppendixC}

\backmatter
\bibliographystyle{ecs}
\bibliography{ECS}
\end{document}
%% ----------------------------------------------------------------


 %% ----------------------------------------------------------------
%% Progress.tex
%% ---------------------------------------------------------------- 
\documentclass{ecsprogress}    % Use the progress Style
\graphicspath{{../figs/}}   % Location of your graphics files
    \usepackage{natbib}            % Use Natbib style for the refs.
\hypersetup{colorlinks=true}   % Set to false for black/white printing
\input{Definitions}            % Include your abbreviations



\usepackage{enumitem}% http://ctan.org/pkg/enumitem
\usepackage{multirow}
\usepackage{float}
\usepackage{amsmath}
\usepackage{multicol}
\usepackage{amssymb}
\usepackage[normalem]{ulem}
\useunder{\uline}{\ul}{}
\usepackage{wrapfig}


\usepackage[table,xcdraw]{xcolor}


%% ----------------------------------------------------------------
\begin{document}
\frontmatter
\title      {Heterogeneous Agent-based Model for Supermarket Competition}
\authors    {\texorpdfstring
             {\href{mailto:sc22g13@ecs.soton.ac.uk}{Stefan J. Collier}}
             {Stefan J. Collier}
            }
\addresses  {\groupname\\\deptname\\\univname}
\date       {\today}
\subject    {}
\keywords   {}
\supervisor {Dr. Maria Polukarov}
\examiner   {Professor Sheng Chen}

\maketitle
\begin{abstract}
This project aim was to model and analyse the effects of competitive pricing behaviors of grocery retailers on the British market. 

This was achieved by creating a multi-agent model, containing retailer and consumer agents. The heterogeneous crowd of retailers employs either a uniform pricing strategy or a ‘local price flexing’ strategy. The actions of these retailers are chosen by predicting the profit of each action, using a perceptron. Following on from the consideration of different economic models, a discrete model was developed so that software agents have a discrete environment to operate within. Within the model, it has been observed how supermarkets with differing behaviors affect a heterogeneous crowd of consumer agents. The model was implemented in Java with Python used to evaluate the results. 

The simulation displays good acceptance with real grocery market behavior, i.e. captures the performance of British retailers thus can be used to determine the impact of changes in their behavior on their competitors and consumers.Furthermore it can be used to provide insight into sustainability of volatile pricing strategies, providing a useful insight in volatility of British supermarket retail industry. 
\end{abstract}
\acknowledgements{
I would like to express my sincere gratitude to Dr Maria Polukarov for her guidance and support which provided me the freedom to take this research in the direction of my interest.\\
\\
I would also like to thank my family and friends for their encouragement and support. To those who quietly listened to my software complaints. To those who worked throughout the nights with me. To those who helped me write what I couldn't say. I cannot thank you enough.
}

\declaration{
I, Stefan Collier, declare that this dissertation and the work presented in it are my own and has been generated by me as the result of my own original research.\\
I confirm that:\\
1. This work was done wholly or mainly while in candidature for a degree at this University;\\
2. Where any part of this dissertation has previously been submitted for any other qualification at this University or any other institution, this has been clearly stated;\\
3. Where I have consulted the published work of others, this is always clearly attributed;\\
4. Where I have quoted from the work of others, the source is always given. With the exception of such quotations, this dissertation is entirely my own work;\\
5. I have acknowledged all main sources of help;\\
6. Where the thesis is based on work done by myself jointly with others, I have made clear exactly what was done by others and what I have contributed myself;\\
7. Either none of this work has been published before submission, or parts of this work have been published by :\\
\\
Stefan Collier\\
April 2016
}
\tableofcontents
\listoffigures
\listoftables

\mainmatter
%% ----------------------------------------------------------------
%\include{Introduction}
%\include{Conclusions}
\include{chapters/1Project/main}
\include{chapters/2Lit/main}
\include{chapters/3Design/HighLevel}
\include{chapters/3Design/InDepth}
\include{chapters/4Impl/main}

\include{chapters/5Experiments/1/main}
\include{chapters/5Experiments/2/main}
\include{chapters/5Experiments/3/main}
\include{chapters/5Experiments/4/main}

\include{chapters/6Conclusion/main}

\appendix
\include{appendix/AppendixB}
\include{appendix/D/main}
\include{appendix/AppendixC}

\backmatter
\bibliographystyle{ecs}
\bibliography{ECS}
\end{document}
%% ----------------------------------------------------------------

 %% ----------------------------------------------------------------
%% Progress.tex
%% ---------------------------------------------------------------- 
\documentclass{ecsprogress}    % Use the progress Style
\graphicspath{{../figs/}}   % Location of your graphics files
    \usepackage{natbib}            % Use Natbib style for the refs.
\hypersetup{colorlinks=true}   % Set to false for black/white printing
\input{Definitions}            % Include your abbreviations



\usepackage{enumitem}% http://ctan.org/pkg/enumitem
\usepackage{multirow}
\usepackage{float}
\usepackage{amsmath}
\usepackage{multicol}
\usepackage{amssymb}
\usepackage[normalem]{ulem}
\useunder{\uline}{\ul}{}
\usepackage{wrapfig}


\usepackage[table,xcdraw]{xcolor}


%% ----------------------------------------------------------------
\begin{document}
\frontmatter
\title      {Heterogeneous Agent-based Model for Supermarket Competition}
\authors    {\texorpdfstring
             {\href{mailto:sc22g13@ecs.soton.ac.uk}{Stefan J. Collier}}
             {Stefan J. Collier}
            }
\addresses  {\groupname\\\deptname\\\univname}
\date       {\today}
\subject    {}
\keywords   {}
\supervisor {Dr. Maria Polukarov}
\examiner   {Professor Sheng Chen}

\maketitle
\begin{abstract}
This project aim was to model and analyse the effects of competitive pricing behaviors of grocery retailers on the British market. 

This was achieved by creating a multi-agent model, containing retailer and consumer agents. The heterogeneous crowd of retailers employs either a uniform pricing strategy or a ‘local price flexing’ strategy. The actions of these retailers are chosen by predicting the profit of each action, using a perceptron. Following on from the consideration of different economic models, a discrete model was developed so that software agents have a discrete environment to operate within. Within the model, it has been observed how supermarkets with differing behaviors affect a heterogeneous crowd of consumer agents. The model was implemented in Java with Python used to evaluate the results. 

The simulation displays good acceptance with real grocery market behavior, i.e. captures the performance of British retailers thus can be used to determine the impact of changes in their behavior on their competitors and consumers.Furthermore it can be used to provide insight into sustainability of volatile pricing strategies, providing a useful insight in volatility of British supermarket retail industry. 
\end{abstract}
\acknowledgements{
I would like to express my sincere gratitude to Dr Maria Polukarov for her guidance and support which provided me the freedom to take this research in the direction of my interest.\\
\\
I would also like to thank my family and friends for their encouragement and support. To those who quietly listened to my software complaints. To those who worked throughout the nights with me. To those who helped me write what I couldn't say. I cannot thank you enough.
}

\declaration{
I, Stefan Collier, declare that this dissertation and the work presented in it are my own and has been generated by me as the result of my own original research.\\
I confirm that:\\
1. This work was done wholly or mainly while in candidature for a degree at this University;\\
2. Where any part of this dissertation has previously been submitted for any other qualification at this University or any other institution, this has been clearly stated;\\
3. Where I have consulted the published work of others, this is always clearly attributed;\\
4. Where I have quoted from the work of others, the source is always given. With the exception of such quotations, this dissertation is entirely my own work;\\
5. I have acknowledged all main sources of help;\\
6. Where the thesis is based on work done by myself jointly with others, I have made clear exactly what was done by others and what I have contributed myself;\\
7. Either none of this work has been published before submission, or parts of this work have been published by :\\
\\
Stefan Collier\\
April 2016
}
\tableofcontents
\listoffigures
\listoftables

\mainmatter
%% ----------------------------------------------------------------
%\include{Introduction}
%\include{Conclusions}
\include{chapters/1Project/main}
\include{chapters/2Lit/main}
\include{chapters/3Design/HighLevel}
\include{chapters/3Design/InDepth}
\include{chapters/4Impl/main}

\include{chapters/5Experiments/1/main}
\include{chapters/5Experiments/2/main}
\include{chapters/5Experiments/3/main}
\include{chapters/5Experiments/4/main}

\include{chapters/6Conclusion/main}

\appendix
\include{appendix/AppendixB}
\include{appendix/D/main}
\include{appendix/AppendixC}

\backmatter
\bibliographystyle{ecs}
\bibliography{ECS}
\end{document}
%% ----------------------------------------------------------------

 %% ----------------------------------------------------------------
%% Progress.tex
%% ---------------------------------------------------------------- 
\documentclass{ecsprogress}    % Use the progress Style
\graphicspath{{../figs/}}   % Location of your graphics files
    \usepackage{natbib}            % Use Natbib style for the refs.
\hypersetup{colorlinks=true}   % Set to false for black/white printing
\input{Definitions}            % Include your abbreviations



\usepackage{enumitem}% http://ctan.org/pkg/enumitem
\usepackage{multirow}
\usepackage{float}
\usepackage{amsmath}
\usepackage{multicol}
\usepackage{amssymb}
\usepackage[normalem]{ulem}
\useunder{\uline}{\ul}{}
\usepackage{wrapfig}


\usepackage[table,xcdraw]{xcolor}


%% ----------------------------------------------------------------
\begin{document}
\frontmatter
\title      {Heterogeneous Agent-based Model for Supermarket Competition}
\authors    {\texorpdfstring
             {\href{mailto:sc22g13@ecs.soton.ac.uk}{Stefan J. Collier}}
             {Stefan J. Collier}
            }
\addresses  {\groupname\\\deptname\\\univname}
\date       {\today}
\subject    {}
\keywords   {}
\supervisor {Dr. Maria Polukarov}
\examiner   {Professor Sheng Chen}

\maketitle
\begin{abstract}
This project aim was to model and analyse the effects of competitive pricing behaviors of grocery retailers on the British market. 

This was achieved by creating a multi-agent model, containing retailer and consumer agents. The heterogeneous crowd of retailers employs either a uniform pricing strategy or a ‘local price flexing’ strategy. The actions of these retailers are chosen by predicting the profit of each action, using a perceptron. Following on from the consideration of different economic models, a discrete model was developed so that software agents have a discrete environment to operate within. Within the model, it has been observed how supermarkets with differing behaviors affect a heterogeneous crowd of consumer agents. The model was implemented in Java with Python used to evaluate the results. 

The simulation displays good acceptance with real grocery market behavior, i.e. captures the performance of British retailers thus can be used to determine the impact of changes in their behavior on their competitors and consumers.Furthermore it can be used to provide insight into sustainability of volatile pricing strategies, providing a useful insight in volatility of British supermarket retail industry. 
\end{abstract}
\acknowledgements{
I would like to express my sincere gratitude to Dr Maria Polukarov for her guidance and support which provided me the freedom to take this research in the direction of my interest.\\
\\
I would also like to thank my family and friends for their encouragement and support. To those who quietly listened to my software complaints. To those who worked throughout the nights with me. To those who helped me write what I couldn't say. I cannot thank you enough.
}

\declaration{
I, Stefan Collier, declare that this dissertation and the work presented in it are my own and has been generated by me as the result of my own original research.\\
I confirm that:\\
1. This work was done wholly or mainly while in candidature for a degree at this University;\\
2. Where any part of this dissertation has previously been submitted for any other qualification at this University or any other institution, this has been clearly stated;\\
3. Where I have consulted the published work of others, this is always clearly attributed;\\
4. Where I have quoted from the work of others, the source is always given. With the exception of such quotations, this dissertation is entirely my own work;\\
5. I have acknowledged all main sources of help;\\
6. Where the thesis is based on work done by myself jointly with others, I have made clear exactly what was done by others and what I have contributed myself;\\
7. Either none of this work has been published before submission, or parts of this work have been published by :\\
\\
Stefan Collier\\
April 2016
}
\tableofcontents
\listoffigures
\listoftables

\mainmatter
%% ----------------------------------------------------------------
%\include{Introduction}
%\include{Conclusions}
\include{chapters/1Project/main}
\include{chapters/2Lit/main}
\include{chapters/3Design/HighLevel}
\include{chapters/3Design/InDepth}
\include{chapters/4Impl/main}

\include{chapters/5Experiments/1/main}
\include{chapters/5Experiments/2/main}
\include{chapters/5Experiments/3/main}
\include{chapters/5Experiments/4/main}

\include{chapters/6Conclusion/main}

\appendix
\include{appendix/AppendixB}
\include{appendix/D/main}
\include{appendix/AppendixC}

\backmatter
\bibliographystyle{ecs}
\bibliography{ECS}
\end{document}
%% ----------------------------------------------------------------

 %% ----------------------------------------------------------------
%% Progress.tex
%% ---------------------------------------------------------------- 
\documentclass{ecsprogress}    % Use the progress Style
\graphicspath{{../figs/}}   % Location of your graphics files
    \usepackage{natbib}            % Use Natbib style for the refs.
\hypersetup{colorlinks=true}   % Set to false for black/white printing
\input{Definitions}            % Include your abbreviations



\usepackage{enumitem}% http://ctan.org/pkg/enumitem
\usepackage{multirow}
\usepackage{float}
\usepackage{amsmath}
\usepackage{multicol}
\usepackage{amssymb}
\usepackage[normalem]{ulem}
\useunder{\uline}{\ul}{}
\usepackage{wrapfig}


\usepackage[table,xcdraw]{xcolor}


%% ----------------------------------------------------------------
\begin{document}
\frontmatter
\title      {Heterogeneous Agent-based Model for Supermarket Competition}
\authors    {\texorpdfstring
             {\href{mailto:sc22g13@ecs.soton.ac.uk}{Stefan J. Collier}}
             {Stefan J. Collier}
            }
\addresses  {\groupname\\\deptname\\\univname}
\date       {\today}
\subject    {}
\keywords   {}
\supervisor {Dr. Maria Polukarov}
\examiner   {Professor Sheng Chen}

\maketitle
\begin{abstract}
This project aim was to model and analyse the effects of competitive pricing behaviors of grocery retailers on the British market. 

This was achieved by creating a multi-agent model, containing retailer and consumer agents. The heterogeneous crowd of retailers employs either a uniform pricing strategy or a ‘local price flexing’ strategy. The actions of these retailers are chosen by predicting the profit of each action, using a perceptron. Following on from the consideration of different economic models, a discrete model was developed so that software agents have a discrete environment to operate within. Within the model, it has been observed how supermarkets with differing behaviors affect a heterogeneous crowd of consumer agents. The model was implemented in Java with Python used to evaluate the results. 

The simulation displays good acceptance with real grocery market behavior, i.e. captures the performance of British retailers thus can be used to determine the impact of changes in their behavior on their competitors and consumers.Furthermore it can be used to provide insight into sustainability of volatile pricing strategies, providing a useful insight in volatility of British supermarket retail industry. 
\end{abstract}
\acknowledgements{
I would like to express my sincere gratitude to Dr Maria Polukarov for her guidance and support which provided me the freedom to take this research in the direction of my interest.\\
\\
I would also like to thank my family and friends for their encouragement and support. To those who quietly listened to my software complaints. To those who worked throughout the nights with me. To those who helped me write what I couldn't say. I cannot thank you enough.
}

\declaration{
I, Stefan Collier, declare that this dissertation and the work presented in it are my own and has been generated by me as the result of my own original research.\\
I confirm that:\\
1. This work was done wholly or mainly while in candidature for a degree at this University;\\
2. Where any part of this dissertation has previously been submitted for any other qualification at this University or any other institution, this has been clearly stated;\\
3. Where I have consulted the published work of others, this is always clearly attributed;\\
4. Where I have quoted from the work of others, the source is always given. With the exception of such quotations, this dissertation is entirely my own work;\\
5. I have acknowledged all main sources of help;\\
6. Where the thesis is based on work done by myself jointly with others, I have made clear exactly what was done by others and what I have contributed myself;\\
7. Either none of this work has been published before submission, or parts of this work have been published by :\\
\\
Stefan Collier\\
April 2016
}
\tableofcontents
\listoffigures
\listoftables

\mainmatter
%% ----------------------------------------------------------------
%\include{Introduction}
%\include{Conclusions}
\include{chapters/1Project/main}
\include{chapters/2Lit/main}
\include{chapters/3Design/HighLevel}
\include{chapters/3Design/InDepth}
\include{chapters/4Impl/main}

\include{chapters/5Experiments/1/main}
\include{chapters/5Experiments/2/main}
\include{chapters/5Experiments/3/main}
\include{chapters/5Experiments/4/main}

\include{chapters/6Conclusion/main}

\appendix
\include{appendix/AppendixB}
\include{appendix/D/main}
\include{appendix/AppendixC}

\backmatter
\bibliographystyle{ecs}
\bibliography{ECS}
\end{document}
%% ----------------------------------------------------------------


 %% ----------------------------------------------------------------
%% Progress.tex
%% ---------------------------------------------------------------- 
\documentclass{ecsprogress}    % Use the progress Style
\graphicspath{{../figs/}}   % Location of your graphics files
    \usepackage{natbib}            % Use Natbib style for the refs.
\hypersetup{colorlinks=true}   % Set to false for black/white printing
\input{Definitions}            % Include your abbreviations



\usepackage{enumitem}% http://ctan.org/pkg/enumitem
\usepackage{multirow}
\usepackage{float}
\usepackage{amsmath}
\usepackage{multicol}
\usepackage{amssymb}
\usepackage[normalem]{ulem}
\useunder{\uline}{\ul}{}
\usepackage{wrapfig}


\usepackage[table,xcdraw]{xcolor}


%% ----------------------------------------------------------------
\begin{document}
\frontmatter
\title      {Heterogeneous Agent-based Model for Supermarket Competition}
\authors    {\texorpdfstring
             {\href{mailto:sc22g13@ecs.soton.ac.uk}{Stefan J. Collier}}
             {Stefan J. Collier}
            }
\addresses  {\groupname\\\deptname\\\univname}
\date       {\today}
\subject    {}
\keywords   {}
\supervisor {Dr. Maria Polukarov}
\examiner   {Professor Sheng Chen}

\maketitle
\begin{abstract}
This project aim was to model and analyse the effects of competitive pricing behaviors of grocery retailers on the British market. 

This was achieved by creating a multi-agent model, containing retailer and consumer agents. The heterogeneous crowd of retailers employs either a uniform pricing strategy or a ‘local price flexing’ strategy. The actions of these retailers are chosen by predicting the profit of each action, using a perceptron. Following on from the consideration of different economic models, a discrete model was developed so that software agents have a discrete environment to operate within. Within the model, it has been observed how supermarkets with differing behaviors affect a heterogeneous crowd of consumer agents. The model was implemented in Java with Python used to evaluate the results. 

The simulation displays good acceptance with real grocery market behavior, i.e. captures the performance of British retailers thus can be used to determine the impact of changes in their behavior on their competitors and consumers.Furthermore it can be used to provide insight into sustainability of volatile pricing strategies, providing a useful insight in volatility of British supermarket retail industry. 
\end{abstract}
\acknowledgements{
I would like to express my sincere gratitude to Dr Maria Polukarov for her guidance and support which provided me the freedom to take this research in the direction of my interest.\\
\\
I would also like to thank my family and friends for their encouragement and support. To those who quietly listened to my software complaints. To those who worked throughout the nights with me. To those who helped me write what I couldn't say. I cannot thank you enough.
}

\declaration{
I, Stefan Collier, declare that this dissertation and the work presented in it are my own and has been generated by me as the result of my own original research.\\
I confirm that:\\
1. This work was done wholly or mainly while in candidature for a degree at this University;\\
2. Where any part of this dissertation has previously been submitted for any other qualification at this University or any other institution, this has been clearly stated;\\
3. Where I have consulted the published work of others, this is always clearly attributed;\\
4. Where I have quoted from the work of others, the source is always given. With the exception of such quotations, this dissertation is entirely my own work;\\
5. I have acknowledged all main sources of help;\\
6. Where the thesis is based on work done by myself jointly with others, I have made clear exactly what was done by others and what I have contributed myself;\\
7. Either none of this work has been published before submission, or parts of this work have been published by :\\
\\
Stefan Collier\\
April 2016
}
\tableofcontents
\listoffigures
\listoftables

\mainmatter
%% ----------------------------------------------------------------
%\include{Introduction}
%\include{Conclusions}
\include{chapters/1Project/main}
\include{chapters/2Lit/main}
\include{chapters/3Design/HighLevel}
\include{chapters/3Design/InDepth}
\include{chapters/4Impl/main}

\include{chapters/5Experiments/1/main}
\include{chapters/5Experiments/2/main}
\include{chapters/5Experiments/3/main}
\include{chapters/5Experiments/4/main}

\include{chapters/6Conclusion/main}

\appendix
\include{appendix/AppendixB}
\include{appendix/D/main}
\include{appendix/AppendixC}

\backmatter
\bibliographystyle{ecs}
\bibliography{ECS}
\end{document}
%% ----------------------------------------------------------------


\appendix
\include{appendix/AppendixB}
 %% ----------------------------------------------------------------
%% Progress.tex
%% ---------------------------------------------------------------- 
\documentclass{ecsprogress}    % Use the progress Style
\graphicspath{{../figs/}}   % Location of your graphics files
    \usepackage{natbib}            % Use Natbib style for the refs.
\hypersetup{colorlinks=true}   % Set to false for black/white printing
\input{Definitions}            % Include your abbreviations



\usepackage{enumitem}% http://ctan.org/pkg/enumitem
\usepackage{multirow}
\usepackage{float}
\usepackage{amsmath}
\usepackage{multicol}
\usepackage{amssymb}
\usepackage[normalem]{ulem}
\useunder{\uline}{\ul}{}
\usepackage{wrapfig}


\usepackage[table,xcdraw]{xcolor}


%% ----------------------------------------------------------------
\begin{document}
\frontmatter
\title      {Heterogeneous Agent-based Model for Supermarket Competition}
\authors    {\texorpdfstring
             {\href{mailto:sc22g13@ecs.soton.ac.uk}{Stefan J. Collier}}
             {Stefan J. Collier}
            }
\addresses  {\groupname\\\deptname\\\univname}
\date       {\today}
\subject    {}
\keywords   {}
\supervisor {Dr. Maria Polukarov}
\examiner   {Professor Sheng Chen}

\maketitle
\begin{abstract}
This project aim was to model and analyse the effects of competitive pricing behaviors of grocery retailers on the British market. 

This was achieved by creating a multi-agent model, containing retailer and consumer agents. The heterogeneous crowd of retailers employs either a uniform pricing strategy or a ‘local price flexing’ strategy. The actions of these retailers are chosen by predicting the profit of each action, using a perceptron. Following on from the consideration of different economic models, a discrete model was developed so that software agents have a discrete environment to operate within. Within the model, it has been observed how supermarkets with differing behaviors affect a heterogeneous crowd of consumer agents. The model was implemented in Java with Python used to evaluate the results. 

The simulation displays good acceptance with real grocery market behavior, i.e. captures the performance of British retailers thus can be used to determine the impact of changes in their behavior on their competitors and consumers.Furthermore it can be used to provide insight into sustainability of volatile pricing strategies, providing a useful insight in volatility of British supermarket retail industry. 
\end{abstract}
\acknowledgements{
I would like to express my sincere gratitude to Dr Maria Polukarov for her guidance and support which provided me the freedom to take this research in the direction of my interest.\\
\\
I would also like to thank my family and friends for their encouragement and support. To those who quietly listened to my software complaints. To those who worked throughout the nights with me. To those who helped me write what I couldn't say. I cannot thank you enough.
}

\declaration{
I, Stefan Collier, declare that this dissertation and the work presented in it are my own and has been generated by me as the result of my own original research.\\
I confirm that:\\
1. This work was done wholly or mainly while in candidature for a degree at this University;\\
2. Where any part of this dissertation has previously been submitted for any other qualification at this University or any other institution, this has been clearly stated;\\
3. Where I have consulted the published work of others, this is always clearly attributed;\\
4. Where I have quoted from the work of others, the source is always given. With the exception of such quotations, this dissertation is entirely my own work;\\
5. I have acknowledged all main sources of help;\\
6. Where the thesis is based on work done by myself jointly with others, I have made clear exactly what was done by others and what I have contributed myself;\\
7. Either none of this work has been published before submission, or parts of this work have been published by :\\
\\
Stefan Collier\\
April 2016
}
\tableofcontents
\listoffigures
\listoftables

\mainmatter
%% ----------------------------------------------------------------
%\include{Introduction}
%\include{Conclusions}
\include{chapters/1Project/main}
\include{chapters/2Lit/main}
\include{chapters/3Design/HighLevel}
\include{chapters/3Design/InDepth}
\include{chapters/4Impl/main}

\include{chapters/5Experiments/1/main}
\include{chapters/5Experiments/2/main}
\include{chapters/5Experiments/3/main}
\include{chapters/5Experiments/4/main}

\include{chapters/6Conclusion/main}

\appendix
\include{appendix/AppendixB}
\include{appendix/D/main}
\include{appendix/AppendixC}

\backmatter
\bibliographystyle{ecs}
\bibliography{ECS}
\end{document}
%% ----------------------------------------------------------------

\include{appendix/AppendixC}

\backmatter
\bibliographystyle{ecs}
\bibliography{ECS}
\end{document}
%% ----------------------------------------------------------------


 %% ----------------------------------------------------------------
%% Progress.tex
%% ---------------------------------------------------------------- 
\documentclass{ecsprogress}    % Use the progress Style
\graphicspath{{../figs/}}   % Location of your graphics files
    \usepackage{natbib}            % Use Natbib style for the refs.
\hypersetup{colorlinks=true}   % Set to false for black/white printing
\input{Definitions}            % Include your abbreviations



\usepackage{enumitem}% http://ctan.org/pkg/enumitem
\usepackage{multirow}
\usepackage{float}
\usepackage{amsmath}
\usepackage{multicol}
\usepackage{amssymb}
\usepackage[normalem]{ulem}
\useunder{\uline}{\ul}{}
\usepackage{wrapfig}


\usepackage[table,xcdraw]{xcolor}


%% ----------------------------------------------------------------
\begin{document}
\frontmatter
\title      {Heterogeneous Agent-based Model for Supermarket Competition}
\authors    {\texorpdfstring
             {\href{mailto:sc22g13@ecs.soton.ac.uk}{Stefan J. Collier}}
             {Stefan J. Collier}
            }
\addresses  {\groupname\\\deptname\\\univname}
\date       {\today}
\subject    {}
\keywords   {}
\supervisor {Dr. Maria Polukarov}
\examiner   {Professor Sheng Chen}

\maketitle
\begin{abstract}
This project aim was to model and analyse the effects of competitive pricing behaviors of grocery retailers on the British market. 

This was achieved by creating a multi-agent model, containing retailer and consumer agents. The heterogeneous crowd of retailers employs either a uniform pricing strategy or a ‘local price flexing’ strategy. The actions of these retailers are chosen by predicting the profit of each action, using a perceptron. Following on from the consideration of different economic models, a discrete model was developed so that software agents have a discrete environment to operate within. Within the model, it has been observed how supermarkets with differing behaviors affect a heterogeneous crowd of consumer agents. The model was implemented in Java with Python used to evaluate the results. 

The simulation displays good acceptance with real grocery market behavior, i.e. captures the performance of British retailers thus can be used to determine the impact of changes in their behavior on their competitors and consumers.Furthermore it can be used to provide insight into sustainability of volatile pricing strategies, providing a useful insight in volatility of British supermarket retail industry. 
\end{abstract}
\acknowledgements{
I would like to express my sincere gratitude to Dr Maria Polukarov for her guidance and support which provided me the freedom to take this research in the direction of my interest.\\
\\
I would also like to thank my family and friends for their encouragement and support. To those who quietly listened to my software complaints. To those who worked throughout the nights with me. To those who helped me write what I couldn't say. I cannot thank you enough.
}

\declaration{
I, Stefan Collier, declare that this dissertation and the work presented in it are my own and has been generated by me as the result of my own original research.\\
I confirm that:\\
1. This work was done wholly or mainly while in candidature for a degree at this University;\\
2. Where any part of this dissertation has previously been submitted for any other qualification at this University or any other institution, this has been clearly stated;\\
3. Where I have consulted the published work of others, this is always clearly attributed;\\
4. Where I have quoted from the work of others, the source is always given. With the exception of such quotations, this dissertation is entirely my own work;\\
5. I have acknowledged all main sources of help;\\
6. Where the thesis is based on work done by myself jointly with others, I have made clear exactly what was done by others and what I have contributed myself;\\
7. Either none of this work has been published before submission, or parts of this work have been published by :\\
\\
Stefan Collier\\
April 2016
}
\tableofcontents
\listoffigures
\listoftables

\mainmatter
%% ----------------------------------------------------------------
%\include{Introduction}
%\include{Conclusions}
 %% ----------------------------------------------------------------
%% Progress.tex
%% ---------------------------------------------------------------- 
\documentclass{ecsprogress}    % Use the progress Style
\graphicspath{{../figs/}}   % Location of your graphics files
    \usepackage{natbib}            % Use Natbib style for the refs.
\hypersetup{colorlinks=true}   % Set to false for black/white printing
\input{Definitions}            % Include your abbreviations



\usepackage{enumitem}% http://ctan.org/pkg/enumitem
\usepackage{multirow}
\usepackage{float}
\usepackage{amsmath}
\usepackage{multicol}
\usepackage{amssymb}
\usepackage[normalem]{ulem}
\useunder{\uline}{\ul}{}
\usepackage{wrapfig}


\usepackage[table,xcdraw]{xcolor}


%% ----------------------------------------------------------------
\begin{document}
\frontmatter
\title      {Heterogeneous Agent-based Model for Supermarket Competition}
\authors    {\texorpdfstring
             {\href{mailto:sc22g13@ecs.soton.ac.uk}{Stefan J. Collier}}
             {Stefan J. Collier}
            }
\addresses  {\groupname\\\deptname\\\univname}
\date       {\today}
\subject    {}
\keywords   {}
\supervisor {Dr. Maria Polukarov}
\examiner   {Professor Sheng Chen}

\maketitle
\begin{abstract}
This project aim was to model and analyse the effects of competitive pricing behaviors of grocery retailers on the British market. 

This was achieved by creating a multi-agent model, containing retailer and consumer agents. The heterogeneous crowd of retailers employs either a uniform pricing strategy or a ‘local price flexing’ strategy. The actions of these retailers are chosen by predicting the profit of each action, using a perceptron. Following on from the consideration of different economic models, a discrete model was developed so that software agents have a discrete environment to operate within. Within the model, it has been observed how supermarkets with differing behaviors affect a heterogeneous crowd of consumer agents. The model was implemented in Java with Python used to evaluate the results. 

The simulation displays good acceptance with real grocery market behavior, i.e. captures the performance of British retailers thus can be used to determine the impact of changes in their behavior on their competitors and consumers.Furthermore it can be used to provide insight into sustainability of volatile pricing strategies, providing a useful insight in volatility of British supermarket retail industry. 
\end{abstract}
\acknowledgements{
I would like to express my sincere gratitude to Dr Maria Polukarov for her guidance and support which provided me the freedom to take this research in the direction of my interest.\\
\\
I would also like to thank my family and friends for their encouragement and support. To those who quietly listened to my software complaints. To those who worked throughout the nights with me. To those who helped me write what I couldn't say. I cannot thank you enough.
}

\declaration{
I, Stefan Collier, declare that this dissertation and the work presented in it are my own and has been generated by me as the result of my own original research.\\
I confirm that:\\
1. This work was done wholly or mainly while in candidature for a degree at this University;\\
2. Where any part of this dissertation has previously been submitted for any other qualification at this University or any other institution, this has been clearly stated;\\
3. Where I have consulted the published work of others, this is always clearly attributed;\\
4. Where I have quoted from the work of others, the source is always given. With the exception of such quotations, this dissertation is entirely my own work;\\
5. I have acknowledged all main sources of help;\\
6. Where the thesis is based on work done by myself jointly with others, I have made clear exactly what was done by others and what I have contributed myself;\\
7. Either none of this work has been published before submission, or parts of this work have been published by :\\
\\
Stefan Collier\\
April 2016
}
\tableofcontents
\listoffigures
\listoftables

\mainmatter
%% ----------------------------------------------------------------
%\include{Introduction}
%\include{Conclusions}
\include{chapters/1Project/main}
\include{chapters/2Lit/main}
\include{chapters/3Design/HighLevel}
\include{chapters/3Design/InDepth}
\include{chapters/4Impl/main}

\include{chapters/5Experiments/1/main}
\include{chapters/5Experiments/2/main}
\include{chapters/5Experiments/3/main}
\include{chapters/5Experiments/4/main}

\include{chapters/6Conclusion/main}

\appendix
\include{appendix/AppendixB}
\include{appendix/D/main}
\include{appendix/AppendixC}

\backmatter
\bibliographystyle{ecs}
\bibliography{ECS}
\end{document}
%% ----------------------------------------------------------------

 %% ----------------------------------------------------------------
%% Progress.tex
%% ---------------------------------------------------------------- 
\documentclass{ecsprogress}    % Use the progress Style
\graphicspath{{../figs/}}   % Location of your graphics files
    \usepackage{natbib}            % Use Natbib style for the refs.
\hypersetup{colorlinks=true}   % Set to false for black/white printing
\input{Definitions}            % Include your abbreviations



\usepackage{enumitem}% http://ctan.org/pkg/enumitem
\usepackage{multirow}
\usepackage{float}
\usepackage{amsmath}
\usepackage{multicol}
\usepackage{amssymb}
\usepackage[normalem]{ulem}
\useunder{\uline}{\ul}{}
\usepackage{wrapfig}


\usepackage[table,xcdraw]{xcolor}


%% ----------------------------------------------------------------
\begin{document}
\frontmatter
\title      {Heterogeneous Agent-based Model for Supermarket Competition}
\authors    {\texorpdfstring
             {\href{mailto:sc22g13@ecs.soton.ac.uk}{Stefan J. Collier}}
             {Stefan J. Collier}
            }
\addresses  {\groupname\\\deptname\\\univname}
\date       {\today}
\subject    {}
\keywords   {}
\supervisor {Dr. Maria Polukarov}
\examiner   {Professor Sheng Chen}

\maketitle
\begin{abstract}
This project aim was to model and analyse the effects of competitive pricing behaviors of grocery retailers on the British market. 

This was achieved by creating a multi-agent model, containing retailer and consumer agents. The heterogeneous crowd of retailers employs either a uniform pricing strategy or a ‘local price flexing’ strategy. The actions of these retailers are chosen by predicting the profit of each action, using a perceptron. Following on from the consideration of different economic models, a discrete model was developed so that software agents have a discrete environment to operate within. Within the model, it has been observed how supermarkets with differing behaviors affect a heterogeneous crowd of consumer agents. The model was implemented in Java with Python used to evaluate the results. 

The simulation displays good acceptance with real grocery market behavior, i.e. captures the performance of British retailers thus can be used to determine the impact of changes in their behavior on their competitors and consumers.Furthermore it can be used to provide insight into sustainability of volatile pricing strategies, providing a useful insight in volatility of British supermarket retail industry. 
\end{abstract}
\acknowledgements{
I would like to express my sincere gratitude to Dr Maria Polukarov for her guidance and support which provided me the freedom to take this research in the direction of my interest.\\
\\
I would also like to thank my family and friends for their encouragement and support. To those who quietly listened to my software complaints. To those who worked throughout the nights with me. To those who helped me write what I couldn't say. I cannot thank you enough.
}

\declaration{
I, Stefan Collier, declare that this dissertation and the work presented in it are my own and has been generated by me as the result of my own original research.\\
I confirm that:\\
1. This work was done wholly or mainly while in candidature for a degree at this University;\\
2. Where any part of this dissertation has previously been submitted for any other qualification at this University or any other institution, this has been clearly stated;\\
3. Where I have consulted the published work of others, this is always clearly attributed;\\
4. Where I have quoted from the work of others, the source is always given. With the exception of such quotations, this dissertation is entirely my own work;\\
5. I have acknowledged all main sources of help;\\
6. Where the thesis is based on work done by myself jointly with others, I have made clear exactly what was done by others and what I have contributed myself;\\
7. Either none of this work has been published before submission, or parts of this work have been published by :\\
\\
Stefan Collier\\
April 2016
}
\tableofcontents
\listoffigures
\listoftables

\mainmatter
%% ----------------------------------------------------------------
%\include{Introduction}
%\include{Conclusions}
\include{chapters/1Project/main}
\include{chapters/2Lit/main}
\include{chapters/3Design/HighLevel}
\include{chapters/3Design/InDepth}
\include{chapters/4Impl/main}

\include{chapters/5Experiments/1/main}
\include{chapters/5Experiments/2/main}
\include{chapters/5Experiments/3/main}
\include{chapters/5Experiments/4/main}

\include{chapters/6Conclusion/main}

\appendix
\include{appendix/AppendixB}
\include{appendix/D/main}
\include{appendix/AppendixC}

\backmatter
\bibliographystyle{ecs}
\bibliography{ECS}
\end{document}
%% ----------------------------------------------------------------

\include{chapters/3Design/HighLevel}
\include{chapters/3Design/InDepth}
 %% ----------------------------------------------------------------
%% Progress.tex
%% ---------------------------------------------------------------- 
\documentclass{ecsprogress}    % Use the progress Style
\graphicspath{{../figs/}}   % Location of your graphics files
    \usepackage{natbib}            % Use Natbib style for the refs.
\hypersetup{colorlinks=true}   % Set to false for black/white printing
\input{Definitions}            % Include your abbreviations



\usepackage{enumitem}% http://ctan.org/pkg/enumitem
\usepackage{multirow}
\usepackage{float}
\usepackage{amsmath}
\usepackage{multicol}
\usepackage{amssymb}
\usepackage[normalem]{ulem}
\useunder{\uline}{\ul}{}
\usepackage{wrapfig}


\usepackage[table,xcdraw]{xcolor}


%% ----------------------------------------------------------------
\begin{document}
\frontmatter
\title      {Heterogeneous Agent-based Model for Supermarket Competition}
\authors    {\texorpdfstring
             {\href{mailto:sc22g13@ecs.soton.ac.uk}{Stefan J. Collier}}
             {Stefan J. Collier}
            }
\addresses  {\groupname\\\deptname\\\univname}
\date       {\today}
\subject    {}
\keywords   {}
\supervisor {Dr. Maria Polukarov}
\examiner   {Professor Sheng Chen}

\maketitle
\begin{abstract}
This project aim was to model and analyse the effects of competitive pricing behaviors of grocery retailers on the British market. 

This was achieved by creating a multi-agent model, containing retailer and consumer agents. The heterogeneous crowd of retailers employs either a uniform pricing strategy or a ‘local price flexing’ strategy. The actions of these retailers are chosen by predicting the profit of each action, using a perceptron. Following on from the consideration of different economic models, a discrete model was developed so that software agents have a discrete environment to operate within. Within the model, it has been observed how supermarkets with differing behaviors affect a heterogeneous crowd of consumer agents. The model was implemented in Java with Python used to evaluate the results. 

The simulation displays good acceptance with real grocery market behavior, i.e. captures the performance of British retailers thus can be used to determine the impact of changes in their behavior on their competitors and consumers.Furthermore it can be used to provide insight into sustainability of volatile pricing strategies, providing a useful insight in volatility of British supermarket retail industry. 
\end{abstract}
\acknowledgements{
I would like to express my sincere gratitude to Dr Maria Polukarov for her guidance and support which provided me the freedom to take this research in the direction of my interest.\\
\\
I would also like to thank my family and friends for their encouragement and support. To those who quietly listened to my software complaints. To those who worked throughout the nights with me. To those who helped me write what I couldn't say. I cannot thank you enough.
}

\declaration{
I, Stefan Collier, declare that this dissertation and the work presented in it are my own and has been generated by me as the result of my own original research.\\
I confirm that:\\
1. This work was done wholly or mainly while in candidature for a degree at this University;\\
2. Where any part of this dissertation has previously been submitted for any other qualification at this University or any other institution, this has been clearly stated;\\
3. Where I have consulted the published work of others, this is always clearly attributed;\\
4. Where I have quoted from the work of others, the source is always given. With the exception of such quotations, this dissertation is entirely my own work;\\
5. I have acknowledged all main sources of help;\\
6. Where the thesis is based on work done by myself jointly with others, I have made clear exactly what was done by others and what I have contributed myself;\\
7. Either none of this work has been published before submission, or parts of this work have been published by :\\
\\
Stefan Collier\\
April 2016
}
\tableofcontents
\listoffigures
\listoftables

\mainmatter
%% ----------------------------------------------------------------
%\include{Introduction}
%\include{Conclusions}
\include{chapters/1Project/main}
\include{chapters/2Lit/main}
\include{chapters/3Design/HighLevel}
\include{chapters/3Design/InDepth}
\include{chapters/4Impl/main}

\include{chapters/5Experiments/1/main}
\include{chapters/5Experiments/2/main}
\include{chapters/5Experiments/3/main}
\include{chapters/5Experiments/4/main}

\include{chapters/6Conclusion/main}

\appendix
\include{appendix/AppendixB}
\include{appendix/D/main}
\include{appendix/AppendixC}

\backmatter
\bibliographystyle{ecs}
\bibliography{ECS}
\end{document}
%% ----------------------------------------------------------------


 %% ----------------------------------------------------------------
%% Progress.tex
%% ---------------------------------------------------------------- 
\documentclass{ecsprogress}    % Use the progress Style
\graphicspath{{../figs/}}   % Location of your graphics files
    \usepackage{natbib}            % Use Natbib style for the refs.
\hypersetup{colorlinks=true}   % Set to false for black/white printing
\input{Definitions}            % Include your abbreviations



\usepackage{enumitem}% http://ctan.org/pkg/enumitem
\usepackage{multirow}
\usepackage{float}
\usepackage{amsmath}
\usepackage{multicol}
\usepackage{amssymb}
\usepackage[normalem]{ulem}
\useunder{\uline}{\ul}{}
\usepackage{wrapfig}


\usepackage[table,xcdraw]{xcolor}


%% ----------------------------------------------------------------
\begin{document}
\frontmatter
\title      {Heterogeneous Agent-based Model for Supermarket Competition}
\authors    {\texorpdfstring
             {\href{mailto:sc22g13@ecs.soton.ac.uk}{Stefan J. Collier}}
             {Stefan J. Collier}
            }
\addresses  {\groupname\\\deptname\\\univname}
\date       {\today}
\subject    {}
\keywords   {}
\supervisor {Dr. Maria Polukarov}
\examiner   {Professor Sheng Chen}

\maketitle
\begin{abstract}
This project aim was to model and analyse the effects of competitive pricing behaviors of grocery retailers on the British market. 

This was achieved by creating a multi-agent model, containing retailer and consumer agents. The heterogeneous crowd of retailers employs either a uniform pricing strategy or a ‘local price flexing’ strategy. The actions of these retailers are chosen by predicting the profit of each action, using a perceptron. Following on from the consideration of different economic models, a discrete model was developed so that software agents have a discrete environment to operate within. Within the model, it has been observed how supermarkets with differing behaviors affect a heterogeneous crowd of consumer agents. The model was implemented in Java with Python used to evaluate the results. 

The simulation displays good acceptance with real grocery market behavior, i.e. captures the performance of British retailers thus can be used to determine the impact of changes in their behavior on their competitors and consumers.Furthermore it can be used to provide insight into sustainability of volatile pricing strategies, providing a useful insight in volatility of British supermarket retail industry. 
\end{abstract}
\acknowledgements{
I would like to express my sincere gratitude to Dr Maria Polukarov for her guidance and support which provided me the freedom to take this research in the direction of my interest.\\
\\
I would also like to thank my family and friends for their encouragement and support. To those who quietly listened to my software complaints. To those who worked throughout the nights with me. To those who helped me write what I couldn't say. I cannot thank you enough.
}

\declaration{
I, Stefan Collier, declare that this dissertation and the work presented in it are my own and has been generated by me as the result of my own original research.\\
I confirm that:\\
1. This work was done wholly or mainly while in candidature for a degree at this University;\\
2. Where any part of this dissertation has previously been submitted for any other qualification at this University or any other institution, this has been clearly stated;\\
3. Where I have consulted the published work of others, this is always clearly attributed;\\
4. Where I have quoted from the work of others, the source is always given. With the exception of such quotations, this dissertation is entirely my own work;\\
5. I have acknowledged all main sources of help;\\
6. Where the thesis is based on work done by myself jointly with others, I have made clear exactly what was done by others and what I have contributed myself;\\
7. Either none of this work has been published before submission, or parts of this work have been published by :\\
\\
Stefan Collier\\
April 2016
}
\tableofcontents
\listoffigures
\listoftables

\mainmatter
%% ----------------------------------------------------------------
%\include{Introduction}
%\include{Conclusions}
\include{chapters/1Project/main}
\include{chapters/2Lit/main}
\include{chapters/3Design/HighLevel}
\include{chapters/3Design/InDepth}
\include{chapters/4Impl/main}

\include{chapters/5Experiments/1/main}
\include{chapters/5Experiments/2/main}
\include{chapters/5Experiments/3/main}
\include{chapters/5Experiments/4/main}

\include{chapters/6Conclusion/main}

\appendix
\include{appendix/AppendixB}
\include{appendix/D/main}
\include{appendix/AppendixC}

\backmatter
\bibliographystyle{ecs}
\bibliography{ECS}
\end{document}
%% ----------------------------------------------------------------

 %% ----------------------------------------------------------------
%% Progress.tex
%% ---------------------------------------------------------------- 
\documentclass{ecsprogress}    % Use the progress Style
\graphicspath{{../figs/}}   % Location of your graphics files
    \usepackage{natbib}            % Use Natbib style for the refs.
\hypersetup{colorlinks=true}   % Set to false for black/white printing
\input{Definitions}            % Include your abbreviations



\usepackage{enumitem}% http://ctan.org/pkg/enumitem
\usepackage{multirow}
\usepackage{float}
\usepackage{amsmath}
\usepackage{multicol}
\usepackage{amssymb}
\usepackage[normalem]{ulem}
\useunder{\uline}{\ul}{}
\usepackage{wrapfig}


\usepackage[table,xcdraw]{xcolor}


%% ----------------------------------------------------------------
\begin{document}
\frontmatter
\title      {Heterogeneous Agent-based Model for Supermarket Competition}
\authors    {\texorpdfstring
             {\href{mailto:sc22g13@ecs.soton.ac.uk}{Stefan J. Collier}}
             {Stefan J. Collier}
            }
\addresses  {\groupname\\\deptname\\\univname}
\date       {\today}
\subject    {}
\keywords   {}
\supervisor {Dr. Maria Polukarov}
\examiner   {Professor Sheng Chen}

\maketitle
\begin{abstract}
This project aim was to model and analyse the effects of competitive pricing behaviors of grocery retailers on the British market. 

This was achieved by creating a multi-agent model, containing retailer and consumer agents. The heterogeneous crowd of retailers employs either a uniform pricing strategy or a ‘local price flexing’ strategy. The actions of these retailers are chosen by predicting the profit of each action, using a perceptron. Following on from the consideration of different economic models, a discrete model was developed so that software agents have a discrete environment to operate within. Within the model, it has been observed how supermarkets with differing behaviors affect a heterogeneous crowd of consumer agents. The model was implemented in Java with Python used to evaluate the results. 

The simulation displays good acceptance with real grocery market behavior, i.e. captures the performance of British retailers thus can be used to determine the impact of changes in their behavior on their competitors and consumers.Furthermore it can be used to provide insight into sustainability of volatile pricing strategies, providing a useful insight in volatility of British supermarket retail industry. 
\end{abstract}
\acknowledgements{
I would like to express my sincere gratitude to Dr Maria Polukarov for her guidance and support which provided me the freedom to take this research in the direction of my interest.\\
\\
I would also like to thank my family and friends for their encouragement and support. To those who quietly listened to my software complaints. To those who worked throughout the nights with me. To those who helped me write what I couldn't say. I cannot thank you enough.
}

\declaration{
I, Stefan Collier, declare that this dissertation and the work presented in it are my own and has been generated by me as the result of my own original research.\\
I confirm that:\\
1. This work was done wholly or mainly while in candidature for a degree at this University;\\
2. Where any part of this dissertation has previously been submitted for any other qualification at this University or any other institution, this has been clearly stated;\\
3. Where I have consulted the published work of others, this is always clearly attributed;\\
4. Where I have quoted from the work of others, the source is always given. With the exception of such quotations, this dissertation is entirely my own work;\\
5. I have acknowledged all main sources of help;\\
6. Where the thesis is based on work done by myself jointly with others, I have made clear exactly what was done by others and what I have contributed myself;\\
7. Either none of this work has been published before submission, or parts of this work have been published by :\\
\\
Stefan Collier\\
April 2016
}
\tableofcontents
\listoffigures
\listoftables

\mainmatter
%% ----------------------------------------------------------------
%\include{Introduction}
%\include{Conclusions}
\include{chapters/1Project/main}
\include{chapters/2Lit/main}
\include{chapters/3Design/HighLevel}
\include{chapters/3Design/InDepth}
\include{chapters/4Impl/main}

\include{chapters/5Experiments/1/main}
\include{chapters/5Experiments/2/main}
\include{chapters/5Experiments/3/main}
\include{chapters/5Experiments/4/main}

\include{chapters/6Conclusion/main}

\appendix
\include{appendix/AppendixB}
\include{appendix/D/main}
\include{appendix/AppendixC}

\backmatter
\bibliographystyle{ecs}
\bibliography{ECS}
\end{document}
%% ----------------------------------------------------------------

 %% ----------------------------------------------------------------
%% Progress.tex
%% ---------------------------------------------------------------- 
\documentclass{ecsprogress}    % Use the progress Style
\graphicspath{{../figs/}}   % Location of your graphics files
    \usepackage{natbib}            % Use Natbib style for the refs.
\hypersetup{colorlinks=true}   % Set to false for black/white printing
\input{Definitions}            % Include your abbreviations



\usepackage{enumitem}% http://ctan.org/pkg/enumitem
\usepackage{multirow}
\usepackage{float}
\usepackage{amsmath}
\usepackage{multicol}
\usepackage{amssymb}
\usepackage[normalem]{ulem}
\useunder{\uline}{\ul}{}
\usepackage{wrapfig}


\usepackage[table,xcdraw]{xcolor}


%% ----------------------------------------------------------------
\begin{document}
\frontmatter
\title      {Heterogeneous Agent-based Model for Supermarket Competition}
\authors    {\texorpdfstring
             {\href{mailto:sc22g13@ecs.soton.ac.uk}{Stefan J. Collier}}
             {Stefan J. Collier}
            }
\addresses  {\groupname\\\deptname\\\univname}
\date       {\today}
\subject    {}
\keywords   {}
\supervisor {Dr. Maria Polukarov}
\examiner   {Professor Sheng Chen}

\maketitle
\begin{abstract}
This project aim was to model and analyse the effects of competitive pricing behaviors of grocery retailers on the British market. 

This was achieved by creating a multi-agent model, containing retailer and consumer agents. The heterogeneous crowd of retailers employs either a uniform pricing strategy or a ‘local price flexing’ strategy. The actions of these retailers are chosen by predicting the profit of each action, using a perceptron. Following on from the consideration of different economic models, a discrete model was developed so that software agents have a discrete environment to operate within. Within the model, it has been observed how supermarkets with differing behaviors affect a heterogeneous crowd of consumer agents. The model was implemented in Java with Python used to evaluate the results. 

The simulation displays good acceptance with real grocery market behavior, i.e. captures the performance of British retailers thus can be used to determine the impact of changes in their behavior on their competitors and consumers.Furthermore it can be used to provide insight into sustainability of volatile pricing strategies, providing a useful insight in volatility of British supermarket retail industry. 
\end{abstract}
\acknowledgements{
I would like to express my sincere gratitude to Dr Maria Polukarov for her guidance and support which provided me the freedom to take this research in the direction of my interest.\\
\\
I would also like to thank my family and friends for their encouragement and support. To those who quietly listened to my software complaints. To those who worked throughout the nights with me. To those who helped me write what I couldn't say. I cannot thank you enough.
}

\declaration{
I, Stefan Collier, declare that this dissertation and the work presented in it are my own and has been generated by me as the result of my own original research.\\
I confirm that:\\
1. This work was done wholly or mainly while in candidature for a degree at this University;\\
2. Where any part of this dissertation has previously been submitted for any other qualification at this University or any other institution, this has been clearly stated;\\
3. Where I have consulted the published work of others, this is always clearly attributed;\\
4. Where I have quoted from the work of others, the source is always given. With the exception of such quotations, this dissertation is entirely my own work;\\
5. I have acknowledged all main sources of help;\\
6. Where the thesis is based on work done by myself jointly with others, I have made clear exactly what was done by others and what I have contributed myself;\\
7. Either none of this work has been published before submission, or parts of this work have been published by :\\
\\
Stefan Collier\\
April 2016
}
\tableofcontents
\listoffigures
\listoftables

\mainmatter
%% ----------------------------------------------------------------
%\include{Introduction}
%\include{Conclusions}
\include{chapters/1Project/main}
\include{chapters/2Lit/main}
\include{chapters/3Design/HighLevel}
\include{chapters/3Design/InDepth}
\include{chapters/4Impl/main}

\include{chapters/5Experiments/1/main}
\include{chapters/5Experiments/2/main}
\include{chapters/5Experiments/3/main}
\include{chapters/5Experiments/4/main}

\include{chapters/6Conclusion/main}

\appendix
\include{appendix/AppendixB}
\include{appendix/D/main}
\include{appendix/AppendixC}

\backmatter
\bibliographystyle{ecs}
\bibliography{ECS}
\end{document}
%% ----------------------------------------------------------------

 %% ----------------------------------------------------------------
%% Progress.tex
%% ---------------------------------------------------------------- 
\documentclass{ecsprogress}    % Use the progress Style
\graphicspath{{../figs/}}   % Location of your graphics files
    \usepackage{natbib}            % Use Natbib style for the refs.
\hypersetup{colorlinks=true}   % Set to false for black/white printing
\input{Definitions}            % Include your abbreviations



\usepackage{enumitem}% http://ctan.org/pkg/enumitem
\usepackage{multirow}
\usepackage{float}
\usepackage{amsmath}
\usepackage{multicol}
\usepackage{amssymb}
\usepackage[normalem]{ulem}
\useunder{\uline}{\ul}{}
\usepackage{wrapfig}


\usepackage[table,xcdraw]{xcolor}


%% ----------------------------------------------------------------
\begin{document}
\frontmatter
\title      {Heterogeneous Agent-based Model for Supermarket Competition}
\authors    {\texorpdfstring
             {\href{mailto:sc22g13@ecs.soton.ac.uk}{Stefan J. Collier}}
             {Stefan J. Collier}
            }
\addresses  {\groupname\\\deptname\\\univname}
\date       {\today}
\subject    {}
\keywords   {}
\supervisor {Dr. Maria Polukarov}
\examiner   {Professor Sheng Chen}

\maketitle
\begin{abstract}
This project aim was to model and analyse the effects of competitive pricing behaviors of grocery retailers on the British market. 

This was achieved by creating a multi-agent model, containing retailer and consumer agents. The heterogeneous crowd of retailers employs either a uniform pricing strategy or a ‘local price flexing’ strategy. The actions of these retailers are chosen by predicting the profit of each action, using a perceptron. Following on from the consideration of different economic models, a discrete model was developed so that software agents have a discrete environment to operate within. Within the model, it has been observed how supermarkets with differing behaviors affect a heterogeneous crowd of consumer agents. The model was implemented in Java with Python used to evaluate the results. 

The simulation displays good acceptance with real grocery market behavior, i.e. captures the performance of British retailers thus can be used to determine the impact of changes in their behavior on their competitors and consumers.Furthermore it can be used to provide insight into sustainability of volatile pricing strategies, providing a useful insight in volatility of British supermarket retail industry. 
\end{abstract}
\acknowledgements{
I would like to express my sincere gratitude to Dr Maria Polukarov for her guidance and support which provided me the freedom to take this research in the direction of my interest.\\
\\
I would also like to thank my family and friends for their encouragement and support. To those who quietly listened to my software complaints. To those who worked throughout the nights with me. To those who helped me write what I couldn't say. I cannot thank you enough.
}

\declaration{
I, Stefan Collier, declare that this dissertation and the work presented in it are my own and has been generated by me as the result of my own original research.\\
I confirm that:\\
1. This work was done wholly or mainly while in candidature for a degree at this University;\\
2. Where any part of this dissertation has previously been submitted for any other qualification at this University or any other institution, this has been clearly stated;\\
3. Where I have consulted the published work of others, this is always clearly attributed;\\
4. Where I have quoted from the work of others, the source is always given. With the exception of such quotations, this dissertation is entirely my own work;\\
5. I have acknowledged all main sources of help;\\
6. Where the thesis is based on work done by myself jointly with others, I have made clear exactly what was done by others and what I have contributed myself;\\
7. Either none of this work has been published before submission, or parts of this work have been published by :\\
\\
Stefan Collier\\
April 2016
}
\tableofcontents
\listoffigures
\listoftables

\mainmatter
%% ----------------------------------------------------------------
%\include{Introduction}
%\include{Conclusions}
\include{chapters/1Project/main}
\include{chapters/2Lit/main}
\include{chapters/3Design/HighLevel}
\include{chapters/3Design/InDepth}
\include{chapters/4Impl/main}

\include{chapters/5Experiments/1/main}
\include{chapters/5Experiments/2/main}
\include{chapters/5Experiments/3/main}
\include{chapters/5Experiments/4/main}

\include{chapters/6Conclusion/main}

\appendix
\include{appendix/AppendixB}
\include{appendix/D/main}
\include{appendix/AppendixC}

\backmatter
\bibliographystyle{ecs}
\bibliography{ECS}
\end{document}
%% ----------------------------------------------------------------


 %% ----------------------------------------------------------------
%% Progress.tex
%% ---------------------------------------------------------------- 
\documentclass{ecsprogress}    % Use the progress Style
\graphicspath{{../figs/}}   % Location of your graphics files
    \usepackage{natbib}            % Use Natbib style for the refs.
\hypersetup{colorlinks=true}   % Set to false for black/white printing
\input{Definitions}            % Include your abbreviations



\usepackage{enumitem}% http://ctan.org/pkg/enumitem
\usepackage{multirow}
\usepackage{float}
\usepackage{amsmath}
\usepackage{multicol}
\usepackage{amssymb}
\usepackage[normalem]{ulem}
\useunder{\uline}{\ul}{}
\usepackage{wrapfig}


\usepackage[table,xcdraw]{xcolor}


%% ----------------------------------------------------------------
\begin{document}
\frontmatter
\title      {Heterogeneous Agent-based Model for Supermarket Competition}
\authors    {\texorpdfstring
             {\href{mailto:sc22g13@ecs.soton.ac.uk}{Stefan J. Collier}}
             {Stefan J. Collier}
            }
\addresses  {\groupname\\\deptname\\\univname}
\date       {\today}
\subject    {}
\keywords   {}
\supervisor {Dr. Maria Polukarov}
\examiner   {Professor Sheng Chen}

\maketitle
\begin{abstract}
This project aim was to model and analyse the effects of competitive pricing behaviors of grocery retailers on the British market. 

This was achieved by creating a multi-agent model, containing retailer and consumer agents. The heterogeneous crowd of retailers employs either a uniform pricing strategy or a ‘local price flexing’ strategy. The actions of these retailers are chosen by predicting the profit of each action, using a perceptron. Following on from the consideration of different economic models, a discrete model was developed so that software agents have a discrete environment to operate within. Within the model, it has been observed how supermarkets with differing behaviors affect a heterogeneous crowd of consumer agents. The model was implemented in Java with Python used to evaluate the results. 

The simulation displays good acceptance with real grocery market behavior, i.e. captures the performance of British retailers thus can be used to determine the impact of changes in their behavior on their competitors and consumers.Furthermore it can be used to provide insight into sustainability of volatile pricing strategies, providing a useful insight in volatility of British supermarket retail industry. 
\end{abstract}
\acknowledgements{
I would like to express my sincere gratitude to Dr Maria Polukarov for her guidance and support which provided me the freedom to take this research in the direction of my interest.\\
\\
I would also like to thank my family and friends for their encouragement and support. To those who quietly listened to my software complaints. To those who worked throughout the nights with me. To those who helped me write what I couldn't say. I cannot thank you enough.
}

\declaration{
I, Stefan Collier, declare that this dissertation and the work presented in it are my own and has been generated by me as the result of my own original research.\\
I confirm that:\\
1. This work was done wholly or mainly while in candidature for a degree at this University;\\
2. Where any part of this dissertation has previously been submitted for any other qualification at this University or any other institution, this has been clearly stated;\\
3. Where I have consulted the published work of others, this is always clearly attributed;\\
4. Where I have quoted from the work of others, the source is always given. With the exception of such quotations, this dissertation is entirely my own work;\\
5. I have acknowledged all main sources of help;\\
6. Where the thesis is based on work done by myself jointly with others, I have made clear exactly what was done by others and what I have contributed myself;\\
7. Either none of this work has been published before submission, or parts of this work have been published by :\\
\\
Stefan Collier\\
April 2016
}
\tableofcontents
\listoffigures
\listoftables

\mainmatter
%% ----------------------------------------------------------------
%\include{Introduction}
%\include{Conclusions}
\include{chapters/1Project/main}
\include{chapters/2Lit/main}
\include{chapters/3Design/HighLevel}
\include{chapters/3Design/InDepth}
\include{chapters/4Impl/main}

\include{chapters/5Experiments/1/main}
\include{chapters/5Experiments/2/main}
\include{chapters/5Experiments/3/main}
\include{chapters/5Experiments/4/main}

\include{chapters/6Conclusion/main}

\appendix
\include{appendix/AppendixB}
\include{appendix/D/main}
\include{appendix/AppendixC}

\backmatter
\bibliographystyle{ecs}
\bibliography{ECS}
\end{document}
%% ----------------------------------------------------------------


\appendix
\include{appendix/AppendixB}
 %% ----------------------------------------------------------------
%% Progress.tex
%% ---------------------------------------------------------------- 
\documentclass{ecsprogress}    % Use the progress Style
\graphicspath{{../figs/}}   % Location of your graphics files
    \usepackage{natbib}            % Use Natbib style for the refs.
\hypersetup{colorlinks=true}   % Set to false for black/white printing
\input{Definitions}            % Include your abbreviations



\usepackage{enumitem}% http://ctan.org/pkg/enumitem
\usepackage{multirow}
\usepackage{float}
\usepackage{amsmath}
\usepackage{multicol}
\usepackage{amssymb}
\usepackage[normalem]{ulem}
\useunder{\uline}{\ul}{}
\usepackage{wrapfig}


\usepackage[table,xcdraw]{xcolor}


%% ----------------------------------------------------------------
\begin{document}
\frontmatter
\title      {Heterogeneous Agent-based Model for Supermarket Competition}
\authors    {\texorpdfstring
             {\href{mailto:sc22g13@ecs.soton.ac.uk}{Stefan J. Collier}}
             {Stefan J. Collier}
            }
\addresses  {\groupname\\\deptname\\\univname}
\date       {\today}
\subject    {}
\keywords   {}
\supervisor {Dr. Maria Polukarov}
\examiner   {Professor Sheng Chen}

\maketitle
\begin{abstract}
This project aim was to model and analyse the effects of competitive pricing behaviors of grocery retailers on the British market. 

This was achieved by creating a multi-agent model, containing retailer and consumer agents. The heterogeneous crowd of retailers employs either a uniform pricing strategy or a ‘local price flexing’ strategy. The actions of these retailers are chosen by predicting the profit of each action, using a perceptron. Following on from the consideration of different economic models, a discrete model was developed so that software agents have a discrete environment to operate within. Within the model, it has been observed how supermarkets with differing behaviors affect a heterogeneous crowd of consumer agents. The model was implemented in Java with Python used to evaluate the results. 

The simulation displays good acceptance with real grocery market behavior, i.e. captures the performance of British retailers thus can be used to determine the impact of changes in their behavior on their competitors and consumers.Furthermore it can be used to provide insight into sustainability of volatile pricing strategies, providing a useful insight in volatility of British supermarket retail industry. 
\end{abstract}
\acknowledgements{
I would like to express my sincere gratitude to Dr Maria Polukarov for her guidance and support which provided me the freedom to take this research in the direction of my interest.\\
\\
I would also like to thank my family and friends for their encouragement and support. To those who quietly listened to my software complaints. To those who worked throughout the nights with me. To those who helped me write what I couldn't say. I cannot thank you enough.
}

\declaration{
I, Stefan Collier, declare that this dissertation and the work presented in it are my own and has been generated by me as the result of my own original research.\\
I confirm that:\\
1. This work was done wholly or mainly while in candidature for a degree at this University;\\
2. Where any part of this dissertation has previously been submitted for any other qualification at this University or any other institution, this has been clearly stated;\\
3. Where I have consulted the published work of others, this is always clearly attributed;\\
4. Where I have quoted from the work of others, the source is always given. With the exception of such quotations, this dissertation is entirely my own work;\\
5. I have acknowledged all main sources of help;\\
6. Where the thesis is based on work done by myself jointly with others, I have made clear exactly what was done by others and what I have contributed myself;\\
7. Either none of this work has been published before submission, or parts of this work have been published by :\\
\\
Stefan Collier\\
April 2016
}
\tableofcontents
\listoffigures
\listoftables

\mainmatter
%% ----------------------------------------------------------------
%\include{Introduction}
%\include{Conclusions}
\include{chapters/1Project/main}
\include{chapters/2Lit/main}
\include{chapters/3Design/HighLevel}
\include{chapters/3Design/InDepth}
\include{chapters/4Impl/main}

\include{chapters/5Experiments/1/main}
\include{chapters/5Experiments/2/main}
\include{chapters/5Experiments/3/main}
\include{chapters/5Experiments/4/main}

\include{chapters/6Conclusion/main}

\appendix
\include{appendix/AppendixB}
\include{appendix/D/main}
\include{appendix/AppendixC}

\backmatter
\bibliographystyle{ecs}
\bibliography{ECS}
\end{document}
%% ----------------------------------------------------------------

\include{appendix/AppendixC}

\backmatter
\bibliographystyle{ecs}
\bibliography{ECS}
\end{document}
%% ----------------------------------------------------------------

 %% ----------------------------------------------------------------
%% Progress.tex
%% ---------------------------------------------------------------- 
\documentclass{ecsprogress}    % Use the progress Style
\graphicspath{{../figs/}}   % Location of your graphics files
    \usepackage{natbib}            % Use Natbib style for the refs.
\hypersetup{colorlinks=true}   % Set to false for black/white printing
\input{Definitions}            % Include your abbreviations



\usepackage{enumitem}% http://ctan.org/pkg/enumitem
\usepackage{multirow}
\usepackage{float}
\usepackage{amsmath}
\usepackage{multicol}
\usepackage{amssymb}
\usepackage[normalem]{ulem}
\useunder{\uline}{\ul}{}
\usepackage{wrapfig}


\usepackage[table,xcdraw]{xcolor}


%% ----------------------------------------------------------------
\begin{document}
\frontmatter
\title      {Heterogeneous Agent-based Model for Supermarket Competition}
\authors    {\texorpdfstring
             {\href{mailto:sc22g13@ecs.soton.ac.uk}{Stefan J. Collier}}
             {Stefan J. Collier}
            }
\addresses  {\groupname\\\deptname\\\univname}
\date       {\today}
\subject    {}
\keywords   {}
\supervisor {Dr. Maria Polukarov}
\examiner   {Professor Sheng Chen}

\maketitle
\begin{abstract}
This project aim was to model and analyse the effects of competitive pricing behaviors of grocery retailers on the British market. 

This was achieved by creating a multi-agent model, containing retailer and consumer agents. The heterogeneous crowd of retailers employs either a uniform pricing strategy or a ‘local price flexing’ strategy. The actions of these retailers are chosen by predicting the profit of each action, using a perceptron. Following on from the consideration of different economic models, a discrete model was developed so that software agents have a discrete environment to operate within. Within the model, it has been observed how supermarkets with differing behaviors affect a heterogeneous crowd of consumer agents. The model was implemented in Java with Python used to evaluate the results. 

The simulation displays good acceptance with real grocery market behavior, i.e. captures the performance of British retailers thus can be used to determine the impact of changes in their behavior on their competitors and consumers.Furthermore it can be used to provide insight into sustainability of volatile pricing strategies, providing a useful insight in volatility of British supermarket retail industry. 
\end{abstract}
\acknowledgements{
I would like to express my sincere gratitude to Dr Maria Polukarov for her guidance and support which provided me the freedom to take this research in the direction of my interest.\\
\\
I would also like to thank my family and friends for their encouragement and support. To those who quietly listened to my software complaints. To those who worked throughout the nights with me. To those who helped me write what I couldn't say. I cannot thank you enough.
}

\declaration{
I, Stefan Collier, declare that this dissertation and the work presented in it are my own and has been generated by me as the result of my own original research.\\
I confirm that:\\
1. This work was done wholly or mainly while in candidature for a degree at this University;\\
2. Where any part of this dissertation has previously been submitted for any other qualification at this University or any other institution, this has been clearly stated;\\
3. Where I have consulted the published work of others, this is always clearly attributed;\\
4. Where I have quoted from the work of others, the source is always given. With the exception of such quotations, this dissertation is entirely my own work;\\
5. I have acknowledged all main sources of help;\\
6. Where the thesis is based on work done by myself jointly with others, I have made clear exactly what was done by others and what I have contributed myself;\\
7. Either none of this work has been published before submission, or parts of this work have been published by :\\
\\
Stefan Collier\\
April 2016
}
\tableofcontents
\listoffigures
\listoftables

\mainmatter
%% ----------------------------------------------------------------
%\include{Introduction}
%\include{Conclusions}
 %% ----------------------------------------------------------------
%% Progress.tex
%% ---------------------------------------------------------------- 
\documentclass{ecsprogress}    % Use the progress Style
\graphicspath{{../figs/}}   % Location of your graphics files
    \usepackage{natbib}            % Use Natbib style for the refs.
\hypersetup{colorlinks=true}   % Set to false for black/white printing
\input{Definitions}            % Include your abbreviations



\usepackage{enumitem}% http://ctan.org/pkg/enumitem
\usepackage{multirow}
\usepackage{float}
\usepackage{amsmath}
\usepackage{multicol}
\usepackage{amssymb}
\usepackage[normalem]{ulem}
\useunder{\uline}{\ul}{}
\usepackage{wrapfig}


\usepackage[table,xcdraw]{xcolor}


%% ----------------------------------------------------------------
\begin{document}
\frontmatter
\title      {Heterogeneous Agent-based Model for Supermarket Competition}
\authors    {\texorpdfstring
             {\href{mailto:sc22g13@ecs.soton.ac.uk}{Stefan J. Collier}}
             {Stefan J. Collier}
            }
\addresses  {\groupname\\\deptname\\\univname}
\date       {\today}
\subject    {}
\keywords   {}
\supervisor {Dr. Maria Polukarov}
\examiner   {Professor Sheng Chen}

\maketitle
\begin{abstract}
This project aim was to model and analyse the effects of competitive pricing behaviors of grocery retailers on the British market. 

This was achieved by creating a multi-agent model, containing retailer and consumer agents. The heterogeneous crowd of retailers employs either a uniform pricing strategy or a ‘local price flexing’ strategy. The actions of these retailers are chosen by predicting the profit of each action, using a perceptron. Following on from the consideration of different economic models, a discrete model was developed so that software agents have a discrete environment to operate within. Within the model, it has been observed how supermarkets with differing behaviors affect a heterogeneous crowd of consumer agents. The model was implemented in Java with Python used to evaluate the results. 

The simulation displays good acceptance with real grocery market behavior, i.e. captures the performance of British retailers thus can be used to determine the impact of changes in their behavior on their competitors and consumers.Furthermore it can be used to provide insight into sustainability of volatile pricing strategies, providing a useful insight in volatility of British supermarket retail industry. 
\end{abstract}
\acknowledgements{
I would like to express my sincere gratitude to Dr Maria Polukarov for her guidance and support which provided me the freedom to take this research in the direction of my interest.\\
\\
I would also like to thank my family and friends for their encouragement and support. To those who quietly listened to my software complaints. To those who worked throughout the nights with me. To those who helped me write what I couldn't say. I cannot thank you enough.
}

\declaration{
I, Stefan Collier, declare that this dissertation and the work presented in it are my own and has been generated by me as the result of my own original research.\\
I confirm that:\\
1. This work was done wholly or mainly while in candidature for a degree at this University;\\
2. Where any part of this dissertation has previously been submitted for any other qualification at this University or any other institution, this has been clearly stated;\\
3. Where I have consulted the published work of others, this is always clearly attributed;\\
4. Where I have quoted from the work of others, the source is always given. With the exception of such quotations, this dissertation is entirely my own work;\\
5. I have acknowledged all main sources of help;\\
6. Where the thesis is based on work done by myself jointly with others, I have made clear exactly what was done by others and what I have contributed myself;\\
7. Either none of this work has been published before submission, or parts of this work have been published by :\\
\\
Stefan Collier\\
April 2016
}
\tableofcontents
\listoffigures
\listoftables

\mainmatter
%% ----------------------------------------------------------------
%\include{Introduction}
%\include{Conclusions}
\include{chapters/1Project/main}
\include{chapters/2Lit/main}
\include{chapters/3Design/HighLevel}
\include{chapters/3Design/InDepth}
\include{chapters/4Impl/main}

\include{chapters/5Experiments/1/main}
\include{chapters/5Experiments/2/main}
\include{chapters/5Experiments/3/main}
\include{chapters/5Experiments/4/main}

\include{chapters/6Conclusion/main}

\appendix
\include{appendix/AppendixB}
\include{appendix/D/main}
\include{appendix/AppendixC}

\backmatter
\bibliographystyle{ecs}
\bibliography{ECS}
\end{document}
%% ----------------------------------------------------------------

 %% ----------------------------------------------------------------
%% Progress.tex
%% ---------------------------------------------------------------- 
\documentclass{ecsprogress}    % Use the progress Style
\graphicspath{{../figs/}}   % Location of your graphics files
    \usepackage{natbib}            % Use Natbib style for the refs.
\hypersetup{colorlinks=true}   % Set to false for black/white printing
\input{Definitions}            % Include your abbreviations



\usepackage{enumitem}% http://ctan.org/pkg/enumitem
\usepackage{multirow}
\usepackage{float}
\usepackage{amsmath}
\usepackage{multicol}
\usepackage{amssymb}
\usepackage[normalem]{ulem}
\useunder{\uline}{\ul}{}
\usepackage{wrapfig}


\usepackage[table,xcdraw]{xcolor}


%% ----------------------------------------------------------------
\begin{document}
\frontmatter
\title      {Heterogeneous Agent-based Model for Supermarket Competition}
\authors    {\texorpdfstring
             {\href{mailto:sc22g13@ecs.soton.ac.uk}{Stefan J. Collier}}
             {Stefan J. Collier}
            }
\addresses  {\groupname\\\deptname\\\univname}
\date       {\today}
\subject    {}
\keywords   {}
\supervisor {Dr. Maria Polukarov}
\examiner   {Professor Sheng Chen}

\maketitle
\begin{abstract}
This project aim was to model and analyse the effects of competitive pricing behaviors of grocery retailers on the British market. 

This was achieved by creating a multi-agent model, containing retailer and consumer agents. The heterogeneous crowd of retailers employs either a uniform pricing strategy or a ‘local price flexing’ strategy. The actions of these retailers are chosen by predicting the profit of each action, using a perceptron. Following on from the consideration of different economic models, a discrete model was developed so that software agents have a discrete environment to operate within. Within the model, it has been observed how supermarkets with differing behaviors affect a heterogeneous crowd of consumer agents. The model was implemented in Java with Python used to evaluate the results. 

The simulation displays good acceptance with real grocery market behavior, i.e. captures the performance of British retailers thus can be used to determine the impact of changes in their behavior on their competitors and consumers.Furthermore it can be used to provide insight into sustainability of volatile pricing strategies, providing a useful insight in volatility of British supermarket retail industry. 
\end{abstract}
\acknowledgements{
I would like to express my sincere gratitude to Dr Maria Polukarov for her guidance and support which provided me the freedom to take this research in the direction of my interest.\\
\\
I would also like to thank my family and friends for their encouragement and support. To those who quietly listened to my software complaints. To those who worked throughout the nights with me. To those who helped me write what I couldn't say. I cannot thank you enough.
}

\declaration{
I, Stefan Collier, declare that this dissertation and the work presented in it are my own and has been generated by me as the result of my own original research.\\
I confirm that:\\
1. This work was done wholly or mainly while in candidature for a degree at this University;\\
2. Where any part of this dissertation has previously been submitted for any other qualification at this University or any other institution, this has been clearly stated;\\
3. Where I have consulted the published work of others, this is always clearly attributed;\\
4. Where I have quoted from the work of others, the source is always given. With the exception of such quotations, this dissertation is entirely my own work;\\
5. I have acknowledged all main sources of help;\\
6. Where the thesis is based on work done by myself jointly with others, I have made clear exactly what was done by others and what I have contributed myself;\\
7. Either none of this work has been published before submission, or parts of this work have been published by :\\
\\
Stefan Collier\\
April 2016
}
\tableofcontents
\listoffigures
\listoftables

\mainmatter
%% ----------------------------------------------------------------
%\include{Introduction}
%\include{Conclusions}
\include{chapters/1Project/main}
\include{chapters/2Lit/main}
\include{chapters/3Design/HighLevel}
\include{chapters/3Design/InDepth}
\include{chapters/4Impl/main}

\include{chapters/5Experiments/1/main}
\include{chapters/5Experiments/2/main}
\include{chapters/5Experiments/3/main}
\include{chapters/5Experiments/4/main}

\include{chapters/6Conclusion/main}

\appendix
\include{appendix/AppendixB}
\include{appendix/D/main}
\include{appendix/AppendixC}

\backmatter
\bibliographystyle{ecs}
\bibliography{ECS}
\end{document}
%% ----------------------------------------------------------------

\include{chapters/3Design/HighLevel}
\include{chapters/3Design/InDepth}
 %% ----------------------------------------------------------------
%% Progress.tex
%% ---------------------------------------------------------------- 
\documentclass{ecsprogress}    % Use the progress Style
\graphicspath{{../figs/}}   % Location of your graphics files
    \usepackage{natbib}            % Use Natbib style for the refs.
\hypersetup{colorlinks=true}   % Set to false for black/white printing
\input{Definitions}            % Include your abbreviations



\usepackage{enumitem}% http://ctan.org/pkg/enumitem
\usepackage{multirow}
\usepackage{float}
\usepackage{amsmath}
\usepackage{multicol}
\usepackage{amssymb}
\usepackage[normalem]{ulem}
\useunder{\uline}{\ul}{}
\usepackage{wrapfig}


\usepackage[table,xcdraw]{xcolor}


%% ----------------------------------------------------------------
\begin{document}
\frontmatter
\title      {Heterogeneous Agent-based Model for Supermarket Competition}
\authors    {\texorpdfstring
             {\href{mailto:sc22g13@ecs.soton.ac.uk}{Stefan J. Collier}}
             {Stefan J. Collier}
            }
\addresses  {\groupname\\\deptname\\\univname}
\date       {\today}
\subject    {}
\keywords   {}
\supervisor {Dr. Maria Polukarov}
\examiner   {Professor Sheng Chen}

\maketitle
\begin{abstract}
This project aim was to model and analyse the effects of competitive pricing behaviors of grocery retailers on the British market. 

This was achieved by creating a multi-agent model, containing retailer and consumer agents. The heterogeneous crowd of retailers employs either a uniform pricing strategy or a ‘local price flexing’ strategy. The actions of these retailers are chosen by predicting the profit of each action, using a perceptron. Following on from the consideration of different economic models, a discrete model was developed so that software agents have a discrete environment to operate within. Within the model, it has been observed how supermarkets with differing behaviors affect a heterogeneous crowd of consumer agents. The model was implemented in Java with Python used to evaluate the results. 

The simulation displays good acceptance with real grocery market behavior, i.e. captures the performance of British retailers thus can be used to determine the impact of changes in their behavior on their competitors and consumers.Furthermore it can be used to provide insight into sustainability of volatile pricing strategies, providing a useful insight in volatility of British supermarket retail industry. 
\end{abstract}
\acknowledgements{
I would like to express my sincere gratitude to Dr Maria Polukarov for her guidance and support which provided me the freedom to take this research in the direction of my interest.\\
\\
I would also like to thank my family and friends for their encouragement and support. To those who quietly listened to my software complaints. To those who worked throughout the nights with me. To those who helped me write what I couldn't say. I cannot thank you enough.
}

\declaration{
I, Stefan Collier, declare that this dissertation and the work presented in it are my own and has been generated by me as the result of my own original research.\\
I confirm that:\\
1. This work was done wholly or mainly while in candidature for a degree at this University;\\
2. Where any part of this dissertation has previously been submitted for any other qualification at this University or any other institution, this has been clearly stated;\\
3. Where I have consulted the published work of others, this is always clearly attributed;\\
4. Where I have quoted from the work of others, the source is always given. With the exception of such quotations, this dissertation is entirely my own work;\\
5. I have acknowledged all main sources of help;\\
6. Where the thesis is based on work done by myself jointly with others, I have made clear exactly what was done by others and what I have contributed myself;\\
7. Either none of this work has been published before submission, or parts of this work have been published by :\\
\\
Stefan Collier\\
April 2016
}
\tableofcontents
\listoffigures
\listoftables

\mainmatter
%% ----------------------------------------------------------------
%\include{Introduction}
%\include{Conclusions}
\include{chapters/1Project/main}
\include{chapters/2Lit/main}
\include{chapters/3Design/HighLevel}
\include{chapters/3Design/InDepth}
\include{chapters/4Impl/main}

\include{chapters/5Experiments/1/main}
\include{chapters/5Experiments/2/main}
\include{chapters/5Experiments/3/main}
\include{chapters/5Experiments/4/main}

\include{chapters/6Conclusion/main}

\appendix
\include{appendix/AppendixB}
\include{appendix/D/main}
\include{appendix/AppendixC}

\backmatter
\bibliographystyle{ecs}
\bibliography{ECS}
\end{document}
%% ----------------------------------------------------------------


 %% ----------------------------------------------------------------
%% Progress.tex
%% ---------------------------------------------------------------- 
\documentclass{ecsprogress}    % Use the progress Style
\graphicspath{{../figs/}}   % Location of your graphics files
    \usepackage{natbib}            % Use Natbib style for the refs.
\hypersetup{colorlinks=true}   % Set to false for black/white printing
\input{Definitions}            % Include your abbreviations



\usepackage{enumitem}% http://ctan.org/pkg/enumitem
\usepackage{multirow}
\usepackage{float}
\usepackage{amsmath}
\usepackage{multicol}
\usepackage{amssymb}
\usepackage[normalem]{ulem}
\useunder{\uline}{\ul}{}
\usepackage{wrapfig}


\usepackage[table,xcdraw]{xcolor}


%% ----------------------------------------------------------------
\begin{document}
\frontmatter
\title      {Heterogeneous Agent-based Model for Supermarket Competition}
\authors    {\texorpdfstring
             {\href{mailto:sc22g13@ecs.soton.ac.uk}{Stefan J. Collier}}
             {Stefan J. Collier}
            }
\addresses  {\groupname\\\deptname\\\univname}
\date       {\today}
\subject    {}
\keywords   {}
\supervisor {Dr. Maria Polukarov}
\examiner   {Professor Sheng Chen}

\maketitle
\begin{abstract}
This project aim was to model and analyse the effects of competitive pricing behaviors of grocery retailers on the British market. 

This was achieved by creating a multi-agent model, containing retailer and consumer agents. The heterogeneous crowd of retailers employs either a uniform pricing strategy or a ‘local price flexing’ strategy. The actions of these retailers are chosen by predicting the profit of each action, using a perceptron. Following on from the consideration of different economic models, a discrete model was developed so that software agents have a discrete environment to operate within. Within the model, it has been observed how supermarkets with differing behaviors affect a heterogeneous crowd of consumer agents. The model was implemented in Java with Python used to evaluate the results. 

The simulation displays good acceptance with real grocery market behavior, i.e. captures the performance of British retailers thus can be used to determine the impact of changes in their behavior on their competitors and consumers.Furthermore it can be used to provide insight into sustainability of volatile pricing strategies, providing a useful insight in volatility of British supermarket retail industry. 
\end{abstract}
\acknowledgements{
I would like to express my sincere gratitude to Dr Maria Polukarov for her guidance and support which provided me the freedom to take this research in the direction of my interest.\\
\\
I would also like to thank my family and friends for their encouragement and support. To those who quietly listened to my software complaints. To those who worked throughout the nights with me. To those who helped me write what I couldn't say. I cannot thank you enough.
}

\declaration{
I, Stefan Collier, declare that this dissertation and the work presented in it are my own and has been generated by me as the result of my own original research.\\
I confirm that:\\
1. This work was done wholly or mainly while in candidature for a degree at this University;\\
2. Where any part of this dissertation has previously been submitted for any other qualification at this University or any other institution, this has been clearly stated;\\
3. Where I have consulted the published work of others, this is always clearly attributed;\\
4. Where I have quoted from the work of others, the source is always given. With the exception of such quotations, this dissertation is entirely my own work;\\
5. I have acknowledged all main sources of help;\\
6. Where the thesis is based on work done by myself jointly with others, I have made clear exactly what was done by others and what I have contributed myself;\\
7. Either none of this work has been published before submission, or parts of this work have been published by :\\
\\
Stefan Collier\\
April 2016
}
\tableofcontents
\listoffigures
\listoftables

\mainmatter
%% ----------------------------------------------------------------
%\include{Introduction}
%\include{Conclusions}
\include{chapters/1Project/main}
\include{chapters/2Lit/main}
\include{chapters/3Design/HighLevel}
\include{chapters/3Design/InDepth}
\include{chapters/4Impl/main}

\include{chapters/5Experiments/1/main}
\include{chapters/5Experiments/2/main}
\include{chapters/5Experiments/3/main}
\include{chapters/5Experiments/4/main}

\include{chapters/6Conclusion/main}

\appendix
\include{appendix/AppendixB}
\include{appendix/D/main}
\include{appendix/AppendixC}

\backmatter
\bibliographystyle{ecs}
\bibliography{ECS}
\end{document}
%% ----------------------------------------------------------------

 %% ----------------------------------------------------------------
%% Progress.tex
%% ---------------------------------------------------------------- 
\documentclass{ecsprogress}    % Use the progress Style
\graphicspath{{../figs/}}   % Location of your graphics files
    \usepackage{natbib}            % Use Natbib style for the refs.
\hypersetup{colorlinks=true}   % Set to false for black/white printing
\input{Definitions}            % Include your abbreviations



\usepackage{enumitem}% http://ctan.org/pkg/enumitem
\usepackage{multirow}
\usepackage{float}
\usepackage{amsmath}
\usepackage{multicol}
\usepackage{amssymb}
\usepackage[normalem]{ulem}
\useunder{\uline}{\ul}{}
\usepackage{wrapfig}


\usepackage[table,xcdraw]{xcolor}


%% ----------------------------------------------------------------
\begin{document}
\frontmatter
\title      {Heterogeneous Agent-based Model for Supermarket Competition}
\authors    {\texorpdfstring
             {\href{mailto:sc22g13@ecs.soton.ac.uk}{Stefan J. Collier}}
             {Stefan J. Collier}
            }
\addresses  {\groupname\\\deptname\\\univname}
\date       {\today}
\subject    {}
\keywords   {}
\supervisor {Dr. Maria Polukarov}
\examiner   {Professor Sheng Chen}

\maketitle
\begin{abstract}
This project aim was to model and analyse the effects of competitive pricing behaviors of grocery retailers on the British market. 

This was achieved by creating a multi-agent model, containing retailer and consumer agents. The heterogeneous crowd of retailers employs either a uniform pricing strategy or a ‘local price flexing’ strategy. The actions of these retailers are chosen by predicting the profit of each action, using a perceptron. Following on from the consideration of different economic models, a discrete model was developed so that software agents have a discrete environment to operate within. Within the model, it has been observed how supermarkets with differing behaviors affect a heterogeneous crowd of consumer agents. The model was implemented in Java with Python used to evaluate the results. 

The simulation displays good acceptance with real grocery market behavior, i.e. captures the performance of British retailers thus can be used to determine the impact of changes in their behavior on their competitors and consumers.Furthermore it can be used to provide insight into sustainability of volatile pricing strategies, providing a useful insight in volatility of British supermarket retail industry. 
\end{abstract}
\acknowledgements{
I would like to express my sincere gratitude to Dr Maria Polukarov for her guidance and support which provided me the freedom to take this research in the direction of my interest.\\
\\
I would also like to thank my family and friends for their encouragement and support. To those who quietly listened to my software complaints. To those who worked throughout the nights with me. To those who helped me write what I couldn't say. I cannot thank you enough.
}

\declaration{
I, Stefan Collier, declare that this dissertation and the work presented in it are my own and has been generated by me as the result of my own original research.\\
I confirm that:\\
1. This work was done wholly or mainly while in candidature for a degree at this University;\\
2. Where any part of this dissertation has previously been submitted for any other qualification at this University or any other institution, this has been clearly stated;\\
3. Where I have consulted the published work of others, this is always clearly attributed;\\
4. Where I have quoted from the work of others, the source is always given. With the exception of such quotations, this dissertation is entirely my own work;\\
5. I have acknowledged all main sources of help;\\
6. Where the thesis is based on work done by myself jointly with others, I have made clear exactly what was done by others and what I have contributed myself;\\
7. Either none of this work has been published before submission, or parts of this work have been published by :\\
\\
Stefan Collier\\
April 2016
}
\tableofcontents
\listoffigures
\listoftables

\mainmatter
%% ----------------------------------------------------------------
%\include{Introduction}
%\include{Conclusions}
\include{chapters/1Project/main}
\include{chapters/2Lit/main}
\include{chapters/3Design/HighLevel}
\include{chapters/3Design/InDepth}
\include{chapters/4Impl/main}

\include{chapters/5Experiments/1/main}
\include{chapters/5Experiments/2/main}
\include{chapters/5Experiments/3/main}
\include{chapters/5Experiments/4/main}

\include{chapters/6Conclusion/main}

\appendix
\include{appendix/AppendixB}
\include{appendix/D/main}
\include{appendix/AppendixC}

\backmatter
\bibliographystyle{ecs}
\bibliography{ECS}
\end{document}
%% ----------------------------------------------------------------

 %% ----------------------------------------------------------------
%% Progress.tex
%% ---------------------------------------------------------------- 
\documentclass{ecsprogress}    % Use the progress Style
\graphicspath{{../figs/}}   % Location of your graphics files
    \usepackage{natbib}            % Use Natbib style for the refs.
\hypersetup{colorlinks=true}   % Set to false for black/white printing
\input{Definitions}            % Include your abbreviations



\usepackage{enumitem}% http://ctan.org/pkg/enumitem
\usepackage{multirow}
\usepackage{float}
\usepackage{amsmath}
\usepackage{multicol}
\usepackage{amssymb}
\usepackage[normalem]{ulem}
\useunder{\uline}{\ul}{}
\usepackage{wrapfig}


\usepackage[table,xcdraw]{xcolor}


%% ----------------------------------------------------------------
\begin{document}
\frontmatter
\title      {Heterogeneous Agent-based Model for Supermarket Competition}
\authors    {\texorpdfstring
             {\href{mailto:sc22g13@ecs.soton.ac.uk}{Stefan J. Collier}}
             {Stefan J. Collier}
            }
\addresses  {\groupname\\\deptname\\\univname}
\date       {\today}
\subject    {}
\keywords   {}
\supervisor {Dr. Maria Polukarov}
\examiner   {Professor Sheng Chen}

\maketitle
\begin{abstract}
This project aim was to model and analyse the effects of competitive pricing behaviors of grocery retailers on the British market. 

This was achieved by creating a multi-agent model, containing retailer and consumer agents. The heterogeneous crowd of retailers employs either a uniform pricing strategy or a ‘local price flexing’ strategy. The actions of these retailers are chosen by predicting the profit of each action, using a perceptron. Following on from the consideration of different economic models, a discrete model was developed so that software agents have a discrete environment to operate within. Within the model, it has been observed how supermarkets with differing behaviors affect a heterogeneous crowd of consumer agents. The model was implemented in Java with Python used to evaluate the results. 

The simulation displays good acceptance with real grocery market behavior, i.e. captures the performance of British retailers thus can be used to determine the impact of changes in their behavior on their competitors and consumers.Furthermore it can be used to provide insight into sustainability of volatile pricing strategies, providing a useful insight in volatility of British supermarket retail industry. 
\end{abstract}
\acknowledgements{
I would like to express my sincere gratitude to Dr Maria Polukarov for her guidance and support which provided me the freedom to take this research in the direction of my interest.\\
\\
I would also like to thank my family and friends for their encouragement and support. To those who quietly listened to my software complaints. To those who worked throughout the nights with me. To those who helped me write what I couldn't say. I cannot thank you enough.
}

\declaration{
I, Stefan Collier, declare that this dissertation and the work presented in it are my own and has been generated by me as the result of my own original research.\\
I confirm that:\\
1. This work was done wholly or mainly while in candidature for a degree at this University;\\
2. Where any part of this dissertation has previously been submitted for any other qualification at this University or any other institution, this has been clearly stated;\\
3. Where I have consulted the published work of others, this is always clearly attributed;\\
4. Where I have quoted from the work of others, the source is always given. With the exception of such quotations, this dissertation is entirely my own work;\\
5. I have acknowledged all main sources of help;\\
6. Where the thesis is based on work done by myself jointly with others, I have made clear exactly what was done by others and what I have contributed myself;\\
7. Either none of this work has been published before submission, or parts of this work have been published by :\\
\\
Stefan Collier\\
April 2016
}
\tableofcontents
\listoffigures
\listoftables

\mainmatter
%% ----------------------------------------------------------------
%\include{Introduction}
%\include{Conclusions}
\include{chapters/1Project/main}
\include{chapters/2Lit/main}
\include{chapters/3Design/HighLevel}
\include{chapters/3Design/InDepth}
\include{chapters/4Impl/main}

\include{chapters/5Experiments/1/main}
\include{chapters/5Experiments/2/main}
\include{chapters/5Experiments/3/main}
\include{chapters/5Experiments/4/main}

\include{chapters/6Conclusion/main}

\appendix
\include{appendix/AppendixB}
\include{appendix/D/main}
\include{appendix/AppendixC}

\backmatter
\bibliographystyle{ecs}
\bibliography{ECS}
\end{document}
%% ----------------------------------------------------------------

 %% ----------------------------------------------------------------
%% Progress.tex
%% ---------------------------------------------------------------- 
\documentclass{ecsprogress}    % Use the progress Style
\graphicspath{{../figs/}}   % Location of your graphics files
    \usepackage{natbib}            % Use Natbib style for the refs.
\hypersetup{colorlinks=true}   % Set to false for black/white printing
\input{Definitions}            % Include your abbreviations



\usepackage{enumitem}% http://ctan.org/pkg/enumitem
\usepackage{multirow}
\usepackage{float}
\usepackage{amsmath}
\usepackage{multicol}
\usepackage{amssymb}
\usepackage[normalem]{ulem}
\useunder{\uline}{\ul}{}
\usepackage{wrapfig}


\usepackage[table,xcdraw]{xcolor}


%% ----------------------------------------------------------------
\begin{document}
\frontmatter
\title      {Heterogeneous Agent-based Model for Supermarket Competition}
\authors    {\texorpdfstring
             {\href{mailto:sc22g13@ecs.soton.ac.uk}{Stefan J. Collier}}
             {Stefan J. Collier}
            }
\addresses  {\groupname\\\deptname\\\univname}
\date       {\today}
\subject    {}
\keywords   {}
\supervisor {Dr. Maria Polukarov}
\examiner   {Professor Sheng Chen}

\maketitle
\begin{abstract}
This project aim was to model and analyse the effects of competitive pricing behaviors of grocery retailers on the British market. 

This was achieved by creating a multi-agent model, containing retailer and consumer agents. The heterogeneous crowd of retailers employs either a uniform pricing strategy or a ‘local price flexing’ strategy. The actions of these retailers are chosen by predicting the profit of each action, using a perceptron. Following on from the consideration of different economic models, a discrete model was developed so that software agents have a discrete environment to operate within. Within the model, it has been observed how supermarkets with differing behaviors affect a heterogeneous crowd of consumer agents. The model was implemented in Java with Python used to evaluate the results. 

The simulation displays good acceptance with real grocery market behavior, i.e. captures the performance of British retailers thus can be used to determine the impact of changes in their behavior on their competitors and consumers.Furthermore it can be used to provide insight into sustainability of volatile pricing strategies, providing a useful insight in volatility of British supermarket retail industry. 
\end{abstract}
\acknowledgements{
I would like to express my sincere gratitude to Dr Maria Polukarov for her guidance and support which provided me the freedom to take this research in the direction of my interest.\\
\\
I would also like to thank my family and friends for their encouragement and support. To those who quietly listened to my software complaints. To those who worked throughout the nights with me. To those who helped me write what I couldn't say. I cannot thank you enough.
}

\declaration{
I, Stefan Collier, declare that this dissertation and the work presented in it are my own and has been generated by me as the result of my own original research.\\
I confirm that:\\
1. This work was done wholly or mainly while in candidature for a degree at this University;\\
2. Where any part of this dissertation has previously been submitted for any other qualification at this University or any other institution, this has been clearly stated;\\
3. Where I have consulted the published work of others, this is always clearly attributed;\\
4. Where I have quoted from the work of others, the source is always given. With the exception of such quotations, this dissertation is entirely my own work;\\
5. I have acknowledged all main sources of help;\\
6. Where the thesis is based on work done by myself jointly with others, I have made clear exactly what was done by others and what I have contributed myself;\\
7. Either none of this work has been published before submission, or parts of this work have been published by :\\
\\
Stefan Collier\\
April 2016
}
\tableofcontents
\listoffigures
\listoftables

\mainmatter
%% ----------------------------------------------------------------
%\include{Introduction}
%\include{Conclusions}
\include{chapters/1Project/main}
\include{chapters/2Lit/main}
\include{chapters/3Design/HighLevel}
\include{chapters/3Design/InDepth}
\include{chapters/4Impl/main}

\include{chapters/5Experiments/1/main}
\include{chapters/5Experiments/2/main}
\include{chapters/5Experiments/3/main}
\include{chapters/5Experiments/4/main}

\include{chapters/6Conclusion/main}

\appendix
\include{appendix/AppendixB}
\include{appendix/D/main}
\include{appendix/AppendixC}

\backmatter
\bibliographystyle{ecs}
\bibliography{ECS}
\end{document}
%% ----------------------------------------------------------------


 %% ----------------------------------------------------------------
%% Progress.tex
%% ---------------------------------------------------------------- 
\documentclass{ecsprogress}    % Use the progress Style
\graphicspath{{../figs/}}   % Location of your graphics files
    \usepackage{natbib}            % Use Natbib style for the refs.
\hypersetup{colorlinks=true}   % Set to false for black/white printing
\input{Definitions}            % Include your abbreviations



\usepackage{enumitem}% http://ctan.org/pkg/enumitem
\usepackage{multirow}
\usepackage{float}
\usepackage{amsmath}
\usepackage{multicol}
\usepackage{amssymb}
\usepackage[normalem]{ulem}
\useunder{\uline}{\ul}{}
\usepackage{wrapfig}


\usepackage[table,xcdraw]{xcolor}


%% ----------------------------------------------------------------
\begin{document}
\frontmatter
\title      {Heterogeneous Agent-based Model for Supermarket Competition}
\authors    {\texorpdfstring
             {\href{mailto:sc22g13@ecs.soton.ac.uk}{Stefan J. Collier}}
             {Stefan J. Collier}
            }
\addresses  {\groupname\\\deptname\\\univname}
\date       {\today}
\subject    {}
\keywords   {}
\supervisor {Dr. Maria Polukarov}
\examiner   {Professor Sheng Chen}

\maketitle
\begin{abstract}
This project aim was to model and analyse the effects of competitive pricing behaviors of grocery retailers on the British market. 

This was achieved by creating a multi-agent model, containing retailer and consumer agents. The heterogeneous crowd of retailers employs either a uniform pricing strategy or a ‘local price flexing’ strategy. The actions of these retailers are chosen by predicting the profit of each action, using a perceptron. Following on from the consideration of different economic models, a discrete model was developed so that software agents have a discrete environment to operate within. Within the model, it has been observed how supermarkets with differing behaviors affect a heterogeneous crowd of consumer agents. The model was implemented in Java with Python used to evaluate the results. 

The simulation displays good acceptance with real grocery market behavior, i.e. captures the performance of British retailers thus can be used to determine the impact of changes in their behavior on their competitors and consumers.Furthermore it can be used to provide insight into sustainability of volatile pricing strategies, providing a useful insight in volatility of British supermarket retail industry. 
\end{abstract}
\acknowledgements{
I would like to express my sincere gratitude to Dr Maria Polukarov for her guidance and support which provided me the freedom to take this research in the direction of my interest.\\
\\
I would also like to thank my family and friends for their encouragement and support. To those who quietly listened to my software complaints. To those who worked throughout the nights with me. To those who helped me write what I couldn't say. I cannot thank you enough.
}

\declaration{
I, Stefan Collier, declare that this dissertation and the work presented in it are my own and has been generated by me as the result of my own original research.\\
I confirm that:\\
1. This work was done wholly or mainly while in candidature for a degree at this University;\\
2. Where any part of this dissertation has previously been submitted for any other qualification at this University or any other institution, this has been clearly stated;\\
3. Where I have consulted the published work of others, this is always clearly attributed;\\
4. Where I have quoted from the work of others, the source is always given. With the exception of such quotations, this dissertation is entirely my own work;\\
5. I have acknowledged all main sources of help;\\
6. Where the thesis is based on work done by myself jointly with others, I have made clear exactly what was done by others and what I have contributed myself;\\
7. Either none of this work has been published before submission, or parts of this work have been published by :\\
\\
Stefan Collier\\
April 2016
}
\tableofcontents
\listoffigures
\listoftables

\mainmatter
%% ----------------------------------------------------------------
%\include{Introduction}
%\include{Conclusions}
\include{chapters/1Project/main}
\include{chapters/2Lit/main}
\include{chapters/3Design/HighLevel}
\include{chapters/3Design/InDepth}
\include{chapters/4Impl/main}

\include{chapters/5Experiments/1/main}
\include{chapters/5Experiments/2/main}
\include{chapters/5Experiments/3/main}
\include{chapters/5Experiments/4/main}

\include{chapters/6Conclusion/main}

\appendix
\include{appendix/AppendixB}
\include{appendix/D/main}
\include{appendix/AppendixC}

\backmatter
\bibliographystyle{ecs}
\bibliography{ECS}
\end{document}
%% ----------------------------------------------------------------


\appendix
\include{appendix/AppendixB}
 %% ----------------------------------------------------------------
%% Progress.tex
%% ---------------------------------------------------------------- 
\documentclass{ecsprogress}    % Use the progress Style
\graphicspath{{../figs/}}   % Location of your graphics files
    \usepackage{natbib}            % Use Natbib style for the refs.
\hypersetup{colorlinks=true}   % Set to false for black/white printing
\input{Definitions}            % Include your abbreviations



\usepackage{enumitem}% http://ctan.org/pkg/enumitem
\usepackage{multirow}
\usepackage{float}
\usepackage{amsmath}
\usepackage{multicol}
\usepackage{amssymb}
\usepackage[normalem]{ulem}
\useunder{\uline}{\ul}{}
\usepackage{wrapfig}


\usepackage[table,xcdraw]{xcolor}


%% ----------------------------------------------------------------
\begin{document}
\frontmatter
\title      {Heterogeneous Agent-based Model for Supermarket Competition}
\authors    {\texorpdfstring
             {\href{mailto:sc22g13@ecs.soton.ac.uk}{Stefan J. Collier}}
             {Stefan J. Collier}
            }
\addresses  {\groupname\\\deptname\\\univname}
\date       {\today}
\subject    {}
\keywords   {}
\supervisor {Dr. Maria Polukarov}
\examiner   {Professor Sheng Chen}

\maketitle
\begin{abstract}
This project aim was to model and analyse the effects of competitive pricing behaviors of grocery retailers on the British market. 

This was achieved by creating a multi-agent model, containing retailer and consumer agents. The heterogeneous crowd of retailers employs either a uniform pricing strategy or a ‘local price flexing’ strategy. The actions of these retailers are chosen by predicting the profit of each action, using a perceptron. Following on from the consideration of different economic models, a discrete model was developed so that software agents have a discrete environment to operate within. Within the model, it has been observed how supermarkets with differing behaviors affect a heterogeneous crowd of consumer agents. The model was implemented in Java with Python used to evaluate the results. 

The simulation displays good acceptance with real grocery market behavior, i.e. captures the performance of British retailers thus can be used to determine the impact of changes in their behavior on their competitors and consumers.Furthermore it can be used to provide insight into sustainability of volatile pricing strategies, providing a useful insight in volatility of British supermarket retail industry. 
\end{abstract}
\acknowledgements{
I would like to express my sincere gratitude to Dr Maria Polukarov for her guidance and support which provided me the freedom to take this research in the direction of my interest.\\
\\
I would also like to thank my family and friends for their encouragement and support. To those who quietly listened to my software complaints. To those who worked throughout the nights with me. To those who helped me write what I couldn't say. I cannot thank you enough.
}

\declaration{
I, Stefan Collier, declare that this dissertation and the work presented in it are my own and has been generated by me as the result of my own original research.\\
I confirm that:\\
1. This work was done wholly or mainly while in candidature for a degree at this University;\\
2. Where any part of this dissertation has previously been submitted for any other qualification at this University or any other institution, this has been clearly stated;\\
3. Where I have consulted the published work of others, this is always clearly attributed;\\
4. Where I have quoted from the work of others, the source is always given. With the exception of such quotations, this dissertation is entirely my own work;\\
5. I have acknowledged all main sources of help;\\
6. Where the thesis is based on work done by myself jointly with others, I have made clear exactly what was done by others and what I have contributed myself;\\
7. Either none of this work has been published before submission, or parts of this work have been published by :\\
\\
Stefan Collier\\
April 2016
}
\tableofcontents
\listoffigures
\listoftables

\mainmatter
%% ----------------------------------------------------------------
%\include{Introduction}
%\include{Conclusions}
\include{chapters/1Project/main}
\include{chapters/2Lit/main}
\include{chapters/3Design/HighLevel}
\include{chapters/3Design/InDepth}
\include{chapters/4Impl/main}

\include{chapters/5Experiments/1/main}
\include{chapters/5Experiments/2/main}
\include{chapters/5Experiments/3/main}
\include{chapters/5Experiments/4/main}

\include{chapters/6Conclusion/main}

\appendix
\include{appendix/AppendixB}
\include{appendix/D/main}
\include{appendix/AppendixC}

\backmatter
\bibliographystyle{ecs}
\bibliography{ECS}
\end{document}
%% ----------------------------------------------------------------

\include{appendix/AppendixC}

\backmatter
\bibliographystyle{ecs}
\bibliography{ECS}
\end{document}
%% ----------------------------------------------------------------

 %% ----------------------------------------------------------------
%% Progress.tex
%% ---------------------------------------------------------------- 
\documentclass{ecsprogress}    % Use the progress Style
\graphicspath{{../figs/}}   % Location of your graphics files
    \usepackage{natbib}            % Use Natbib style for the refs.
\hypersetup{colorlinks=true}   % Set to false for black/white printing
\input{Definitions}            % Include your abbreviations



\usepackage{enumitem}% http://ctan.org/pkg/enumitem
\usepackage{multirow}
\usepackage{float}
\usepackage{amsmath}
\usepackage{multicol}
\usepackage{amssymb}
\usepackage[normalem]{ulem}
\useunder{\uline}{\ul}{}
\usepackage{wrapfig}


\usepackage[table,xcdraw]{xcolor}


%% ----------------------------------------------------------------
\begin{document}
\frontmatter
\title      {Heterogeneous Agent-based Model for Supermarket Competition}
\authors    {\texorpdfstring
             {\href{mailto:sc22g13@ecs.soton.ac.uk}{Stefan J. Collier}}
             {Stefan J. Collier}
            }
\addresses  {\groupname\\\deptname\\\univname}
\date       {\today}
\subject    {}
\keywords   {}
\supervisor {Dr. Maria Polukarov}
\examiner   {Professor Sheng Chen}

\maketitle
\begin{abstract}
This project aim was to model and analyse the effects of competitive pricing behaviors of grocery retailers on the British market. 

This was achieved by creating a multi-agent model, containing retailer and consumer agents. The heterogeneous crowd of retailers employs either a uniform pricing strategy or a ‘local price flexing’ strategy. The actions of these retailers are chosen by predicting the profit of each action, using a perceptron. Following on from the consideration of different economic models, a discrete model was developed so that software agents have a discrete environment to operate within. Within the model, it has been observed how supermarkets with differing behaviors affect a heterogeneous crowd of consumer agents. The model was implemented in Java with Python used to evaluate the results. 

The simulation displays good acceptance with real grocery market behavior, i.e. captures the performance of British retailers thus can be used to determine the impact of changes in their behavior on their competitors and consumers.Furthermore it can be used to provide insight into sustainability of volatile pricing strategies, providing a useful insight in volatility of British supermarket retail industry. 
\end{abstract}
\acknowledgements{
I would like to express my sincere gratitude to Dr Maria Polukarov for her guidance and support which provided me the freedom to take this research in the direction of my interest.\\
\\
I would also like to thank my family and friends for their encouragement and support. To those who quietly listened to my software complaints. To those who worked throughout the nights with me. To those who helped me write what I couldn't say. I cannot thank you enough.
}

\declaration{
I, Stefan Collier, declare that this dissertation and the work presented in it are my own and has been generated by me as the result of my own original research.\\
I confirm that:\\
1. This work was done wholly or mainly while in candidature for a degree at this University;\\
2. Where any part of this dissertation has previously been submitted for any other qualification at this University or any other institution, this has been clearly stated;\\
3. Where I have consulted the published work of others, this is always clearly attributed;\\
4. Where I have quoted from the work of others, the source is always given. With the exception of such quotations, this dissertation is entirely my own work;\\
5. I have acknowledged all main sources of help;\\
6. Where the thesis is based on work done by myself jointly with others, I have made clear exactly what was done by others and what I have contributed myself;\\
7. Either none of this work has been published before submission, or parts of this work have been published by :\\
\\
Stefan Collier\\
April 2016
}
\tableofcontents
\listoffigures
\listoftables

\mainmatter
%% ----------------------------------------------------------------
%\include{Introduction}
%\include{Conclusions}
 %% ----------------------------------------------------------------
%% Progress.tex
%% ---------------------------------------------------------------- 
\documentclass{ecsprogress}    % Use the progress Style
\graphicspath{{../figs/}}   % Location of your graphics files
    \usepackage{natbib}            % Use Natbib style for the refs.
\hypersetup{colorlinks=true}   % Set to false for black/white printing
\input{Definitions}            % Include your abbreviations



\usepackage{enumitem}% http://ctan.org/pkg/enumitem
\usepackage{multirow}
\usepackage{float}
\usepackage{amsmath}
\usepackage{multicol}
\usepackage{amssymb}
\usepackage[normalem]{ulem}
\useunder{\uline}{\ul}{}
\usepackage{wrapfig}


\usepackage[table,xcdraw]{xcolor}


%% ----------------------------------------------------------------
\begin{document}
\frontmatter
\title      {Heterogeneous Agent-based Model for Supermarket Competition}
\authors    {\texorpdfstring
             {\href{mailto:sc22g13@ecs.soton.ac.uk}{Stefan J. Collier}}
             {Stefan J. Collier}
            }
\addresses  {\groupname\\\deptname\\\univname}
\date       {\today}
\subject    {}
\keywords   {}
\supervisor {Dr. Maria Polukarov}
\examiner   {Professor Sheng Chen}

\maketitle
\begin{abstract}
This project aim was to model and analyse the effects of competitive pricing behaviors of grocery retailers on the British market. 

This was achieved by creating a multi-agent model, containing retailer and consumer agents. The heterogeneous crowd of retailers employs either a uniform pricing strategy or a ‘local price flexing’ strategy. The actions of these retailers are chosen by predicting the profit of each action, using a perceptron. Following on from the consideration of different economic models, a discrete model was developed so that software agents have a discrete environment to operate within. Within the model, it has been observed how supermarkets with differing behaviors affect a heterogeneous crowd of consumer agents. The model was implemented in Java with Python used to evaluate the results. 

The simulation displays good acceptance with real grocery market behavior, i.e. captures the performance of British retailers thus can be used to determine the impact of changes in their behavior on their competitors and consumers.Furthermore it can be used to provide insight into sustainability of volatile pricing strategies, providing a useful insight in volatility of British supermarket retail industry. 
\end{abstract}
\acknowledgements{
I would like to express my sincere gratitude to Dr Maria Polukarov for her guidance and support which provided me the freedom to take this research in the direction of my interest.\\
\\
I would also like to thank my family and friends for their encouragement and support. To those who quietly listened to my software complaints. To those who worked throughout the nights with me. To those who helped me write what I couldn't say. I cannot thank you enough.
}

\declaration{
I, Stefan Collier, declare that this dissertation and the work presented in it are my own and has been generated by me as the result of my own original research.\\
I confirm that:\\
1. This work was done wholly or mainly while in candidature for a degree at this University;\\
2. Where any part of this dissertation has previously been submitted for any other qualification at this University or any other institution, this has been clearly stated;\\
3. Where I have consulted the published work of others, this is always clearly attributed;\\
4. Where I have quoted from the work of others, the source is always given. With the exception of such quotations, this dissertation is entirely my own work;\\
5. I have acknowledged all main sources of help;\\
6. Where the thesis is based on work done by myself jointly with others, I have made clear exactly what was done by others and what I have contributed myself;\\
7. Either none of this work has been published before submission, or parts of this work have been published by :\\
\\
Stefan Collier\\
April 2016
}
\tableofcontents
\listoffigures
\listoftables

\mainmatter
%% ----------------------------------------------------------------
%\include{Introduction}
%\include{Conclusions}
\include{chapters/1Project/main}
\include{chapters/2Lit/main}
\include{chapters/3Design/HighLevel}
\include{chapters/3Design/InDepth}
\include{chapters/4Impl/main}

\include{chapters/5Experiments/1/main}
\include{chapters/5Experiments/2/main}
\include{chapters/5Experiments/3/main}
\include{chapters/5Experiments/4/main}

\include{chapters/6Conclusion/main}

\appendix
\include{appendix/AppendixB}
\include{appendix/D/main}
\include{appendix/AppendixC}

\backmatter
\bibliographystyle{ecs}
\bibliography{ECS}
\end{document}
%% ----------------------------------------------------------------

 %% ----------------------------------------------------------------
%% Progress.tex
%% ---------------------------------------------------------------- 
\documentclass{ecsprogress}    % Use the progress Style
\graphicspath{{../figs/}}   % Location of your graphics files
    \usepackage{natbib}            % Use Natbib style for the refs.
\hypersetup{colorlinks=true}   % Set to false for black/white printing
\input{Definitions}            % Include your abbreviations



\usepackage{enumitem}% http://ctan.org/pkg/enumitem
\usepackage{multirow}
\usepackage{float}
\usepackage{amsmath}
\usepackage{multicol}
\usepackage{amssymb}
\usepackage[normalem]{ulem}
\useunder{\uline}{\ul}{}
\usepackage{wrapfig}


\usepackage[table,xcdraw]{xcolor}


%% ----------------------------------------------------------------
\begin{document}
\frontmatter
\title      {Heterogeneous Agent-based Model for Supermarket Competition}
\authors    {\texorpdfstring
             {\href{mailto:sc22g13@ecs.soton.ac.uk}{Stefan J. Collier}}
             {Stefan J. Collier}
            }
\addresses  {\groupname\\\deptname\\\univname}
\date       {\today}
\subject    {}
\keywords   {}
\supervisor {Dr. Maria Polukarov}
\examiner   {Professor Sheng Chen}

\maketitle
\begin{abstract}
This project aim was to model and analyse the effects of competitive pricing behaviors of grocery retailers on the British market. 

This was achieved by creating a multi-agent model, containing retailer and consumer agents. The heterogeneous crowd of retailers employs either a uniform pricing strategy or a ‘local price flexing’ strategy. The actions of these retailers are chosen by predicting the profit of each action, using a perceptron. Following on from the consideration of different economic models, a discrete model was developed so that software agents have a discrete environment to operate within. Within the model, it has been observed how supermarkets with differing behaviors affect a heterogeneous crowd of consumer agents. The model was implemented in Java with Python used to evaluate the results. 

The simulation displays good acceptance with real grocery market behavior, i.e. captures the performance of British retailers thus can be used to determine the impact of changes in their behavior on their competitors and consumers.Furthermore it can be used to provide insight into sustainability of volatile pricing strategies, providing a useful insight in volatility of British supermarket retail industry. 
\end{abstract}
\acknowledgements{
I would like to express my sincere gratitude to Dr Maria Polukarov for her guidance and support which provided me the freedom to take this research in the direction of my interest.\\
\\
I would also like to thank my family and friends for their encouragement and support. To those who quietly listened to my software complaints. To those who worked throughout the nights with me. To those who helped me write what I couldn't say. I cannot thank you enough.
}

\declaration{
I, Stefan Collier, declare that this dissertation and the work presented in it are my own and has been generated by me as the result of my own original research.\\
I confirm that:\\
1. This work was done wholly or mainly while in candidature for a degree at this University;\\
2. Where any part of this dissertation has previously been submitted for any other qualification at this University or any other institution, this has been clearly stated;\\
3. Where I have consulted the published work of others, this is always clearly attributed;\\
4. Where I have quoted from the work of others, the source is always given. With the exception of such quotations, this dissertation is entirely my own work;\\
5. I have acknowledged all main sources of help;\\
6. Where the thesis is based on work done by myself jointly with others, I have made clear exactly what was done by others and what I have contributed myself;\\
7. Either none of this work has been published before submission, or parts of this work have been published by :\\
\\
Stefan Collier\\
April 2016
}
\tableofcontents
\listoffigures
\listoftables

\mainmatter
%% ----------------------------------------------------------------
%\include{Introduction}
%\include{Conclusions}
\include{chapters/1Project/main}
\include{chapters/2Lit/main}
\include{chapters/3Design/HighLevel}
\include{chapters/3Design/InDepth}
\include{chapters/4Impl/main}

\include{chapters/5Experiments/1/main}
\include{chapters/5Experiments/2/main}
\include{chapters/5Experiments/3/main}
\include{chapters/5Experiments/4/main}

\include{chapters/6Conclusion/main}

\appendix
\include{appendix/AppendixB}
\include{appendix/D/main}
\include{appendix/AppendixC}

\backmatter
\bibliographystyle{ecs}
\bibliography{ECS}
\end{document}
%% ----------------------------------------------------------------

\include{chapters/3Design/HighLevel}
\include{chapters/3Design/InDepth}
 %% ----------------------------------------------------------------
%% Progress.tex
%% ---------------------------------------------------------------- 
\documentclass{ecsprogress}    % Use the progress Style
\graphicspath{{../figs/}}   % Location of your graphics files
    \usepackage{natbib}            % Use Natbib style for the refs.
\hypersetup{colorlinks=true}   % Set to false for black/white printing
\input{Definitions}            % Include your abbreviations



\usepackage{enumitem}% http://ctan.org/pkg/enumitem
\usepackage{multirow}
\usepackage{float}
\usepackage{amsmath}
\usepackage{multicol}
\usepackage{amssymb}
\usepackage[normalem]{ulem}
\useunder{\uline}{\ul}{}
\usepackage{wrapfig}


\usepackage[table,xcdraw]{xcolor}


%% ----------------------------------------------------------------
\begin{document}
\frontmatter
\title      {Heterogeneous Agent-based Model for Supermarket Competition}
\authors    {\texorpdfstring
             {\href{mailto:sc22g13@ecs.soton.ac.uk}{Stefan J. Collier}}
             {Stefan J. Collier}
            }
\addresses  {\groupname\\\deptname\\\univname}
\date       {\today}
\subject    {}
\keywords   {}
\supervisor {Dr. Maria Polukarov}
\examiner   {Professor Sheng Chen}

\maketitle
\begin{abstract}
This project aim was to model and analyse the effects of competitive pricing behaviors of grocery retailers on the British market. 

This was achieved by creating a multi-agent model, containing retailer and consumer agents. The heterogeneous crowd of retailers employs either a uniform pricing strategy or a ‘local price flexing’ strategy. The actions of these retailers are chosen by predicting the profit of each action, using a perceptron. Following on from the consideration of different economic models, a discrete model was developed so that software agents have a discrete environment to operate within. Within the model, it has been observed how supermarkets with differing behaviors affect a heterogeneous crowd of consumer agents. The model was implemented in Java with Python used to evaluate the results. 

The simulation displays good acceptance with real grocery market behavior, i.e. captures the performance of British retailers thus can be used to determine the impact of changes in their behavior on their competitors and consumers.Furthermore it can be used to provide insight into sustainability of volatile pricing strategies, providing a useful insight in volatility of British supermarket retail industry. 
\end{abstract}
\acknowledgements{
I would like to express my sincere gratitude to Dr Maria Polukarov for her guidance and support which provided me the freedom to take this research in the direction of my interest.\\
\\
I would also like to thank my family and friends for their encouragement and support. To those who quietly listened to my software complaints. To those who worked throughout the nights with me. To those who helped me write what I couldn't say. I cannot thank you enough.
}

\declaration{
I, Stefan Collier, declare that this dissertation and the work presented in it are my own and has been generated by me as the result of my own original research.\\
I confirm that:\\
1. This work was done wholly or mainly while in candidature for a degree at this University;\\
2. Where any part of this dissertation has previously been submitted for any other qualification at this University or any other institution, this has been clearly stated;\\
3. Where I have consulted the published work of others, this is always clearly attributed;\\
4. Where I have quoted from the work of others, the source is always given. With the exception of such quotations, this dissertation is entirely my own work;\\
5. I have acknowledged all main sources of help;\\
6. Where the thesis is based on work done by myself jointly with others, I have made clear exactly what was done by others and what I have contributed myself;\\
7. Either none of this work has been published before submission, or parts of this work have been published by :\\
\\
Stefan Collier\\
April 2016
}
\tableofcontents
\listoffigures
\listoftables

\mainmatter
%% ----------------------------------------------------------------
%\include{Introduction}
%\include{Conclusions}
\include{chapters/1Project/main}
\include{chapters/2Lit/main}
\include{chapters/3Design/HighLevel}
\include{chapters/3Design/InDepth}
\include{chapters/4Impl/main}

\include{chapters/5Experiments/1/main}
\include{chapters/5Experiments/2/main}
\include{chapters/5Experiments/3/main}
\include{chapters/5Experiments/4/main}

\include{chapters/6Conclusion/main}

\appendix
\include{appendix/AppendixB}
\include{appendix/D/main}
\include{appendix/AppendixC}

\backmatter
\bibliographystyle{ecs}
\bibliography{ECS}
\end{document}
%% ----------------------------------------------------------------


 %% ----------------------------------------------------------------
%% Progress.tex
%% ---------------------------------------------------------------- 
\documentclass{ecsprogress}    % Use the progress Style
\graphicspath{{../figs/}}   % Location of your graphics files
    \usepackage{natbib}            % Use Natbib style for the refs.
\hypersetup{colorlinks=true}   % Set to false for black/white printing
\input{Definitions}            % Include your abbreviations



\usepackage{enumitem}% http://ctan.org/pkg/enumitem
\usepackage{multirow}
\usepackage{float}
\usepackage{amsmath}
\usepackage{multicol}
\usepackage{amssymb}
\usepackage[normalem]{ulem}
\useunder{\uline}{\ul}{}
\usepackage{wrapfig}


\usepackage[table,xcdraw]{xcolor}


%% ----------------------------------------------------------------
\begin{document}
\frontmatter
\title      {Heterogeneous Agent-based Model for Supermarket Competition}
\authors    {\texorpdfstring
             {\href{mailto:sc22g13@ecs.soton.ac.uk}{Stefan J. Collier}}
             {Stefan J. Collier}
            }
\addresses  {\groupname\\\deptname\\\univname}
\date       {\today}
\subject    {}
\keywords   {}
\supervisor {Dr. Maria Polukarov}
\examiner   {Professor Sheng Chen}

\maketitle
\begin{abstract}
This project aim was to model and analyse the effects of competitive pricing behaviors of grocery retailers on the British market. 

This was achieved by creating a multi-agent model, containing retailer and consumer agents. The heterogeneous crowd of retailers employs either a uniform pricing strategy or a ‘local price flexing’ strategy. The actions of these retailers are chosen by predicting the profit of each action, using a perceptron. Following on from the consideration of different economic models, a discrete model was developed so that software agents have a discrete environment to operate within. Within the model, it has been observed how supermarkets with differing behaviors affect a heterogeneous crowd of consumer agents. The model was implemented in Java with Python used to evaluate the results. 

The simulation displays good acceptance with real grocery market behavior, i.e. captures the performance of British retailers thus can be used to determine the impact of changes in their behavior on their competitors and consumers.Furthermore it can be used to provide insight into sustainability of volatile pricing strategies, providing a useful insight in volatility of British supermarket retail industry. 
\end{abstract}
\acknowledgements{
I would like to express my sincere gratitude to Dr Maria Polukarov for her guidance and support which provided me the freedom to take this research in the direction of my interest.\\
\\
I would also like to thank my family and friends for their encouragement and support. To those who quietly listened to my software complaints. To those who worked throughout the nights with me. To those who helped me write what I couldn't say. I cannot thank you enough.
}

\declaration{
I, Stefan Collier, declare that this dissertation and the work presented in it are my own and has been generated by me as the result of my own original research.\\
I confirm that:\\
1. This work was done wholly or mainly while in candidature for a degree at this University;\\
2. Where any part of this dissertation has previously been submitted for any other qualification at this University or any other institution, this has been clearly stated;\\
3. Where I have consulted the published work of others, this is always clearly attributed;\\
4. Where I have quoted from the work of others, the source is always given. With the exception of such quotations, this dissertation is entirely my own work;\\
5. I have acknowledged all main sources of help;\\
6. Where the thesis is based on work done by myself jointly with others, I have made clear exactly what was done by others and what I have contributed myself;\\
7. Either none of this work has been published before submission, or parts of this work have been published by :\\
\\
Stefan Collier\\
April 2016
}
\tableofcontents
\listoffigures
\listoftables

\mainmatter
%% ----------------------------------------------------------------
%\include{Introduction}
%\include{Conclusions}
\include{chapters/1Project/main}
\include{chapters/2Lit/main}
\include{chapters/3Design/HighLevel}
\include{chapters/3Design/InDepth}
\include{chapters/4Impl/main}

\include{chapters/5Experiments/1/main}
\include{chapters/5Experiments/2/main}
\include{chapters/5Experiments/3/main}
\include{chapters/5Experiments/4/main}

\include{chapters/6Conclusion/main}

\appendix
\include{appendix/AppendixB}
\include{appendix/D/main}
\include{appendix/AppendixC}

\backmatter
\bibliographystyle{ecs}
\bibliography{ECS}
\end{document}
%% ----------------------------------------------------------------

 %% ----------------------------------------------------------------
%% Progress.tex
%% ---------------------------------------------------------------- 
\documentclass{ecsprogress}    % Use the progress Style
\graphicspath{{../figs/}}   % Location of your graphics files
    \usepackage{natbib}            % Use Natbib style for the refs.
\hypersetup{colorlinks=true}   % Set to false for black/white printing
\input{Definitions}            % Include your abbreviations



\usepackage{enumitem}% http://ctan.org/pkg/enumitem
\usepackage{multirow}
\usepackage{float}
\usepackage{amsmath}
\usepackage{multicol}
\usepackage{amssymb}
\usepackage[normalem]{ulem}
\useunder{\uline}{\ul}{}
\usepackage{wrapfig}


\usepackage[table,xcdraw]{xcolor}


%% ----------------------------------------------------------------
\begin{document}
\frontmatter
\title      {Heterogeneous Agent-based Model for Supermarket Competition}
\authors    {\texorpdfstring
             {\href{mailto:sc22g13@ecs.soton.ac.uk}{Stefan J. Collier}}
             {Stefan J. Collier}
            }
\addresses  {\groupname\\\deptname\\\univname}
\date       {\today}
\subject    {}
\keywords   {}
\supervisor {Dr. Maria Polukarov}
\examiner   {Professor Sheng Chen}

\maketitle
\begin{abstract}
This project aim was to model and analyse the effects of competitive pricing behaviors of grocery retailers on the British market. 

This was achieved by creating a multi-agent model, containing retailer and consumer agents. The heterogeneous crowd of retailers employs either a uniform pricing strategy or a ‘local price flexing’ strategy. The actions of these retailers are chosen by predicting the profit of each action, using a perceptron. Following on from the consideration of different economic models, a discrete model was developed so that software agents have a discrete environment to operate within. Within the model, it has been observed how supermarkets with differing behaviors affect a heterogeneous crowd of consumer agents. The model was implemented in Java with Python used to evaluate the results. 

The simulation displays good acceptance with real grocery market behavior, i.e. captures the performance of British retailers thus can be used to determine the impact of changes in their behavior on their competitors and consumers.Furthermore it can be used to provide insight into sustainability of volatile pricing strategies, providing a useful insight in volatility of British supermarket retail industry. 
\end{abstract}
\acknowledgements{
I would like to express my sincere gratitude to Dr Maria Polukarov for her guidance and support which provided me the freedom to take this research in the direction of my interest.\\
\\
I would also like to thank my family and friends for their encouragement and support. To those who quietly listened to my software complaints. To those who worked throughout the nights with me. To those who helped me write what I couldn't say. I cannot thank you enough.
}

\declaration{
I, Stefan Collier, declare that this dissertation and the work presented in it are my own and has been generated by me as the result of my own original research.\\
I confirm that:\\
1. This work was done wholly or mainly while in candidature for a degree at this University;\\
2. Where any part of this dissertation has previously been submitted for any other qualification at this University or any other institution, this has been clearly stated;\\
3. Where I have consulted the published work of others, this is always clearly attributed;\\
4. Where I have quoted from the work of others, the source is always given. With the exception of such quotations, this dissertation is entirely my own work;\\
5. I have acknowledged all main sources of help;\\
6. Where the thesis is based on work done by myself jointly with others, I have made clear exactly what was done by others and what I have contributed myself;\\
7. Either none of this work has been published before submission, or parts of this work have been published by :\\
\\
Stefan Collier\\
April 2016
}
\tableofcontents
\listoffigures
\listoftables

\mainmatter
%% ----------------------------------------------------------------
%\include{Introduction}
%\include{Conclusions}
\include{chapters/1Project/main}
\include{chapters/2Lit/main}
\include{chapters/3Design/HighLevel}
\include{chapters/3Design/InDepth}
\include{chapters/4Impl/main}

\include{chapters/5Experiments/1/main}
\include{chapters/5Experiments/2/main}
\include{chapters/5Experiments/3/main}
\include{chapters/5Experiments/4/main}

\include{chapters/6Conclusion/main}

\appendix
\include{appendix/AppendixB}
\include{appendix/D/main}
\include{appendix/AppendixC}

\backmatter
\bibliographystyle{ecs}
\bibliography{ECS}
\end{document}
%% ----------------------------------------------------------------

 %% ----------------------------------------------------------------
%% Progress.tex
%% ---------------------------------------------------------------- 
\documentclass{ecsprogress}    % Use the progress Style
\graphicspath{{../figs/}}   % Location of your graphics files
    \usepackage{natbib}            % Use Natbib style for the refs.
\hypersetup{colorlinks=true}   % Set to false for black/white printing
\input{Definitions}            % Include your abbreviations



\usepackage{enumitem}% http://ctan.org/pkg/enumitem
\usepackage{multirow}
\usepackage{float}
\usepackage{amsmath}
\usepackage{multicol}
\usepackage{amssymb}
\usepackage[normalem]{ulem}
\useunder{\uline}{\ul}{}
\usepackage{wrapfig}


\usepackage[table,xcdraw]{xcolor}


%% ----------------------------------------------------------------
\begin{document}
\frontmatter
\title      {Heterogeneous Agent-based Model for Supermarket Competition}
\authors    {\texorpdfstring
             {\href{mailto:sc22g13@ecs.soton.ac.uk}{Stefan J. Collier}}
             {Stefan J. Collier}
            }
\addresses  {\groupname\\\deptname\\\univname}
\date       {\today}
\subject    {}
\keywords   {}
\supervisor {Dr. Maria Polukarov}
\examiner   {Professor Sheng Chen}

\maketitle
\begin{abstract}
This project aim was to model and analyse the effects of competitive pricing behaviors of grocery retailers on the British market. 

This was achieved by creating a multi-agent model, containing retailer and consumer agents. The heterogeneous crowd of retailers employs either a uniform pricing strategy or a ‘local price flexing’ strategy. The actions of these retailers are chosen by predicting the profit of each action, using a perceptron. Following on from the consideration of different economic models, a discrete model was developed so that software agents have a discrete environment to operate within. Within the model, it has been observed how supermarkets with differing behaviors affect a heterogeneous crowd of consumer agents. The model was implemented in Java with Python used to evaluate the results. 

The simulation displays good acceptance with real grocery market behavior, i.e. captures the performance of British retailers thus can be used to determine the impact of changes in their behavior on their competitors and consumers.Furthermore it can be used to provide insight into sustainability of volatile pricing strategies, providing a useful insight in volatility of British supermarket retail industry. 
\end{abstract}
\acknowledgements{
I would like to express my sincere gratitude to Dr Maria Polukarov for her guidance and support which provided me the freedom to take this research in the direction of my interest.\\
\\
I would also like to thank my family and friends for their encouragement and support. To those who quietly listened to my software complaints. To those who worked throughout the nights with me. To those who helped me write what I couldn't say. I cannot thank you enough.
}

\declaration{
I, Stefan Collier, declare that this dissertation and the work presented in it are my own and has been generated by me as the result of my own original research.\\
I confirm that:\\
1. This work was done wholly or mainly while in candidature for a degree at this University;\\
2. Where any part of this dissertation has previously been submitted for any other qualification at this University or any other institution, this has been clearly stated;\\
3. Where I have consulted the published work of others, this is always clearly attributed;\\
4. Where I have quoted from the work of others, the source is always given. With the exception of such quotations, this dissertation is entirely my own work;\\
5. I have acknowledged all main sources of help;\\
6. Where the thesis is based on work done by myself jointly with others, I have made clear exactly what was done by others and what I have contributed myself;\\
7. Either none of this work has been published before submission, or parts of this work have been published by :\\
\\
Stefan Collier\\
April 2016
}
\tableofcontents
\listoffigures
\listoftables

\mainmatter
%% ----------------------------------------------------------------
%\include{Introduction}
%\include{Conclusions}
\include{chapters/1Project/main}
\include{chapters/2Lit/main}
\include{chapters/3Design/HighLevel}
\include{chapters/3Design/InDepth}
\include{chapters/4Impl/main}

\include{chapters/5Experiments/1/main}
\include{chapters/5Experiments/2/main}
\include{chapters/5Experiments/3/main}
\include{chapters/5Experiments/4/main}

\include{chapters/6Conclusion/main}

\appendix
\include{appendix/AppendixB}
\include{appendix/D/main}
\include{appendix/AppendixC}

\backmatter
\bibliographystyle{ecs}
\bibliography{ECS}
\end{document}
%% ----------------------------------------------------------------

 %% ----------------------------------------------------------------
%% Progress.tex
%% ---------------------------------------------------------------- 
\documentclass{ecsprogress}    % Use the progress Style
\graphicspath{{../figs/}}   % Location of your graphics files
    \usepackage{natbib}            % Use Natbib style for the refs.
\hypersetup{colorlinks=true}   % Set to false for black/white printing
\input{Definitions}            % Include your abbreviations



\usepackage{enumitem}% http://ctan.org/pkg/enumitem
\usepackage{multirow}
\usepackage{float}
\usepackage{amsmath}
\usepackage{multicol}
\usepackage{amssymb}
\usepackage[normalem]{ulem}
\useunder{\uline}{\ul}{}
\usepackage{wrapfig}


\usepackage[table,xcdraw]{xcolor}


%% ----------------------------------------------------------------
\begin{document}
\frontmatter
\title      {Heterogeneous Agent-based Model for Supermarket Competition}
\authors    {\texorpdfstring
             {\href{mailto:sc22g13@ecs.soton.ac.uk}{Stefan J. Collier}}
             {Stefan J. Collier}
            }
\addresses  {\groupname\\\deptname\\\univname}
\date       {\today}
\subject    {}
\keywords   {}
\supervisor {Dr. Maria Polukarov}
\examiner   {Professor Sheng Chen}

\maketitle
\begin{abstract}
This project aim was to model and analyse the effects of competitive pricing behaviors of grocery retailers on the British market. 

This was achieved by creating a multi-agent model, containing retailer and consumer agents. The heterogeneous crowd of retailers employs either a uniform pricing strategy or a ‘local price flexing’ strategy. The actions of these retailers are chosen by predicting the profit of each action, using a perceptron. Following on from the consideration of different economic models, a discrete model was developed so that software agents have a discrete environment to operate within. Within the model, it has been observed how supermarkets with differing behaviors affect a heterogeneous crowd of consumer agents. The model was implemented in Java with Python used to evaluate the results. 

The simulation displays good acceptance with real grocery market behavior, i.e. captures the performance of British retailers thus can be used to determine the impact of changes in their behavior on their competitors and consumers.Furthermore it can be used to provide insight into sustainability of volatile pricing strategies, providing a useful insight in volatility of British supermarket retail industry. 
\end{abstract}
\acknowledgements{
I would like to express my sincere gratitude to Dr Maria Polukarov for her guidance and support which provided me the freedom to take this research in the direction of my interest.\\
\\
I would also like to thank my family and friends for their encouragement and support. To those who quietly listened to my software complaints. To those who worked throughout the nights with me. To those who helped me write what I couldn't say. I cannot thank you enough.
}

\declaration{
I, Stefan Collier, declare that this dissertation and the work presented in it are my own and has been generated by me as the result of my own original research.\\
I confirm that:\\
1. This work was done wholly or mainly while in candidature for a degree at this University;\\
2. Where any part of this dissertation has previously been submitted for any other qualification at this University or any other institution, this has been clearly stated;\\
3. Where I have consulted the published work of others, this is always clearly attributed;\\
4. Where I have quoted from the work of others, the source is always given. With the exception of such quotations, this dissertation is entirely my own work;\\
5. I have acknowledged all main sources of help;\\
6. Where the thesis is based on work done by myself jointly with others, I have made clear exactly what was done by others and what I have contributed myself;\\
7. Either none of this work has been published before submission, or parts of this work have been published by :\\
\\
Stefan Collier\\
April 2016
}
\tableofcontents
\listoffigures
\listoftables

\mainmatter
%% ----------------------------------------------------------------
%\include{Introduction}
%\include{Conclusions}
\include{chapters/1Project/main}
\include{chapters/2Lit/main}
\include{chapters/3Design/HighLevel}
\include{chapters/3Design/InDepth}
\include{chapters/4Impl/main}

\include{chapters/5Experiments/1/main}
\include{chapters/5Experiments/2/main}
\include{chapters/5Experiments/3/main}
\include{chapters/5Experiments/4/main}

\include{chapters/6Conclusion/main}

\appendix
\include{appendix/AppendixB}
\include{appendix/D/main}
\include{appendix/AppendixC}

\backmatter
\bibliographystyle{ecs}
\bibliography{ECS}
\end{document}
%% ----------------------------------------------------------------


 %% ----------------------------------------------------------------
%% Progress.tex
%% ---------------------------------------------------------------- 
\documentclass{ecsprogress}    % Use the progress Style
\graphicspath{{../figs/}}   % Location of your graphics files
    \usepackage{natbib}            % Use Natbib style for the refs.
\hypersetup{colorlinks=true}   % Set to false for black/white printing
\input{Definitions}            % Include your abbreviations



\usepackage{enumitem}% http://ctan.org/pkg/enumitem
\usepackage{multirow}
\usepackage{float}
\usepackage{amsmath}
\usepackage{multicol}
\usepackage{amssymb}
\usepackage[normalem]{ulem}
\useunder{\uline}{\ul}{}
\usepackage{wrapfig}


\usepackage[table,xcdraw]{xcolor}


%% ----------------------------------------------------------------
\begin{document}
\frontmatter
\title      {Heterogeneous Agent-based Model for Supermarket Competition}
\authors    {\texorpdfstring
             {\href{mailto:sc22g13@ecs.soton.ac.uk}{Stefan J. Collier}}
             {Stefan J. Collier}
            }
\addresses  {\groupname\\\deptname\\\univname}
\date       {\today}
\subject    {}
\keywords   {}
\supervisor {Dr. Maria Polukarov}
\examiner   {Professor Sheng Chen}

\maketitle
\begin{abstract}
This project aim was to model and analyse the effects of competitive pricing behaviors of grocery retailers on the British market. 

This was achieved by creating a multi-agent model, containing retailer and consumer agents. The heterogeneous crowd of retailers employs either a uniform pricing strategy or a ‘local price flexing’ strategy. The actions of these retailers are chosen by predicting the profit of each action, using a perceptron. Following on from the consideration of different economic models, a discrete model was developed so that software agents have a discrete environment to operate within. Within the model, it has been observed how supermarkets with differing behaviors affect a heterogeneous crowd of consumer agents. The model was implemented in Java with Python used to evaluate the results. 

The simulation displays good acceptance with real grocery market behavior, i.e. captures the performance of British retailers thus can be used to determine the impact of changes in their behavior on their competitors and consumers.Furthermore it can be used to provide insight into sustainability of volatile pricing strategies, providing a useful insight in volatility of British supermarket retail industry. 
\end{abstract}
\acknowledgements{
I would like to express my sincere gratitude to Dr Maria Polukarov for her guidance and support which provided me the freedom to take this research in the direction of my interest.\\
\\
I would also like to thank my family and friends for their encouragement and support. To those who quietly listened to my software complaints. To those who worked throughout the nights with me. To those who helped me write what I couldn't say. I cannot thank you enough.
}

\declaration{
I, Stefan Collier, declare that this dissertation and the work presented in it are my own and has been generated by me as the result of my own original research.\\
I confirm that:\\
1. This work was done wholly or mainly while in candidature for a degree at this University;\\
2. Where any part of this dissertation has previously been submitted for any other qualification at this University or any other institution, this has been clearly stated;\\
3. Where I have consulted the published work of others, this is always clearly attributed;\\
4. Where I have quoted from the work of others, the source is always given. With the exception of such quotations, this dissertation is entirely my own work;\\
5. I have acknowledged all main sources of help;\\
6. Where the thesis is based on work done by myself jointly with others, I have made clear exactly what was done by others and what I have contributed myself;\\
7. Either none of this work has been published before submission, or parts of this work have been published by :\\
\\
Stefan Collier\\
April 2016
}
\tableofcontents
\listoffigures
\listoftables

\mainmatter
%% ----------------------------------------------------------------
%\include{Introduction}
%\include{Conclusions}
\include{chapters/1Project/main}
\include{chapters/2Lit/main}
\include{chapters/3Design/HighLevel}
\include{chapters/3Design/InDepth}
\include{chapters/4Impl/main}

\include{chapters/5Experiments/1/main}
\include{chapters/5Experiments/2/main}
\include{chapters/5Experiments/3/main}
\include{chapters/5Experiments/4/main}

\include{chapters/6Conclusion/main}

\appendix
\include{appendix/AppendixB}
\include{appendix/D/main}
\include{appendix/AppendixC}

\backmatter
\bibliographystyle{ecs}
\bibliography{ECS}
\end{document}
%% ----------------------------------------------------------------


\appendix
\include{appendix/AppendixB}
 %% ----------------------------------------------------------------
%% Progress.tex
%% ---------------------------------------------------------------- 
\documentclass{ecsprogress}    % Use the progress Style
\graphicspath{{../figs/}}   % Location of your graphics files
    \usepackage{natbib}            % Use Natbib style for the refs.
\hypersetup{colorlinks=true}   % Set to false for black/white printing
\input{Definitions}            % Include your abbreviations



\usepackage{enumitem}% http://ctan.org/pkg/enumitem
\usepackage{multirow}
\usepackage{float}
\usepackage{amsmath}
\usepackage{multicol}
\usepackage{amssymb}
\usepackage[normalem]{ulem}
\useunder{\uline}{\ul}{}
\usepackage{wrapfig}


\usepackage[table,xcdraw]{xcolor}


%% ----------------------------------------------------------------
\begin{document}
\frontmatter
\title      {Heterogeneous Agent-based Model for Supermarket Competition}
\authors    {\texorpdfstring
             {\href{mailto:sc22g13@ecs.soton.ac.uk}{Stefan J. Collier}}
             {Stefan J. Collier}
            }
\addresses  {\groupname\\\deptname\\\univname}
\date       {\today}
\subject    {}
\keywords   {}
\supervisor {Dr. Maria Polukarov}
\examiner   {Professor Sheng Chen}

\maketitle
\begin{abstract}
This project aim was to model and analyse the effects of competitive pricing behaviors of grocery retailers on the British market. 

This was achieved by creating a multi-agent model, containing retailer and consumer agents. The heterogeneous crowd of retailers employs either a uniform pricing strategy or a ‘local price flexing’ strategy. The actions of these retailers are chosen by predicting the profit of each action, using a perceptron. Following on from the consideration of different economic models, a discrete model was developed so that software agents have a discrete environment to operate within. Within the model, it has been observed how supermarkets with differing behaviors affect a heterogeneous crowd of consumer agents. The model was implemented in Java with Python used to evaluate the results. 

The simulation displays good acceptance with real grocery market behavior, i.e. captures the performance of British retailers thus can be used to determine the impact of changes in their behavior on their competitors and consumers.Furthermore it can be used to provide insight into sustainability of volatile pricing strategies, providing a useful insight in volatility of British supermarket retail industry. 
\end{abstract}
\acknowledgements{
I would like to express my sincere gratitude to Dr Maria Polukarov for her guidance and support which provided me the freedom to take this research in the direction of my interest.\\
\\
I would also like to thank my family and friends for their encouragement and support. To those who quietly listened to my software complaints. To those who worked throughout the nights with me. To those who helped me write what I couldn't say. I cannot thank you enough.
}

\declaration{
I, Stefan Collier, declare that this dissertation and the work presented in it are my own and has been generated by me as the result of my own original research.\\
I confirm that:\\
1. This work was done wholly or mainly while in candidature for a degree at this University;\\
2. Where any part of this dissertation has previously been submitted for any other qualification at this University or any other institution, this has been clearly stated;\\
3. Where I have consulted the published work of others, this is always clearly attributed;\\
4. Where I have quoted from the work of others, the source is always given. With the exception of such quotations, this dissertation is entirely my own work;\\
5. I have acknowledged all main sources of help;\\
6. Where the thesis is based on work done by myself jointly with others, I have made clear exactly what was done by others and what I have contributed myself;\\
7. Either none of this work has been published before submission, or parts of this work have been published by :\\
\\
Stefan Collier\\
April 2016
}
\tableofcontents
\listoffigures
\listoftables

\mainmatter
%% ----------------------------------------------------------------
%\include{Introduction}
%\include{Conclusions}
\include{chapters/1Project/main}
\include{chapters/2Lit/main}
\include{chapters/3Design/HighLevel}
\include{chapters/3Design/InDepth}
\include{chapters/4Impl/main}

\include{chapters/5Experiments/1/main}
\include{chapters/5Experiments/2/main}
\include{chapters/5Experiments/3/main}
\include{chapters/5Experiments/4/main}

\include{chapters/6Conclusion/main}

\appendix
\include{appendix/AppendixB}
\include{appendix/D/main}
\include{appendix/AppendixC}

\backmatter
\bibliographystyle{ecs}
\bibliography{ECS}
\end{document}
%% ----------------------------------------------------------------

\include{appendix/AppendixC}

\backmatter
\bibliographystyle{ecs}
\bibliography{ECS}
\end{document}
%% ----------------------------------------------------------------

 %% ----------------------------------------------------------------
%% Progress.tex
%% ---------------------------------------------------------------- 
\documentclass{ecsprogress}    % Use the progress Style
\graphicspath{{../figs/}}   % Location of your graphics files
    \usepackage{natbib}            % Use Natbib style for the refs.
\hypersetup{colorlinks=true}   % Set to false for black/white printing
\input{Definitions}            % Include your abbreviations



\usepackage{enumitem}% http://ctan.org/pkg/enumitem
\usepackage{multirow}
\usepackage{float}
\usepackage{amsmath}
\usepackage{multicol}
\usepackage{amssymb}
\usepackage[normalem]{ulem}
\useunder{\uline}{\ul}{}
\usepackage{wrapfig}


\usepackage[table,xcdraw]{xcolor}


%% ----------------------------------------------------------------
\begin{document}
\frontmatter
\title      {Heterogeneous Agent-based Model for Supermarket Competition}
\authors    {\texorpdfstring
             {\href{mailto:sc22g13@ecs.soton.ac.uk}{Stefan J. Collier}}
             {Stefan J. Collier}
            }
\addresses  {\groupname\\\deptname\\\univname}
\date       {\today}
\subject    {}
\keywords   {}
\supervisor {Dr. Maria Polukarov}
\examiner   {Professor Sheng Chen}

\maketitle
\begin{abstract}
This project aim was to model and analyse the effects of competitive pricing behaviors of grocery retailers on the British market. 

This was achieved by creating a multi-agent model, containing retailer and consumer agents. The heterogeneous crowd of retailers employs either a uniform pricing strategy or a ‘local price flexing’ strategy. The actions of these retailers are chosen by predicting the profit of each action, using a perceptron. Following on from the consideration of different economic models, a discrete model was developed so that software agents have a discrete environment to operate within. Within the model, it has been observed how supermarkets with differing behaviors affect a heterogeneous crowd of consumer agents. The model was implemented in Java with Python used to evaluate the results. 

The simulation displays good acceptance with real grocery market behavior, i.e. captures the performance of British retailers thus can be used to determine the impact of changes in their behavior on their competitors and consumers.Furthermore it can be used to provide insight into sustainability of volatile pricing strategies, providing a useful insight in volatility of British supermarket retail industry. 
\end{abstract}
\acknowledgements{
I would like to express my sincere gratitude to Dr Maria Polukarov for her guidance and support which provided me the freedom to take this research in the direction of my interest.\\
\\
I would also like to thank my family and friends for their encouragement and support. To those who quietly listened to my software complaints. To those who worked throughout the nights with me. To those who helped me write what I couldn't say. I cannot thank you enough.
}

\declaration{
I, Stefan Collier, declare that this dissertation and the work presented in it are my own and has been generated by me as the result of my own original research.\\
I confirm that:\\
1. This work was done wholly or mainly while in candidature for a degree at this University;\\
2. Where any part of this dissertation has previously been submitted for any other qualification at this University or any other institution, this has been clearly stated;\\
3. Where I have consulted the published work of others, this is always clearly attributed;\\
4. Where I have quoted from the work of others, the source is always given. With the exception of such quotations, this dissertation is entirely my own work;\\
5. I have acknowledged all main sources of help;\\
6. Where the thesis is based on work done by myself jointly with others, I have made clear exactly what was done by others and what I have contributed myself;\\
7. Either none of this work has been published before submission, or parts of this work have been published by :\\
\\
Stefan Collier\\
April 2016
}
\tableofcontents
\listoffigures
\listoftables

\mainmatter
%% ----------------------------------------------------------------
%\include{Introduction}
%\include{Conclusions}
 %% ----------------------------------------------------------------
%% Progress.tex
%% ---------------------------------------------------------------- 
\documentclass{ecsprogress}    % Use the progress Style
\graphicspath{{../figs/}}   % Location of your graphics files
    \usepackage{natbib}            % Use Natbib style for the refs.
\hypersetup{colorlinks=true}   % Set to false for black/white printing
\input{Definitions}            % Include your abbreviations



\usepackage{enumitem}% http://ctan.org/pkg/enumitem
\usepackage{multirow}
\usepackage{float}
\usepackage{amsmath}
\usepackage{multicol}
\usepackage{amssymb}
\usepackage[normalem]{ulem}
\useunder{\uline}{\ul}{}
\usepackage{wrapfig}


\usepackage[table,xcdraw]{xcolor}


%% ----------------------------------------------------------------
\begin{document}
\frontmatter
\title      {Heterogeneous Agent-based Model for Supermarket Competition}
\authors    {\texorpdfstring
             {\href{mailto:sc22g13@ecs.soton.ac.uk}{Stefan J. Collier}}
             {Stefan J. Collier}
            }
\addresses  {\groupname\\\deptname\\\univname}
\date       {\today}
\subject    {}
\keywords   {}
\supervisor {Dr. Maria Polukarov}
\examiner   {Professor Sheng Chen}

\maketitle
\begin{abstract}
This project aim was to model and analyse the effects of competitive pricing behaviors of grocery retailers on the British market. 

This was achieved by creating a multi-agent model, containing retailer and consumer agents. The heterogeneous crowd of retailers employs either a uniform pricing strategy or a ‘local price flexing’ strategy. The actions of these retailers are chosen by predicting the profit of each action, using a perceptron. Following on from the consideration of different economic models, a discrete model was developed so that software agents have a discrete environment to operate within. Within the model, it has been observed how supermarkets with differing behaviors affect a heterogeneous crowd of consumer agents. The model was implemented in Java with Python used to evaluate the results. 

The simulation displays good acceptance with real grocery market behavior, i.e. captures the performance of British retailers thus can be used to determine the impact of changes in their behavior on their competitors and consumers.Furthermore it can be used to provide insight into sustainability of volatile pricing strategies, providing a useful insight in volatility of British supermarket retail industry. 
\end{abstract}
\acknowledgements{
I would like to express my sincere gratitude to Dr Maria Polukarov for her guidance and support which provided me the freedom to take this research in the direction of my interest.\\
\\
I would also like to thank my family and friends for their encouragement and support. To those who quietly listened to my software complaints. To those who worked throughout the nights with me. To those who helped me write what I couldn't say. I cannot thank you enough.
}

\declaration{
I, Stefan Collier, declare that this dissertation and the work presented in it are my own and has been generated by me as the result of my own original research.\\
I confirm that:\\
1. This work was done wholly or mainly while in candidature for a degree at this University;\\
2. Where any part of this dissertation has previously been submitted for any other qualification at this University or any other institution, this has been clearly stated;\\
3. Where I have consulted the published work of others, this is always clearly attributed;\\
4. Where I have quoted from the work of others, the source is always given. With the exception of such quotations, this dissertation is entirely my own work;\\
5. I have acknowledged all main sources of help;\\
6. Where the thesis is based on work done by myself jointly with others, I have made clear exactly what was done by others and what I have contributed myself;\\
7. Either none of this work has been published before submission, or parts of this work have been published by :\\
\\
Stefan Collier\\
April 2016
}
\tableofcontents
\listoffigures
\listoftables

\mainmatter
%% ----------------------------------------------------------------
%\include{Introduction}
%\include{Conclusions}
\include{chapters/1Project/main}
\include{chapters/2Lit/main}
\include{chapters/3Design/HighLevel}
\include{chapters/3Design/InDepth}
\include{chapters/4Impl/main}

\include{chapters/5Experiments/1/main}
\include{chapters/5Experiments/2/main}
\include{chapters/5Experiments/3/main}
\include{chapters/5Experiments/4/main}

\include{chapters/6Conclusion/main}

\appendix
\include{appendix/AppendixB}
\include{appendix/D/main}
\include{appendix/AppendixC}

\backmatter
\bibliographystyle{ecs}
\bibliography{ECS}
\end{document}
%% ----------------------------------------------------------------

 %% ----------------------------------------------------------------
%% Progress.tex
%% ---------------------------------------------------------------- 
\documentclass{ecsprogress}    % Use the progress Style
\graphicspath{{../figs/}}   % Location of your graphics files
    \usepackage{natbib}            % Use Natbib style for the refs.
\hypersetup{colorlinks=true}   % Set to false for black/white printing
\input{Definitions}            % Include your abbreviations



\usepackage{enumitem}% http://ctan.org/pkg/enumitem
\usepackage{multirow}
\usepackage{float}
\usepackage{amsmath}
\usepackage{multicol}
\usepackage{amssymb}
\usepackage[normalem]{ulem}
\useunder{\uline}{\ul}{}
\usepackage{wrapfig}


\usepackage[table,xcdraw]{xcolor}


%% ----------------------------------------------------------------
\begin{document}
\frontmatter
\title      {Heterogeneous Agent-based Model for Supermarket Competition}
\authors    {\texorpdfstring
             {\href{mailto:sc22g13@ecs.soton.ac.uk}{Stefan J. Collier}}
             {Stefan J. Collier}
            }
\addresses  {\groupname\\\deptname\\\univname}
\date       {\today}
\subject    {}
\keywords   {}
\supervisor {Dr. Maria Polukarov}
\examiner   {Professor Sheng Chen}

\maketitle
\begin{abstract}
This project aim was to model and analyse the effects of competitive pricing behaviors of grocery retailers on the British market. 

This was achieved by creating a multi-agent model, containing retailer and consumer agents. The heterogeneous crowd of retailers employs either a uniform pricing strategy or a ‘local price flexing’ strategy. The actions of these retailers are chosen by predicting the profit of each action, using a perceptron. Following on from the consideration of different economic models, a discrete model was developed so that software agents have a discrete environment to operate within. Within the model, it has been observed how supermarkets with differing behaviors affect a heterogeneous crowd of consumer agents. The model was implemented in Java with Python used to evaluate the results. 

The simulation displays good acceptance with real grocery market behavior, i.e. captures the performance of British retailers thus can be used to determine the impact of changes in their behavior on their competitors and consumers.Furthermore it can be used to provide insight into sustainability of volatile pricing strategies, providing a useful insight in volatility of British supermarket retail industry. 
\end{abstract}
\acknowledgements{
I would like to express my sincere gratitude to Dr Maria Polukarov for her guidance and support which provided me the freedom to take this research in the direction of my interest.\\
\\
I would also like to thank my family and friends for their encouragement and support. To those who quietly listened to my software complaints. To those who worked throughout the nights with me. To those who helped me write what I couldn't say. I cannot thank you enough.
}

\declaration{
I, Stefan Collier, declare that this dissertation and the work presented in it are my own and has been generated by me as the result of my own original research.\\
I confirm that:\\
1. This work was done wholly or mainly while in candidature for a degree at this University;\\
2. Where any part of this dissertation has previously been submitted for any other qualification at this University or any other institution, this has been clearly stated;\\
3. Where I have consulted the published work of others, this is always clearly attributed;\\
4. Where I have quoted from the work of others, the source is always given. With the exception of such quotations, this dissertation is entirely my own work;\\
5. I have acknowledged all main sources of help;\\
6. Where the thesis is based on work done by myself jointly with others, I have made clear exactly what was done by others and what I have contributed myself;\\
7. Either none of this work has been published before submission, or parts of this work have been published by :\\
\\
Stefan Collier\\
April 2016
}
\tableofcontents
\listoffigures
\listoftables

\mainmatter
%% ----------------------------------------------------------------
%\include{Introduction}
%\include{Conclusions}
\include{chapters/1Project/main}
\include{chapters/2Lit/main}
\include{chapters/3Design/HighLevel}
\include{chapters/3Design/InDepth}
\include{chapters/4Impl/main}

\include{chapters/5Experiments/1/main}
\include{chapters/5Experiments/2/main}
\include{chapters/5Experiments/3/main}
\include{chapters/5Experiments/4/main}

\include{chapters/6Conclusion/main}

\appendix
\include{appendix/AppendixB}
\include{appendix/D/main}
\include{appendix/AppendixC}

\backmatter
\bibliographystyle{ecs}
\bibliography{ECS}
\end{document}
%% ----------------------------------------------------------------

\include{chapters/3Design/HighLevel}
\include{chapters/3Design/InDepth}
 %% ----------------------------------------------------------------
%% Progress.tex
%% ---------------------------------------------------------------- 
\documentclass{ecsprogress}    % Use the progress Style
\graphicspath{{../figs/}}   % Location of your graphics files
    \usepackage{natbib}            % Use Natbib style for the refs.
\hypersetup{colorlinks=true}   % Set to false for black/white printing
\input{Definitions}            % Include your abbreviations



\usepackage{enumitem}% http://ctan.org/pkg/enumitem
\usepackage{multirow}
\usepackage{float}
\usepackage{amsmath}
\usepackage{multicol}
\usepackage{amssymb}
\usepackage[normalem]{ulem}
\useunder{\uline}{\ul}{}
\usepackage{wrapfig}


\usepackage[table,xcdraw]{xcolor}


%% ----------------------------------------------------------------
\begin{document}
\frontmatter
\title      {Heterogeneous Agent-based Model for Supermarket Competition}
\authors    {\texorpdfstring
             {\href{mailto:sc22g13@ecs.soton.ac.uk}{Stefan J. Collier}}
             {Stefan J. Collier}
            }
\addresses  {\groupname\\\deptname\\\univname}
\date       {\today}
\subject    {}
\keywords   {}
\supervisor {Dr. Maria Polukarov}
\examiner   {Professor Sheng Chen}

\maketitle
\begin{abstract}
This project aim was to model and analyse the effects of competitive pricing behaviors of grocery retailers on the British market. 

This was achieved by creating a multi-agent model, containing retailer and consumer agents. The heterogeneous crowd of retailers employs either a uniform pricing strategy or a ‘local price flexing’ strategy. The actions of these retailers are chosen by predicting the profit of each action, using a perceptron. Following on from the consideration of different economic models, a discrete model was developed so that software agents have a discrete environment to operate within. Within the model, it has been observed how supermarkets with differing behaviors affect a heterogeneous crowd of consumer agents. The model was implemented in Java with Python used to evaluate the results. 

The simulation displays good acceptance with real grocery market behavior, i.e. captures the performance of British retailers thus can be used to determine the impact of changes in their behavior on their competitors and consumers.Furthermore it can be used to provide insight into sustainability of volatile pricing strategies, providing a useful insight in volatility of British supermarket retail industry. 
\end{abstract}
\acknowledgements{
I would like to express my sincere gratitude to Dr Maria Polukarov for her guidance and support which provided me the freedom to take this research in the direction of my interest.\\
\\
I would also like to thank my family and friends for their encouragement and support. To those who quietly listened to my software complaints. To those who worked throughout the nights with me. To those who helped me write what I couldn't say. I cannot thank you enough.
}

\declaration{
I, Stefan Collier, declare that this dissertation and the work presented in it are my own and has been generated by me as the result of my own original research.\\
I confirm that:\\
1. This work was done wholly or mainly while in candidature for a degree at this University;\\
2. Where any part of this dissertation has previously been submitted for any other qualification at this University or any other institution, this has been clearly stated;\\
3. Where I have consulted the published work of others, this is always clearly attributed;\\
4. Where I have quoted from the work of others, the source is always given. With the exception of such quotations, this dissertation is entirely my own work;\\
5. I have acknowledged all main sources of help;\\
6. Where the thesis is based on work done by myself jointly with others, I have made clear exactly what was done by others and what I have contributed myself;\\
7. Either none of this work has been published before submission, or parts of this work have been published by :\\
\\
Stefan Collier\\
April 2016
}
\tableofcontents
\listoffigures
\listoftables

\mainmatter
%% ----------------------------------------------------------------
%\include{Introduction}
%\include{Conclusions}
\include{chapters/1Project/main}
\include{chapters/2Lit/main}
\include{chapters/3Design/HighLevel}
\include{chapters/3Design/InDepth}
\include{chapters/4Impl/main}

\include{chapters/5Experiments/1/main}
\include{chapters/5Experiments/2/main}
\include{chapters/5Experiments/3/main}
\include{chapters/5Experiments/4/main}

\include{chapters/6Conclusion/main}

\appendix
\include{appendix/AppendixB}
\include{appendix/D/main}
\include{appendix/AppendixC}

\backmatter
\bibliographystyle{ecs}
\bibliography{ECS}
\end{document}
%% ----------------------------------------------------------------


 %% ----------------------------------------------------------------
%% Progress.tex
%% ---------------------------------------------------------------- 
\documentclass{ecsprogress}    % Use the progress Style
\graphicspath{{../figs/}}   % Location of your graphics files
    \usepackage{natbib}            % Use Natbib style for the refs.
\hypersetup{colorlinks=true}   % Set to false for black/white printing
\input{Definitions}            % Include your abbreviations



\usepackage{enumitem}% http://ctan.org/pkg/enumitem
\usepackage{multirow}
\usepackage{float}
\usepackage{amsmath}
\usepackage{multicol}
\usepackage{amssymb}
\usepackage[normalem]{ulem}
\useunder{\uline}{\ul}{}
\usepackage{wrapfig}


\usepackage[table,xcdraw]{xcolor}


%% ----------------------------------------------------------------
\begin{document}
\frontmatter
\title      {Heterogeneous Agent-based Model for Supermarket Competition}
\authors    {\texorpdfstring
             {\href{mailto:sc22g13@ecs.soton.ac.uk}{Stefan J. Collier}}
             {Stefan J. Collier}
            }
\addresses  {\groupname\\\deptname\\\univname}
\date       {\today}
\subject    {}
\keywords   {}
\supervisor {Dr. Maria Polukarov}
\examiner   {Professor Sheng Chen}

\maketitle
\begin{abstract}
This project aim was to model and analyse the effects of competitive pricing behaviors of grocery retailers on the British market. 

This was achieved by creating a multi-agent model, containing retailer and consumer agents. The heterogeneous crowd of retailers employs either a uniform pricing strategy or a ‘local price flexing’ strategy. The actions of these retailers are chosen by predicting the profit of each action, using a perceptron. Following on from the consideration of different economic models, a discrete model was developed so that software agents have a discrete environment to operate within. Within the model, it has been observed how supermarkets with differing behaviors affect a heterogeneous crowd of consumer agents. The model was implemented in Java with Python used to evaluate the results. 

The simulation displays good acceptance with real grocery market behavior, i.e. captures the performance of British retailers thus can be used to determine the impact of changes in their behavior on their competitors and consumers.Furthermore it can be used to provide insight into sustainability of volatile pricing strategies, providing a useful insight in volatility of British supermarket retail industry. 
\end{abstract}
\acknowledgements{
I would like to express my sincere gratitude to Dr Maria Polukarov for her guidance and support which provided me the freedom to take this research in the direction of my interest.\\
\\
I would also like to thank my family and friends for their encouragement and support. To those who quietly listened to my software complaints. To those who worked throughout the nights with me. To those who helped me write what I couldn't say. I cannot thank you enough.
}

\declaration{
I, Stefan Collier, declare that this dissertation and the work presented in it are my own and has been generated by me as the result of my own original research.\\
I confirm that:\\
1. This work was done wholly or mainly while in candidature for a degree at this University;\\
2. Where any part of this dissertation has previously been submitted for any other qualification at this University or any other institution, this has been clearly stated;\\
3. Where I have consulted the published work of others, this is always clearly attributed;\\
4. Where I have quoted from the work of others, the source is always given. With the exception of such quotations, this dissertation is entirely my own work;\\
5. I have acknowledged all main sources of help;\\
6. Where the thesis is based on work done by myself jointly with others, I have made clear exactly what was done by others and what I have contributed myself;\\
7. Either none of this work has been published before submission, or parts of this work have been published by :\\
\\
Stefan Collier\\
April 2016
}
\tableofcontents
\listoffigures
\listoftables

\mainmatter
%% ----------------------------------------------------------------
%\include{Introduction}
%\include{Conclusions}
\include{chapters/1Project/main}
\include{chapters/2Lit/main}
\include{chapters/3Design/HighLevel}
\include{chapters/3Design/InDepth}
\include{chapters/4Impl/main}

\include{chapters/5Experiments/1/main}
\include{chapters/5Experiments/2/main}
\include{chapters/5Experiments/3/main}
\include{chapters/5Experiments/4/main}

\include{chapters/6Conclusion/main}

\appendix
\include{appendix/AppendixB}
\include{appendix/D/main}
\include{appendix/AppendixC}

\backmatter
\bibliographystyle{ecs}
\bibliography{ECS}
\end{document}
%% ----------------------------------------------------------------

 %% ----------------------------------------------------------------
%% Progress.tex
%% ---------------------------------------------------------------- 
\documentclass{ecsprogress}    % Use the progress Style
\graphicspath{{../figs/}}   % Location of your graphics files
    \usepackage{natbib}            % Use Natbib style for the refs.
\hypersetup{colorlinks=true}   % Set to false for black/white printing
\input{Definitions}            % Include your abbreviations



\usepackage{enumitem}% http://ctan.org/pkg/enumitem
\usepackage{multirow}
\usepackage{float}
\usepackage{amsmath}
\usepackage{multicol}
\usepackage{amssymb}
\usepackage[normalem]{ulem}
\useunder{\uline}{\ul}{}
\usepackage{wrapfig}


\usepackage[table,xcdraw]{xcolor}


%% ----------------------------------------------------------------
\begin{document}
\frontmatter
\title      {Heterogeneous Agent-based Model for Supermarket Competition}
\authors    {\texorpdfstring
             {\href{mailto:sc22g13@ecs.soton.ac.uk}{Stefan J. Collier}}
             {Stefan J. Collier}
            }
\addresses  {\groupname\\\deptname\\\univname}
\date       {\today}
\subject    {}
\keywords   {}
\supervisor {Dr. Maria Polukarov}
\examiner   {Professor Sheng Chen}

\maketitle
\begin{abstract}
This project aim was to model and analyse the effects of competitive pricing behaviors of grocery retailers on the British market. 

This was achieved by creating a multi-agent model, containing retailer and consumer agents. The heterogeneous crowd of retailers employs either a uniform pricing strategy or a ‘local price flexing’ strategy. The actions of these retailers are chosen by predicting the profit of each action, using a perceptron. Following on from the consideration of different economic models, a discrete model was developed so that software agents have a discrete environment to operate within. Within the model, it has been observed how supermarkets with differing behaviors affect a heterogeneous crowd of consumer agents. The model was implemented in Java with Python used to evaluate the results. 

The simulation displays good acceptance with real grocery market behavior, i.e. captures the performance of British retailers thus can be used to determine the impact of changes in their behavior on their competitors and consumers.Furthermore it can be used to provide insight into sustainability of volatile pricing strategies, providing a useful insight in volatility of British supermarket retail industry. 
\end{abstract}
\acknowledgements{
I would like to express my sincere gratitude to Dr Maria Polukarov for her guidance and support which provided me the freedom to take this research in the direction of my interest.\\
\\
I would also like to thank my family and friends for their encouragement and support. To those who quietly listened to my software complaints. To those who worked throughout the nights with me. To those who helped me write what I couldn't say. I cannot thank you enough.
}

\declaration{
I, Stefan Collier, declare that this dissertation and the work presented in it are my own and has been generated by me as the result of my own original research.\\
I confirm that:\\
1. This work was done wholly or mainly while in candidature for a degree at this University;\\
2. Where any part of this dissertation has previously been submitted for any other qualification at this University or any other institution, this has been clearly stated;\\
3. Where I have consulted the published work of others, this is always clearly attributed;\\
4. Where I have quoted from the work of others, the source is always given. With the exception of such quotations, this dissertation is entirely my own work;\\
5. I have acknowledged all main sources of help;\\
6. Where the thesis is based on work done by myself jointly with others, I have made clear exactly what was done by others and what I have contributed myself;\\
7. Either none of this work has been published before submission, or parts of this work have been published by :\\
\\
Stefan Collier\\
April 2016
}
\tableofcontents
\listoffigures
\listoftables

\mainmatter
%% ----------------------------------------------------------------
%\include{Introduction}
%\include{Conclusions}
\include{chapters/1Project/main}
\include{chapters/2Lit/main}
\include{chapters/3Design/HighLevel}
\include{chapters/3Design/InDepth}
\include{chapters/4Impl/main}

\include{chapters/5Experiments/1/main}
\include{chapters/5Experiments/2/main}
\include{chapters/5Experiments/3/main}
\include{chapters/5Experiments/4/main}

\include{chapters/6Conclusion/main}

\appendix
\include{appendix/AppendixB}
\include{appendix/D/main}
\include{appendix/AppendixC}

\backmatter
\bibliographystyle{ecs}
\bibliography{ECS}
\end{document}
%% ----------------------------------------------------------------

 %% ----------------------------------------------------------------
%% Progress.tex
%% ---------------------------------------------------------------- 
\documentclass{ecsprogress}    % Use the progress Style
\graphicspath{{../figs/}}   % Location of your graphics files
    \usepackage{natbib}            % Use Natbib style for the refs.
\hypersetup{colorlinks=true}   % Set to false for black/white printing
\input{Definitions}            % Include your abbreviations



\usepackage{enumitem}% http://ctan.org/pkg/enumitem
\usepackage{multirow}
\usepackage{float}
\usepackage{amsmath}
\usepackage{multicol}
\usepackage{amssymb}
\usepackage[normalem]{ulem}
\useunder{\uline}{\ul}{}
\usepackage{wrapfig}


\usepackage[table,xcdraw]{xcolor}


%% ----------------------------------------------------------------
\begin{document}
\frontmatter
\title      {Heterogeneous Agent-based Model for Supermarket Competition}
\authors    {\texorpdfstring
             {\href{mailto:sc22g13@ecs.soton.ac.uk}{Stefan J. Collier}}
             {Stefan J. Collier}
            }
\addresses  {\groupname\\\deptname\\\univname}
\date       {\today}
\subject    {}
\keywords   {}
\supervisor {Dr. Maria Polukarov}
\examiner   {Professor Sheng Chen}

\maketitle
\begin{abstract}
This project aim was to model and analyse the effects of competitive pricing behaviors of grocery retailers on the British market. 

This was achieved by creating a multi-agent model, containing retailer and consumer agents. The heterogeneous crowd of retailers employs either a uniform pricing strategy or a ‘local price flexing’ strategy. The actions of these retailers are chosen by predicting the profit of each action, using a perceptron. Following on from the consideration of different economic models, a discrete model was developed so that software agents have a discrete environment to operate within. Within the model, it has been observed how supermarkets with differing behaviors affect a heterogeneous crowd of consumer agents. The model was implemented in Java with Python used to evaluate the results. 

The simulation displays good acceptance with real grocery market behavior, i.e. captures the performance of British retailers thus can be used to determine the impact of changes in their behavior on their competitors and consumers.Furthermore it can be used to provide insight into sustainability of volatile pricing strategies, providing a useful insight in volatility of British supermarket retail industry. 
\end{abstract}
\acknowledgements{
I would like to express my sincere gratitude to Dr Maria Polukarov for her guidance and support which provided me the freedom to take this research in the direction of my interest.\\
\\
I would also like to thank my family and friends for their encouragement and support. To those who quietly listened to my software complaints. To those who worked throughout the nights with me. To those who helped me write what I couldn't say. I cannot thank you enough.
}

\declaration{
I, Stefan Collier, declare that this dissertation and the work presented in it are my own and has been generated by me as the result of my own original research.\\
I confirm that:\\
1. This work was done wholly or mainly while in candidature for a degree at this University;\\
2. Where any part of this dissertation has previously been submitted for any other qualification at this University or any other institution, this has been clearly stated;\\
3. Where I have consulted the published work of others, this is always clearly attributed;\\
4. Where I have quoted from the work of others, the source is always given. With the exception of such quotations, this dissertation is entirely my own work;\\
5. I have acknowledged all main sources of help;\\
6. Where the thesis is based on work done by myself jointly with others, I have made clear exactly what was done by others and what I have contributed myself;\\
7. Either none of this work has been published before submission, or parts of this work have been published by :\\
\\
Stefan Collier\\
April 2016
}
\tableofcontents
\listoffigures
\listoftables

\mainmatter
%% ----------------------------------------------------------------
%\include{Introduction}
%\include{Conclusions}
\include{chapters/1Project/main}
\include{chapters/2Lit/main}
\include{chapters/3Design/HighLevel}
\include{chapters/3Design/InDepth}
\include{chapters/4Impl/main}

\include{chapters/5Experiments/1/main}
\include{chapters/5Experiments/2/main}
\include{chapters/5Experiments/3/main}
\include{chapters/5Experiments/4/main}

\include{chapters/6Conclusion/main}

\appendix
\include{appendix/AppendixB}
\include{appendix/D/main}
\include{appendix/AppendixC}

\backmatter
\bibliographystyle{ecs}
\bibliography{ECS}
\end{document}
%% ----------------------------------------------------------------

 %% ----------------------------------------------------------------
%% Progress.tex
%% ---------------------------------------------------------------- 
\documentclass{ecsprogress}    % Use the progress Style
\graphicspath{{../figs/}}   % Location of your graphics files
    \usepackage{natbib}            % Use Natbib style for the refs.
\hypersetup{colorlinks=true}   % Set to false for black/white printing
\input{Definitions}            % Include your abbreviations



\usepackage{enumitem}% http://ctan.org/pkg/enumitem
\usepackage{multirow}
\usepackage{float}
\usepackage{amsmath}
\usepackage{multicol}
\usepackage{amssymb}
\usepackage[normalem]{ulem}
\useunder{\uline}{\ul}{}
\usepackage{wrapfig}


\usepackage[table,xcdraw]{xcolor}


%% ----------------------------------------------------------------
\begin{document}
\frontmatter
\title      {Heterogeneous Agent-based Model for Supermarket Competition}
\authors    {\texorpdfstring
             {\href{mailto:sc22g13@ecs.soton.ac.uk}{Stefan J. Collier}}
             {Stefan J. Collier}
            }
\addresses  {\groupname\\\deptname\\\univname}
\date       {\today}
\subject    {}
\keywords   {}
\supervisor {Dr. Maria Polukarov}
\examiner   {Professor Sheng Chen}

\maketitle
\begin{abstract}
This project aim was to model and analyse the effects of competitive pricing behaviors of grocery retailers on the British market. 

This was achieved by creating a multi-agent model, containing retailer and consumer agents. The heterogeneous crowd of retailers employs either a uniform pricing strategy or a ‘local price flexing’ strategy. The actions of these retailers are chosen by predicting the profit of each action, using a perceptron. Following on from the consideration of different economic models, a discrete model was developed so that software agents have a discrete environment to operate within. Within the model, it has been observed how supermarkets with differing behaviors affect a heterogeneous crowd of consumer agents. The model was implemented in Java with Python used to evaluate the results. 

The simulation displays good acceptance with real grocery market behavior, i.e. captures the performance of British retailers thus can be used to determine the impact of changes in their behavior on their competitors and consumers.Furthermore it can be used to provide insight into sustainability of volatile pricing strategies, providing a useful insight in volatility of British supermarket retail industry. 
\end{abstract}
\acknowledgements{
I would like to express my sincere gratitude to Dr Maria Polukarov for her guidance and support which provided me the freedom to take this research in the direction of my interest.\\
\\
I would also like to thank my family and friends for their encouragement and support. To those who quietly listened to my software complaints. To those who worked throughout the nights with me. To those who helped me write what I couldn't say. I cannot thank you enough.
}

\declaration{
I, Stefan Collier, declare that this dissertation and the work presented in it are my own and has been generated by me as the result of my own original research.\\
I confirm that:\\
1. This work was done wholly or mainly while in candidature for a degree at this University;\\
2. Where any part of this dissertation has previously been submitted for any other qualification at this University or any other institution, this has been clearly stated;\\
3. Where I have consulted the published work of others, this is always clearly attributed;\\
4. Where I have quoted from the work of others, the source is always given. With the exception of such quotations, this dissertation is entirely my own work;\\
5. I have acknowledged all main sources of help;\\
6. Where the thesis is based on work done by myself jointly with others, I have made clear exactly what was done by others and what I have contributed myself;\\
7. Either none of this work has been published before submission, or parts of this work have been published by :\\
\\
Stefan Collier\\
April 2016
}
\tableofcontents
\listoffigures
\listoftables

\mainmatter
%% ----------------------------------------------------------------
%\include{Introduction}
%\include{Conclusions}
\include{chapters/1Project/main}
\include{chapters/2Lit/main}
\include{chapters/3Design/HighLevel}
\include{chapters/3Design/InDepth}
\include{chapters/4Impl/main}

\include{chapters/5Experiments/1/main}
\include{chapters/5Experiments/2/main}
\include{chapters/5Experiments/3/main}
\include{chapters/5Experiments/4/main}

\include{chapters/6Conclusion/main}

\appendix
\include{appendix/AppendixB}
\include{appendix/D/main}
\include{appendix/AppendixC}

\backmatter
\bibliographystyle{ecs}
\bibliography{ECS}
\end{document}
%% ----------------------------------------------------------------


 %% ----------------------------------------------------------------
%% Progress.tex
%% ---------------------------------------------------------------- 
\documentclass{ecsprogress}    % Use the progress Style
\graphicspath{{../figs/}}   % Location of your graphics files
    \usepackage{natbib}            % Use Natbib style for the refs.
\hypersetup{colorlinks=true}   % Set to false for black/white printing
\input{Definitions}            % Include your abbreviations



\usepackage{enumitem}% http://ctan.org/pkg/enumitem
\usepackage{multirow}
\usepackage{float}
\usepackage{amsmath}
\usepackage{multicol}
\usepackage{amssymb}
\usepackage[normalem]{ulem}
\useunder{\uline}{\ul}{}
\usepackage{wrapfig}


\usepackage[table,xcdraw]{xcolor}


%% ----------------------------------------------------------------
\begin{document}
\frontmatter
\title      {Heterogeneous Agent-based Model for Supermarket Competition}
\authors    {\texorpdfstring
             {\href{mailto:sc22g13@ecs.soton.ac.uk}{Stefan J. Collier}}
             {Stefan J. Collier}
            }
\addresses  {\groupname\\\deptname\\\univname}
\date       {\today}
\subject    {}
\keywords   {}
\supervisor {Dr. Maria Polukarov}
\examiner   {Professor Sheng Chen}

\maketitle
\begin{abstract}
This project aim was to model and analyse the effects of competitive pricing behaviors of grocery retailers on the British market. 

This was achieved by creating a multi-agent model, containing retailer and consumer agents. The heterogeneous crowd of retailers employs either a uniform pricing strategy or a ‘local price flexing’ strategy. The actions of these retailers are chosen by predicting the profit of each action, using a perceptron. Following on from the consideration of different economic models, a discrete model was developed so that software agents have a discrete environment to operate within. Within the model, it has been observed how supermarkets with differing behaviors affect a heterogeneous crowd of consumer agents. The model was implemented in Java with Python used to evaluate the results. 

The simulation displays good acceptance with real grocery market behavior, i.e. captures the performance of British retailers thus can be used to determine the impact of changes in their behavior on their competitors and consumers.Furthermore it can be used to provide insight into sustainability of volatile pricing strategies, providing a useful insight in volatility of British supermarket retail industry. 
\end{abstract}
\acknowledgements{
I would like to express my sincere gratitude to Dr Maria Polukarov for her guidance and support which provided me the freedom to take this research in the direction of my interest.\\
\\
I would also like to thank my family and friends for their encouragement and support. To those who quietly listened to my software complaints. To those who worked throughout the nights with me. To those who helped me write what I couldn't say. I cannot thank you enough.
}

\declaration{
I, Stefan Collier, declare that this dissertation and the work presented in it are my own and has been generated by me as the result of my own original research.\\
I confirm that:\\
1. This work was done wholly or mainly while in candidature for a degree at this University;\\
2. Where any part of this dissertation has previously been submitted for any other qualification at this University or any other institution, this has been clearly stated;\\
3. Where I have consulted the published work of others, this is always clearly attributed;\\
4. Where I have quoted from the work of others, the source is always given. With the exception of such quotations, this dissertation is entirely my own work;\\
5. I have acknowledged all main sources of help;\\
6. Where the thesis is based on work done by myself jointly with others, I have made clear exactly what was done by others and what I have contributed myself;\\
7. Either none of this work has been published before submission, or parts of this work have been published by :\\
\\
Stefan Collier\\
April 2016
}
\tableofcontents
\listoffigures
\listoftables

\mainmatter
%% ----------------------------------------------------------------
%\include{Introduction}
%\include{Conclusions}
\include{chapters/1Project/main}
\include{chapters/2Lit/main}
\include{chapters/3Design/HighLevel}
\include{chapters/3Design/InDepth}
\include{chapters/4Impl/main}

\include{chapters/5Experiments/1/main}
\include{chapters/5Experiments/2/main}
\include{chapters/5Experiments/3/main}
\include{chapters/5Experiments/4/main}

\include{chapters/6Conclusion/main}

\appendix
\include{appendix/AppendixB}
\include{appendix/D/main}
\include{appendix/AppendixC}

\backmatter
\bibliographystyle{ecs}
\bibliography{ECS}
\end{document}
%% ----------------------------------------------------------------


\appendix
\include{appendix/AppendixB}
 %% ----------------------------------------------------------------
%% Progress.tex
%% ---------------------------------------------------------------- 
\documentclass{ecsprogress}    % Use the progress Style
\graphicspath{{../figs/}}   % Location of your graphics files
    \usepackage{natbib}            % Use Natbib style for the refs.
\hypersetup{colorlinks=true}   % Set to false for black/white printing
\input{Definitions}            % Include your abbreviations



\usepackage{enumitem}% http://ctan.org/pkg/enumitem
\usepackage{multirow}
\usepackage{float}
\usepackage{amsmath}
\usepackage{multicol}
\usepackage{amssymb}
\usepackage[normalem]{ulem}
\useunder{\uline}{\ul}{}
\usepackage{wrapfig}


\usepackage[table,xcdraw]{xcolor}


%% ----------------------------------------------------------------
\begin{document}
\frontmatter
\title      {Heterogeneous Agent-based Model for Supermarket Competition}
\authors    {\texorpdfstring
             {\href{mailto:sc22g13@ecs.soton.ac.uk}{Stefan J. Collier}}
             {Stefan J. Collier}
            }
\addresses  {\groupname\\\deptname\\\univname}
\date       {\today}
\subject    {}
\keywords   {}
\supervisor {Dr. Maria Polukarov}
\examiner   {Professor Sheng Chen}

\maketitle
\begin{abstract}
This project aim was to model and analyse the effects of competitive pricing behaviors of grocery retailers on the British market. 

This was achieved by creating a multi-agent model, containing retailer and consumer agents. The heterogeneous crowd of retailers employs either a uniform pricing strategy or a ‘local price flexing’ strategy. The actions of these retailers are chosen by predicting the profit of each action, using a perceptron. Following on from the consideration of different economic models, a discrete model was developed so that software agents have a discrete environment to operate within. Within the model, it has been observed how supermarkets with differing behaviors affect a heterogeneous crowd of consumer agents. The model was implemented in Java with Python used to evaluate the results. 

The simulation displays good acceptance with real grocery market behavior, i.e. captures the performance of British retailers thus can be used to determine the impact of changes in their behavior on their competitors and consumers.Furthermore it can be used to provide insight into sustainability of volatile pricing strategies, providing a useful insight in volatility of British supermarket retail industry. 
\end{abstract}
\acknowledgements{
I would like to express my sincere gratitude to Dr Maria Polukarov for her guidance and support which provided me the freedom to take this research in the direction of my interest.\\
\\
I would also like to thank my family and friends for their encouragement and support. To those who quietly listened to my software complaints. To those who worked throughout the nights with me. To those who helped me write what I couldn't say. I cannot thank you enough.
}

\declaration{
I, Stefan Collier, declare that this dissertation and the work presented in it are my own and has been generated by me as the result of my own original research.\\
I confirm that:\\
1. This work was done wholly or mainly while in candidature for a degree at this University;\\
2. Where any part of this dissertation has previously been submitted for any other qualification at this University or any other institution, this has been clearly stated;\\
3. Where I have consulted the published work of others, this is always clearly attributed;\\
4. Where I have quoted from the work of others, the source is always given. With the exception of such quotations, this dissertation is entirely my own work;\\
5. I have acknowledged all main sources of help;\\
6. Where the thesis is based on work done by myself jointly with others, I have made clear exactly what was done by others and what I have contributed myself;\\
7. Either none of this work has been published before submission, or parts of this work have been published by :\\
\\
Stefan Collier\\
April 2016
}
\tableofcontents
\listoffigures
\listoftables

\mainmatter
%% ----------------------------------------------------------------
%\include{Introduction}
%\include{Conclusions}
\include{chapters/1Project/main}
\include{chapters/2Lit/main}
\include{chapters/3Design/HighLevel}
\include{chapters/3Design/InDepth}
\include{chapters/4Impl/main}

\include{chapters/5Experiments/1/main}
\include{chapters/5Experiments/2/main}
\include{chapters/5Experiments/3/main}
\include{chapters/5Experiments/4/main}

\include{chapters/6Conclusion/main}

\appendix
\include{appendix/AppendixB}
\include{appendix/D/main}
\include{appendix/AppendixC}

\backmatter
\bibliographystyle{ecs}
\bibliography{ECS}
\end{document}
%% ----------------------------------------------------------------

\include{appendix/AppendixC}

\backmatter
\bibliographystyle{ecs}
\bibliography{ECS}
\end{document}
%% ----------------------------------------------------------------


 %% ----------------------------------------------------------------
%% Progress.tex
%% ---------------------------------------------------------------- 
\documentclass{ecsprogress}    % Use the progress Style
\graphicspath{{../figs/}}   % Location of your graphics files
    \usepackage{natbib}            % Use Natbib style for the refs.
\hypersetup{colorlinks=true}   % Set to false for black/white printing
\input{Definitions}            % Include your abbreviations



\usepackage{enumitem}% http://ctan.org/pkg/enumitem
\usepackage{multirow}
\usepackage{float}
\usepackage{amsmath}
\usepackage{multicol}
\usepackage{amssymb}
\usepackage[normalem]{ulem}
\useunder{\uline}{\ul}{}
\usepackage{wrapfig}


\usepackage[table,xcdraw]{xcolor}


%% ----------------------------------------------------------------
\begin{document}
\frontmatter
\title      {Heterogeneous Agent-based Model for Supermarket Competition}
\authors    {\texorpdfstring
             {\href{mailto:sc22g13@ecs.soton.ac.uk}{Stefan J. Collier}}
             {Stefan J. Collier}
            }
\addresses  {\groupname\\\deptname\\\univname}
\date       {\today}
\subject    {}
\keywords   {}
\supervisor {Dr. Maria Polukarov}
\examiner   {Professor Sheng Chen}

\maketitle
\begin{abstract}
This project aim was to model and analyse the effects of competitive pricing behaviors of grocery retailers on the British market. 

This was achieved by creating a multi-agent model, containing retailer and consumer agents. The heterogeneous crowd of retailers employs either a uniform pricing strategy or a ‘local price flexing’ strategy. The actions of these retailers are chosen by predicting the profit of each action, using a perceptron. Following on from the consideration of different economic models, a discrete model was developed so that software agents have a discrete environment to operate within. Within the model, it has been observed how supermarkets with differing behaviors affect a heterogeneous crowd of consumer agents. The model was implemented in Java with Python used to evaluate the results. 

The simulation displays good acceptance with real grocery market behavior, i.e. captures the performance of British retailers thus can be used to determine the impact of changes in their behavior on their competitors and consumers.Furthermore it can be used to provide insight into sustainability of volatile pricing strategies, providing a useful insight in volatility of British supermarket retail industry. 
\end{abstract}
\acknowledgements{
I would like to express my sincere gratitude to Dr Maria Polukarov for her guidance and support which provided me the freedom to take this research in the direction of my interest.\\
\\
I would also like to thank my family and friends for their encouragement and support. To those who quietly listened to my software complaints. To those who worked throughout the nights with me. To those who helped me write what I couldn't say. I cannot thank you enough.
}

\declaration{
I, Stefan Collier, declare that this dissertation and the work presented in it are my own and has been generated by me as the result of my own original research.\\
I confirm that:\\
1. This work was done wholly or mainly while in candidature for a degree at this University;\\
2. Where any part of this dissertation has previously been submitted for any other qualification at this University or any other institution, this has been clearly stated;\\
3. Where I have consulted the published work of others, this is always clearly attributed;\\
4. Where I have quoted from the work of others, the source is always given. With the exception of such quotations, this dissertation is entirely my own work;\\
5. I have acknowledged all main sources of help;\\
6. Where the thesis is based on work done by myself jointly with others, I have made clear exactly what was done by others and what I have contributed myself;\\
7. Either none of this work has been published before submission, or parts of this work have been published by :\\
\\
Stefan Collier\\
April 2016
}
\tableofcontents
\listoffigures
\listoftables

\mainmatter
%% ----------------------------------------------------------------
%\include{Introduction}
%\include{Conclusions}
 %% ----------------------------------------------------------------
%% Progress.tex
%% ---------------------------------------------------------------- 
\documentclass{ecsprogress}    % Use the progress Style
\graphicspath{{../figs/}}   % Location of your graphics files
    \usepackage{natbib}            % Use Natbib style for the refs.
\hypersetup{colorlinks=true}   % Set to false for black/white printing
\input{Definitions}            % Include your abbreviations



\usepackage{enumitem}% http://ctan.org/pkg/enumitem
\usepackage{multirow}
\usepackage{float}
\usepackage{amsmath}
\usepackage{multicol}
\usepackage{amssymb}
\usepackage[normalem]{ulem}
\useunder{\uline}{\ul}{}
\usepackage{wrapfig}


\usepackage[table,xcdraw]{xcolor}


%% ----------------------------------------------------------------
\begin{document}
\frontmatter
\title      {Heterogeneous Agent-based Model for Supermarket Competition}
\authors    {\texorpdfstring
             {\href{mailto:sc22g13@ecs.soton.ac.uk}{Stefan J. Collier}}
             {Stefan J. Collier}
            }
\addresses  {\groupname\\\deptname\\\univname}
\date       {\today}
\subject    {}
\keywords   {}
\supervisor {Dr. Maria Polukarov}
\examiner   {Professor Sheng Chen}

\maketitle
\begin{abstract}
This project aim was to model and analyse the effects of competitive pricing behaviors of grocery retailers on the British market. 

This was achieved by creating a multi-agent model, containing retailer and consumer agents. The heterogeneous crowd of retailers employs either a uniform pricing strategy or a ‘local price flexing’ strategy. The actions of these retailers are chosen by predicting the profit of each action, using a perceptron. Following on from the consideration of different economic models, a discrete model was developed so that software agents have a discrete environment to operate within. Within the model, it has been observed how supermarkets with differing behaviors affect a heterogeneous crowd of consumer agents. The model was implemented in Java with Python used to evaluate the results. 

The simulation displays good acceptance with real grocery market behavior, i.e. captures the performance of British retailers thus can be used to determine the impact of changes in their behavior on their competitors and consumers.Furthermore it can be used to provide insight into sustainability of volatile pricing strategies, providing a useful insight in volatility of British supermarket retail industry. 
\end{abstract}
\acknowledgements{
I would like to express my sincere gratitude to Dr Maria Polukarov for her guidance and support which provided me the freedom to take this research in the direction of my interest.\\
\\
I would also like to thank my family and friends for their encouragement and support. To those who quietly listened to my software complaints. To those who worked throughout the nights with me. To those who helped me write what I couldn't say. I cannot thank you enough.
}

\declaration{
I, Stefan Collier, declare that this dissertation and the work presented in it are my own and has been generated by me as the result of my own original research.\\
I confirm that:\\
1. This work was done wholly or mainly while in candidature for a degree at this University;\\
2. Where any part of this dissertation has previously been submitted for any other qualification at this University or any other institution, this has been clearly stated;\\
3. Where I have consulted the published work of others, this is always clearly attributed;\\
4. Where I have quoted from the work of others, the source is always given. With the exception of such quotations, this dissertation is entirely my own work;\\
5. I have acknowledged all main sources of help;\\
6. Where the thesis is based on work done by myself jointly with others, I have made clear exactly what was done by others and what I have contributed myself;\\
7. Either none of this work has been published before submission, or parts of this work have been published by :\\
\\
Stefan Collier\\
April 2016
}
\tableofcontents
\listoffigures
\listoftables

\mainmatter
%% ----------------------------------------------------------------
%\include{Introduction}
%\include{Conclusions}
\include{chapters/1Project/main}
\include{chapters/2Lit/main}
\include{chapters/3Design/HighLevel}
\include{chapters/3Design/InDepth}
\include{chapters/4Impl/main}

\include{chapters/5Experiments/1/main}
\include{chapters/5Experiments/2/main}
\include{chapters/5Experiments/3/main}
\include{chapters/5Experiments/4/main}

\include{chapters/6Conclusion/main}

\appendix
\include{appendix/AppendixB}
\include{appendix/D/main}
\include{appendix/AppendixC}

\backmatter
\bibliographystyle{ecs}
\bibliography{ECS}
\end{document}
%% ----------------------------------------------------------------

 %% ----------------------------------------------------------------
%% Progress.tex
%% ---------------------------------------------------------------- 
\documentclass{ecsprogress}    % Use the progress Style
\graphicspath{{../figs/}}   % Location of your graphics files
    \usepackage{natbib}            % Use Natbib style for the refs.
\hypersetup{colorlinks=true}   % Set to false for black/white printing
\input{Definitions}            % Include your abbreviations



\usepackage{enumitem}% http://ctan.org/pkg/enumitem
\usepackage{multirow}
\usepackage{float}
\usepackage{amsmath}
\usepackage{multicol}
\usepackage{amssymb}
\usepackage[normalem]{ulem}
\useunder{\uline}{\ul}{}
\usepackage{wrapfig}


\usepackage[table,xcdraw]{xcolor}


%% ----------------------------------------------------------------
\begin{document}
\frontmatter
\title      {Heterogeneous Agent-based Model for Supermarket Competition}
\authors    {\texorpdfstring
             {\href{mailto:sc22g13@ecs.soton.ac.uk}{Stefan J. Collier}}
             {Stefan J. Collier}
            }
\addresses  {\groupname\\\deptname\\\univname}
\date       {\today}
\subject    {}
\keywords   {}
\supervisor {Dr. Maria Polukarov}
\examiner   {Professor Sheng Chen}

\maketitle
\begin{abstract}
This project aim was to model and analyse the effects of competitive pricing behaviors of grocery retailers on the British market. 

This was achieved by creating a multi-agent model, containing retailer and consumer agents. The heterogeneous crowd of retailers employs either a uniform pricing strategy or a ‘local price flexing’ strategy. The actions of these retailers are chosen by predicting the profit of each action, using a perceptron. Following on from the consideration of different economic models, a discrete model was developed so that software agents have a discrete environment to operate within. Within the model, it has been observed how supermarkets with differing behaviors affect a heterogeneous crowd of consumer agents. The model was implemented in Java with Python used to evaluate the results. 

The simulation displays good acceptance with real grocery market behavior, i.e. captures the performance of British retailers thus can be used to determine the impact of changes in their behavior on their competitors and consumers.Furthermore it can be used to provide insight into sustainability of volatile pricing strategies, providing a useful insight in volatility of British supermarket retail industry. 
\end{abstract}
\acknowledgements{
I would like to express my sincere gratitude to Dr Maria Polukarov for her guidance and support which provided me the freedom to take this research in the direction of my interest.\\
\\
I would also like to thank my family and friends for their encouragement and support. To those who quietly listened to my software complaints. To those who worked throughout the nights with me. To those who helped me write what I couldn't say. I cannot thank you enough.
}

\declaration{
I, Stefan Collier, declare that this dissertation and the work presented in it are my own and has been generated by me as the result of my own original research.\\
I confirm that:\\
1. This work was done wholly or mainly while in candidature for a degree at this University;\\
2. Where any part of this dissertation has previously been submitted for any other qualification at this University or any other institution, this has been clearly stated;\\
3. Where I have consulted the published work of others, this is always clearly attributed;\\
4. Where I have quoted from the work of others, the source is always given. With the exception of such quotations, this dissertation is entirely my own work;\\
5. I have acknowledged all main sources of help;\\
6. Where the thesis is based on work done by myself jointly with others, I have made clear exactly what was done by others and what I have contributed myself;\\
7. Either none of this work has been published before submission, or parts of this work have been published by :\\
\\
Stefan Collier\\
April 2016
}
\tableofcontents
\listoffigures
\listoftables

\mainmatter
%% ----------------------------------------------------------------
%\include{Introduction}
%\include{Conclusions}
\include{chapters/1Project/main}
\include{chapters/2Lit/main}
\include{chapters/3Design/HighLevel}
\include{chapters/3Design/InDepth}
\include{chapters/4Impl/main}

\include{chapters/5Experiments/1/main}
\include{chapters/5Experiments/2/main}
\include{chapters/5Experiments/3/main}
\include{chapters/5Experiments/4/main}

\include{chapters/6Conclusion/main}

\appendix
\include{appendix/AppendixB}
\include{appendix/D/main}
\include{appendix/AppendixC}

\backmatter
\bibliographystyle{ecs}
\bibliography{ECS}
\end{document}
%% ----------------------------------------------------------------

\include{chapters/3Design/HighLevel}
\include{chapters/3Design/InDepth}
 %% ----------------------------------------------------------------
%% Progress.tex
%% ---------------------------------------------------------------- 
\documentclass{ecsprogress}    % Use the progress Style
\graphicspath{{../figs/}}   % Location of your graphics files
    \usepackage{natbib}            % Use Natbib style for the refs.
\hypersetup{colorlinks=true}   % Set to false for black/white printing
\input{Definitions}            % Include your abbreviations



\usepackage{enumitem}% http://ctan.org/pkg/enumitem
\usepackage{multirow}
\usepackage{float}
\usepackage{amsmath}
\usepackage{multicol}
\usepackage{amssymb}
\usepackage[normalem]{ulem}
\useunder{\uline}{\ul}{}
\usepackage{wrapfig}


\usepackage[table,xcdraw]{xcolor}


%% ----------------------------------------------------------------
\begin{document}
\frontmatter
\title      {Heterogeneous Agent-based Model for Supermarket Competition}
\authors    {\texorpdfstring
             {\href{mailto:sc22g13@ecs.soton.ac.uk}{Stefan J. Collier}}
             {Stefan J. Collier}
            }
\addresses  {\groupname\\\deptname\\\univname}
\date       {\today}
\subject    {}
\keywords   {}
\supervisor {Dr. Maria Polukarov}
\examiner   {Professor Sheng Chen}

\maketitle
\begin{abstract}
This project aim was to model and analyse the effects of competitive pricing behaviors of grocery retailers on the British market. 

This was achieved by creating a multi-agent model, containing retailer and consumer agents. The heterogeneous crowd of retailers employs either a uniform pricing strategy or a ‘local price flexing’ strategy. The actions of these retailers are chosen by predicting the profit of each action, using a perceptron. Following on from the consideration of different economic models, a discrete model was developed so that software agents have a discrete environment to operate within. Within the model, it has been observed how supermarkets with differing behaviors affect a heterogeneous crowd of consumer agents. The model was implemented in Java with Python used to evaluate the results. 

The simulation displays good acceptance with real grocery market behavior, i.e. captures the performance of British retailers thus can be used to determine the impact of changes in their behavior on their competitors and consumers.Furthermore it can be used to provide insight into sustainability of volatile pricing strategies, providing a useful insight in volatility of British supermarket retail industry. 
\end{abstract}
\acknowledgements{
I would like to express my sincere gratitude to Dr Maria Polukarov for her guidance and support which provided me the freedom to take this research in the direction of my interest.\\
\\
I would also like to thank my family and friends for their encouragement and support. To those who quietly listened to my software complaints. To those who worked throughout the nights with me. To those who helped me write what I couldn't say. I cannot thank you enough.
}

\declaration{
I, Stefan Collier, declare that this dissertation and the work presented in it are my own and has been generated by me as the result of my own original research.\\
I confirm that:\\
1. This work was done wholly or mainly while in candidature for a degree at this University;\\
2. Where any part of this dissertation has previously been submitted for any other qualification at this University or any other institution, this has been clearly stated;\\
3. Where I have consulted the published work of others, this is always clearly attributed;\\
4. Where I have quoted from the work of others, the source is always given. With the exception of such quotations, this dissertation is entirely my own work;\\
5. I have acknowledged all main sources of help;\\
6. Where the thesis is based on work done by myself jointly with others, I have made clear exactly what was done by others and what I have contributed myself;\\
7. Either none of this work has been published before submission, or parts of this work have been published by :\\
\\
Stefan Collier\\
April 2016
}
\tableofcontents
\listoffigures
\listoftables

\mainmatter
%% ----------------------------------------------------------------
%\include{Introduction}
%\include{Conclusions}
\include{chapters/1Project/main}
\include{chapters/2Lit/main}
\include{chapters/3Design/HighLevel}
\include{chapters/3Design/InDepth}
\include{chapters/4Impl/main}

\include{chapters/5Experiments/1/main}
\include{chapters/5Experiments/2/main}
\include{chapters/5Experiments/3/main}
\include{chapters/5Experiments/4/main}

\include{chapters/6Conclusion/main}

\appendix
\include{appendix/AppendixB}
\include{appendix/D/main}
\include{appendix/AppendixC}

\backmatter
\bibliographystyle{ecs}
\bibliography{ECS}
\end{document}
%% ----------------------------------------------------------------


 %% ----------------------------------------------------------------
%% Progress.tex
%% ---------------------------------------------------------------- 
\documentclass{ecsprogress}    % Use the progress Style
\graphicspath{{../figs/}}   % Location of your graphics files
    \usepackage{natbib}            % Use Natbib style for the refs.
\hypersetup{colorlinks=true}   % Set to false for black/white printing
\input{Definitions}            % Include your abbreviations



\usepackage{enumitem}% http://ctan.org/pkg/enumitem
\usepackage{multirow}
\usepackage{float}
\usepackage{amsmath}
\usepackage{multicol}
\usepackage{amssymb}
\usepackage[normalem]{ulem}
\useunder{\uline}{\ul}{}
\usepackage{wrapfig}


\usepackage[table,xcdraw]{xcolor}


%% ----------------------------------------------------------------
\begin{document}
\frontmatter
\title      {Heterogeneous Agent-based Model for Supermarket Competition}
\authors    {\texorpdfstring
             {\href{mailto:sc22g13@ecs.soton.ac.uk}{Stefan J. Collier}}
             {Stefan J. Collier}
            }
\addresses  {\groupname\\\deptname\\\univname}
\date       {\today}
\subject    {}
\keywords   {}
\supervisor {Dr. Maria Polukarov}
\examiner   {Professor Sheng Chen}

\maketitle
\begin{abstract}
This project aim was to model and analyse the effects of competitive pricing behaviors of grocery retailers on the British market. 

This was achieved by creating a multi-agent model, containing retailer and consumer agents. The heterogeneous crowd of retailers employs either a uniform pricing strategy or a ‘local price flexing’ strategy. The actions of these retailers are chosen by predicting the profit of each action, using a perceptron. Following on from the consideration of different economic models, a discrete model was developed so that software agents have a discrete environment to operate within. Within the model, it has been observed how supermarkets with differing behaviors affect a heterogeneous crowd of consumer agents. The model was implemented in Java with Python used to evaluate the results. 

The simulation displays good acceptance with real grocery market behavior, i.e. captures the performance of British retailers thus can be used to determine the impact of changes in their behavior on their competitors and consumers.Furthermore it can be used to provide insight into sustainability of volatile pricing strategies, providing a useful insight in volatility of British supermarket retail industry. 
\end{abstract}
\acknowledgements{
I would like to express my sincere gratitude to Dr Maria Polukarov for her guidance and support which provided me the freedom to take this research in the direction of my interest.\\
\\
I would also like to thank my family and friends for their encouragement and support. To those who quietly listened to my software complaints. To those who worked throughout the nights with me. To those who helped me write what I couldn't say. I cannot thank you enough.
}

\declaration{
I, Stefan Collier, declare that this dissertation and the work presented in it are my own and has been generated by me as the result of my own original research.\\
I confirm that:\\
1. This work was done wholly or mainly while in candidature for a degree at this University;\\
2. Where any part of this dissertation has previously been submitted for any other qualification at this University or any other institution, this has been clearly stated;\\
3. Where I have consulted the published work of others, this is always clearly attributed;\\
4. Where I have quoted from the work of others, the source is always given. With the exception of such quotations, this dissertation is entirely my own work;\\
5. I have acknowledged all main sources of help;\\
6. Where the thesis is based on work done by myself jointly with others, I have made clear exactly what was done by others and what I have contributed myself;\\
7. Either none of this work has been published before submission, or parts of this work have been published by :\\
\\
Stefan Collier\\
April 2016
}
\tableofcontents
\listoffigures
\listoftables

\mainmatter
%% ----------------------------------------------------------------
%\include{Introduction}
%\include{Conclusions}
\include{chapters/1Project/main}
\include{chapters/2Lit/main}
\include{chapters/3Design/HighLevel}
\include{chapters/3Design/InDepth}
\include{chapters/4Impl/main}

\include{chapters/5Experiments/1/main}
\include{chapters/5Experiments/2/main}
\include{chapters/5Experiments/3/main}
\include{chapters/5Experiments/4/main}

\include{chapters/6Conclusion/main}

\appendix
\include{appendix/AppendixB}
\include{appendix/D/main}
\include{appendix/AppendixC}

\backmatter
\bibliographystyle{ecs}
\bibliography{ECS}
\end{document}
%% ----------------------------------------------------------------

 %% ----------------------------------------------------------------
%% Progress.tex
%% ---------------------------------------------------------------- 
\documentclass{ecsprogress}    % Use the progress Style
\graphicspath{{../figs/}}   % Location of your graphics files
    \usepackage{natbib}            % Use Natbib style for the refs.
\hypersetup{colorlinks=true}   % Set to false for black/white printing
\input{Definitions}            % Include your abbreviations



\usepackage{enumitem}% http://ctan.org/pkg/enumitem
\usepackage{multirow}
\usepackage{float}
\usepackage{amsmath}
\usepackage{multicol}
\usepackage{amssymb}
\usepackage[normalem]{ulem}
\useunder{\uline}{\ul}{}
\usepackage{wrapfig}


\usepackage[table,xcdraw]{xcolor}


%% ----------------------------------------------------------------
\begin{document}
\frontmatter
\title      {Heterogeneous Agent-based Model for Supermarket Competition}
\authors    {\texorpdfstring
             {\href{mailto:sc22g13@ecs.soton.ac.uk}{Stefan J. Collier}}
             {Stefan J. Collier}
            }
\addresses  {\groupname\\\deptname\\\univname}
\date       {\today}
\subject    {}
\keywords   {}
\supervisor {Dr. Maria Polukarov}
\examiner   {Professor Sheng Chen}

\maketitle
\begin{abstract}
This project aim was to model and analyse the effects of competitive pricing behaviors of grocery retailers on the British market. 

This was achieved by creating a multi-agent model, containing retailer and consumer agents. The heterogeneous crowd of retailers employs either a uniform pricing strategy or a ‘local price flexing’ strategy. The actions of these retailers are chosen by predicting the profit of each action, using a perceptron. Following on from the consideration of different economic models, a discrete model was developed so that software agents have a discrete environment to operate within. Within the model, it has been observed how supermarkets with differing behaviors affect a heterogeneous crowd of consumer agents. The model was implemented in Java with Python used to evaluate the results. 

The simulation displays good acceptance with real grocery market behavior, i.e. captures the performance of British retailers thus can be used to determine the impact of changes in their behavior on their competitors and consumers.Furthermore it can be used to provide insight into sustainability of volatile pricing strategies, providing a useful insight in volatility of British supermarket retail industry. 
\end{abstract}
\acknowledgements{
I would like to express my sincere gratitude to Dr Maria Polukarov for her guidance and support which provided me the freedom to take this research in the direction of my interest.\\
\\
I would also like to thank my family and friends for their encouragement and support. To those who quietly listened to my software complaints. To those who worked throughout the nights with me. To those who helped me write what I couldn't say. I cannot thank you enough.
}

\declaration{
I, Stefan Collier, declare that this dissertation and the work presented in it are my own and has been generated by me as the result of my own original research.\\
I confirm that:\\
1. This work was done wholly or mainly while in candidature for a degree at this University;\\
2. Where any part of this dissertation has previously been submitted for any other qualification at this University or any other institution, this has been clearly stated;\\
3. Where I have consulted the published work of others, this is always clearly attributed;\\
4. Where I have quoted from the work of others, the source is always given. With the exception of such quotations, this dissertation is entirely my own work;\\
5. I have acknowledged all main sources of help;\\
6. Where the thesis is based on work done by myself jointly with others, I have made clear exactly what was done by others and what I have contributed myself;\\
7. Either none of this work has been published before submission, or parts of this work have been published by :\\
\\
Stefan Collier\\
April 2016
}
\tableofcontents
\listoffigures
\listoftables

\mainmatter
%% ----------------------------------------------------------------
%\include{Introduction}
%\include{Conclusions}
\include{chapters/1Project/main}
\include{chapters/2Lit/main}
\include{chapters/3Design/HighLevel}
\include{chapters/3Design/InDepth}
\include{chapters/4Impl/main}

\include{chapters/5Experiments/1/main}
\include{chapters/5Experiments/2/main}
\include{chapters/5Experiments/3/main}
\include{chapters/5Experiments/4/main}

\include{chapters/6Conclusion/main}

\appendix
\include{appendix/AppendixB}
\include{appendix/D/main}
\include{appendix/AppendixC}

\backmatter
\bibliographystyle{ecs}
\bibliography{ECS}
\end{document}
%% ----------------------------------------------------------------

 %% ----------------------------------------------------------------
%% Progress.tex
%% ---------------------------------------------------------------- 
\documentclass{ecsprogress}    % Use the progress Style
\graphicspath{{../figs/}}   % Location of your graphics files
    \usepackage{natbib}            % Use Natbib style for the refs.
\hypersetup{colorlinks=true}   % Set to false for black/white printing
\input{Definitions}            % Include your abbreviations



\usepackage{enumitem}% http://ctan.org/pkg/enumitem
\usepackage{multirow}
\usepackage{float}
\usepackage{amsmath}
\usepackage{multicol}
\usepackage{amssymb}
\usepackage[normalem]{ulem}
\useunder{\uline}{\ul}{}
\usepackage{wrapfig}


\usepackage[table,xcdraw]{xcolor}


%% ----------------------------------------------------------------
\begin{document}
\frontmatter
\title      {Heterogeneous Agent-based Model for Supermarket Competition}
\authors    {\texorpdfstring
             {\href{mailto:sc22g13@ecs.soton.ac.uk}{Stefan J. Collier}}
             {Stefan J. Collier}
            }
\addresses  {\groupname\\\deptname\\\univname}
\date       {\today}
\subject    {}
\keywords   {}
\supervisor {Dr. Maria Polukarov}
\examiner   {Professor Sheng Chen}

\maketitle
\begin{abstract}
This project aim was to model and analyse the effects of competitive pricing behaviors of grocery retailers on the British market. 

This was achieved by creating a multi-agent model, containing retailer and consumer agents. The heterogeneous crowd of retailers employs either a uniform pricing strategy or a ‘local price flexing’ strategy. The actions of these retailers are chosen by predicting the profit of each action, using a perceptron. Following on from the consideration of different economic models, a discrete model was developed so that software agents have a discrete environment to operate within. Within the model, it has been observed how supermarkets with differing behaviors affect a heterogeneous crowd of consumer agents. The model was implemented in Java with Python used to evaluate the results. 

The simulation displays good acceptance with real grocery market behavior, i.e. captures the performance of British retailers thus can be used to determine the impact of changes in their behavior on their competitors and consumers.Furthermore it can be used to provide insight into sustainability of volatile pricing strategies, providing a useful insight in volatility of British supermarket retail industry. 
\end{abstract}
\acknowledgements{
I would like to express my sincere gratitude to Dr Maria Polukarov for her guidance and support which provided me the freedom to take this research in the direction of my interest.\\
\\
I would also like to thank my family and friends for their encouragement and support. To those who quietly listened to my software complaints. To those who worked throughout the nights with me. To those who helped me write what I couldn't say. I cannot thank you enough.
}

\declaration{
I, Stefan Collier, declare that this dissertation and the work presented in it are my own and has been generated by me as the result of my own original research.\\
I confirm that:\\
1. This work was done wholly or mainly while in candidature for a degree at this University;\\
2. Where any part of this dissertation has previously been submitted for any other qualification at this University or any other institution, this has been clearly stated;\\
3. Where I have consulted the published work of others, this is always clearly attributed;\\
4. Where I have quoted from the work of others, the source is always given. With the exception of such quotations, this dissertation is entirely my own work;\\
5. I have acknowledged all main sources of help;\\
6. Where the thesis is based on work done by myself jointly with others, I have made clear exactly what was done by others and what I have contributed myself;\\
7. Either none of this work has been published before submission, or parts of this work have been published by :\\
\\
Stefan Collier\\
April 2016
}
\tableofcontents
\listoffigures
\listoftables

\mainmatter
%% ----------------------------------------------------------------
%\include{Introduction}
%\include{Conclusions}
\include{chapters/1Project/main}
\include{chapters/2Lit/main}
\include{chapters/3Design/HighLevel}
\include{chapters/3Design/InDepth}
\include{chapters/4Impl/main}

\include{chapters/5Experiments/1/main}
\include{chapters/5Experiments/2/main}
\include{chapters/5Experiments/3/main}
\include{chapters/5Experiments/4/main}

\include{chapters/6Conclusion/main}

\appendix
\include{appendix/AppendixB}
\include{appendix/D/main}
\include{appendix/AppendixC}

\backmatter
\bibliographystyle{ecs}
\bibliography{ECS}
\end{document}
%% ----------------------------------------------------------------

 %% ----------------------------------------------------------------
%% Progress.tex
%% ---------------------------------------------------------------- 
\documentclass{ecsprogress}    % Use the progress Style
\graphicspath{{../figs/}}   % Location of your graphics files
    \usepackage{natbib}            % Use Natbib style for the refs.
\hypersetup{colorlinks=true}   % Set to false for black/white printing
\input{Definitions}            % Include your abbreviations



\usepackage{enumitem}% http://ctan.org/pkg/enumitem
\usepackage{multirow}
\usepackage{float}
\usepackage{amsmath}
\usepackage{multicol}
\usepackage{amssymb}
\usepackage[normalem]{ulem}
\useunder{\uline}{\ul}{}
\usepackage{wrapfig}


\usepackage[table,xcdraw]{xcolor}


%% ----------------------------------------------------------------
\begin{document}
\frontmatter
\title      {Heterogeneous Agent-based Model for Supermarket Competition}
\authors    {\texorpdfstring
             {\href{mailto:sc22g13@ecs.soton.ac.uk}{Stefan J. Collier}}
             {Stefan J. Collier}
            }
\addresses  {\groupname\\\deptname\\\univname}
\date       {\today}
\subject    {}
\keywords   {}
\supervisor {Dr. Maria Polukarov}
\examiner   {Professor Sheng Chen}

\maketitle
\begin{abstract}
This project aim was to model and analyse the effects of competitive pricing behaviors of grocery retailers on the British market. 

This was achieved by creating a multi-agent model, containing retailer and consumer agents. The heterogeneous crowd of retailers employs either a uniform pricing strategy or a ‘local price flexing’ strategy. The actions of these retailers are chosen by predicting the profit of each action, using a perceptron. Following on from the consideration of different economic models, a discrete model was developed so that software agents have a discrete environment to operate within. Within the model, it has been observed how supermarkets with differing behaviors affect a heterogeneous crowd of consumer agents. The model was implemented in Java with Python used to evaluate the results. 

The simulation displays good acceptance with real grocery market behavior, i.e. captures the performance of British retailers thus can be used to determine the impact of changes in their behavior on their competitors and consumers.Furthermore it can be used to provide insight into sustainability of volatile pricing strategies, providing a useful insight in volatility of British supermarket retail industry. 
\end{abstract}
\acknowledgements{
I would like to express my sincere gratitude to Dr Maria Polukarov for her guidance and support which provided me the freedom to take this research in the direction of my interest.\\
\\
I would also like to thank my family and friends for their encouragement and support. To those who quietly listened to my software complaints. To those who worked throughout the nights with me. To those who helped me write what I couldn't say. I cannot thank you enough.
}

\declaration{
I, Stefan Collier, declare that this dissertation and the work presented in it are my own and has been generated by me as the result of my own original research.\\
I confirm that:\\
1. This work was done wholly or mainly while in candidature for a degree at this University;\\
2. Where any part of this dissertation has previously been submitted for any other qualification at this University or any other institution, this has been clearly stated;\\
3. Where I have consulted the published work of others, this is always clearly attributed;\\
4. Where I have quoted from the work of others, the source is always given. With the exception of such quotations, this dissertation is entirely my own work;\\
5. I have acknowledged all main sources of help;\\
6. Where the thesis is based on work done by myself jointly with others, I have made clear exactly what was done by others and what I have contributed myself;\\
7. Either none of this work has been published before submission, or parts of this work have been published by :\\
\\
Stefan Collier\\
April 2016
}
\tableofcontents
\listoffigures
\listoftables

\mainmatter
%% ----------------------------------------------------------------
%\include{Introduction}
%\include{Conclusions}
\include{chapters/1Project/main}
\include{chapters/2Lit/main}
\include{chapters/3Design/HighLevel}
\include{chapters/3Design/InDepth}
\include{chapters/4Impl/main}

\include{chapters/5Experiments/1/main}
\include{chapters/5Experiments/2/main}
\include{chapters/5Experiments/3/main}
\include{chapters/5Experiments/4/main}

\include{chapters/6Conclusion/main}

\appendix
\include{appendix/AppendixB}
\include{appendix/D/main}
\include{appendix/AppendixC}

\backmatter
\bibliographystyle{ecs}
\bibliography{ECS}
\end{document}
%% ----------------------------------------------------------------


 %% ----------------------------------------------------------------
%% Progress.tex
%% ---------------------------------------------------------------- 
\documentclass{ecsprogress}    % Use the progress Style
\graphicspath{{../figs/}}   % Location of your graphics files
    \usepackage{natbib}            % Use Natbib style for the refs.
\hypersetup{colorlinks=true}   % Set to false for black/white printing
\input{Definitions}            % Include your abbreviations



\usepackage{enumitem}% http://ctan.org/pkg/enumitem
\usepackage{multirow}
\usepackage{float}
\usepackage{amsmath}
\usepackage{multicol}
\usepackage{amssymb}
\usepackage[normalem]{ulem}
\useunder{\uline}{\ul}{}
\usepackage{wrapfig}


\usepackage[table,xcdraw]{xcolor}


%% ----------------------------------------------------------------
\begin{document}
\frontmatter
\title      {Heterogeneous Agent-based Model for Supermarket Competition}
\authors    {\texorpdfstring
             {\href{mailto:sc22g13@ecs.soton.ac.uk}{Stefan J. Collier}}
             {Stefan J. Collier}
            }
\addresses  {\groupname\\\deptname\\\univname}
\date       {\today}
\subject    {}
\keywords   {}
\supervisor {Dr. Maria Polukarov}
\examiner   {Professor Sheng Chen}

\maketitle
\begin{abstract}
This project aim was to model and analyse the effects of competitive pricing behaviors of grocery retailers on the British market. 

This was achieved by creating a multi-agent model, containing retailer and consumer agents. The heterogeneous crowd of retailers employs either a uniform pricing strategy or a ‘local price flexing’ strategy. The actions of these retailers are chosen by predicting the profit of each action, using a perceptron. Following on from the consideration of different economic models, a discrete model was developed so that software agents have a discrete environment to operate within. Within the model, it has been observed how supermarkets with differing behaviors affect a heterogeneous crowd of consumer agents. The model was implemented in Java with Python used to evaluate the results. 

The simulation displays good acceptance with real grocery market behavior, i.e. captures the performance of British retailers thus can be used to determine the impact of changes in their behavior on their competitors and consumers.Furthermore it can be used to provide insight into sustainability of volatile pricing strategies, providing a useful insight in volatility of British supermarket retail industry. 
\end{abstract}
\acknowledgements{
I would like to express my sincere gratitude to Dr Maria Polukarov for her guidance and support which provided me the freedom to take this research in the direction of my interest.\\
\\
I would also like to thank my family and friends for their encouragement and support. To those who quietly listened to my software complaints. To those who worked throughout the nights with me. To those who helped me write what I couldn't say. I cannot thank you enough.
}

\declaration{
I, Stefan Collier, declare that this dissertation and the work presented in it are my own and has been generated by me as the result of my own original research.\\
I confirm that:\\
1. This work was done wholly or mainly while in candidature for a degree at this University;\\
2. Where any part of this dissertation has previously been submitted for any other qualification at this University or any other institution, this has been clearly stated;\\
3. Where I have consulted the published work of others, this is always clearly attributed;\\
4. Where I have quoted from the work of others, the source is always given. With the exception of such quotations, this dissertation is entirely my own work;\\
5. I have acknowledged all main sources of help;\\
6. Where the thesis is based on work done by myself jointly with others, I have made clear exactly what was done by others and what I have contributed myself;\\
7. Either none of this work has been published before submission, or parts of this work have been published by :\\
\\
Stefan Collier\\
April 2016
}
\tableofcontents
\listoffigures
\listoftables

\mainmatter
%% ----------------------------------------------------------------
%\include{Introduction}
%\include{Conclusions}
\include{chapters/1Project/main}
\include{chapters/2Lit/main}
\include{chapters/3Design/HighLevel}
\include{chapters/3Design/InDepth}
\include{chapters/4Impl/main}

\include{chapters/5Experiments/1/main}
\include{chapters/5Experiments/2/main}
\include{chapters/5Experiments/3/main}
\include{chapters/5Experiments/4/main}

\include{chapters/6Conclusion/main}

\appendix
\include{appendix/AppendixB}
\include{appendix/D/main}
\include{appendix/AppendixC}

\backmatter
\bibliographystyle{ecs}
\bibliography{ECS}
\end{document}
%% ----------------------------------------------------------------


\appendix
\include{appendix/AppendixB}
 %% ----------------------------------------------------------------
%% Progress.tex
%% ---------------------------------------------------------------- 
\documentclass{ecsprogress}    % Use the progress Style
\graphicspath{{../figs/}}   % Location of your graphics files
    \usepackage{natbib}            % Use Natbib style for the refs.
\hypersetup{colorlinks=true}   % Set to false for black/white printing
\input{Definitions}            % Include your abbreviations



\usepackage{enumitem}% http://ctan.org/pkg/enumitem
\usepackage{multirow}
\usepackage{float}
\usepackage{amsmath}
\usepackage{multicol}
\usepackage{amssymb}
\usepackage[normalem]{ulem}
\useunder{\uline}{\ul}{}
\usepackage{wrapfig}


\usepackage[table,xcdraw]{xcolor}


%% ----------------------------------------------------------------
\begin{document}
\frontmatter
\title      {Heterogeneous Agent-based Model for Supermarket Competition}
\authors    {\texorpdfstring
             {\href{mailto:sc22g13@ecs.soton.ac.uk}{Stefan J. Collier}}
             {Stefan J. Collier}
            }
\addresses  {\groupname\\\deptname\\\univname}
\date       {\today}
\subject    {}
\keywords   {}
\supervisor {Dr. Maria Polukarov}
\examiner   {Professor Sheng Chen}

\maketitle
\begin{abstract}
This project aim was to model and analyse the effects of competitive pricing behaviors of grocery retailers on the British market. 

This was achieved by creating a multi-agent model, containing retailer and consumer agents. The heterogeneous crowd of retailers employs either a uniform pricing strategy or a ‘local price flexing’ strategy. The actions of these retailers are chosen by predicting the profit of each action, using a perceptron. Following on from the consideration of different economic models, a discrete model was developed so that software agents have a discrete environment to operate within. Within the model, it has been observed how supermarkets with differing behaviors affect a heterogeneous crowd of consumer agents. The model was implemented in Java with Python used to evaluate the results. 

The simulation displays good acceptance with real grocery market behavior, i.e. captures the performance of British retailers thus can be used to determine the impact of changes in their behavior on their competitors and consumers.Furthermore it can be used to provide insight into sustainability of volatile pricing strategies, providing a useful insight in volatility of British supermarket retail industry. 
\end{abstract}
\acknowledgements{
I would like to express my sincere gratitude to Dr Maria Polukarov for her guidance and support which provided me the freedom to take this research in the direction of my interest.\\
\\
I would also like to thank my family and friends for their encouragement and support. To those who quietly listened to my software complaints. To those who worked throughout the nights with me. To those who helped me write what I couldn't say. I cannot thank you enough.
}

\declaration{
I, Stefan Collier, declare that this dissertation and the work presented in it are my own and has been generated by me as the result of my own original research.\\
I confirm that:\\
1. This work was done wholly or mainly while in candidature for a degree at this University;\\
2. Where any part of this dissertation has previously been submitted for any other qualification at this University or any other institution, this has been clearly stated;\\
3. Where I have consulted the published work of others, this is always clearly attributed;\\
4. Where I have quoted from the work of others, the source is always given. With the exception of such quotations, this dissertation is entirely my own work;\\
5. I have acknowledged all main sources of help;\\
6. Where the thesis is based on work done by myself jointly with others, I have made clear exactly what was done by others and what I have contributed myself;\\
7. Either none of this work has been published before submission, or parts of this work have been published by :\\
\\
Stefan Collier\\
April 2016
}
\tableofcontents
\listoffigures
\listoftables

\mainmatter
%% ----------------------------------------------------------------
%\include{Introduction}
%\include{Conclusions}
\include{chapters/1Project/main}
\include{chapters/2Lit/main}
\include{chapters/3Design/HighLevel}
\include{chapters/3Design/InDepth}
\include{chapters/4Impl/main}

\include{chapters/5Experiments/1/main}
\include{chapters/5Experiments/2/main}
\include{chapters/5Experiments/3/main}
\include{chapters/5Experiments/4/main}

\include{chapters/6Conclusion/main}

\appendix
\include{appendix/AppendixB}
\include{appendix/D/main}
\include{appendix/AppendixC}

\backmatter
\bibliographystyle{ecs}
\bibliography{ECS}
\end{document}
%% ----------------------------------------------------------------

\include{appendix/AppendixC}

\backmatter
\bibliographystyle{ecs}
\bibliography{ECS}
\end{document}
%% ----------------------------------------------------------------


\appendix
\include{appendix/AppendixB}
 %% ----------------------------------------------------------------
%% Progress.tex
%% ---------------------------------------------------------------- 
\documentclass{ecsprogress}    % Use the progress Style
\graphicspath{{../figs/}}   % Location of your graphics files
    \usepackage{natbib}            % Use Natbib style for the refs.
\hypersetup{colorlinks=true}   % Set to false for black/white printing
\input{Definitions}            % Include your abbreviations



\usepackage{enumitem}% http://ctan.org/pkg/enumitem
\usepackage{multirow}
\usepackage{float}
\usepackage{amsmath}
\usepackage{multicol}
\usepackage{amssymb}
\usepackage[normalem]{ulem}
\useunder{\uline}{\ul}{}
\usepackage{wrapfig}


\usepackage[table,xcdraw]{xcolor}


%% ----------------------------------------------------------------
\begin{document}
\frontmatter
\title      {Heterogeneous Agent-based Model for Supermarket Competition}
\authors    {\texorpdfstring
             {\href{mailto:sc22g13@ecs.soton.ac.uk}{Stefan J. Collier}}
             {Stefan J. Collier}
            }
\addresses  {\groupname\\\deptname\\\univname}
\date       {\today}
\subject    {}
\keywords   {}
\supervisor {Dr. Maria Polukarov}
\examiner   {Professor Sheng Chen}

\maketitle
\begin{abstract}
This project aim was to model and analyse the effects of competitive pricing behaviors of grocery retailers on the British market. 

This was achieved by creating a multi-agent model, containing retailer and consumer agents. The heterogeneous crowd of retailers employs either a uniform pricing strategy or a ‘local price flexing’ strategy. The actions of these retailers are chosen by predicting the profit of each action, using a perceptron. Following on from the consideration of different economic models, a discrete model was developed so that software agents have a discrete environment to operate within. Within the model, it has been observed how supermarkets with differing behaviors affect a heterogeneous crowd of consumer agents. The model was implemented in Java with Python used to evaluate the results. 

The simulation displays good acceptance with real grocery market behavior, i.e. captures the performance of British retailers thus can be used to determine the impact of changes in their behavior on their competitors and consumers.Furthermore it can be used to provide insight into sustainability of volatile pricing strategies, providing a useful insight in volatility of British supermarket retail industry. 
\end{abstract}
\acknowledgements{
I would like to express my sincere gratitude to Dr Maria Polukarov for her guidance and support which provided me the freedom to take this research in the direction of my interest.\\
\\
I would also like to thank my family and friends for their encouragement and support. To those who quietly listened to my software complaints. To those who worked throughout the nights with me. To those who helped me write what I couldn't say. I cannot thank you enough.
}

\declaration{
I, Stefan Collier, declare that this dissertation and the work presented in it are my own and has been generated by me as the result of my own original research.\\
I confirm that:\\
1. This work was done wholly or mainly while in candidature for a degree at this University;\\
2. Where any part of this dissertation has previously been submitted for any other qualification at this University or any other institution, this has been clearly stated;\\
3. Where I have consulted the published work of others, this is always clearly attributed;\\
4. Where I have quoted from the work of others, the source is always given. With the exception of such quotations, this dissertation is entirely my own work;\\
5. I have acknowledged all main sources of help;\\
6. Where the thesis is based on work done by myself jointly with others, I have made clear exactly what was done by others and what I have contributed myself;\\
7. Either none of this work has been published before submission, or parts of this work have been published by :\\
\\
Stefan Collier\\
April 2016
}
\tableofcontents
\listoffigures
\listoftables

\mainmatter
%% ----------------------------------------------------------------
%\include{Introduction}
%\include{Conclusions}
 %% ----------------------------------------------------------------
%% Progress.tex
%% ---------------------------------------------------------------- 
\documentclass{ecsprogress}    % Use the progress Style
\graphicspath{{../figs/}}   % Location of your graphics files
    \usepackage{natbib}            % Use Natbib style for the refs.
\hypersetup{colorlinks=true}   % Set to false for black/white printing
\input{Definitions}            % Include your abbreviations



\usepackage{enumitem}% http://ctan.org/pkg/enumitem
\usepackage{multirow}
\usepackage{float}
\usepackage{amsmath}
\usepackage{multicol}
\usepackage{amssymb}
\usepackage[normalem]{ulem}
\useunder{\uline}{\ul}{}
\usepackage{wrapfig}


\usepackage[table,xcdraw]{xcolor}


%% ----------------------------------------------------------------
\begin{document}
\frontmatter
\title      {Heterogeneous Agent-based Model for Supermarket Competition}
\authors    {\texorpdfstring
             {\href{mailto:sc22g13@ecs.soton.ac.uk}{Stefan J. Collier}}
             {Stefan J. Collier}
            }
\addresses  {\groupname\\\deptname\\\univname}
\date       {\today}
\subject    {}
\keywords   {}
\supervisor {Dr. Maria Polukarov}
\examiner   {Professor Sheng Chen}

\maketitle
\begin{abstract}
This project aim was to model and analyse the effects of competitive pricing behaviors of grocery retailers on the British market. 

This was achieved by creating a multi-agent model, containing retailer and consumer agents. The heterogeneous crowd of retailers employs either a uniform pricing strategy or a ‘local price flexing’ strategy. The actions of these retailers are chosen by predicting the profit of each action, using a perceptron. Following on from the consideration of different economic models, a discrete model was developed so that software agents have a discrete environment to operate within. Within the model, it has been observed how supermarkets with differing behaviors affect a heterogeneous crowd of consumer agents. The model was implemented in Java with Python used to evaluate the results. 

The simulation displays good acceptance with real grocery market behavior, i.e. captures the performance of British retailers thus can be used to determine the impact of changes in their behavior on their competitors and consumers.Furthermore it can be used to provide insight into sustainability of volatile pricing strategies, providing a useful insight in volatility of British supermarket retail industry. 
\end{abstract}
\acknowledgements{
I would like to express my sincere gratitude to Dr Maria Polukarov for her guidance and support which provided me the freedom to take this research in the direction of my interest.\\
\\
I would also like to thank my family and friends for their encouragement and support. To those who quietly listened to my software complaints. To those who worked throughout the nights with me. To those who helped me write what I couldn't say. I cannot thank you enough.
}

\declaration{
I, Stefan Collier, declare that this dissertation and the work presented in it are my own and has been generated by me as the result of my own original research.\\
I confirm that:\\
1. This work was done wholly or mainly while in candidature for a degree at this University;\\
2. Where any part of this dissertation has previously been submitted for any other qualification at this University or any other institution, this has been clearly stated;\\
3. Where I have consulted the published work of others, this is always clearly attributed;\\
4. Where I have quoted from the work of others, the source is always given. With the exception of such quotations, this dissertation is entirely my own work;\\
5. I have acknowledged all main sources of help;\\
6. Where the thesis is based on work done by myself jointly with others, I have made clear exactly what was done by others and what I have contributed myself;\\
7. Either none of this work has been published before submission, or parts of this work have been published by :\\
\\
Stefan Collier\\
April 2016
}
\tableofcontents
\listoffigures
\listoftables

\mainmatter
%% ----------------------------------------------------------------
%\include{Introduction}
%\include{Conclusions}
\include{chapters/1Project/main}
\include{chapters/2Lit/main}
\include{chapters/3Design/HighLevel}
\include{chapters/3Design/InDepth}
\include{chapters/4Impl/main}

\include{chapters/5Experiments/1/main}
\include{chapters/5Experiments/2/main}
\include{chapters/5Experiments/3/main}
\include{chapters/5Experiments/4/main}

\include{chapters/6Conclusion/main}

\appendix
\include{appendix/AppendixB}
\include{appendix/D/main}
\include{appendix/AppendixC}

\backmatter
\bibliographystyle{ecs}
\bibliography{ECS}
\end{document}
%% ----------------------------------------------------------------

 %% ----------------------------------------------------------------
%% Progress.tex
%% ---------------------------------------------------------------- 
\documentclass{ecsprogress}    % Use the progress Style
\graphicspath{{../figs/}}   % Location of your graphics files
    \usepackage{natbib}            % Use Natbib style for the refs.
\hypersetup{colorlinks=true}   % Set to false for black/white printing
\input{Definitions}            % Include your abbreviations



\usepackage{enumitem}% http://ctan.org/pkg/enumitem
\usepackage{multirow}
\usepackage{float}
\usepackage{amsmath}
\usepackage{multicol}
\usepackage{amssymb}
\usepackage[normalem]{ulem}
\useunder{\uline}{\ul}{}
\usepackage{wrapfig}


\usepackage[table,xcdraw]{xcolor}


%% ----------------------------------------------------------------
\begin{document}
\frontmatter
\title      {Heterogeneous Agent-based Model for Supermarket Competition}
\authors    {\texorpdfstring
             {\href{mailto:sc22g13@ecs.soton.ac.uk}{Stefan J. Collier}}
             {Stefan J. Collier}
            }
\addresses  {\groupname\\\deptname\\\univname}
\date       {\today}
\subject    {}
\keywords   {}
\supervisor {Dr. Maria Polukarov}
\examiner   {Professor Sheng Chen}

\maketitle
\begin{abstract}
This project aim was to model and analyse the effects of competitive pricing behaviors of grocery retailers on the British market. 

This was achieved by creating a multi-agent model, containing retailer and consumer agents. The heterogeneous crowd of retailers employs either a uniform pricing strategy or a ‘local price flexing’ strategy. The actions of these retailers are chosen by predicting the profit of each action, using a perceptron. Following on from the consideration of different economic models, a discrete model was developed so that software agents have a discrete environment to operate within. Within the model, it has been observed how supermarkets with differing behaviors affect a heterogeneous crowd of consumer agents. The model was implemented in Java with Python used to evaluate the results. 

The simulation displays good acceptance with real grocery market behavior, i.e. captures the performance of British retailers thus can be used to determine the impact of changes in their behavior on their competitors and consumers.Furthermore it can be used to provide insight into sustainability of volatile pricing strategies, providing a useful insight in volatility of British supermarket retail industry. 
\end{abstract}
\acknowledgements{
I would like to express my sincere gratitude to Dr Maria Polukarov for her guidance and support which provided me the freedom to take this research in the direction of my interest.\\
\\
I would also like to thank my family and friends for their encouragement and support. To those who quietly listened to my software complaints. To those who worked throughout the nights with me. To those who helped me write what I couldn't say. I cannot thank you enough.
}

\declaration{
I, Stefan Collier, declare that this dissertation and the work presented in it are my own and has been generated by me as the result of my own original research.\\
I confirm that:\\
1. This work was done wholly or mainly while in candidature for a degree at this University;\\
2. Where any part of this dissertation has previously been submitted for any other qualification at this University or any other institution, this has been clearly stated;\\
3. Where I have consulted the published work of others, this is always clearly attributed;\\
4. Where I have quoted from the work of others, the source is always given. With the exception of such quotations, this dissertation is entirely my own work;\\
5. I have acknowledged all main sources of help;\\
6. Where the thesis is based on work done by myself jointly with others, I have made clear exactly what was done by others and what I have contributed myself;\\
7. Either none of this work has been published before submission, or parts of this work have been published by :\\
\\
Stefan Collier\\
April 2016
}
\tableofcontents
\listoffigures
\listoftables

\mainmatter
%% ----------------------------------------------------------------
%\include{Introduction}
%\include{Conclusions}
\include{chapters/1Project/main}
\include{chapters/2Lit/main}
\include{chapters/3Design/HighLevel}
\include{chapters/3Design/InDepth}
\include{chapters/4Impl/main}

\include{chapters/5Experiments/1/main}
\include{chapters/5Experiments/2/main}
\include{chapters/5Experiments/3/main}
\include{chapters/5Experiments/4/main}

\include{chapters/6Conclusion/main}

\appendix
\include{appendix/AppendixB}
\include{appendix/D/main}
\include{appendix/AppendixC}

\backmatter
\bibliographystyle{ecs}
\bibliography{ECS}
\end{document}
%% ----------------------------------------------------------------

\include{chapters/3Design/HighLevel}
\include{chapters/3Design/InDepth}
 %% ----------------------------------------------------------------
%% Progress.tex
%% ---------------------------------------------------------------- 
\documentclass{ecsprogress}    % Use the progress Style
\graphicspath{{../figs/}}   % Location of your graphics files
    \usepackage{natbib}            % Use Natbib style for the refs.
\hypersetup{colorlinks=true}   % Set to false for black/white printing
\input{Definitions}            % Include your abbreviations



\usepackage{enumitem}% http://ctan.org/pkg/enumitem
\usepackage{multirow}
\usepackage{float}
\usepackage{amsmath}
\usepackage{multicol}
\usepackage{amssymb}
\usepackage[normalem]{ulem}
\useunder{\uline}{\ul}{}
\usepackage{wrapfig}


\usepackage[table,xcdraw]{xcolor}


%% ----------------------------------------------------------------
\begin{document}
\frontmatter
\title      {Heterogeneous Agent-based Model for Supermarket Competition}
\authors    {\texorpdfstring
             {\href{mailto:sc22g13@ecs.soton.ac.uk}{Stefan J. Collier}}
             {Stefan J. Collier}
            }
\addresses  {\groupname\\\deptname\\\univname}
\date       {\today}
\subject    {}
\keywords   {}
\supervisor {Dr. Maria Polukarov}
\examiner   {Professor Sheng Chen}

\maketitle
\begin{abstract}
This project aim was to model and analyse the effects of competitive pricing behaviors of grocery retailers on the British market. 

This was achieved by creating a multi-agent model, containing retailer and consumer agents. The heterogeneous crowd of retailers employs either a uniform pricing strategy or a ‘local price flexing’ strategy. The actions of these retailers are chosen by predicting the profit of each action, using a perceptron. Following on from the consideration of different economic models, a discrete model was developed so that software agents have a discrete environment to operate within. Within the model, it has been observed how supermarkets with differing behaviors affect a heterogeneous crowd of consumer agents. The model was implemented in Java with Python used to evaluate the results. 

The simulation displays good acceptance with real grocery market behavior, i.e. captures the performance of British retailers thus can be used to determine the impact of changes in their behavior on their competitors and consumers.Furthermore it can be used to provide insight into sustainability of volatile pricing strategies, providing a useful insight in volatility of British supermarket retail industry. 
\end{abstract}
\acknowledgements{
I would like to express my sincere gratitude to Dr Maria Polukarov for her guidance and support which provided me the freedom to take this research in the direction of my interest.\\
\\
I would also like to thank my family and friends for their encouragement and support. To those who quietly listened to my software complaints. To those who worked throughout the nights with me. To those who helped me write what I couldn't say. I cannot thank you enough.
}

\declaration{
I, Stefan Collier, declare that this dissertation and the work presented in it are my own and has been generated by me as the result of my own original research.\\
I confirm that:\\
1. This work was done wholly or mainly while in candidature for a degree at this University;\\
2. Where any part of this dissertation has previously been submitted for any other qualification at this University or any other institution, this has been clearly stated;\\
3. Where I have consulted the published work of others, this is always clearly attributed;\\
4. Where I have quoted from the work of others, the source is always given. With the exception of such quotations, this dissertation is entirely my own work;\\
5. I have acknowledged all main sources of help;\\
6. Where the thesis is based on work done by myself jointly with others, I have made clear exactly what was done by others and what I have contributed myself;\\
7. Either none of this work has been published before submission, or parts of this work have been published by :\\
\\
Stefan Collier\\
April 2016
}
\tableofcontents
\listoffigures
\listoftables

\mainmatter
%% ----------------------------------------------------------------
%\include{Introduction}
%\include{Conclusions}
\include{chapters/1Project/main}
\include{chapters/2Lit/main}
\include{chapters/3Design/HighLevel}
\include{chapters/3Design/InDepth}
\include{chapters/4Impl/main}

\include{chapters/5Experiments/1/main}
\include{chapters/5Experiments/2/main}
\include{chapters/5Experiments/3/main}
\include{chapters/5Experiments/4/main}

\include{chapters/6Conclusion/main}

\appendix
\include{appendix/AppendixB}
\include{appendix/D/main}
\include{appendix/AppendixC}

\backmatter
\bibliographystyle{ecs}
\bibliography{ECS}
\end{document}
%% ----------------------------------------------------------------


 %% ----------------------------------------------------------------
%% Progress.tex
%% ---------------------------------------------------------------- 
\documentclass{ecsprogress}    % Use the progress Style
\graphicspath{{../figs/}}   % Location of your graphics files
    \usepackage{natbib}            % Use Natbib style for the refs.
\hypersetup{colorlinks=true}   % Set to false for black/white printing
\input{Definitions}            % Include your abbreviations



\usepackage{enumitem}% http://ctan.org/pkg/enumitem
\usepackage{multirow}
\usepackage{float}
\usepackage{amsmath}
\usepackage{multicol}
\usepackage{amssymb}
\usepackage[normalem]{ulem}
\useunder{\uline}{\ul}{}
\usepackage{wrapfig}


\usepackage[table,xcdraw]{xcolor}


%% ----------------------------------------------------------------
\begin{document}
\frontmatter
\title      {Heterogeneous Agent-based Model for Supermarket Competition}
\authors    {\texorpdfstring
             {\href{mailto:sc22g13@ecs.soton.ac.uk}{Stefan J. Collier}}
             {Stefan J. Collier}
            }
\addresses  {\groupname\\\deptname\\\univname}
\date       {\today}
\subject    {}
\keywords   {}
\supervisor {Dr. Maria Polukarov}
\examiner   {Professor Sheng Chen}

\maketitle
\begin{abstract}
This project aim was to model and analyse the effects of competitive pricing behaviors of grocery retailers on the British market. 

This was achieved by creating a multi-agent model, containing retailer and consumer agents. The heterogeneous crowd of retailers employs either a uniform pricing strategy or a ‘local price flexing’ strategy. The actions of these retailers are chosen by predicting the profit of each action, using a perceptron. Following on from the consideration of different economic models, a discrete model was developed so that software agents have a discrete environment to operate within. Within the model, it has been observed how supermarkets with differing behaviors affect a heterogeneous crowd of consumer agents. The model was implemented in Java with Python used to evaluate the results. 

The simulation displays good acceptance with real grocery market behavior, i.e. captures the performance of British retailers thus can be used to determine the impact of changes in their behavior on their competitors and consumers.Furthermore it can be used to provide insight into sustainability of volatile pricing strategies, providing a useful insight in volatility of British supermarket retail industry. 
\end{abstract}
\acknowledgements{
I would like to express my sincere gratitude to Dr Maria Polukarov for her guidance and support which provided me the freedom to take this research in the direction of my interest.\\
\\
I would also like to thank my family and friends for their encouragement and support. To those who quietly listened to my software complaints. To those who worked throughout the nights with me. To those who helped me write what I couldn't say. I cannot thank you enough.
}

\declaration{
I, Stefan Collier, declare that this dissertation and the work presented in it are my own and has been generated by me as the result of my own original research.\\
I confirm that:\\
1. This work was done wholly or mainly while in candidature for a degree at this University;\\
2. Where any part of this dissertation has previously been submitted for any other qualification at this University or any other institution, this has been clearly stated;\\
3. Where I have consulted the published work of others, this is always clearly attributed;\\
4. Where I have quoted from the work of others, the source is always given. With the exception of such quotations, this dissertation is entirely my own work;\\
5. I have acknowledged all main sources of help;\\
6. Where the thesis is based on work done by myself jointly with others, I have made clear exactly what was done by others and what I have contributed myself;\\
7. Either none of this work has been published before submission, or parts of this work have been published by :\\
\\
Stefan Collier\\
April 2016
}
\tableofcontents
\listoffigures
\listoftables

\mainmatter
%% ----------------------------------------------------------------
%\include{Introduction}
%\include{Conclusions}
\include{chapters/1Project/main}
\include{chapters/2Lit/main}
\include{chapters/3Design/HighLevel}
\include{chapters/3Design/InDepth}
\include{chapters/4Impl/main}

\include{chapters/5Experiments/1/main}
\include{chapters/5Experiments/2/main}
\include{chapters/5Experiments/3/main}
\include{chapters/5Experiments/4/main}

\include{chapters/6Conclusion/main}

\appendix
\include{appendix/AppendixB}
\include{appendix/D/main}
\include{appendix/AppendixC}

\backmatter
\bibliographystyle{ecs}
\bibliography{ECS}
\end{document}
%% ----------------------------------------------------------------

 %% ----------------------------------------------------------------
%% Progress.tex
%% ---------------------------------------------------------------- 
\documentclass{ecsprogress}    % Use the progress Style
\graphicspath{{../figs/}}   % Location of your graphics files
    \usepackage{natbib}            % Use Natbib style for the refs.
\hypersetup{colorlinks=true}   % Set to false for black/white printing
\input{Definitions}            % Include your abbreviations



\usepackage{enumitem}% http://ctan.org/pkg/enumitem
\usepackage{multirow}
\usepackage{float}
\usepackage{amsmath}
\usepackage{multicol}
\usepackage{amssymb}
\usepackage[normalem]{ulem}
\useunder{\uline}{\ul}{}
\usepackage{wrapfig}


\usepackage[table,xcdraw]{xcolor}


%% ----------------------------------------------------------------
\begin{document}
\frontmatter
\title      {Heterogeneous Agent-based Model for Supermarket Competition}
\authors    {\texorpdfstring
             {\href{mailto:sc22g13@ecs.soton.ac.uk}{Stefan J. Collier}}
             {Stefan J. Collier}
            }
\addresses  {\groupname\\\deptname\\\univname}
\date       {\today}
\subject    {}
\keywords   {}
\supervisor {Dr. Maria Polukarov}
\examiner   {Professor Sheng Chen}

\maketitle
\begin{abstract}
This project aim was to model and analyse the effects of competitive pricing behaviors of grocery retailers on the British market. 

This was achieved by creating a multi-agent model, containing retailer and consumer agents. The heterogeneous crowd of retailers employs either a uniform pricing strategy or a ‘local price flexing’ strategy. The actions of these retailers are chosen by predicting the profit of each action, using a perceptron. Following on from the consideration of different economic models, a discrete model was developed so that software agents have a discrete environment to operate within. Within the model, it has been observed how supermarkets with differing behaviors affect a heterogeneous crowd of consumer agents. The model was implemented in Java with Python used to evaluate the results. 

The simulation displays good acceptance with real grocery market behavior, i.e. captures the performance of British retailers thus can be used to determine the impact of changes in their behavior on their competitors and consumers.Furthermore it can be used to provide insight into sustainability of volatile pricing strategies, providing a useful insight in volatility of British supermarket retail industry. 
\end{abstract}
\acknowledgements{
I would like to express my sincere gratitude to Dr Maria Polukarov for her guidance and support which provided me the freedom to take this research in the direction of my interest.\\
\\
I would also like to thank my family and friends for their encouragement and support. To those who quietly listened to my software complaints. To those who worked throughout the nights with me. To those who helped me write what I couldn't say. I cannot thank you enough.
}

\declaration{
I, Stefan Collier, declare that this dissertation and the work presented in it are my own and has been generated by me as the result of my own original research.\\
I confirm that:\\
1. This work was done wholly or mainly while in candidature for a degree at this University;\\
2. Where any part of this dissertation has previously been submitted for any other qualification at this University or any other institution, this has been clearly stated;\\
3. Where I have consulted the published work of others, this is always clearly attributed;\\
4. Where I have quoted from the work of others, the source is always given. With the exception of such quotations, this dissertation is entirely my own work;\\
5. I have acknowledged all main sources of help;\\
6. Where the thesis is based on work done by myself jointly with others, I have made clear exactly what was done by others and what I have contributed myself;\\
7. Either none of this work has been published before submission, or parts of this work have been published by :\\
\\
Stefan Collier\\
April 2016
}
\tableofcontents
\listoffigures
\listoftables

\mainmatter
%% ----------------------------------------------------------------
%\include{Introduction}
%\include{Conclusions}
\include{chapters/1Project/main}
\include{chapters/2Lit/main}
\include{chapters/3Design/HighLevel}
\include{chapters/3Design/InDepth}
\include{chapters/4Impl/main}

\include{chapters/5Experiments/1/main}
\include{chapters/5Experiments/2/main}
\include{chapters/5Experiments/3/main}
\include{chapters/5Experiments/4/main}

\include{chapters/6Conclusion/main}

\appendix
\include{appendix/AppendixB}
\include{appendix/D/main}
\include{appendix/AppendixC}

\backmatter
\bibliographystyle{ecs}
\bibliography{ECS}
\end{document}
%% ----------------------------------------------------------------

 %% ----------------------------------------------------------------
%% Progress.tex
%% ---------------------------------------------------------------- 
\documentclass{ecsprogress}    % Use the progress Style
\graphicspath{{../figs/}}   % Location of your graphics files
    \usepackage{natbib}            % Use Natbib style for the refs.
\hypersetup{colorlinks=true}   % Set to false for black/white printing
\input{Definitions}            % Include your abbreviations



\usepackage{enumitem}% http://ctan.org/pkg/enumitem
\usepackage{multirow}
\usepackage{float}
\usepackage{amsmath}
\usepackage{multicol}
\usepackage{amssymb}
\usepackage[normalem]{ulem}
\useunder{\uline}{\ul}{}
\usepackage{wrapfig}


\usepackage[table,xcdraw]{xcolor}


%% ----------------------------------------------------------------
\begin{document}
\frontmatter
\title      {Heterogeneous Agent-based Model for Supermarket Competition}
\authors    {\texorpdfstring
             {\href{mailto:sc22g13@ecs.soton.ac.uk}{Stefan J. Collier}}
             {Stefan J. Collier}
            }
\addresses  {\groupname\\\deptname\\\univname}
\date       {\today}
\subject    {}
\keywords   {}
\supervisor {Dr. Maria Polukarov}
\examiner   {Professor Sheng Chen}

\maketitle
\begin{abstract}
This project aim was to model and analyse the effects of competitive pricing behaviors of grocery retailers on the British market. 

This was achieved by creating a multi-agent model, containing retailer and consumer agents. The heterogeneous crowd of retailers employs either a uniform pricing strategy or a ‘local price flexing’ strategy. The actions of these retailers are chosen by predicting the profit of each action, using a perceptron. Following on from the consideration of different economic models, a discrete model was developed so that software agents have a discrete environment to operate within. Within the model, it has been observed how supermarkets with differing behaviors affect a heterogeneous crowd of consumer agents. The model was implemented in Java with Python used to evaluate the results. 

The simulation displays good acceptance with real grocery market behavior, i.e. captures the performance of British retailers thus can be used to determine the impact of changes in their behavior on their competitors and consumers.Furthermore it can be used to provide insight into sustainability of volatile pricing strategies, providing a useful insight in volatility of British supermarket retail industry. 
\end{abstract}
\acknowledgements{
I would like to express my sincere gratitude to Dr Maria Polukarov for her guidance and support which provided me the freedom to take this research in the direction of my interest.\\
\\
I would also like to thank my family and friends for their encouragement and support. To those who quietly listened to my software complaints. To those who worked throughout the nights with me. To those who helped me write what I couldn't say. I cannot thank you enough.
}

\declaration{
I, Stefan Collier, declare that this dissertation and the work presented in it are my own and has been generated by me as the result of my own original research.\\
I confirm that:\\
1. This work was done wholly or mainly while in candidature for a degree at this University;\\
2. Where any part of this dissertation has previously been submitted for any other qualification at this University or any other institution, this has been clearly stated;\\
3. Where I have consulted the published work of others, this is always clearly attributed;\\
4. Where I have quoted from the work of others, the source is always given. With the exception of such quotations, this dissertation is entirely my own work;\\
5. I have acknowledged all main sources of help;\\
6. Where the thesis is based on work done by myself jointly with others, I have made clear exactly what was done by others and what I have contributed myself;\\
7. Either none of this work has been published before submission, or parts of this work have been published by :\\
\\
Stefan Collier\\
April 2016
}
\tableofcontents
\listoffigures
\listoftables

\mainmatter
%% ----------------------------------------------------------------
%\include{Introduction}
%\include{Conclusions}
\include{chapters/1Project/main}
\include{chapters/2Lit/main}
\include{chapters/3Design/HighLevel}
\include{chapters/3Design/InDepth}
\include{chapters/4Impl/main}

\include{chapters/5Experiments/1/main}
\include{chapters/5Experiments/2/main}
\include{chapters/5Experiments/3/main}
\include{chapters/5Experiments/4/main}

\include{chapters/6Conclusion/main}

\appendix
\include{appendix/AppendixB}
\include{appendix/D/main}
\include{appendix/AppendixC}

\backmatter
\bibliographystyle{ecs}
\bibliography{ECS}
\end{document}
%% ----------------------------------------------------------------

 %% ----------------------------------------------------------------
%% Progress.tex
%% ---------------------------------------------------------------- 
\documentclass{ecsprogress}    % Use the progress Style
\graphicspath{{../figs/}}   % Location of your graphics files
    \usepackage{natbib}            % Use Natbib style for the refs.
\hypersetup{colorlinks=true}   % Set to false for black/white printing
\input{Definitions}            % Include your abbreviations



\usepackage{enumitem}% http://ctan.org/pkg/enumitem
\usepackage{multirow}
\usepackage{float}
\usepackage{amsmath}
\usepackage{multicol}
\usepackage{amssymb}
\usepackage[normalem]{ulem}
\useunder{\uline}{\ul}{}
\usepackage{wrapfig}


\usepackage[table,xcdraw]{xcolor}


%% ----------------------------------------------------------------
\begin{document}
\frontmatter
\title      {Heterogeneous Agent-based Model for Supermarket Competition}
\authors    {\texorpdfstring
             {\href{mailto:sc22g13@ecs.soton.ac.uk}{Stefan J. Collier}}
             {Stefan J. Collier}
            }
\addresses  {\groupname\\\deptname\\\univname}
\date       {\today}
\subject    {}
\keywords   {}
\supervisor {Dr. Maria Polukarov}
\examiner   {Professor Sheng Chen}

\maketitle
\begin{abstract}
This project aim was to model and analyse the effects of competitive pricing behaviors of grocery retailers on the British market. 

This was achieved by creating a multi-agent model, containing retailer and consumer agents. The heterogeneous crowd of retailers employs either a uniform pricing strategy or a ‘local price flexing’ strategy. The actions of these retailers are chosen by predicting the profit of each action, using a perceptron. Following on from the consideration of different economic models, a discrete model was developed so that software agents have a discrete environment to operate within. Within the model, it has been observed how supermarkets with differing behaviors affect a heterogeneous crowd of consumer agents. The model was implemented in Java with Python used to evaluate the results. 

The simulation displays good acceptance with real grocery market behavior, i.e. captures the performance of British retailers thus can be used to determine the impact of changes in their behavior on their competitors and consumers.Furthermore it can be used to provide insight into sustainability of volatile pricing strategies, providing a useful insight in volatility of British supermarket retail industry. 
\end{abstract}
\acknowledgements{
I would like to express my sincere gratitude to Dr Maria Polukarov for her guidance and support which provided me the freedom to take this research in the direction of my interest.\\
\\
I would also like to thank my family and friends for their encouragement and support. To those who quietly listened to my software complaints. To those who worked throughout the nights with me. To those who helped me write what I couldn't say. I cannot thank you enough.
}

\declaration{
I, Stefan Collier, declare that this dissertation and the work presented in it are my own and has been generated by me as the result of my own original research.\\
I confirm that:\\
1. This work was done wholly or mainly while in candidature for a degree at this University;\\
2. Where any part of this dissertation has previously been submitted for any other qualification at this University or any other institution, this has been clearly stated;\\
3. Where I have consulted the published work of others, this is always clearly attributed;\\
4. Where I have quoted from the work of others, the source is always given. With the exception of such quotations, this dissertation is entirely my own work;\\
5. I have acknowledged all main sources of help;\\
6. Where the thesis is based on work done by myself jointly with others, I have made clear exactly what was done by others and what I have contributed myself;\\
7. Either none of this work has been published before submission, or parts of this work have been published by :\\
\\
Stefan Collier\\
April 2016
}
\tableofcontents
\listoffigures
\listoftables

\mainmatter
%% ----------------------------------------------------------------
%\include{Introduction}
%\include{Conclusions}
\include{chapters/1Project/main}
\include{chapters/2Lit/main}
\include{chapters/3Design/HighLevel}
\include{chapters/3Design/InDepth}
\include{chapters/4Impl/main}

\include{chapters/5Experiments/1/main}
\include{chapters/5Experiments/2/main}
\include{chapters/5Experiments/3/main}
\include{chapters/5Experiments/4/main}

\include{chapters/6Conclusion/main}

\appendix
\include{appendix/AppendixB}
\include{appendix/D/main}
\include{appendix/AppendixC}

\backmatter
\bibliographystyle{ecs}
\bibliography{ECS}
\end{document}
%% ----------------------------------------------------------------


 %% ----------------------------------------------------------------
%% Progress.tex
%% ---------------------------------------------------------------- 
\documentclass{ecsprogress}    % Use the progress Style
\graphicspath{{../figs/}}   % Location of your graphics files
    \usepackage{natbib}            % Use Natbib style for the refs.
\hypersetup{colorlinks=true}   % Set to false for black/white printing
\input{Definitions}            % Include your abbreviations



\usepackage{enumitem}% http://ctan.org/pkg/enumitem
\usepackage{multirow}
\usepackage{float}
\usepackage{amsmath}
\usepackage{multicol}
\usepackage{amssymb}
\usepackage[normalem]{ulem}
\useunder{\uline}{\ul}{}
\usepackage{wrapfig}


\usepackage[table,xcdraw]{xcolor}


%% ----------------------------------------------------------------
\begin{document}
\frontmatter
\title      {Heterogeneous Agent-based Model for Supermarket Competition}
\authors    {\texorpdfstring
             {\href{mailto:sc22g13@ecs.soton.ac.uk}{Stefan J. Collier}}
             {Stefan J. Collier}
            }
\addresses  {\groupname\\\deptname\\\univname}
\date       {\today}
\subject    {}
\keywords   {}
\supervisor {Dr. Maria Polukarov}
\examiner   {Professor Sheng Chen}

\maketitle
\begin{abstract}
This project aim was to model and analyse the effects of competitive pricing behaviors of grocery retailers on the British market. 

This was achieved by creating a multi-agent model, containing retailer and consumer agents. The heterogeneous crowd of retailers employs either a uniform pricing strategy or a ‘local price flexing’ strategy. The actions of these retailers are chosen by predicting the profit of each action, using a perceptron. Following on from the consideration of different economic models, a discrete model was developed so that software agents have a discrete environment to operate within. Within the model, it has been observed how supermarkets with differing behaviors affect a heterogeneous crowd of consumer agents. The model was implemented in Java with Python used to evaluate the results. 

The simulation displays good acceptance with real grocery market behavior, i.e. captures the performance of British retailers thus can be used to determine the impact of changes in their behavior on their competitors and consumers.Furthermore it can be used to provide insight into sustainability of volatile pricing strategies, providing a useful insight in volatility of British supermarket retail industry. 
\end{abstract}
\acknowledgements{
I would like to express my sincere gratitude to Dr Maria Polukarov for her guidance and support which provided me the freedom to take this research in the direction of my interest.\\
\\
I would also like to thank my family and friends for their encouragement and support. To those who quietly listened to my software complaints. To those who worked throughout the nights with me. To those who helped me write what I couldn't say. I cannot thank you enough.
}

\declaration{
I, Stefan Collier, declare that this dissertation and the work presented in it are my own and has been generated by me as the result of my own original research.\\
I confirm that:\\
1. This work was done wholly or mainly while in candidature for a degree at this University;\\
2. Where any part of this dissertation has previously been submitted for any other qualification at this University or any other institution, this has been clearly stated;\\
3. Where I have consulted the published work of others, this is always clearly attributed;\\
4. Where I have quoted from the work of others, the source is always given. With the exception of such quotations, this dissertation is entirely my own work;\\
5. I have acknowledged all main sources of help;\\
6. Where the thesis is based on work done by myself jointly with others, I have made clear exactly what was done by others and what I have contributed myself;\\
7. Either none of this work has been published before submission, or parts of this work have been published by :\\
\\
Stefan Collier\\
April 2016
}
\tableofcontents
\listoffigures
\listoftables

\mainmatter
%% ----------------------------------------------------------------
%\include{Introduction}
%\include{Conclusions}
\include{chapters/1Project/main}
\include{chapters/2Lit/main}
\include{chapters/3Design/HighLevel}
\include{chapters/3Design/InDepth}
\include{chapters/4Impl/main}

\include{chapters/5Experiments/1/main}
\include{chapters/5Experiments/2/main}
\include{chapters/5Experiments/3/main}
\include{chapters/5Experiments/4/main}

\include{chapters/6Conclusion/main}

\appendix
\include{appendix/AppendixB}
\include{appendix/D/main}
\include{appendix/AppendixC}

\backmatter
\bibliographystyle{ecs}
\bibliography{ECS}
\end{document}
%% ----------------------------------------------------------------


\appendix
\include{appendix/AppendixB}
 %% ----------------------------------------------------------------
%% Progress.tex
%% ---------------------------------------------------------------- 
\documentclass{ecsprogress}    % Use the progress Style
\graphicspath{{../figs/}}   % Location of your graphics files
    \usepackage{natbib}            % Use Natbib style for the refs.
\hypersetup{colorlinks=true}   % Set to false for black/white printing
\input{Definitions}            % Include your abbreviations



\usepackage{enumitem}% http://ctan.org/pkg/enumitem
\usepackage{multirow}
\usepackage{float}
\usepackage{amsmath}
\usepackage{multicol}
\usepackage{amssymb}
\usepackage[normalem]{ulem}
\useunder{\uline}{\ul}{}
\usepackage{wrapfig}


\usepackage[table,xcdraw]{xcolor}


%% ----------------------------------------------------------------
\begin{document}
\frontmatter
\title      {Heterogeneous Agent-based Model for Supermarket Competition}
\authors    {\texorpdfstring
             {\href{mailto:sc22g13@ecs.soton.ac.uk}{Stefan J. Collier}}
             {Stefan J. Collier}
            }
\addresses  {\groupname\\\deptname\\\univname}
\date       {\today}
\subject    {}
\keywords   {}
\supervisor {Dr. Maria Polukarov}
\examiner   {Professor Sheng Chen}

\maketitle
\begin{abstract}
This project aim was to model and analyse the effects of competitive pricing behaviors of grocery retailers on the British market. 

This was achieved by creating a multi-agent model, containing retailer and consumer agents. The heterogeneous crowd of retailers employs either a uniform pricing strategy or a ‘local price flexing’ strategy. The actions of these retailers are chosen by predicting the profit of each action, using a perceptron. Following on from the consideration of different economic models, a discrete model was developed so that software agents have a discrete environment to operate within. Within the model, it has been observed how supermarkets with differing behaviors affect a heterogeneous crowd of consumer agents. The model was implemented in Java with Python used to evaluate the results. 

The simulation displays good acceptance with real grocery market behavior, i.e. captures the performance of British retailers thus can be used to determine the impact of changes in their behavior on their competitors and consumers.Furthermore it can be used to provide insight into sustainability of volatile pricing strategies, providing a useful insight in volatility of British supermarket retail industry. 
\end{abstract}
\acknowledgements{
I would like to express my sincere gratitude to Dr Maria Polukarov for her guidance and support which provided me the freedom to take this research in the direction of my interest.\\
\\
I would also like to thank my family and friends for their encouragement and support. To those who quietly listened to my software complaints. To those who worked throughout the nights with me. To those who helped me write what I couldn't say. I cannot thank you enough.
}

\declaration{
I, Stefan Collier, declare that this dissertation and the work presented in it are my own and has been generated by me as the result of my own original research.\\
I confirm that:\\
1. This work was done wholly or mainly while in candidature for a degree at this University;\\
2. Where any part of this dissertation has previously been submitted for any other qualification at this University or any other institution, this has been clearly stated;\\
3. Where I have consulted the published work of others, this is always clearly attributed;\\
4. Where I have quoted from the work of others, the source is always given. With the exception of such quotations, this dissertation is entirely my own work;\\
5. I have acknowledged all main sources of help;\\
6. Where the thesis is based on work done by myself jointly with others, I have made clear exactly what was done by others and what I have contributed myself;\\
7. Either none of this work has been published before submission, or parts of this work have been published by :\\
\\
Stefan Collier\\
April 2016
}
\tableofcontents
\listoffigures
\listoftables

\mainmatter
%% ----------------------------------------------------------------
%\include{Introduction}
%\include{Conclusions}
\include{chapters/1Project/main}
\include{chapters/2Lit/main}
\include{chapters/3Design/HighLevel}
\include{chapters/3Design/InDepth}
\include{chapters/4Impl/main}

\include{chapters/5Experiments/1/main}
\include{chapters/5Experiments/2/main}
\include{chapters/5Experiments/3/main}
\include{chapters/5Experiments/4/main}

\include{chapters/6Conclusion/main}

\appendix
\include{appendix/AppendixB}
\include{appendix/D/main}
\include{appendix/AppendixC}

\backmatter
\bibliographystyle{ecs}
\bibliography{ECS}
\end{document}
%% ----------------------------------------------------------------

\include{appendix/AppendixC}

\backmatter
\bibliographystyle{ecs}
\bibliography{ECS}
\end{document}
%% ----------------------------------------------------------------

\include{appendix/AppendixC}

\backmatter
\bibliographystyle{ecs}
\bibliography{ECS}
\end{document}
%% ----------------------------------------------------------------

 %% ----------------------------------------------------------------
%% Progress.tex
%% ---------------------------------------------------------------- 
\documentclass{ecsprogress}    % Use the progress Style
\graphicspath{{../figs/}}   % Location of your graphics files
    \usepackage{natbib}            % Use Natbib style for the refs.
\hypersetup{colorlinks=true}   % Set to false for black/white printing
\input{Definitions}            % Include your abbreviations



\usepackage{enumitem}% http://ctan.org/pkg/enumitem
\usepackage{multirow}
\usepackage{float}
\usepackage{amsmath}
\usepackage{multicol}
\usepackage{amssymb}
\usepackage[normalem]{ulem}
\useunder{\uline}{\ul}{}
\usepackage{wrapfig}


\usepackage[table,xcdraw]{xcolor}


%% ----------------------------------------------------------------
\begin{document}
\frontmatter
\title      {Heterogeneous Agent-based Model for Supermarket Competition}
\authors    {\texorpdfstring
             {\href{mailto:sc22g13@ecs.soton.ac.uk}{Stefan J. Collier}}
             {Stefan J. Collier}
            }
\addresses  {\groupname\\\deptname\\\univname}
\date       {\today}
\subject    {}
\keywords   {}
\supervisor {Dr. Maria Polukarov}
\examiner   {Professor Sheng Chen}

\maketitle
\begin{abstract}
This project aim was to model and analyse the effects of competitive pricing behaviors of grocery retailers on the British market. 

This was achieved by creating a multi-agent model, containing retailer and consumer agents. The heterogeneous crowd of retailers employs either a uniform pricing strategy or a ‘local price flexing’ strategy. The actions of these retailers are chosen by predicting the profit of each action, using a perceptron. Following on from the consideration of different economic models, a discrete model was developed so that software agents have a discrete environment to operate within. Within the model, it has been observed how supermarkets with differing behaviors affect a heterogeneous crowd of consumer agents. The model was implemented in Java with Python used to evaluate the results. 

The simulation displays good acceptance with real grocery market behavior, i.e. captures the performance of British retailers thus can be used to determine the impact of changes in their behavior on their competitors and consumers.Furthermore it can be used to provide insight into sustainability of volatile pricing strategies, providing a useful insight in volatility of British supermarket retail industry. 
\end{abstract}
\acknowledgements{
I would like to express my sincere gratitude to Dr Maria Polukarov for her guidance and support which provided me the freedom to take this research in the direction of my interest.\\
\\
I would also like to thank my family and friends for their encouragement and support. To those who quietly listened to my software complaints. To those who worked throughout the nights with me. To those who helped me write what I couldn't say. I cannot thank you enough.
}

\declaration{
I, Stefan Collier, declare that this dissertation and the work presented in it are my own and has been generated by me as the result of my own original research.\\
I confirm that:\\
1. This work was done wholly or mainly while in candidature for a degree at this University;\\
2. Where any part of this dissertation has previously been submitted for any other qualification at this University or any other institution, this has been clearly stated;\\
3. Where I have consulted the published work of others, this is always clearly attributed;\\
4. Where I have quoted from the work of others, the source is always given. With the exception of such quotations, this dissertation is entirely my own work;\\
5. I have acknowledged all main sources of help;\\
6. Where the thesis is based on work done by myself jointly with others, I have made clear exactly what was done by others and what I have contributed myself;\\
7. Either none of this work has been published before submission, or parts of this work have been published by :\\
\\
Stefan Collier\\
April 2016
}
\tableofcontents
\listoffigures
\listoftables

\mainmatter
%% ----------------------------------------------------------------
%\include{Introduction}
%\include{Conclusions}
 %% ----------------------------------------------------------------
%% Progress.tex
%% ---------------------------------------------------------------- 
\documentclass{ecsprogress}    % Use the progress Style
\graphicspath{{../figs/}}   % Location of your graphics files
    \usepackage{natbib}            % Use Natbib style for the refs.
\hypersetup{colorlinks=true}   % Set to false for black/white printing
\input{Definitions}            % Include your abbreviations



\usepackage{enumitem}% http://ctan.org/pkg/enumitem
\usepackage{multirow}
\usepackage{float}
\usepackage{amsmath}
\usepackage{multicol}
\usepackage{amssymb}
\usepackage[normalem]{ulem}
\useunder{\uline}{\ul}{}
\usepackage{wrapfig}


\usepackage[table,xcdraw]{xcolor}


%% ----------------------------------------------------------------
\begin{document}
\frontmatter
\title      {Heterogeneous Agent-based Model for Supermarket Competition}
\authors    {\texorpdfstring
             {\href{mailto:sc22g13@ecs.soton.ac.uk}{Stefan J. Collier}}
             {Stefan J. Collier}
            }
\addresses  {\groupname\\\deptname\\\univname}
\date       {\today}
\subject    {}
\keywords   {}
\supervisor {Dr. Maria Polukarov}
\examiner   {Professor Sheng Chen}

\maketitle
\begin{abstract}
This project aim was to model and analyse the effects of competitive pricing behaviors of grocery retailers on the British market. 

This was achieved by creating a multi-agent model, containing retailer and consumer agents. The heterogeneous crowd of retailers employs either a uniform pricing strategy or a ‘local price flexing’ strategy. The actions of these retailers are chosen by predicting the profit of each action, using a perceptron. Following on from the consideration of different economic models, a discrete model was developed so that software agents have a discrete environment to operate within. Within the model, it has been observed how supermarkets with differing behaviors affect a heterogeneous crowd of consumer agents. The model was implemented in Java with Python used to evaluate the results. 

The simulation displays good acceptance with real grocery market behavior, i.e. captures the performance of British retailers thus can be used to determine the impact of changes in their behavior on their competitors and consumers.Furthermore it can be used to provide insight into sustainability of volatile pricing strategies, providing a useful insight in volatility of British supermarket retail industry. 
\end{abstract}
\acknowledgements{
I would like to express my sincere gratitude to Dr Maria Polukarov for her guidance and support which provided me the freedom to take this research in the direction of my interest.\\
\\
I would also like to thank my family and friends for their encouragement and support. To those who quietly listened to my software complaints. To those who worked throughout the nights with me. To those who helped me write what I couldn't say. I cannot thank you enough.
}

\declaration{
I, Stefan Collier, declare that this dissertation and the work presented in it are my own and has been generated by me as the result of my own original research.\\
I confirm that:\\
1. This work was done wholly or mainly while in candidature for a degree at this University;\\
2. Where any part of this dissertation has previously been submitted for any other qualification at this University or any other institution, this has been clearly stated;\\
3. Where I have consulted the published work of others, this is always clearly attributed;\\
4. Where I have quoted from the work of others, the source is always given. With the exception of such quotations, this dissertation is entirely my own work;\\
5. I have acknowledged all main sources of help;\\
6. Where the thesis is based on work done by myself jointly with others, I have made clear exactly what was done by others and what I have contributed myself;\\
7. Either none of this work has been published before submission, or parts of this work have been published by :\\
\\
Stefan Collier\\
April 2016
}
\tableofcontents
\listoffigures
\listoftables

\mainmatter
%% ----------------------------------------------------------------
%\include{Introduction}
%\include{Conclusions}
 %% ----------------------------------------------------------------
%% Progress.tex
%% ---------------------------------------------------------------- 
\documentclass{ecsprogress}    % Use the progress Style
\graphicspath{{../figs/}}   % Location of your graphics files
    \usepackage{natbib}            % Use Natbib style for the refs.
\hypersetup{colorlinks=true}   % Set to false for black/white printing
\input{Definitions}            % Include your abbreviations



\usepackage{enumitem}% http://ctan.org/pkg/enumitem
\usepackage{multirow}
\usepackage{float}
\usepackage{amsmath}
\usepackage{multicol}
\usepackage{amssymb}
\usepackage[normalem]{ulem}
\useunder{\uline}{\ul}{}
\usepackage{wrapfig}


\usepackage[table,xcdraw]{xcolor}


%% ----------------------------------------------------------------
\begin{document}
\frontmatter
\title      {Heterogeneous Agent-based Model for Supermarket Competition}
\authors    {\texorpdfstring
             {\href{mailto:sc22g13@ecs.soton.ac.uk}{Stefan J. Collier}}
             {Stefan J. Collier}
            }
\addresses  {\groupname\\\deptname\\\univname}
\date       {\today}
\subject    {}
\keywords   {}
\supervisor {Dr. Maria Polukarov}
\examiner   {Professor Sheng Chen}

\maketitle
\begin{abstract}
This project aim was to model and analyse the effects of competitive pricing behaviors of grocery retailers on the British market. 

This was achieved by creating a multi-agent model, containing retailer and consumer agents. The heterogeneous crowd of retailers employs either a uniform pricing strategy or a ‘local price flexing’ strategy. The actions of these retailers are chosen by predicting the profit of each action, using a perceptron. Following on from the consideration of different economic models, a discrete model was developed so that software agents have a discrete environment to operate within. Within the model, it has been observed how supermarkets with differing behaviors affect a heterogeneous crowd of consumer agents. The model was implemented in Java with Python used to evaluate the results. 

The simulation displays good acceptance with real grocery market behavior, i.e. captures the performance of British retailers thus can be used to determine the impact of changes in their behavior on their competitors and consumers.Furthermore it can be used to provide insight into sustainability of volatile pricing strategies, providing a useful insight in volatility of British supermarket retail industry. 
\end{abstract}
\acknowledgements{
I would like to express my sincere gratitude to Dr Maria Polukarov for her guidance and support which provided me the freedom to take this research in the direction of my interest.\\
\\
I would also like to thank my family and friends for their encouragement and support. To those who quietly listened to my software complaints. To those who worked throughout the nights with me. To those who helped me write what I couldn't say. I cannot thank you enough.
}

\declaration{
I, Stefan Collier, declare that this dissertation and the work presented in it are my own and has been generated by me as the result of my own original research.\\
I confirm that:\\
1. This work was done wholly or mainly while in candidature for a degree at this University;\\
2. Where any part of this dissertation has previously been submitted for any other qualification at this University or any other institution, this has been clearly stated;\\
3. Where I have consulted the published work of others, this is always clearly attributed;\\
4. Where I have quoted from the work of others, the source is always given. With the exception of such quotations, this dissertation is entirely my own work;\\
5. I have acknowledged all main sources of help;\\
6. Where the thesis is based on work done by myself jointly with others, I have made clear exactly what was done by others and what I have contributed myself;\\
7. Either none of this work has been published before submission, or parts of this work have been published by :\\
\\
Stefan Collier\\
April 2016
}
\tableofcontents
\listoffigures
\listoftables

\mainmatter
%% ----------------------------------------------------------------
%\include{Introduction}
%\include{Conclusions}
\include{chapters/1Project/main}
\include{chapters/2Lit/main}
\include{chapters/3Design/HighLevel}
\include{chapters/3Design/InDepth}
\include{chapters/4Impl/main}

\include{chapters/5Experiments/1/main}
\include{chapters/5Experiments/2/main}
\include{chapters/5Experiments/3/main}
\include{chapters/5Experiments/4/main}

\include{chapters/6Conclusion/main}

\appendix
\include{appendix/AppendixB}
\include{appendix/D/main}
\include{appendix/AppendixC}

\backmatter
\bibliographystyle{ecs}
\bibliography{ECS}
\end{document}
%% ----------------------------------------------------------------

 %% ----------------------------------------------------------------
%% Progress.tex
%% ---------------------------------------------------------------- 
\documentclass{ecsprogress}    % Use the progress Style
\graphicspath{{../figs/}}   % Location of your graphics files
    \usepackage{natbib}            % Use Natbib style for the refs.
\hypersetup{colorlinks=true}   % Set to false for black/white printing
\input{Definitions}            % Include your abbreviations



\usepackage{enumitem}% http://ctan.org/pkg/enumitem
\usepackage{multirow}
\usepackage{float}
\usepackage{amsmath}
\usepackage{multicol}
\usepackage{amssymb}
\usepackage[normalem]{ulem}
\useunder{\uline}{\ul}{}
\usepackage{wrapfig}


\usepackage[table,xcdraw]{xcolor}


%% ----------------------------------------------------------------
\begin{document}
\frontmatter
\title      {Heterogeneous Agent-based Model for Supermarket Competition}
\authors    {\texorpdfstring
             {\href{mailto:sc22g13@ecs.soton.ac.uk}{Stefan J. Collier}}
             {Stefan J. Collier}
            }
\addresses  {\groupname\\\deptname\\\univname}
\date       {\today}
\subject    {}
\keywords   {}
\supervisor {Dr. Maria Polukarov}
\examiner   {Professor Sheng Chen}

\maketitle
\begin{abstract}
This project aim was to model and analyse the effects of competitive pricing behaviors of grocery retailers on the British market. 

This was achieved by creating a multi-agent model, containing retailer and consumer agents. The heterogeneous crowd of retailers employs either a uniform pricing strategy or a ‘local price flexing’ strategy. The actions of these retailers are chosen by predicting the profit of each action, using a perceptron. Following on from the consideration of different economic models, a discrete model was developed so that software agents have a discrete environment to operate within. Within the model, it has been observed how supermarkets with differing behaviors affect a heterogeneous crowd of consumer agents. The model was implemented in Java with Python used to evaluate the results. 

The simulation displays good acceptance with real grocery market behavior, i.e. captures the performance of British retailers thus can be used to determine the impact of changes in their behavior on their competitors and consumers.Furthermore it can be used to provide insight into sustainability of volatile pricing strategies, providing a useful insight in volatility of British supermarket retail industry. 
\end{abstract}
\acknowledgements{
I would like to express my sincere gratitude to Dr Maria Polukarov for her guidance and support which provided me the freedom to take this research in the direction of my interest.\\
\\
I would also like to thank my family and friends for their encouragement and support. To those who quietly listened to my software complaints. To those who worked throughout the nights with me. To those who helped me write what I couldn't say. I cannot thank you enough.
}

\declaration{
I, Stefan Collier, declare that this dissertation and the work presented in it are my own and has been generated by me as the result of my own original research.\\
I confirm that:\\
1. This work was done wholly or mainly while in candidature for a degree at this University;\\
2. Where any part of this dissertation has previously been submitted for any other qualification at this University or any other institution, this has been clearly stated;\\
3. Where I have consulted the published work of others, this is always clearly attributed;\\
4. Where I have quoted from the work of others, the source is always given. With the exception of such quotations, this dissertation is entirely my own work;\\
5. I have acknowledged all main sources of help;\\
6. Where the thesis is based on work done by myself jointly with others, I have made clear exactly what was done by others and what I have contributed myself;\\
7. Either none of this work has been published before submission, or parts of this work have been published by :\\
\\
Stefan Collier\\
April 2016
}
\tableofcontents
\listoffigures
\listoftables

\mainmatter
%% ----------------------------------------------------------------
%\include{Introduction}
%\include{Conclusions}
\include{chapters/1Project/main}
\include{chapters/2Lit/main}
\include{chapters/3Design/HighLevel}
\include{chapters/3Design/InDepth}
\include{chapters/4Impl/main}

\include{chapters/5Experiments/1/main}
\include{chapters/5Experiments/2/main}
\include{chapters/5Experiments/3/main}
\include{chapters/5Experiments/4/main}

\include{chapters/6Conclusion/main}

\appendix
\include{appendix/AppendixB}
\include{appendix/D/main}
\include{appendix/AppendixC}

\backmatter
\bibliographystyle{ecs}
\bibliography{ECS}
\end{document}
%% ----------------------------------------------------------------

\include{chapters/3Design/HighLevel}
\include{chapters/3Design/InDepth}
 %% ----------------------------------------------------------------
%% Progress.tex
%% ---------------------------------------------------------------- 
\documentclass{ecsprogress}    % Use the progress Style
\graphicspath{{../figs/}}   % Location of your graphics files
    \usepackage{natbib}            % Use Natbib style for the refs.
\hypersetup{colorlinks=true}   % Set to false for black/white printing
\input{Definitions}            % Include your abbreviations



\usepackage{enumitem}% http://ctan.org/pkg/enumitem
\usepackage{multirow}
\usepackage{float}
\usepackage{amsmath}
\usepackage{multicol}
\usepackage{amssymb}
\usepackage[normalem]{ulem}
\useunder{\uline}{\ul}{}
\usepackage{wrapfig}


\usepackage[table,xcdraw]{xcolor}


%% ----------------------------------------------------------------
\begin{document}
\frontmatter
\title      {Heterogeneous Agent-based Model for Supermarket Competition}
\authors    {\texorpdfstring
             {\href{mailto:sc22g13@ecs.soton.ac.uk}{Stefan J. Collier}}
             {Stefan J. Collier}
            }
\addresses  {\groupname\\\deptname\\\univname}
\date       {\today}
\subject    {}
\keywords   {}
\supervisor {Dr. Maria Polukarov}
\examiner   {Professor Sheng Chen}

\maketitle
\begin{abstract}
This project aim was to model and analyse the effects of competitive pricing behaviors of grocery retailers on the British market. 

This was achieved by creating a multi-agent model, containing retailer and consumer agents. The heterogeneous crowd of retailers employs either a uniform pricing strategy or a ‘local price flexing’ strategy. The actions of these retailers are chosen by predicting the profit of each action, using a perceptron. Following on from the consideration of different economic models, a discrete model was developed so that software agents have a discrete environment to operate within. Within the model, it has been observed how supermarkets with differing behaviors affect a heterogeneous crowd of consumer agents. The model was implemented in Java with Python used to evaluate the results. 

The simulation displays good acceptance with real grocery market behavior, i.e. captures the performance of British retailers thus can be used to determine the impact of changes in their behavior on their competitors and consumers.Furthermore it can be used to provide insight into sustainability of volatile pricing strategies, providing a useful insight in volatility of British supermarket retail industry. 
\end{abstract}
\acknowledgements{
I would like to express my sincere gratitude to Dr Maria Polukarov for her guidance and support which provided me the freedom to take this research in the direction of my interest.\\
\\
I would also like to thank my family and friends for their encouragement and support. To those who quietly listened to my software complaints. To those who worked throughout the nights with me. To those who helped me write what I couldn't say. I cannot thank you enough.
}

\declaration{
I, Stefan Collier, declare that this dissertation and the work presented in it are my own and has been generated by me as the result of my own original research.\\
I confirm that:\\
1. This work was done wholly or mainly while in candidature for a degree at this University;\\
2. Where any part of this dissertation has previously been submitted for any other qualification at this University or any other institution, this has been clearly stated;\\
3. Where I have consulted the published work of others, this is always clearly attributed;\\
4. Where I have quoted from the work of others, the source is always given. With the exception of such quotations, this dissertation is entirely my own work;\\
5. I have acknowledged all main sources of help;\\
6. Where the thesis is based on work done by myself jointly with others, I have made clear exactly what was done by others and what I have contributed myself;\\
7. Either none of this work has been published before submission, or parts of this work have been published by :\\
\\
Stefan Collier\\
April 2016
}
\tableofcontents
\listoffigures
\listoftables

\mainmatter
%% ----------------------------------------------------------------
%\include{Introduction}
%\include{Conclusions}
\include{chapters/1Project/main}
\include{chapters/2Lit/main}
\include{chapters/3Design/HighLevel}
\include{chapters/3Design/InDepth}
\include{chapters/4Impl/main}

\include{chapters/5Experiments/1/main}
\include{chapters/5Experiments/2/main}
\include{chapters/5Experiments/3/main}
\include{chapters/5Experiments/4/main}

\include{chapters/6Conclusion/main}

\appendix
\include{appendix/AppendixB}
\include{appendix/D/main}
\include{appendix/AppendixC}

\backmatter
\bibliographystyle{ecs}
\bibliography{ECS}
\end{document}
%% ----------------------------------------------------------------


 %% ----------------------------------------------------------------
%% Progress.tex
%% ---------------------------------------------------------------- 
\documentclass{ecsprogress}    % Use the progress Style
\graphicspath{{../figs/}}   % Location of your graphics files
    \usepackage{natbib}            % Use Natbib style for the refs.
\hypersetup{colorlinks=true}   % Set to false for black/white printing
\input{Definitions}            % Include your abbreviations



\usepackage{enumitem}% http://ctan.org/pkg/enumitem
\usepackage{multirow}
\usepackage{float}
\usepackage{amsmath}
\usepackage{multicol}
\usepackage{amssymb}
\usepackage[normalem]{ulem}
\useunder{\uline}{\ul}{}
\usepackage{wrapfig}


\usepackage[table,xcdraw]{xcolor}


%% ----------------------------------------------------------------
\begin{document}
\frontmatter
\title      {Heterogeneous Agent-based Model for Supermarket Competition}
\authors    {\texorpdfstring
             {\href{mailto:sc22g13@ecs.soton.ac.uk}{Stefan J. Collier}}
             {Stefan J. Collier}
            }
\addresses  {\groupname\\\deptname\\\univname}
\date       {\today}
\subject    {}
\keywords   {}
\supervisor {Dr. Maria Polukarov}
\examiner   {Professor Sheng Chen}

\maketitle
\begin{abstract}
This project aim was to model and analyse the effects of competitive pricing behaviors of grocery retailers on the British market. 

This was achieved by creating a multi-agent model, containing retailer and consumer agents. The heterogeneous crowd of retailers employs either a uniform pricing strategy or a ‘local price flexing’ strategy. The actions of these retailers are chosen by predicting the profit of each action, using a perceptron. Following on from the consideration of different economic models, a discrete model was developed so that software agents have a discrete environment to operate within. Within the model, it has been observed how supermarkets with differing behaviors affect a heterogeneous crowd of consumer agents. The model was implemented in Java with Python used to evaluate the results. 

The simulation displays good acceptance with real grocery market behavior, i.e. captures the performance of British retailers thus can be used to determine the impact of changes in their behavior on their competitors and consumers.Furthermore it can be used to provide insight into sustainability of volatile pricing strategies, providing a useful insight in volatility of British supermarket retail industry. 
\end{abstract}
\acknowledgements{
I would like to express my sincere gratitude to Dr Maria Polukarov for her guidance and support which provided me the freedom to take this research in the direction of my interest.\\
\\
I would also like to thank my family and friends for their encouragement and support. To those who quietly listened to my software complaints. To those who worked throughout the nights with me. To those who helped me write what I couldn't say. I cannot thank you enough.
}

\declaration{
I, Stefan Collier, declare that this dissertation and the work presented in it are my own and has been generated by me as the result of my own original research.\\
I confirm that:\\
1. This work was done wholly or mainly while in candidature for a degree at this University;\\
2. Where any part of this dissertation has previously been submitted for any other qualification at this University or any other institution, this has been clearly stated;\\
3. Where I have consulted the published work of others, this is always clearly attributed;\\
4. Where I have quoted from the work of others, the source is always given. With the exception of such quotations, this dissertation is entirely my own work;\\
5. I have acknowledged all main sources of help;\\
6. Where the thesis is based on work done by myself jointly with others, I have made clear exactly what was done by others and what I have contributed myself;\\
7. Either none of this work has been published before submission, or parts of this work have been published by :\\
\\
Stefan Collier\\
April 2016
}
\tableofcontents
\listoffigures
\listoftables

\mainmatter
%% ----------------------------------------------------------------
%\include{Introduction}
%\include{Conclusions}
\include{chapters/1Project/main}
\include{chapters/2Lit/main}
\include{chapters/3Design/HighLevel}
\include{chapters/3Design/InDepth}
\include{chapters/4Impl/main}

\include{chapters/5Experiments/1/main}
\include{chapters/5Experiments/2/main}
\include{chapters/5Experiments/3/main}
\include{chapters/5Experiments/4/main}

\include{chapters/6Conclusion/main}

\appendix
\include{appendix/AppendixB}
\include{appendix/D/main}
\include{appendix/AppendixC}

\backmatter
\bibliographystyle{ecs}
\bibliography{ECS}
\end{document}
%% ----------------------------------------------------------------

 %% ----------------------------------------------------------------
%% Progress.tex
%% ---------------------------------------------------------------- 
\documentclass{ecsprogress}    % Use the progress Style
\graphicspath{{../figs/}}   % Location of your graphics files
    \usepackage{natbib}            % Use Natbib style for the refs.
\hypersetup{colorlinks=true}   % Set to false for black/white printing
\input{Definitions}            % Include your abbreviations



\usepackage{enumitem}% http://ctan.org/pkg/enumitem
\usepackage{multirow}
\usepackage{float}
\usepackage{amsmath}
\usepackage{multicol}
\usepackage{amssymb}
\usepackage[normalem]{ulem}
\useunder{\uline}{\ul}{}
\usepackage{wrapfig}


\usepackage[table,xcdraw]{xcolor}


%% ----------------------------------------------------------------
\begin{document}
\frontmatter
\title      {Heterogeneous Agent-based Model for Supermarket Competition}
\authors    {\texorpdfstring
             {\href{mailto:sc22g13@ecs.soton.ac.uk}{Stefan J. Collier}}
             {Stefan J. Collier}
            }
\addresses  {\groupname\\\deptname\\\univname}
\date       {\today}
\subject    {}
\keywords   {}
\supervisor {Dr. Maria Polukarov}
\examiner   {Professor Sheng Chen}

\maketitle
\begin{abstract}
This project aim was to model and analyse the effects of competitive pricing behaviors of grocery retailers on the British market. 

This was achieved by creating a multi-agent model, containing retailer and consumer agents. The heterogeneous crowd of retailers employs either a uniform pricing strategy or a ‘local price flexing’ strategy. The actions of these retailers are chosen by predicting the profit of each action, using a perceptron. Following on from the consideration of different economic models, a discrete model was developed so that software agents have a discrete environment to operate within. Within the model, it has been observed how supermarkets with differing behaviors affect a heterogeneous crowd of consumer agents. The model was implemented in Java with Python used to evaluate the results. 

The simulation displays good acceptance with real grocery market behavior, i.e. captures the performance of British retailers thus can be used to determine the impact of changes in their behavior on their competitors and consumers.Furthermore it can be used to provide insight into sustainability of volatile pricing strategies, providing a useful insight in volatility of British supermarket retail industry. 
\end{abstract}
\acknowledgements{
I would like to express my sincere gratitude to Dr Maria Polukarov for her guidance and support which provided me the freedom to take this research in the direction of my interest.\\
\\
I would also like to thank my family and friends for their encouragement and support. To those who quietly listened to my software complaints. To those who worked throughout the nights with me. To those who helped me write what I couldn't say. I cannot thank you enough.
}

\declaration{
I, Stefan Collier, declare that this dissertation and the work presented in it are my own and has been generated by me as the result of my own original research.\\
I confirm that:\\
1. This work was done wholly or mainly while in candidature for a degree at this University;\\
2. Where any part of this dissertation has previously been submitted for any other qualification at this University or any other institution, this has been clearly stated;\\
3. Where I have consulted the published work of others, this is always clearly attributed;\\
4. Where I have quoted from the work of others, the source is always given. With the exception of such quotations, this dissertation is entirely my own work;\\
5. I have acknowledged all main sources of help;\\
6. Where the thesis is based on work done by myself jointly with others, I have made clear exactly what was done by others and what I have contributed myself;\\
7. Either none of this work has been published before submission, or parts of this work have been published by :\\
\\
Stefan Collier\\
April 2016
}
\tableofcontents
\listoffigures
\listoftables

\mainmatter
%% ----------------------------------------------------------------
%\include{Introduction}
%\include{Conclusions}
\include{chapters/1Project/main}
\include{chapters/2Lit/main}
\include{chapters/3Design/HighLevel}
\include{chapters/3Design/InDepth}
\include{chapters/4Impl/main}

\include{chapters/5Experiments/1/main}
\include{chapters/5Experiments/2/main}
\include{chapters/5Experiments/3/main}
\include{chapters/5Experiments/4/main}

\include{chapters/6Conclusion/main}

\appendix
\include{appendix/AppendixB}
\include{appendix/D/main}
\include{appendix/AppendixC}

\backmatter
\bibliographystyle{ecs}
\bibliography{ECS}
\end{document}
%% ----------------------------------------------------------------

 %% ----------------------------------------------------------------
%% Progress.tex
%% ---------------------------------------------------------------- 
\documentclass{ecsprogress}    % Use the progress Style
\graphicspath{{../figs/}}   % Location of your graphics files
    \usepackage{natbib}            % Use Natbib style for the refs.
\hypersetup{colorlinks=true}   % Set to false for black/white printing
\input{Definitions}            % Include your abbreviations



\usepackage{enumitem}% http://ctan.org/pkg/enumitem
\usepackage{multirow}
\usepackage{float}
\usepackage{amsmath}
\usepackage{multicol}
\usepackage{amssymb}
\usepackage[normalem]{ulem}
\useunder{\uline}{\ul}{}
\usepackage{wrapfig}


\usepackage[table,xcdraw]{xcolor}


%% ----------------------------------------------------------------
\begin{document}
\frontmatter
\title      {Heterogeneous Agent-based Model for Supermarket Competition}
\authors    {\texorpdfstring
             {\href{mailto:sc22g13@ecs.soton.ac.uk}{Stefan J. Collier}}
             {Stefan J. Collier}
            }
\addresses  {\groupname\\\deptname\\\univname}
\date       {\today}
\subject    {}
\keywords   {}
\supervisor {Dr. Maria Polukarov}
\examiner   {Professor Sheng Chen}

\maketitle
\begin{abstract}
This project aim was to model and analyse the effects of competitive pricing behaviors of grocery retailers on the British market. 

This was achieved by creating a multi-agent model, containing retailer and consumer agents. The heterogeneous crowd of retailers employs either a uniform pricing strategy or a ‘local price flexing’ strategy. The actions of these retailers are chosen by predicting the profit of each action, using a perceptron. Following on from the consideration of different economic models, a discrete model was developed so that software agents have a discrete environment to operate within. Within the model, it has been observed how supermarkets with differing behaviors affect a heterogeneous crowd of consumer agents. The model was implemented in Java with Python used to evaluate the results. 

The simulation displays good acceptance with real grocery market behavior, i.e. captures the performance of British retailers thus can be used to determine the impact of changes in their behavior on their competitors and consumers.Furthermore it can be used to provide insight into sustainability of volatile pricing strategies, providing a useful insight in volatility of British supermarket retail industry. 
\end{abstract}
\acknowledgements{
I would like to express my sincere gratitude to Dr Maria Polukarov for her guidance and support which provided me the freedom to take this research in the direction of my interest.\\
\\
I would also like to thank my family and friends for their encouragement and support. To those who quietly listened to my software complaints. To those who worked throughout the nights with me. To those who helped me write what I couldn't say. I cannot thank you enough.
}

\declaration{
I, Stefan Collier, declare that this dissertation and the work presented in it are my own and has been generated by me as the result of my own original research.\\
I confirm that:\\
1. This work was done wholly or mainly while in candidature for a degree at this University;\\
2. Where any part of this dissertation has previously been submitted for any other qualification at this University or any other institution, this has been clearly stated;\\
3. Where I have consulted the published work of others, this is always clearly attributed;\\
4. Where I have quoted from the work of others, the source is always given. With the exception of such quotations, this dissertation is entirely my own work;\\
5. I have acknowledged all main sources of help;\\
6. Where the thesis is based on work done by myself jointly with others, I have made clear exactly what was done by others and what I have contributed myself;\\
7. Either none of this work has been published before submission, or parts of this work have been published by :\\
\\
Stefan Collier\\
April 2016
}
\tableofcontents
\listoffigures
\listoftables

\mainmatter
%% ----------------------------------------------------------------
%\include{Introduction}
%\include{Conclusions}
\include{chapters/1Project/main}
\include{chapters/2Lit/main}
\include{chapters/3Design/HighLevel}
\include{chapters/3Design/InDepth}
\include{chapters/4Impl/main}

\include{chapters/5Experiments/1/main}
\include{chapters/5Experiments/2/main}
\include{chapters/5Experiments/3/main}
\include{chapters/5Experiments/4/main}

\include{chapters/6Conclusion/main}

\appendix
\include{appendix/AppendixB}
\include{appendix/D/main}
\include{appendix/AppendixC}

\backmatter
\bibliographystyle{ecs}
\bibliography{ECS}
\end{document}
%% ----------------------------------------------------------------

 %% ----------------------------------------------------------------
%% Progress.tex
%% ---------------------------------------------------------------- 
\documentclass{ecsprogress}    % Use the progress Style
\graphicspath{{../figs/}}   % Location of your graphics files
    \usepackage{natbib}            % Use Natbib style for the refs.
\hypersetup{colorlinks=true}   % Set to false for black/white printing
\input{Definitions}            % Include your abbreviations



\usepackage{enumitem}% http://ctan.org/pkg/enumitem
\usepackage{multirow}
\usepackage{float}
\usepackage{amsmath}
\usepackage{multicol}
\usepackage{amssymb}
\usepackage[normalem]{ulem}
\useunder{\uline}{\ul}{}
\usepackage{wrapfig}


\usepackage[table,xcdraw]{xcolor}


%% ----------------------------------------------------------------
\begin{document}
\frontmatter
\title      {Heterogeneous Agent-based Model for Supermarket Competition}
\authors    {\texorpdfstring
             {\href{mailto:sc22g13@ecs.soton.ac.uk}{Stefan J. Collier}}
             {Stefan J. Collier}
            }
\addresses  {\groupname\\\deptname\\\univname}
\date       {\today}
\subject    {}
\keywords   {}
\supervisor {Dr. Maria Polukarov}
\examiner   {Professor Sheng Chen}

\maketitle
\begin{abstract}
This project aim was to model and analyse the effects of competitive pricing behaviors of grocery retailers on the British market. 

This was achieved by creating a multi-agent model, containing retailer and consumer agents. The heterogeneous crowd of retailers employs either a uniform pricing strategy or a ‘local price flexing’ strategy. The actions of these retailers are chosen by predicting the profit of each action, using a perceptron. Following on from the consideration of different economic models, a discrete model was developed so that software agents have a discrete environment to operate within. Within the model, it has been observed how supermarkets with differing behaviors affect a heterogeneous crowd of consumer agents. The model was implemented in Java with Python used to evaluate the results. 

The simulation displays good acceptance with real grocery market behavior, i.e. captures the performance of British retailers thus can be used to determine the impact of changes in their behavior on their competitors and consumers.Furthermore it can be used to provide insight into sustainability of volatile pricing strategies, providing a useful insight in volatility of British supermarket retail industry. 
\end{abstract}
\acknowledgements{
I would like to express my sincere gratitude to Dr Maria Polukarov for her guidance and support which provided me the freedom to take this research in the direction of my interest.\\
\\
I would also like to thank my family and friends for their encouragement and support. To those who quietly listened to my software complaints. To those who worked throughout the nights with me. To those who helped me write what I couldn't say. I cannot thank you enough.
}

\declaration{
I, Stefan Collier, declare that this dissertation and the work presented in it are my own and has been generated by me as the result of my own original research.\\
I confirm that:\\
1. This work was done wholly or mainly while in candidature for a degree at this University;\\
2. Where any part of this dissertation has previously been submitted for any other qualification at this University or any other institution, this has been clearly stated;\\
3. Where I have consulted the published work of others, this is always clearly attributed;\\
4. Where I have quoted from the work of others, the source is always given. With the exception of such quotations, this dissertation is entirely my own work;\\
5. I have acknowledged all main sources of help;\\
6. Where the thesis is based on work done by myself jointly with others, I have made clear exactly what was done by others and what I have contributed myself;\\
7. Either none of this work has been published before submission, or parts of this work have been published by :\\
\\
Stefan Collier\\
April 2016
}
\tableofcontents
\listoffigures
\listoftables

\mainmatter
%% ----------------------------------------------------------------
%\include{Introduction}
%\include{Conclusions}
\include{chapters/1Project/main}
\include{chapters/2Lit/main}
\include{chapters/3Design/HighLevel}
\include{chapters/3Design/InDepth}
\include{chapters/4Impl/main}

\include{chapters/5Experiments/1/main}
\include{chapters/5Experiments/2/main}
\include{chapters/5Experiments/3/main}
\include{chapters/5Experiments/4/main}

\include{chapters/6Conclusion/main}

\appendix
\include{appendix/AppendixB}
\include{appendix/D/main}
\include{appendix/AppendixC}

\backmatter
\bibliographystyle{ecs}
\bibliography{ECS}
\end{document}
%% ----------------------------------------------------------------


 %% ----------------------------------------------------------------
%% Progress.tex
%% ---------------------------------------------------------------- 
\documentclass{ecsprogress}    % Use the progress Style
\graphicspath{{../figs/}}   % Location of your graphics files
    \usepackage{natbib}            % Use Natbib style for the refs.
\hypersetup{colorlinks=true}   % Set to false for black/white printing
\input{Definitions}            % Include your abbreviations



\usepackage{enumitem}% http://ctan.org/pkg/enumitem
\usepackage{multirow}
\usepackage{float}
\usepackage{amsmath}
\usepackage{multicol}
\usepackage{amssymb}
\usepackage[normalem]{ulem}
\useunder{\uline}{\ul}{}
\usepackage{wrapfig}


\usepackage[table,xcdraw]{xcolor}


%% ----------------------------------------------------------------
\begin{document}
\frontmatter
\title      {Heterogeneous Agent-based Model for Supermarket Competition}
\authors    {\texorpdfstring
             {\href{mailto:sc22g13@ecs.soton.ac.uk}{Stefan J. Collier}}
             {Stefan J. Collier}
            }
\addresses  {\groupname\\\deptname\\\univname}
\date       {\today}
\subject    {}
\keywords   {}
\supervisor {Dr. Maria Polukarov}
\examiner   {Professor Sheng Chen}

\maketitle
\begin{abstract}
This project aim was to model and analyse the effects of competitive pricing behaviors of grocery retailers on the British market. 

This was achieved by creating a multi-agent model, containing retailer and consumer agents. The heterogeneous crowd of retailers employs either a uniform pricing strategy or a ‘local price flexing’ strategy. The actions of these retailers are chosen by predicting the profit of each action, using a perceptron. Following on from the consideration of different economic models, a discrete model was developed so that software agents have a discrete environment to operate within. Within the model, it has been observed how supermarkets with differing behaviors affect a heterogeneous crowd of consumer agents. The model was implemented in Java with Python used to evaluate the results. 

The simulation displays good acceptance with real grocery market behavior, i.e. captures the performance of British retailers thus can be used to determine the impact of changes in their behavior on their competitors and consumers.Furthermore it can be used to provide insight into sustainability of volatile pricing strategies, providing a useful insight in volatility of British supermarket retail industry. 
\end{abstract}
\acknowledgements{
I would like to express my sincere gratitude to Dr Maria Polukarov for her guidance and support which provided me the freedom to take this research in the direction of my interest.\\
\\
I would also like to thank my family and friends for their encouragement and support. To those who quietly listened to my software complaints. To those who worked throughout the nights with me. To those who helped me write what I couldn't say. I cannot thank you enough.
}

\declaration{
I, Stefan Collier, declare that this dissertation and the work presented in it are my own and has been generated by me as the result of my own original research.\\
I confirm that:\\
1. This work was done wholly or mainly while in candidature for a degree at this University;\\
2. Where any part of this dissertation has previously been submitted for any other qualification at this University or any other institution, this has been clearly stated;\\
3. Where I have consulted the published work of others, this is always clearly attributed;\\
4. Where I have quoted from the work of others, the source is always given. With the exception of such quotations, this dissertation is entirely my own work;\\
5. I have acknowledged all main sources of help;\\
6. Where the thesis is based on work done by myself jointly with others, I have made clear exactly what was done by others and what I have contributed myself;\\
7. Either none of this work has been published before submission, or parts of this work have been published by :\\
\\
Stefan Collier\\
April 2016
}
\tableofcontents
\listoffigures
\listoftables

\mainmatter
%% ----------------------------------------------------------------
%\include{Introduction}
%\include{Conclusions}
\include{chapters/1Project/main}
\include{chapters/2Lit/main}
\include{chapters/3Design/HighLevel}
\include{chapters/3Design/InDepth}
\include{chapters/4Impl/main}

\include{chapters/5Experiments/1/main}
\include{chapters/5Experiments/2/main}
\include{chapters/5Experiments/3/main}
\include{chapters/5Experiments/4/main}

\include{chapters/6Conclusion/main}

\appendix
\include{appendix/AppendixB}
\include{appendix/D/main}
\include{appendix/AppendixC}

\backmatter
\bibliographystyle{ecs}
\bibliography{ECS}
\end{document}
%% ----------------------------------------------------------------


\appendix
\include{appendix/AppendixB}
 %% ----------------------------------------------------------------
%% Progress.tex
%% ---------------------------------------------------------------- 
\documentclass{ecsprogress}    % Use the progress Style
\graphicspath{{../figs/}}   % Location of your graphics files
    \usepackage{natbib}            % Use Natbib style for the refs.
\hypersetup{colorlinks=true}   % Set to false for black/white printing
\input{Definitions}            % Include your abbreviations



\usepackage{enumitem}% http://ctan.org/pkg/enumitem
\usepackage{multirow}
\usepackage{float}
\usepackage{amsmath}
\usepackage{multicol}
\usepackage{amssymb}
\usepackage[normalem]{ulem}
\useunder{\uline}{\ul}{}
\usepackage{wrapfig}


\usepackage[table,xcdraw]{xcolor}


%% ----------------------------------------------------------------
\begin{document}
\frontmatter
\title      {Heterogeneous Agent-based Model for Supermarket Competition}
\authors    {\texorpdfstring
             {\href{mailto:sc22g13@ecs.soton.ac.uk}{Stefan J. Collier}}
             {Stefan J. Collier}
            }
\addresses  {\groupname\\\deptname\\\univname}
\date       {\today}
\subject    {}
\keywords   {}
\supervisor {Dr. Maria Polukarov}
\examiner   {Professor Sheng Chen}

\maketitle
\begin{abstract}
This project aim was to model and analyse the effects of competitive pricing behaviors of grocery retailers on the British market. 

This was achieved by creating a multi-agent model, containing retailer and consumer agents. The heterogeneous crowd of retailers employs either a uniform pricing strategy or a ‘local price flexing’ strategy. The actions of these retailers are chosen by predicting the profit of each action, using a perceptron. Following on from the consideration of different economic models, a discrete model was developed so that software agents have a discrete environment to operate within. Within the model, it has been observed how supermarkets with differing behaviors affect a heterogeneous crowd of consumer agents. The model was implemented in Java with Python used to evaluate the results. 

The simulation displays good acceptance with real grocery market behavior, i.e. captures the performance of British retailers thus can be used to determine the impact of changes in their behavior on their competitors and consumers.Furthermore it can be used to provide insight into sustainability of volatile pricing strategies, providing a useful insight in volatility of British supermarket retail industry. 
\end{abstract}
\acknowledgements{
I would like to express my sincere gratitude to Dr Maria Polukarov for her guidance and support which provided me the freedom to take this research in the direction of my interest.\\
\\
I would also like to thank my family and friends for their encouragement and support. To those who quietly listened to my software complaints. To those who worked throughout the nights with me. To those who helped me write what I couldn't say. I cannot thank you enough.
}

\declaration{
I, Stefan Collier, declare that this dissertation and the work presented in it are my own and has been generated by me as the result of my own original research.\\
I confirm that:\\
1. This work was done wholly or mainly while in candidature for a degree at this University;\\
2. Where any part of this dissertation has previously been submitted for any other qualification at this University or any other institution, this has been clearly stated;\\
3. Where I have consulted the published work of others, this is always clearly attributed;\\
4. Where I have quoted from the work of others, the source is always given. With the exception of such quotations, this dissertation is entirely my own work;\\
5. I have acknowledged all main sources of help;\\
6. Where the thesis is based on work done by myself jointly with others, I have made clear exactly what was done by others and what I have contributed myself;\\
7. Either none of this work has been published before submission, or parts of this work have been published by :\\
\\
Stefan Collier\\
April 2016
}
\tableofcontents
\listoffigures
\listoftables

\mainmatter
%% ----------------------------------------------------------------
%\include{Introduction}
%\include{Conclusions}
\include{chapters/1Project/main}
\include{chapters/2Lit/main}
\include{chapters/3Design/HighLevel}
\include{chapters/3Design/InDepth}
\include{chapters/4Impl/main}

\include{chapters/5Experiments/1/main}
\include{chapters/5Experiments/2/main}
\include{chapters/5Experiments/3/main}
\include{chapters/5Experiments/4/main}

\include{chapters/6Conclusion/main}

\appendix
\include{appendix/AppendixB}
\include{appendix/D/main}
\include{appendix/AppendixC}

\backmatter
\bibliographystyle{ecs}
\bibliography{ECS}
\end{document}
%% ----------------------------------------------------------------

\include{appendix/AppendixC}

\backmatter
\bibliographystyle{ecs}
\bibliography{ECS}
\end{document}
%% ----------------------------------------------------------------

 %% ----------------------------------------------------------------
%% Progress.tex
%% ---------------------------------------------------------------- 
\documentclass{ecsprogress}    % Use the progress Style
\graphicspath{{../figs/}}   % Location of your graphics files
    \usepackage{natbib}            % Use Natbib style for the refs.
\hypersetup{colorlinks=true}   % Set to false for black/white printing
\input{Definitions}            % Include your abbreviations



\usepackage{enumitem}% http://ctan.org/pkg/enumitem
\usepackage{multirow}
\usepackage{float}
\usepackage{amsmath}
\usepackage{multicol}
\usepackage{amssymb}
\usepackage[normalem]{ulem}
\useunder{\uline}{\ul}{}
\usepackage{wrapfig}


\usepackage[table,xcdraw]{xcolor}


%% ----------------------------------------------------------------
\begin{document}
\frontmatter
\title      {Heterogeneous Agent-based Model for Supermarket Competition}
\authors    {\texorpdfstring
             {\href{mailto:sc22g13@ecs.soton.ac.uk}{Stefan J. Collier}}
             {Stefan J. Collier}
            }
\addresses  {\groupname\\\deptname\\\univname}
\date       {\today}
\subject    {}
\keywords   {}
\supervisor {Dr. Maria Polukarov}
\examiner   {Professor Sheng Chen}

\maketitle
\begin{abstract}
This project aim was to model and analyse the effects of competitive pricing behaviors of grocery retailers on the British market. 

This was achieved by creating a multi-agent model, containing retailer and consumer agents. The heterogeneous crowd of retailers employs either a uniform pricing strategy or a ‘local price flexing’ strategy. The actions of these retailers are chosen by predicting the profit of each action, using a perceptron. Following on from the consideration of different economic models, a discrete model was developed so that software agents have a discrete environment to operate within. Within the model, it has been observed how supermarkets with differing behaviors affect a heterogeneous crowd of consumer agents. The model was implemented in Java with Python used to evaluate the results. 

The simulation displays good acceptance with real grocery market behavior, i.e. captures the performance of British retailers thus can be used to determine the impact of changes in their behavior on their competitors and consumers.Furthermore it can be used to provide insight into sustainability of volatile pricing strategies, providing a useful insight in volatility of British supermarket retail industry. 
\end{abstract}
\acknowledgements{
I would like to express my sincere gratitude to Dr Maria Polukarov for her guidance and support which provided me the freedom to take this research in the direction of my interest.\\
\\
I would also like to thank my family and friends for their encouragement and support. To those who quietly listened to my software complaints. To those who worked throughout the nights with me. To those who helped me write what I couldn't say. I cannot thank you enough.
}

\declaration{
I, Stefan Collier, declare that this dissertation and the work presented in it are my own and has been generated by me as the result of my own original research.\\
I confirm that:\\
1. This work was done wholly or mainly while in candidature for a degree at this University;\\
2. Where any part of this dissertation has previously been submitted for any other qualification at this University or any other institution, this has been clearly stated;\\
3. Where I have consulted the published work of others, this is always clearly attributed;\\
4. Where I have quoted from the work of others, the source is always given. With the exception of such quotations, this dissertation is entirely my own work;\\
5. I have acknowledged all main sources of help;\\
6. Where the thesis is based on work done by myself jointly with others, I have made clear exactly what was done by others and what I have contributed myself;\\
7. Either none of this work has been published before submission, or parts of this work have been published by :\\
\\
Stefan Collier\\
April 2016
}
\tableofcontents
\listoffigures
\listoftables

\mainmatter
%% ----------------------------------------------------------------
%\include{Introduction}
%\include{Conclusions}
 %% ----------------------------------------------------------------
%% Progress.tex
%% ---------------------------------------------------------------- 
\documentclass{ecsprogress}    % Use the progress Style
\graphicspath{{../figs/}}   % Location of your graphics files
    \usepackage{natbib}            % Use Natbib style for the refs.
\hypersetup{colorlinks=true}   % Set to false for black/white printing
\input{Definitions}            % Include your abbreviations



\usepackage{enumitem}% http://ctan.org/pkg/enumitem
\usepackage{multirow}
\usepackage{float}
\usepackage{amsmath}
\usepackage{multicol}
\usepackage{amssymb}
\usepackage[normalem]{ulem}
\useunder{\uline}{\ul}{}
\usepackage{wrapfig}


\usepackage[table,xcdraw]{xcolor}


%% ----------------------------------------------------------------
\begin{document}
\frontmatter
\title      {Heterogeneous Agent-based Model for Supermarket Competition}
\authors    {\texorpdfstring
             {\href{mailto:sc22g13@ecs.soton.ac.uk}{Stefan J. Collier}}
             {Stefan J. Collier}
            }
\addresses  {\groupname\\\deptname\\\univname}
\date       {\today}
\subject    {}
\keywords   {}
\supervisor {Dr. Maria Polukarov}
\examiner   {Professor Sheng Chen}

\maketitle
\begin{abstract}
This project aim was to model and analyse the effects of competitive pricing behaviors of grocery retailers on the British market. 

This was achieved by creating a multi-agent model, containing retailer and consumer agents. The heterogeneous crowd of retailers employs either a uniform pricing strategy or a ‘local price flexing’ strategy. The actions of these retailers are chosen by predicting the profit of each action, using a perceptron. Following on from the consideration of different economic models, a discrete model was developed so that software agents have a discrete environment to operate within. Within the model, it has been observed how supermarkets with differing behaviors affect a heterogeneous crowd of consumer agents. The model was implemented in Java with Python used to evaluate the results. 

The simulation displays good acceptance with real grocery market behavior, i.e. captures the performance of British retailers thus can be used to determine the impact of changes in their behavior on their competitors and consumers.Furthermore it can be used to provide insight into sustainability of volatile pricing strategies, providing a useful insight in volatility of British supermarket retail industry. 
\end{abstract}
\acknowledgements{
I would like to express my sincere gratitude to Dr Maria Polukarov for her guidance and support which provided me the freedom to take this research in the direction of my interest.\\
\\
I would also like to thank my family and friends for their encouragement and support. To those who quietly listened to my software complaints. To those who worked throughout the nights with me. To those who helped me write what I couldn't say. I cannot thank you enough.
}

\declaration{
I, Stefan Collier, declare that this dissertation and the work presented in it are my own and has been generated by me as the result of my own original research.\\
I confirm that:\\
1. This work was done wholly or mainly while in candidature for a degree at this University;\\
2. Where any part of this dissertation has previously been submitted for any other qualification at this University or any other institution, this has been clearly stated;\\
3. Where I have consulted the published work of others, this is always clearly attributed;\\
4. Where I have quoted from the work of others, the source is always given. With the exception of such quotations, this dissertation is entirely my own work;\\
5. I have acknowledged all main sources of help;\\
6. Where the thesis is based on work done by myself jointly with others, I have made clear exactly what was done by others and what I have contributed myself;\\
7. Either none of this work has been published before submission, or parts of this work have been published by :\\
\\
Stefan Collier\\
April 2016
}
\tableofcontents
\listoffigures
\listoftables

\mainmatter
%% ----------------------------------------------------------------
%\include{Introduction}
%\include{Conclusions}
\include{chapters/1Project/main}
\include{chapters/2Lit/main}
\include{chapters/3Design/HighLevel}
\include{chapters/3Design/InDepth}
\include{chapters/4Impl/main}

\include{chapters/5Experiments/1/main}
\include{chapters/5Experiments/2/main}
\include{chapters/5Experiments/3/main}
\include{chapters/5Experiments/4/main}

\include{chapters/6Conclusion/main}

\appendix
\include{appendix/AppendixB}
\include{appendix/D/main}
\include{appendix/AppendixC}

\backmatter
\bibliographystyle{ecs}
\bibliography{ECS}
\end{document}
%% ----------------------------------------------------------------

 %% ----------------------------------------------------------------
%% Progress.tex
%% ---------------------------------------------------------------- 
\documentclass{ecsprogress}    % Use the progress Style
\graphicspath{{../figs/}}   % Location of your graphics files
    \usepackage{natbib}            % Use Natbib style for the refs.
\hypersetup{colorlinks=true}   % Set to false for black/white printing
\input{Definitions}            % Include your abbreviations



\usepackage{enumitem}% http://ctan.org/pkg/enumitem
\usepackage{multirow}
\usepackage{float}
\usepackage{amsmath}
\usepackage{multicol}
\usepackage{amssymb}
\usepackage[normalem]{ulem}
\useunder{\uline}{\ul}{}
\usepackage{wrapfig}


\usepackage[table,xcdraw]{xcolor}


%% ----------------------------------------------------------------
\begin{document}
\frontmatter
\title      {Heterogeneous Agent-based Model for Supermarket Competition}
\authors    {\texorpdfstring
             {\href{mailto:sc22g13@ecs.soton.ac.uk}{Stefan J. Collier}}
             {Stefan J. Collier}
            }
\addresses  {\groupname\\\deptname\\\univname}
\date       {\today}
\subject    {}
\keywords   {}
\supervisor {Dr. Maria Polukarov}
\examiner   {Professor Sheng Chen}

\maketitle
\begin{abstract}
This project aim was to model and analyse the effects of competitive pricing behaviors of grocery retailers on the British market. 

This was achieved by creating a multi-agent model, containing retailer and consumer agents. The heterogeneous crowd of retailers employs either a uniform pricing strategy or a ‘local price flexing’ strategy. The actions of these retailers are chosen by predicting the profit of each action, using a perceptron. Following on from the consideration of different economic models, a discrete model was developed so that software agents have a discrete environment to operate within. Within the model, it has been observed how supermarkets with differing behaviors affect a heterogeneous crowd of consumer agents. The model was implemented in Java with Python used to evaluate the results. 

The simulation displays good acceptance with real grocery market behavior, i.e. captures the performance of British retailers thus can be used to determine the impact of changes in their behavior on their competitors and consumers.Furthermore it can be used to provide insight into sustainability of volatile pricing strategies, providing a useful insight in volatility of British supermarket retail industry. 
\end{abstract}
\acknowledgements{
I would like to express my sincere gratitude to Dr Maria Polukarov for her guidance and support which provided me the freedom to take this research in the direction of my interest.\\
\\
I would also like to thank my family and friends for their encouragement and support. To those who quietly listened to my software complaints. To those who worked throughout the nights with me. To those who helped me write what I couldn't say. I cannot thank you enough.
}

\declaration{
I, Stefan Collier, declare that this dissertation and the work presented in it are my own and has been generated by me as the result of my own original research.\\
I confirm that:\\
1. This work was done wholly or mainly while in candidature for a degree at this University;\\
2. Where any part of this dissertation has previously been submitted for any other qualification at this University or any other institution, this has been clearly stated;\\
3. Where I have consulted the published work of others, this is always clearly attributed;\\
4. Where I have quoted from the work of others, the source is always given. With the exception of such quotations, this dissertation is entirely my own work;\\
5. I have acknowledged all main sources of help;\\
6. Where the thesis is based on work done by myself jointly with others, I have made clear exactly what was done by others and what I have contributed myself;\\
7. Either none of this work has been published before submission, or parts of this work have been published by :\\
\\
Stefan Collier\\
April 2016
}
\tableofcontents
\listoffigures
\listoftables

\mainmatter
%% ----------------------------------------------------------------
%\include{Introduction}
%\include{Conclusions}
\include{chapters/1Project/main}
\include{chapters/2Lit/main}
\include{chapters/3Design/HighLevel}
\include{chapters/3Design/InDepth}
\include{chapters/4Impl/main}

\include{chapters/5Experiments/1/main}
\include{chapters/5Experiments/2/main}
\include{chapters/5Experiments/3/main}
\include{chapters/5Experiments/4/main}

\include{chapters/6Conclusion/main}

\appendix
\include{appendix/AppendixB}
\include{appendix/D/main}
\include{appendix/AppendixC}

\backmatter
\bibliographystyle{ecs}
\bibliography{ECS}
\end{document}
%% ----------------------------------------------------------------

\include{chapters/3Design/HighLevel}
\include{chapters/3Design/InDepth}
 %% ----------------------------------------------------------------
%% Progress.tex
%% ---------------------------------------------------------------- 
\documentclass{ecsprogress}    % Use the progress Style
\graphicspath{{../figs/}}   % Location of your graphics files
    \usepackage{natbib}            % Use Natbib style for the refs.
\hypersetup{colorlinks=true}   % Set to false for black/white printing
\input{Definitions}            % Include your abbreviations



\usepackage{enumitem}% http://ctan.org/pkg/enumitem
\usepackage{multirow}
\usepackage{float}
\usepackage{amsmath}
\usepackage{multicol}
\usepackage{amssymb}
\usepackage[normalem]{ulem}
\useunder{\uline}{\ul}{}
\usepackage{wrapfig}


\usepackage[table,xcdraw]{xcolor}


%% ----------------------------------------------------------------
\begin{document}
\frontmatter
\title      {Heterogeneous Agent-based Model for Supermarket Competition}
\authors    {\texorpdfstring
             {\href{mailto:sc22g13@ecs.soton.ac.uk}{Stefan J. Collier}}
             {Stefan J. Collier}
            }
\addresses  {\groupname\\\deptname\\\univname}
\date       {\today}
\subject    {}
\keywords   {}
\supervisor {Dr. Maria Polukarov}
\examiner   {Professor Sheng Chen}

\maketitle
\begin{abstract}
This project aim was to model and analyse the effects of competitive pricing behaviors of grocery retailers on the British market. 

This was achieved by creating a multi-agent model, containing retailer and consumer agents. The heterogeneous crowd of retailers employs either a uniform pricing strategy or a ‘local price flexing’ strategy. The actions of these retailers are chosen by predicting the profit of each action, using a perceptron. Following on from the consideration of different economic models, a discrete model was developed so that software agents have a discrete environment to operate within. Within the model, it has been observed how supermarkets with differing behaviors affect a heterogeneous crowd of consumer agents. The model was implemented in Java with Python used to evaluate the results. 

The simulation displays good acceptance with real grocery market behavior, i.e. captures the performance of British retailers thus can be used to determine the impact of changes in their behavior on their competitors and consumers.Furthermore it can be used to provide insight into sustainability of volatile pricing strategies, providing a useful insight in volatility of British supermarket retail industry. 
\end{abstract}
\acknowledgements{
I would like to express my sincere gratitude to Dr Maria Polukarov for her guidance and support which provided me the freedom to take this research in the direction of my interest.\\
\\
I would also like to thank my family and friends for their encouragement and support. To those who quietly listened to my software complaints. To those who worked throughout the nights with me. To those who helped me write what I couldn't say. I cannot thank you enough.
}

\declaration{
I, Stefan Collier, declare that this dissertation and the work presented in it are my own and has been generated by me as the result of my own original research.\\
I confirm that:\\
1. This work was done wholly or mainly while in candidature for a degree at this University;\\
2. Where any part of this dissertation has previously been submitted for any other qualification at this University or any other institution, this has been clearly stated;\\
3. Where I have consulted the published work of others, this is always clearly attributed;\\
4. Where I have quoted from the work of others, the source is always given. With the exception of such quotations, this dissertation is entirely my own work;\\
5. I have acknowledged all main sources of help;\\
6. Where the thesis is based on work done by myself jointly with others, I have made clear exactly what was done by others and what I have contributed myself;\\
7. Either none of this work has been published before submission, or parts of this work have been published by :\\
\\
Stefan Collier\\
April 2016
}
\tableofcontents
\listoffigures
\listoftables

\mainmatter
%% ----------------------------------------------------------------
%\include{Introduction}
%\include{Conclusions}
\include{chapters/1Project/main}
\include{chapters/2Lit/main}
\include{chapters/3Design/HighLevel}
\include{chapters/3Design/InDepth}
\include{chapters/4Impl/main}

\include{chapters/5Experiments/1/main}
\include{chapters/5Experiments/2/main}
\include{chapters/5Experiments/3/main}
\include{chapters/5Experiments/4/main}

\include{chapters/6Conclusion/main}

\appendix
\include{appendix/AppendixB}
\include{appendix/D/main}
\include{appendix/AppendixC}

\backmatter
\bibliographystyle{ecs}
\bibliography{ECS}
\end{document}
%% ----------------------------------------------------------------


 %% ----------------------------------------------------------------
%% Progress.tex
%% ---------------------------------------------------------------- 
\documentclass{ecsprogress}    % Use the progress Style
\graphicspath{{../figs/}}   % Location of your graphics files
    \usepackage{natbib}            % Use Natbib style for the refs.
\hypersetup{colorlinks=true}   % Set to false for black/white printing
\input{Definitions}            % Include your abbreviations



\usepackage{enumitem}% http://ctan.org/pkg/enumitem
\usepackage{multirow}
\usepackage{float}
\usepackage{amsmath}
\usepackage{multicol}
\usepackage{amssymb}
\usepackage[normalem]{ulem}
\useunder{\uline}{\ul}{}
\usepackage{wrapfig}


\usepackage[table,xcdraw]{xcolor}


%% ----------------------------------------------------------------
\begin{document}
\frontmatter
\title      {Heterogeneous Agent-based Model for Supermarket Competition}
\authors    {\texorpdfstring
             {\href{mailto:sc22g13@ecs.soton.ac.uk}{Stefan J. Collier}}
             {Stefan J. Collier}
            }
\addresses  {\groupname\\\deptname\\\univname}
\date       {\today}
\subject    {}
\keywords   {}
\supervisor {Dr. Maria Polukarov}
\examiner   {Professor Sheng Chen}

\maketitle
\begin{abstract}
This project aim was to model and analyse the effects of competitive pricing behaviors of grocery retailers on the British market. 

This was achieved by creating a multi-agent model, containing retailer and consumer agents. The heterogeneous crowd of retailers employs either a uniform pricing strategy or a ‘local price flexing’ strategy. The actions of these retailers are chosen by predicting the profit of each action, using a perceptron. Following on from the consideration of different economic models, a discrete model was developed so that software agents have a discrete environment to operate within. Within the model, it has been observed how supermarkets with differing behaviors affect a heterogeneous crowd of consumer agents. The model was implemented in Java with Python used to evaluate the results. 

The simulation displays good acceptance with real grocery market behavior, i.e. captures the performance of British retailers thus can be used to determine the impact of changes in their behavior on their competitors and consumers.Furthermore it can be used to provide insight into sustainability of volatile pricing strategies, providing a useful insight in volatility of British supermarket retail industry. 
\end{abstract}
\acknowledgements{
I would like to express my sincere gratitude to Dr Maria Polukarov for her guidance and support which provided me the freedom to take this research in the direction of my interest.\\
\\
I would also like to thank my family and friends for their encouragement and support. To those who quietly listened to my software complaints. To those who worked throughout the nights with me. To those who helped me write what I couldn't say. I cannot thank you enough.
}

\declaration{
I, Stefan Collier, declare that this dissertation and the work presented in it are my own and has been generated by me as the result of my own original research.\\
I confirm that:\\
1. This work was done wholly or mainly while in candidature for a degree at this University;\\
2. Where any part of this dissertation has previously been submitted for any other qualification at this University or any other institution, this has been clearly stated;\\
3. Where I have consulted the published work of others, this is always clearly attributed;\\
4. Where I have quoted from the work of others, the source is always given. With the exception of such quotations, this dissertation is entirely my own work;\\
5. I have acknowledged all main sources of help;\\
6. Where the thesis is based on work done by myself jointly with others, I have made clear exactly what was done by others and what I have contributed myself;\\
7. Either none of this work has been published before submission, or parts of this work have been published by :\\
\\
Stefan Collier\\
April 2016
}
\tableofcontents
\listoffigures
\listoftables

\mainmatter
%% ----------------------------------------------------------------
%\include{Introduction}
%\include{Conclusions}
\include{chapters/1Project/main}
\include{chapters/2Lit/main}
\include{chapters/3Design/HighLevel}
\include{chapters/3Design/InDepth}
\include{chapters/4Impl/main}

\include{chapters/5Experiments/1/main}
\include{chapters/5Experiments/2/main}
\include{chapters/5Experiments/3/main}
\include{chapters/5Experiments/4/main}

\include{chapters/6Conclusion/main}

\appendix
\include{appendix/AppendixB}
\include{appendix/D/main}
\include{appendix/AppendixC}

\backmatter
\bibliographystyle{ecs}
\bibliography{ECS}
\end{document}
%% ----------------------------------------------------------------

 %% ----------------------------------------------------------------
%% Progress.tex
%% ---------------------------------------------------------------- 
\documentclass{ecsprogress}    % Use the progress Style
\graphicspath{{../figs/}}   % Location of your graphics files
    \usepackage{natbib}            % Use Natbib style for the refs.
\hypersetup{colorlinks=true}   % Set to false for black/white printing
\input{Definitions}            % Include your abbreviations



\usepackage{enumitem}% http://ctan.org/pkg/enumitem
\usepackage{multirow}
\usepackage{float}
\usepackage{amsmath}
\usepackage{multicol}
\usepackage{amssymb}
\usepackage[normalem]{ulem}
\useunder{\uline}{\ul}{}
\usepackage{wrapfig}


\usepackage[table,xcdraw]{xcolor}


%% ----------------------------------------------------------------
\begin{document}
\frontmatter
\title      {Heterogeneous Agent-based Model for Supermarket Competition}
\authors    {\texorpdfstring
             {\href{mailto:sc22g13@ecs.soton.ac.uk}{Stefan J. Collier}}
             {Stefan J. Collier}
            }
\addresses  {\groupname\\\deptname\\\univname}
\date       {\today}
\subject    {}
\keywords   {}
\supervisor {Dr. Maria Polukarov}
\examiner   {Professor Sheng Chen}

\maketitle
\begin{abstract}
This project aim was to model and analyse the effects of competitive pricing behaviors of grocery retailers on the British market. 

This was achieved by creating a multi-agent model, containing retailer and consumer agents. The heterogeneous crowd of retailers employs either a uniform pricing strategy or a ‘local price flexing’ strategy. The actions of these retailers are chosen by predicting the profit of each action, using a perceptron. Following on from the consideration of different economic models, a discrete model was developed so that software agents have a discrete environment to operate within. Within the model, it has been observed how supermarkets with differing behaviors affect a heterogeneous crowd of consumer agents. The model was implemented in Java with Python used to evaluate the results. 

The simulation displays good acceptance with real grocery market behavior, i.e. captures the performance of British retailers thus can be used to determine the impact of changes in their behavior on their competitors and consumers.Furthermore it can be used to provide insight into sustainability of volatile pricing strategies, providing a useful insight in volatility of British supermarket retail industry. 
\end{abstract}
\acknowledgements{
I would like to express my sincere gratitude to Dr Maria Polukarov for her guidance and support which provided me the freedom to take this research in the direction of my interest.\\
\\
I would also like to thank my family and friends for their encouragement and support. To those who quietly listened to my software complaints. To those who worked throughout the nights with me. To those who helped me write what I couldn't say. I cannot thank you enough.
}

\declaration{
I, Stefan Collier, declare that this dissertation and the work presented in it are my own and has been generated by me as the result of my own original research.\\
I confirm that:\\
1. This work was done wholly or mainly while in candidature for a degree at this University;\\
2. Where any part of this dissertation has previously been submitted for any other qualification at this University or any other institution, this has been clearly stated;\\
3. Where I have consulted the published work of others, this is always clearly attributed;\\
4. Where I have quoted from the work of others, the source is always given. With the exception of such quotations, this dissertation is entirely my own work;\\
5. I have acknowledged all main sources of help;\\
6. Where the thesis is based on work done by myself jointly with others, I have made clear exactly what was done by others and what I have contributed myself;\\
7. Either none of this work has been published before submission, or parts of this work have been published by :\\
\\
Stefan Collier\\
April 2016
}
\tableofcontents
\listoffigures
\listoftables

\mainmatter
%% ----------------------------------------------------------------
%\include{Introduction}
%\include{Conclusions}
\include{chapters/1Project/main}
\include{chapters/2Lit/main}
\include{chapters/3Design/HighLevel}
\include{chapters/3Design/InDepth}
\include{chapters/4Impl/main}

\include{chapters/5Experiments/1/main}
\include{chapters/5Experiments/2/main}
\include{chapters/5Experiments/3/main}
\include{chapters/5Experiments/4/main}

\include{chapters/6Conclusion/main}

\appendix
\include{appendix/AppendixB}
\include{appendix/D/main}
\include{appendix/AppendixC}

\backmatter
\bibliographystyle{ecs}
\bibliography{ECS}
\end{document}
%% ----------------------------------------------------------------

 %% ----------------------------------------------------------------
%% Progress.tex
%% ---------------------------------------------------------------- 
\documentclass{ecsprogress}    % Use the progress Style
\graphicspath{{../figs/}}   % Location of your graphics files
    \usepackage{natbib}            % Use Natbib style for the refs.
\hypersetup{colorlinks=true}   % Set to false for black/white printing
\input{Definitions}            % Include your abbreviations



\usepackage{enumitem}% http://ctan.org/pkg/enumitem
\usepackage{multirow}
\usepackage{float}
\usepackage{amsmath}
\usepackage{multicol}
\usepackage{amssymb}
\usepackage[normalem]{ulem}
\useunder{\uline}{\ul}{}
\usepackage{wrapfig}


\usepackage[table,xcdraw]{xcolor}


%% ----------------------------------------------------------------
\begin{document}
\frontmatter
\title      {Heterogeneous Agent-based Model for Supermarket Competition}
\authors    {\texorpdfstring
             {\href{mailto:sc22g13@ecs.soton.ac.uk}{Stefan J. Collier}}
             {Stefan J. Collier}
            }
\addresses  {\groupname\\\deptname\\\univname}
\date       {\today}
\subject    {}
\keywords   {}
\supervisor {Dr. Maria Polukarov}
\examiner   {Professor Sheng Chen}

\maketitle
\begin{abstract}
This project aim was to model and analyse the effects of competitive pricing behaviors of grocery retailers on the British market. 

This was achieved by creating a multi-agent model, containing retailer and consumer agents. The heterogeneous crowd of retailers employs either a uniform pricing strategy or a ‘local price flexing’ strategy. The actions of these retailers are chosen by predicting the profit of each action, using a perceptron. Following on from the consideration of different economic models, a discrete model was developed so that software agents have a discrete environment to operate within. Within the model, it has been observed how supermarkets with differing behaviors affect a heterogeneous crowd of consumer agents. The model was implemented in Java with Python used to evaluate the results. 

The simulation displays good acceptance with real grocery market behavior, i.e. captures the performance of British retailers thus can be used to determine the impact of changes in their behavior on their competitors and consumers.Furthermore it can be used to provide insight into sustainability of volatile pricing strategies, providing a useful insight in volatility of British supermarket retail industry. 
\end{abstract}
\acknowledgements{
I would like to express my sincere gratitude to Dr Maria Polukarov for her guidance and support which provided me the freedom to take this research in the direction of my interest.\\
\\
I would also like to thank my family and friends for their encouragement and support. To those who quietly listened to my software complaints. To those who worked throughout the nights with me. To those who helped me write what I couldn't say. I cannot thank you enough.
}

\declaration{
I, Stefan Collier, declare that this dissertation and the work presented in it are my own and has been generated by me as the result of my own original research.\\
I confirm that:\\
1. This work was done wholly or mainly while in candidature for a degree at this University;\\
2. Where any part of this dissertation has previously been submitted for any other qualification at this University or any other institution, this has been clearly stated;\\
3. Where I have consulted the published work of others, this is always clearly attributed;\\
4. Where I have quoted from the work of others, the source is always given. With the exception of such quotations, this dissertation is entirely my own work;\\
5. I have acknowledged all main sources of help;\\
6. Where the thesis is based on work done by myself jointly with others, I have made clear exactly what was done by others and what I have contributed myself;\\
7. Either none of this work has been published before submission, or parts of this work have been published by :\\
\\
Stefan Collier\\
April 2016
}
\tableofcontents
\listoffigures
\listoftables

\mainmatter
%% ----------------------------------------------------------------
%\include{Introduction}
%\include{Conclusions}
\include{chapters/1Project/main}
\include{chapters/2Lit/main}
\include{chapters/3Design/HighLevel}
\include{chapters/3Design/InDepth}
\include{chapters/4Impl/main}

\include{chapters/5Experiments/1/main}
\include{chapters/5Experiments/2/main}
\include{chapters/5Experiments/3/main}
\include{chapters/5Experiments/4/main}

\include{chapters/6Conclusion/main}

\appendix
\include{appendix/AppendixB}
\include{appendix/D/main}
\include{appendix/AppendixC}

\backmatter
\bibliographystyle{ecs}
\bibliography{ECS}
\end{document}
%% ----------------------------------------------------------------

 %% ----------------------------------------------------------------
%% Progress.tex
%% ---------------------------------------------------------------- 
\documentclass{ecsprogress}    % Use the progress Style
\graphicspath{{../figs/}}   % Location of your graphics files
    \usepackage{natbib}            % Use Natbib style for the refs.
\hypersetup{colorlinks=true}   % Set to false for black/white printing
\input{Definitions}            % Include your abbreviations



\usepackage{enumitem}% http://ctan.org/pkg/enumitem
\usepackage{multirow}
\usepackage{float}
\usepackage{amsmath}
\usepackage{multicol}
\usepackage{amssymb}
\usepackage[normalem]{ulem}
\useunder{\uline}{\ul}{}
\usepackage{wrapfig}


\usepackage[table,xcdraw]{xcolor}


%% ----------------------------------------------------------------
\begin{document}
\frontmatter
\title      {Heterogeneous Agent-based Model for Supermarket Competition}
\authors    {\texorpdfstring
             {\href{mailto:sc22g13@ecs.soton.ac.uk}{Stefan J. Collier}}
             {Stefan J. Collier}
            }
\addresses  {\groupname\\\deptname\\\univname}
\date       {\today}
\subject    {}
\keywords   {}
\supervisor {Dr. Maria Polukarov}
\examiner   {Professor Sheng Chen}

\maketitle
\begin{abstract}
This project aim was to model and analyse the effects of competitive pricing behaviors of grocery retailers on the British market. 

This was achieved by creating a multi-agent model, containing retailer and consumer agents. The heterogeneous crowd of retailers employs either a uniform pricing strategy or a ‘local price flexing’ strategy. The actions of these retailers are chosen by predicting the profit of each action, using a perceptron. Following on from the consideration of different economic models, a discrete model was developed so that software agents have a discrete environment to operate within. Within the model, it has been observed how supermarkets with differing behaviors affect a heterogeneous crowd of consumer agents. The model was implemented in Java with Python used to evaluate the results. 

The simulation displays good acceptance with real grocery market behavior, i.e. captures the performance of British retailers thus can be used to determine the impact of changes in their behavior on their competitors and consumers.Furthermore it can be used to provide insight into sustainability of volatile pricing strategies, providing a useful insight in volatility of British supermarket retail industry. 
\end{abstract}
\acknowledgements{
I would like to express my sincere gratitude to Dr Maria Polukarov for her guidance and support which provided me the freedom to take this research in the direction of my interest.\\
\\
I would also like to thank my family and friends for their encouragement and support. To those who quietly listened to my software complaints. To those who worked throughout the nights with me. To those who helped me write what I couldn't say. I cannot thank you enough.
}

\declaration{
I, Stefan Collier, declare that this dissertation and the work presented in it are my own and has been generated by me as the result of my own original research.\\
I confirm that:\\
1. This work was done wholly or mainly while in candidature for a degree at this University;\\
2. Where any part of this dissertation has previously been submitted for any other qualification at this University or any other institution, this has been clearly stated;\\
3. Where I have consulted the published work of others, this is always clearly attributed;\\
4. Where I have quoted from the work of others, the source is always given. With the exception of such quotations, this dissertation is entirely my own work;\\
5. I have acknowledged all main sources of help;\\
6. Where the thesis is based on work done by myself jointly with others, I have made clear exactly what was done by others and what I have contributed myself;\\
7. Either none of this work has been published before submission, or parts of this work have been published by :\\
\\
Stefan Collier\\
April 2016
}
\tableofcontents
\listoffigures
\listoftables

\mainmatter
%% ----------------------------------------------------------------
%\include{Introduction}
%\include{Conclusions}
\include{chapters/1Project/main}
\include{chapters/2Lit/main}
\include{chapters/3Design/HighLevel}
\include{chapters/3Design/InDepth}
\include{chapters/4Impl/main}

\include{chapters/5Experiments/1/main}
\include{chapters/5Experiments/2/main}
\include{chapters/5Experiments/3/main}
\include{chapters/5Experiments/4/main}

\include{chapters/6Conclusion/main}

\appendix
\include{appendix/AppendixB}
\include{appendix/D/main}
\include{appendix/AppendixC}

\backmatter
\bibliographystyle{ecs}
\bibliography{ECS}
\end{document}
%% ----------------------------------------------------------------


 %% ----------------------------------------------------------------
%% Progress.tex
%% ---------------------------------------------------------------- 
\documentclass{ecsprogress}    % Use the progress Style
\graphicspath{{../figs/}}   % Location of your graphics files
    \usepackage{natbib}            % Use Natbib style for the refs.
\hypersetup{colorlinks=true}   % Set to false for black/white printing
\input{Definitions}            % Include your abbreviations



\usepackage{enumitem}% http://ctan.org/pkg/enumitem
\usepackage{multirow}
\usepackage{float}
\usepackage{amsmath}
\usepackage{multicol}
\usepackage{amssymb}
\usepackage[normalem]{ulem}
\useunder{\uline}{\ul}{}
\usepackage{wrapfig}


\usepackage[table,xcdraw]{xcolor}


%% ----------------------------------------------------------------
\begin{document}
\frontmatter
\title      {Heterogeneous Agent-based Model for Supermarket Competition}
\authors    {\texorpdfstring
             {\href{mailto:sc22g13@ecs.soton.ac.uk}{Stefan J. Collier}}
             {Stefan J. Collier}
            }
\addresses  {\groupname\\\deptname\\\univname}
\date       {\today}
\subject    {}
\keywords   {}
\supervisor {Dr. Maria Polukarov}
\examiner   {Professor Sheng Chen}

\maketitle
\begin{abstract}
This project aim was to model and analyse the effects of competitive pricing behaviors of grocery retailers on the British market. 

This was achieved by creating a multi-agent model, containing retailer and consumer agents. The heterogeneous crowd of retailers employs either a uniform pricing strategy or a ‘local price flexing’ strategy. The actions of these retailers are chosen by predicting the profit of each action, using a perceptron. Following on from the consideration of different economic models, a discrete model was developed so that software agents have a discrete environment to operate within. Within the model, it has been observed how supermarkets with differing behaviors affect a heterogeneous crowd of consumer agents. The model was implemented in Java with Python used to evaluate the results. 

The simulation displays good acceptance with real grocery market behavior, i.e. captures the performance of British retailers thus can be used to determine the impact of changes in their behavior on their competitors and consumers.Furthermore it can be used to provide insight into sustainability of volatile pricing strategies, providing a useful insight in volatility of British supermarket retail industry. 
\end{abstract}
\acknowledgements{
I would like to express my sincere gratitude to Dr Maria Polukarov for her guidance and support which provided me the freedom to take this research in the direction of my interest.\\
\\
I would also like to thank my family and friends for their encouragement and support. To those who quietly listened to my software complaints. To those who worked throughout the nights with me. To those who helped me write what I couldn't say. I cannot thank you enough.
}

\declaration{
I, Stefan Collier, declare that this dissertation and the work presented in it are my own and has been generated by me as the result of my own original research.\\
I confirm that:\\
1. This work was done wholly or mainly while in candidature for a degree at this University;\\
2. Where any part of this dissertation has previously been submitted for any other qualification at this University or any other institution, this has been clearly stated;\\
3. Where I have consulted the published work of others, this is always clearly attributed;\\
4. Where I have quoted from the work of others, the source is always given. With the exception of such quotations, this dissertation is entirely my own work;\\
5. I have acknowledged all main sources of help;\\
6. Where the thesis is based on work done by myself jointly with others, I have made clear exactly what was done by others and what I have contributed myself;\\
7. Either none of this work has been published before submission, or parts of this work have been published by :\\
\\
Stefan Collier\\
April 2016
}
\tableofcontents
\listoffigures
\listoftables

\mainmatter
%% ----------------------------------------------------------------
%\include{Introduction}
%\include{Conclusions}
\include{chapters/1Project/main}
\include{chapters/2Lit/main}
\include{chapters/3Design/HighLevel}
\include{chapters/3Design/InDepth}
\include{chapters/4Impl/main}

\include{chapters/5Experiments/1/main}
\include{chapters/5Experiments/2/main}
\include{chapters/5Experiments/3/main}
\include{chapters/5Experiments/4/main}

\include{chapters/6Conclusion/main}

\appendix
\include{appendix/AppendixB}
\include{appendix/D/main}
\include{appendix/AppendixC}

\backmatter
\bibliographystyle{ecs}
\bibliography{ECS}
\end{document}
%% ----------------------------------------------------------------


\appendix
\include{appendix/AppendixB}
 %% ----------------------------------------------------------------
%% Progress.tex
%% ---------------------------------------------------------------- 
\documentclass{ecsprogress}    % Use the progress Style
\graphicspath{{../figs/}}   % Location of your graphics files
    \usepackage{natbib}            % Use Natbib style for the refs.
\hypersetup{colorlinks=true}   % Set to false for black/white printing
\input{Definitions}            % Include your abbreviations



\usepackage{enumitem}% http://ctan.org/pkg/enumitem
\usepackage{multirow}
\usepackage{float}
\usepackage{amsmath}
\usepackage{multicol}
\usepackage{amssymb}
\usepackage[normalem]{ulem}
\useunder{\uline}{\ul}{}
\usepackage{wrapfig}


\usepackage[table,xcdraw]{xcolor}


%% ----------------------------------------------------------------
\begin{document}
\frontmatter
\title      {Heterogeneous Agent-based Model for Supermarket Competition}
\authors    {\texorpdfstring
             {\href{mailto:sc22g13@ecs.soton.ac.uk}{Stefan J. Collier}}
             {Stefan J. Collier}
            }
\addresses  {\groupname\\\deptname\\\univname}
\date       {\today}
\subject    {}
\keywords   {}
\supervisor {Dr. Maria Polukarov}
\examiner   {Professor Sheng Chen}

\maketitle
\begin{abstract}
This project aim was to model and analyse the effects of competitive pricing behaviors of grocery retailers on the British market. 

This was achieved by creating a multi-agent model, containing retailer and consumer agents. The heterogeneous crowd of retailers employs either a uniform pricing strategy or a ‘local price flexing’ strategy. The actions of these retailers are chosen by predicting the profit of each action, using a perceptron. Following on from the consideration of different economic models, a discrete model was developed so that software agents have a discrete environment to operate within. Within the model, it has been observed how supermarkets with differing behaviors affect a heterogeneous crowd of consumer agents. The model was implemented in Java with Python used to evaluate the results. 

The simulation displays good acceptance with real grocery market behavior, i.e. captures the performance of British retailers thus can be used to determine the impact of changes in their behavior on their competitors and consumers.Furthermore it can be used to provide insight into sustainability of volatile pricing strategies, providing a useful insight in volatility of British supermarket retail industry. 
\end{abstract}
\acknowledgements{
I would like to express my sincere gratitude to Dr Maria Polukarov for her guidance and support which provided me the freedom to take this research in the direction of my interest.\\
\\
I would also like to thank my family and friends for their encouragement and support. To those who quietly listened to my software complaints. To those who worked throughout the nights with me. To those who helped me write what I couldn't say. I cannot thank you enough.
}

\declaration{
I, Stefan Collier, declare that this dissertation and the work presented in it are my own and has been generated by me as the result of my own original research.\\
I confirm that:\\
1. This work was done wholly or mainly while in candidature for a degree at this University;\\
2. Where any part of this dissertation has previously been submitted for any other qualification at this University or any other institution, this has been clearly stated;\\
3. Where I have consulted the published work of others, this is always clearly attributed;\\
4. Where I have quoted from the work of others, the source is always given. With the exception of such quotations, this dissertation is entirely my own work;\\
5. I have acknowledged all main sources of help;\\
6. Where the thesis is based on work done by myself jointly with others, I have made clear exactly what was done by others and what I have contributed myself;\\
7. Either none of this work has been published before submission, or parts of this work have been published by :\\
\\
Stefan Collier\\
April 2016
}
\tableofcontents
\listoffigures
\listoftables

\mainmatter
%% ----------------------------------------------------------------
%\include{Introduction}
%\include{Conclusions}
\include{chapters/1Project/main}
\include{chapters/2Lit/main}
\include{chapters/3Design/HighLevel}
\include{chapters/3Design/InDepth}
\include{chapters/4Impl/main}

\include{chapters/5Experiments/1/main}
\include{chapters/5Experiments/2/main}
\include{chapters/5Experiments/3/main}
\include{chapters/5Experiments/4/main}

\include{chapters/6Conclusion/main}

\appendix
\include{appendix/AppendixB}
\include{appendix/D/main}
\include{appendix/AppendixC}

\backmatter
\bibliographystyle{ecs}
\bibliography{ECS}
\end{document}
%% ----------------------------------------------------------------

\include{appendix/AppendixC}

\backmatter
\bibliographystyle{ecs}
\bibliography{ECS}
\end{document}
%% ----------------------------------------------------------------

\include{chapters/3Design/HighLevel}
\include{chapters/3Design/InDepth}
 %% ----------------------------------------------------------------
%% Progress.tex
%% ---------------------------------------------------------------- 
\documentclass{ecsprogress}    % Use the progress Style
\graphicspath{{../figs/}}   % Location of your graphics files
    \usepackage{natbib}            % Use Natbib style for the refs.
\hypersetup{colorlinks=true}   % Set to false for black/white printing
\input{Definitions}            % Include your abbreviations



\usepackage{enumitem}% http://ctan.org/pkg/enumitem
\usepackage{multirow}
\usepackage{float}
\usepackage{amsmath}
\usepackage{multicol}
\usepackage{amssymb}
\usepackage[normalem]{ulem}
\useunder{\uline}{\ul}{}
\usepackage{wrapfig}


\usepackage[table,xcdraw]{xcolor}


%% ----------------------------------------------------------------
\begin{document}
\frontmatter
\title      {Heterogeneous Agent-based Model for Supermarket Competition}
\authors    {\texorpdfstring
             {\href{mailto:sc22g13@ecs.soton.ac.uk}{Stefan J. Collier}}
             {Stefan J. Collier}
            }
\addresses  {\groupname\\\deptname\\\univname}
\date       {\today}
\subject    {}
\keywords   {}
\supervisor {Dr. Maria Polukarov}
\examiner   {Professor Sheng Chen}

\maketitle
\begin{abstract}
This project aim was to model and analyse the effects of competitive pricing behaviors of grocery retailers on the British market. 

This was achieved by creating a multi-agent model, containing retailer and consumer agents. The heterogeneous crowd of retailers employs either a uniform pricing strategy or a ‘local price flexing’ strategy. The actions of these retailers are chosen by predicting the profit of each action, using a perceptron. Following on from the consideration of different economic models, a discrete model was developed so that software agents have a discrete environment to operate within. Within the model, it has been observed how supermarkets with differing behaviors affect a heterogeneous crowd of consumer agents. The model was implemented in Java with Python used to evaluate the results. 

The simulation displays good acceptance with real grocery market behavior, i.e. captures the performance of British retailers thus can be used to determine the impact of changes in their behavior on their competitors and consumers.Furthermore it can be used to provide insight into sustainability of volatile pricing strategies, providing a useful insight in volatility of British supermarket retail industry. 
\end{abstract}
\acknowledgements{
I would like to express my sincere gratitude to Dr Maria Polukarov for her guidance and support which provided me the freedom to take this research in the direction of my interest.\\
\\
I would also like to thank my family and friends for their encouragement and support. To those who quietly listened to my software complaints. To those who worked throughout the nights with me. To those who helped me write what I couldn't say. I cannot thank you enough.
}

\declaration{
I, Stefan Collier, declare that this dissertation and the work presented in it are my own and has been generated by me as the result of my own original research.\\
I confirm that:\\
1. This work was done wholly or mainly while in candidature for a degree at this University;\\
2. Where any part of this dissertation has previously been submitted for any other qualification at this University or any other institution, this has been clearly stated;\\
3. Where I have consulted the published work of others, this is always clearly attributed;\\
4. Where I have quoted from the work of others, the source is always given. With the exception of such quotations, this dissertation is entirely my own work;\\
5. I have acknowledged all main sources of help;\\
6. Where the thesis is based on work done by myself jointly with others, I have made clear exactly what was done by others and what I have contributed myself;\\
7. Either none of this work has been published before submission, or parts of this work have been published by :\\
\\
Stefan Collier\\
April 2016
}
\tableofcontents
\listoffigures
\listoftables

\mainmatter
%% ----------------------------------------------------------------
%\include{Introduction}
%\include{Conclusions}
 %% ----------------------------------------------------------------
%% Progress.tex
%% ---------------------------------------------------------------- 
\documentclass{ecsprogress}    % Use the progress Style
\graphicspath{{../figs/}}   % Location of your graphics files
    \usepackage{natbib}            % Use Natbib style for the refs.
\hypersetup{colorlinks=true}   % Set to false for black/white printing
\input{Definitions}            % Include your abbreviations



\usepackage{enumitem}% http://ctan.org/pkg/enumitem
\usepackage{multirow}
\usepackage{float}
\usepackage{amsmath}
\usepackage{multicol}
\usepackage{amssymb}
\usepackage[normalem]{ulem}
\useunder{\uline}{\ul}{}
\usepackage{wrapfig}


\usepackage[table,xcdraw]{xcolor}


%% ----------------------------------------------------------------
\begin{document}
\frontmatter
\title      {Heterogeneous Agent-based Model for Supermarket Competition}
\authors    {\texorpdfstring
             {\href{mailto:sc22g13@ecs.soton.ac.uk}{Stefan J. Collier}}
             {Stefan J. Collier}
            }
\addresses  {\groupname\\\deptname\\\univname}
\date       {\today}
\subject    {}
\keywords   {}
\supervisor {Dr. Maria Polukarov}
\examiner   {Professor Sheng Chen}

\maketitle
\begin{abstract}
This project aim was to model and analyse the effects of competitive pricing behaviors of grocery retailers on the British market. 

This was achieved by creating a multi-agent model, containing retailer and consumer agents. The heterogeneous crowd of retailers employs either a uniform pricing strategy or a ‘local price flexing’ strategy. The actions of these retailers are chosen by predicting the profit of each action, using a perceptron. Following on from the consideration of different economic models, a discrete model was developed so that software agents have a discrete environment to operate within. Within the model, it has been observed how supermarkets with differing behaviors affect a heterogeneous crowd of consumer agents. The model was implemented in Java with Python used to evaluate the results. 

The simulation displays good acceptance with real grocery market behavior, i.e. captures the performance of British retailers thus can be used to determine the impact of changes in their behavior on their competitors and consumers.Furthermore it can be used to provide insight into sustainability of volatile pricing strategies, providing a useful insight in volatility of British supermarket retail industry. 
\end{abstract}
\acknowledgements{
I would like to express my sincere gratitude to Dr Maria Polukarov for her guidance and support which provided me the freedom to take this research in the direction of my interest.\\
\\
I would also like to thank my family and friends for their encouragement and support. To those who quietly listened to my software complaints. To those who worked throughout the nights with me. To those who helped me write what I couldn't say. I cannot thank you enough.
}

\declaration{
I, Stefan Collier, declare that this dissertation and the work presented in it are my own and has been generated by me as the result of my own original research.\\
I confirm that:\\
1. This work was done wholly or mainly while in candidature for a degree at this University;\\
2. Where any part of this dissertation has previously been submitted for any other qualification at this University or any other institution, this has been clearly stated;\\
3. Where I have consulted the published work of others, this is always clearly attributed;\\
4. Where I have quoted from the work of others, the source is always given. With the exception of such quotations, this dissertation is entirely my own work;\\
5. I have acknowledged all main sources of help;\\
6. Where the thesis is based on work done by myself jointly with others, I have made clear exactly what was done by others and what I have contributed myself;\\
7. Either none of this work has been published before submission, or parts of this work have been published by :\\
\\
Stefan Collier\\
April 2016
}
\tableofcontents
\listoffigures
\listoftables

\mainmatter
%% ----------------------------------------------------------------
%\include{Introduction}
%\include{Conclusions}
\include{chapters/1Project/main}
\include{chapters/2Lit/main}
\include{chapters/3Design/HighLevel}
\include{chapters/3Design/InDepth}
\include{chapters/4Impl/main}

\include{chapters/5Experiments/1/main}
\include{chapters/5Experiments/2/main}
\include{chapters/5Experiments/3/main}
\include{chapters/5Experiments/4/main}

\include{chapters/6Conclusion/main}

\appendix
\include{appendix/AppendixB}
\include{appendix/D/main}
\include{appendix/AppendixC}

\backmatter
\bibliographystyle{ecs}
\bibliography{ECS}
\end{document}
%% ----------------------------------------------------------------

 %% ----------------------------------------------------------------
%% Progress.tex
%% ---------------------------------------------------------------- 
\documentclass{ecsprogress}    % Use the progress Style
\graphicspath{{../figs/}}   % Location of your graphics files
    \usepackage{natbib}            % Use Natbib style for the refs.
\hypersetup{colorlinks=true}   % Set to false for black/white printing
\input{Definitions}            % Include your abbreviations



\usepackage{enumitem}% http://ctan.org/pkg/enumitem
\usepackage{multirow}
\usepackage{float}
\usepackage{amsmath}
\usepackage{multicol}
\usepackage{amssymb}
\usepackage[normalem]{ulem}
\useunder{\uline}{\ul}{}
\usepackage{wrapfig}


\usepackage[table,xcdraw]{xcolor}


%% ----------------------------------------------------------------
\begin{document}
\frontmatter
\title      {Heterogeneous Agent-based Model for Supermarket Competition}
\authors    {\texorpdfstring
             {\href{mailto:sc22g13@ecs.soton.ac.uk}{Stefan J. Collier}}
             {Stefan J. Collier}
            }
\addresses  {\groupname\\\deptname\\\univname}
\date       {\today}
\subject    {}
\keywords   {}
\supervisor {Dr. Maria Polukarov}
\examiner   {Professor Sheng Chen}

\maketitle
\begin{abstract}
This project aim was to model and analyse the effects of competitive pricing behaviors of grocery retailers on the British market. 

This was achieved by creating a multi-agent model, containing retailer and consumer agents. The heterogeneous crowd of retailers employs either a uniform pricing strategy or a ‘local price flexing’ strategy. The actions of these retailers are chosen by predicting the profit of each action, using a perceptron. Following on from the consideration of different economic models, a discrete model was developed so that software agents have a discrete environment to operate within. Within the model, it has been observed how supermarkets with differing behaviors affect a heterogeneous crowd of consumer agents. The model was implemented in Java with Python used to evaluate the results. 

The simulation displays good acceptance with real grocery market behavior, i.e. captures the performance of British retailers thus can be used to determine the impact of changes in their behavior on their competitors and consumers.Furthermore it can be used to provide insight into sustainability of volatile pricing strategies, providing a useful insight in volatility of British supermarket retail industry. 
\end{abstract}
\acknowledgements{
I would like to express my sincere gratitude to Dr Maria Polukarov for her guidance and support which provided me the freedom to take this research in the direction of my interest.\\
\\
I would also like to thank my family and friends for their encouragement and support. To those who quietly listened to my software complaints. To those who worked throughout the nights with me. To those who helped me write what I couldn't say. I cannot thank you enough.
}

\declaration{
I, Stefan Collier, declare that this dissertation and the work presented in it are my own and has been generated by me as the result of my own original research.\\
I confirm that:\\
1. This work was done wholly or mainly while in candidature for a degree at this University;\\
2. Where any part of this dissertation has previously been submitted for any other qualification at this University or any other institution, this has been clearly stated;\\
3. Where I have consulted the published work of others, this is always clearly attributed;\\
4. Where I have quoted from the work of others, the source is always given. With the exception of such quotations, this dissertation is entirely my own work;\\
5. I have acknowledged all main sources of help;\\
6. Where the thesis is based on work done by myself jointly with others, I have made clear exactly what was done by others and what I have contributed myself;\\
7. Either none of this work has been published before submission, or parts of this work have been published by :\\
\\
Stefan Collier\\
April 2016
}
\tableofcontents
\listoffigures
\listoftables

\mainmatter
%% ----------------------------------------------------------------
%\include{Introduction}
%\include{Conclusions}
\include{chapters/1Project/main}
\include{chapters/2Lit/main}
\include{chapters/3Design/HighLevel}
\include{chapters/3Design/InDepth}
\include{chapters/4Impl/main}

\include{chapters/5Experiments/1/main}
\include{chapters/5Experiments/2/main}
\include{chapters/5Experiments/3/main}
\include{chapters/5Experiments/4/main}

\include{chapters/6Conclusion/main}

\appendix
\include{appendix/AppendixB}
\include{appendix/D/main}
\include{appendix/AppendixC}

\backmatter
\bibliographystyle{ecs}
\bibliography{ECS}
\end{document}
%% ----------------------------------------------------------------

\include{chapters/3Design/HighLevel}
\include{chapters/3Design/InDepth}
 %% ----------------------------------------------------------------
%% Progress.tex
%% ---------------------------------------------------------------- 
\documentclass{ecsprogress}    % Use the progress Style
\graphicspath{{../figs/}}   % Location of your graphics files
    \usepackage{natbib}            % Use Natbib style for the refs.
\hypersetup{colorlinks=true}   % Set to false for black/white printing
\input{Definitions}            % Include your abbreviations



\usepackage{enumitem}% http://ctan.org/pkg/enumitem
\usepackage{multirow}
\usepackage{float}
\usepackage{amsmath}
\usepackage{multicol}
\usepackage{amssymb}
\usepackage[normalem]{ulem}
\useunder{\uline}{\ul}{}
\usepackage{wrapfig}


\usepackage[table,xcdraw]{xcolor}


%% ----------------------------------------------------------------
\begin{document}
\frontmatter
\title      {Heterogeneous Agent-based Model for Supermarket Competition}
\authors    {\texorpdfstring
             {\href{mailto:sc22g13@ecs.soton.ac.uk}{Stefan J. Collier}}
             {Stefan J. Collier}
            }
\addresses  {\groupname\\\deptname\\\univname}
\date       {\today}
\subject    {}
\keywords   {}
\supervisor {Dr. Maria Polukarov}
\examiner   {Professor Sheng Chen}

\maketitle
\begin{abstract}
This project aim was to model and analyse the effects of competitive pricing behaviors of grocery retailers on the British market. 

This was achieved by creating a multi-agent model, containing retailer and consumer agents. The heterogeneous crowd of retailers employs either a uniform pricing strategy or a ‘local price flexing’ strategy. The actions of these retailers are chosen by predicting the profit of each action, using a perceptron. Following on from the consideration of different economic models, a discrete model was developed so that software agents have a discrete environment to operate within. Within the model, it has been observed how supermarkets with differing behaviors affect a heterogeneous crowd of consumer agents. The model was implemented in Java with Python used to evaluate the results. 

The simulation displays good acceptance with real grocery market behavior, i.e. captures the performance of British retailers thus can be used to determine the impact of changes in their behavior on their competitors and consumers.Furthermore it can be used to provide insight into sustainability of volatile pricing strategies, providing a useful insight in volatility of British supermarket retail industry. 
\end{abstract}
\acknowledgements{
I would like to express my sincere gratitude to Dr Maria Polukarov for her guidance and support which provided me the freedom to take this research in the direction of my interest.\\
\\
I would also like to thank my family and friends for their encouragement and support. To those who quietly listened to my software complaints. To those who worked throughout the nights with me. To those who helped me write what I couldn't say. I cannot thank you enough.
}

\declaration{
I, Stefan Collier, declare that this dissertation and the work presented in it are my own and has been generated by me as the result of my own original research.\\
I confirm that:\\
1. This work was done wholly or mainly while in candidature for a degree at this University;\\
2. Where any part of this dissertation has previously been submitted for any other qualification at this University or any other institution, this has been clearly stated;\\
3. Where I have consulted the published work of others, this is always clearly attributed;\\
4. Where I have quoted from the work of others, the source is always given. With the exception of such quotations, this dissertation is entirely my own work;\\
5. I have acknowledged all main sources of help;\\
6. Where the thesis is based on work done by myself jointly with others, I have made clear exactly what was done by others and what I have contributed myself;\\
7. Either none of this work has been published before submission, or parts of this work have been published by :\\
\\
Stefan Collier\\
April 2016
}
\tableofcontents
\listoffigures
\listoftables

\mainmatter
%% ----------------------------------------------------------------
%\include{Introduction}
%\include{Conclusions}
\include{chapters/1Project/main}
\include{chapters/2Lit/main}
\include{chapters/3Design/HighLevel}
\include{chapters/3Design/InDepth}
\include{chapters/4Impl/main}

\include{chapters/5Experiments/1/main}
\include{chapters/5Experiments/2/main}
\include{chapters/5Experiments/3/main}
\include{chapters/5Experiments/4/main}

\include{chapters/6Conclusion/main}

\appendix
\include{appendix/AppendixB}
\include{appendix/D/main}
\include{appendix/AppendixC}

\backmatter
\bibliographystyle{ecs}
\bibliography{ECS}
\end{document}
%% ----------------------------------------------------------------


 %% ----------------------------------------------------------------
%% Progress.tex
%% ---------------------------------------------------------------- 
\documentclass{ecsprogress}    % Use the progress Style
\graphicspath{{../figs/}}   % Location of your graphics files
    \usepackage{natbib}            % Use Natbib style for the refs.
\hypersetup{colorlinks=true}   % Set to false for black/white printing
\input{Definitions}            % Include your abbreviations



\usepackage{enumitem}% http://ctan.org/pkg/enumitem
\usepackage{multirow}
\usepackage{float}
\usepackage{amsmath}
\usepackage{multicol}
\usepackage{amssymb}
\usepackage[normalem]{ulem}
\useunder{\uline}{\ul}{}
\usepackage{wrapfig}


\usepackage[table,xcdraw]{xcolor}


%% ----------------------------------------------------------------
\begin{document}
\frontmatter
\title      {Heterogeneous Agent-based Model for Supermarket Competition}
\authors    {\texorpdfstring
             {\href{mailto:sc22g13@ecs.soton.ac.uk}{Stefan J. Collier}}
             {Stefan J. Collier}
            }
\addresses  {\groupname\\\deptname\\\univname}
\date       {\today}
\subject    {}
\keywords   {}
\supervisor {Dr. Maria Polukarov}
\examiner   {Professor Sheng Chen}

\maketitle
\begin{abstract}
This project aim was to model and analyse the effects of competitive pricing behaviors of grocery retailers on the British market. 

This was achieved by creating a multi-agent model, containing retailer and consumer agents. The heterogeneous crowd of retailers employs either a uniform pricing strategy or a ‘local price flexing’ strategy. The actions of these retailers are chosen by predicting the profit of each action, using a perceptron. Following on from the consideration of different economic models, a discrete model was developed so that software agents have a discrete environment to operate within. Within the model, it has been observed how supermarkets with differing behaviors affect a heterogeneous crowd of consumer agents. The model was implemented in Java with Python used to evaluate the results. 

The simulation displays good acceptance with real grocery market behavior, i.e. captures the performance of British retailers thus can be used to determine the impact of changes in their behavior on their competitors and consumers.Furthermore it can be used to provide insight into sustainability of volatile pricing strategies, providing a useful insight in volatility of British supermarket retail industry. 
\end{abstract}
\acknowledgements{
I would like to express my sincere gratitude to Dr Maria Polukarov for her guidance and support which provided me the freedom to take this research in the direction of my interest.\\
\\
I would also like to thank my family and friends for their encouragement and support. To those who quietly listened to my software complaints. To those who worked throughout the nights with me. To those who helped me write what I couldn't say. I cannot thank you enough.
}

\declaration{
I, Stefan Collier, declare that this dissertation and the work presented in it are my own and has been generated by me as the result of my own original research.\\
I confirm that:\\
1. This work was done wholly or mainly while in candidature for a degree at this University;\\
2. Where any part of this dissertation has previously been submitted for any other qualification at this University or any other institution, this has been clearly stated;\\
3. Where I have consulted the published work of others, this is always clearly attributed;\\
4. Where I have quoted from the work of others, the source is always given. With the exception of such quotations, this dissertation is entirely my own work;\\
5. I have acknowledged all main sources of help;\\
6. Where the thesis is based on work done by myself jointly with others, I have made clear exactly what was done by others and what I have contributed myself;\\
7. Either none of this work has been published before submission, or parts of this work have been published by :\\
\\
Stefan Collier\\
April 2016
}
\tableofcontents
\listoffigures
\listoftables

\mainmatter
%% ----------------------------------------------------------------
%\include{Introduction}
%\include{Conclusions}
\include{chapters/1Project/main}
\include{chapters/2Lit/main}
\include{chapters/3Design/HighLevel}
\include{chapters/3Design/InDepth}
\include{chapters/4Impl/main}

\include{chapters/5Experiments/1/main}
\include{chapters/5Experiments/2/main}
\include{chapters/5Experiments/3/main}
\include{chapters/5Experiments/4/main}

\include{chapters/6Conclusion/main}

\appendix
\include{appendix/AppendixB}
\include{appendix/D/main}
\include{appendix/AppendixC}

\backmatter
\bibliographystyle{ecs}
\bibliography{ECS}
\end{document}
%% ----------------------------------------------------------------

 %% ----------------------------------------------------------------
%% Progress.tex
%% ---------------------------------------------------------------- 
\documentclass{ecsprogress}    % Use the progress Style
\graphicspath{{../figs/}}   % Location of your graphics files
    \usepackage{natbib}            % Use Natbib style for the refs.
\hypersetup{colorlinks=true}   % Set to false for black/white printing
\input{Definitions}            % Include your abbreviations



\usepackage{enumitem}% http://ctan.org/pkg/enumitem
\usepackage{multirow}
\usepackage{float}
\usepackage{amsmath}
\usepackage{multicol}
\usepackage{amssymb}
\usepackage[normalem]{ulem}
\useunder{\uline}{\ul}{}
\usepackage{wrapfig}


\usepackage[table,xcdraw]{xcolor}


%% ----------------------------------------------------------------
\begin{document}
\frontmatter
\title      {Heterogeneous Agent-based Model for Supermarket Competition}
\authors    {\texorpdfstring
             {\href{mailto:sc22g13@ecs.soton.ac.uk}{Stefan J. Collier}}
             {Stefan J. Collier}
            }
\addresses  {\groupname\\\deptname\\\univname}
\date       {\today}
\subject    {}
\keywords   {}
\supervisor {Dr. Maria Polukarov}
\examiner   {Professor Sheng Chen}

\maketitle
\begin{abstract}
This project aim was to model and analyse the effects of competitive pricing behaviors of grocery retailers on the British market. 

This was achieved by creating a multi-agent model, containing retailer and consumer agents. The heterogeneous crowd of retailers employs either a uniform pricing strategy or a ‘local price flexing’ strategy. The actions of these retailers are chosen by predicting the profit of each action, using a perceptron. Following on from the consideration of different economic models, a discrete model was developed so that software agents have a discrete environment to operate within. Within the model, it has been observed how supermarkets with differing behaviors affect a heterogeneous crowd of consumer agents. The model was implemented in Java with Python used to evaluate the results. 

The simulation displays good acceptance with real grocery market behavior, i.e. captures the performance of British retailers thus can be used to determine the impact of changes in their behavior on their competitors and consumers.Furthermore it can be used to provide insight into sustainability of volatile pricing strategies, providing a useful insight in volatility of British supermarket retail industry. 
\end{abstract}
\acknowledgements{
I would like to express my sincere gratitude to Dr Maria Polukarov for her guidance and support which provided me the freedom to take this research in the direction of my interest.\\
\\
I would also like to thank my family and friends for their encouragement and support. To those who quietly listened to my software complaints. To those who worked throughout the nights with me. To those who helped me write what I couldn't say. I cannot thank you enough.
}

\declaration{
I, Stefan Collier, declare that this dissertation and the work presented in it are my own and has been generated by me as the result of my own original research.\\
I confirm that:\\
1. This work was done wholly or mainly while in candidature for a degree at this University;\\
2. Where any part of this dissertation has previously been submitted for any other qualification at this University or any other institution, this has been clearly stated;\\
3. Where I have consulted the published work of others, this is always clearly attributed;\\
4. Where I have quoted from the work of others, the source is always given. With the exception of such quotations, this dissertation is entirely my own work;\\
5. I have acknowledged all main sources of help;\\
6. Where the thesis is based on work done by myself jointly with others, I have made clear exactly what was done by others and what I have contributed myself;\\
7. Either none of this work has been published before submission, or parts of this work have been published by :\\
\\
Stefan Collier\\
April 2016
}
\tableofcontents
\listoffigures
\listoftables

\mainmatter
%% ----------------------------------------------------------------
%\include{Introduction}
%\include{Conclusions}
\include{chapters/1Project/main}
\include{chapters/2Lit/main}
\include{chapters/3Design/HighLevel}
\include{chapters/3Design/InDepth}
\include{chapters/4Impl/main}

\include{chapters/5Experiments/1/main}
\include{chapters/5Experiments/2/main}
\include{chapters/5Experiments/3/main}
\include{chapters/5Experiments/4/main}

\include{chapters/6Conclusion/main}

\appendix
\include{appendix/AppendixB}
\include{appendix/D/main}
\include{appendix/AppendixC}

\backmatter
\bibliographystyle{ecs}
\bibliography{ECS}
\end{document}
%% ----------------------------------------------------------------

 %% ----------------------------------------------------------------
%% Progress.tex
%% ---------------------------------------------------------------- 
\documentclass{ecsprogress}    % Use the progress Style
\graphicspath{{../figs/}}   % Location of your graphics files
    \usepackage{natbib}            % Use Natbib style for the refs.
\hypersetup{colorlinks=true}   % Set to false for black/white printing
\input{Definitions}            % Include your abbreviations



\usepackage{enumitem}% http://ctan.org/pkg/enumitem
\usepackage{multirow}
\usepackage{float}
\usepackage{amsmath}
\usepackage{multicol}
\usepackage{amssymb}
\usepackage[normalem]{ulem}
\useunder{\uline}{\ul}{}
\usepackage{wrapfig}


\usepackage[table,xcdraw]{xcolor}


%% ----------------------------------------------------------------
\begin{document}
\frontmatter
\title      {Heterogeneous Agent-based Model for Supermarket Competition}
\authors    {\texorpdfstring
             {\href{mailto:sc22g13@ecs.soton.ac.uk}{Stefan J. Collier}}
             {Stefan J. Collier}
            }
\addresses  {\groupname\\\deptname\\\univname}
\date       {\today}
\subject    {}
\keywords   {}
\supervisor {Dr. Maria Polukarov}
\examiner   {Professor Sheng Chen}

\maketitle
\begin{abstract}
This project aim was to model and analyse the effects of competitive pricing behaviors of grocery retailers on the British market. 

This was achieved by creating a multi-agent model, containing retailer and consumer agents. The heterogeneous crowd of retailers employs either a uniform pricing strategy or a ‘local price flexing’ strategy. The actions of these retailers are chosen by predicting the profit of each action, using a perceptron. Following on from the consideration of different economic models, a discrete model was developed so that software agents have a discrete environment to operate within. Within the model, it has been observed how supermarkets with differing behaviors affect a heterogeneous crowd of consumer agents. The model was implemented in Java with Python used to evaluate the results. 

The simulation displays good acceptance with real grocery market behavior, i.e. captures the performance of British retailers thus can be used to determine the impact of changes in their behavior on their competitors and consumers.Furthermore it can be used to provide insight into sustainability of volatile pricing strategies, providing a useful insight in volatility of British supermarket retail industry. 
\end{abstract}
\acknowledgements{
I would like to express my sincere gratitude to Dr Maria Polukarov for her guidance and support which provided me the freedom to take this research in the direction of my interest.\\
\\
I would also like to thank my family and friends for their encouragement and support. To those who quietly listened to my software complaints. To those who worked throughout the nights with me. To those who helped me write what I couldn't say. I cannot thank you enough.
}

\declaration{
I, Stefan Collier, declare that this dissertation and the work presented in it are my own and has been generated by me as the result of my own original research.\\
I confirm that:\\
1. This work was done wholly or mainly while in candidature for a degree at this University;\\
2. Where any part of this dissertation has previously been submitted for any other qualification at this University or any other institution, this has been clearly stated;\\
3. Where I have consulted the published work of others, this is always clearly attributed;\\
4. Where I have quoted from the work of others, the source is always given. With the exception of such quotations, this dissertation is entirely my own work;\\
5. I have acknowledged all main sources of help;\\
6. Where the thesis is based on work done by myself jointly with others, I have made clear exactly what was done by others and what I have contributed myself;\\
7. Either none of this work has been published before submission, or parts of this work have been published by :\\
\\
Stefan Collier\\
April 2016
}
\tableofcontents
\listoffigures
\listoftables

\mainmatter
%% ----------------------------------------------------------------
%\include{Introduction}
%\include{Conclusions}
\include{chapters/1Project/main}
\include{chapters/2Lit/main}
\include{chapters/3Design/HighLevel}
\include{chapters/3Design/InDepth}
\include{chapters/4Impl/main}

\include{chapters/5Experiments/1/main}
\include{chapters/5Experiments/2/main}
\include{chapters/5Experiments/3/main}
\include{chapters/5Experiments/4/main}

\include{chapters/6Conclusion/main}

\appendix
\include{appendix/AppendixB}
\include{appendix/D/main}
\include{appendix/AppendixC}

\backmatter
\bibliographystyle{ecs}
\bibliography{ECS}
\end{document}
%% ----------------------------------------------------------------

 %% ----------------------------------------------------------------
%% Progress.tex
%% ---------------------------------------------------------------- 
\documentclass{ecsprogress}    % Use the progress Style
\graphicspath{{../figs/}}   % Location of your graphics files
    \usepackage{natbib}            % Use Natbib style for the refs.
\hypersetup{colorlinks=true}   % Set to false for black/white printing
\input{Definitions}            % Include your abbreviations



\usepackage{enumitem}% http://ctan.org/pkg/enumitem
\usepackage{multirow}
\usepackage{float}
\usepackage{amsmath}
\usepackage{multicol}
\usepackage{amssymb}
\usepackage[normalem]{ulem}
\useunder{\uline}{\ul}{}
\usepackage{wrapfig}


\usepackage[table,xcdraw]{xcolor}


%% ----------------------------------------------------------------
\begin{document}
\frontmatter
\title      {Heterogeneous Agent-based Model for Supermarket Competition}
\authors    {\texorpdfstring
             {\href{mailto:sc22g13@ecs.soton.ac.uk}{Stefan J. Collier}}
             {Stefan J. Collier}
            }
\addresses  {\groupname\\\deptname\\\univname}
\date       {\today}
\subject    {}
\keywords   {}
\supervisor {Dr. Maria Polukarov}
\examiner   {Professor Sheng Chen}

\maketitle
\begin{abstract}
This project aim was to model and analyse the effects of competitive pricing behaviors of grocery retailers on the British market. 

This was achieved by creating a multi-agent model, containing retailer and consumer agents. The heterogeneous crowd of retailers employs either a uniform pricing strategy or a ‘local price flexing’ strategy. The actions of these retailers are chosen by predicting the profit of each action, using a perceptron. Following on from the consideration of different economic models, a discrete model was developed so that software agents have a discrete environment to operate within. Within the model, it has been observed how supermarkets with differing behaviors affect a heterogeneous crowd of consumer agents. The model was implemented in Java with Python used to evaluate the results. 

The simulation displays good acceptance with real grocery market behavior, i.e. captures the performance of British retailers thus can be used to determine the impact of changes in their behavior on their competitors and consumers.Furthermore it can be used to provide insight into sustainability of volatile pricing strategies, providing a useful insight in volatility of British supermarket retail industry. 
\end{abstract}
\acknowledgements{
I would like to express my sincere gratitude to Dr Maria Polukarov for her guidance and support which provided me the freedom to take this research in the direction of my interest.\\
\\
I would also like to thank my family and friends for their encouragement and support. To those who quietly listened to my software complaints. To those who worked throughout the nights with me. To those who helped me write what I couldn't say. I cannot thank you enough.
}

\declaration{
I, Stefan Collier, declare that this dissertation and the work presented in it are my own and has been generated by me as the result of my own original research.\\
I confirm that:\\
1. This work was done wholly or mainly while in candidature for a degree at this University;\\
2. Where any part of this dissertation has previously been submitted for any other qualification at this University or any other institution, this has been clearly stated;\\
3. Where I have consulted the published work of others, this is always clearly attributed;\\
4. Where I have quoted from the work of others, the source is always given. With the exception of such quotations, this dissertation is entirely my own work;\\
5. I have acknowledged all main sources of help;\\
6. Where the thesis is based on work done by myself jointly with others, I have made clear exactly what was done by others and what I have contributed myself;\\
7. Either none of this work has been published before submission, or parts of this work have been published by :\\
\\
Stefan Collier\\
April 2016
}
\tableofcontents
\listoffigures
\listoftables

\mainmatter
%% ----------------------------------------------------------------
%\include{Introduction}
%\include{Conclusions}
\include{chapters/1Project/main}
\include{chapters/2Lit/main}
\include{chapters/3Design/HighLevel}
\include{chapters/3Design/InDepth}
\include{chapters/4Impl/main}

\include{chapters/5Experiments/1/main}
\include{chapters/5Experiments/2/main}
\include{chapters/5Experiments/3/main}
\include{chapters/5Experiments/4/main}

\include{chapters/6Conclusion/main}

\appendix
\include{appendix/AppendixB}
\include{appendix/D/main}
\include{appendix/AppendixC}

\backmatter
\bibliographystyle{ecs}
\bibliography{ECS}
\end{document}
%% ----------------------------------------------------------------


 %% ----------------------------------------------------------------
%% Progress.tex
%% ---------------------------------------------------------------- 
\documentclass{ecsprogress}    % Use the progress Style
\graphicspath{{../figs/}}   % Location of your graphics files
    \usepackage{natbib}            % Use Natbib style for the refs.
\hypersetup{colorlinks=true}   % Set to false for black/white printing
\input{Definitions}            % Include your abbreviations



\usepackage{enumitem}% http://ctan.org/pkg/enumitem
\usepackage{multirow}
\usepackage{float}
\usepackage{amsmath}
\usepackage{multicol}
\usepackage{amssymb}
\usepackage[normalem]{ulem}
\useunder{\uline}{\ul}{}
\usepackage{wrapfig}


\usepackage[table,xcdraw]{xcolor}


%% ----------------------------------------------------------------
\begin{document}
\frontmatter
\title      {Heterogeneous Agent-based Model for Supermarket Competition}
\authors    {\texorpdfstring
             {\href{mailto:sc22g13@ecs.soton.ac.uk}{Stefan J. Collier}}
             {Stefan J. Collier}
            }
\addresses  {\groupname\\\deptname\\\univname}
\date       {\today}
\subject    {}
\keywords   {}
\supervisor {Dr. Maria Polukarov}
\examiner   {Professor Sheng Chen}

\maketitle
\begin{abstract}
This project aim was to model and analyse the effects of competitive pricing behaviors of grocery retailers on the British market. 

This was achieved by creating a multi-agent model, containing retailer and consumer agents. The heterogeneous crowd of retailers employs either a uniform pricing strategy or a ‘local price flexing’ strategy. The actions of these retailers are chosen by predicting the profit of each action, using a perceptron. Following on from the consideration of different economic models, a discrete model was developed so that software agents have a discrete environment to operate within. Within the model, it has been observed how supermarkets with differing behaviors affect a heterogeneous crowd of consumer agents. The model was implemented in Java with Python used to evaluate the results. 

The simulation displays good acceptance with real grocery market behavior, i.e. captures the performance of British retailers thus can be used to determine the impact of changes in their behavior on their competitors and consumers.Furthermore it can be used to provide insight into sustainability of volatile pricing strategies, providing a useful insight in volatility of British supermarket retail industry. 
\end{abstract}
\acknowledgements{
I would like to express my sincere gratitude to Dr Maria Polukarov for her guidance and support which provided me the freedom to take this research in the direction of my interest.\\
\\
I would also like to thank my family and friends for their encouragement and support. To those who quietly listened to my software complaints. To those who worked throughout the nights with me. To those who helped me write what I couldn't say. I cannot thank you enough.
}

\declaration{
I, Stefan Collier, declare that this dissertation and the work presented in it are my own and has been generated by me as the result of my own original research.\\
I confirm that:\\
1. This work was done wholly or mainly while in candidature for a degree at this University;\\
2. Where any part of this dissertation has previously been submitted for any other qualification at this University or any other institution, this has been clearly stated;\\
3. Where I have consulted the published work of others, this is always clearly attributed;\\
4. Where I have quoted from the work of others, the source is always given. With the exception of such quotations, this dissertation is entirely my own work;\\
5. I have acknowledged all main sources of help;\\
6. Where the thesis is based on work done by myself jointly with others, I have made clear exactly what was done by others and what I have contributed myself;\\
7. Either none of this work has been published before submission, or parts of this work have been published by :\\
\\
Stefan Collier\\
April 2016
}
\tableofcontents
\listoffigures
\listoftables

\mainmatter
%% ----------------------------------------------------------------
%\include{Introduction}
%\include{Conclusions}
\include{chapters/1Project/main}
\include{chapters/2Lit/main}
\include{chapters/3Design/HighLevel}
\include{chapters/3Design/InDepth}
\include{chapters/4Impl/main}

\include{chapters/5Experiments/1/main}
\include{chapters/5Experiments/2/main}
\include{chapters/5Experiments/3/main}
\include{chapters/5Experiments/4/main}

\include{chapters/6Conclusion/main}

\appendix
\include{appendix/AppendixB}
\include{appendix/D/main}
\include{appendix/AppendixC}

\backmatter
\bibliographystyle{ecs}
\bibliography{ECS}
\end{document}
%% ----------------------------------------------------------------


\appendix
\include{appendix/AppendixB}
 %% ----------------------------------------------------------------
%% Progress.tex
%% ---------------------------------------------------------------- 
\documentclass{ecsprogress}    % Use the progress Style
\graphicspath{{../figs/}}   % Location of your graphics files
    \usepackage{natbib}            % Use Natbib style for the refs.
\hypersetup{colorlinks=true}   % Set to false for black/white printing
\input{Definitions}            % Include your abbreviations



\usepackage{enumitem}% http://ctan.org/pkg/enumitem
\usepackage{multirow}
\usepackage{float}
\usepackage{amsmath}
\usepackage{multicol}
\usepackage{amssymb}
\usepackage[normalem]{ulem}
\useunder{\uline}{\ul}{}
\usepackage{wrapfig}


\usepackage[table,xcdraw]{xcolor}


%% ----------------------------------------------------------------
\begin{document}
\frontmatter
\title      {Heterogeneous Agent-based Model for Supermarket Competition}
\authors    {\texorpdfstring
             {\href{mailto:sc22g13@ecs.soton.ac.uk}{Stefan J. Collier}}
             {Stefan J. Collier}
            }
\addresses  {\groupname\\\deptname\\\univname}
\date       {\today}
\subject    {}
\keywords   {}
\supervisor {Dr. Maria Polukarov}
\examiner   {Professor Sheng Chen}

\maketitle
\begin{abstract}
This project aim was to model and analyse the effects of competitive pricing behaviors of grocery retailers on the British market. 

This was achieved by creating a multi-agent model, containing retailer and consumer agents. The heterogeneous crowd of retailers employs either a uniform pricing strategy or a ‘local price flexing’ strategy. The actions of these retailers are chosen by predicting the profit of each action, using a perceptron. Following on from the consideration of different economic models, a discrete model was developed so that software agents have a discrete environment to operate within. Within the model, it has been observed how supermarkets with differing behaviors affect a heterogeneous crowd of consumer agents. The model was implemented in Java with Python used to evaluate the results. 

The simulation displays good acceptance with real grocery market behavior, i.e. captures the performance of British retailers thus can be used to determine the impact of changes in their behavior on their competitors and consumers.Furthermore it can be used to provide insight into sustainability of volatile pricing strategies, providing a useful insight in volatility of British supermarket retail industry. 
\end{abstract}
\acknowledgements{
I would like to express my sincere gratitude to Dr Maria Polukarov for her guidance and support which provided me the freedom to take this research in the direction of my interest.\\
\\
I would also like to thank my family and friends for their encouragement and support. To those who quietly listened to my software complaints. To those who worked throughout the nights with me. To those who helped me write what I couldn't say. I cannot thank you enough.
}

\declaration{
I, Stefan Collier, declare that this dissertation and the work presented in it are my own and has been generated by me as the result of my own original research.\\
I confirm that:\\
1. This work was done wholly or mainly while in candidature for a degree at this University;\\
2. Where any part of this dissertation has previously been submitted for any other qualification at this University or any other institution, this has been clearly stated;\\
3. Where I have consulted the published work of others, this is always clearly attributed;\\
4. Where I have quoted from the work of others, the source is always given. With the exception of such quotations, this dissertation is entirely my own work;\\
5. I have acknowledged all main sources of help;\\
6. Where the thesis is based on work done by myself jointly with others, I have made clear exactly what was done by others and what I have contributed myself;\\
7. Either none of this work has been published before submission, or parts of this work have been published by :\\
\\
Stefan Collier\\
April 2016
}
\tableofcontents
\listoffigures
\listoftables

\mainmatter
%% ----------------------------------------------------------------
%\include{Introduction}
%\include{Conclusions}
\include{chapters/1Project/main}
\include{chapters/2Lit/main}
\include{chapters/3Design/HighLevel}
\include{chapters/3Design/InDepth}
\include{chapters/4Impl/main}

\include{chapters/5Experiments/1/main}
\include{chapters/5Experiments/2/main}
\include{chapters/5Experiments/3/main}
\include{chapters/5Experiments/4/main}

\include{chapters/6Conclusion/main}

\appendix
\include{appendix/AppendixB}
\include{appendix/D/main}
\include{appendix/AppendixC}

\backmatter
\bibliographystyle{ecs}
\bibliography{ECS}
\end{document}
%% ----------------------------------------------------------------

\include{appendix/AppendixC}

\backmatter
\bibliographystyle{ecs}
\bibliography{ECS}
\end{document}
%% ----------------------------------------------------------------


 %% ----------------------------------------------------------------
%% Progress.tex
%% ---------------------------------------------------------------- 
\documentclass{ecsprogress}    % Use the progress Style
\graphicspath{{../figs/}}   % Location of your graphics files
    \usepackage{natbib}            % Use Natbib style for the refs.
\hypersetup{colorlinks=true}   % Set to false for black/white printing
\input{Definitions}            % Include your abbreviations



\usepackage{enumitem}% http://ctan.org/pkg/enumitem
\usepackage{multirow}
\usepackage{float}
\usepackage{amsmath}
\usepackage{multicol}
\usepackage{amssymb}
\usepackage[normalem]{ulem}
\useunder{\uline}{\ul}{}
\usepackage{wrapfig}


\usepackage[table,xcdraw]{xcolor}


%% ----------------------------------------------------------------
\begin{document}
\frontmatter
\title      {Heterogeneous Agent-based Model for Supermarket Competition}
\authors    {\texorpdfstring
             {\href{mailto:sc22g13@ecs.soton.ac.uk}{Stefan J. Collier}}
             {Stefan J. Collier}
            }
\addresses  {\groupname\\\deptname\\\univname}
\date       {\today}
\subject    {}
\keywords   {}
\supervisor {Dr. Maria Polukarov}
\examiner   {Professor Sheng Chen}

\maketitle
\begin{abstract}
This project aim was to model and analyse the effects of competitive pricing behaviors of grocery retailers on the British market. 

This was achieved by creating a multi-agent model, containing retailer and consumer agents. The heterogeneous crowd of retailers employs either a uniform pricing strategy or a ‘local price flexing’ strategy. The actions of these retailers are chosen by predicting the profit of each action, using a perceptron. Following on from the consideration of different economic models, a discrete model was developed so that software agents have a discrete environment to operate within. Within the model, it has been observed how supermarkets with differing behaviors affect a heterogeneous crowd of consumer agents. The model was implemented in Java with Python used to evaluate the results. 

The simulation displays good acceptance with real grocery market behavior, i.e. captures the performance of British retailers thus can be used to determine the impact of changes in their behavior on their competitors and consumers.Furthermore it can be used to provide insight into sustainability of volatile pricing strategies, providing a useful insight in volatility of British supermarket retail industry. 
\end{abstract}
\acknowledgements{
I would like to express my sincere gratitude to Dr Maria Polukarov for her guidance and support which provided me the freedom to take this research in the direction of my interest.\\
\\
I would also like to thank my family and friends for their encouragement and support. To those who quietly listened to my software complaints. To those who worked throughout the nights with me. To those who helped me write what I couldn't say. I cannot thank you enough.
}

\declaration{
I, Stefan Collier, declare that this dissertation and the work presented in it are my own and has been generated by me as the result of my own original research.\\
I confirm that:\\
1. This work was done wholly or mainly while in candidature for a degree at this University;\\
2. Where any part of this dissertation has previously been submitted for any other qualification at this University or any other institution, this has been clearly stated;\\
3. Where I have consulted the published work of others, this is always clearly attributed;\\
4. Where I have quoted from the work of others, the source is always given. With the exception of such quotations, this dissertation is entirely my own work;\\
5. I have acknowledged all main sources of help;\\
6. Where the thesis is based on work done by myself jointly with others, I have made clear exactly what was done by others and what I have contributed myself;\\
7. Either none of this work has been published before submission, or parts of this work have been published by :\\
\\
Stefan Collier\\
April 2016
}
\tableofcontents
\listoffigures
\listoftables

\mainmatter
%% ----------------------------------------------------------------
%\include{Introduction}
%\include{Conclusions}
 %% ----------------------------------------------------------------
%% Progress.tex
%% ---------------------------------------------------------------- 
\documentclass{ecsprogress}    % Use the progress Style
\graphicspath{{../figs/}}   % Location of your graphics files
    \usepackage{natbib}            % Use Natbib style for the refs.
\hypersetup{colorlinks=true}   % Set to false for black/white printing
\input{Definitions}            % Include your abbreviations



\usepackage{enumitem}% http://ctan.org/pkg/enumitem
\usepackage{multirow}
\usepackage{float}
\usepackage{amsmath}
\usepackage{multicol}
\usepackage{amssymb}
\usepackage[normalem]{ulem}
\useunder{\uline}{\ul}{}
\usepackage{wrapfig}


\usepackage[table,xcdraw]{xcolor}


%% ----------------------------------------------------------------
\begin{document}
\frontmatter
\title      {Heterogeneous Agent-based Model for Supermarket Competition}
\authors    {\texorpdfstring
             {\href{mailto:sc22g13@ecs.soton.ac.uk}{Stefan J. Collier}}
             {Stefan J. Collier}
            }
\addresses  {\groupname\\\deptname\\\univname}
\date       {\today}
\subject    {}
\keywords   {}
\supervisor {Dr. Maria Polukarov}
\examiner   {Professor Sheng Chen}

\maketitle
\begin{abstract}
This project aim was to model and analyse the effects of competitive pricing behaviors of grocery retailers on the British market. 

This was achieved by creating a multi-agent model, containing retailer and consumer agents. The heterogeneous crowd of retailers employs either a uniform pricing strategy or a ‘local price flexing’ strategy. The actions of these retailers are chosen by predicting the profit of each action, using a perceptron. Following on from the consideration of different economic models, a discrete model was developed so that software agents have a discrete environment to operate within. Within the model, it has been observed how supermarkets with differing behaviors affect a heterogeneous crowd of consumer agents. The model was implemented in Java with Python used to evaluate the results. 

The simulation displays good acceptance with real grocery market behavior, i.e. captures the performance of British retailers thus can be used to determine the impact of changes in their behavior on their competitors and consumers.Furthermore it can be used to provide insight into sustainability of volatile pricing strategies, providing a useful insight in volatility of British supermarket retail industry. 
\end{abstract}
\acknowledgements{
I would like to express my sincere gratitude to Dr Maria Polukarov for her guidance and support which provided me the freedom to take this research in the direction of my interest.\\
\\
I would also like to thank my family and friends for their encouragement and support. To those who quietly listened to my software complaints. To those who worked throughout the nights with me. To those who helped me write what I couldn't say. I cannot thank you enough.
}

\declaration{
I, Stefan Collier, declare that this dissertation and the work presented in it are my own and has been generated by me as the result of my own original research.\\
I confirm that:\\
1. This work was done wholly or mainly while in candidature for a degree at this University;\\
2. Where any part of this dissertation has previously been submitted for any other qualification at this University or any other institution, this has been clearly stated;\\
3. Where I have consulted the published work of others, this is always clearly attributed;\\
4. Where I have quoted from the work of others, the source is always given. With the exception of such quotations, this dissertation is entirely my own work;\\
5. I have acknowledged all main sources of help;\\
6. Where the thesis is based on work done by myself jointly with others, I have made clear exactly what was done by others and what I have contributed myself;\\
7. Either none of this work has been published before submission, or parts of this work have been published by :\\
\\
Stefan Collier\\
April 2016
}
\tableofcontents
\listoffigures
\listoftables

\mainmatter
%% ----------------------------------------------------------------
%\include{Introduction}
%\include{Conclusions}
\include{chapters/1Project/main}
\include{chapters/2Lit/main}
\include{chapters/3Design/HighLevel}
\include{chapters/3Design/InDepth}
\include{chapters/4Impl/main}

\include{chapters/5Experiments/1/main}
\include{chapters/5Experiments/2/main}
\include{chapters/5Experiments/3/main}
\include{chapters/5Experiments/4/main}

\include{chapters/6Conclusion/main}

\appendix
\include{appendix/AppendixB}
\include{appendix/D/main}
\include{appendix/AppendixC}

\backmatter
\bibliographystyle{ecs}
\bibliography{ECS}
\end{document}
%% ----------------------------------------------------------------

 %% ----------------------------------------------------------------
%% Progress.tex
%% ---------------------------------------------------------------- 
\documentclass{ecsprogress}    % Use the progress Style
\graphicspath{{../figs/}}   % Location of your graphics files
    \usepackage{natbib}            % Use Natbib style for the refs.
\hypersetup{colorlinks=true}   % Set to false for black/white printing
\input{Definitions}            % Include your abbreviations



\usepackage{enumitem}% http://ctan.org/pkg/enumitem
\usepackage{multirow}
\usepackage{float}
\usepackage{amsmath}
\usepackage{multicol}
\usepackage{amssymb}
\usepackage[normalem]{ulem}
\useunder{\uline}{\ul}{}
\usepackage{wrapfig}


\usepackage[table,xcdraw]{xcolor}


%% ----------------------------------------------------------------
\begin{document}
\frontmatter
\title      {Heterogeneous Agent-based Model for Supermarket Competition}
\authors    {\texorpdfstring
             {\href{mailto:sc22g13@ecs.soton.ac.uk}{Stefan J. Collier}}
             {Stefan J. Collier}
            }
\addresses  {\groupname\\\deptname\\\univname}
\date       {\today}
\subject    {}
\keywords   {}
\supervisor {Dr. Maria Polukarov}
\examiner   {Professor Sheng Chen}

\maketitle
\begin{abstract}
This project aim was to model and analyse the effects of competitive pricing behaviors of grocery retailers on the British market. 

This was achieved by creating a multi-agent model, containing retailer and consumer agents. The heterogeneous crowd of retailers employs either a uniform pricing strategy or a ‘local price flexing’ strategy. The actions of these retailers are chosen by predicting the profit of each action, using a perceptron. Following on from the consideration of different economic models, a discrete model was developed so that software agents have a discrete environment to operate within. Within the model, it has been observed how supermarkets with differing behaviors affect a heterogeneous crowd of consumer agents. The model was implemented in Java with Python used to evaluate the results. 

The simulation displays good acceptance with real grocery market behavior, i.e. captures the performance of British retailers thus can be used to determine the impact of changes in their behavior on their competitors and consumers.Furthermore it can be used to provide insight into sustainability of volatile pricing strategies, providing a useful insight in volatility of British supermarket retail industry. 
\end{abstract}
\acknowledgements{
I would like to express my sincere gratitude to Dr Maria Polukarov for her guidance and support which provided me the freedom to take this research in the direction of my interest.\\
\\
I would also like to thank my family and friends for their encouragement and support. To those who quietly listened to my software complaints. To those who worked throughout the nights with me. To those who helped me write what I couldn't say. I cannot thank you enough.
}

\declaration{
I, Stefan Collier, declare that this dissertation and the work presented in it are my own and has been generated by me as the result of my own original research.\\
I confirm that:\\
1. This work was done wholly or mainly while in candidature for a degree at this University;\\
2. Where any part of this dissertation has previously been submitted for any other qualification at this University or any other institution, this has been clearly stated;\\
3. Where I have consulted the published work of others, this is always clearly attributed;\\
4. Where I have quoted from the work of others, the source is always given. With the exception of such quotations, this dissertation is entirely my own work;\\
5. I have acknowledged all main sources of help;\\
6. Where the thesis is based on work done by myself jointly with others, I have made clear exactly what was done by others and what I have contributed myself;\\
7. Either none of this work has been published before submission, or parts of this work have been published by :\\
\\
Stefan Collier\\
April 2016
}
\tableofcontents
\listoffigures
\listoftables

\mainmatter
%% ----------------------------------------------------------------
%\include{Introduction}
%\include{Conclusions}
\include{chapters/1Project/main}
\include{chapters/2Lit/main}
\include{chapters/3Design/HighLevel}
\include{chapters/3Design/InDepth}
\include{chapters/4Impl/main}

\include{chapters/5Experiments/1/main}
\include{chapters/5Experiments/2/main}
\include{chapters/5Experiments/3/main}
\include{chapters/5Experiments/4/main}

\include{chapters/6Conclusion/main}

\appendix
\include{appendix/AppendixB}
\include{appendix/D/main}
\include{appendix/AppendixC}

\backmatter
\bibliographystyle{ecs}
\bibliography{ECS}
\end{document}
%% ----------------------------------------------------------------

\include{chapters/3Design/HighLevel}
\include{chapters/3Design/InDepth}
 %% ----------------------------------------------------------------
%% Progress.tex
%% ---------------------------------------------------------------- 
\documentclass{ecsprogress}    % Use the progress Style
\graphicspath{{../figs/}}   % Location of your graphics files
    \usepackage{natbib}            % Use Natbib style for the refs.
\hypersetup{colorlinks=true}   % Set to false for black/white printing
\input{Definitions}            % Include your abbreviations



\usepackage{enumitem}% http://ctan.org/pkg/enumitem
\usepackage{multirow}
\usepackage{float}
\usepackage{amsmath}
\usepackage{multicol}
\usepackage{amssymb}
\usepackage[normalem]{ulem}
\useunder{\uline}{\ul}{}
\usepackage{wrapfig}


\usepackage[table,xcdraw]{xcolor}


%% ----------------------------------------------------------------
\begin{document}
\frontmatter
\title      {Heterogeneous Agent-based Model for Supermarket Competition}
\authors    {\texorpdfstring
             {\href{mailto:sc22g13@ecs.soton.ac.uk}{Stefan J. Collier}}
             {Stefan J. Collier}
            }
\addresses  {\groupname\\\deptname\\\univname}
\date       {\today}
\subject    {}
\keywords   {}
\supervisor {Dr. Maria Polukarov}
\examiner   {Professor Sheng Chen}

\maketitle
\begin{abstract}
This project aim was to model and analyse the effects of competitive pricing behaviors of grocery retailers on the British market. 

This was achieved by creating a multi-agent model, containing retailer and consumer agents. The heterogeneous crowd of retailers employs either a uniform pricing strategy or a ‘local price flexing’ strategy. The actions of these retailers are chosen by predicting the profit of each action, using a perceptron. Following on from the consideration of different economic models, a discrete model was developed so that software agents have a discrete environment to operate within. Within the model, it has been observed how supermarkets with differing behaviors affect a heterogeneous crowd of consumer agents. The model was implemented in Java with Python used to evaluate the results. 

The simulation displays good acceptance with real grocery market behavior, i.e. captures the performance of British retailers thus can be used to determine the impact of changes in their behavior on their competitors and consumers.Furthermore it can be used to provide insight into sustainability of volatile pricing strategies, providing a useful insight in volatility of British supermarket retail industry. 
\end{abstract}
\acknowledgements{
I would like to express my sincere gratitude to Dr Maria Polukarov for her guidance and support which provided me the freedom to take this research in the direction of my interest.\\
\\
I would also like to thank my family and friends for their encouragement and support. To those who quietly listened to my software complaints. To those who worked throughout the nights with me. To those who helped me write what I couldn't say. I cannot thank you enough.
}

\declaration{
I, Stefan Collier, declare that this dissertation and the work presented in it are my own and has been generated by me as the result of my own original research.\\
I confirm that:\\
1. This work was done wholly or mainly while in candidature for a degree at this University;\\
2. Where any part of this dissertation has previously been submitted for any other qualification at this University or any other institution, this has been clearly stated;\\
3. Where I have consulted the published work of others, this is always clearly attributed;\\
4. Where I have quoted from the work of others, the source is always given. With the exception of such quotations, this dissertation is entirely my own work;\\
5. I have acknowledged all main sources of help;\\
6. Where the thesis is based on work done by myself jointly with others, I have made clear exactly what was done by others and what I have contributed myself;\\
7. Either none of this work has been published before submission, or parts of this work have been published by :\\
\\
Stefan Collier\\
April 2016
}
\tableofcontents
\listoffigures
\listoftables

\mainmatter
%% ----------------------------------------------------------------
%\include{Introduction}
%\include{Conclusions}
\include{chapters/1Project/main}
\include{chapters/2Lit/main}
\include{chapters/3Design/HighLevel}
\include{chapters/3Design/InDepth}
\include{chapters/4Impl/main}

\include{chapters/5Experiments/1/main}
\include{chapters/5Experiments/2/main}
\include{chapters/5Experiments/3/main}
\include{chapters/5Experiments/4/main}

\include{chapters/6Conclusion/main}

\appendix
\include{appendix/AppendixB}
\include{appendix/D/main}
\include{appendix/AppendixC}

\backmatter
\bibliographystyle{ecs}
\bibliography{ECS}
\end{document}
%% ----------------------------------------------------------------


 %% ----------------------------------------------------------------
%% Progress.tex
%% ---------------------------------------------------------------- 
\documentclass{ecsprogress}    % Use the progress Style
\graphicspath{{../figs/}}   % Location of your graphics files
    \usepackage{natbib}            % Use Natbib style for the refs.
\hypersetup{colorlinks=true}   % Set to false for black/white printing
\input{Definitions}            % Include your abbreviations



\usepackage{enumitem}% http://ctan.org/pkg/enumitem
\usepackage{multirow}
\usepackage{float}
\usepackage{amsmath}
\usepackage{multicol}
\usepackage{amssymb}
\usepackage[normalem]{ulem}
\useunder{\uline}{\ul}{}
\usepackage{wrapfig}


\usepackage[table,xcdraw]{xcolor}


%% ----------------------------------------------------------------
\begin{document}
\frontmatter
\title      {Heterogeneous Agent-based Model for Supermarket Competition}
\authors    {\texorpdfstring
             {\href{mailto:sc22g13@ecs.soton.ac.uk}{Stefan J. Collier}}
             {Stefan J. Collier}
            }
\addresses  {\groupname\\\deptname\\\univname}
\date       {\today}
\subject    {}
\keywords   {}
\supervisor {Dr. Maria Polukarov}
\examiner   {Professor Sheng Chen}

\maketitle
\begin{abstract}
This project aim was to model and analyse the effects of competitive pricing behaviors of grocery retailers on the British market. 

This was achieved by creating a multi-agent model, containing retailer and consumer agents. The heterogeneous crowd of retailers employs either a uniform pricing strategy or a ‘local price flexing’ strategy. The actions of these retailers are chosen by predicting the profit of each action, using a perceptron. Following on from the consideration of different economic models, a discrete model was developed so that software agents have a discrete environment to operate within. Within the model, it has been observed how supermarkets with differing behaviors affect a heterogeneous crowd of consumer agents. The model was implemented in Java with Python used to evaluate the results. 

The simulation displays good acceptance with real grocery market behavior, i.e. captures the performance of British retailers thus can be used to determine the impact of changes in their behavior on their competitors and consumers.Furthermore it can be used to provide insight into sustainability of volatile pricing strategies, providing a useful insight in volatility of British supermarket retail industry. 
\end{abstract}
\acknowledgements{
I would like to express my sincere gratitude to Dr Maria Polukarov for her guidance and support which provided me the freedom to take this research in the direction of my interest.\\
\\
I would also like to thank my family and friends for their encouragement and support. To those who quietly listened to my software complaints. To those who worked throughout the nights with me. To those who helped me write what I couldn't say. I cannot thank you enough.
}

\declaration{
I, Stefan Collier, declare that this dissertation and the work presented in it are my own and has been generated by me as the result of my own original research.\\
I confirm that:\\
1. This work was done wholly or mainly while in candidature for a degree at this University;\\
2. Where any part of this dissertation has previously been submitted for any other qualification at this University or any other institution, this has been clearly stated;\\
3. Where I have consulted the published work of others, this is always clearly attributed;\\
4. Where I have quoted from the work of others, the source is always given. With the exception of such quotations, this dissertation is entirely my own work;\\
5. I have acknowledged all main sources of help;\\
6. Where the thesis is based on work done by myself jointly with others, I have made clear exactly what was done by others and what I have contributed myself;\\
7. Either none of this work has been published before submission, or parts of this work have been published by :\\
\\
Stefan Collier\\
April 2016
}
\tableofcontents
\listoffigures
\listoftables

\mainmatter
%% ----------------------------------------------------------------
%\include{Introduction}
%\include{Conclusions}
\include{chapters/1Project/main}
\include{chapters/2Lit/main}
\include{chapters/3Design/HighLevel}
\include{chapters/3Design/InDepth}
\include{chapters/4Impl/main}

\include{chapters/5Experiments/1/main}
\include{chapters/5Experiments/2/main}
\include{chapters/5Experiments/3/main}
\include{chapters/5Experiments/4/main}

\include{chapters/6Conclusion/main}

\appendix
\include{appendix/AppendixB}
\include{appendix/D/main}
\include{appendix/AppendixC}

\backmatter
\bibliographystyle{ecs}
\bibliography{ECS}
\end{document}
%% ----------------------------------------------------------------

 %% ----------------------------------------------------------------
%% Progress.tex
%% ---------------------------------------------------------------- 
\documentclass{ecsprogress}    % Use the progress Style
\graphicspath{{../figs/}}   % Location of your graphics files
    \usepackage{natbib}            % Use Natbib style for the refs.
\hypersetup{colorlinks=true}   % Set to false for black/white printing
\input{Definitions}            % Include your abbreviations



\usepackage{enumitem}% http://ctan.org/pkg/enumitem
\usepackage{multirow}
\usepackage{float}
\usepackage{amsmath}
\usepackage{multicol}
\usepackage{amssymb}
\usepackage[normalem]{ulem}
\useunder{\uline}{\ul}{}
\usepackage{wrapfig}


\usepackage[table,xcdraw]{xcolor}


%% ----------------------------------------------------------------
\begin{document}
\frontmatter
\title      {Heterogeneous Agent-based Model for Supermarket Competition}
\authors    {\texorpdfstring
             {\href{mailto:sc22g13@ecs.soton.ac.uk}{Stefan J. Collier}}
             {Stefan J. Collier}
            }
\addresses  {\groupname\\\deptname\\\univname}
\date       {\today}
\subject    {}
\keywords   {}
\supervisor {Dr. Maria Polukarov}
\examiner   {Professor Sheng Chen}

\maketitle
\begin{abstract}
This project aim was to model and analyse the effects of competitive pricing behaviors of grocery retailers on the British market. 

This was achieved by creating a multi-agent model, containing retailer and consumer agents. The heterogeneous crowd of retailers employs either a uniform pricing strategy or a ‘local price flexing’ strategy. The actions of these retailers are chosen by predicting the profit of each action, using a perceptron. Following on from the consideration of different economic models, a discrete model was developed so that software agents have a discrete environment to operate within. Within the model, it has been observed how supermarkets with differing behaviors affect a heterogeneous crowd of consumer agents. The model was implemented in Java with Python used to evaluate the results. 

The simulation displays good acceptance with real grocery market behavior, i.e. captures the performance of British retailers thus can be used to determine the impact of changes in their behavior on their competitors and consumers.Furthermore it can be used to provide insight into sustainability of volatile pricing strategies, providing a useful insight in volatility of British supermarket retail industry. 
\end{abstract}
\acknowledgements{
I would like to express my sincere gratitude to Dr Maria Polukarov for her guidance and support which provided me the freedom to take this research in the direction of my interest.\\
\\
I would also like to thank my family and friends for their encouragement and support. To those who quietly listened to my software complaints. To those who worked throughout the nights with me. To those who helped me write what I couldn't say. I cannot thank you enough.
}

\declaration{
I, Stefan Collier, declare that this dissertation and the work presented in it are my own and has been generated by me as the result of my own original research.\\
I confirm that:\\
1. This work was done wholly or mainly while in candidature for a degree at this University;\\
2. Where any part of this dissertation has previously been submitted for any other qualification at this University or any other institution, this has been clearly stated;\\
3. Where I have consulted the published work of others, this is always clearly attributed;\\
4. Where I have quoted from the work of others, the source is always given. With the exception of such quotations, this dissertation is entirely my own work;\\
5. I have acknowledged all main sources of help;\\
6. Where the thesis is based on work done by myself jointly with others, I have made clear exactly what was done by others and what I have contributed myself;\\
7. Either none of this work has been published before submission, or parts of this work have been published by :\\
\\
Stefan Collier\\
April 2016
}
\tableofcontents
\listoffigures
\listoftables

\mainmatter
%% ----------------------------------------------------------------
%\include{Introduction}
%\include{Conclusions}
\include{chapters/1Project/main}
\include{chapters/2Lit/main}
\include{chapters/3Design/HighLevel}
\include{chapters/3Design/InDepth}
\include{chapters/4Impl/main}

\include{chapters/5Experiments/1/main}
\include{chapters/5Experiments/2/main}
\include{chapters/5Experiments/3/main}
\include{chapters/5Experiments/4/main}

\include{chapters/6Conclusion/main}

\appendix
\include{appendix/AppendixB}
\include{appendix/D/main}
\include{appendix/AppendixC}

\backmatter
\bibliographystyle{ecs}
\bibliography{ECS}
\end{document}
%% ----------------------------------------------------------------

 %% ----------------------------------------------------------------
%% Progress.tex
%% ---------------------------------------------------------------- 
\documentclass{ecsprogress}    % Use the progress Style
\graphicspath{{../figs/}}   % Location of your graphics files
    \usepackage{natbib}            % Use Natbib style for the refs.
\hypersetup{colorlinks=true}   % Set to false for black/white printing
\input{Definitions}            % Include your abbreviations



\usepackage{enumitem}% http://ctan.org/pkg/enumitem
\usepackage{multirow}
\usepackage{float}
\usepackage{amsmath}
\usepackage{multicol}
\usepackage{amssymb}
\usepackage[normalem]{ulem}
\useunder{\uline}{\ul}{}
\usepackage{wrapfig}


\usepackage[table,xcdraw]{xcolor}


%% ----------------------------------------------------------------
\begin{document}
\frontmatter
\title      {Heterogeneous Agent-based Model for Supermarket Competition}
\authors    {\texorpdfstring
             {\href{mailto:sc22g13@ecs.soton.ac.uk}{Stefan J. Collier}}
             {Stefan J. Collier}
            }
\addresses  {\groupname\\\deptname\\\univname}
\date       {\today}
\subject    {}
\keywords   {}
\supervisor {Dr. Maria Polukarov}
\examiner   {Professor Sheng Chen}

\maketitle
\begin{abstract}
This project aim was to model and analyse the effects of competitive pricing behaviors of grocery retailers on the British market. 

This was achieved by creating a multi-agent model, containing retailer and consumer agents. The heterogeneous crowd of retailers employs either a uniform pricing strategy or a ‘local price flexing’ strategy. The actions of these retailers are chosen by predicting the profit of each action, using a perceptron. Following on from the consideration of different economic models, a discrete model was developed so that software agents have a discrete environment to operate within. Within the model, it has been observed how supermarkets with differing behaviors affect a heterogeneous crowd of consumer agents. The model was implemented in Java with Python used to evaluate the results. 

The simulation displays good acceptance with real grocery market behavior, i.e. captures the performance of British retailers thus can be used to determine the impact of changes in their behavior on their competitors and consumers.Furthermore it can be used to provide insight into sustainability of volatile pricing strategies, providing a useful insight in volatility of British supermarket retail industry. 
\end{abstract}
\acknowledgements{
I would like to express my sincere gratitude to Dr Maria Polukarov for her guidance and support which provided me the freedom to take this research in the direction of my interest.\\
\\
I would also like to thank my family and friends for their encouragement and support. To those who quietly listened to my software complaints. To those who worked throughout the nights with me. To those who helped me write what I couldn't say. I cannot thank you enough.
}

\declaration{
I, Stefan Collier, declare that this dissertation and the work presented in it are my own and has been generated by me as the result of my own original research.\\
I confirm that:\\
1. This work was done wholly or mainly while in candidature for a degree at this University;\\
2. Where any part of this dissertation has previously been submitted for any other qualification at this University or any other institution, this has been clearly stated;\\
3. Where I have consulted the published work of others, this is always clearly attributed;\\
4. Where I have quoted from the work of others, the source is always given. With the exception of such quotations, this dissertation is entirely my own work;\\
5. I have acknowledged all main sources of help;\\
6. Where the thesis is based on work done by myself jointly with others, I have made clear exactly what was done by others and what I have contributed myself;\\
7. Either none of this work has been published before submission, or parts of this work have been published by :\\
\\
Stefan Collier\\
April 2016
}
\tableofcontents
\listoffigures
\listoftables

\mainmatter
%% ----------------------------------------------------------------
%\include{Introduction}
%\include{Conclusions}
\include{chapters/1Project/main}
\include{chapters/2Lit/main}
\include{chapters/3Design/HighLevel}
\include{chapters/3Design/InDepth}
\include{chapters/4Impl/main}

\include{chapters/5Experiments/1/main}
\include{chapters/5Experiments/2/main}
\include{chapters/5Experiments/3/main}
\include{chapters/5Experiments/4/main}

\include{chapters/6Conclusion/main}

\appendix
\include{appendix/AppendixB}
\include{appendix/D/main}
\include{appendix/AppendixC}

\backmatter
\bibliographystyle{ecs}
\bibliography{ECS}
\end{document}
%% ----------------------------------------------------------------

 %% ----------------------------------------------------------------
%% Progress.tex
%% ---------------------------------------------------------------- 
\documentclass{ecsprogress}    % Use the progress Style
\graphicspath{{../figs/}}   % Location of your graphics files
    \usepackage{natbib}            % Use Natbib style for the refs.
\hypersetup{colorlinks=true}   % Set to false for black/white printing
\input{Definitions}            % Include your abbreviations



\usepackage{enumitem}% http://ctan.org/pkg/enumitem
\usepackage{multirow}
\usepackage{float}
\usepackage{amsmath}
\usepackage{multicol}
\usepackage{amssymb}
\usepackage[normalem]{ulem}
\useunder{\uline}{\ul}{}
\usepackage{wrapfig}


\usepackage[table,xcdraw]{xcolor}


%% ----------------------------------------------------------------
\begin{document}
\frontmatter
\title      {Heterogeneous Agent-based Model for Supermarket Competition}
\authors    {\texorpdfstring
             {\href{mailto:sc22g13@ecs.soton.ac.uk}{Stefan J. Collier}}
             {Stefan J. Collier}
            }
\addresses  {\groupname\\\deptname\\\univname}
\date       {\today}
\subject    {}
\keywords   {}
\supervisor {Dr. Maria Polukarov}
\examiner   {Professor Sheng Chen}

\maketitle
\begin{abstract}
This project aim was to model and analyse the effects of competitive pricing behaviors of grocery retailers on the British market. 

This was achieved by creating a multi-agent model, containing retailer and consumer agents. The heterogeneous crowd of retailers employs either a uniform pricing strategy or a ‘local price flexing’ strategy. The actions of these retailers are chosen by predicting the profit of each action, using a perceptron. Following on from the consideration of different economic models, a discrete model was developed so that software agents have a discrete environment to operate within. Within the model, it has been observed how supermarkets with differing behaviors affect a heterogeneous crowd of consumer agents. The model was implemented in Java with Python used to evaluate the results. 

The simulation displays good acceptance with real grocery market behavior, i.e. captures the performance of British retailers thus can be used to determine the impact of changes in their behavior on their competitors and consumers.Furthermore it can be used to provide insight into sustainability of volatile pricing strategies, providing a useful insight in volatility of British supermarket retail industry. 
\end{abstract}
\acknowledgements{
I would like to express my sincere gratitude to Dr Maria Polukarov for her guidance and support which provided me the freedom to take this research in the direction of my interest.\\
\\
I would also like to thank my family and friends for their encouragement and support. To those who quietly listened to my software complaints. To those who worked throughout the nights with me. To those who helped me write what I couldn't say. I cannot thank you enough.
}

\declaration{
I, Stefan Collier, declare that this dissertation and the work presented in it are my own and has been generated by me as the result of my own original research.\\
I confirm that:\\
1. This work was done wholly or mainly while in candidature for a degree at this University;\\
2. Where any part of this dissertation has previously been submitted for any other qualification at this University or any other institution, this has been clearly stated;\\
3. Where I have consulted the published work of others, this is always clearly attributed;\\
4. Where I have quoted from the work of others, the source is always given. With the exception of such quotations, this dissertation is entirely my own work;\\
5. I have acknowledged all main sources of help;\\
6. Where the thesis is based on work done by myself jointly with others, I have made clear exactly what was done by others and what I have contributed myself;\\
7. Either none of this work has been published before submission, or parts of this work have been published by :\\
\\
Stefan Collier\\
April 2016
}
\tableofcontents
\listoffigures
\listoftables

\mainmatter
%% ----------------------------------------------------------------
%\include{Introduction}
%\include{Conclusions}
\include{chapters/1Project/main}
\include{chapters/2Lit/main}
\include{chapters/3Design/HighLevel}
\include{chapters/3Design/InDepth}
\include{chapters/4Impl/main}

\include{chapters/5Experiments/1/main}
\include{chapters/5Experiments/2/main}
\include{chapters/5Experiments/3/main}
\include{chapters/5Experiments/4/main}

\include{chapters/6Conclusion/main}

\appendix
\include{appendix/AppendixB}
\include{appendix/D/main}
\include{appendix/AppendixC}

\backmatter
\bibliographystyle{ecs}
\bibliography{ECS}
\end{document}
%% ----------------------------------------------------------------


 %% ----------------------------------------------------------------
%% Progress.tex
%% ---------------------------------------------------------------- 
\documentclass{ecsprogress}    % Use the progress Style
\graphicspath{{../figs/}}   % Location of your graphics files
    \usepackage{natbib}            % Use Natbib style for the refs.
\hypersetup{colorlinks=true}   % Set to false for black/white printing
\input{Definitions}            % Include your abbreviations



\usepackage{enumitem}% http://ctan.org/pkg/enumitem
\usepackage{multirow}
\usepackage{float}
\usepackage{amsmath}
\usepackage{multicol}
\usepackage{amssymb}
\usepackage[normalem]{ulem}
\useunder{\uline}{\ul}{}
\usepackage{wrapfig}


\usepackage[table,xcdraw]{xcolor}


%% ----------------------------------------------------------------
\begin{document}
\frontmatter
\title      {Heterogeneous Agent-based Model for Supermarket Competition}
\authors    {\texorpdfstring
             {\href{mailto:sc22g13@ecs.soton.ac.uk}{Stefan J. Collier}}
             {Stefan J. Collier}
            }
\addresses  {\groupname\\\deptname\\\univname}
\date       {\today}
\subject    {}
\keywords   {}
\supervisor {Dr. Maria Polukarov}
\examiner   {Professor Sheng Chen}

\maketitle
\begin{abstract}
This project aim was to model and analyse the effects of competitive pricing behaviors of grocery retailers on the British market. 

This was achieved by creating a multi-agent model, containing retailer and consumer agents. The heterogeneous crowd of retailers employs either a uniform pricing strategy or a ‘local price flexing’ strategy. The actions of these retailers are chosen by predicting the profit of each action, using a perceptron. Following on from the consideration of different economic models, a discrete model was developed so that software agents have a discrete environment to operate within. Within the model, it has been observed how supermarkets with differing behaviors affect a heterogeneous crowd of consumer agents. The model was implemented in Java with Python used to evaluate the results. 

The simulation displays good acceptance with real grocery market behavior, i.e. captures the performance of British retailers thus can be used to determine the impact of changes in their behavior on their competitors and consumers.Furthermore it can be used to provide insight into sustainability of volatile pricing strategies, providing a useful insight in volatility of British supermarket retail industry. 
\end{abstract}
\acknowledgements{
I would like to express my sincere gratitude to Dr Maria Polukarov for her guidance and support which provided me the freedom to take this research in the direction of my interest.\\
\\
I would also like to thank my family and friends for their encouragement and support. To those who quietly listened to my software complaints. To those who worked throughout the nights with me. To those who helped me write what I couldn't say. I cannot thank you enough.
}

\declaration{
I, Stefan Collier, declare that this dissertation and the work presented in it are my own and has been generated by me as the result of my own original research.\\
I confirm that:\\
1. This work was done wholly or mainly while in candidature for a degree at this University;\\
2. Where any part of this dissertation has previously been submitted for any other qualification at this University or any other institution, this has been clearly stated;\\
3. Where I have consulted the published work of others, this is always clearly attributed;\\
4. Where I have quoted from the work of others, the source is always given. With the exception of such quotations, this dissertation is entirely my own work;\\
5. I have acknowledged all main sources of help;\\
6. Where the thesis is based on work done by myself jointly with others, I have made clear exactly what was done by others and what I have contributed myself;\\
7. Either none of this work has been published before submission, or parts of this work have been published by :\\
\\
Stefan Collier\\
April 2016
}
\tableofcontents
\listoffigures
\listoftables

\mainmatter
%% ----------------------------------------------------------------
%\include{Introduction}
%\include{Conclusions}
\include{chapters/1Project/main}
\include{chapters/2Lit/main}
\include{chapters/3Design/HighLevel}
\include{chapters/3Design/InDepth}
\include{chapters/4Impl/main}

\include{chapters/5Experiments/1/main}
\include{chapters/5Experiments/2/main}
\include{chapters/5Experiments/3/main}
\include{chapters/5Experiments/4/main}

\include{chapters/6Conclusion/main}

\appendix
\include{appendix/AppendixB}
\include{appendix/D/main}
\include{appendix/AppendixC}

\backmatter
\bibliographystyle{ecs}
\bibliography{ECS}
\end{document}
%% ----------------------------------------------------------------


\appendix
\include{appendix/AppendixB}
 %% ----------------------------------------------------------------
%% Progress.tex
%% ---------------------------------------------------------------- 
\documentclass{ecsprogress}    % Use the progress Style
\graphicspath{{../figs/}}   % Location of your graphics files
    \usepackage{natbib}            % Use Natbib style for the refs.
\hypersetup{colorlinks=true}   % Set to false for black/white printing
\input{Definitions}            % Include your abbreviations



\usepackage{enumitem}% http://ctan.org/pkg/enumitem
\usepackage{multirow}
\usepackage{float}
\usepackage{amsmath}
\usepackage{multicol}
\usepackage{amssymb}
\usepackage[normalem]{ulem}
\useunder{\uline}{\ul}{}
\usepackage{wrapfig}


\usepackage[table,xcdraw]{xcolor}


%% ----------------------------------------------------------------
\begin{document}
\frontmatter
\title      {Heterogeneous Agent-based Model for Supermarket Competition}
\authors    {\texorpdfstring
             {\href{mailto:sc22g13@ecs.soton.ac.uk}{Stefan J. Collier}}
             {Stefan J. Collier}
            }
\addresses  {\groupname\\\deptname\\\univname}
\date       {\today}
\subject    {}
\keywords   {}
\supervisor {Dr. Maria Polukarov}
\examiner   {Professor Sheng Chen}

\maketitle
\begin{abstract}
This project aim was to model and analyse the effects of competitive pricing behaviors of grocery retailers on the British market. 

This was achieved by creating a multi-agent model, containing retailer and consumer agents. The heterogeneous crowd of retailers employs either a uniform pricing strategy or a ‘local price flexing’ strategy. The actions of these retailers are chosen by predicting the profit of each action, using a perceptron. Following on from the consideration of different economic models, a discrete model was developed so that software agents have a discrete environment to operate within. Within the model, it has been observed how supermarkets with differing behaviors affect a heterogeneous crowd of consumer agents. The model was implemented in Java with Python used to evaluate the results. 

The simulation displays good acceptance with real grocery market behavior, i.e. captures the performance of British retailers thus can be used to determine the impact of changes in their behavior on their competitors and consumers.Furthermore it can be used to provide insight into sustainability of volatile pricing strategies, providing a useful insight in volatility of British supermarket retail industry. 
\end{abstract}
\acknowledgements{
I would like to express my sincere gratitude to Dr Maria Polukarov for her guidance and support which provided me the freedom to take this research in the direction of my interest.\\
\\
I would also like to thank my family and friends for their encouragement and support. To those who quietly listened to my software complaints. To those who worked throughout the nights with me. To those who helped me write what I couldn't say. I cannot thank you enough.
}

\declaration{
I, Stefan Collier, declare that this dissertation and the work presented in it are my own and has been generated by me as the result of my own original research.\\
I confirm that:\\
1. This work was done wholly or mainly while in candidature for a degree at this University;\\
2. Where any part of this dissertation has previously been submitted for any other qualification at this University or any other institution, this has been clearly stated;\\
3. Where I have consulted the published work of others, this is always clearly attributed;\\
4. Where I have quoted from the work of others, the source is always given. With the exception of such quotations, this dissertation is entirely my own work;\\
5. I have acknowledged all main sources of help;\\
6. Where the thesis is based on work done by myself jointly with others, I have made clear exactly what was done by others and what I have contributed myself;\\
7. Either none of this work has been published before submission, or parts of this work have been published by :\\
\\
Stefan Collier\\
April 2016
}
\tableofcontents
\listoffigures
\listoftables

\mainmatter
%% ----------------------------------------------------------------
%\include{Introduction}
%\include{Conclusions}
\include{chapters/1Project/main}
\include{chapters/2Lit/main}
\include{chapters/3Design/HighLevel}
\include{chapters/3Design/InDepth}
\include{chapters/4Impl/main}

\include{chapters/5Experiments/1/main}
\include{chapters/5Experiments/2/main}
\include{chapters/5Experiments/3/main}
\include{chapters/5Experiments/4/main}

\include{chapters/6Conclusion/main}

\appendix
\include{appendix/AppendixB}
\include{appendix/D/main}
\include{appendix/AppendixC}

\backmatter
\bibliographystyle{ecs}
\bibliography{ECS}
\end{document}
%% ----------------------------------------------------------------

\include{appendix/AppendixC}

\backmatter
\bibliographystyle{ecs}
\bibliography{ECS}
\end{document}
%% ----------------------------------------------------------------

 %% ----------------------------------------------------------------
%% Progress.tex
%% ---------------------------------------------------------------- 
\documentclass{ecsprogress}    % Use the progress Style
\graphicspath{{../figs/}}   % Location of your graphics files
    \usepackage{natbib}            % Use Natbib style for the refs.
\hypersetup{colorlinks=true}   % Set to false for black/white printing
\input{Definitions}            % Include your abbreviations



\usepackage{enumitem}% http://ctan.org/pkg/enumitem
\usepackage{multirow}
\usepackage{float}
\usepackage{amsmath}
\usepackage{multicol}
\usepackage{amssymb}
\usepackage[normalem]{ulem}
\useunder{\uline}{\ul}{}
\usepackage{wrapfig}


\usepackage[table,xcdraw]{xcolor}


%% ----------------------------------------------------------------
\begin{document}
\frontmatter
\title      {Heterogeneous Agent-based Model for Supermarket Competition}
\authors    {\texorpdfstring
             {\href{mailto:sc22g13@ecs.soton.ac.uk}{Stefan J. Collier}}
             {Stefan J. Collier}
            }
\addresses  {\groupname\\\deptname\\\univname}
\date       {\today}
\subject    {}
\keywords   {}
\supervisor {Dr. Maria Polukarov}
\examiner   {Professor Sheng Chen}

\maketitle
\begin{abstract}
This project aim was to model and analyse the effects of competitive pricing behaviors of grocery retailers on the British market. 

This was achieved by creating a multi-agent model, containing retailer and consumer agents. The heterogeneous crowd of retailers employs either a uniform pricing strategy or a ‘local price flexing’ strategy. The actions of these retailers are chosen by predicting the profit of each action, using a perceptron. Following on from the consideration of different economic models, a discrete model was developed so that software agents have a discrete environment to operate within. Within the model, it has been observed how supermarkets with differing behaviors affect a heterogeneous crowd of consumer agents. The model was implemented in Java with Python used to evaluate the results. 

The simulation displays good acceptance with real grocery market behavior, i.e. captures the performance of British retailers thus can be used to determine the impact of changes in their behavior on their competitors and consumers.Furthermore it can be used to provide insight into sustainability of volatile pricing strategies, providing a useful insight in volatility of British supermarket retail industry. 
\end{abstract}
\acknowledgements{
I would like to express my sincere gratitude to Dr Maria Polukarov for her guidance and support which provided me the freedom to take this research in the direction of my interest.\\
\\
I would also like to thank my family and friends for their encouragement and support. To those who quietly listened to my software complaints. To those who worked throughout the nights with me. To those who helped me write what I couldn't say. I cannot thank you enough.
}

\declaration{
I, Stefan Collier, declare that this dissertation and the work presented in it are my own and has been generated by me as the result of my own original research.\\
I confirm that:\\
1. This work was done wholly or mainly while in candidature for a degree at this University;\\
2. Where any part of this dissertation has previously been submitted for any other qualification at this University or any other institution, this has been clearly stated;\\
3. Where I have consulted the published work of others, this is always clearly attributed;\\
4. Where I have quoted from the work of others, the source is always given. With the exception of such quotations, this dissertation is entirely my own work;\\
5. I have acknowledged all main sources of help;\\
6. Where the thesis is based on work done by myself jointly with others, I have made clear exactly what was done by others and what I have contributed myself;\\
7. Either none of this work has been published before submission, or parts of this work have been published by :\\
\\
Stefan Collier\\
April 2016
}
\tableofcontents
\listoffigures
\listoftables

\mainmatter
%% ----------------------------------------------------------------
%\include{Introduction}
%\include{Conclusions}
 %% ----------------------------------------------------------------
%% Progress.tex
%% ---------------------------------------------------------------- 
\documentclass{ecsprogress}    % Use the progress Style
\graphicspath{{../figs/}}   % Location of your graphics files
    \usepackage{natbib}            % Use Natbib style for the refs.
\hypersetup{colorlinks=true}   % Set to false for black/white printing
\input{Definitions}            % Include your abbreviations



\usepackage{enumitem}% http://ctan.org/pkg/enumitem
\usepackage{multirow}
\usepackage{float}
\usepackage{amsmath}
\usepackage{multicol}
\usepackage{amssymb}
\usepackage[normalem]{ulem}
\useunder{\uline}{\ul}{}
\usepackage{wrapfig}


\usepackage[table,xcdraw]{xcolor}


%% ----------------------------------------------------------------
\begin{document}
\frontmatter
\title      {Heterogeneous Agent-based Model for Supermarket Competition}
\authors    {\texorpdfstring
             {\href{mailto:sc22g13@ecs.soton.ac.uk}{Stefan J. Collier}}
             {Stefan J. Collier}
            }
\addresses  {\groupname\\\deptname\\\univname}
\date       {\today}
\subject    {}
\keywords   {}
\supervisor {Dr. Maria Polukarov}
\examiner   {Professor Sheng Chen}

\maketitle
\begin{abstract}
This project aim was to model and analyse the effects of competitive pricing behaviors of grocery retailers on the British market. 

This was achieved by creating a multi-agent model, containing retailer and consumer agents. The heterogeneous crowd of retailers employs either a uniform pricing strategy or a ‘local price flexing’ strategy. The actions of these retailers are chosen by predicting the profit of each action, using a perceptron. Following on from the consideration of different economic models, a discrete model was developed so that software agents have a discrete environment to operate within. Within the model, it has been observed how supermarkets with differing behaviors affect a heterogeneous crowd of consumer agents. The model was implemented in Java with Python used to evaluate the results. 

The simulation displays good acceptance with real grocery market behavior, i.e. captures the performance of British retailers thus can be used to determine the impact of changes in their behavior on their competitors and consumers.Furthermore it can be used to provide insight into sustainability of volatile pricing strategies, providing a useful insight in volatility of British supermarket retail industry. 
\end{abstract}
\acknowledgements{
I would like to express my sincere gratitude to Dr Maria Polukarov for her guidance and support which provided me the freedom to take this research in the direction of my interest.\\
\\
I would also like to thank my family and friends for their encouragement and support. To those who quietly listened to my software complaints. To those who worked throughout the nights with me. To those who helped me write what I couldn't say. I cannot thank you enough.
}

\declaration{
I, Stefan Collier, declare that this dissertation and the work presented in it are my own and has been generated by me as the result of my own original research.\\
I confirm that:\\
1. This work was done wholly or mainly while in candidature for a degree at this University;\\
2. Where any part of this dissertation has previously been submitted for any other qualification at this University or any other institution, this has been clearly stated;\\
3. Where I have consulted the published work of others, this is always clearly attributed;\\
4. Where I have quoted from the work of others, the source is always given. With the exception of such quotations, this dissertation is entirely my own work;\\
5. I have acknowledged all main sources of help;\\
6. Where the thesis is based on work done by myself jointly with others, I have made clear exactly what was done by others and what I have contributed myself;\\
7. Either none of this work has been published before submission, or parts of this work have been published by :\\
\\
Stefan Collier\\
April 2016
}
\tableofcontents
\listoffigures
\listoftables

\mainmatter
%% ----------------------------------------------------------------
%\include{Introduction}
%\include{Conclusions}
\include{chapters/1Project/main}
\include{chapters/2Lit/main}
\include{chapters/3Design/HighLevel}
\include{chapters/3Design/InDepth}
\include{chapters/4Impl/main}

\include{chapters/5Experiments/1/main}
\include{chapters/5Experiments/2/main}
\include{chapters/5Experiments/3/main}
\include{chapters/5Experiments/4/main}

\include{chapters/6Conclusion/main}

\appendix
\include{appendix/AppendixB}
\include{appendix/D/main}
\include{appendix/AppendixC}

\backmatter
\bibliographystyle{ecs}
\bibliography{ECS}
\end{document}
%% ----------------------------------------------------------------

 %% ----------------------------------------------------------------
%% Progress.tex
%% ---------------------------------------------------------------- 
\documentclass{ecsprogress}    % Use the progress Style
\graphicspath{{../figs/}}   % Location of your graphics files
    \usepackage{natbib}            % Use Natbib style for the refs.
\hypersetup{colorlinks=true}   % Set to false for black/white printing
\input{Definitions}            % Include your abbreviations



\usepackage{enumitem}% http://ctan.org/pkg/enumitem
\usepackage{multirow}
\usepackage{float}
\usepackage{amsmath}
\usepackage{multicol}
\usepackage{amssymb}
\usepackage[normalem]{ulem}
\useunder{\uline}{\ul}{}
\usepackage{wrapfig}


\usepackage[table,xcdraw]{xcolor}


%% ----------------------------------------------------------------
\begin{document}
\frontmatter
\title      {Heterogeneous Agent-based Model for Supermarket Competition}
\authors    {\texorpdfstring
             {\href{mailto:sc22g13@ecs.soton.ac.uk}{Stefan J. Collier}}
             {Stefan J. Collier}
            }
\addresses  {\groupname\\\deptname\\\univname}
\date       {\today}
\subject    {}
\keywords   {}
\supervisor {Dr. Maria Polukarov}
\examiner   {Professor Sheng Chen}

\maketitle
\begin{abstract}
This project aim was to model and analyse the effects of competitive pricing behaviors of grocery retailers on the British market. 

This was achieved by creating a multi-agent model, containing retailer and consumer agents. The heterogeneous crowd of retailers employs either a uniform pricing strategy or a ‘local price flexing’ strategy. The actions of these retailers are chosen by predicting the profit of each action, using a perceptron. Following on from the consideration of different economic models, a discrete model was developed so that software agents have a discrete environment to operate within. Within the model, it has been observed how supermarkets with differing behaviors affect a heterogeneous crowd of consumer agents. The model was implemented in Java with Python used to evaluate the results. 

The simulation displays good acceptance with real grocery market behavior, i.e. captures the performance of British retailers thus can be used to determine the impact of changes in their behavior on their competitors and consumers.Furthermore it can be used to provide insight into sustainability of volatile pricing strategies, providing a useful insight in volatility of British supermarket retail industry. 
\end{abstract}
\acknowledgements{
I would like to express my sincere gratitude to Dr Maria Polukarov for her guidance and support which provided me the freedom to take this research in the direction of my interest.\\
\\
I would also like to thank my family and friends for their encouragement and support. To those who quietly listened to my software complaints. To those who worked throughout the nights with me. To those who helped me write what I couldn't say. I cannot thank you enough.
}

\declaration{
I, Stefan Collier, declare that this dissertation and the work presented in it are my own and has been generated by me as the result of my own original research.\\
I confirm that:\\
1. This work was done wholly or mainly while in candidature for a degree at this University;\\
2. Where any part of this dissertation has previously been submitted for any other qualification at this University or any other institution, this has been clearly stated;\\
3. Where I have consulted the published work of others, this is always clearly attributed;\\
4. Where I have quoted from the work of others, the source is always given. With the exception of such quotations, this dissertation is entirely my own work;\\
5. I have acknowledged all main sources of help;\\
6. Where the thesis is based on work done by myself jointly with others, I have made clear exactly what was done by others and what I have contributed myself;\\
7. Either none of this work has been published before submission, or parts of this work have been published by :\\
\\
Stefan Collier\\
April 2016
}
\tableofcontents
\listoffigures
\listoftables

\mainmatter
%% ----------------------------------------------------------------
%\include{Introduction}
%\include{Conclusions}
\include{chapters/1Project/main}
\include{chapters/2Lit/main}
\include{chapters/3Design/HighLevel}
\include{chapters/3Design/InDepth}
\include{chapters/4Impl/main}

\include{chapters/5Experiments/1/main}
\include{chapters/5Experiments/2/main}
\include{chapters/5Experiments/3/main}
\include{chapters/5Experiments/4/main}

\include{chapters/6Conclusion/main}

\appendix
\include{appendix/AppendixB}
\include{appendix/D/main}
\include{appendix/AppendixC}

\backmatter
\bibliographystyle{ecs}
\bibliography{ECS}
\end{document}
%% ----------------------------------------------------------------

\include{chapters/3Design/HighLevel}
\include{chapters/3Design/InDepth}
 %% ----------------------------------------------------------------
%% Progress.tex
%% ---------------------------------------------------------------- 
\documentclass{ecsprogress}    % Use the progress Style
\graphicspath{{../figs/}}   % Location of your graphics files
    \usepackage{natbib}            % Use Natbib style for the refs.
\hypersetup{colorlinks=true}   % Set to false for black/white printing
\input{Definitions}            % Include your abbreviations



\usepackage{enumitem}% http://ctan.org/pkg/enumitem
\usepackage{multirow}
\usepackage{float}
\usepackage{amsmath}
\usepackage{multicol}
\usepackage{amssymb}
\usepackage[normalem]{ulem}
\useunder{\uline}{\ul}{}
\usepackage{wrapfig}


\usepackage[table,xcdraw]{xcolor}


%% ----------------------------------------------------------------
\begin{document}
\frontmatter
\title      {Heterogeneous Agent-based Model for Supermarket Competition}
\authors    {\texorpdfstring
             {\href{mailto:sc22g13@ecs.soton.ac.uk}{Stefan J. Collier}}
             {Stefan J. Collier}
            }
\addresses  {\groupname\\\deptname\\\univname}
\date       {\today}
\subject    {}
\keywords   {}
\supervisor {Dr. Maria Polukarov}
\examiner   {Professor Sheng Chen}

\maketitle
\begin{abstract}
This project aim was to model and analyse the effects of competitive pricing behaviors of grocery retailers on the British market. 

This was achieved by creating a multi-agent model, containing retailer and consumer agents. The heterogeneous crowd of retailers employs either a uniform pricing strategy or a ‘local price flexing’ strategy. The actions of these retailers are chosen by predicting the profit of each action, using a perceptron. Following on from the consideration of different economic models, a discrete model was developed so that software agents have a discrete environment to operate within. Within the model, it has been observed how supermarkets with differing behaviors affect a heterogeneous crowd of consumer agents. The model was implemented in Java with Python used to evaluate the results. 

The simulation displays good acceptance with real grocery market behavior, i.e. captures the performance of British retailers thus can be used to determine the impact of changes in their behavior on their competitors and consumers.Furthermore it can be used to provide insight into sustainability of volatile pricing strategies, providing a useful insight in volatility of British supermarket retail industry. 
\end{abstract}
\acknowledgements{
I would like to express my sincere gratitude to Dr Maria Polukarov for her guidance and support which provided me the freedom to take this research in the direction of my interest.\\
\\
I would also like to thank my family and friends for their encouragement and support. To those who quietly listened to my software complaints. To those who worked throughout the nights with me. To those who helped me write what I couldn't say. I cannot thank you enough.
}

\declaration{
I, Stefan Collier, declare that this dissertation and the work presented in it are my own and has been generated by me as the result of my own original research.\\
I confirm that:\\
1. This work was done wholly or mainly while in candidature for a degree at this University;\\
2. Where any part of this dissertation has previously been submitted for any other qualification at this University or any other institution, this has been clearly stated;\\
3. Where I have consulted the published work of others, this is always clearly attributed;\\
4. Where I have quoted from the work of others, the source is always given. With the exception of such quotations, this dissertation is entirely my own work;\\
5. I have acknowledged all main sources of help;\\
6. Where the thesis is based on work done by myself jointly with others, I have made clear exactly what was done by others and what I have contributed myself;\\
7. Either none of this work has been published before submission, or parts of this work have been published by :\\
\\
Stefan Collier\\
April 2016
}
\tableofcontents
\listoffigures
\listoftables

\mainmatter
%% ----------------------------------------------------------------
%\include{Introduction}
%\include{Conclusions}
\include{chapters/1Project/main}
\include{chapters/2Lit/main}
\include{chapters/3Design/HighLevel}
\include{chapters/3Design/InDepth}
\include{chapters/4Impl/main}

\include{chapters/5Experiments/1/main}
\include{chapters/5Experiments/2/main}
\include{chapters/5Experiments/3/main}
\include{chapters/5Experiments/4/main}

\include{chapters/6Conclusion/main}

\appendix
\include{appendix/AppendixB}
\include{appendix/D/main}
\include{appendix/AppendixC}

\backmatter
\bibliographystyle{ecs}
\bibliography{ECS}
\end{document}
%% ----------------------------------------------------------------


 %% ----------------------------------------------------------------
%% Progress.tex
%% ---------------------------------------------------------------- 
\documentclass{ecsprogress}    % Use the progress Style
\graphicspath{{../figs/}}   % Location of your graphics files
    \usepackage{natbib}            % Use Natbib style for the refs.
\hypersetup{colorlinks=true}   % Set to false for black/white printing
\input{Definitions}            % Include your abbreviations



\usepackage{enumitem}% http://ctan.org/pkg/enumitem
\usepackage{multirow}
\usepackage{float}
\usepackage{amsmath}
\usepackage{multicol}
\usepackage{amssymb}
\usepackage[normalem]{ulem}
\useunder{\uline}{\ul}{}
\usepackage{wrapfig}


\usepackage[table,xcdraw]{xcolor}


%% ----------------------------------------------------------------
\begin{document}
\frontmatter
\title      {Heterogeneous Agent-based Model for Supermarket Competition}
\authors    {\texorpdfstring
             {\href{mailto:sc22g13@ecs.soton.ac.uk}{Stefan J. Collier}}
             {Stefan J. Collier}
            }
\addresses  {\groupname\\\deptname\\\univname}
\date       {\today}
\subject    {}
\keywords   {}
\supervisor {Dr. Maria Polukarov}
\examiner   {Professor Sheng Chen}

\maketitle
\begin{abstract}
This project aim was to model and analyse the effects of competitive pricing behaviors of grocery retailers on the British market. 

This was achieved by creating a multi-agent model, containing retailer and consumer agents. The heterogeneous crowd of retailers employs either a uniform pricing strategy or a ‘local price flexing’ strategy. The actions of these retailers are chosen by predicting the profit of each action, using a perceptron. Following on from the consideration of different economic models, a discrete model was developed so that software agents have a discrete environment to operate within. Within the model, it has been observed how supermarkets with differing behaviors affect a heterogeneous crowd of consumer agents. The model was implemented in Java with Python used to evaluate the results. 

The simulation displays good acceptance with real grocery market behavior, i.e. captures the performance of British retailers thus can be used to determine the impact of changes in their behavior on their competitors and consumers.Furthermore it can be used to provide insight into sustainability of volatile pricing strategies, providing a useful insight in volatility of British supermarket retail industry. 
\end{abstract}
\acknowledgements{
I would like to express my sincere gratitude to Dr Maria Polukarov for her guidance and support which provided me the freedom to take this research in the direction of my interest.\\
\\
I would also like to thank my family and friends for their encouragement and support. To those who quietly listened to my software complaints. To those who worked throughout the nights with me. To those who helped me write what I couldn't say. I cannot thank you enough.
}

\declaration{
I, Stefan Collier, declare that this dissertation and the work presented in it are my own and has been generated by me as the result of my own original research.\\
I confirm that:\\
1. This work was done wholly or mainly while in candidature for a degree at this University;\\
2. Where any part of this dissertation has previously been submitted for any other qualification at this University or any other institution, this has been clearly stated;\\
3. Where I have consulted the published work of others, this is always clearly attributed;\\
4. Where I have quoted from the work of others, the source is always given. With the exception of such quotations, this dissertation is entirely my own work;\\
5. I have acknowledged all main sources of help;\\
6. Where the thesis is based on work done by myself jointly with others, I have made clear exactly what was done by others and what I have contributed myself;\\
7. Either none of this work has been published before submission, or parts of this work have been published by :\\
\\
Stefan Collier\\
April 2016
}
\tableofcontents
\listoffigures
\listoftables

\mainmatter
%% ----------------------------------------------------------------
%\include{Introduction}
%\include{Conclusions}
\include{chapters/1Project/main}
\include{chapters/2Lit/main}
\include{chapters/3Design/HighLevel}
\include{chapters/3Design/InDepth}
\include{chapters/4Impl/main}

\include{chapters/5Experiments/1/main}
\include{chapters/5Experiments/2/main}
\include{chapters/5Experiments/3/main}
\include{chapters/5Experiments/4/main}

\include{chapters/6Conclusion/main}

\appendix
\include{appendix/AppendixB}
\include{appendix/D/main}
\include{appendix/AppendixC}

\backmatter
\bibliographystyle{ecs}
\bibliography{ECS}
\end{document}
%% ----------------------------------------------------------------

 %% ----------------------------------------------------------------
%% Progress.tex
%% ---------------------------------------------------------------- 
\documentclass{ecsprogress}    % Use the progress Style
\graphicspath{{../figs/}}   % Location of your graphics files
    \usepackage{natbib}            % Use Natbib style for the refs.
\hypersetup{colorlinks=true}   % Set to false for black/white printing
\input{Definitions}            % Include your abbreviations



\usepackage{enumitem}% http://ctan.org/pkg/enumitem
\usepackage{multirow}
\usepackage{float}
\usepackage{amsmath}
\usepackage{multicol}
\usepackage{amssymb}
\usepackage[normalem]{ulem}
\useunder{\uline}{\ul}{}
\usepackage{wrapfig}


\usepackage[table,xcdraw]{xcolor}


%% ----------------------------------------------------------------
\begin{document}
\frontmatter
\title      {Heterogeneous Agent-based Model for Supermarket Competition}
\authors    {\texorpdfstring
             {\href{mailto:sc22g13@ecs.soton.ac.uk}{Stefan J. Collier}}
             {Stefan J. Collier}
            }
\addresses  {\groupname\\\deptname\\\univname}
\date       {\today}
\subject    {}
\keywords   {}
\supervisor {Dr. Maria Polukarov}
\examiner   {Professor Sheng Chen}

\maketitle
\begin{abstract}
This project aim was to model and analyse the effects of competitive pricing behaviors of grocery retailers on the British market. 

This was achieved by creating a multi-agent model, containing retailer and consumer agents. The heterogeneous crowd of retailers employs either a uniform pricing strategy or a ‘local price flexing’ strategy. The actions of these retailers are chosen by predicting the profit of each action, using a perceptron. Following on from the consideration of different economic models, a discrete model was developed so that software agents have a discrete environment to operate within. Within the model, it has been observed how supermarkets with differing behaviors affect a heterogeneous crowd of consumer agents. The model was implemented in Java with Python used to evaluate the results. 

The simulation displays good acceptance with real grocery market behavior, i.e. captures the performance of British retailers thus can be used to determine the impact of changes in their behavior on their competitors and consumers.Furthermore it can be used to provide insight into sustainability of volatile pricing strategies, providing a useful insight in volatility of British supermarket retail industry. 
\end{abstract}
\acknowledgements{
I would like to express my sincere gratitude to Dr Maria Polukarov for her guidance and support which provided me the freedom to take this research in the direction of my interest.\\
\\
I would also like to thank my family and friends for their encouragement and support. To those who quietly listened to my software complaints. To those who worked throughout the nights with me. To those who helped me write what I couldn't say. I cannot thank you enough.
}

\declaration{
I, Stefan Collier, declare that this dissertation and the work presented in it are my own and has been generated by me as the result of my own original research.\\
I confirm that:\\
1. This work was done wholly or mainly while in candidature for a degree at this University;\\
2. Where any part of this dissertation has previously been submitted for any other qualification at this University or any other institution, this has been clearly stated;\\
3. Where I have consulted the published work of others, this is always clearly attributed;\\
4. Where I have quoted from the work of others, the source is always given. With the exception of such quotations, this dissertation is entirely my own work;\\
5. I have acknowledged all main sources of help;\\
6. Where the thesis is based on work done by myself jointly with others, I have made clear exactly what was done by others and what I have contributed myself;\\
7. Either none of this work has been published before submission, or parts of this work have been published by :\\
\\
Stefan Collier\\
April 2016
}
\tableofcontents
\listoffigures
\listoftables

\mainmatter
%% ----------------------------------------------------------------
%\include{Introduction}
%\include{Conclusions}
\include{chapters/1Project/main}
\include{chapters/2Lit/main}
\include{chapters/3Design/HighLevel}
\include{chapters/3Design/InDepth}
\include{chapters/4Impl/main}

\include{chapters/5Experiments/1/main}
\include{chapters/5Experiments/2/main}
\include{chapters/5Experiments/3/main}
\include{chapters/5Experiments/4/main}

\include{chapters/6Conclusion/main}

\appendix
\include{appendix/AppendixB}
\include{appendix/D/main}
\include{appendix/AppendixC}

\backmatter
\bibliographystyle{ecs}
\bibliography{ECS}
\end{document}
%% ----------------------------------------------------------------

 %% ----------------------------------------------------------------
%% Progress.tex
%% ---------------------------------------------------------------- 
\documentclass{ecsprogress}    % Use the progress Style
\graphicspath{{../figs/}}   % Location of your graphics files
    \usepackage{natbib}            % Use Natbib style for the refs.
\hypersetup{colorlinks=true}   % Set to false for black/white printing
\input{Definitions}            % Include your abbreviations



\usepackage{enumitem}% http://ctan.org/pkg/enumitem
\usepackage{multirow}
\usepackage{float}
\usepackage{amsmath}
\usepackage{multicol}
\usepackage{amssymb}
\usepackage[normalem]{ulem}
\useunder{\uline}{\ul}{}
\usepackage{wrapfig}


\usepackage[table,xcdraw]{xcolor}


%% ----------------------------------------------------------------
\begin{document}
\frontmatter
\title      {Heterogeneous Agent-based Model for Supermarket Competition}
\authors    {\texorpdfstring
             {\href{mailto:sc22g13@ecs.soton.ac.uk}{Stefan J. Collier}}
             {Stefan J. Collier}
            }
\addresses  {\groupname\\\deptname\\\univname}
\date       {\today}
\subject    {}
\keywords   {}
\supervisor {Dr. Maria Polukarov}
\examiner   {Professor Sheng Chen}

\maketitle
\begin{abstract}
This project aim was to model and analyse the effects of competitive pricing behaviors of grocery retailers on the British market. 

This was achieved by creating a multi-agent model, containing retailer and consumer agents. The heterogeneous crowd of retailers employs either a uniform pricing strategy or a ‘local price flexing’ strategy. The actions of these retailers are chosen by predicting the profit of each action, using a perceptron. Following on from the consideration of different economic models, a discrete model was developed so that software agents have a discrete environment to operate within. Within the model, it has been observed how supermarkets with differing behaviors affect a heterogeneous crowd of consumer agents. The model was implemented in Java with Python used to evaluate the results. 

The simulation displays good acceptance with real grocery market behavior, i.e. captures the performance of British retailers thus can be used to determine the impact of changes in their behavior on their competitors and consumers.Furthermore it can be used to provide insight into sustainability of volatile pricing strategies, providing a useful insight in volatility of British supermarket retail industry. 
\end{abstract}
\acknowledgements{
I would like to express my sincere gratitude to Dr Maria Polukarov for her guidance and support which provided me the freedom to take this research in the direction of my interest.\\
\\
I would also like to thank my family and friends for their encouragement and support. To those who quietly listened to my software complaints. To those who worked throughout the nights with me. To those who helped me write what I couldn't say. I cannot thank you enough.
}

\declaration{
I, Stefan Collier, declare that this dissertation and the work presented in it are my own and has been generated by me as the result of my own original research.\\
I confirm that:\\
1. This work was done wholly or mainly while in candidature for a degree at this University;\\
2. Where any part of this dissertation has previously been submitted for any other qualification at this University or any other institution, this has been clearly stated;\\
3. Where I have consulted the published work of others, this is always clearly attributed;\\
4. Where I have quoted from the work of others, the source is always given. With the exception of such quotations, this dissertation is entirely my own work;\\
5. I have acknowledged all main sources of help;\\
6. Where the thesis is based on work done by myself jointly with others, I have made clear exactly what was done by others and what I have contributed myself;\\
7. Either none of this work has been published before submission, or parts of this work have been published by :\\
\\
Stefan Collier\\
April 2016
}
\tableofcontents
\listoffigures
\listoftables

\mainmatter
%% ----------------------------------------------------------------
%\include{Introduction}
%\include{Conclusions}
\include{chapters/1Project/main}
\include{chapters/2Lit/main}
\include{chapters/3Design/HighLevel}
\include{chapters/3Design/InDepth}
\include{chapters/4Impl/main}

\include{chapters/5Experiments/1/main}
\include{chapters/5Experiments/2/main}
\include{chapters/5Experiments/3/main}
\include{chapters/5Experiments/4/main}

\include{chapters/6Conclusion/main}

\appendix
\include{appendix/AppendixB}
\include{appendix/D/main}
\include{appendix/AppendixC}

\backmatter
\bibliographystyle{ecs}
\bibliography{ECS}
\end{document}
%% ----------------------------------------------------------------

 %% ----------------------------------------------------------------
%% Progress.tex
%% ---------------------------------------------------------------- 
\documentclass{ecsprogress}    % Use the progress Style
\graphicspath{{../figs/}}   % Location of your graphics files
    \usepackage{natbib}            % Use Natbib style for the refs.
\hypersetup{colorlinks=true}   % Set to false for black/white printing
\input{Definitions}            % Include your abbreviations



\usepackage{enumitem}% http://ctan.org/pkg/enumitem
\usepackage{multirow}
\usepackage{float}
\usepackage{amsmath}
\usepackage{multicol}
\usepackage{amssymb}
\usepackage[normalem]{ulem}
\useunder{\uline}{\ul}{}
\usepackage{wrapfig}


\usepackage[table,xcdraw]{xcolor}


%% ----------------------------------------------------------------
\begin{document}
\frontmatter
\title      {Heterogeneous Agent-based Model for Supermarket Competition}
\authors    {\texorpdfstring
             {\href{mailto:sc22g13@ecs.soton.ac.uk}{Stefan J. Collier}}
             {Stefan J. Collier}
            }
\addresses  {\groupname\\\deptname\\\univname}
\date       {\today}
\subject    {}
\keywords   {}
\supervisor {Dr. Maria Polukarov}
\examiner   {Professor Sheng Chen}

\maketitle
\begin{abstract}
This project aim was to model and analyse the effects of competitive pricing behaviors of grocery retailers on the British market. 

This was achieved by creating a multi-agent model, containing retailer and consumer agents. The heterogeneous crowd of retailers employs either a uniform pricing strategy or a ‘local price flexing’ strategy. The actions of these retailers are chosen by predicting the profit of each action, using a perceptron. Following on from the consideration of different economic models, a discrete model was developed so that software agents have a discrete environment to operate within. Within the model, it has been observed how supermarkets with differing behaviors affect a heterogeneous crowd of consumer agents. The model was implemented in Java with Python used to evaluate the results. 

The simulation displays good acceptance with real grocery market behavior, i.e. captures the performance of British retailers thus can be used to determine the impact of changes in their behavior on their competitors and consumers.Furthermore it can be used to provide insight into sustainability of volatile pricing strategies, providing a useful insight in volatility of British supermarket retail industry. 
\end{abstract}
\acknowledgements{
I would like to express my sincere gratitude to Dr Maria Polukarov for her guidance and support which provided me the freedom to take this research in the direction of my interest.\\
\\
I would also like to thank my family and friends for their encouragement and support. To those who quietly listened to my software complaints. To those who worked throughout the nights with me. To those who helped me write what I couldn't say. I cannot thank you enough.
}

\declaration{
I, Stefan Collier, declare that this dissertation and the work presented in it are my own and has been generated by me as the result of my own original research.\\
I confirm that:\\
1. This work was done wholly or mainly while in candidature for a degree at this University;\\
2. Where any part of this dissertation has previously been submitted for any other qualification at this University or any other institution, this has been clearly stated;\\
3. Where I have consulted the published work of others, this is always clearly attributed;\\
4. Where I have quoted from the work of others, the source is always given. With the exception of such quotations, this dissertation is entirely my own work;\\
5. I have acknowledged all main sources of help;\\
6. Where the thesis is based on work done by myself jointly with others, I have made clear exactly what was done by others and what I have contributed myself;\\
7. Either none of this work has been published before submission, or parts of this work have been published by :\\
\\
Stefan Collier\\
April 2016
}
\tableofcontents
\listoffigures
\listoftables

\mainmatter
%% ----------------------------------------------------------------
%\include{Introduction}
%\include{Conclusions}
\include{chapters/1Project/main}
\include{chapters/2Lit/main}
\include{chapters/3Design/HighLevel}
\include{chapters/3Design/InDepth}
\include{chapters/4Impl/main}

\include{chapters/5Experiments/1/main}
\include{chapters/5Experiments/2/main}
\include{chapters/5Experiments/3/main}
\include{chapters/5Experiments/4/main}

\include{chapters/6Conclusion/main}

\appendix
\include{appendix/AppendixB}
\include{appendix/D/main}
\include{appendix/AppendixC}

\backmatter
\bibliographystyle{ecs}
\bibliography{ECS}
\end{document}
%% ----------------------------------------------------------------


 %% ----------------------------------------------------------------
%% Progress.tex
%% ---------------------------------------------------------------- 
\documentclass{ecsprogress}    % Use the progress Style
\graphicspath{{../figs/}}   % Location of your graphics files
    \usepackage{natbib}            % Use Natbib style for the refs.
\hypersetup{colorlinks=true}   % Set to false for black/white printing
\input{Definitions}            % Include your abbreviations



\usepackage{enumitem}% http://ctan.org/pkg/enumitem
\usepackage{multirow}
\usepackage{float}
\usepackage{amsmath}
\usepackage{multicol}
\usepackage{amssymb}
\usepackage[normalem]{ulem}
\useunder{\uline}{\ul}{}
\usepackage{wrapfig}


\usepackage[table,xcdraw]{xcolor}


%% ----------------------------------------------------------------
\begin{document}
\frontmatter
\title      {Heterogeneous Agent-based Model for Supermarket Competition}
\authors    {\texorpdfstring
             {\href{mailto:sc22g13@ecs.soton.ac.uk}{Stefan J. Collier}}
             {Stefan J. Collier}
            }
\addresses  {\groupname\\\deptname\\\univname}
\date       {\today}
\subject    {}
\keywords   {}
\supervisor {Dr. Maria Polukarov}
\examiner   {Professor Sheng Chen}

\maketitle
\begin{abstract}
This project aim was to model and analyse the effects of competitive pricing behaviors of grocery retailers on the British market. 

This was achieved by creating a multi-agent model, containing retailer and consumer agents. The heterogeneous crowd of retailers employs either a uniform pricing strategy or a ‘local price flexing’ strategy. The actions of these retailers are chosen by predicting the profit of each action, using a perceptron. Following on from the consideration of different economic models, a discrete model was developed so that software agents have a discrete environment to operate within. Within the model, it has been observed how supermarkets with differing behaviors affect a heterogeneous crowd of consumer agents. The model was implemented in Java with Python used to evaluate the results. 

The simulation displays good acceptance with real grocery market behavior, i.e. captures the performance of British retailers thus can be used to determine the impact of changes in their behavior on their competitors and consumers.Furthermore it can be used to provide insight into sustainability of volatile pricing strategies, providing a useful insight in volatility of British supermarket retail industry. 
\end{abstract}
\acknowledgements{
I would like to express my sincere gratitude to Dr Maria Polukarov for her guidance and support which provided me the freedom to take this research in the direction of my interest.\\
\\
I would also like to thank my family and friends for their encouragement and support. To those who quietly listened to my software complaints. To those who worked throughout the nights with me. To those who helped me write what I couldn't say. I cannot thank you enough.
}

\declaration{
I, Stefan Collier, declare that this dissertation and the work presented in it are my own and has been generated by me as the result of my own original research.\\
I confirm that:\\
1. This work was done wholly or mainly while in candidature for a degree at this University;\\
2. Where any part of this dissertation has previously been submitted for any other qualification at this University or any other institution, this has been clearly stated;\\
3. Where I have consulted the published work of others, this is always clearly attributed;\\
4. Where I have quoted from the work of others, the source is always given. With the exception of such quotations, this dissertation is entirely my own work;\\
5. I have acknowledged all main sources of help;\\
6. Where the thesis is based on work done by myself jointly with others, I have made clear exactly what was done by others and what I have contributed myself;\\
7. Either none of this work has been published before submission, or parts of this work have been published by :\\
\\
Stefan Collier\\
April 2016
}
\tableofcontents
\listoffigures
\listoftables

\mainmatter
%% ----------------------------------------------------------------
%\include{Introduction}
%\include{Conclusions}
\include{chapters/1Project/main}
\include{chapters/2Lit/main}
\include{chapters/3Design/HighLevel}
\include{chapters/3Design/InDepth}
\include{chapters/4Impl/main}

\include{chapters/5Experiments/1/main}
\include{chapters/5Experiments/2/main}
\include{chapters/5Experiments/3/main}
\include{chapters/5Experiments/4/main}

\include{chapters/6Conclusion/main}

\appendix
\include{appendix/AppendixB}
\include{appendix/D/main}
\include{appendix/AppendixC}

\backmatter
\bibliographystyle{ecs}
\bibliography{ECS}
\end{document}
%% ----------------------------------------------------------------


\appendix
\include{appendix/AppendixB}
 %% ----------------------------------------------------------------
%% Progress.tex
%% ---------------------------------------------------------------- 
\documentclass{ecsprogress}    % Use the progress Style
\graphicspath{{../figs/}}   % Location of your graphics files
    \usepackage{natbib}            % Use Natbib style for the refs.
\hypersetup{colorlinks=true}   % Set to false for black/white printing
\input{Definitions}            % Include your abbreviations



\usepackage{enumitem}% http://ctan.org/pkg/enumitem
\usepackage{multirow}
\usepackage{float}
\usepackage{amsmath}
\usepackage{multicol}
\usepackage{amssymb}
\usepackage[normalem]{ulem}
\useunder{\uline}{\ul}{}
\usepackage{wrapfig}


\usepackage[table,xcdraw]{xcolor}


%% ----------------------------------------------------------------
\begin{document}
\frontmatter
\title      {Heterogeneous Agent-based Model for Supermarket Competition}
\authors    {\texorpdfstring
             {\href{mailto:sc22g13@ecs.soton.ac.uk}{Stefan J. Collier}}
             {Stefan J. Collier}
            }
\addresses  {\groupname\\\deptname\\\univname}
\date       {\today}
\subject    {}
\keywords   {}
\supervisor {Dr. Maria Polukarov}
\examiner   {Professor Sheng Chen}

\maketitle
\begin{abstract}
This project aim was to model and analyse the effects of competitive pricing behaviors of grocery retailers on the British market. 

This was achieved by creating a multi-agent model, containing retailer and consumer agents. The heterogeneous crowd of retailers employs either a uniform pricing strategy or a ‘local price flexing’ strategy. The actions of these retailers are chosen by predicting the profit of each action, using a perceptron. Following on from the consideration of different economic models, a discrete model was developed so that software agents have a discrete environment to operate within. Within the model, it has been observed how supermarkets with differing behaviors affect a heterogeneous crowd of consumer agents. The model was implemented in Java with Python used to evaluate the results. 

The simulation displays good acceptance with real grocery market behavior, i.e. captures the performance of British retailers thus can be used to determine the impact of changes in their behavior on their competitors and consumers.Furthermore it can be used to provide insight into sustainability of volatile pricing strategies, providing a useful insight in volatility of British supermarket retail industry. 
\end{abstract}
\acknowledgements{
I would like to express my sincere gratitude to Dr Maria Polukarov for her guidance and support which provided me the freedom to take this research in the direction of my interest.\\
\\
I would also like to thank my family and friends for their encouragement and support. To those who quietly listened to my software complaints. To those who worked throughout the nights with me. To those who helped me write what I couldn't say. I cannot thank you enough.
}

\declaration{
I, Stefan Collier, declare that this dissertation and the work presented in it are my own and has been generated by me as the result of my own original research.\\
I confirm that:\\
1. This work was done wholly or mainly while in candidature for a degree at this University;\\
2. Where any part of this dissertation has previously been submitted for any other qualification at this University or any other institution, this has been clearly stated;\\
3. Where I have consulted the published work of others, this is always clearly attributed;\\
4. Where I have quoted from the work of others, the source is always given. With the exception of such quotations, this dissertation is entirely my own work;\\
5. I have acknowledged all main sources of help;\\
6. Where the thesis is based on work done by myself jointly with others, I have made clear exactly what was done by others and what I have contributed myself;\\
7. Either none of this work has been published before submission, or parts of this work have been published by :\\
\\
Stefan Collier\\
April 2016
}
\tableofcontents
\listoffigures
\listoftables

\mainmatter
%% ----------------------------------------------------------------
%\include{Introduction}
%\include{Conclusions}
\include{chapters/1Project/main}
\include{chapters/2Lit/main}
\include{chapters/3Design/HighLevel}
\include{chapters/3Design/InDepth}
\include{chapters/4Impl/main}

\include{chapters/5Experiments/1/main}
\include{chapters/5Experiments/2/main}
\include{chapters/5Experiments/3/main}
\include{chapters/5Experiments/4/main}

\include{chapters/6Conclusion/main}

\appendix
\include{appendix/AppendixB}
\include{appendix/D/main}
\include{appendix/AppendixC}

\backmatter
\bibliographystyle{ecs}
\bibliography{ECS}
\end{document}
%% ----------------------------------------------------------------

\include{appendix/AppendixC}

\backmatter
\bibliographystyle{ecs}
\bibliography{ECS}
\end{document}
%% ----------------------------------------------------------------

 %% ----------------------------------------------------------------
%% Progress.tex
%% ---------------------------------------------------------------- 
\documentclass{ecsprogress}    % Use the progress Style
\graphicspath{{../figs/}}   % Location of your graphics files
    \usepackage{natbib}            % Use Natbib style for the refs.
\hypersetup{colorlinks=true}   % Set to false for black/white printing
\input{Definitions}            % Include your abbreviations



\usepackage{enumitem}% http://ctan.org/pkg/enumitem
\usepackage{multirow}
\usepackage{float}
\usepackage{amsmath}
\usepackage{multicol}
\usepackage{amssymb}
\usepackage[normalem]{ulem}
\useunder{\uline}{\ul}{}
\usepackage{wrapfig}


\usepackage[table,xcdraw]{xcolor}


%% ----------------------------------------------------------------
\begin{document}
\frontmatter
\title      {Heterogeneous Agent-based Model for Supermarket Competition}
\authors    {\texorpdfstring
             {\href{mailto:sc22g13@ecs.soton.ac.uk}{Stefan J. Collier}}
             {Stefan J. Collier}
            }
\addresses  {\groupname\\\deptname\\\univname}
\date       {\today}
\subject    {}
\keywords   {}
\supervisor {Dr. Maria Polukarov}
\examiner   {Professor Sheng Chen}

\maketitle
\begin{abstract}
This project aim was to model and analyse the effects of competitive pricing behaviors of grocery retailers on the British market. 

This was achieved by creating a multi-agent model, containing retailer and consumer agents. The heterogeneous crowd of retailers employs either a uniform pricing strategy or a ‘local price flexing’ strategy. The actions of these retailers are chosen by predicting the profit of each action, using a perceptron. Following on from the consideration of different economic models, a discrete model was developed so that software agents have a discrete environment to operate within. Within the model, it has been observed how supermarkets with differing behaviors affect a heterogeneous crowd of consumer agents. The model was implemented in Java with Python used to evaluate the results. 

The simulation displays good acceptance with real grocery market behavior, i.e. captures the performance of British retailers thus can be used to determine the impact of changes in their behavior on their competitors and consumers.Furthermore it can be used to provide insight into sustainability of volatile pricing strategies, providing a useful insight in volatility of British supermarket retail industry. 
\end{abstract}
\acknowledgements{
I would like to express my sincere gratitude to Dr Maria Polukarov for her guidance and support which provided me the freedom to take this research in the direction of my interest.\\
\\
I would also like to thank my family and friends for their encouragement and support. To those who quietly listened to my software complaints. To those who worked throughout the nights with me. To those who helped me write what I couldn't say. I cannot thank you enough.
}

\declaration{
I, Stefan Collier, declare that this dissertation and the work presented in it are my own and has been generated by me as the result of my own original research.\\
I confirm that:\\
1. This work was done wholly or mainly while in candidature for a degree at this University;\\
2. Where any part of this dissertation has previously been submitted for any other qualification at this University or any other institution, this has been clearly stated;\\
3. Where I have consulted the published work of others, this is always clearly attributed;\\
4. Where I have quoted from the work of others, the source is always given. With the exception of such quotations, this dissertation is entirely my own work;\\
5. I have acknowledged all main sources of help;\\
6. Where the thesis is based on work done by myself jointly with others, I have made clear exactly what was done by others and what I have contributed myself;\\
7. Either none of this work has been published before submission, or parts of this work have been published by :\\
\\
Stefan Collier\\
April 2016
}
\tableofcontents
\listoffigures
\listoftables

\mainmatter
%% ----------------------------------------------------------------
%\include{Introduction}
%\include{Conclusions}
 %% ----------------------------------------------------------------
%% Progress.tex
%% ---------------------------------------------------------------- 
\documentclass{ecsprogress}    % Use the progress Style
\graphicspath{{../figs/}}   % Location of your graphics files
    \usepackage{natbib}            % Use Natbib style for the refs.
\hypersetup{colorlinks=true}   % Set to false for black/white printing
\input{Definitions}            % Include your abbreviations



\usepackage{enumitem}% http://ctan.org/pkg/enumitem
\usepackage{multirow}
\usepackage{float}
\usepackage{amsmath}
\usepackage{multicol}
\usepackage{amssymb}
\usepackage[normalem]{ulem}
\useunder{\uline}{\ul}{}
\usepackage{wrapfig}


\usepackage[table,xcdraw]{xcolor}


%% ----------------------------------------------------------------
\begin{document}
\frontmatter
\title      {Heterogeneous Agent-based Model for Supermarket Competition}
\authors    {\texorpdfstring
             {\href{mailto:sc22g13@ecs.soton.ac.uk}{Stefan J. Collier}}
             {Stefan J. Collier}
            }
\addresses  {\groupname\\\deptname\\\univname}
\date       {\today}
\subject    {}
\keywords   {}
\supervisor {Dr. Maria Polukarov}
\examiner   {Professor Sheng Chen}

\maketitle
\begin{abstract}
This project aim was to model and analyse the effects of competitive pricing behaviors of grocery retailers on the British market. 

This was achieved by creating a multi-agent model, containing retailer and consumer agents. The heterogeneous crowd of retailers employs either a uniform pricing strategy or a ‘local price flexing’ strategy. The actions of these retailers are chosen by predicting the profit of each action, using a perceptron. Following on from the consideration of different economic models, a discrete model was developed so that software agents have a discrete environment to operate within. Within the model, it has been observed how supermarkets with differing behaviors affect a heterogeneous crowd of consumer agents. The model was implemented in Java with Python used to evaluate the results. 

The simulation displays good acceptance with real grocery market behavior, i.e. captures the performance of British retailers thus can be used to determine the impact of changes in their behavior on their competitors and consumers.Furthermore it can be used to provide insight into sustainability of volatile pricing strategies, providing a useful insight in volatility of British supermarket retail industry. 
\end{abstract}
\acknowledgements{
I would like to express my sincere gratitude to Dr Maria Polukarov for her guidance and support which provided me the freedom to take this research in the direction of my interest.\\
\\
I would also like to thank my family and friends for their encouragement and support. To those who quietly listened to my software complaints. To those who worked throughout the nights with me. To those who helped me write what I couldn't say. I cannot thank you enough.
}

\declaration{
I, Stefan Collier, declare that this dissertation and the work presented in it are my own and has been generated by me as the result of my own original research.\\
I confirm that:\\
1. This work was done wholly or mainly while in candidature for a degree at this University;\\
2. Where any part of this dissertation has previously been submitted for any other qualification at this University or any other institution, this has been clearly stated;\\
3. Where I have consulted the published work of others, this is always clearly attributed;\\
4. Where I have quoted from the work of others, the source is always given. With the exception of such quotations, this dissertation is entirely my own work;\\
5. I have acknowledged all main sources of help;\\
6. Where the thesis is based on work done by myself jointly with others, I have made clear exactly what was done by others and what I have contributed myself;\\
7. Either none of this work has been published before submission, or parts of this work have been published by :\\
\\
Stefan Collier\\
April 2016
}
\tableofcontents
\listoffigures
\listoftables

\mainmatter
%% ----------------------------------------------------------------
%\include{Introduction}
%\include{Conclusions}
\include{chapters/1Project/main}
\include{chapters/2Lit/main}
\include{chapters/3Design/HighLevel}
\include{chapters/3Design/InDepth}
\include{chapters/4Impl/main}

\include{chapters/5Experiments/1/main}
\include{chapters/5Experiments/2/main}
\include{chapters/5Experiments/3/main}
\include{chapters/5Experiments/4/main}

\include{chapters/6Conclusion/main}

\appendix
\include{appendix/AppendixB}
\include{appendix/D/main}
\include{appendix/AppendixC}

\backmatter
\bibliographystyle{ecs}
\bibliography{ECS}
\end{document}
%% ----------------------------------------------------------------

 %% ----------------------------------------------------------------
%% Progress.tex
%% ---------------------------------------------------------------- 
\documentclass{ecsprogress}    % Use the progress Style
\graphicspath{{../figs/}}   % Location of your graphics files
    \usepackage{natbib}            % Use Natbib style for the refs.
\hypersetup{colorlinks=true}   % Set to false for black/white printing
\input{Definitions}            % Include your abbreviations



\usepackage{enumitem}% http://ctan.org/pkg/enumitem
\usepackage{multirow}
\usepackage{float}
\usepackage{amsmath}
\usepackage{multicol}
\usepackage{amssymb}
\usepackage[normalem]{ulem}
\useunder{\uline}{\ul}{}
\usepackage{wrapfig}


\usepackage[table,xcdraw]{xcolor}


%% ----------------------------------------------------------------
\begin{document}
\frontmatter
\title      {Heterogeneous Agent-based Model for Supermarket Competition}
\authors    {\texorpdfstring
             {\href{mailto:sc22g13@ecs.soton.ac.uk}{Stefan J. Collier}}
             {Stefan J. Collier}
            }
\addresses  {\groupname\\\deptname\\\univname}
\date       {\today}
\subject    {}
\keywords   {}
\supervisor {Dr. Maria Polukarov}
\examiner   {Professor Sheng Chen}

\maketitle
\begin{abstract}
This project aim was to model and analyse the effects of competitive pricing behaviors of grocery retailers on the British market. 

This was achieved by creating a multi-agent model, containing retailer and consumer agents. The heterogeneous crowd of retailers employs either a uniform pricing strategy or a ‘local price flexing’ strategy. The actions of these retailers are chosen by predicting the profit of each action, using a perceptron. Following on from the consideration of different economic models, a discrete model was developed so that software agents have a discrete environment to operate within. Within the model, it has been observed how supermarkets with differing behaviors affect a heterogeneous crowd of consumer agents. The model was implemented in Java with Python used to evaluate the results. 

The simulation displays good acceptance with real grocery market behavior, i.e. captures the performance of British retailers thus can be used to determine the impact of changes in their behavior on their competitors and consumers.Furthermore it can be used to provide insight into sustainability of volatile pricing strategies, providing a useful insight in volatility of British supermarket retail industry. 
\end{abstract}
\acknowledgements{
I would like to express my sincere gratitude to Dr Maria Polukarov for her guidance and support which provided me the freedom to take this research in the direction of my interest.\\
\\
I would also like to thank my family and friends for their encouragement and support. To those who quietly listened to my software complaints. To those who worked throughout the nights with me. To those who helped me write what I couldn't say. I cannot thank you enough.
}

\declaration{
I, Stefan Collier, declare that this dissertation and the work presented in it are my own and has been generated by me as the result of my own original research.\\
I confirm that:\\
1. This work was done wholly or mainly while in candidature for a degree at this University;\\
2. Where any part of this dissertation has previously been submitted for any other qualification at this University or any other institution, this has been clearly stated;\\
3. Where I have consulted the published work of others, this is always clearly attributed;\\
4. Where I have quoted from the work of others, the source is always given. With the exception of such quotations, this dissertation is entirely my own work;\\
5. I have acknowledged all main sources of help;\\
6. Where the thesis is based on work done by myself jointly with others, I have made clear exactly what was done by others and what I have contributed myself;\\
7. Either none of this work has been published before submission, or parts of this work have been published by :\\
\\
Stefan Collier\\
April 2016
}
\tableofcontents
\listoffigures
\listoftables

\mainmatter
%% ----------------------------------------------------------------
%\include{Introduction}
%\include{Conclusions}
\include{chapters/1Project/main}
\include{chapters/2Lit/main}
\include{chapters/3Design/HighLevel}
\include{chapters/3Design/InDepth}
\include{chapters/4Impl/main}

\include{chapters/5Experiments/1/main}
\include{chapters/5Experiments/2/main}
\include{chapters/5Experiments/3/main}
\include{chapters/5Experiments/4/main}

\include{chapters/6Conclusion/main}

\appendix
\include{appendix/AppendixB}
\include{appendix/D/main}
\include{appendix/AppendixC}

\backmatter
\bibliographystyle{ecs}
\bibliography{ECS}
\end{document}
%% ----------------------------------------------------------------

\include{chapters/3Design/HighLevel}
\include{chapters/3Design/InDepth}
 %% ----------------------------------------------------------------
%% Progress.tex
%% ---------------------------------------------------------------- 
\documentclass{ecsprogress}    % Use the progress Style
\graphicspath{{../figs/}}   % Location of your graphics files
    \usepackage{natbib}            % Use Natbib style for the refs.
\hypersetup{colorlinks=true}   % Set to false for black/white printing
\input{Definitions}            % Include your abbreviations



\usepackage{enumitem}% http://ctan.org/pkg/enumitem
\usepackage{multirow}
\usepackage{float}
\usepackage{amsmath}
\usepackage{multicol}
\usepackage{amssymb}
\usepackage[normalem]{ulem}
\useunder{\uline}{\ul}{}
\usepackage{wrapfig}


\usepackage[table,xcdraw]{xcolor}


%% ----------------------------------------------------------------
\begin{document}
\frontmatter
\title      {Heterogeneous Agent-based Model for Supermarket Competition}
\authors    {\texorpdfstring
             {\href{mailto:sc22g13@ecs.soton.ac.uk}{Stefan J. Collier}}
             {Stefan J. Collier}
            }
\addresses  {\groupname\\\deptname\\\univname}
\date       {\today}
\subject    {}
\keywords   {}
\supervisor {Dr. Maria Polukarov}
\examiner   {Professor Sheng Chen}

\maketitle
\begin{abstract}
This project aim was to model and analyse the effects of competitive pricing behaviors of grocery retailers on the British market. 

This was achieved by creating a multi-agent model, containing retailer and consumer agents. The heterogeneous crowd of retailers employs either a uniform pricing strategy or a ‘local price flexing’ strategy. The actions of these retailers are chosen by predicting the profit of each action, using a perceptron. Following on from the consideration of different economic models, a discrete model was developed so that software agents have a discrete environment to operate within. Within the model, it has been observed how supermarkets with differing behaviors affect a heterogeneous crowd of consumer agents. The model was implemented in Java with Python used to evaluate the results. 

The simulation displays good acceptance with real grocery market behavior, i.e. captures the performance of British retailers thus can be used to determine the impact of changes in their behavior on their competitors and consumers.Furthermore it can be used to provide insight into sustainability of volatile pricing strategies, providing a useful insight in volatility of British supermarket retail industry. 
\end{abstract}
\acknowledgements{
I would like to express my sincere gratitude to Dr Maria Polukarov for her guidance and support which provided me the freedom to take this research in the direction of my interest.\\
\\
I would also like to thank my family and friends for their encouragement and support. To those who quietly listened to my software complaints. To those who worked throughout the nights with me. To those who helped me write what I couldn't say. I cannot thank you enough.
}

\declaration{
I, Stefan Collier, declare that this dissertation and the work presented in it are my own and has been generated by me as the result of my own original research.\\
I confirm that:\\
1. This work was done wholly or mainly while in candidature for a degree at this University;\\
2. Where any part of this dissertation has previously been submitted for any other qualification at this University or any other institution, this has been clearly stated;\\
3. Where I have consulted the published work of others, this is always clearly attributed;\\
4. Where I have quoted from the work of others, the source is always given. With the exception of such quotations, this dissertation is entirely my own work;\\
5. I have acknowledged all main sources of help;\\
6. Where the thesis is based on work done by myself jointly with others, I have made clear exactly what was done by others and what I have contributed myself;\\
7. Either none of this work has been published before submission, or parts of this work have been published by :\\
\\
Stefan Collier\\
April 2016
}
\tableofcontents
\listoffigures
\listoftables

\mainmatter
%% ----------------------------------------------------------------
%\include{Introduction}
%\include{Conclusions}
\include{chapters/1Project/main}
\include{chapters/2Lit/main}
\include{chapters/3Design/HighLevel}
\include{chapters/3Design/InDepth}
\include{chapters/4Impl/main}

\include{chapters/5Experiments/1/main}
\include{chapters/5Experiments/2/main}
\include{chapters/5Experiments/3/main}
\include{chapters/5Experiments/4/main}

\include{chapters/6Conclusion/main}

\appendix
\include{appendix/AppendixB}
\include{appendix/D/main}
\include{appendix/AppendixC}

\backmatter
\bibliographystyle{ecs}
\bibliography{ECS}
\end{document}
%% ----------------------------------------------------------------


 %% ----------------------------------------------------------------
%% Progress.tex
%% ---------------------------------------------------------------- 
\documentclass{ecsprogress}    % Use the progress Style
\graphicspath{{../figs/}}   % Location of your graphics files
    \usepackage{natbib}            % Use Natbib style for the refs.
\hypersetup{colorlinks=true}   % Set to false for black/white printing
\input{Definitions}            % Include your abbreviations



\usepackage{enumitem}% http://ctan.org/pkg/enumitem
\usepackage{multirow}
\usepackage{float}
\usepackage{amsmath}
\usepackage{multicol}
\usepackage{amssymb}
\usepackage[normalem]{ulem}
\useunder{\uline}{\ul}{}
\usepackage{wrapfig}


\usepackage[table,xcdraw]{xcolor}


%% ----------------------------------------------------------------
\begin{document}
\frontmatter
\title      {Heterogeneous Agent-based Model for Supermarket Competition}
\authors    {\texorpdfstring
             {\href{mailto:sc22g13@ecs.soton.ac.uk}{Stefan J. Collier}}
             {Stefan J. Collier}
            }
\addresses  {\groupname\\\deptname\\\univname}
\date       {\today}
\subject    {}
\keywords   {}
\supervisor {Dr. Maria Polukarov}
\examiner   {Professor Sheng Chen}

\maketitle
\begin{abstract}
This project aim was to model and analyse the effects of competitive pricing behaviors of grocery retailers on the British market. 

This was achieved by creating a multi-agent model, containing retailer and consumer agents. The heterogeneous crowd of retailers employs either a uniform pricing strategy or a ‘local price flexing’ strategy. The actions of these retailers are chosen by predicting the profit of each action, using a perceptron. Following on from the consideration of different economic models, a discrete model was developed so that software agents have a discrete environment to operate within. Within the model, it has been observed how supermarkets with differing behaviors affect a heterogeneous crowd of consumer agents. The model was implemented in Java with Python used to evaluate the results. 

The simulation displays good acceptance with real grocery market behavior, i.e. captures the performance of British retailers thus can be used to determine the impact of changes in their behavior on their competitors and consumers.Furthermore it can be used to provide insight into sustainability of volatile pricing strategies, providing a useful insight in volatility of British supermarket retail industry. 
\end{abstract}
\acknowledgements{
I would like to express my sincere gratitude to Dr Maria Polukarov for her guidance and support which provided me the freedom to take this research in the direction of my interest.\\
\\
I would also like to thank my family and friends for their encouragement and support. To those who quietly listened to my software complaints. To those who worked throughout the nights with me. To those who helped me write what I couldn't say. I cannot thank you enough.
}

\declaration{
I, Stefan Collier, declare that this dissertation and the work presented in it are my own and has been generated by me as the result of my own original research.\\
I confirm that:\\
1. This work was done wholly or mainly while in candidature for a degree at this University;\\
2. Where any part of this dissertation has previously been submitted for any other qualification at this University or any other institution, this has been clearly stated;\\
3. Where I have consulted the published work of others, this is always clearly attributed;\\
4. Where I have quoted from the work of others, the source is always given. With the exception of such quotations, this dissertation is entirely my own work;\\
5. I have acknowledged all main sources of help;\\
6. Where the thesis is based on work done by myself jointly with others, I have made clear exactly what was done by others and what I have contributed myself;\\
7. Either none of this work has been published before submission, or parts of this work have been published by :\\
\\
Stefan Collier\\
April 2016
}
\tableofcontents
\listoffigures
\listoftables

\mainmatter
%% ----------------------------------------------------------------
%\include{Introduction}
%\include{Conclusions}
\include{chapters/1Project/main}
\include{chapters/2Lit/main}
\include{chapters/3Design/HighLevel}
\include{chapters/3Design/InDepth}
\include{chapters/4Impl/main}

\include{chapters/5Experiments/1/main}
\include{chapters/5Experiments/2/main}
\include{chapters/5Experiments/3/main}
\include{chapters/5Experiments/4/main}

\include{chapters/6Conclusion/main}

\appendix
\include{appendix/AppendixB}
\include{appendix/D/main}
\include{appendix/AppendixC}

\backmatter
\bibliographystyle{ecs}
\bibliography{ECS}
\end{document}
%% ----------------------------------------------------------------

 %% ----------------------------------------------------------------
%% Progress.tex
%% ---------------------------------------------------------------- 
\documentclass{ecsprogress}    % Use the progress Style
\graphicspath{{../figs/}}   % Location of your graphics files
    \usepackage{natbib}            % Use Natbib style for the refs.
\hypersetup{colorlinks=true}   % Set to false for black/white printing
\input{Definitions}            % Include your abbreviations



\usepackage{enumitem}% http://ctan.org/pkg/enumitem
\usepackage{multirow}
\usepackage{float}
\usepackage{amsmath}
\usepackage{multicol}
\usepackage{amssymb}
\usepackage[normalem]{ulem}
\useunder{\uline}{\ul}{}
\usepackage{wrapfig}


\usepackage[table,xcdraw]{xcolor}


%% ----------------------------------------------------------------
\begin{document}
\frontmatter
\title      {Heterogeneous Agent-based Model for Supermarket Competition}
\authors    {\texorpdfstring
             {\href{mailto:sc22g13@ecs.soton.ac.uk}{Stefan J. Collier}}
             {Stefan J. Collier}
            }
\addresses  {\groupname\\\deptname\\\univname}
\date       {\today}
\subject    {}
\keywords   {}
\supervisor {Dr. Maria Polukarov}
\examiner   {Professor Sheng Chen}

\maketitle
\begin{abstract}
This project aim was to model and analyse the effects of competitive pricing behaviors of grocery retailers on the British market. 

This was achieved by creating a multi-agent model, containing retailer and consumer agents. The heterogeneous crowd of retailers employs either a uniform pricing strategy or a ‘local price flexing’ strategy. The actions of these retailers are chosen by predicting the profit of each action, using a perceptron. Following on from the consideration of different economic models, a discrete model was developed so that software agents have a discrete environment to operate within. Within the model, it has been observed how supermarkets with differing behaviors affect a heterogeneous crowd of consumer agents. The model was implemented in Java with Python used to evaluate the results. 

The simulation displays good acceptance with real grocery market behavior, i.e. captures the performance of British retailers thus can be used to determine the impact of changes in their behavior on their competitors and consumers.Furthermore it can be used to provide insight into sustainability of volatile pricing strategies, providing a useful insight in volatility of British supermarket retail industry. 
\end{abstract}
\acknowledgements{
I would like to express my sincere gratitude to Dr Maria Polukarov for her guidance and support which provided me the freedom to take this research in the direction of my interest.\\
\\
I would also like to thank my family and friends for their encouragement and support. To those who quietly listened to my software complaints. To those who worked throughout the nights with me. To those who helped me write what I couldn't say. I cannot thank you enough.
}

\declaration{
I, Stefan Collier, declare that this dissertation and the work presented in it are my own and has been generated by me as the result of my own original research.\\
I confirm that:\\
1. This work was done wholly or mainly while in candidature for a degree at this University;\\
2. Where any part of this dissertation has previously been submitted for any other qualification at this University or any other institution, this has been clearly stated;\\
3. Where I have consulted the published work of others, this is always clearly attributed;\\
4. Where I have quoted from the work of others, the source is always given. With the exception of such quotations, this dissertation is entirely my own work;\\
5. I have acknowledged all main sources of help;\\
6. Where the thesis is based on work done by myself jointly with others, I have made clear exactly what was done by others and what I have contributed myself;\\
7. Either none of this work has been published before submission, or parts of this work have been published by :\\
\\
Stefan Collier\\
April 2016
}
\tableofcontents
\listoffigures
\listoftables

\mainmatter
%% ----------------------------------------------------------------
%\include{Introduction}
%\include{Conclusions}
\include{chapters/1Project/main}
\include{chapters/2Lit/main}
\include{chapters/3Design/HighLevel}
\include{chapters/3Design/InDepth}
\include{chapters/4Impl/main}

\include{chapters/5Experiments/1/main}
\include{chapters/5Experiments/2/main}
\include{chapters/5Experiments/3/main}
\include{chapters/5Experiments/4/main}

\include{chapters/6Conclusion/main}

\appendix
\include{appendix/AppendixB}
\include{appendix/D/main}
\include{appendix/AppendixC}

\backmatter
\bibliographystyle{ecs}
\bibliography{ECS}
\end{document}
%% ----------------------------------------------------------------

 %% ----------------------------------------------------------------
%% Progress.tex
%% ---------------------------------------------------------------- 
\documentclass{ecsprogress}    % Use the progress Style
\graphicspath{{../figs/}}   % Location of your graphics files
    \usepackage{natbib}            % Use Natbib style for the refs.
\hypersetup{colorlinks=true}   % Set to false for black/white printing
\input{Definitions}            % Include your abbreviations



\usepackage{enumitem}% http://ctan.org/pkg/enumitem
\usepackage{multirow}
\usepackage{float}
\usepackage{amsmath}
\usepackage{multicol}
\usepackage{amssymb}
\usepackage[normalem]{ulem}
\useunder{\uline}{\ul}{}
\usepackage{wrapfig}


\usepackage[table,xcdraw]{xcolor}


%% ----------------------------------------------------------------
\begin{document}
\frontmatter
\title      {Heterogeneous Agent-based Model for Supermarket Competition}
\authors    {\texorpdfstring
             {\href{mailto:sc22g13@ecs.soton.ac.uk}{Stefan J. Collier}}
             {Stefan J. Collier}
            }
\addresses  {\groupname\\\deptname\\\univname}
\date       {\today}
\subject    {}
\keywords   {}
\supervisor {Dr. Maria Polukarov}
\examiner   {Professor Sheng Chen}

\maketitle
\begin{abstract}
This project aim was to model and analyse the effects of competitive pricing behaviors of grocery retailers on the British market. 

This was achieved by creating a multi-agent model, containing retailer and consumer agents. The heterogeneous crowd of retailers employs either a uniform pricing strategy or a ‘local price flexing’ strategy. The actions of these retailers are chosen by predicting the profit of each action, using a perceptron. Following on from the consideration of different economic models, a discrete model was developed so that software agents have a discrete environment to operate within. Within the model, it has been observed how supermarkets with differing behaviors affect a heterogeneous crowd of consumer agents. The model was implemented in Java with Python used to evaluate the results. 

The simulation displays good acceptance with real grocery market behavior, i.e. captures the performance of British retailers thus can be used to determine the impact of changes in their behavior on their competitors and consumers.Furthermore it can be used to provide insight into sustainability of volatile pricing strategies, providing a useful insight in volatility of British supermarket retail industry. 
\end{abstract}
\acknowledgements{
I would like to express my sincere gratitude to Dr Maria Polukarov for her guidance and support which provided me the freedom to take this research in the direction of my interest.\\
\\
I would also like to thank my family and friends for their encouragement and support. To those who quietly listened to my software complaints. To those who worked throughout the nights with me. To those who helped me write what I couldn't say. I cannot thank you enough.
}

\declaration{
I, Stefan Collier, declare that this dissertation and the work presented in it are my own and has been generated by me as the result of my own original research.\\
I confirm that:\\
1. This work was done wholly or mainly while in candidature for a degree at this University;\\
2. Where any part of this dissertation has previously been submitted for any other qualification at this University or any other institution, this has been clearly stated;\\
3. Where I have consulted the published work of others, this is always clearly attributed;\\
4. Where I have quoted from the work of others, the source is always given. With the exception of such quotations, this dissertation is entirely my own work;\\
5. I have acknowledged all main sources of help;\\
6. Where the thesis is based on work done by myself jointly with others, I have made clear exactly what was done by others and what I have contributed myself;\\
7. Either none of this work has been published before submission, or parts of this work have been published by :\\
\\
Stefan Collier\\
April 2016
}
\tableofcontents
\listoffigures
\listoftables

\mainmatter
%% ----------------------------------------------------------------
%\include{Introduction}
%\include{Conclusions}
\include{chapters/1Project/main}
\include{chapters/2Lit/main}
\include{chapters/3Design/HighLevel}
\include{chapters/3Design/InDepth}
\include{chapters/4Impl/main}

\include{chapters/5Experiments/1/main}
\include{chapters/5Experiments/2/main}
\include{chapters/5Experiments/3/main}
\include{chapters/5Experiments/4/main}

\include{chapters/6Conclusion/main}

\appendix
\include{appendix/AppendixB}
\include{appendix/D/main}
\include{appendix/AppendixC}

\backmatter
\bibliographystyle{ecs}
\bibliography{ECS}
\end{document}
%% ----------------------------------------------------------------

 %% ----------------------------------------------------------------
%% Progress.tex
%% ---------------------------------------------------------------- 
\documentclass{ecsprogress}    % Use the progress Style
\graphicspath{{../figs/}}   % Location of your graphics files
    \usepackage{natbib}            % Use Natbib style for the refs.
\hypersetup{colorlinks=true}   % Set to false for black/white printing
\input{Definitions}            % Include your abbreviations



\usepackage{enumitem}% http://ctan.org/pkg/enumitem
\usepackage{multirow}
\usepackage{float}
\usepackage{amsmath}
\usepackage{multicol}
\usepackage{amssymb}
\usepackage[normalem]{ulem}
\useunder{\uline}{\ul}{}
\usepackage{wrapfig}


\usepackage[table,xcdraw]{xcolor}


%% ----------------------------------------------------------------
\begin{document}
\frontmatter
\title      {Heterogeneous Agent-based Model for Supermarket Competition}
\authors    {\texorpdfstring
             {\href{mailto:sc22g13@ecs.soton.ac.uk}{Stefan J. Collier}}
             {Stefan J. Collier}
            }
\addresses  {\groupname\\\deptname\\\univname}
\date       {\today}
\subject    {}
\keywords   {}
\supervisor {Dr. Maria Polukarov}
\examiner   {Professor Sheng Chen}

\maketitle
\begin{abstract}
This project aim was to model and analyse the effects of competitive pricing behaviors of grocery retailers on the British market. 

This was achieved by creating a multi-agent model, containing retailer and consumer agents. The heterogeneous crowd of retailers employs either a uniform pricing strategy or a ‘local price flexing’ strategy. The actions of these retailers are chosen by predicting the profit of each action, using a perceptron. Following on from the consideration of different economic models, a discrete model was developed so that software agents have a discrete environment to operate within. Within the model, it has been observed how supermarkets with differing behaviors affect a heterogeneous crowd of consumer agents. The model was implemented in Java with Python used to evaluate the results. 

The simulation displays good acceptance with real grocery market behavior, i.e. captures the performance of British retailers thus can be used to determine the impact of changes in their behavior on their competitors and consumers.Furthermore it can be used to provide insight into sustainability of volatile pricing strategies, providing a useful insight in volatility of British supermarket retail industry. 
\end{abstract}
\acknowledgements{
I would like to express my sincere gratitude to Dr Maria Polukarov for her guidance and support which provided me the freedom to take this research in the direction of my interest.\\
\\
I would also like to thank my family and friends for their encouragement and support. To those who quietly listened to my software complaints. To those who worked throughout the nights with me. To those who helped me write what I couldn't say. I cannot thank you enough.
}

\declaration{
I, Stefan Collier, declare that this dissertation and the work presented in it are my own and has been generated by me as the result of my own original research.\\
I confirm that:\\
1. This work was done wholly or mainly while in candidature for a degree at this University;\\
2. Where any part of this dissertation has previously been submitted for any other qualification at this University or any other institution, this has been clearly stated;\\
3. Where I have consulted the published work of others, this is always clearly attributed;\\
4. Where I have quoted from the work of others, the source is always given. With the exception of such quotations, this dissertation is entirely my own work;\\
5. I have acknowledged all main sources of help;\\
6. Where the thesis is based on work done by myself jointly with others, I have made clear exactly what was done by others and what I have contributed myself;\\
7. Either none of this work has been published before submission, or parts of this work have been published by :\\
\\
Stefan Collier\\
April 2016
}
\tableofcontents
\listoffigures
\listoftables

\mainmatter
%% ----------------------------------------------------------------
%\include{Introduction}
%\include{Conclusions}
\include{chapters/1Project/main}
\include{chapters/2Lit/main}
\include{chapters/3Design/HighLevel}
\include{chapters/3Design/InDepth}
\include{chapters/4Impl/main}

\include{chapters/5Experiments/1/main}
\include{chapters/5Experiments/2/main}
\include{chapters/5Experiments/3/main}
\include{chapters/5Experiments/4/main}

\include{chapters/6Conclusion/main}

\appendix
\include{appendix/AppendixB}
\include{appendix/D/main}
\include{appendix/AppendixC}

\backmatter
\bibliographystyle{ecs}
\bibliography{ECS}
\end{document}
%% ----------------------------------------------------------------


 %% ----------------------------------------------------------------
%% Progress.tex
%% ---------------------------------------------------------------- 
\documentclass{ecsprogress}    % Use the progress Style
\graphicspath{{../figs/}}   % Location of your graphics files
    \usepackage{natbib}            % Use Natbib style for the refs.
\hypersetup{colorlinks=true}   % Set to false for black/white printing
\input{Definitions}            % Include your abbreviations



\usepackage{enumitem}% http://ctan.org/pkg/enumitem
\usepackage{multirow}
\usepackage{float}
\usepackage{amsmath}
\usepackage{multicol}
\usepackage{amssymb}
\usepackage[normalem]{ulem}
\useunder{\uline}{\ul}{}
\usepackage{wrapfig}


\usepackage[table,xcdraw]{xcolor}


%% ----------------------------------------------------------------
\begin{document}
\frontmatter
\title      {Heterogeneous Agent-based Model for Supermarket Competition}
\authors    {\texorpdfstring
             {\href{mailto:sc22g13@ecs.soton.ac.uk}{Stefan J. Collier}}
             {Stefan J. Collier}
            }
\addresses  {\groupname\\\deptname\\\univname}
\date       {\today}
\subject    {}
\keywords   {}
\supervisor {Dr. Maria Polukarov}
\examiner   {Professor Sheng Chen}

\maketitle
\begin{abstract}
This project aim was to model and analyse the effects of competitive pricing behaviors of grocery retailers on the British market. 

This was achieved by creating a multi-agent model, containing retailer and consumer agents. The heterogeneous crowd of retailers employs either a uniform pricing strategy or a ‘local price flexing’ strategy. The actions of these retailers are chosen by predicting the profit of each action, using a perceptron. Following on from the consideration of different economic models, a discrete model was developed so that software agents have a discrete environment to operate within. Within the model, it has been observed how supermarkets with differing behaviors affect a heterogeneous crowd of consumer agents. The model was implemented in Java with Python used to evaluate the results. 

The simulation displays good acceptance with real grocery market behavior, i.e. captures the performance of British retailers thus can be used to determine the impact of changes in their behavior on their competitors and consumers.Furthermore it can be used to provide insight into sustainability of volatile pricing strategies, providing a useful insight in volatility of British supermarket retail industry. 
\end{abstract}
\acknowledgements{
I would like to express my sincere gratitude to Dr Maria Polukarov for her guidance and support which provided me the freedom to take this research in the direction of my interest.\\
\\
I would also like to thank my family and friends for their encouragement and support. To those who quietly listened to my software complaints. To those who worked throughout the nights with me. To those who helped me write what I couldn't say. I cannot thank you enough.
}

\declaration{
I, Stefan Collier, declare that this dissertation and the work presented in it are my own and has been generated by me as the result of my own original research.\\
I confirm that:\\
1. This work was done wholly or mainly while in candidature for a degree at this University;\\
2. Where any part of this dissertation has previously been submitted for any other qualification at this University or any other institution, this has been clearly stated;\\
3. Where I have consulted the published work of others, this is always clearly attributed;\\
4. Where I have quoted from the work of others, the source is always given. With the exception of such quotations, this dissertation is entirely my own work;\\
5. I have acknowledged all main sources of help;\\
6. Where the thesis is based on work done by myself jointly with others, I have made clear exactly what was done by others and what I have contributed myself;\\
7. Either none of this work has been published before submission, or parts of this work have been published by :\\
\\
Stefan Collier\\
April 2016
}
\tableofcontents
\listoffigures
\listoftables

\mainmatter
%% ----------------------------------------------------------------
%\include{Introduction}
%\include{Conclusions}
\include{chapters/1Project/main}
\include{chapters/2Lit/main}
\include{chapters/3Design/HighLevel}
\include{chapters/3Design/InDepth}
\include{chapters/4Impl/main}

\include{chapters/5Experiments/1/main}
\include{chapters/5Experiments/2/main}
\include{chapters/5Experiments/3/main}
\include{chapters/5Experiments/4/main}

\include{chapters/6Conclusion/main}

\appendix
\include{appendix/AppendixB}
\include{appendix/D/main}
\include{appendix/AppendixC}

\backmatter
\bibliographystyle{ecs}
\bibliography{ECS}
\end{document}
%% ----------------------------------------------------------------


\appendix
\include{appendix/AppendixB}
 %% ----------------------------------------------------------------
%% Progress.tex
%% ---------------------------------------------------------------- 
\documentclass{ecsprogress}    % Use the progress Style
\graphicspath{{../figs/}}   % Location of your graphics files
    \usepackage{natbib}            % Use Natbib style for the refs.
\hypersetup{colorlinks=true}   % Set to false for black/white printing
\input{Definitions}            % Include your abbreviations



\usepackage{enumitem}% http://ctan.org/pkg/enumitem
\usepackage{multirow}
\usepackage{float}
\usepackage{amsmath}
\usepackage{multicol}
\usepackage{amssymb}
\usepackage[normalem]{ulem}
\useunder{\uline}{\ul}{}
\usepackage{wrapfig}


\usepackage[table,xcdraw]{xcolor}


%% ----------------------------------------------------------------
\begin{document}
\frontmatter
\title      {Heterogeneous Agent-based Model for Supermarket Competition}
\authors    {\texorpdfstring
             {\href{mailto:sc22g13@ecs.soton.ac.uk}{Stefan J. Collier}}
             {Stefan J. Collier}
            }
\addresses  {\groupname\\\deptname\\\univname}
\date       {\today}
\subject    {}
\keywords   {}
\supervisor {Dr. Maria Polukarov}
\examiner   {Professor Sheng Chen}

\maketitle
\begin{abstract}
This project aim was to model and analyse the effects of competitive pricing behaviors of grocery retailers on the British market. 

This was achieved by creating a multi-agent model, containing retailer and consumer agents. The heterogeneous crowd of retailers employs either a uniform pricing strategy or a ‘local price flexing’ strategy. The actions of these retailers are chosen by predicting the profit of each action, using a perceptron. Following on from the consideration of different economic models, a discrete model was developed so that software agents have a discrete environment to operate within. Within the model, it has been observed how supermarkets with differing behaviors affect a heterogeneous crowd of consumer agents. The model was implemented in Java with Python used to evaluate the results. 

The simulation displays good acceptance with real grocery market behavior, i.e. captures the performance of British retailers thus can be used to determine the impact of changes in their behavior on their competitors and consumers.Furthermore it can be used to provide insight into sustainability of volatile pricing strategies, providing a useful insight in volatility of British supermarket retail industry. 
\end{abstract}
\acknowledgements{
I would like to express my sincere gratitude to Dr Maria Polukarov for her guidance and support which provided me the freedom to take this research in the direction of my interest.\\
\\
I would also like to thank my family and friends for their encouragement and support. To those who quietly listened to my software complaints. To those who worked throughout the nights with me. To those who helped me write what I couldn't say. I cannot thank you enough.
}

\declaration{
I, Stefan Collier, declare that this dissertation and the work presented in it are my own and has been generated by me as the result of my own original research.\\
I confirm that:\\
1. This work was done wholly or mainly while in candidature for a degree at this University;\\
2. Where any part of this dissertation has previously been submitted for any other qualification at this University or any other institution, this has been clearly stated;\\
3. Where I have consulted the published work of others, this is always clearly attributed;\\
4. Where I have quoted from the work of others, the source is always given. With the exception of such quotations, this dissertation is entirely my own work;\\
5. I have acknowledged all main sources of help;\\
6. Where the thesis is based on work done by myself jointly with others, I have made clear exactly what was done by others and what I have contributed myself;\\
7. Either none of this work has been published before submission, or parts of this work have been published by :\\
\\
Stefan Collier\\
April 2016
}
\tableofcontents
\listoffigures
\listoftables

\mainmatter
%% ----------------------------------------------------------------
%\include{Introduction}
%\include{Conclusions}
\include{chapters/1Project/main}
\include{chapters/2Lit/main}
\include{chapters/3Design/HighLevel}
\include{chapters/3Design/InDepth}
\include{chapters/4Impl/main}

\include{chapters/5Experiments/1/main}
\include{chapters/5Experiments/2/main}
\include{chapters/5Experiments/3/main}
\include{chapters/5Experiments/4/main}

\include{chapters/6Conclusion/main}

\appendix
\include{appendix/AppendixB}
\include{appendix/D/main}
\include{appendix/AppendixC}

\backmatter
\bibliographystyle{ecs}
\bibliography{ECS}
\end{document}
%% ----------------------------------------------------------------

\include{appendix/AppendixC}

\backmatter
\bibliographystyle{ecs}
\bibliography{ECS}
\end{document}
%% ----------------------------------------------------------------

 %% ----------------------------------------------------------------
%% Progress.tex
%% ---------------------------------------------------------------- 
\documentclass{ecsprogress}    % Use the progress Style
\graphicspath{{../figs/}}   % Location of your graphics files
    \usepackage{natbib}            % Use Natbib style for the refs.
\hypersetup{colorlinks=true}   % Set to false for black/white printing
\input{Definitions}            % Include your abbreviations



\usepackage{enumitem}% http://ctan.org/pkg/enumitem
\usepackage{multirow}
\usepackage{float}
\usepackage{amsmath}
\usepackage{multicol}
\usepackage{amssymb}
\usepackage[normalem]{ulem}
\useunder{\uline}{\ul}{}
\usepackage{wrapfig}


\usepackage[table,xcdraw]{xcolor}


%% ----------------------------------------------------------------
\begin{document}
\frontmatter
\title      {Heterogeneous Agent-based Model for Supermarket Competition}
\authors    {\texorpdfstring
             {\href{mailto:sc22g13@ecs.soton.ac.uk}{Stefan J. Collier}}
             {Stefan J. Collier}
            }
\addresses  {\groupname\\\deptname\\\univname}
\date       {\today}
\subject    {}
\keywords   {}
\supervisor {Dr. Maria Polukarov}
\examiner   {Professor Sheng Chen}

\maketitle
\begin{abstract}
This project aim was to model and analyse the effects of competitive pricing behaviors of grocery retailers on the British market. 

This was achieved by creating a multi-agent model, containing retailer and consumer agents. The heterogeneous crowd of retailers employs either a uniform pricing strategy or a ‘local price flexing’ strategy. The actions of these retailers are chosen by predicting the profit of each action, using a perceptron. Following on from the consideration of different economic models, a discrete model was developed so that software agents have a discrete environment to operate within. Within the model, it has been observed how supermarkets with differing behaviors affect a heterogeneous crowd of consumer agents. The model was implemented in Java with Python used to evaluate the results. 

The simulation displays good acceptance with real grocery market behavior, i.e. captures the performance of British retailers thus can be used to determine the impact of changes in their behavior on their competitors and consumers.Furthermore it can be used to provide insight into sustainability of volatile pricing strategies, providing a useful insight in volatility of British supermarket retail industry. 
\end{abstract}
\acknowledgements{
I would like to express my sincere gratitude to Dr Maria Polukarov for her guidance and support which provided me the freedom to take this research in the direction of my interest.\\
\\
I would also like to thank my family and friends for their encouragement and support. To those who quietly listened to my software complaints. To those who worked throughout the nights with me. To those who helped me write what I couldn't say. I cannot thank you enough.
}

\declaration{
I, Stefan Collier, declare that this dissertation and the work presented in it are my own and has been generated by me as the result of my own original research.\\
I confirm that:\\
1. This work was done wholly or mainly while in candidature for a degree at this University;\\
2. Where any part of this dissertation has previously been submitted for any other qualification at this University or any other institution, this has been clearly stated;\\
3. Where I have consulted the published work of others, this is always clearly attributed;\\
4. Where I have quoted from the work of others, the source is always given. With the exception of such quotations, this dissertation is entirely my own work;\\
5. I have acknowledged all main sources of help;\\
6. Where the thesis is based on work done by myself jointly with others, I have made clear exactly what was done by others and what I have contributed myself;\\
7. Either none of this work has been published before submission, or parts of this work have been published by :\\
\\
Stefan Collier\\
April 2016
}
\tableofcontents
\listoffigures
\listoftables

\mainmatter
%% ----------------------------------------------------------------
%\include{Introduction}
%\include{Conclusions}
 %% ----------------------------------------------------------------
%% Progress.tex
%% ---------------------------------------------------------------- 
\documentclass{ecsprogress}    % Use the progress Style
\graphicspath{{../figs/}}   % Location of your graphics files
    \usepackage{natbib}            % Use Natbib style for the refs.
\hypersetup{colorlinks=true}   % Set to false for black/white printing
\input{Definitions}            % Include your abbreviations



\usepackage{enumitem}% http://ctan.org/pkg/enumitem
\usepackage{multirow}
\usepackage{float}
\usepackage{amsmath}
\usepackage{multicol}
\usepackage{amssymb}
\usepackage[normalem]{ulem}
\useunder{\uline}{\ul}{}
\usepackage{wrapfig}


\usepackage[table,xcdraw]{xcolor}


%% ----------------------------------------------------------------
\begin{document}
\frontmatter
\title      {Heterogeneous Agent-based Model for Supermarket Competition}
\authors    {\texorpdfstring
             {\href{mailto:sc22g13@ecs.soton.ac.uk}{Stefan J. Collier}}
             {Stefan J. Collier}
            }
\addresses  {\groupname\\\deptname\\\univname}
\date       {\today}
\subject    {}
\keywords   {}
\supervisor {Dr. Maria Polukarov}
\examiner   {Professor Sheng Chen}

\maketitle
\begin{abstract}
This project aim was to model and analyse the effects of competitive pricing behaviors of grocery retailers on the British market. 

This was achieved by creating a multi-agent model, containing retailer and consumer agents. The heterogeneous crowd of retailers employs either a uniform pricing strategy or a ‘local price flexing’ strategy. The actions of these retailers are chosen by predicting the profit of each action, using a perceptron. Following on from the consideration of different economic models, a discrete model was developed so that software agents have a discrete environment to operate within. Within the model, it has been observed how supermarkets with differing behaviors affect a heterogeneous crowd of consumer agents. The model was implemented in Java with Python used to evaluate the results. 

The simulation displays good acceptance with real grocery market behavior, i.e. captures the performance of British retailers thus can be used to determine the impact of changes in their behavior on their competitors and consumers.Furthermore it can be used to provide insight into sustainability of volatile pricing strategies, providing a useful insight in volatility of British supermarket retail industry. 
\end{abstract}
\acknowledgements{
I would like to express my sincere gratitude to Dr Maria Polukarov for her guidance and support which provided me the freedom to take this research in the direction of my interest.\\
\\
I would also like to thank my family and friends for their encouragement and support. To those who quietly listened to my software complaints. To those who worked throughout the nights with me. To those who helped me write what I couldn't say. I cannot thank you enough.
}

\declaration{
I, Stefan Collier, declare that this dissertation and the work presented in it are my own and has been generated by me as the result of my own original research.\\
I confirm that:\\
1. This work was done wholly or mainly while in candidature for a degree at this University;\\
2. Where any part of this dissertation has previously been submitted for any other qualification at this University or any other institution, this has been clearly stated;\\
3. Where I have consulted the published work of others, this is always clearly attributed;\\
4. Where I have quoted from the work of others, the source is always given. With the exception of such quotations, this dissertation is entirely my own work;\\
5. I have acknowledged all main sources of help;\\
6. Where the thesis is based on work done by myself jointly with others, I have made clear exactly what was done by others and what I have contributed myself;\\
7. Either none of this work has been published before submission, or parts of this work have been published by :\\
\\
Stefan Collier\\
April 2016
}
\tableofcontents
\listoffigures
\listoftables

\mainmatter
%% ----------------------------------------------------------------
%\include{Introduction}
%\include{Conclusions}
\include{chapters/1Project/main}
\include{chapters/2Lit/main}
\include{chapters/3Design/HighLevel}
\include{chapters/3Design/InDepth}
\include{chapters/4Impl/main}

\include{chapters/5Experiments/1/main}
\include{chapters/5Experiments/2/main}
\include{chapters/5Experiments/3/main}
\include{chapters/5Experiments/4/main}

\include{chapters/6Conclusion/main}

\appendix
\include{appendix/AppendixB}
\include{appendix/D/main}
\include{appendix/AppendixC}

\backmatter
\bibliographystyle{ecs}
\bibliography{ECS}
\end{document}
%% ----------------------------------------------------------------

 %% ----------------------------------------------------------------
%% Progress.tex
%% ---------------------------------------------------------------- 
\documentclass{ecsprogress}    % Use the progress Style
\graphicspath{{../figs/}}   % Location of your graphics files
    \usepackage{natbib}            % Use Natbib style for the refs.
\hypersetup{colorlinks=true}   % Set to false for black/white printing
\input{Definitions}            % Include your abbreviations



\usepackage{enumitem}% http://ctan.org/pkg/enumitem
\usepackage{multirow}
\usepackage{float}
\usepackage{amsmath}
\usepackage{multicol}
\usepackage{amssymb}
\usepackage[normalem]{ulem}
\useunder{\uline}{\ul}{}
\usepackage{wrapfig}


\usepackage[table,xcdraw]{xcolor}


%% ----------------------------------------------------------------
\begin{document}
\frontmatter
\title      {Heterogeneous Agent-based Model for Supermarket Competition}
\authors    {\texorpdfstring
             {\href{mailto:sc22g13@ecs.soton.ac.uk}{Stefan J. Collier}}
             {Stefan J. Collier}
            }
\addresses  {\groupname\\\deptname\\\univname}
\date       {\today}
\subject    {}
\keywords   {}
\supervisor {Dr. Maria Polukarov}
\examiner   {Professor Sheng Chen}

\maketitle
\begin{abstract}
This project aim was to model and analyse the effects of competitive pricing behaviors of grocery retailers on the British market. 

This was achieved by creating a multi-agent model, containing retailer and consumer agents. The heterogeneous crowd of retailers employs either a uniform pricing strategy or a ‘local price flexing’ strategy. The actions of these retailers are chosen by predicting the profit of each action, using a perceptron. Following on from the consideration of different economic models, a discrete model was developed so that software agents have a discrete environment to operate within. Within the model, it has been observed how supermarkets with differing behaviors affect a heterogeneous crowd of consumer agents. The model was implemented in Java with Python used to evaluate the results. 

The simulation displays good acceptance with real grocery market behavior, i.e. captures the performance of British retailers thus can be used to determine the impact of changes in their behavior on their competitors and consumers.Furthermore it can be used to provide insight into sustainability of volatile pricing strategies, providing a useful insight in volatility of British supermarket retail industry. 
\end{abstract}
\acknowledgements{
I would like to express my sincere gratitude to Dr Maria Polukarov for her guidance and support which provided me the freedom to take this research in the direction of my interest.\\
\\
I would also like to thank my family and friends for their encouragement and support. To those who quietly listened to my software complaints. To those who worked throughout the nights with me. To those who helped me write what I couldn't say. I cannot thank you enough.
}

\declaration{
I, Stefan Collier, declare that this dissertation and the work presented in it are my own and has been generated by me as the result of my own original research.\\
I confirm that:\\
1. This work was done wholly or mainly while in candidature for a degree at this University;\\
2. Where any part of this dissertation has previously been submitted for any other qualification at this University or any other institution, this has been clearly stated;\\
3. Where I have consulted the published work of others, this is always clearly attributed;\\
4. Where I have quoted from the work of others, the source is always given. With the exception of such quotations, this dissertation is entirely my own work;\\
5. I have acknowledged all main sources of help;\\
6. Where the thesis is based on work done by myself jointly with others, I have made clear exactly what was done by others and what I have contributed myself;\\
7. Either none of this work has been published before submission, or parts of this work have been published by :\\
\\
Stefan Collier\\
April 2016
}
\tableofcontents
\listoffigures
\listoftables

\mainmatter
%% ----------------------------------------------------------------
%\include{Introduction}
%\include{Conclusions}
\include{chapters/1Project/main}
\include{chapters/2Lit/main}
\include{chapters/3Design/HighLevel}
\include{chapters/3Design/InDepth}
\include{chapters/4Impl/main}

\include{chapters/5Experiments/1/main}
\include{chapters/5Experiments/2/main}
\include{chapters/5Experiments/3/main}
\include{chapters/5Experiments/4/main}

\include{chapters/6Conclusion/main}

\appendix
\include{appendix/AppendixB}
\include{appendix/D/main}
\include{appendix/AppendixC}

\backmatter
\bibliographystyle{ecs}
\bibliography{ECS}
\end{document}
%% ----------------------------------------------------------------

\include{chapters/3Design/HighLevel}
\include{chapters/3Design/InDepth}
 %% ----------------------------------------------------------------
%% Progress.tex
%% ---------------------------------------------------------------- 
\documentclass{ecsprogress}    % Use the progress Style
\graphicspath{{../figs/}}   % Location of your graphics files
    \usepackage{natbib}            % Use Natbib style for the refs.
\hypersetup{colorlinks=true}   % Set to false for black/white printing
\input{Definitions}            % Include your abbreviations



\usepackage{enumitem}% http://ctan.org/pkg/enumitem
\usepackage{multirow}
\usepackage{float}
\usepackage{amsmath}
\usepackage{multicol}
\usepackage{amssymb}
\usepackage[normalem]{ulem}
\useunder{\uline}{\ul}{}
\usepackage{wrapfig}


\usepackage[table,xcdraw]{xcolor}


%% ----------------------------------------------------------------
\begin{document}
\frontmatter
\title      {Heterogeneous Agent-based Model for Supermarket Competition}
\authors    {\texorpdfstring
             {\href{mailto:sc22g13@ecs.soton.ac.uk}{Stefan J. Collier}}
             {Stefan J. Collier}
            }
\addresses  {\groupname\\\deptname\\\univname}
\date       {\today}
\subject    {}
\keywords   {}
\supervisor {Dr. Maria Polukarov}
\examiner   {Professor Sheng Chen}

\maketitle
\begin{abstract}
This project aim was to model and analyse the effects of competitive pricing behaviors of grocery retailers on the British market. 

This was achieved by creating a multi-agent model, containing retailer and consumer agents. The heterogeneous crowd of retailers employs either a uniform pricing strategy or a ‘local price flexing’ strategy. The actions of these retailers are chosen by predicting the profit of each action, using a perceptron. Following on from the consideration of different economic models, a discrete model was developed so that software agents have a discrete environment to operate within. Within the model, it has been observed how supermarkets with differing behaviors affect a heterogeneous crowd of consumer agents. The model was implemented in Java with Python used to evaluate the results. 

The simulation displays good acceptance with real grocery market behavior, i.e. captures the performance of British retailers thus can be used to determine the impact of changes in their behavior on their competitors and consumers.Furthermore it can be used to provide insight into sustainability of volatile pricing strategies, providing a useful insight in volatility of British supermarket retail industry. 
\end{abstract}
\acknowledgements{
I would like to express my sincere gratitude to Dr Maria Polukarov for her guidance and support which provided me the freedom to take this research in the direction of my interest.\\
\\
I would also like to thank my family and friends for their encouragement and support. To those who quietly listened to my software complaints. To those who worked throughout the nights with me. To those who helped me write what I couldn't say. I cannot thank you enough.
}

\declaration{
I, Stefan Collier, declare that this dissertation and the work presented in it are my own and has been generated by me as the result of my own original research.\\
I confirm that:\\
1. This work was done wholly or mainly while in candidature for a degree at this University;\\
2. Where any part of this dissertation has previously been submitted for any other qualification at this University or any other institution, this has been clearly stated;\\
3. Where I have consulted the published work of others, this is always clearly attributed;\\
4. Where I have quoted from the work of others, the source is always given. With the exception of such quotations, this dissertation is entirely my own work;\\
5. I have acknowledged all main sources of help;\\
6. Where the thesis is based on work done by myself jointly with others, I have made clear exactly what was done by others and what I have contributed myself;\\
7. Either none of this work has been published before submission, or parts of this work have been published by :\\
\\
Stefan Collier\\
April 2016
}
\tableofcontents
\listoffigures
\listoftables

\mainmatter
%% ----------------------------------------------------------------
%\include{Introduction}
%\include{Conclusions}
\include{chapters/1Project/main}
\include{chapters/2Lit/main}
\include{chapters/3Design/HighLevel}
\include{chapters/3Design/InDepth}
\include{chapters/4Impl/main}

\include{chapters/5Experiments/1/main}
\include{chapters/5Experiments/2/main}
\include{chapters/5Experiments/3/main}
\include{chapters/5Experiments/4/main}

\include{chapters/6Conclusion/main}

\appendix
\include{appendix/AppendixB}
\include{appendix/D/main}
\include{appendix/AppendixC}

\backmatter
\bibliographystyle{ecs}
\bibliography{ECS}
\end{document}
%% ----------------------------------------------------------------


 %% ----------------------------------------------------------------
%% Progress.tex
%% ---------------------------------------------------------------- 
\documentclass{ecsprogress}    % Use the progress Style
\graphicspath{{../figs/}}   % Location of your graphics files
    \usepackage{natbib}            % Use Natbib style for the refs.
\hypersetup{colorlinks=true}   % Set to false for black/white printing
\input{Definitions}            % Include your abbreviations



\usepackage{enumitem}% http://ctan.org/pkg/enumitem
\usepackage{multirow}
\usepackage{float}
\usepackage{amsmath}
\usepackage{multicol}
\usepackage{amssymb}
\usepackage[normalem]{ulem}
\useunder{\uline}{\ul}{}
\usepackage{wrapfig}


\usepackage[table,xcdraw]{xcolor}


%% ----------------------------------------------------------------
\begin{document}
\frontmatter
\title      {Heterogeneous Agent-based Model for Supermarket Competition}
\authors    {\texorpdfstring
             {\href{mailto:sc22g13@ecs.soton.ac.uk}{Stefan J. Collier}}
             {Stefan J. Collier}
            }
\addresses  {\groupname\\\deptname\\\univname}
\date       {\today}
\subject    {}
\keywords   {}
\supervisor {Dr. Maria Polukarov}
\examiner   {Professor Sheng Chen}

\maketitle
\begin{abstract}
This project aim was to model and analyse the effects of competitive pricing behaviors of grocery retailers on the British market. 

This was achieved by creating a multi-agent model, containing retailer and consumer agents. The heterogeneous crowd of retailers employs either a uniform pricing strategy or a ‘local price flexing’ strategy. The actions of these retailers are chosen by predicting the profit of each action, using a perceptron. Following on from the consideration of different economic models, a discrete model was developed so that software agents have a discrete environment to operate within. Within the model, it has been observed how supermarkets with differing behaviors affect a heterogeneous crowd of consumer agents. The model was implemented in Java with Python used to evaluate the results. 

The simulation displays good acceptance with real grocery market behavior, i.e. captures the performance of British retailers thus can be used to determine the impact of changes in their behavior on their competitors and consumers.Furthermore it can be used to provide insight into sustainability of volatile pricing strategies, providing a useful insight in volatility of British supermarket retail industry. 
\end{abstract}
\acknowledgements{
I would like to express my sincere gratitude to Dr Maria Polukarov for her guidance and support which provided me the freedom to take this research in the direction of my interest.\\
\\
I would also like to thank my family and friends for their encouragement and support. To those who quietly listened to my software complaints. To those who worked throughout the nights with me. To those who helped me write what I couldn't say. I cannot thank you enough.
}

\declaration{
I, Stefan Collier, declare that this dissertation and the work presented in it are my own and has been generated by me as the result of my own original research.\\
I confirm that:\\
1. This work was done wholly or mainly while in candidature for a degree at this University;\\
2. Where any part of this dissertation has previously been submitted for any other qualification at this University or any other institution, this has been clearly stated;\\
3. Where I have consulted the published work of others, this is always clearly attributed;\\
4. Where I have quoted from the work of others, the source is always given. With the exception of such quotations, this dissertation is entirely my own work;\\
5. I have acknowledged all main sources of help;\\
6. Where the thesis is based on work done by myself jointly with others, I have made clear exactly what was done by others and what I have contributed myself;\\
7. Either none of this work has been published before submission, or parts of this work have been published by :\\
\\
Stefan Collier\\
April 2016
}
\tableofcontents
\listoffigures
\listoftables

\mainmatter
%% ----------------------------------------------------------------
%\include{Introduction}
%\include{Conclusions}
\include{chapters/1Project/main}
\include{chapters/2Lit/main}
\include{chapters/3Design/HighLevel}
\include{chapters/3Design/InDepth}
\include{chapters/4Impl/main}

\include{chapters/5Experiments/1/main}
\include{chapters/5Experiments/2/main}
\include{chapters/5Experiments/3/main}
\include{chapters/5Experiments/4/main}

\include{chapters/6Conclusion/main}

\appendix
\include{appendix/AppendixB}
\include{appendix/D/main}
\include{appendix/AppendixC}

\backmatter
\bibliographystyle{ecs}
\bibliography{ECS}
\end{document}
%% ----------------------------------------------------------------

 %% ----------------------------------------------------------------
%% Progress.tex
%% ---------------------------------------------------------------- 
\documentclass{ecsprogress}    % Use the progress Style
\graphicspath{{../figs/}}   % Location of your graphics files
    \usepackage{natbib}            % Use Natbib style for the refs.
\hypersetup{colorlinks=true}   % Set to false for black/white printing
\input{Definitions}            % Include your abbreviations



\usepackage{enumitem}% http://ctan.org/pkg/enumitem
\usepackage{multirow}
\usepackage{float}
\usepackage{amsmath}
\usepackage{multicol}
\usepackage{amssymb}
\usepackage[normalem]{ulem}
\useunder{\uline}{\ul}{}
\usepackage{wrapfig}


\usepackage[table,xcdraw]{xcolor}


%% ----------------------------------------------------------------
\begin{document}
\frontmatter
\title      {Heterogeneous Agent-based Model for Supermarket Competition}
\authors    {\texorpdfstring
             {\href{mailto:sc22g13@ecs.soton.ac.uk}{Stefan J. Collier}}
             {Stefan J. Collier}
            }
\addresses  {\groupname\\\deptname\\\univname}
\date       {\today}
\subject    {}
\keywords   {}
\supervisor {Dr. Maria Polukarov}
\examiner   {Professor Sheng Chen}

\maketitle
\begin{abstract}
This project aim was to model and analyse the effects of competitive pricing behaviors of grocery retailers on the British market. 

This was achieved by creating a multi-agent model, containing retailer and consumer agents. The heterogeneous crowd of retailers employs either a uniform pricing strategy or a ‘local price flexing’ strategy. The actions of these retailers are chosen by predicting the profit of each action, using a perceptron. Following on from the consideration of different economic models, a discrete model was developed so that software agents have a discrete environment to operate within. Within the model, it has been observed how supermarkets with differing behaviors affect a heterogeneous crowd of consumer agents. The model was implemented in Java with Python used to evaluate the results. 

The simulation displays good acceptance with real grocery market behavior, i.e. captures the performance of British retailers thus can be used to determine the impact of changes in their behavior on their competitors and consumers.Furthermore it can be used to provide insight into sustainability of volatile pricing strategies, providing a useful insight in volatility of British supermarket retail industry. 
\end{abstract}
\acknowledgements{
I would like to express my sincere gratitude to Dr Maria Polukarov for her guidance and support which provided me the freedom to take this research in the direction of my interest.\\
\\
I would also like to thank my family and friends for their encouragement and support. To those who quietly listened to my software complaints. To those who worked throughout the nights with me. To those who helped me write what I couldn't say. I cannot thank you enough.
}

\declaration{
I, Stefan Collier, declare that this dissertation and the work presented in it are my own and has been generated by me as the result of my own original research.\\
I confirm that:\\
1. This work was done wholly or mainly while in candidature for a degree at this University;\\
2. Where any part of this dissertation has previously been submitted for any other qualification at this University or any other institution, this has been clearly stated;\\
3. Where I have consulted the published work of others, this is always clearly attributed;\\
4. Where I have quoted from the work of others, the source is always given. With the exception of such quotations, this dissertation is entirely my own work;\\
5. I have acknowledged all main sources of help;\\
6. Where the thesis is based on work done by myself jointly with others, I have made clear exactly what was done by others and what I have contributed myself;\\
7. Either none of this work has been published before submission, or parts of this work have been published by :\\
\\
Stefan Collier\\
April 2016
}
\tableofcontents
\listoffigures
\listoftables

\mainmatter
%% ----------------------------------------------------------------
%\include{Introduction}
%\include{Conclusions}
\include{chapters/1Project/main}
\include{chapters/2Lit/main}
\include{chapters/3Design/HighLevel}
\include{chapters/3Design/InDepth}
\include{chapters/4Impl/main}

\include{chapters/5Experiments/1/main}
\include{chapters/5Experiments/2/main}
\include{chapters/5Experiments/3/main}
\include{chapters/5Experiments/4/main}

\include{chapters/6Conclusion/main}

\appendix
\include{appendix/AppendixB}
\include{appendix/D/main}
\include{appendix/AppendixC}

\backmatter
\bibliographystyle{ecs}
\bibliography{ECS}
\end{document}
%% ----------------------------------------------------------------

 %% ----------------------------------------------------------------
%% Progress.tex
%% ---------------------------------------------------------------- 
\documentclass{ecsprogress}    % Use the progress Style
\graphicspath{{../figs/}}   % Location of your graphics files
    \usepackage{natbib}            % Use Natbib style for the refs.
\hypersetup{colorlinks=true}   % Set to false for black/white printing
\input{Definitions}            % Include your abbreviations



\usepackage{enumitem}% http://ctan.org/pkg/enumitem
\usepackage{multirow}
\usepackage{float}
\usepackage{amsmath}
\usepackage{multicol}
\usepackage{amssymb}
\usepackage[normalem]{ulem}
\useunder{\uline}{\ul}{}
\usepackage{wrapfig}


\usepackage[table,xcdraw]{xcolor}


%% ----------------------------------------------------------------
\begin{document}
\frontmatter
\title      {Heterogeneous Agent-based Model for Supermarket Competition}
\authors    {\texorpdfstring
             {\href{mailto:sc22g13@ecs.soton.ac.uk}{Stefan J. Collier}}
             {Stefan J. Collier}
            }
\addresses  {\groupname\\\deptname\\\univname}
\date       {\today}
\subject    {}
\keywords   {}
\supervisor {Dr. Maria Polukarov}
\examiner   {Professor Sheng Chen}

\maketitle
\begin{abstract}
This project aim was to model and analyse the effects of competitive pricing behaviors of grocery retailers on the British market. 

This was achieved by creating a multi-agent model, containing retailer and consumer agents. The heterogeneous crowd of retailers employs either a uniform pricing strategy or a ‘local price flexing’ strategy. The actions of these retailers are chosen by predicting the profit of each action, using a perceptron. Following on from the consideration of different economic models, a discrete model was developed so that software agents have a discrete environment to operate within. Within the model, it has been observed how supermarkets with differing behaviors affect a heterogeneous crowd of consumer agents. The model was implemented in Java with Python used to evaluate the results. 

The simulation displays good acceptance with real grocery market behavior, i.e. captures the performance of British retailers thus can be used to determine the impact of changes in their behavior on their competitors and consumers.Furthermore it can be used to provide insight into sustainability of volatile pricing strategies, providing a useful insight in volatility of British supermarket retail industry. 
\end{abstract}
\acknowledgements{
I would like to express my sincere gratitude to Dr Maria Polukarov for her guidance and support which provided me the freedom to take this research in the direction of my interest.\\
\\
I would also like to thank my family and friends for their encouragement and support. To those who quietly listened to my software complaints. To those who worked throughout the nights with me. To those who helped me write what I couldn't say. I cannot thank you enough.
}

\declaration{
I, Stefan Collier, declare that this dissertation and the work presented in it are my own and has been generated by me as the result of my own original research.\\
I confirm that:\\
1. This work was done wholly or mainly while in candidature for a degree at this University;\\
2. Where any part of this dissertation has previously been submitted for any other qualification at this University or any other institution, this has been clearly stated;\\
3. Where I have consulted the published work of others, this is always clearly attributed;\\
4. Where I have quoted from the work of others, the source is always given. With the exception of such quotations, this dissertation is entirely my own work;\\
5. I have acknowledged all main sources of help;\\
6. Where the thesis is based on work done by myself jointly with others, I have made clear exactly what was done by others and what I have contributed myself;\\
7. Either none of this work has been published before submission, or parts of this work have been published by :\\
\\
Stefan Collier\\
April 2016
}
\tableofcontents
\listoffigures
\listoftables

\mainmatter
%% ----------------------------------------------------------------
%\include{Introduction}
%\include{Conclusions}
\include{chapters/1Project/main}
\include{chapters/2Lit/main}
\include{chapters/3Design/HighLevel}
\include{chapters/3Design/InDepth}
\include{chapters/4Impl/main}

\include{chapters/5Experiments/1/main}
\include{chapters/5Experiments/2/main}
\include{chapters/5Experiments/3/main}
\include{chapters/5Experiments/4/main}

\include{chapters/6Conclusion/main}

\appendix
\include{appendix/AppendixB}
\include{appendix/D/main}
\include{appendix/AppendixC}

\backmatter
\bibliographystyle{ecs}
\bibliography{ECS}
\end{document}
%% ----------------------------------------------------------------

 %% ----------------------------------------------------------------
%% Progress.tex
%% ---------------------------------------------------------------- 
\documentclass{ecsprogress}    % Use the progress Style
\graphicspath{{../figs/}}   % Location of your graphics files
    \usepackage{natbib}            % Use Natbib style for the refs.
\hypersetup{colorlinks=true}   % Set to false for black/white printing
\input{Definitions}            % Include your abbreviations



\usepackage{enumitem}% http://ctan.org/pkg/enumitem
\usepackage{multirow}
\usepackage{float}
\usepackage{amsmath}
\usepackage{multicol}
\usepackage{amssymb}
\usepackage[normalem]{ulem}
\useunder{\uline}{\ul}{}
\usepackage{wrapfig}


\usepackage[table,xcdraw]{xcolor}


%% ----------------------------------------------------------------
\begin{document}
\frontmatter
\title      {Heterogeneous Agent-based Model for Supermarket Competition}
\authors    {\texorpdfstring
             {\href{mailto:sc22g13@ecs.soton.ac.uk}{Stefan J. Collier}}
             {Stefan J. Collier}
            }
\addresses  {\groupname\\\deptname\\\univname}
\date       {\today}
\subject    {}
\keywords   {}
\supervisor {Dr. Maria Polukarov}
\examiner   {Professor Sheng Chen}

\maketitle
\begin{abstract}
This project aim was to model and analyse the effects of competitive pricing behaviors of grocery retailers on the British market. 

This was achieved by creating a multi-agent model, containing retailer and consumer agents. The heterogeneous crowd of retailers employs either a uniform pricing strategy or a ‘local price flexing’ strategy. The actions of these retailers are chosen by predicting the profit of each action, using a perceptron. Following on from the consideration of different economic models, a discrete model was developed so that software agents have a discrete environment to operate within. Within the model, it has been observed how supermarkets with differing behaviors affect a heterogeneous crowd of consumer agents. The model was implemented in Java with Python used to evaluate the results. 

The simulation displays good acceptance with real grocery market behavior, i.e. captures the performance of British retailers thus can be used to determine the impact of changes in their behavior on their competitors and consumers.Furthermore it can be used to provide insight into sustainability of volatile pricing strategies, providing a useful insight in volatility of British supermarket retail industry. 
\end{abstract}
\acknowledgements{
I would like to express my sincere gratitude to Dr Maria Polukarov for her guidance and support which provided me the freedom to take this research in the direction of my interest.\\
\\
I would also like to thank my family and friends for their encouragement and support. To those who quietly listened to my software complaints. To those who worked throughout the nights with me. To those who helped me write what I couldn't say. I cannot thank you enough.
}

\declaration{
I, Stefan Collier, declare that this dissertation and the work presented in it are my own and has been generated by me as the result of my own original research.\\
I confirm that:\\
1. This work was done wholly or mainly while in candidature for a degree at this University;\\
2. Where any part of this dissertation has previously been submitted for any other qualification at this University or any other institution, this has been clearly stated;\\
3. Where I have consulted the published work of others, this is always clearly attributed;\\
4. Where I have quoted from the work of others, the source is always given. With the exception of such quotations, this dissertation is entirely my own work;\\
5. I have acknowledged all main sources of help;\\
6. Where the thesis is based on work done by myself jointly with others, I have made clear exactly what was done by others and what I have contributed myself;\\
7. Either none of this work has been published before submission, or parts of this work have been published by :\\
\\
Stefan Collier\\
April 2016
}
\tableofcontents
\listoffigures
\listoftables

\mainmatter
%% ----------------------------------------------------------------
%\include{Introduction}
%\include{Conclusions}
\include{chapters/1Project/main}
\include{chapters/2Lit/main}
\include{chapters/3Design/HighLevel}
\include{chapters/3Design/InDepth}
\include{chapters/4Impl/main}

\include{chapters/5Experiments/1/main}
\include{chapters/5Experiments/2/main}
\include{chapters/5Experiments/3/main}
\include{chapters/5Experiments/4/main}

\include{chapters/6Conclusion/main}

\appendix
\include{appendix/AppendixB}
\include{appendix/D/main}
\include{appendix/AppendixC}

\backmatter
\bibliographystyle{ecs}
\bibliography{ECS}
\end{document}
%% ----------------------------------------------------------------


 %% ----------------------------------------------------------------
%% Progress.tex
%% ---------------------------------------------------------------- 
\documentclass{ecsprogress}    % Use the progress Style
\graphicspath{{../figs/}}   % Location of your graphics files
    \usepackage{natbib}            % Use Natbib style for the refs.
\hypersetup{colorlinks=true}   % Set to false for black/white printing
\input{Definitions}            % Include your abbreviations



\usepackage{enumitem}% http://ctan.org/pkg/enumitem
\usepackage{multirow}
\usepackage{float}
\usepackage{amsmath}
\usepackage{multicol}
\usepackage{amssymb}
\usepackage[normalem]{ulem}
\useunder{\uline}{\ul}{}
\usepackage{wrapfig}


\usepackage[table,xcdraw]{xcolor}


%% ----------------------------------------------------------------
\begin{document}
\frontmatter
\title      {Heterogeneous Agent-based Model for Supermarket Competition}
\authors    {\texorpdfstring
             {\href{mailto:sc22g13@ecs.soton.ac.uk}{Stefan J. Collier}}
             {Stefan J. Collier}
            }
\addresses  {\groupname\\\deptname\\\univname}
\date       {\today}
\subject    {}
\keywords   {}
\supervisor {Dr. Maria Polukarov}
\examiner   {Professor Sheng Chen}

\maketitle
\begin{abstract}
This project aim was to model and analyse the effects of competitive pricing behaviors of grocery retailers on the British market. 

This was achieved by creating a multi-agent model, containing retailer and consumer agents. The heterogeneous crowd of retailers employs either a uniform pricing strategy or a ‘local price flexing’ strategy. The actions of these retailers are chosen by predicting the profit of each action, using a perceptron. Following on from the consideration of different economic models, a discrete model was developed so that software agents have a discrete environment to operate within. Within the model, it has been observed how supermarkets with differing behaviors affect a heterogeneous crowd of consumer agents. The model was implemented in Java with Python used to evaluate the results. 

The simulation displays good acceptance with real grocery market behavior, i.e. captures the performance of British retailers thus can be used to determine the impact of changes in their behavior on their competitors and consumers.Furthermore it can be used to provide insight into sustainability of volatile pricing strategies, providing a useful insight in volatility of British supermarket retail industry. 
\end{abstract}
\acknowledgements{
I would like to express my sincere gratitude to Dr Maria Polukarov for her guidance and support which provided me the freedom to take this research in the direction of my interest.\\
\\
I would also like to thank my family and friends for their encouragement and support. To those who quietly listened to my software complaints. To those who worked throughout the nights with me. To those who helped me write what I couldn't say. I cannot thank you enough.
}

\declaration{
I, Stefan Collier, declare that this dissertation and the work presented in it are my own and has been generated by me as the result of my own original research.\\
I confirm that:\\
1. This work was done wholly or mainly while in candidature for a degree at this University;\\
2. Where any part of this dissertation has previously been submitted for any other qualification at this University or any other institution, this has been clearly stated;\\
3. Where I have consulted the published work of others, this is always clearly attributed;\\
4. Where I have quoted from the work of others, the source is always given. With the exception of such quotations, this dissertation is entirely my own work;\\
5. I have acknowledged all main sources of help;\\
6. Where the thesis is based on work done by myself jointly with others, I have made clear exactly what was done by others and what I have contributed myself;\\
7. Either none of this work has been published before submission, or parts of this work have been published by :\\
\\
Stefan Collier\\
April 2016
}
\tableofcontents
\listoffigures
\listoftables

\mainmatter
%% ----------------------------------------------------------------
%\include{Introduction}
%\include{Conclusions}
\include{chapters/1Project/main}
\include{chapters/2Lit/main}
\include{chapters/3Design/HighLevel}
\include{chapters/3Design/InDepth}
\include{chapters/4Impl/main}

\include{chapters/5Experiments/1/main}
\include{chapters/5Experiments/2/main}
\include{chapters/5Experiments/3/main}
\include{chapters/5Experiments/4/main}

\include{chapters/6Conclusion/main}

\appendix
\include{appendix/AppendixB}
\include{appendix/D/main}
\include{appendix/AppendixC}

\backmatter
\bibliographystyle{ecs}
\bibliography{ECS}
\end{document}
%% ----------------------------------------------------------------


\appendix
\include{appendix/AppendixB}
 %% ----------------------------------------------------------------
%% Progress.tex
%% ---------------------------------------------------------------- 
\documentclass{ecsprogress}    % Use the progress Style
\graphicspath{{../figs/}}   % Location of your graphics files
    \usepackage{natbib}            % Use Natbib style for the refs.
\hypersetup{colorlinks=true}   % Set to false for black/white printing
\input{Definitions}            % Include your abbreviations



\usepackage{enumitem}% http://ctan.org/pkg/enumitem
\usepackage{multirow}
\usepackage{float}
\usepackage{amsmath}
\usepackage{multicol}
\usepackage{amssymb}
\usepackage[normalem]{ulem}
\useunder{\uline}{\ul}{}
\usepackage{wrapfig}


\usepackage[table,xcdraw]{xcolor}


%% ----------------------------------------------------------------
\begin{document}
\frontmatter
\title      {Heterogeneous Agent-based Model for Supermarket Competition}
\authors    {\texorpdfstring
             {\href{mailto:sc22g13@ecs.soton.ac.uk}{Stefan J. Collier}}
             {Stefan J. Collier}
            }
\addresses  {\groupname\\\deptname\\\univname}
\date       {\today}
\subject    {}
\keywords   {}
\supervisor {Dr. Maria Polukarov}
\examiner   {Professor Sheng Chen}

\maketitle
\begin{abstract}
This project aim was to model and analyse the effects of competitive pricing behaviors of grocery retailers on the British market. 

This was achieved by creating a multi-agent model, containing retailer and consumer agents. The heterogeneous crowd of retailers employs either a uniform pricing strategy or a ‘local price flexing’ strategy. The actions of these retailers are chosen by predicting the profit of each action, using a perceptron. Following on from the consideration of different economic models, a discrete model was developed so that software agents have a discrete environment to operate within. Within the model, it has been observed how supermarkets with differing behaviors affect a heterogeneous crowd of consumer agents. The model was implemented in Java with Python used to evaluate the results. 

The simulation displays good acceptance with real grocery market behavior, i.e. captures the performance of British retailers thus can be used to determine the impact of changes in their behavior on their competitors and consumers.Furthermore it can be used to provide insight into sustainability of volatile pricing strategies, providing a useful insight in volatility of British supermarket retail industry. 
\end{abstract}
\acknowledgements{
I would like to express my sincere gratitude to Dr Maria Polukarov for her guidance and support which provided me the freedom to take this research in the direction of my interest.\\
\\
I would also like to thank my family and friends for their encouragement and support. To those who quietly listened to my software complaints. To those who worked throughout the nights with me. To those who helped me write what I couldn't say. I cannot thank you enough.
}

\declaration{
I, Stefan Collier, declare that this dissertation and the work presented in it are my own and has been generated by me as the result of my own original research.\\
I confirm that:\\
1. This work was done wholly or mainly while in candidature for a degree at this University;\\
2. Where any part of this dissertation has previously been submitted for any other qualification at this University or any other institution, this has been clearly stated;\\
3. Where I have consulted the published work of others, this is always clearly attributed;\\
4. Where I have quoted from the work of others, the source is always given. With the exception of such quotations, this dissertation is entirely my own work;\\
5. I have acknowledged all main sources of help;\\
6. Where the thesis is based on work done by myself jointly with others, I have made clear exactly what was done by others and what I have contributed myself;\\
7. Either none of this work has been published before submission, or parts of this work have been published by :\\
\\
Stefan Collier\\
April 2016
}
\tableofcontents
\listoffigures
\listoftables

\mainmatter
%% ----------------------------------------------------------------
%\include{Introduction}
%\include{Conclusions}
\include{chapters/1Project/main}
\include{chapters/2Lit/main}
\include{chapters/3Design/HighLevel}
\include{chapters/3Design/InDepth}
\include{chapters/4Impl/main}

\include{chapters/5Experiments/1/main}
\include{chapters/5Experiments/2/main}
\include{chapters/5Experiments/3/main}
\include{chapters/5Experiments/4/main}

\include{chapters/6Conclusion/main}

\appendix
\include{appendix/AppendixB}
\include{appendix/D/main}
\include{appendix/AppendixC}

\backmatter
\bibliographystyle{ecs}
\bibliography{ECS}
\end{document}
%% ----------------------------------------------------------------

\include{appendix/AppendixC}

\backmatter
\bibliographystyle{ecs}
\bibliography{ECS}
\end{document}
%% ----------------------------------------------------------------


 %% ----------------------------------------------------------------
%% Progress.tex
%% ---------------------------------------------------------------- 
\documentclass{ecsprogress}    % Use the progress Style
\graphicspath{{../figs/}}   % Location of your graphics files
    \usepackage{natbib}            % Use Natbib style for the refs.
\hypersetup{colorlinks=true}   % Set to false for black/white printing
\input{Definitions}            % Include your abbreviations



\usepackage{enumitem}% http://ctan.org/pkg/enumitem
\usepackage{multirow}
\usepackage{float}
\usepackage{amsmath}
\usepackage{multicol}
\usepackage{amssymb}
\usepackage[normalem]{ulem}
\useunder{\uline}{\ul}{}
\usepackage{wrapfig}


\usepackage[table,xcdraw]{xcolor}


%% ----------------------------------------------------------------
\begin{document}
\frontmatter
\title      {Heterogeneous Agent-based Model for Supermarket Competition}
\authors    {\texorpdfstring
             {\href{mailto:sc22g13@ecs.soton.ac.uk}{Stefan J. Collier}}
             {Stefan J. Collier}
            }
\addresses  {\groupname\\\deptname\\\univname}
\date       {\today}
\subject    {}
\keywords   {}
\supervisor {Dr. Maria Polukarov}
\examiner   {Professor Sheng Chen}

\maketitle
\begin{abstract}
This project aim was to model and analyse the effects of competitive pricing behaviors of grocery retailers on the British market. 

This was achieved by creating a multi-agent model, containing retailer and consumer agents. The heterogeneous crowd of retailers employs either a uniform pricing strategy or a ‘local price flexing’ strategy. The actions of these retailers are chosen by predicting the profit of each action, using a perceptron. Following on from the consideration of different economic models, a discrete model was developed so that software agents have a discrete environment to operate within. Within the model, it has been observed how supermarkets with differing behaviors affect a heterogeneous crowd of consumer agents. The model was implemented in Java with Python used to evaluate the results. 

The simulation displays good acceptance with real grocery market behavior, i.e. captures the performance of British retailers thus can be used to determine the impact of changes in their behavior on their competitors and consumers.Furthermore it can be used to provide insight into sustainability of volatile pricing strategies, providing a useful insight in volatility of British supermarket retail industry. 
\end{abstract}
\acknowledgements{
I would like to express my sincere gratitude to Dr Maria Polukarov for her guidance and support which provided me the freedom to take this research in the direction of my interest.\\
\\
I would also like to thank my family and friends for their encouragement and support. To those who quietly listened to my software complaints. To those who worked throughout the nights with me. To those who helped me write what I couldn't say. I cannot thank you enough.
}

\declaration{
I, Stefan Collier, declare that this dissertation and the work presented in it are my own and has been generated by me as the result of my own original research.\\
I confirm that:\\
1. This work was done wholly or mainly while in candidature for a degree at this University;\\
2. Where any part of this dissertation has previously been submitted for any other qualification at this University or any other institution, this has been clearly stated;\\
3. Where I have consulted the published work of others, this is always clearly attributed;\\
4. Where I have quoted from the work of others, the source is always given. With the exception of such quotations, this dissertation is entirely my own work;\\
5. I have acknowledged all main sources of help;\\
6. Where the thesis is based on work done by myself jointly with others, I have made clear exactly what was done by others and what I have contributed myself;\\
7. Either none of this work has been published before submission, or parts of this work have been published by :\\
\\
Stefan Collier\\
April 2016
}
\tableofcontents
\listoffigures
\listoftables

\mainmatter
%% ----------------------------------------------------------------
%\include{Introduction}
%\include{Conclusions}
 %% ----------------------------------------------------------------
%% Progress.tex
%% ---------------------------------------------------------------- 
\documentclass{ecsprogress}    % Use the progress Style
\graphicspath{{../figs/}}   % Location of your graphics files
    \usepackage{natbib}            % Use Natbib style for the refs.
\hypersetup{colorlinks=true}   % Set to false for black/white printing
\input{Definitions}            % Include your abbreviations



\usepackage{enumitem}% http://ctan.org/pkg/enumitem
\usepackage{multirow}
\usepackage{float}
\usepackage{amsmath}
\usepackage{multicol}
\usepackage{amssymb}
\usepackage[normalem]{ulem}
\useunder{\uline}{\ul}{}
\usepackage{wrapfig}


\usepackage[table,xcdraw]{xcolor}


%% ----------------------------------------------------------------
\begin{document}
\frontmatter
\title      {Heterogeneous Agent-based Model for Supermarket Competition}
\authors    {\texorpdfstring
             {\href{mailto:sc22g13@ecs.soton.ac.uk}{Stefan J. Collier}}
             {Stefan J. Collier}
            }
\addresses  {\groupname\\\deptname\\\univname}
\date       {\today}
\subject    {}
\keywords   {}
\supervisor {Dr. Maria Polukarov}
\examiner   {Professor Sheng Chen}

\maketitle
\begin{abstract}
This project aim was to model and analyse the effects of competitive pricing behaviors of grocery retailers on the British market. 

This was achieved by creating a multi-agent model, containing retailer and consumer agents. The heterogeneous crowd of retailers employs either a uniform pricing strategy or a ‘local price flexing’ strategy. The actions of these retailers are chosen by predicting the profit of each action, using a perceptron. Following on from the consideration of different economic models, a discrete model was developed so that software agents have a discrete environment to operate within. Within the model, it has been observed how supermarkets with differing behaviors affect a heterogeneous crowd of consumer agents. The model was implemented in Java with Python used to evaluate the results. 

The simulation displays good acceptance with real grocery market behavior, i.e. captures the performance of British retailers thus can be used to determine the impact of changes in their behavior on their competitors and consumers.Furthermore it can be used to provide insight into sustainability of volatile pricing strategies, providing a useful insight in volatility of British supermarket retail industry. 
\end{abstract}
\acknowledgements{
I would like to express my sincere gratitude to Dr Maria Polukarov for her guidance and support which provided me the freedom to take this research in the direction of my interest.\\
\\
I would also like to thank my family and friends for their encouragement and support. To those who quietly listened to my software complaints. To those who worked throughout the nights with me. To those who helped me write what I couldn't say. I cannot thank you enough.
}

\declaration{
I, Stefan Collier, declare that this dissertation and the work presented in it are my own and has been generated by me as the result of my own original research.\\
I confirm that:\\
1. This work was done wholly or mainly while in candidature for a degree at this University;\\
2. Where any part of this dissertation has previously been submitted for any other qualification at this University or any other institution, this has been clearly stated;\\
3. Where I have consulted the published work of others, this is always clearly attributed;\\
4. Where I have quoted from the work of others, the source is always given. With the exception of such quotations, this dissertation is entirely my own work;\\
5. I have acknowledged all main sources of help;\\
6. Where the thesis is based on work done by myself jointly with others, I have made clear exactly what was done by others and what I have contributed myself;\\
7. Either none of this work has been published before submission, or parts of this work have been published by :\\
\\
Stefan Collier\\
April 2016
}
\tableofcontents
\listoffigures
\listoftables

\mainmatter
%% ----------------------------------------------------------------
%\include{Introduction}
%\include{Conclusions}
\include{chapters/1Project/main}
\include{chapters/2Lit/main}
\include{chapters/3Design/HighLevel}
\include{chapters/3Design/InDepth}
\include{chapters/4Impl/main}

\include{chapters/5Experiments/1/main}
\include{chapters/5Experiments/2/main}
\include{chapters/5Experiments/3/main}
\include{chapters/5Experiments/4/main}

\include{chapters/6Conclusion/main}

\appendix
\include{appendix/AppendixB}
\include{appendix/D/main}
\include{appendix/AppendixC}

\backmatter
\bibliographystyle{ecs}
\bibliography{ECS}
\end{document}
%% ----------------------------------------------------------------

 %% ----------------------------------------------------------------
%% Progress.tex
%% ---------------------------------------------------------------- 
\documentclass{ecsprogress}    % Use the progress Style
\graphicspath{{../figs/}}   % Location of your graphics files
    \usepackage{natbib}            % Use Natbib style for the refs.
\hypersetup{colorlinks=true}   % Set to false for black/white printing
\input{Definitions}            % Include your abbreviations



\usepackage{enumitem}% http://ctan.org/pkg/enumitem
\usepackage{multirow}
\usepackage{float}
\usepackage{amsmath}
\usepackage{multicol}
\usepackage{amssymb}
\usepackage[normalem]{ulem}
\useunder{\uline}{\ul}{}
\usepackage{wrapfig}


\usepackage[table,xcdraw]{xcolor}


%% ----------------------------------------------------------------
\begin{document}
\frontmatter
\title      {Heterogeneous Agent-based Model for Supermarket Competition}
\authors    {\texorpdfstring
             {\href{mailto:sc22g13@ecs.soton.ac.uk}{Stefan J. Collier}}
             {Stefan J. Collier}
            }
\addresses  {\groupname\\\deptname\\\univname}
\date       {\today}
\subject    {}
\keywords   {}
\supervisor {Dr. Maria Polukarov}
\examiner   {Professor Sheng Chen}

\maketitle
\begin{abstract}
This project aim was to model and analyse the effects of competitive pricing behaviors of grocery retailers on the British market. 

This was achieved by creating a multi-agent model, containing retailer and consumer agents. The heterogeneous crowd of retailers employs either a uniform pricing strategy or a ‘local price flexing’ strategy. The actions of these retailers are chosen by predicting the profit of each action, using a perceptron. Following on from the consideration of different economic models, a discrete model was developed so that software agents have a discrete environment to operate within. Within the model, it has been observed how supermarkets with differing behaviors affect a heterogeneous crowd of consumer agents. The model was implemented in Java with Python used to evaluate the results. 

The simulation displays good acceptance with real grocery market behavior, i.e. captures the performance of British retailers thus can be used to determine the impact of changes in their behavior on their competitors and consumers.Furthermore it can be used to provide insight into sustainability of volatile pricing strategies, providing a useful insight in volatility of British supermarket retail industry. 
\end{abstract}
\acknowledgements{
I would like to express my sincere gratitude to Dr Maria Polukarov for her guidance and support which provided me the freedom to take this research in the direction of my interest.\\
\\
I would also like to thank my family and friends for their encouragement and support. To those who quietly listened to my software complaints. To those who worked throughout the nights with me. To those who helped me write what I couldn't say. I cannot thank you enough.
}

\declaration{
I, Stefan Collier, declare that this dissertation and the work presented in it are my own and has been generated by me as the result of my own original research.\\
I confirm that:\\
1. This work was done wholly or mainly while in candidature for a degree at this University;\\
2. Where any part of this dissertation has previously been submitted for any other qualification at this University or any other institution, this has been clearly stated;\\
3. Where I have consulted the published work of others, this is always clearly attributed;\\
4. Where I have quoted from the work of others, the source is always given. With the exception of such quotations, this dissertation is entirely my own work;\\
5. I have acknowledged all main sources of help;\\
6. Where the thesis is based on work done by myself jointly with others, I have made clear exactly what was done by others and what I have contributed myself;\\
7. Either none of this work has been published before submission, or parts of this work have been published by :\\
\\
Stefan Collier\\
April 2016
}
\tableofcontents
\listoffigures
\listoftables

\mainmatter
%% ----------------------------------------------------------------
%\include{Introduction}
%\include{Conclusions}
\include{chapters/1Project/main}
\include{chapters/2Lit/main}
\include{chapters/3Design/HighLevel}
\include{chapters/3Design/InDepth}
\include{chapters/4Impl/main}

\include{chapters/5Experiments/1/main}
\include{chapters/5Experiments/2/main}
\include{chapters/5Experiments/3/main}
\include{chapters/5Experiments/4/main}

\include{chapters/6Conclusion/main}

\appendix
\include{appendix/AppendixB}
\include{appendix/D/main}
\include{appendix/AppendixC}

\backmatter
\bibliographystyle{ecs}
\bibliography{ECS}
\end{document}
%% ----------------------------------------------------------------

\include{chapters/3Design/HighLevel}
\include{chapters/3Design/InDepth}
 %% ----------------------------------------------------------------
%% Progress.tex
%% ---------------------------------------------------------------- 
\documentclass{ecsprogress}    % Use the progress Style
\graphicspath{{../figs/}}   % Location of your graphics files
    \usepackage{natbib}            % Use Natbib style for the refs.
\hypersetup{colorlinks=true}   % Set to false for black/white printing
\input{Definitions}            % Include your abbreviations



\usepackage{enumitem}% http://ctan.org/pkg/enumitem
\usepackage{multirow}
\usepackage{float}
\usepackage{amsmath}
\usepackage{multicol}
\usepackage{amssymb}
\usepackage[normalem]{ulem}
\useunder{\uline}{\ul}{}
\usepackage{wrapfig}


\usepackage[table,xcdraw]{xcolor}


%% ----------------------------------------------------------------
\begin{document}
\frontmatter
\title      {Heterogeneous Agent-based Model for Supermarket Competition}
\authors    {\texorpdfstring
             {\href{mailto:sc22g13@ecs.soton.ac.uk}{Stefan J. Collier}}
             {Stefan J. Collier}
            }
\addresses  {\groupname\\\deptname\\\univname}
\date       {\today}
\subject    {}
\keywords   {}
\supervisor {Dr. Maria Polukarov}
\examiner   {Professor Sheng Chen}

\maketitle
\begin{abstract}
This project aim was to model and analyse the effects of competitive pricing behaviors of grocery retailers on the British market. 

This was achieved by creating a multi-agent model, containing retailer and consumer agents. The heterogeneous crowd of retailers employs either a uniform pricing strategy or a ‘local price flexing’ strategy. The actions of these retailers are chosen by predicting the profit of each action, using a perceptron. Following on from the consideration of different economic models, a discrete model was developed so that software agents have a discrete environment to operate within. Within the model, it has been observed how supermarkets with differing behaviors affect a heterogeneous crowd of consumer agents. The model was implemented in Java with Python used to evaluate the results. 

The simulation displays good acceptance with real grocery market behavior, i.e. captures the performance of British retailers thus can be used to determine the impact of changes in their behavior on their competitors and consumers.Furthermore it can be used to provide insight into sustainability of volatile pricing strategies, providing a useful insight in volatility of British supermarket retail industry. 
\end{abstract}
\acknowledgements{
I would like to express my sincere gratitude to Dr Maria Polukarov for her guidance and support which provided me the freedom to take this research in the direction of my interest.\\
\\
I would also like to thank my family and friends for their encouragement and support. To those who quietly listened to my software complaints. To those who worked throughout the nights with me. To those who helped me write what I couldn't say. I cannot thank you enough.
}

\declaration{
I, Stefan Collier, declare that this dissertation and the work presented in it are my own and has been generated by me as the result of my own original research.\\
I confirm that:\\
1. This work was done wholly or mainly while in candidature for a degree at this University;\\
2. Where any part of this dissertation has previously been submitted for any other qualification at this University or any other institution, this has been clearly stated;\\
3. Where I have consulted the published work of others, this is always clearly attributed;\\
4. Where I have quoted from the work of others, the source is always given. With the exception of such quotations, this dissertation is entirely my own work;\\
5. I have acknowledged all main sources of help;\\
6. Where the thesis is based on work done by myself jointly with others, I have made clear exactly what was done by others and what I have contributed myself;\\
7. Either none of this work has been published before submission, or parts of this work have been published by :\\
\\
Stefan Collier\\
April 2016
}
\tableofcontents
\listoffigures
\listoftables

\mainmatter
%% ----------------------------------------------------------------
%\include{Introduction}
%\include{Conclusions}
\include{chapters/1Project/main}
\include{chapters/2Lit/main}
\include{chapters/3Design/HighLevel}
\include{chapters/3Design/InDepth}
\include{chapters/4Impl/main}

\include{chapters/5Experiments/1/main}
\include{chapters/5Experiments/2/main}
\include{chapters/5Experiments/3/main}
\include{chapters/5Experiments/4/main}

\include{chapters/6Conclusion/main}

\appendix
\include{appendix/AppendixB}
\include{appendix/D/main}
\include{appendix/AppendixC}

\backmatter
\bibliographystyle{ecs}
\bibliography{ECS}
\end{document}
%% ----------------------------------------------------------------


 %% ----------------------------------------------------------------
%% Progress.tex
%% ---------------------------------------------------------------- 
\documentclass{ecsprogress}    % Use the progress Style
\graphicspath{{../figs/}}   % Location of your graphics files
    \usepackage{natbib}            % Use Natbib style for the refs.
\hypersetup{colorlinks=true}   % Set to false for black/white printing
\input{Definitions}            % Include your abbreviations



\usepackage{enumitem}% http://ctan.org/pkg/enumitem
\usepackage{multirow}
\usepackage{float}
\usepackage{amsmath}
\usepackage{multicol}
\usepackage{amssymb}
\usepackage[normalem]{ulem}
\useunder{\uline}{\ul}{}
\usepackage{wrapfig}


\usepackage[table,xcdraw]{xcolor}


%% ----------------------------------------------------------------
\begin{document}
\frontmatter
\title      {Heterogeneous Agent-based Model for Supermarket Competition}
\authors    {\texorpdfstring
             {\href{mailto:sc22g13@ecs.soton.ac.uk}{Stefan J. Collier}}
             {Stefan J. Collier}
            }
\addresses  {\groupname\\\deptname\\\univname}
\date       {\today}
\subject    {}
\keywords   {}
\supervisor {Dr. Maria Polukarov}
\examiner   {Professor Sheng Chen}

\maketitle
\begin{abstract}
This project aim was to model and analyse the effects of competitive pricing behaviors of grocery retailers on the British market. 

This was achieved by creating a multi-agent model, containing retailer and consumer agents. The heterogeneous crowd of retailers employs either a uniform pricing strategy or a ‘local price flexing’ strategy. The actions of these retailers are chosen by predicting the profit of each action, using a perceptron. Following on from the consideration of different economic models, a discrete model was developed so that software agents have a discrete environment to operate within. Within the model, it has been observed how supermarkets with differing behaviors affect a heterogeneous crowd of consumer agents. The model was implemented in Java with Python used to evaluate the results. 

The simulation displays good acceptance with real grocery market behavior, i.e. captures the performance of British retailers thus can be used to determine the impact of changes in their behavior on their competitors and consumers.Furthermore it can be used to provide insight into sustainability of volatile pricing strategies, providing a useful insight in volatility of British supermarket retail industry. 
\end{abstract}
\acknowledgements{
I would like to express my sincere gratitude to Dr Maria Polukarov for her guidance and support which provided me the freedom to take this research in the direction of my interest.\\
\\
I would also like to thank my family and friends for their encouragement and support. To those who quietly listened to my software complaints. To those who worked throughout the nights with me. To those who helped me write what I couldn't say. I cannot thank you enough.
}

\declaration{
I, Stefan Collier, declare that this dissertation and the work presented in it are my own and has been generated by me as the result of my own original research.\\
I confirm that:\\
1. This work was done wholly or mainly while in candidature for a degree at this University;\\
2. Where any part of this dissertation has previously been submitted for any other qualification at this University or any other institution, this has been clearly stated;\\
3. Where I have consulted the published work of others, this is always clearly attributed;\\
4. Where I have quoted from the work of others, the source is always given. With the exception of such quotations, this dissertation is entirely my own work;\\
5. I have acknowledged all main sources of help;\\
6. Where the thesis is based on work done by myself jointly with others, I have made clear exactly what was done by others and what I have contributed myself;\\
7. Either none of this work has been published before submission, or parts of this work have been published by :\\
\\
Stefan Collier\\
April 2016
}
\tableofcontents
\listoffigures
\listoftables

\mainmatter
%% ----------------------------------------------------------------
%\include{Introduction}
%\include{Conclusions}
\include{chapters/1Project/main}
\include{chapters/2Lit/main}
\include{chapters/3Design/HighLevel}
\include{chapters/3Design/InDepth}
\include{chapters/4Impl/main}

\include{chapters/5Experiments/1/main}
\include{chapters/5Experiments/2/main}
\include{chapters/5Experiments/3/main}
\include{chapters/5Experiments/4/main}

\include{chapters/6Conclusion/main}

\appendix
\include{appendix/AppendixB}
\include{appendix/D/main}
\include{appendix/AppendixC}

\backmatter
\bibliographystyle{ecs}
\bibliography{ECS}
\end{document}
%% ----------------------------------------------------------------

 %% ----------------------------------------------------------------
%% Progress.tex
%% ---------------------------------------------------------------- 
\documentclass{ecsprogress}    % Use the progress Style
\graphicspath{{../figs/}}   % Location of your graphics files
    \usepackage{natbib}            % Use Natbib style for the refs.
\hypersetup{colorlinks=true}   % Set to false for black/white printing
\input{Definitions}            % Include your abbreviations



\usepackage{enumitem}% http://ctan.org/pkg/enumitem
\usepackage{multirow}
\usepackage{float}
\usepackage{amsmath}
\usepackage{multicol}
\usepackage{amssymb}
\usepackage[normalem]{ulem}
\useunder{\uline}{\ul}{}
\usepackage{wrapfig}


\usepackage[table,xcdraw]{xcolor}


%% ----------------------------------------------------------------
\begin{document}
\frontmatter
\title      {Heterogeneous Agent-based Model for Supermarket Competition}
\authors    {\texorpdfstring
             {\href{mailto:sc22g13@ecs.soton.ac.uk}{Stefan J. Collier}}
             {Stefan J. Collier}
            }
\addresses  {\groupname\\\deptname\\\univname}
\date       {\today}
\subject    {}
\keywords   {}
\supervisor {Dr. Maria Polukarov}
\examiner   {Professor Sheng Chen}

\maketitle
\begin{abstract}
This project aim was to model and analyse the effects of competitive pricing behaviors of grocery retailers on the British market. 

This was achieved by creating a multi-agent model, containing retailer and consumer agents. The heterogeneous crowd of retailers employs either a uniform pricing strategy or a ‘local price flexing’ strategy. The actions of these retailers are chosen by predicting the profit of each action, using a perceptron. Following on from the consideration of different economic models, a discrete model was developed so that software agents have a discrete environment to operate within. Within the model, it has been observed how supermarkets with differing behaviors affect a heterogeneous crowd of consumer agents. The model was implemented in Java with Python used to evaluate the results. 

The simulation displays good acceptance with real grocery market behavior, i.e. captures the performance of British retailers thus can be used to determine the impact of changes in their behavior on their competitors and consumers.Furthermore it can be used to provide insight into sustainability of volatile pricing strategies, providing a useful insight in volatility of British supermarket retail industry. 
\end{abstract}
\acknowledgements{
I would like to express my sincere gratitude to Dr Maria Polukarov for her guidance and support which provided me the freedom to take this research in the direction of my interest.\\
\\
I would also like to thank my family and friends for their encouragement and support. To those who quietly listened to my software complaints. To those who worked throughout the nights with me. To those who helped me write what I couldn't say. I cannot thank you enough.
}

\declaration{
I, Stefan Collier, declare that this dissertation and the work presented in it are my own and has been generated by me as the result of my own original research.\\
I confirm that:\\
1. This work was done wholly or mainly while in candidature for a degree at this University;\\
2. Where any part of this dissertation has previously been submitted for any other qualification at this University or any other institution, this has been clearly stated;\\
3. Where I have consulted the published work of others, this is always clearly attributed;\\
4. Where I have quoted from the work of others, the source is always given. With the exception of such quotations, this dissertation is entirely my own work;\\
5. I have acknowledged all main sources of help;\\
6. Where the thesis is based on work done by myself jointly with others, I have made clear exactly what was done by others and what I have contributed myself;\\
7. Either none of this work has been published before submission, or parts of this work have been published by :\\
\\
Stefan Collier\\
April 2016
}
\tableofcontents
\listoffigures
\listoftables

\mainmatter
%% ----------------------------------------------------------------
%\include{Introduction}
%\include{Conclusions}
\include{chapters/1Project/main}
\include{chapters/2Lit/main}
\include{chapters/3Design/HighLevel}
\include{chapters/3Design/InDepth}
\include{chapters/4Impl/main}

\include{chapters/5Experiments/1/main}
\include{chapters/5Experiments/2/main}
\include{chapters/5Experiments/3/main}
\include{chapters/5Experiments/4/main}

\include{chapters/6Conclusion/main}

\appendix
\include{appendix/AppendixB}
\include{appendix/D/main}
\include{appendix/AppendixC}

\backmatter
\bibliographystyle{ecs}
\bibliography{ECS}
\end{document}
%% ----------------------------------------------------------------

 %% ----------------------------------------------------------------
%% Progress.tex
%% ---------------------------------------------------------------- 
\documentclass{ecsprogress}    % Use the progress Style
\graphicspath{{../figs/}}   % Location of your graphics files
    \usepackage{natbib}            % Use Natbib style for the refs.
\hypersetup{colorlinks=true}   % Set to false for black/white printing
\input{Definitions}            % Include your abbreviations



\usepackage{enumitem}% http://ctan.org/pkg/enumitem
\usepackage{multirow}
\usepackage{float}
\usepackage{amsmath}
\usepackage{multicol}
\usepackage{amssymb}
\usepackage[normalem]{ulem}
\useunder{\uline}{\ul}{}
\usepackage{wrapfig}


\usepackage[table,xcdraw]{xcolor}


%% ----------------------------------------------------------------
\begin{document}
\frontmatter
\title      {Heterogeneous Agent-based Model for Supermarket Competition}
\authors    {\texorpdfstring
             {\href{mailto:sc22g13@ecs.soton.ac.uk}{Stefan J. Collier}}
             {Stefan J. Collier}
            }
\addresses  {\groupname\\\deptname\\\univname}
\date       {\today}
\subject    {}
\keywords   {}
\supervisor {Dr. Maria Polukarov}
\examiner   {Professor Sheng Chen}

\maketitle
\begin{abstract}
This project aim was to model and analyse the effects of competitive pricing behaviors of grocery retailers on the British market. 

This was achieved by creating a multi-agent model, containing retailer and consumer agents. The heterogeneous crowd of retailers employs either a uniform pricing strategy or a ‘local price flexing’ strategy. The actions of these retailers are chosen by predicting the profit of each action, using a perceptron. Following on from the consideration of different economic models, a discrete model was developed so that software agents have a discrete environment to operate within. Within the model, it has been observed how supermarkets with differing behaviors affect a heterogeneous crowd of consumer agents. The model was implemented in Java with Python used to evaluate the results. 

The simulation displays good acceptance with real grocery market behavior, i.e. captures the performance of British retailers thus can be used to determine the impact of changes in their behavior on their competitors and consumers.Furthermore it can be used to provide insight into sustainability of volatile pricing strategies, providing a useful insight in volatility of British supermarket retail industry. 
\end{abstract}
\acknowledgements{
I would like to express my sincere gratitude to Dr Maria Polukarov for her guidance and support which provided me the freedom to take this research in the direction of my interest.\\
\\
I would also like to thank my family and friends for their encouragement and support. To those who quietly listened to my software complaints. To those who worked throughout the nights with me. To those who helped me write what I couldn't say. I cannot thank you enough.
}

\declaration{
I, Stefan Collier, declare that this dissertation and the work presented in it are my own and has been generated by me as the result of my own original research.\\
I confirm that:\\
1. This work was done wholly or mainly while in candidature for a degree at this University;\\
2. Where any part of this dissertation has previously been submitted for any other qualification at this University or any other institution, this has been clearly stated;\\
3. Where I have consulted the published work of others, this is always clearly attributed;\\
4. Where I have quoted from the work of others, the source is always given. With the exception of such quotations, this dissertation is entirely my own work;\\
5. I have acknowledged all main sources of help;\\
6. Where the thesis is based on work done by myself jointly with others, I have made clear exactly what was done by others and what I have contributed myself;\\
7. Either none of this work has been published before submission, or parts of this work have been published by :\\
\\
Stefan Collier\\
April 2016
}
\tableofcontents
\listoffigures
\listoftables

\mainmatter
%% ----------------------------------------------------------------
%\include{Introduction}
%\include{Conclusions}
\include{chapters/1Project/main}
\include{chapters/2Lit/main}
\include{chapters/3Design/HighLevel}
\include{chapters/3Design/InDepth}
\include{chapters/4Impl/main}

\include{chapters/5Experiments/1/main}
\include{chapters/5Experiments/2/main}
\include{chapters/5Experiments/3/main}
\include{chapters/5Experiments/4/main}

\include{chapters/6Conclusion/main}

\appendix
\include{appendix/AppendixB}
\include{appendix/D/main}
\include{appendix/AppendixC}

\backmatter
\bibliographystyle{ecs}
\bibliography{ECS}
\end{document}
%% ----------------------------------------------------------------

 %% ----------------------------------------------------------------
%% Progress.tex
%% ---------------------------------------------------------------- 
\documentclass{ecsprogress}    % Use the progress Style
\graphicspath{{../figs/}}   % Location of your graphics files
    \usepackage{natbib}            % Use Natbib style for the refs.
\hypersetup{colorlinks=true}   % Set to false for black/white printing
\input{Definitions}            % Include your abbreviations



\usepackage{enumitem}% http://ctan.org/pkg/enumitem
\usepackage{multirow}
\usepackage{float}
\usepackage{amsmath}
\usepackage{multicol}
\usepackage{amssymb}
\usepackage[normalem]{ulem}
\useunder{\uline}{\ul}{}
\usepackage{wrapfig}


\usepackage[table,xcdraw]{xcolor}


%% ----------------------------------------------------------------
\begin{document}
\frontmatter
\title      {Heterogeneous Agent-based Model for Supermarket Competition}
\authors    {\texorpdfstring
             {\href{mailto:sc22g13@ecs.soton.ac.uk}{Stefan J. Collier}}
             {Stefan J. Collier}
            }
\addresses  {\groupname\\\deptname\\\univname}
\date       {\today}
\subject    {}
\keywords   {}
\supervisor {Dr. Maria Polukarov}
\examiner   {Professor Sheng Chen}

\maketitle
\begin{abstract}
This project aim was to model and analyse the effects of competitive pricing behaviors of grocery retailers on the British market. 

This was achieved by creating a multi-agent model, containing retailer and consumer agents. The heterogeneous crowd of retailers employs either a uniform pricing strategy or a ‘local price flexing’ strategy. The actions of these retailers are chosen by predicting the profit of each action, using a perceptron. Following on from the consideration of different economic models, a discrete model was developed so that software agents have a discrete environment to operate within. Within the model, it has been observed how supermarkets with differing behaviors affect a heterogeneous crowd of consumer agents. The model was implemented in Java with Python used to evaluate the results. 

The simulation displays good acceptance with real grocery market behavior, i.e. captures the performance of British retailers thus can be used to determine the impact of changes in their behavior on their competitors and consumers.Furthermore it can be used to provide insight into sustainability of volatile pricing strategies, providing a useful insight in volatility of British supermarket retail industry. 
\end{abstract}
\acknowledgements{
I would like to express my sincere gratitude to Dr Maria Polukarov for her guidance and support which provided me the freedom to take this research in the direction of my interest.\\
\\
I would also like to thank my family and friends for their encouragement and support. To those who quietly listened to my software complaints. To those who worked throughout the nights with me. To those who helped me write what I couldn't say. I cannot thank you enough.
}

\declaration{
I, Stefan Collier, declare that this dissertation and the work presented in it are my own and has been generated by me as the result of my own original research.\\
I confirm that:\\
1. This work was done wholly or mainly while in candidature for a degree at this University;\\
2. Where any part of this dissertation has previously been submitted for any other qualification at this University or any other institution, this has been clearly stated;\\
3. Where I have consulted the published work of others, this is always clearly attributed;\\
4. Where I have quoted from the work of others, the source is always given. With the exception of such quotations, this dissertation is entirely my own work;\\
5. I have acknowledged all main sources of help;\\
6. Where the thesis is based on work done by myself jointly with others, I have made clear exactly what was done by others and what I have contributed myself;\\
7. Either none of this work has been published before submission, or parts of this work have been published by :\\
\\
Stefan Collier\\
April 2016
}
\tableofcontents
\listoffigures
\listoftables

\mainmatter
%% ----------------------------------------------------------------
%\include{Introduction}
%\include{Conclusions}
\include{chapters/1Project/main}
\include{chapters/2Lit/main}
\include{chapters/3Design/HighLevel}
\include{chapters/3Design/InDepth}
\include{chapters/4Impl/main}

\include{chapters/5Experiments/1/main}
\include{chapters/5Experiments/2/main}
\include{chapters/5Experiments/3/main}
\include{chapters/5Experiments/4/main}

\include{chapters/6Conclusion/main}

\appendix
\include{appendix/AppendixB}
\include{appendix/D/main}
\include{appendix/AppendixC}

\backmatter
\bibliographystyle{ecs}
\bibliography{ECS}
\end{document}
%% ----------------------------------------------------------------


 %% ----------------------------------------------------------------
%% Progress.tex
%% ---------------------------------------------------------------- 
\documentclass{ecsprogress}    % Use the progress Style
\graphicspath{{../figs/}}   % Location of your graphics files
    \usepackage{natbib}            % Use Natbib style for the refs.
\hypersetup{colorlinks=true}   % Set to false for black/white printing
\input{Definitions}            % Include your abbreviations



\usepackage{enumitem}% http://ctan.org/pkg/enumitem
\usepackage{multirow}
\usepackage{float}
\usepackage{amsmath}
\usepackage{multicol}
\usepackage{amssymb}
\usepackage[normalem]{ulem}
\useunder{\uline}{\ul}{}
\usepackage{wrapfig}


\usepackage[table,xcdraw]{xcolor}


%% ----------------------------------------------------------------
\begin{document}
\frontmatter
\title      {Heterogeneous Agent-based Model for Supermarket Competition}
\authors    {\texorpdfstring
             {\href{mailto:sc22g13@ecs.soton.ac.uk}{Stefan J. Collier}}
             {Stefan J. Collier}
            }
\addresses  {\groupname\\\deptname\\\univname}
\date       {\today}
\subject    {}
\keywords   {}
\supervisor {Dr. Maria Polukarov}
\examiner   {Professor Sheng Chen}

\maketitle
\begin{abstract}
This project aim was to model and analyse the effects of competitive pricing behaviors of grocery retailers on the British market. 

This was achieved by creating a multi-agent model, containing retailer and consumer agents. The heterogeneous crowd of retailers employs either a uniform pricing strategy or a ‘local price flexing’ strategy. The actions of these retailers are chosen by predicting the profit of each action, using a perceptron. Following on from the consideration of different economic models, a discrete model was developed so that software agents have a discrete environment to operate within. Within the model, it has been observed how supermarkets with differing behaviors affect a heterogeneous crowd of consumer agents. The model was implemented in Java with Python used to evaluate the results. 

The simulation displays good acceptance with real grocery market behavior, i.e. captures the performance of British retailers thus can be used to determine the impact of changes in their behavior on their competitors and consumers.Furthermore it can be used to provide insight into sustainability of volatile pricing strategies, providing a useful insight in volatility of British supermarket retail industry. 
\end{abstract}
\acknowledgements{
I would like to express my sincere gratitude to Dr Maria Polukarov for her guidance and support which provided me the freedom to take this research in the direction of my interest.\\
\\
I would also like to thank my family and friends for their encouragement and support. To those who quietly listened to my software complaints. To those who worked throughout the nights with me. To those who helped me write what I couldn't say. I cannot thank you enough.
}

\declaration{
I, Stefan Collier, declare that this dissertation and the work presented in it are my own and has been generated by me as the result of my own original research.\\
I confirm that:\\
1. This work was done wholly or mainly while in candidature for a degree at this University;\\
2. Where any part of this dissertation has previously been submitted for any other qualification at this University or any other institution, this has been clearly stated;\\
3. Where I have consulted the published work of others, this is always clearly attributed;\\
4. Where I have quoted from the work of others, the source is always given. With the exception of such quotations, this dissertation is entirely my own work;\\
5. I have acknowledged all main sources of help;\\
6. Where the thesis is based on work done by myself jointly with others, I have made clear exactly what was done by others and what I have contributed myself;\\
7. Either none of this work has been published before submission, or parts of this work have been published by :\\
\\
Stefan Collier\\
April 2016
}
\tableofcontents
\listoffigures
\listoftables

\mainmatter
%% ----------------------------------------------------------------
%\include{Introduction}
%\include{Conclusions}
\include{chapters/1Project/main}
\include{chapters/2Lit/main}
\include{chapters/3Design/HighLevel}
\include{chapters/3Design/InDepth}
\include{chapters/4Impl/main}

\include{chapters/5Experiments/1/main}
\include{chapters/5Experiments/2/main}
\include{chapters/5Experiments/3/main}
\include{chapters/5Experiments/4/main}

\include{chapters/6Conclusion/main}

\appendix
\include{appendix/AppendixB}
\include{appendix/D/main}
\include{appendix/AppendixC}

\backmatter
\bibliographystyle{ecs}
\bibliography{ECS}
\end{document}
%% ----------------------------------------------------------------


\appendix
\include{appendix/AppendixB}
 %% ----------------------------------------------------------------
%% Progress.tex
%% ---------------------------------------------------------------- 
\documentclass{ecsprogress}    % Use the progress Style
\graphicspath{{../figs/}}   % Location of your graphics files
    \usepackage{natbib}            % Use Natbib style for the refs.
\hypersetup{colorlinks=true}   % Set to false for black/white printing
\input{Definitions}            % Include your abbreviations



\usepackage{enumitem}% http://ctan.org/pkg/enumitem
\usepackage{multirow}
\usepackage{float}
\usepackage{amsmath}
\usepackage{multicol}
\usepackage{amssymb}
\usepackage[normalem]{ulem}
\useunder{\uline}{\ul}{}
\usepackage{wrapfig}


\usepackage[table,xcdraw]{xcolor}


%% ----------------------------------------------------------------
\begin{document}
\frontmatter
\title      {Heterogeneous Agent-based Model for Supermarket Competition}
\authors    {\texorpdfstring
             {\href{mailto:sc22g13@ecs.soton.ac.uk}{Stefan J. Collier}}
             {Stefan J. Collier}
            }
\addresses  {\groupname\\\deptname\\\univname}
\date       {\today}
\subject    {}
\keywords   {}
\supervisor {Dr. Maria Polukarov}
\examiner   {Professor Sheng Chen}

\maketitle
\begin{abstract}
This project aim was to model and analyse the effects of competitive pricing behaviors of grocery retailers on the British market. 

This was achieved by creating a multi-agent model, containing retailer and consumer agents. The heterogeneous crowd of retailers employs either a uniform pricing strategy or a ‘local price flexing’ strategy. The actions of these retailers are chosen by predicting the profit of each action, using a perceptron. Following on from the consideration of different economic models, a discrete model was developed so that software agents have a discrete environment to operate within. Within the model, it has been observed how supermarkets with differing behaviors affect a heterogeneous crowd of consumer agents. The model was implemented in Java with Python used to evaluate the results. 

The simulation displays good acceptance with real grocery market behavior, i.e. captures the performance of British retailers thus can be used to determine the impact of changes in their behavior on their competitors and consumers.Furthermore it can be used to provide insight into sustainability of volatile pricing strategies, providing a useful insight in volatility of British supermarket retail industry. 
\end{abstract}
\acknowledgements{
I would like to express my sincere gratitude to Dr Maria Polukarov for her guidance and support which provided me the freedom to take this research in the direction of my interest.\\
\\
I would also like to thank my family and friends for their encouragement and support. To those who quietly listened to my software complaints. To those who worked throughout the nights with me. To those who helped me write what I couldn't say. I cannot thank you enough.
}

\declaration{
I, Stefan Collier, declare that this dissertation and the work presented in it are my own and has been generated by me as the result of my own original research.\\
I confirm that:\\
1. This work was done wholly or mainly while in candidature for a degree at this University;\\
2. Where any part of this dissertation has previously been submitted for any other qualification at this University or any other institution, this has been clearly stated;\\
3. Where I have consulted the published work of others, this is always clearly attributed;\\
4. Where I have quoted from the work of others, the source is always given. With the exception of such quotations, this dissertation is entirely my own work;\\
5. I have acknowledged all main sources of help;\\
6. Where the thesis is based on work done by myself jointly with others, I have made clear exactly what was done by others and what I have contributed myself;\\
7. Either none of this work has been published before submission, or parts of this work have been published by :\\
\\
Stefan Collier\\
April 2016
}
\tableofcontents
\listoffigures
\listoftables

\mainmatter
%% ----------------------------------------------------------------
%\include{Introduction}
%\include{Conclusions}
\include{chapters/1Project/main}
\include{chapters/2Lit/main}
\include{chapters/3Design/HighLevel}
\include{chapters/3Design/InDepth}
\include{chapters/4Impl/main}

\include{chapters/5Experiments/1/main}
\include{chapters/5Experiments/2/main}
\include{chapters/5Experiments/3/main}
\include{chapters/5Experiments/4/main}

\include{chapters/6Conclusion/main}

\appendix
\include{appendix/AppendixB}
\include{appendix/D/main}
\include{appendix/AppendixC}

\backmatter
\bibliographystyle{ecs}
\bibliography{ECS}
\end{document}
%% ----------------------------------------------------------------

\include{appendix/AppendixC}

\backmatter
\bibliographystyle{ecs}
\bibliography{ECS}
\end{document}
%% ----------------------------------------------------------------


\appendix
\include{appendix/AppendixB}
 %% ----------------------------------------------------------------
%% Progress.tex
%% ---------------------------------------------------------------- 
\documentclass{ecsprogress}    % Use the progress Style
\graphicspath{{../figs/}}   % Location of your graphics files
    \usepackage{natbib}            % Use Natbib style for the refs.
\hypersetup{colorlinks=true}   % Set to false for black/white printing
\input{Definitions}            % Include your abbreviations



\usepackage{enumitem}% http://ctan.org/pkg/enumitem
\usepackage{multirow}
\usepackage{float}
\usepackage{amsmath}
\usepackage{multicol}
\usepackage{amssymb}
\usepackage[normalem]{ulem}
\useunder{\uline}{\ul}{}
\usepackage{wrapfig}


\usepackage[table,xcdraw]{xcolor}


%% ----------------------------------------------------------------
\begin{document}
\frontmatter
\title      {Heterogeneous Agent-based Model for Supermarket Competition}
\authors    {\texorpdfstring
             {\href{mailto:sc22g13@ecs.soton.ac.uk}{Stefan J. Collier}}
             {Stefan J. Collier}
            }
\addresses  {\groupname\\\deptname\\\univname}
\date       {\today}
\subject    {}
\keywords   {}
\supervisor {Dr. Maria Polukarov}
\examiner   {Professor Sheng Chen}

\maketitle
\begin{abstract}
This project aim was to model and analyse the effects of competitive pricing behaviors of grocery retailers on the British market. 

This was achieved by creating a multi-agent model, containing retailer and consumer agents. The heterogeneous crowd of retailers employs either a uniform pricing strategy or a ‘local price flexing’ strategy. The actions of these retailers are chosen by predicting the profit of each action, using a perceptron. Following on from the consideration of different economic models, a discrete model was developed so that software agents have a discrete environment to operate within. Within the model, it has been observed how supermarkets with differing behaviors affect a heterogeneous crowd of consumer agents. The model was implemented in Java with Python used to evaluate the results. 

The simulation displays good acceptance with real grocery market behavior, i.e. captures the performance of British retailers thus can be used to determine the impact of changes in their behavior on their competitors and consumers.Furthermore it can be used to provide insight into sustainability of volatile pricing strategies, providing a useful insight in volatility of British supermarket retail industry. 
\end{abstract}
\acknowledgements{
I would like to express my sincere gratitude to Dr Maria Polukarov for her guidance and support which provided me the freedom to take this research in the direction of my interest.\\
\\
I would also like to thank my family and friends for their encouragement and support. To those who quietly listened to my software complaints. To those who worked throughout the nights with me. To those who helped me write what I couldn't say. I cannot thank you enough.
}

\declaration{
I, Stefan Collier, declare that this dissertation and the work presented in it are my own and has been generated by me as the result of my own original research.\\
I confirm that:\\
1. This work was done wholly or mainly while in candidature for a degree at this University;\\
2. Where any part of this dissertation has previously been submitted for any other qualification at this University or any other institution, this has been clearly stated;\\
3. Where I have consulted the published work of others, this is always clearly attributed;\\
4. Where I have quoted from the work of others, the source is always given. With the exception of such quotations, this dissertation is entirely my own work;\\
5. I have acknowledged all main sources of help;\\
6. Where the thesis is based on work done by myself jointly with others, I have made clear exactly what was done by others and what I have contributed myself;\\
7. Either none of this work has been published before submission, or parts of this work have been published by :\\
\\
Stefan Collier\\
April 2016
}
\tableofcontents
\listoffigures
\listoftables

\mainmatter
%% ----------------------------------------------------------------
%\include{Introduction}
%\include{Conclusions}
 %% ----------------------------------------------------------------
%% Progress.tex
%% ---------------------------------------------------------------- 
\documentclass{ecsprogress}    % Use the progress Style
\graphicspath{{../figs/}}   % Location of your graphics files
    \usepackage{natbib}            % Use Natbib style for the refs.
\hypersetup{colorlinks=true}   % Set to false for black/white printing
\input{Definitions}            % Include your abbreviations



\usepackage{enumitem}% http://ctan.org/pkg/enumitem
\usepackage{multirow}
\usepackage{float}
\usepackage{amsmath}
\usepackage{multicol}
\usepackage{amssymb}
\usepackage[normalem]{ulem}
\useunder{\uline}{\ul}{}
\usepackage{wrapfig}


\usepackage[table,xcdraw]{xcolor}


%% ----------------------------------------------------------------
\begin{document}
\frontmatter
\title      {Heterogeneous Agent-based Model for Supermarket Competition}
\authors    {\texorpdfstring
             {\href{mailto:sc22g13@ecs.soton.ac.uk}{Stefan J. Collier}}
             {Stefan J. Collier}
            }
\addresses  {\groupname\\\deptname\\\univname}
\date       {\today}
\subject    {}
\keywords   {}
\supervisor {Dr. Maria Polukarov}
\examiner   {Professor Sheng Chen}

\maketitle
\begin{abstract}
This project aim was to model and analyse the effects of competitive pricing behaviors of grocery retailers on the British market. 

This was achieved by creating a multi-agent model, containing retailer and consumer agents. The heterogeneous crowd of retailers employs either a uniform pricing strategy or a ‘local price flexing’ strategy. The actions of these retailers are chosen by predicting the profit of each action, using a perceptron. Following on from the consideration of different economic models, a discrete model was developed so that software agents have a discrete environment to operate within. Within the model, it has been observed how supermarkets with differing behaviors affect a heterogeneous crowd of consumer agents. The model was implemented in Java with Python used to evaluate the results. 

The simulation displays good acceptance with real grocery market behavior, i.e. captures the performance of British retailers thus can be used to determine the impact of changes in their behavior on their competitors and consumers.Furthermore it can be used to provide insight into sustainability of volatile pricing strategies, providing a useful insight in volatility of British supermarket retail industry. 
\end{abstract}
\acknowledgements{
I would like to express my sincere gratitude to Dr Maria Polukarov for her guidance and support which provided me the freedom to take this research in the direction of my interest.\\
\\
I would also like to thank my family and friends for their encouragement and support. To those who quietly listened to my software complaints. To those who worked throughout the nights with me. To those who helped me write what I couldn't say. I cannot thank you enough.
}

\declaration{
I, Stefan Collier, declare that this dissertation and the work presented in it are my own and has been generated by me as the result of my own original research.\\
I confirm that:\\
1. This work was done wholly or mainly while in candidature for a degree at this University;\\
2. Where any part of this dissertation has previously been submitted for any other qualification at this University or any other institution, this has been clearly stated;\\
3. Where I have consulted the published work of others, this is always clearly attributed;\\
4. Where I have quoted from the work of others, the source is always given. With the exception of such quotations, this dissertation is entirely my own work;\\
5. I have acknowledged all main sources of help;\\
6. Where the thesis is based on work done by myself jointly with others, I have made clear exactly what was done by others and what I have contributed myself;\\
7. Either none of this work has been published before submission, or parts of this work have been published by :\\
\\
Stefan Collier\\
April 2016
}
\tableofcontents
\listoffigures
\listoftables

\mainmatter
%% ----------------------------------------------------------------
%\include{Introduction}
%\include{Conclusions}
\include{chapters/1Project/main}
\include{chapters/2Lit/main}
\include{chapters/3Design/HighLevel}
\include{chapters/3Design/InDepth}
\include{chapters/4Impl/main}

\include{chapters/5Experiments/1/main}
\include{chapters/5Experiments/2/main}
\include{chapters/5Experiments/3/main}
\include{chapters/5Experiments/4/main}

\include{chapters/6Conclusion/main}

\appendix
\include{appendix/AppendixB}
\include{appendix/D/main}
\include{appendix/AppendixC}

\backmatter
\bibliographystyle{ecs}
\bibliography{ECS}
\end{document}
%% ----------------------------------------------------------------

 %% ----------------------------------------------------------------
%% Progress.tex
%% ---------------------------------------------------------------- 
\documentclass{ecsprogress}    % Use the progress Style
\graphicspath{{../figs/}}   % Location of your graphics files
    \usepackage{natbib}            % Use Natbib style for the refs.
\hypersetup{colorlinks=true}   % Set to false for black/white printing
\input{Definitions}            % Include your abbreviations



\usepackage{enumitem}% http://ctan.org/pkg/enumitem
\usepackage{multirow}
\usepackage{float}
\usepackage{amsmath}
\usepackage{multicol}
\usepackage{amssymb}
\usepackage[normalem]{ulem}
\useunder{\uline}{\ul}{}
\usepackage{wrapfig}


\usepackage[table,xcdraw]{xcolor}


%% ----------------------------------------------------------------
\begin{document}
\frontmatter
\title      {Heterogeneous Agent-based Model for Supermarket Competition}
\authors    {\texorpdfstring
             {\href{mailto:sc22g13@ecs.soton.ac.uk}{Stefan J. Collier}}
             {Stefan J. Collier}
            }
\addresses  {\groupname\\\deptname\\\univname}
\date       {\today}
\subject    {}
\keywords   {}
\supervisor {Dr. Maria Polukarov}
\examiner   {Professor Sheng Chen}

\maketitle
\begin{abstract}
This project aim was to model and analyse the effects of competitive pricing behaviors of grocery retailers on the British market. 

This was achieved by creating a multi-agent model, containing retailer and consumer agents. The heterogeneous crowd of retailers employs either a uniform pricing strategy or a ‘local price flexing’ strategy. The actions of these retailers are chosen by predicting the profit of each action, using a perceptron. Following on from the consideration of different economic models, a discrete model was developed so that software agents have a discrete environment to operate within. Within the model, it has been observed how supermarkets with differing behaviors affect a heterogeneous crowd of consumer agents. The model was implemented in Java with Python used to evaluate the results. 

The simulation displays good acceptance with real grocery market behavior, i.e. captures the performance of British retailers thus can be used to determine the impact of changes in their behavior on their competitors and consumers.Furthermore it can be used to provide insight into sustainability of volatile pricing strategies, providing a useful insight in volatility of British supermarket retail industry. 
\end{abstract}
\acknowledgements{
I would like to express my sincere gratitude to Dr Maria Polukarov for her guidance and support which provided me the freedom to take this research in the direction of my interest.\\
\\
I would also like to thank my family and friends for their encouragement and support. To those who quietly listened to my software complaints. To those who worked throughout the nights with me. To those who helped me write what I couldn't say. I cannot thank you enough.
}

\declaration{
I, Stefan Collier, declare that this dissertation and the work presented in it are my own and has been generated by me as the result of my own original research.\\
I confirm that:\\
1. This work was done wholly or mainly while in candidature for a degree at this University;\\
2. Where any part of this dissertation has previously been submitted for any other qualification at this University or any other institution, this has been clearly stated;\\
3. Where I have consulted the published work of others, this is always clearly attributed;\\
4. Where I have quoted from the work of others, the source is always given. With the exception of such quotations, this dissertation is entirely my own work;\\
5. I have acknowledged all main sources of help;\\
6. Where the thesis is based on work done by myself jointly with others, I have made clear exactly what was done by others and what I have contributed myself;\\
7. Either none of this work has been published before submission, or parts of this work have been published by :\\
\\
Stefan Collier\\
April 2016
}
\tableofcontents
\listoffigures
\listoftables

\mainmatter
%% ----------------------------------------------------------------
%\include{Introduction}
%\include{Conclusions}
\include{chapters/1Project/main}
\include{chapters/2Lit/main}
\include{chapters/3Design/HighLevel}
\include{chapters/3Design/InDepth}
\include{chapters/4Impl/main}

\include{chapters/5Experiments/1/main}
\include{chapters/5Experiments/2/main}
\include{chapters/5Experiments/3/main}
\include{chapters/5Experiments/4/main}

\include{chapters/6Conclusion/main}

\appendix
\include{appendix/AppendixB}
\include{appendix/D/main}
\include{appendix/AppendixC}

\backmatter
\bibliographystyle{ecs}
\bibliography{ECS}
\end{document}
%% ----------------------------------------------------------------

\include{chapters/3Design/HighLevel}
\include{chapters/3Design/InDepth}
 %% ----------------------------------------------------------------
%% Progress.tex
%% ---------------------------------------------------------------- 
\documentclass{ecsprogress}    % Use the progress Style
\graphicspath{{../figs/}}   % Location of your graphics files
    \usepackage{natbib}            % Use Natbib style for the refs.
\hypersetup{colorlinks=true}   % Set to false for black/white printing
\input{Definitions}            % Include your abbreviations



\usepackage{enumitem}% http://ctan.org/pkg/enumitem
\usepackage{multirow}
\usepackage{float}
\usepackage{amsmath}
\usepackage{multicol}
\usepackage{amssymb}
\usepackage[normalem]{ulem}
\useunder{\uline}{\ul}{}
\usepackage{wrapfig}


\usepackage[table,xcdraw]{xcolor}


%% ----------------------------------------------------------------
\begin{document}
\frontmatter
\title      {Heterogeneous Agent-based Model for Supermarket Competition}
\authors    {\texorpdfstring
             {\href{mailto:sc22g13@ecs.soton.ac.uk}{Stefan J. Collier}}
             {Stefan J. Collier}
            }
\addresses  {\groupname\\\deptname\\\univname}
\date       {\today}
\subject    {}
\keywords   {}
\supervisor {Dr. Maria Polukarov}
\examiner   {Professor Sheng Chen}

\maketitle
\begin{abstract}
This project aim was to model and analyse the effects of competitive pricing behaviors of grocery retailers on the British market. 

This was achieved by creating a multi-agent model, containing retailer and consumer agents. The heterogeneous crowd of retailers employs either a uniform pricing strategy or a ‘local price flexing’ strategy. The actions of these retailers are chosen by predicting the profit of each action, using a perceptron. Following on from the consideration of different economic models, a discrete model was developed so that software agents have a discrete environment to operate within. Within the model, it has been observed how supermarkets with differing behaviors affect a heterogeneous crowd of consumer agents. The model was implemented in Java with Python used to evaluate the results. 

The simulation displays good acceptance with real grocery market behavior, i.e. captures the performance of British retailers thus can be used to determine the impact of changes in their behavior on their competitors and consumers.Furthermore it can be used to provide insight into sustainability of volatile pricing strategies, providing a useful insight in volatility of British supermarket retail industry. 
\end{abstract}
\acknowledgements{
I would like to express my sincere gratitude to Dr Maria Polukarov for her guidance and support which provided me the freedom to take this research in the direction of my interest.\\
\\
I would also like to thank my family and friends for their encouragement and support. To those who quietly listened to my software complaints. To those who worked throughout the nights with me. To those who helped me write what I couldn't say. I cannot thank you enough.
}

\declaration{
I, Stefan Collier, declare that this dissertation and the work presented in it are my own and has been generated by me as the result of my own original research.\\
I confirm that:\\
1. This work was done wholly or mainly while in candidature for a degree at this University;\\
2. Where any part of this dissertation has previously been submitted for any other qualification at this University or any other institution, this has been clearly stated;\\
3. Where I have consulted the published work of others, this is always clearly attributed;\\
4. Where I have quoted from the work of others, the source is always given. With the exception of such quotations, this dissertation is entirely my own work;\\
5. I have acknowledged all main sources of help;\\
6. Where the thesis is based on work done by myself jointly with others, I have made clear exactly what was done by others and what I have contributed myself;\\
7. Either none of this work has been published before submission, or parts of this work have been published by :\\
\\
Stefan Collier\\
April 2016
}
\tableofcontents
\listoffigures
\listoftables

\mainmatter
%% ----------------------------------------------------------------
%\include{Introduction}
%\include{Conclusions}
\include{chapters/1Project/main}
\include{chapters/2Lit/main}
\include{chapters/3Design/HighLevel}
\include{chapters/3Design/InDepth}
\include{chapters/4Impl/main}

\include{chapters/5Experiments/1/main}
\include{chapters/5Experiments/2/main}
\include{chapters/5Experiments/3/main}
\include{chapters/5Experiments/4/main}

\include{chapters/6Conclusion/main}

\appendix
\include{appendix/AppendixB}
\include{appendix/D/main}
\include{appendix/AppendixC}

\backmatter
\bibliographystyle{ecs}
\bibliography{ECS}
\end{document}
%% ----------------------------------------------------------------


 %% ----------------------------------------------------------------
%% Progress.tex
%% ---------------------------------------------------------------- 
\documentclass{ecsprogress}    % Use the progress Style
\graphicspath{{../figs/}}   % Location of your graphics files
    \usepackage{natbib}            % Use Natbib style for the refs.
\hypersetup{colorlinks=true}   % Set to false for black/white printing
\input{Definitions}            % Include your abbreviations



\usepackage{enumitem}% http://ctan.org/pkg/enumitem
\usepackage{multirow}
\usepackage{float}
\usepackage{amsmath}
\usepackage{multicol}
\usepackage{amssymb}
\usepackage[normalem]{ulem}
\useunder{\uline}{\ul}{}
\usepackage{wrapfig}


\usepackage[table,xcdraw]{xcolor}


%% ----------------------------------------------------------------
\begin{document}
\frontmatter
\title      {Heterogeneous Agent-based Model for Supermarket Competition}
\authors    {\texorpdfstring
             {\href{mailto:sc22g13@ecs.soton.ac.uk}{Stefan J. Collier}}
             {Stefan J. Collier}
            }
\addresses  {\groupname\\\deptname\\\univname}
\date       {\today}
\subject    {}
\keywords   {}
\supervisor {Dr. Maria Polukarov}
\examiner   {Professor Sheng Chen}

\maketitle
\begin{abstract}
This project aim was to model and analyse the effects of competitive pricing behaviors of grocery retailers on the British market. 

This was achieved by creating a multi-agent model, containing retailer and consumer agents. The heterogeneous crowd of retailers employs either a uniform pricing strategy or a ‘local price flexing’ strategy. The actions of these retailers are chosen by predicting the profit of each action, using a perceptron. Following on from the consideration of different economic models, a discrete model was developed so that software agents have a discrete environment to operate within. Within the model, it has been observed how supermarkets with differing behaviors affect a heterogeneous crowd of consumer agents. The model was implemented in Java with Python used to evaluate the results. 

The simulation displays good acceptance with real grocery market behavior, i.e. captures the performance of British retailers thus can be used to determine the impact of changes in their behavior on their competitors and consumers.Furthermore it can be used to provide insight into sustainability of volatile pricing strategies, providing a useful insight in volatility of British supermarket retail industry. 
\end{abstract}
\acknowledgements{
I would like to express my sincere gratitude to Dr Maria Polukarov for her guidance and support which provided me the freedom to take this research in the direction of my interest.\\
\\
I would also like to thank my family and friends for their encouragement and support. To those who quietly listened to my software complaints. To those who worked throughout the nights with me. To those who helped me write what I couldn't say. I cannot thank you enough.
}

\declaration{
I, Stefan Collier, declare that this dissertation and the work presented in it are my own and has been generated by me as the result of my own original research.\\
I confirm that:\\
1. This work was done wholly or mainly while in candidature for a degree at this University;\\
2. Where any part of this dissertation has previously been submitted for any other qualification at this University or any other institution, this has been clearly stated;\\
3. Where I have consulted the published work of others, this is always clearly attributed;\\
4. Where I have quoted from the work of others, the source is always given. With the exception of such quotations, this dissertation is entirely my own work;\\
5. I have acknowledged all main sources of help;\\
6. Where the thesis is based on work done by myself jointly with others, I have made clear exactly what was done by others and what I have contributed myself;\\
7. Either none of this work has been published before submission, or parts of this work have been published by :\\
\\
Stefan Collier\\
April 2016
}
\tableofcontents
\listoffigures
\listoftables

\mainmatter
%% ----------------------------------------------------------------
%\include{Introduction}
%\include{Conclusions}
\include{chapters/1Project/main}
\include{chapters/2Lit/main}
\include{chapters/3Design/HighLevel}
\include{chapters/3Design/InDepth}
\include{chapters/4Impl/main}

\include{chapters/5Experiments/1/main}
\include{chapters/5Experiments/2/main}
\include{chapters/5Experiments/3/main}
\include{chapters/5Experiments/4/main}

\include{chapters/6Conclusion/main}

\appendix
\include{appendix/AppendixB}
\include{appendix/D/main}
\include{appendix/AppendixC}

\backmatter
\bibliographystyle{ecs}
\bibliography{ECS}
\end{document}
%% ----------------------------------------------------------------

 %% ----------------------------------------------------------------
%% Progress.tex
%% ---------------------------------------------------------------- 
\documentclass{ecsprogress}    % Use the progress Style
\graphicspath{{../figs/}}   % Location of your graphics files
    \usepackage{natbib}            % Use Natbib style for the refs.
\hypersetup{colorlinks=true}   % Set to false for black/white printing
\input{Definitions}            % Include your abbreviations



\usepackage{enumitem}% http://ctan.org/pkg/enumitem
\usepackage{multirow}
\usepackage{float}
\usepackage{amsmath}
\usepackage{multicol}
\usepackage{amssymb}
\usepackage[normalem]{ulem}
\useunder{\uline}{\ul}{}
\usepackage{wrapfig}


\usepackage[table,xcdraw]{xcolor}


%% ----------------------------------------------------------------
\begin{document}
\frontmatter
\title      {Heterogeneous Agent-based Model for Supermarket Competition}
\authors    {\texorpdfstring
             {\href{mailto:sc22g13@ecs.soton.ac.uk}{Stefan J. Collier}}
             {Stefan J. Collier}
            }
\addresses  {\groupname\\\deptname\\\univname}
\date       {\today}
\subject    {}
\keywords   {}
\supervisor {Dr. Maria Polukarov}
\examiner   {Professor Sheng Chen}

\maketitle
\begin{abstract}
This project aim was to model and analyse the effects of competitive pricing behaviors of grocery retailers on the British market. 

This was achieved by creating a multi-agent model, containing retailer and consumer agents. The heterogeneous crowd of retailers employs either a uniform pricing strategy or a ‘local price flexing’ strategy. The actions of these retailers are chosen by predicting the profit of each action, using a perceptron. Following on from the consideration of different economic models, a discrete model was developed so that software agents have a discrete environment to operate within. Within the model, it has been observed how supermarkets with differing behaviors affect a heterogeneous crowd of consumer agents. The model was implemented in Java with Python used to evaluate the results. 

The simulation displays good acceptance with real grocery market behavior, i.e. captures the performance of British retailers thus can be used to determine the impact of changes in their behavior on their competitors and consumers.Furthermore it can be used to provide insight into sustainability of volatile pricing strategies, providing a useful insight in volatility of British supermarket retail industry. 
\end{abstract}
\acknowledgements{
I would like to express my sincere gratitude to Dr Maria Polukarov for her guidance and support which provided me the freedom to take this research in the direction of my interest.\\
\\
I would also like to thank my family and friends for their encouragement and support. To those who quietly listened to my software complaints. To those who worked throughout the nights with me. To those who helped me write what I couldn't say. I cannot thank you enough.
}

\declaration{
I, Stefan Collier, declare that this dissertation and the work presented in it are my own and has been generated by me as the result of my own original research.\\
I confirm that:\\
1. This work was done wholly or mainly while in candidature for a degree at this University;\\
2. Where any part of this dissertation has previously been submitted for any other qualification at this University or any other institution, this has been clearly stated;\\
3. Where I have consulted the published work of others, this is always clearly attributed;\\
4. Where I have quoted from the work of others, the source is always given. With the exception of such quotations, this dissertation is entirely my own work;\\
5. I have acknowledged all main sources of help;\\
6. Where the thesis is based on work done by myself jointly with others, I have made clear exactly what was done by others and what I have contributed myself;\\
7. Either none of this work has been published before submission, or parts of this work have been published by :\\
\\
Stefan Collier\\
April 2016
}
\tableofcontents
\listoffigures
\listoftables

\mainmatter
%% ----------------------------------------------------------------
%\include{Introduction}
%\include{Conclusions}
\include{chapters/1Project/main}
\include{chapters/2Lit/main}
\include{chapters/3Design/HighLevel}
\include{chapters/3Design/InDepth}
\include{chapters/4Impl/main}

\include{chapters/5Experiments/1/main}
\include{chapters/5Experiments/2/main}
\include{chapters/5Experiments/3/main}
\include{chapters/5Experiments/4/main}

\include{chapters/6Conclusion/main}

\appendix
\include{appendix/AppendixB}
\include{appendix/D/main}
\include{appendix/AppendixC}

\backmatter
\bibliographystyle{ecs}
\bibliography{ECS}
\end{document}
%% ----------------------------------------------------------------

 %% ----------------------------------------------------------------
%% Progress.tex
%% ---------------------------------------------------------------- 
\documentclass{ecsprogress}    % Use the progress Style
\graphicspath{{../figs/}}   % Location of your graphics files
    \usepackage{natbib}            % Use Natbib style for the refs.
\hypersetup{colorlinks=true}   % Set to false for black/white printing
\input{Definitions}            % Include your abbreviations



\usepackage{enumitem}% http://ctan.org/pkg/enumitem
\usepackage{multirow}
\usepackage{float}
\usepackage{amsmath}
\usepackage{multicol}
\usepackage{amssymb}
\usepackage[normalem]{ulem}
\useunder{\uline}{\ul}{}
\usepackage{wrapfig}


\usepackage[table,xcdraw]{xcolor}


%% ----------------------------------------------------------------
\begin{document}
\frontmatter
\title      {Heterogeneous Agent-based Model for Supermarket Competition}
\authors    {\texorpdfstring
             {\href{mailto:sc22g13@ecs.soton.ac.uk}{Stefan J. Collier}}
             {Stefan J. Collier}
            }
\addresses  {\groupname\\\deptname\\\univname}
\date       {\today}
\subject    {}
\keywords   {}
\supervisor {Dr. Maria Polukarov}
\examiner   {Professor Sheng Chen}

\maketitle
\begin{abstract}
This project aim was to model and analyse the effects of competitive pricing behaviors of grocery retailers on the British market. 

This was achieved by creating a multi-agent model, containing retailer and consumer agents. The heterogeneous crowd of retailers employs either a uniform pricing strategy or a ‘local price flexing’ strategy. The actions of these retailers are chosen by predicting the profit of each action, using a perceptron. Following on from the consideration of different economic models, a discrete model was developed so that software agents have a discrete environment to operate within. Within the model, it has been observed how supermarkets with differing behaviors affect a heterogeneous crowd of consumer agents. The model was implemented in Java with Python used to evaluate the results. 

The simulation displays good acceptance with real grocery market behavior, i.e. captures the performance of British retailers thus can be used to determine the impact of changes in their behavior on their competitors and consumers.Furthermore it can be used to provide insight into sustainability of volatile pricing strategies, providing a useful insight in volatility of British supermarket retail industry. 
\end{abstract}
\acknowledgements{
I would like to express my sincere gratitude to Dr Maria Polukarov for her guidance and support which provided me the freedom to take this research in the direction of my interest.\\
\\
I would also like to thank my family and friends for their encouragement and support. To those who quietly listened to my software complaints. To those who worked throughout the nights with me. To those who helped me write what I couldn't say. I cannot thank you enough.
}

\declaration{
I, Stefan Collier, declare that this dissertation and the work presented in it are my own and has been generated by me as the result of my own original research.\\
I confirm that:\\
1. This work was done wholly or mainly while in candidature for a degree at this University;\\
2. Where any part of this dissertation has previously been submitted for any other qualification at this University or any other institution, this has been clearly stated;\\
3. Where I have consulted the published work of others, this is always clearly attributed;\\
4. Where I have quoted from the work of others, the source is always given. With the exception of such quotations, this dissertation is entirely my own work;\\
5. I have acknowledged all main sources of help;\\
6. Where the thesis is based on work done by myself jointly with others, I have made clear exactly what was done by others and what I have contributed myself;\\
7. Either none of this work has been published before submission, or parts of this work have been published by :\\
\\
Stefan Collier\\
April 2016
}
\tableofcontents
\listoffigures
\listoftables

\mainmatter
%% ----------------------------------------------------------------
%\include{Introduction}
%\include{Conclusions}
\include{chapters/1Project/main}
\include{chapters/2Lit/main}
\include{chapters/3Design/HighLevel}
\include{chapters/3Design/InDepth}
\include{chapters/4Impl/main}

\include{chapters/5Experiments/1/main}
\include{chapters/5Experiments/2/main}
\include{chapters/5Experiments/3/main}
\include{chapters/5Experiments/4/main}

\include{chapters/6Conclusion/main}

\appendix
\include{appendix/AppendixB}
\include{appendix/D/main}
\include{appendix/AppendixC}

\backmatter
\bibliographystyle{ecs}
\bibliography{ECS}
\end{document}
%% ----------------------------------------------------------------

 %% ----------------------------------------------------------------
%% Progress.tex
%% ---------------------------------------------------------------- 
\documentclass{ecsprogress}    % Use the progress Style
\graphicspath{{../figs/}}   % Location of your graphics files
    \usepackage{natbib}            % Use Natbib style for the refs.
\hypersetup{colorlinks=true}   % Set to false for black/white printing
\input{Definitions}            % Include your abbreviations



\usepackage{enumitem}% http://ctan.org/pkg/enumitem
\usepackage{multirow}
\usepackage{float}
\usepackage{amsmath}
\usepackage{multicol}
\usepackage{amssymb}
\usepackage[normalem]{ulem}
\useunder{\uline}{\ul}{}
\usepackage{wrapfig}


\usepackage[table,xcdraw]{xcolor}


%% ----------------------------------------------------------------
\begin{document}
\frontmatter
\title      {Heterogeneous Agent-based Model for Supermarket Competition}
\authors    {\texorpdfstring
             {\href{mailto:sc22g13@ecs.soton.ac.uk}{Stefan J. Collier}}
             {Stefan J. Collier}
            }
\addresses  {\groupname\\\deptname\\\univname}
\date       {\today}
\subject    {}
\keywords   {}
\supervisor {Dr. Maria Polukarov}
\examiner   {Professor Sheng Chen}

\maketitle
\begin{abstract}
This project aim was to model and analyse the effects of competitive pricing behaviors of grocery retailers on the British market. 

This was achieved by creating a multi-agent model, containing retailer and consumer agents. The heterogeneous crowd of retailers employs either a uniform pricing strategy or a ‘local price flexing’ strategy. The actions of these retailers are chosen by predicting the profit of each action, using a perceptron. Following on from the consideration of different economic models, a discrete model was developed so that software agents have a discrete environment to operate within. Within the model, it has been observed how supermarkets with differing behaviors affect a heterogeneous crowd of consumer agents. The model was implemented in Java with Python used to evaluate the results. 

The simulation displays good acceptance with real grocery market behavior, i.e. captures the performance of British retailers thus can be used to determine the impact of changes in their behavior on their competitors and consumers.Furthermore it can be used to provide insight into sustainability of volatile pricing strategies, providing a useful insight in volatility of British supermarket retail industry. 
\end{abstract}
\acknowledgements{
I would like to express my sincere gratitude to Dr Maria Polukarov for her guidance and support which provided me the freedom to take this research in the direction of my interest.\\
\\
I would also like to thank my family and friends for their encouragement and support. To those who quietly listened to my software complaints. To those who worked throughout the nights with me. To those who helped me write what I couldn't say. I cannot thank you enough.
}

\declaration{
I, Stefan Collier, declare that this dissertation and the work presented in it are my own and has been generated by me as the result of my own original research.\\
I confirm that:\\
1. This work was done wholly or mainly while in candidature for a degree at this University;\\
2. Where any part of this dissertation has previously been submitted for any other qualification at this University or any other institution, this has been clearly stated;\\
3. Where I have consulted the published work of others, this is always clearly attributed;\\
4. Where I have quoted from the work of others, the source is always given. With the exception of such quotations, this dissertation is entirely my own work;\\
5. I have acknowledged all main sources of help;\\
6. Where the thesis is based on work done by myself jointly with others, I have made clear exactly what was done by others and what I have contributed myself;\\
7. Either none of this work has been published before submission, or parts of this work have been published by :\\
\\
Stefan Collier\\
April 2016
}
\tableofcontents
\listoffigures
\listoftables

\mainmatter
%% ----------------------------------------------------------------
%\include{Introduction}
%\include{Conclusions}
\include{chapters/1Project/main}
\include{chapters/2Lit/main}
\include{chapters/3Design/HighLevel}
\include{chapters/3Design/InDepth}
\include{chapters/4Impl/main}

\include{chapters/5Experiments/1/main}
\include{chapters/5Experiments/2/main}
\include{chapters/5Experiments/3/main}
\include{chapters/5Experiments/4/main}

\include{chapters/6Conclusion/main}

\appendix
\include{appendix/AppendixB}
\include{appendix/D/main}
\include{appendix/AppendixC}

\backmatter
\bibliographystyle{ecs}
\bibliography{ECS}
\end{document}
%% ----------------------------------------------------------------


 %% ----------------------------------------------------------------
%% Progress.tex
%% ---------------------------------------------------------------- 
\documentclass{ecsprogress}    % Use the progress Style
\graphicspath{{../figs/}}   % Location of your graphics files
    \usepackage{natbib}            % Use Natbib style for the refs.
\hypersetup{colorlinks=true}   % Set to false for black/white printing
\input{Definitions}            % Include your abbreviations



\usepackage{enumitem}% http://ctan.org/pkg/enumitem
\usepackage{multirow}
\usepackage{float}
\usepackage{amsmath}
\usepackage{multicol}
\usepackage{amssymb}
\usepackage[normalem]{ulem}
\useunder{\uline}{\ul}{}
\usepackage{wrapfig}


\usepackage[table,xcdraw]{xcolor}


%% ----------------------------------------------------------------
\begin{document}
\frontmatter
\title      {Heterogeneous Agent-based Model for Supermarket Competition}
\authors    {\texorpdfstring
             {\href{mailto:sc22g13@ecs.soton.ac.uk}{Stefan J. Collier}}
             {Stefan J. Collier}
            }
\addresses  {\groupname\\\deptname\\\univname}
\date       {\today}
\subject    {}
\keywords   {}
\supervisor {Dr. Maria Polukarov}
\examiner   {Professor Sheng Chen}

\maketitle
\begin{abstract}
This project aim was to model and analyse the effects of competitive pricing behaviors of grocery retailers on the British market. 

This was achieved by creating a multi-agent model, containing retailer and consumer agents. The heterogeneous crowd of retailers employs either a uniform pricing strategy or a ‘local price flexing’ strategy. The actions of these retailers are chosen by predicting the profit of each action, using a perceptron. Following on from the consideration of different economic models, a discrete model was developed so that software agents have a discrete environment to operate within. Within the model, it has been observed how supermarkets with differing behaviors affect a heterogeneous crowd of consumer agents. The model was implemented in Java with Python used to evaluate the results. 

The simulation displays good acceptance with real grocery market behavior, i.e. captures the performance of British retailers thus can be used to determine the impact of changes in their behavior on their competitors and consumers.Furthermore it can be used to provide insight into sustainability of volatile pricing strategies, providing a useful insight in volatility of British supermarket retail industry. 
\end{abstract}
\acknowledgements{
I would like to express my sincere gratitude to Dr Maria Polukarov for her guidance and support which provided me the freedom to take this research in the direction of my interest.\\
\\
I would also like to thank my family and friends for their encouragement and support. To those who quietly listened to my software complaints. To those who worked throughout the nights with me. To those who helped me write what I couldn't say. I cannot thank you enough.
}

\declaration{
I, Stefan Collier, declare that this dissertation and the work presented in it are my own and has been generated by me as the result of my own original research.\\
I confirm that:\\
1. This work was done wholly or mainly while in candidature for a degree at this University;\\
2. Where any part of this dissertation has previously been submitted for any other qualification at this University or any other institution, this has been clearly stated;\\
3. Where I have consulted the published work of others, this is always clearly attributed;\\
4. Where I have quoted from the work of others, the source is always given. With the exception of such quotations, this dissertation is entirely my own work;\\
5. I have acknowledged all main sources of help;\\
6. Where the thesis is based on work done by myself jointly with others, I have made clear exactly what was done by others and what I have contributed myself;\\
7. Either none of this work has been published before submission, or parts of this work have been published by :\\
\\
Stefan Collier\\
April 2016
}
\tableofcontents
\listoffigures
\listoftables

\mainmatter
%% ----------------------------------------------------------------
%\include{Introduction}
%\include{Conclusions}
\include{chapters/1Project/main}
\include{chapters/2Lit/main}
\include{chapters/3Design/HighLevel}
\include{chapters/3Design/InDepth}
\include{chapters/4Impl/main}

\include{chapters/5Experiments/1/main}
\include{chapters/5Experiments/2/main}
\include{chapters/5Experiments/3/main}
\include{chapters/5Experiments/4/main}

\include{chapters/6Conclusion/main}

\appendix
\include{appendix/AppendixB}
\include{appendix/D/main}
\include{appendix/AppendixC}

\backmatter
\bibliographystyle{ecs}
\bibliography{ECS}
\end{document}
%% ----------------------------------------------------------------


\appendix
\include{appendix/AppendixB}
 %% ----------------------------------------------------------------
%% Progress.tex
%% ---------------------------------------------------------------- 
\documentclass{ecsprogress}    % Use the progress Style
\graphicspath{{../figs/}}   % Location of your graphics files
    \usepackage{natbib}            % Use Natbib style for the refs.
\hypersetup{colorlinks=true}   % Set to false for black/white printing
\input{Definitions}            % Include your abbreviations



\usepackage{enumitem}% http://ctan.org/pkg/enumitem
\usepackage{multirow}
\usepackage{float}
\usepackage{amsmath}
\usepackage{multicol}
\usepackage{amssymb}
\usepackage[normalem]{ulem}
\useunder{\uline}{\ul}{}
\usepackage{wrapfig}


\usepackage[table,xcdraw]{xcolor}


%% ----------------------------------------------------------------
\begin{document}
\frontmatter
\title      {Heterogeneous Agent-based Model for Supermarket Competition}
\authors    {\texorpdfstring
             {\href{mailto:sc22g13@ecs.soton.ac.uk}{Stefan J. Collier}}
             {Stefan J. Collier}
            }
\addresses  {\groupname\\\deptname\\\univname}
\date       {\today}
\subject    {}
\keywords   {}
\supervisor {Dr. Maria Polukarov}
\examiner   {Professor Sheng Chen}

\maketitle
\begin{abstract}
This project aim was to model and analyse the effects of competitive pricing behaviors of grocery retailers on the British market. 

This was achieved by creating a multi-agent model, containing retailer and consumer agents. The heterogeneous crowd of retailers employs either a uniform pricing strategy or a ‘local price flexing’ strategy. The actions of these retailers are chosen by predicting the profit of each action, using a perceptron. Following on from the consideration of different economic models, a discrete model was developed so that software agents have a discrete environment to operate within. Within the model, it has been observed how supermarkets with differing behaviors affect a heterogeneous crowd of consumer agents. The model was implemented in Java with Python used to evaluate the results. 

The simulation displays good acceptance with real grocery market behavior, i.e. captures the performance of British retailers thus can be used to determine the impact of changes in their behavior on their competitors and consumers.Furthermore it can be used to provide insight into sustainability of volatile pricing strategies, providing a useful insight in volatility of British supermarket retail industry. 
\end{abstract}
\acknowledgements{
I would like to express my sincere gratitude to Dr Maria Polukarov for her guidance and support which provided me the freedom to take this research in the direction of my interest.\\
\\
I would also like to thank my family and friends for their encouragement and support. To those who quietly listened to my software complaints. To those who worked throughout the nights with me. To those who helped me write what I couldn't say. I cannot thank you enough.
}

\declaration{
I, Stefan Collier, declare that this dissertation and the work presented in it are my own and has been generated by me as the result of my own original research.\\
I confirm that:\\
1. This work was done wholly or mainly while in candidature for a degree at this University;\\
2. Where any part of this dissertation has previously been submitted for any other qualification at this University or any other institution, this has been clearly stated;\\
3. Where I have consulted the published work of others, this is always clearly attributed;\\
4. Where I have quoted from the work of others, the source is always given. With the exception of such quotations, this dissertation is entirely my own work;\\
5. I have acknowledged all main sources of help;\\
6. Where the thesis is based on work done by myself jointly with others, I have made clear exactly what was done by others and what I have contributed myself;\\
7. Either none of this work has been published before submission, or parts of this work have been published by :\\
\\
Stefan Collier\\
April 2016
}
\tableofcontents
\listoffigures
\listoftables

\mainmatter
%% ----------------------------------------------------------------
%\include{Introduction}
%\include{Conclusions}
\include{chapters/1Project/main}
\include{chapters/2Lit/main}
\include{chapters/3Design/HighLevel}
\include{chapters/3Design/InDepth}
\include{chapters/4Impl/main}

\include{chapters/5Experiments/1/main}
\include{chapters/5Experiments/2/main}
\include{chapters/5Experiments/3/main}
\include{chapters/5Experiments/4/main}

\include{chapters/6Conclusion/main}

\appendix
\include{appendix/AppendixB}
\include{appendix/D/main}
\include{appendix/AppendixC}

\backmatter
\bibliographystyle{ecs}
\bibliography{ECS}
\end{document}
%% ----------------------------------------------------------------

\include{appendix/AppendixC}

\backmatter
\bibliographystyle{ecs}
\bibliography{ECS}
\end{document}
%% ----------------------------------------------------------------

\include{appendix/AppendixC}

\backmatter
\bibliographystyle{ecs}
\bibliography{ECS}
\end{document}
%% ----------------------------------------------------------------

 %% ----------------------------------------------------------------
%% Progress.tex
%% ---------------------------------------------------------------- 
\documentclass{ecsprogress}    % Use the progress Style
\graphicspath{{../figs/}}   % Location of your graphics files
    \usepackage{natbib}            % Use Natbib style for the refs.
\hypersetup{colorlinks=true}   % Set to false for black/white printing
\input{Definitions}            % Include your abbreviations



\usepackage{enumitem}% http://ctan.org/pkg/enumitem
\usepackage{multirow}
\usepackage{float}
\usepackage{amsmath}
\usepackage{multicol}
\usepackage{amssymb}
\usepackage[normalem]{ulem}
\useunder{\uline}{\ul}{}
\usepackage{wrapfig}


\usepackage[table,xcdraw]{xcolor}


%% ----------------------------------------------------------------
\begin{document}
\frontmatter
\title      {Heterogeneous Agent-based Model for Supermarket Competition}
\authors    {\texorpdfstring
             {\href{mailto:sc22g13@ecs.soton.ac.uk}{Stefan J. Collier}}
             {Stefan J. Collier}
            }
\addresses  {\groupname\\\deptname\\\univname}
\date       {\today}
\subject    {}
\keywords   {}
\supervisor {Dr. Maria Polukarov}
\examiner   {Professor Sheng Chen}

\maketitle
\begin{abstract}
This project aim was to model and analyse the effects of competitive pricing behaviors of grocery retailers on the British market. 

This was achieved by creating a multi-agent model, containing retailer and consumer agents. The heterogeneous crowd of retailers employs either a uniform pricing strategy or a ‘local price flexing’ strategy. The actions of these retailers are chosen by predicting the profit of each action, using a perceptron. Following on from the consideration of different economic models, a discrete model was developed so that software agents have a discrete environment to operate within. Within the model, it has been observed how supermarkets with differing behaviors affect a heterogeneous crowd of consumer agents. The model was implemented in Java with Python used to evaluate the results. 

The simulation displays good acceptance with real grocery market behavior, i.e. captures the performance of British retailers thus can be used to determine the impact of changes in their behavior on their competitors and consumers.Furthermore it can be used to provide insight into sustainability of volatile pricing strategies, providing a useful insight in volatility of British supermarket retail industry. 
\end{abstract}
\acknowledgements{
I would like to express my sincere gratitude to Dr Maria Polukarov for her guidance and support which provided me the freedom to take this research in the direction of my interest.\\
\\
I would also like to thank my family and friends for their encouragement and support. To those who quietly listened to my software complaints. To those who worked throughout the nights with me. To those who helped me write what I couldn't say. I cannot thank you enough.
}

\declaration{
I, Stefan Collier, declare that this dissertation and the work presented in it are my own and has been generated by me as the result of my own original research.\\
I confirm that:\\
1. This work was done wholly or mainly while in candidature for a degree at this University;\\
2. Where any part of this dissertation has previously been submitted for any other qualification at this University or any other institution, this has been clearly stated;\\
3. Where I have consulted the published work of others, this is always clearly attributed;\\
4. Where I have quoted from the work of others, the source is always given. With the exception of such quotations, this dissertation is entirely my own work;\\
5. I have acknowledged all main sources of help;\\
6. Where the thesis is based on work done by myself jointly with others, I have made clear exactly what was done by others and what I have contributed myself;\\
7. Either none of this work has been published before submission, or parts of this work have been published by :\\
\\
Stefan Collier\\
April 2016
}
\tableofcontents
\listoffigures
\listoftables

\mainmatter
%% ----------------------------------------------------------------
%\include{Introduction}
%\include{Conclusions}
 %% ----------------------------------------------------------------
%% Progress.tex
%% ---------------------------------------------------------------- 
\documentclass{ecsprogress}    % Use the progress Style
\graphicspath{{../figs/}}   % Location of your graphics files
    \usepackage{natbib}            % Use Natbib style for the refs.
\hypersetup{colorlinks=true}   % Set to false for black/white printing
\input{Definitions}            % Include your abbreviations



\usepackage{enumitem}% http://ctan.org/pkg/enumitem
\usepackage{multirow}
\usepackage{float}
\usepackage{amsmath}
\usepackage{multicol}
\usepackage{amssymb}
\usepackage[normalem]{ulem}
\useunder{\uline}{\ul}{}
\usepackage{wrapfig}


\usepackage[table,xcdraw]{xcolor}


%% ----------------------------------------------------------------
\begin{document}
\frontmatter
\title      {Heterogeneous Agent-based Model for Supermarket Competition}
\authors    {\texorpdfstring
             {\href{mailto:sc22g13@ecs.soton.ac.uk}{Stefan J. Collier}}
             {Stefan J. Collier}
            }
\addresses  {\groupname\\\deptname\\\univname}
\date       {\today}
\subject    {}
\keywords   {}
\supervisor {Dr. Maria Polukarov}
\examiner   {Professor Sheng Chen}

\maketitle
\begin{abstract}
This project aim was to model and analyse the effects of competitive pricing behaviors of grocery retailers on the British market. 

This was achieved by creating a multi-agent model, containing retailer and consumer agents. The heterogeneous crowd of retailers employs either a uniform pricing strategy or a ‘local price flexing’ strategy. The actions of these retailers are chosen by predicting the profit of each action, using a perceptron. Following on from the consideration of different economic models, a discrete model was developed so that software agents have a discrete environment to operate within. Within the model, it has been observed how supermarkets with differing behaviors affect a heterogeneous crowd of consumer agents. The model was implemented in Java with Python used to evaluate the results. 

The simulation displays good acceptance with real grocery market behavior, i.e. captures the performance of British retailers thus can be used to determine the impact of changes in their behavior on their competitors and consumers.Furthermore it can be used to provide insight into sustainability of volatile pricing strategies, providing a useful insight in volatility of British supermarket retail industry. 
\end{abstract}
\acknowledgements{
I would like to express my sincere gratitude to Dr Maria Polukarov for her guidance and support which provided me the freedom to take this research in the direction of my interest.\\
\\
I would also like to thank my family and friends for their encouragement and support. To those who quietly listened to my software complaints. To those who worked throughout the nights with me. To those who helped me write what I couldn't say. I cannot thank you enough.
}

\declaration{
I, Stefan Collier, declare that this dissertation and the work presented in it are my own and has been generated by me as the result of my own original research.\\
I confirm that:\\
1. This work was done wholly or mainly while in candidature for a degree at this University;\\
2. Where any part of this dissertation has previously been submitted for any other qualification at this University or any other institution, this has been clearly stated;\\
3. Where I have consulted the published work of others, this is always clearly attributed;\\
4. Where I have quoted from the work of others, the source is always given. With the exception of such quotations, this dissertation is entirely my own work;\\
5. I have acknowledged all main sources of help;\\
6. Where the thesis is based on work done by myself jointly with others, I have made clear exactly what was done by others and what I have contributed myself;\\
7. Either none of this work has been published before submission, or parts of this work have been published by :\\
\\
Stefan Collier\\
April 2016
}
\tableofcontents
\listoffigures
\listoftables

\mainmatter
%% ----------------------------------------------------------------
%\include{Introduction}
%\include{Conclusions}
 %% ----------------------------------------------------------------
%% Progress.tex
%% ---------------------------------------------------------------- 
\documentclass{ecsprogress}    % Use the progress Style
\graphicspath{{../figs/}}   % Location of your graphics files
    \usepackage{natbib}            % Use Natbib style for the refs.
\hypersetup{colorlinks=true}   % Set to false for black/white printing
\input{Definitions}            % Include your abbreviations



\usepackage{enumitem}% http://ctan.org/pkg/enumitem
\usepackage{multirow}
\usepackage{float}
\usepackage{amsmath}
\usepackage{multicol}
\usepackage{amssymb}
\usepackage[normalem]{ulem}
\useunder{\uline}{\ul}{}
\usepackage{wrapfig}


\usepackage[table,xcdraw]{xcolor}


%% ----------------------------------------------------------------
\begin{document}
\frontmatter
\title      {Heterogeneous Agent-based Model for Supermarket Competition}
\authors    {\texorpdfstring
             {\href{mailto:sc22g13@ecs.soton.ac.uk}{Stefan J. Collier}}
             {Stefan J. Collier}
            }
\addresses  {\groupname\\\deptname\\\univname}
\date       {\today}
\subject    {}
\keywords   {}
\supervisor {Dr. Maria Polukarov}
\examiner   {Professor Sheng Chen}

\maketitle
\begin{abstract}
This project aim was to model and analyse the effects of competitive pricing behaviors of grocery retailers on the British market. 

This was achieved by creating a multi-agent model, containing retailer and consumer agents. The heterogeneous crowd of retailers employs either a uniform pricing strategy or a ‘local price flexing’ strategy. The actions of these retailers are chosen by predicting the profit of each action, using a perceptron. Following on from the consideration of different economic models, a discrete model was developed so that software agents have a discrete environment to operate within. Within the model, it has been observed how supermarkets with differing behaviors affect a heterogeneous crowd of consumer agents. The model was implemented in Java with Python used to evaluate the results. 

The simulation displays good acceptance with real grocery market behavior, i.e. captures the performance of British retailers thus can be used to determine the impact of changes in their behavior on their competitors and consumers.Furthermore it can be used to provide insight into sustainability of volatile pricing strategies, providing a useful insight in volatility of British supermarket retail industry. 
\end{abstract}
\acknowledgements{
I would like to express my sincere gratitude to Dr Maria Polukarov for her guidance and support which provided me the freedom to take this research in the direction of my interest.\\
\\
I would also like to thank my family and friends for their encouragement and support. To those who quietly listened to my software complaints. To those who worked throughout the nights with me. To those who helped me write what I couldn't say. I cannot thank you enough.
}

\declaration{
I, Stefan Collier, declare that this dissertation and the work presented in it are my own and has been generated by me as the result of my own original research.\\
I confirm that:\\
1. This work was done wholly or mainly while in candidature for a degree at this University;\\
2. Where any part of this dissertation has previously been submitted for any other qualification at this University or any other institution, this has been clearly stated;\\
3. Where I have consulted the published work of others, this is always clearly attributed;\\
4. Where I have quoted from the work of others, the source is always given. With the exception of such quotations, this dissertation is entirely my own work;\\
5. I have acknowledged all main sources of help;\\
6. Where the thesis is based on work done by myself jointly with others, I have made clear exactly what was done by others and what I have contributed myself;\\
7. Either none of this work has been published before submission, or parts of this work have been published by :\\
\\
Stefan Collier\\
April 2016
}
\tableofcontents
\listoffigures
\listoftables

\mainmatter
%% ----------------------------------------------------------------
%\include{Introduction}
%\include{Conclusions}
\include{chapters/1Project/main}
\include{chapters/2Lit/main}
\include{chapters/3Design/HighLevel}
\include{chapters/3Design/InDepth}
\include{chapters/4Impl/main}

\include{chapters/5Experiments/1/main}
\include{chapters/5Experiments/2/main}
\include{chapters/5Experiments/3/main}
\include{chapters/5Experiments/4/main}

\include{chapters/6Conclusion/main}

\appendix
\include{appendix/AppendixB}
\include{appendix/D/main}
\include{appendix/AppendixC}

\backmatter
\bibliographystyle{ecs}
\bibliography{ECS}
\end{document}
%% ----------------------------------------------------------------

 %% ----------------------------------------------------------------
%% Progress.tex
%% ---------------------------------------------------------------- 
\documentclass{ecsprogress}    % Use the progress Style
\graphicspath{{../figs/}}   % Location of your graphics files
    \usepackage{natbib}            % Use Natbib style for the refs.
\hypersetup{colorlinks=true}   % Set to false for black/white printing
\input{Definitions}            % Include your abbreviations



\usepackage{enumitem}% http://ctan.org/pkg/enumitem
\usepackage{multirow}
\usepackage{float}
\usepackage{amsmath}
\usepackage{multicol}
\usepackage{amssymb}
\usepackage[normalem]{ulem}
\useunder{\uline}{\ul}{}
\usepackage{wrapfig}


\usepackage[table,xcdraw]{xcolor}


%% ----------------------------------------------------------------
\begin{document}
\frontmatter
\title      {Heterogeneous Agent-based Model for Supermarket Competition}
\authors    {\texorpdfstring
             {\href{mailto:sc22g13@ecs.soton.ac.uk}{Stefan J. Collier}}
             {Stefan J. Collier}
            }
\addresses  {\groupname\\\deptname\\\univname}
\date       {\today}
\subject    {}
\keywords   {}
\supervisor {Dr. Maria Polukarov}
\examiner   {Professor Sheng Chen}

\maketitle
\begin{abstract}
This project aim was to model and analyse the effects of competitive pricing behaviors of grocery retailers on the British market. 

This was achieved by creating a multi-agent model, containing retailer and consumer agents. The heterogeneous crowd of retailers employs either a uniform pricing strategy or a ‘local price flexing’ strategy. The actions of these retailers are chosen by predicting the profit of each action, using a perceptron. Following on from the consideration of different economic models, a discrete model was developed so that software agents have a discrete environment to operate within. Within the model, it has been observed how supermarkets with differing behaviors affect a heterogeneous crowd of consumer agents. The model was implemented in Java with Python used to evaluate the results. 

The simulation displays good acceptance with real grocery market behavior, i.e. captures the performance of British retailers thus can be used to determine the impact of changes in their behavior on their competitors and consumers.Furthermore it can be used to provide insight into sustainability of volatile pricing strategies, providing a useful insight in volatility of British supermarket retail industry. 
\end{abstract}
\acknowledgements{
I would like to express my sincere gratitude to Dr Maria Polukarov for her guidance and support which provided me the freedom to take this research in the direction of my interest.\\
\\
I would also like to thank my family and friends for their encouragement and support. To those who quietly listened to my software complaints. To those who worked throughout the nights with me. To those who helped me write what I couldn't say. I cannot thank you enough.
}

\declaration{
I, Stefan Collier, declare that this dissertation and the work presented in it are my own and has been generated by me as the result of my own original research.\\
I confirm that:\\
1. This work was done wholly or mainly while in candidature for a degree at this University;\\
2. Where any part of this dissertation has previously been submitted for any other qualification at this University or any other institution, this has been clearly stated;\\
3. Where I have consulted the published work of others, this is always clearly attributed;\\
4. Where I have quoted from the work of others, the source is always given. With the exception of such quotations, this dissertation is entirely my own work;\\
5. I have acknowledged all main sources of help;\\
6. Where the thesis is based on work done by myself jointly with others, I have made clear exactly what was done by others and what I have contributed myself;\\
7. Either none of this work has been published before submission, or parts of this work have been published by :\\
\\
Stefan Collier\\
April 2016
}
\tableofcontents
\listoffigures
\listoftables

\mainmatter
%% ----------------------------------------------------------------
%\include{Introduction}
%\include{Conclusions}
\include{chapters/1Project/main}
\include{chapters/2Lit/main}
\include{chapters/3Design/HighLevel}
\include{chapters/3Design/InDepth}
\include{chapters/4Impl/main}

\include{chapters/5Experiments/1/main}
\include{chapters/5Experiments/2/main}
\include{chapters/5Experiments/3/main}
\include{chapters/5Experiments/4/main}

\include{chapters/6Conclusion/main}

\appendix
\include{appendix/AppendixB}
\include{appendix/D/main}
\include{appendix/AppendixC}

\backmatter
\bibliographystyle{ecs}
\bibliography{ECS}
\end{document}
%% ----------------------------------------------------------------

\include{chapters/3Design/HighLevel}
\include{chapters/3Design/InDepth}
 %% ----------------------------------------------------------------
%% Progress.tex
%% ---------------------------------------------------------------- 
\documentclass{ecsprogress}    % Use the progress Style
\graphicspath{{../figs/}}   % Location of your graphics files
    \usepackage{natbib}            % Use Natbib style for the refs.
\hypersetup{colorlinks=true}   % Set to false for black/white printing
\input{Definitions}            % Include your abbreviations



\usepackage{enumitem}% http://ctan.org/pkg/enumitem
\usepackage{multirow}
\usepackage{float}
\usepackage{amsmath}
\usepackage{multicol}
\usepackage{amssymb}
\usepackage[normalem]{ulem}
\useunder{\uline}{\ul}{}
\usepackage{wrapfig}


\usepackage[table,xcdraw]{xcolor}


%% ----------------------------------------------------------------
\begin{document}
\frontmatter
\title      {Heterogeneous Agent-based Model for Supermarket Competition}
\authors    {\texorpdfstring
             {\href{mailto:sc22g13@ecs.soton.ac.uk}{Stefan J. Collier}}
             {Stefan J. Collier}
            }
\addresses  {\groupname\\\deptname\\\univname}
\date       {\today}
\subject    {}
\keywords   {}
\supervisor {Dr. Maria Polukarov}
\examiner   {Professor Sheng Chen}

\maketitle
\begin{abstract}
This project aim was to model and analyse the effects of competitive pricing behaviors of grocery retailers on the British market. 

This was achieved by creating a multi-agent model, containing retailer and consumer agents. The heterogeneous crowd of retailers employs either a uniform pricing strategy or a ‘local price flexing’ strategy. The actions of these retailers are chosen by predicting the profit of each action, using a perceptron. Following on from the consideration of different economic models, a discrete model was developed so that software agents have a discrete environment to operate within. Within the model, it has been observed how supermarkets with differing behaviors affect a heterogeneous crowd of consumer agents. The model was implemented in Java with Python used to evaluate the results. 

The simulation displays good acceptance with real grocery market behavior, i.e. captures the performance of British retailers thus can be used to determine the impact of changes in their behavior on their competitors and consumers.Furthermore it can be used to provide insight into sustainability of volatile pricing strategies, providing a useful insight in volatility of British supermarket retail industry. 
\end{abstract}
\acknowledgements{
I would like to express my sincere gratitude to Dr Maria Polukarov for her guidance and support which provided me the freedom to take this research in the direction of my interest.\\
\\
I would also like to thank my family and friends for their encouragement and support. To those who quietly listened to my software complaints. To those who worked throughout the nights with me. To those who helped me write what I couldn't say. I cannot thank you enough.
}

\declaration{
I, Stefan Collier, declare that this dissertation and the work presented in it are my own and has been generated by me as the result of my own original research.\\
I confirm that:\\
1. This work was done wholly or mainly while in candidature for a degree at this University;\\
2. Where any part of this dissertation has previously been submitted for any other qualification at this University or any other institution, this has been clearly stated;\\
3. Where I have consulted the published work of others, this is always clearly attributed;\\
4. Where I have quoted from the work of others, the source is always given. With the exception of such quotations, this dissertation is entirely my own work;\\
5. I have acknowledged all main sources of help;\\
6. Where the thesis is based on work done by myself jointly with others, I have made clear exactly what was done by others and what I have contributed myself;\\
7. Either none of this work has been published before submission, or parts of this work have been published by :\\
\\
Stefan Collier\\
April 2016
}
\tableofcontents
\listoffigures
\listoftables

\mainmatter
%% ----------------------------------------------------------------
%\include{Introduction}
%\include{Conclusions}
\include{chapters/1Project/main}
\include{chapters/2Lit/main}
\include{chapters/3Design/HighLevel}
\include{chapters/3Design/InDepth}
\include{chapters/4Impl/main}

\include{chapters/5Experiments/1/main}
\include{chapters/5Experiments/2/main}
\include{chapters/5Experiments/3/main}
\include{chapters/5Experiments/4/main}

\include{chapters/6Conclusion/main}

\appendix
\include{appendix/AppendixB}
\include{appendix/D/main}
\include{appendix/AppendixC}

\backmatter
\bibliographystyle{ecs}
\bibliography{ECS}
\end{document}
%% ----------------------------------------------------------------


 %% ----------------------------------------------------------------
%% Progress.tex
%% ---------------------------------------------------------------- 
\documentclass{ecsprogress}    % Use the progress Style
\graphicspath{{../figs/}}   % Location of your graphics files
    \usepackage{natbib}            % Use Natbib style for the refs.
\hypersetup{colorlinks=true}   % Set to false for black/white printing
\input{Definitions}            % Include your abbreviations



\usepackage{enumitem}% http://ctan.org/pkg/enumitem
\usepackage{multirow}
\usepackage{float}
\usepackage{amsmath}
\usepackage{multicol}
\usepackage{amssymb}
\usepackage[normalem]{ulem}
\useunder{\uline}{\ul}{}
\usepackage{wrapfig}


\usepackage[table,xcdraw]{xcolor}


%% ----------------------------------------------------------------
\begin{document}
\frontmatter
\title      {Heterogeneous Agent-based Model for Supermarket Competition}
\authors    {\texorpdfstring
             {\href{mailto:sc22g13@ecs.soton.ac.uk}{Stefan J. Collier}}
             {Stefan J. Collier}
            }
\addresses  {\groupname\\\deptname\\\univname}
\date       {\today}
\subject    {}
\keywords   {}
\supervisor {Dr. Maria Polukarov}
\examiner   {Professor Sheng Chen}

\maketitle
\begin{abstract}
This project aim was to model and analyse the effects of competitive pricing behaviors of grocery retailers on the British market. 

This was achieved by creating a multi-agent model, containing retailer and consumer agents. The heterogeneous crowd of retailers employs either a uniform pricing strategy or a ‘local price flexing’ strategy. The actions of these retailers are chosen by predicting the profit of each action, using a perceptron. Following on from the consideration of different economic models, a discrete model was developed so that software agents have a discrete environment to operate within. Within the model, it has been observed how supermarkets with differing behaviors affect a heterogeneous crowd of consumer agents. The model was implemented in Java with Python used to evaluate the results. 

The simulation displays good acceptance with real grocery market behavior, i.e. captures the performance of British retailers thus can be used to determine the impact of changes in their behavior on their competitors and consumers.Furthermore it can be used to provide insight into sustainability of volatile pricing strategies, providing a useful insight in volatility of British supermarket retail industry. 
\end{abstract}
\acknowledgements{
I would like to express my sincere gratitude to Dr Maria Polukarov for her guidance and support which provided me the freedom to take this research in the direction of my interest.\\
\\
I would also like to thank my family and friends for their encouragement and support. To those who quietly listened to my software complaints. To those who worked throughout the nights with me. To those who helped me write what I couldn't say. I cannot thank you enough.
}

\declaration{
I, Stefan Collier, declare that this dissertation and the work presented in it are my own and has been generated by me as the result of my own original research.\\
I confirm that:\\
1. This work was done wholly or mainly while in candidature for a degree at this University;\\
2. Where any part of this dissertation has previously been submitted for any other qualification at this University or any other institution, this has been clearly stated;\\
3. Where I have consulted the published work of others, this is always clearly attributed;\\
4. Where I have quoted from the work of others, the source is always given. With the exception of such quotations, this dissertation is entirely my own work;\\
5. I have acknowledged all main sources of help;\\
6. Where the thesis is based on work done by myself jointly with others, I have made clear exactly what was done by others and what I have contributed myself;\\
7. Either none of this work has been published before submission, or parts of this work have been published by :\\
\\
Stefan Collier\\
April 2016
}
\tableofcontents
\listoffigures
\listoftables

\mainmatter
%% ----------------------------------------------------------------
%\include{Introduction}
%\include{Conclusions}
\include{chapters/1Project/main}
\include{chapters/2Lit/main}
\include{chapters/3Design/HighLevel}
\include{chapters/3Design/InDepth}
\include{chapters/4Impl/main}

\include{chapters/5Experiments/1/main}
\include{chapters/5Experiments/2/main}
\include{chapters/5Experiments/3/main}
\include{chapters/5Experiments/4/main}

\include{chapters/6Conclusion/main}

\appendix
\include{appendix/AppendixB}
\include{appendix/D/main}
\include{appendix/AppendixC}

\backmatter
\bibliographystyle{ecs}
\bibliography{ECS}
\end{document}
%% ----------------------------------------------------------------

 %% ----------------------------------------------------------------
%% Progress.tex
%% ---------------------------------------------------------------- 
\documentclass{ecsprogress}    % Use the progress Style
\graphicspath{{../figs/}}   % Location of your graphics files
    \usepackage{natbib}            % Use Natbib style for the refs.
\hypersetup{colorlinks=true}   % Set to false for black/white printing
\input{Definitions}            % Include your abbreviations



\usepackage{enumitem}% http://ctan.org/pkg/enumitem
\usepackage{multirow}
\usepackage{float}
\usepackage{amsmath}
\usepackage{multicol}
\usepackage{amssymb}
\usepackage[normalem]{ulem}
\useunder{\uline}{\ul}{}
\usepackage{wrapfig}


\usepackage[table,xcdraw]{xcolor}


%% ----------------------------------------------------------------
\begin{document}
\frontmatter
\title      {Heterogeneous Agent-based Model for Supermarket Competition}
\authors    {\texorpdfstring
             {\href{mailto:sc22g13@ecs.soton.ac.uk}{Stefan J. Collier}}
             {Stefan J. Collier}
            }
\addresses  {\groupname\\\deptname\\\univname}
\date       {\today}
\subject    {}
\keywords   {}
\supervisor {Dr. Maria Polukarov}
\examiner   {Professor Sheng Chen}

\maketitle
\begin{abstract}
This project aim was to model and analyse the effects of competitive pricing behaviors of grocery retailers on the British market. 

This was achieved by creating a multi-agent model, containing retailer and consumer agents. The heterogeneous crowd of retailers employs either a uniform pricing strategy or a ‘local price flexing’ strategy. The actions of these retailers are chosen by predicting the profit of each action, using a perceptron. Following on from the consideration of different economic models, a discrete model was developed so that software agents have a discrete environment to operate within. Within the model, it has been observed how supermarkets with differing behaviors affect a heterogeneous crowd of consumer agents. The model was implemented in Java with Python used to evaluate the results. 

The simulation displays good acceptance with real grocery market behavior, i.e. captures the performance of British retailers thus can be used to determine the impact of changes in their behavior on their competitors and consumers.Furthermore it can be used to provide insight into sustainability of volatile pricing strategies, providing a useful insight in volatility of British supermarket retail industry. 
\end{abstract}
\acknowledgements{
I would like to express my sincere gratitude to Dr Maria Polukarov for her guidance and support which provided me the freedom to take this research in the direction of my interest.\\
\\
I would also like to thank my family and friends for their encouragement and support. To those who quietly listened to my software complaints. To those who worked throughout the nights with me. To those who helped me write what I couldn't say. I cannot thank you enough.
}

\declaration{
I, Stefan Collier, declare that this dissertation and the work presented in it are my own and has been generated by me as the result of my own original research.\\
I confirm that:\\
1. This work was done wholly or mainly while in candidature for a degree at this University;\\
2. Where any part of this dissertation has previously been submitted for any other qualification at this University or any other institution, this has been clearly stated;\\
3. Where I have consulted the published work of others, this is always clearly attributed;\\
4. Where I have quoted from the work of others, the source is always given. With the exception of such quotations, this dissertation is entirely my own work;\\
5. I have acknowledged all main sources of help;\\
6. Where the thesis is based on work done by myself jointly with others, I have made clear exactly what was done by others and what I have contributed myself;\\
7. Either none of this work has been published before submission, or parts of this work have been published by :\\
\\
Stefan Collier\\
April 2016
}
\tableofcontents
\listoffigures
\listoftables

\mainmatter
%% ----------------------------------------------------------------
%\include{Introduction}
%\include{Conclusions}
\include{chapters/1Project/main}
\include{chapters/2Lit/main}
\include{chapters/3Design/HighLevel}
\include{chapters/3Design/InDepth}
\include{chapters/4Impl/main}

\include{chapters/5Experiments/1/main}
\include{chapters/5Experiments/2/main}
\include{chapters/5Experiments/3/main}
\include{chapters/5Experiments/4/main}

\include{chapters/6Conclusion/main}

\appendix
\include{appendix/AppendixB}
\include{appendix/D/main}
\include{appendix/AppendixC}

\backmatter
\bibliographystyle{ecs}
\bibliography{ECS}
\end{document}
%% ----------------------------------------------------------------

 %% ----------------------------------------------------------------
%% Progress.tex
%% ---------------------------------------------------------------- 
\documentclass{ecsprogress}    % Use the progress Style
\graphicspath{{../figs/}}   % Location of your graphics files
    \usepackage{natbib}            % Use Natbib style for the refs.
\hypersetup{colorlinks=true}   % Set to false for black/white printing
\input{Definitions}            % Include your abbreviations



\usepackage{enumitem}% http://ctan.org/pkg/enumitem
\usepackage{multirow}
\usepackage{float}
\usepackage{amsmath}
\usepackage{multicol}
\usepackage{amssymb}
\usepackage[normalem]{ulem}
\useunder{\uline}{\ul}{}
\usepackage{wrapfig}


\usepackage[table,xcdraw]{xcolor}


%% ----------------------------------------------------------------
\begin{document}
\frontmatter
\title      {Heterogeneous Agent-based Model for Supermarket Competition}
\authors    {\texorpdfstring
             {\href{mailto:sc22g13@ecs.soton.ac.uk}{Stefan J. Collier}}
             {Stefan J. Collier}
            }
\addresses  {\groupname\\\deptname\\\univname}
\date       {\today}
\subject    {}
\keywords   {}
\supervisor {Dr. Maria Polukarov}
\examiner   {Professor Sheng Chen}

\maketitle
\begin{abstract}
This project aim was to model and analyse the effects of competitive pricing behaviors of grocery retailers on the British market. 

This was achieved by creating a multi-agent model, containing retailer and consumer agents. The heterogeneous crowd of retailers employs either a uniform pricing strategy or a ‘local price flexing’ strategy. The actions of these retailers are chosen by predicting the profit of each action, using a perceptron. Following on from the consideration of different economic models, a discrete model was developed so that software agents have a discrete environment to operate within. Within the model, it has been observed how supermarkets with differing behaviors affect a heterogeneous crowd of consumer agents. The model was implemented in Java with Python used to evaluate the results. 

The simulation displays good acceptance with real grocery market behavior, i.e. captures the performance of British retailers thus can be used to determine the impact of changes in their behavior on their competitors and consumers.Furthermore it can be used to provide insight into sustainability of volatile pricing strategies, providing a useful insight in volatility of British supermarket retail industry. 
\end{abstract}
\acknowledgements{
I would like to express my sincere gratitude to Dr Maria Polukarov for her guidance and support which provided me the freedom to take this research in the direction of my interest.\\
\\
I would also like to thank my family and friends for their encouragement and support. To those who quietly listened to my software complaints. To those who worked throughout the nights with me. To those who helped me write what I couldn't say. I cannot thank you enough.
}

\declaration{
I, Stefan Collier, declare that this dissertation and the work presented in it are my own and has been generated by me as the result of my own original research.\\
I confirm that:\\
1. This work was done wholly or mainly while in candidature for a degree at this University;\\
2. Where any part of this dissertation has previously been submitted for any other qualification at this University or any other institution, this has been clearly stated;\\
3. Where I have consulted the published work of others, this is always clearly attributed;\\
4. Where I have quoted from the work of others, the source is always given. With the exception of such quotations, this dissertation is entirely my own work;\\
5. I have acknowledged all main sources of help;\\
6. Where the thesis is based on work done by myself jointly with others, I have made clear exactly what was done by others and what I have contributed myself;\\
7. Either none of this work has been published before submission, or parts of this work have been published by :\\
\\
Stefan Collier\\
April 2016
}
\tableofcontents
\listoffigures
\listoftables

\mainmatter
%% ----------------------------------------------------------------
%\include{Introduction}
%\include{Conclusions}
\include{chapters/1Project/main}
\include{chapters/2Lit/main}
\include{chapters/3Design/HighLevel}
\include{chapters/3Design/InDepth}
\include{chapters/4Impl/main}

\include{chapters/5Experiments/1/main}
\include{chapters/5Experiments/2/main}
\include{chapters/5Experiments/3/main}
\include{chapters/5Experiments/4/main}

\include{chapters/6Conclusion/main}

\appendix
\include{appendix/AppendixB}
\include{appendix/D/main}
\include{appendix/AppendixC}

\backmatter
\bibliographystyle{ecs}
\bibliography{ECS}
\end{document}
%% ----------------------------------------------------------------

 %% ----------------------------------------------------------------
%% Progress.tex
%% ---------------------------------------------------------------- 
\documentclass{ecsprogress}    % Use the progress Style
\graphicspath{{../figs/}}   % Location of your graphics files
    \usepackage{natbib}            % Use Natbib style for the refs.
\hypersetup{colorlinks=true}   % Set to false for black/white printing
\input{Definitions}            % Include your abbreviations



\usepackage{enumitem}% http://ctan.org/pkg/enumitem
\usepackage{multirow}
\usepackage{float}
\usepackage{amsmath}
\usepackage{multicol}
\usepackage{amssymb}
\usepackage[normalem]{ulem}
\useunder{\uline}{\ul}{}
\usepackage{wrapfig}


\usepackage[table,xcdraw]{xcolor}


%% ----------------------------------------------------------------
\begin{document}
\frontmatter
\title      {Heterogeneous Agent-based Model for Supermarket Competition}
\authors    {\texorpdfstring
             {\href{mailto:sc22g13@ecs.soton.ac.uk}{Stefan J. Collier}}
             {Stefan J. Collier}
            }
\addresses  {\groupname\\\deptname\\\univname}
\date       {\today}
\subject    {}
\keywords   {}
\supervisor {Dr. Maria Polukarov}
\examiner   {Professor Sheng Chen}

\maketitle
\begin{abstract}
This project aim was to model and analyse the effects of competitive pricing behaviors of grocery retailers on the British market. 

This was achieved by creating a multi-agent model, containing retailer and consumer agents. The heterogeneous crowd of retailers employs either a uniform pricing strategy or a ‘local price flexing’ strategy. The actions of these retailers are chosen by predicting the profit of each action, using a perceptron. Following on from the consideration of different economic models, a discrete model was developed so that software agents have a discrete environment to operate within. Within the model, it has been observed how supermarkets with differing behaviors affect a heterogeneous crowd of consumer agents. The model was implemented in Java with Python used to evaluate the results. 

The simulation displays good acceptance with real grocery market behavior, i.e. captures the performance of British retailers thus can be used to determine the impact of changes in their behavior on their competitors and consumers.Furthermore it can be used to provide insight into sustainability of volatile pricing strategies, providing a useful insight in volatility of British supermarket retail industry. 
\end{abstract}
\acknowledgements{
I would like to express my sincere gratitude to Dr Maria Polukarov for her guidance and support which provided me the freedom to take this research in the direction of my interest.\\
\\
I would also like to thank my family and friends for their encouragement and support. To those who quietly listened to my software complaints. To those who worked throughout the nights with me. To those who helped me write what I couldn't say. I cannot thank you enough.
}

\declaration{
I, Stefan Collier, declare that this dissertation and the work presented in it are my own and has been generated by me as the result of my own original research.\\
I confirm that:\\
1. This work was done wholly or mainly while in candidature for a degree at this University;\\
2. Where any part of this dissertation has previously been submitted for any other qualification at this University or any other institution, this has been clearly stated;\\
3. Where I have consulted the published work of others, this is always clearly attributed;\\
4. Where I have quoted from the work of others, the source is always given. With the exception of such quotations, this dissertation is entirely my own work;\\
5. I have acknowledged all main sources of help;\\
6. Where the thesis is based on work done by myself jointly with others, I have made clear exactly what was done by others and what I have contributed myself;\\
7. Either none of this work has been published before submission, or parts of this work have been published by :\\
\\
Stefan Collier\\
April 2016
}
\tableofcontents
\listoffigures
\listoftables

\mainmatter
%% ----------------------------------------------------------------
%\include{Introduction}
%\include{Conclusions}
\include{chapters/1Project/main}
\include{chapters/2Lit/main}
\include{chapters/3Design/HighLevel}
\include{chapters/3Design/InDepth}
\include{chapters/4Impl/main}

\include{chapters/5Experiments/1/main}
\include{chapters/5Experiments/2/main}
\include{chapters/5Experiments/3/main}
\include{chapters/5Experiments/4/main}

\include{chapters/6Conclusion/main}

\appendix
\include{appendix/AppendixB}
\include{appendix/D/main}
\include{appendix/AppendixC}

\backmatter
\bibliographystyle{ecs}
\bibliography{ECS}
\end{document}
%% ----------------------------------------------------------------


 %% ----------------------------------------------------------------
%% Progress.tex
%% ---------------------------------------------------------------- 
\documentclass{ecsprogress}    % Use the progress Style
\graphicspath{{../figs/}}   % Location of your graphics files
    \usepackage{natbib}            % Use Natbib style for the refs.
\hypersetup{colorlinks=true}   % Set to false for black/white printing
\input{Definitions}            % Include your abbreviations



\usepackage{enumitem}% http://ctan.org/pkg/enumitem
\usepackage{multirow}
\usepackage{float}
\usepackage{amsmath}
\usepackage{multicol}
\usepackage{amssymb}
\usepackage[normalem]{ulem}
\useunder{\uline}{\ul}{}
\usepackage{wrapfig}


\usepackage[table,xcdraw]{xcolor}


%% ----------------------------------------------------------------
\begin{document}
\frontmatter
\title      {Heterogeneous Agent-based Model for Supermarket Competition}
\authors    {\texorpdfstring
             {\href{mailto:sc22g13@ecs.soton.ac.uk}{Stefan J. Collier}}
             {Stefan J. Collier}
            }
\addresses  {\groupname\\\deptname\\\univname}
\date       {\today}
\subject    {}
\keywords   {}
\supervisor {Dr. Maria Polukarov}
\examiner   {Professor Sheng Chen}

\maketitle
\begin{abstract}
This project aim was to model and analyse the effects of competitive pricing behaviors of grocery retailers on the British market. 

This was achieved by creating a multi-agent model, containing retailer and consumer agents. The heterogeneous crowd of retailers employs either a uniform pricing strategy or a ‘local price flexing’ strategy. The actions of these retailers are chosen by predicting the profit of each action, using a perceptron. Following on from the consideration of different economic models, a discrete model was developed so that software agents have a discrete environment to operate within. Within the model, it has been observed how supermarkets with differing behaviors affect a heterogeneous crowd of consumer agents. The model was implemented in Java with Python used to evaluate the results. 

The simulation displays good acceptance with real grocery market behavior, i.e. captures the performance of British retailers thus can be used to determine the impact of changes in their behavior on their competitors and consumers.Furthermore it can be used to provide insight into sustainability of volatile pricing strategies, providing a useful insight in volatility of British supermarket retail industry. 
\end{abstract}
\acknowledgements{
I would like to express my sincere gratitude to Dr Maria Polukarov for her guidance and support which provided me the freedom to take this research in the direction of my interest.\\
\\
I would also like to thank my family and friends for their encouragement and support. To those who quietly listened to my software complaints. To those who worked throughout the nights with me. To those who helped me write what I couldn't say. I cannot thank you enough.
}

\declaration{
I, Stefan Collier, declare that this dissertation and the work presented in it are my own and has been generated by me as the result of my own original research.\\
I confirm that:\\
1. This work was done wholly or mainly while in candidature for a degree at this University;\\
2. Where any part of this dissertation has previously been submitted for any other qualification at this University or any other institution, this has been clearly stated;\\
3. Where I have consulted the published work of others, this is always clearly attributed;\\
4. Where I have quoted from the work of others, the source is always given. With the exception of such quotations, this dissertation is entirely my own work;\\
5. I have acknowledged all main sources of help;\\
6. Where the thesis is based on work done by myself jointly with others, I have made clear exactly what was done by others and what I have contributed myself;\\
7. Either none of this work has been published before submission, or parts of this work have been published by :\\
\\
Stefan Collier\\
April 2016
}
\tableofcontents
\listoffigures
\listoftables

\mainmatter
%% ----------------------------------------------------------------
%\include{Introduction}
%\include{Conclusions}
\include{chapters/1Project/main}
\include{chapters/2Lit/main}
\include{chapters/3Design/HighLevel}
\include{chapters/3Design/InDepth}
\include{chapters/4Impl/main}

\include{chapters/5Experiments/1/main}
\include{chapters/5Experiments/2/main}
\include{chapters/5Experiments/3/main}
\include{chapters/5Experiments/4/main}

\include{chapters/6Conclusion/main}

\appendix
\include{appendix/AppendixB}
\include{appendix/D/main}
\include{appendix/AppendixC}

\backmatter
\bibliographystyle{ecs}
\bibliography{ECS}
\end{document}
%% ----------------------------------------------------------------


\appendix
\include{appendix/AppendixB}
 %% ----------------------------------------------------------------
%% Progress.tex
%% ---------------------------------------------------------------- 
\documentclass{ecsprogress}    % Use the progress Style
\graphicspath{{../figs/}}   % Location of your graphics files
    \usepackage{natbib}            % Use Natbib style for the refs.
\hypersetup{colorlinks=true}   % Set to false for black/white printing
\input{Definitions}            % Include your abbreviations



\usepackage{enumitem}% http://ctan.org/pkg/enumitem
\usepackage{multirow}
\usepackage{float}
\usepackage{amsmath}
\usepackage{multicol}
\usepackage{amssymb}
\usepackage[normalem]{ulem}
\useunder{\uline}{\ul}{}
\usepackage{wrapfig}


\usepackage[table,xcdraw]{xcolor}


%% ----------------------------------------------------------------
\begin{document}
\frontmatter
\title      {Heterogeneous Agent-based Model for Supermarket Competition}
\authors    {\texorpdfstring
             {\href{mailto:sc22g13@ecs.soton.ac.uk}{Stefan J. Collier}}
             {Stefan J. Collier}
            }
\addresses  {\groupname\\\deptname\\\univname}
\date       {\today}
\subject    {}
\keywords   {}
\supervisor {Dr. Maria Polukarov}
\examiner   {Professor Sheng Chen}

\maketitle
\begin{abstract}
This project aim was to model and analyse the effects of competitive pricing behaviors of grocery retailers on the British market. 

This was achieved by creating a multi-agent model, containing retailer and consumer agents. The heterogeneous crowd of retailers employs either a uniform pricing strategy or a ‘local price flexing’ strategy. The actions of these retailers are chosen by predicting the profit of each action, using a perceptron. Following on from the consideration of different economic models, a discrete model was developed so that software agents have a discrete environment to operate within. Within the model, it has been observed how supermarkets with differing behaviors affect a heterogeneous crowd of consumer agents. The model was implemented in Java with Python used to evaluate the results. 

The simulation displays good acceptance with real grocery market behavior, i.e. captures the performance of British retailers thus can be used to determine the impact of changes in their behavior on their competitors and consumers.Furthermore it can be used to provide insight into sustainability of volatile pricing strategies, providing a useful insight in volatility of British supermarket retail industry. 
\end{abstract}
\acknowledgements{
I would like to express my sincere gratitude to Dr Maria Polukarov for her guidance and support which provided me the freedom to take this research in the direction of my interest.\\
\\
I would also like to thank my family and friends for their encouragement and support. To those who quietly listened to my software complaints. To those who worked throughout the nights with me. To those who helped me write what I couldn't say. I cannot thank you enough.
}

\declaration{
I, Stefan Collier, declare that this dissertation and the work presented in it are my own and has been generated by me as the result of my own original research.\\
I confirm that:\\
1. This work was done wholly or mainly while in candidature for a degree at this University;\\
2. Where any part of this dissertation has previously been submitted for any other qualification at this University or any other institution, this has been clearly stated;\\
3. Where I have consulted the published work of others, this is always clearly attributed;\\
4. Where I have quoted from the work of others, the source is always given. With the exception of such quotations, this dissertation is entirely my own work;\\
5. I have acknowledged all main sources of help;\\
6. Where the thesis is based on work done by myself jointly with others, I have made clear exactly what was done by others and what I have contributed myself;\\
7. Either none of this work has been published before submission, or parts of this work have been published by :\\
\\
Stefan Collier\\
April 2016
}
\tableofcontents
\listoffigures
\listoftables

\mainmatter
%% ----------------------------------------------------------------
%\include{Introduction}
%\include{Conclusions}
\include{chapters/1Project/main}
\include{chapters/2Lit/main}
\include{chapters/3Design/HighLevel}
\include{chapters/3Design/InDepth}
\include{chapters/4Impl/main}

\include{chapters/5Experiments/1/main}
\include{chapters/5Experiments/2/main}
\include{chapters/5Experiments/3/main}
\include{chapters/5Experiments/4/main}

\include{chapters/6Conclusion/main}

\appendix
\include{appendix/AppendixB}
\include{appendix/D/main}
\include{appendix/AppendixC}

\backmatter
\bibliographystyle{ecs}
\bibliography{ECS}
\end{document}
%% ----------------------------------------------------------------

\include{appendix/AppendixC}

\backmatter
\bibliographystyle{ecs}
\bibliography{ECS}
\end{document}
%% ----------------------------------------------------------------

 %% ----------------------------------------------------------------
%% Progress.tex
%% ---------------------------------------------------------------- 
\documentclass{ecsprogress}    % Use the progress Style
\graphicspath{{../figs/}}   % Location of your graphics files
    \usepackage{natbib}            % Use Natbib style for the refs.
\hypersetup{colorlinks=true}   % Set to false for black/white printing
\input{Definitions}            % Include your abbreviations



\usepackage{enumitem}% http://ctan.org/pkg/enumitem
\usepackage{multirow}
\usepackage{float}
\usepackage{amsmath}
\usepackage{multicol}
\usepackage{amssymb}
\usepackage[normalem]{ulem}
\useunder{\uline}{\ul}{}
\usepackage{wrapfig}


\usepackage[table,xcdraw]{xcolor}


%% ----------------------------------------------------------------
\begin{document}
\frontmatter
\title      {Heterogeneous Agent-based Model for Supermarket Competition}
\authors    {\texorpdfstring
             {\href{mailto:sc22g13@ecs.soton.ac.uk}{Stefan J. Collier}}
             {Stefan J. Collier}
            }
\addresses  {\groupname\\\deptname\\\univname}
\date       {\today}
\subject    {}
\keywords   {}
\supervisor {Dr. Maria Polukarov}
\examiner   {Professor Sheng Chen}

\maketitle
\begin{abstract}
This project aim was to model and analyse the effects of competitive pricing behaviors of grocery retailers on the British market. 

This was achieved by creating a multi-agent model, containing retailer and consumer agents. The heterogeneous crowd of retailers employs either a uniform pricing strategy or a ‘local price flexing’ strategy. The actions of these retailers are chosen by predicting the profit of each action, using a perceptron. Following on from the consideration of different economic models, a discrete model was developed so that software agents have a discrete environment to operate within. Within the model, it has been observed how supermarkets with differing behaviors affect a heterogeneous crowd of consumer agents. The model was implemented in Java with Python used to evaluate the results. 

The simulation displays good acceptance with real grocery market behavior, i.e. captures the performance of British retailers thus can be used to determine the impact of changes in their behavior on their competitors and consumers.Furthermore it can be used to provide insight into sustainability of volatile pricing strategies, providing a useful insight in volatility of British supermarket retail industry. 
\end{abstract}
\acknowledgements{
I would like to express my sincere gratitude to Dr Maria Polukarov for her guidance and support which provided me the freedom to take this research in the direction of my interest.\\
\\
I would also like to thank my family and friends for their encouragement and support. To those who quietly listened to my software complaints. To those who worked throughout the nights with me. To those who helped me write what I couldn't say. I cannot thank you enough.
}

\declaration{
I, Stefan Collier, declare that this dissertation and the work presented in it are my own and has been generated by me as the result of my own original research.\\
I confirm that:\\
1. This work was done wholly or mainly while in candidature for a degree at this University;\\
2. Where any part of this dissertation has previously been submitted for any other qualification at this University or any other institution, this has been clearly stated;\\
3. Where I have consulted the published work of others, this is always clearly attributed;\\
4. Where I have quoted from the work of others, the source is always given. With the exception of such quotations, this dissertation is entirely my own work;\\
5. I have acknowledged all main sources of help;\\
6. Where the thesis is based on work done by myself jointly with others, I have made clear exactly what was done by others and what I have contributed myself;\\
7. Either none of this work has been published before submission, or parts of this work have been published by :\\
\\
Stefan Collier\\
April 2016
}
\tableofcontents
\listoffigures
\listoftables

\mainmatter
%% ----------------------------------------------------------------
%\include{Introduction}
%\include{Conclusions}
 %% ----------------------------------------------------------------
%% Progress.tex
%% ---------------------------------------------------------------- 
\documentclass{ecsprogress}    % Use the progress Style
\graphicspath{{../figs/}}   % Location of your graphics files
    \usepackage{natbib}            % Use Natbib style for the refs.
\hypersetup{colorlinks=true}   % Set to false for black/white printing
\input{Definitions}            % Include your abbreviations



\usepackage{enumitem}% http://ctan.org/pkg/enumitem
\usepackage{multirow}
\usepackage{float}
\usepackage{amsmath}
\usepackage{multicol}
\usepackage{amssymb}
\usepackage[normalem]{ulem}
\useunder{\uline}{\ul}{}
\usepackage{wrapfig}


\usepackage[table,xcdraw]{xcolor}


%% ----------------------------------------------------------------
\begin{document}
\frontmatter
\title      {Heterogeneous Agent-based Model for Supermarket Competition}
\authors    {\texorpdfstring
             {\href{mailto:sc22g13@ecs.soton.ac.uk}{Stefan J. Collier}}
             {Stefan J. Collier}
            }
\addresses  {\groupname\\\deptname\\\univname}
\date       {\today}
\subject    {}
\keywords   {}
\supervisor {Dr. Maria Polukarov}
\examiner   {Professor Sheng Chen}

\maketitle
\begin{abstract}
This project aim was to model and analyse the effects of competitive pricing behaviors of grocery retailers on the British market. 

This was achieved by creating a multi-agent model, containing retailer and consumer agents. The heterogeneous crowd of retailers employs either a uniform pricing strategy or a ‘local price flexing’ strategy. The actions of these retailers are chosen by predicting the profit of each action, using a perceptron. Following on from the consideration of different economic models, a discrete model was developed so that software agents have a discrete environment to operate within. Within the model, it has been observed how supermarkets with differing behaviors affect a heterogeneous crowd of consumer agents. The model was implemented in Java with Python used to evaluate the results. 

The simulation displays good acceptance with real grocery market behavior, i.e. captures the performance of British retailers thus can be used to determine the impact of changes in their behavior on their competitors and consumers.Furthermore it can be used to provide insight into sustainability of volatile pricing strategies, providing a useful insight in volatility of British supermarket retail industry. 
\end{abstract}
\acknowledgements{
I would like to express my sincere gratitude to Dr Maria Polukarov for her guidance and support which provided me the freedom to take this research in the direction of my interest.\\
\\
I would also like to thank my family and friends for their encouragement and support. To those who quietly listened to my software complaints. To those who worked throughout the nights with me. To those who helped me write what I couldn't say. I cannot thank you enough.
}

\declaration{
I, Stefan Collier, declare that this dissertation and the work presented in it are my own and has been generated by me as the result of my own original research.\\
I confirm that:\\
1. This work was done wholly or mainly while in candidature for a degree at this University;\\
2. Where any part of this dissertation has previously been submitted for any other qualification at this University or any other institution, this has been clearly stated;\\
3. Where I have consulted the published work of others, this is always clearly attributed;\\
4. Where I have quoted from the work of others, the source is always given. With the exception of such quotations, this dissertation is entirely my own work;\\
5. I have acknowledged all main sources of help;\\
6. Where the thesis is based on work done by myself jointly with others, I have made clear exactly what was done by others and what I have contributed myself;\\
7. Either none of this work has been published before submission, or parts of this work have been published by :\\
\\
Stefan Collier\\
April 2016
}
\tableofcontents
\listoffigures
\listoftables

\mainmatter
%% ----------------------------------------------------------------
%\include{Introduction}
%\include{Conclusions}
\include{chapters/1Project/main}
\include{chapters/2Lit/main}
\include{chapters/3Design/HighLevel}
\include{chapters/3Design/InDepth}
\include{chapters/4Impl/main}

\include{chapters/5Experiments/1/main}
\include{chapters/5Experiments/2/main}
\include{chapters/5Experiments/3/main}
\include{chapters/5Experiments/4/main}

\include{chapters/6Conclusion/main}

\appendix
\include{appendix/AppendixB}
\include{appendix/D/main}
\include{appendix/AppendixC}

\backmatter
\bibliographystyle{ecs}
\bibliography{ECS}
\end{document}
%% ----------------------------------------------------------------

 %% ----------------------------------------------------------------
%% Progress.tex
%% ---------------------------------------------------------------- 
\documentclass{ecsprogress}    % Use the progress Style
\graphicspath{{../figs/}}   % Location of your graphics files
    \usepackage{natbib}            % Use Natbib style for the refs.
\hypersetup{colorlinks=true}   % Set to false for black/white printing
\input{Definitions}            % Include your abbreviations



\usepackage{enumitem}% http://ctan.org/pkg/enumitem
\usepackage{multirow}
\usepackage{float}
\usepackage{amsmath}
\usepackage{multicol}
\usepackage{amssymb}
\usepackage[normalem]{ulem}
\useunder{\uline}{\ul}{}
\usepackage{wrapfig}


\usepackage[table,xcdraw]{xcolor}


%% ----------------------------------------------------------------
\begin{document}
\frontmatter
\title      {Heterogeneous Agent-based Model for Supermarket Competition}
\authors    {\texorpdfstring
             {\href{mailto:sc22g13@ecs.soton.ac.uk}{Stefan J. Collier}}
             {Stefan J. Collier}
            }
\addresses  {\groupname\\\deptname\\\univname}
\date       {\today}
\subject    {}
\keywords   {}
\supervisor {Dr. Maria Polukarov}
\examiner   {Professor Sheng Chen}

\maketitle
\begin{abstract}
This project aim was to model and analyse the effects of competitive pricing behaviors of grocery retailers on the British market. 

This was achieved by creating a multi-agent model, containing retailer and consumer agents. The heterogeneous crowd of retailers employs either a uniform pricing strategy or a ‘local price flexing’ strategy. The actions of these retailers are chosen by predicting the profit of each action, using a perceptron. Following on from the consideration of different economic models, a discrete model was developed so that software agents have a discrete environment to operate within. Within the model, it has been observed how supermarkets with differing behaviors affect a heterogeneous crowd of consumer agents. The model was implemented in Java with Python used to evaluate the results. 

The simulation displays good acceptance with real grocery market behavior, i.e. captures the performance of British retailers thus can be used to determine the impact of changes in their behavior on their competitors and consumers.Furthermore it can be used to provide insight into sustainability of volatile pricing strategies, providing a useful insight in volatility of British supermarket retail industry. 
\end{abstract}
\acknowledgements{
I would like to express my sincere gratitude to Dr Maria Polukarov for her guidance and support which provided me the freedom to take this research in the direction of my interest.\\
\\
I would also like to thank my family and friends for their encouragement and support. To those who quietly listened to my software complaints. To those who worked throughout the nights with me. To those who helped me write what I couldn't say. I cannot thank you enough.
}

\declaration{
I, Stefan Collier, declare that this dissertation and the work presented in it are my own and has been generated by me as the result of my own original research.\\
I confirm that:\\
1. This work was done wholly or mainly while in candidature for a degree at this University;\\
2. Where any part of this dissertation has previously been submitted for any other qualification at this University or any other institution, this has been clearly stated;\\
3. Where I have consulted the published work of others, this is always clearly attributed;\\
4. Where I have quoted from the work of others, the source is always given. With the exception of such quotations, this dissertation is entirely my own work;\\
5. I have acknowledged all main sources of help;\\
6. Where the thesis is based on work done by myself jointly with others, I have made clear exactly what was done by others and what I have contributed myself;\\
7. Either none of this work has been published before submission, or parts of this work have been published by :\\
\\
Stefan Collier\\
April 2016
}
\tableofcontents
\listoffigures
\listoftables

\mainmatter
%% ----------------------------------------------------------------
%\include{Introduction}
%\include{Conclusions}
\include{chapters/1Project/main}
\include{chapters/2Lit/main}
\include{chapters/3Design/HighLevel}
\include{chapters/3Design/InDepth}
\include{chapters/4Impl/main}

\include{chapters/5Experiments/1/main}
\include{chapters/5Experiments/2/main}
\include{chapters/5Experiments/3/main}
\include{chapters/5Experiments/4/main}

\include{chapters/6Conclusion/main}

\appendix
\include{appendix/AppendixB}
\include{appendix/D/main}
\include{appendix/AppendixC}

\backmatter
\bibliographystyle{ecs}
\bibliography{ECS}
\end{document}
%% ----------------------------------------------------------------

\include{chapters/3Design/HighLevel}
\include{chapters/3Design/InDepth}
 %% ----------------------------------------------------------------
%% Progress.tex
%% ---------------------------------------------------------------- 
\documentclass{ecsprogress}    % Use the progress Style
\graphicspath{{../figs/}}   % Location of your graphics files
    \usepackage{natbib}            % Use Natbib style for the refs.
\hypersetup{colorlinks=true}   % Set to false for black/white printing
\input{Definitions}            % Include your abbreviations



\usepackage{enumitem}% http://ctan.org/pkg/enumitem
\usepackage{multirow}
\usepackage{float}
\usepackage{amsmath}
\usepackage{multicol}
\usepackage{amssymb}
\usepackage[normalem]{ulem}
\useunder{\uline}{\ul}{}
\usepackage{wrapfig}


\usepackage[table,xcdraw]{xcolor}


%% ----------------------------------------------------------------
\begin{document}
\frontmatter
\title      {Heterogeneous Agent-based Model for Supermarket Competition}
\authors    {\texorpdfstring
             {\href{mailto:sc22g13@ecs.soton.ac.uk}{Stefan J. Collier}}
             {Stefan J. Collier}
            }
\addresses  {\groupname\\\deptname\\\univname}
\date       {\today}
\subject    {}
\keywords   {}
\supervisor {Dr. Maria Polukarov}
\examiner   {Professor Sheng Chen}

\maketitle
\begin{abstract}
This project aim was to model and analyse the effects of competitive pricing behaviors of grocery retailers on the British market. 

This was achieved by creating a multi-agent model, containing retailer and consumer agents. The heterogeneous crowd of retailers employs either a uniform pricing strategy or a ‘local price flexing’ strategy. The actions of these retailers are chosen by predicting the profit of each action, using a perceptron. Following on from the consideration of different economic models, a discrete model was developed so that software agents have a discrete environment to operate within. Within the model, it has been observed how supermarkets with differing behaviors affect a heterogeneous crowd of consumer agents. The model was implemented in Java with Python used to evaluate the results. 

The simulation displays good acceptance with real grocery market behavior, i.e. captures the performance of British retailers thus can be used to determine the impact of changes in their behavior on their competitors and consumers.Furthermore it can be used to provide insight into sustainability of volatile pricing strategies, providing a useful insight in volatility of British supermarket retail industry. 
\end{abstract}
\acknowledgements{
I would like to express my sincere gratitude to Dr Maria Polukarov for her guidance and support which provided me the freedom to take this research in the direction of my interest.\\
\\
I would also like to thank my family and friends for their encouragement and support. To those who quietly listened to my software complaints. To those who worked throughout the nights with me. To those who helped me write what I couldn't say. I cannot thank you enough.
}

\declaration{
I, Stefan Collier, declare that this dissertation and the work presented in it are my own and has been generated by me as the result of my own original research.\\
I confirm that:\\
1. This work was done wholly or mainly while in candidature for a degree at this University;\\
2. Where any part of this dissertation has previously been submitted for any other qualification at this University or any other institution, this has been clearly stated;\\
3. Where I have consulted the published work of others, this is always clearly attributed;\\
4. Where I have quoted from the work of others, the source is always given. With the exception of such quotations, this dissertation is entirely my own work;\\
5. I have acknowledged all main sources of help;\\
6. Where the thesis is based on work done by myself jointly with others, I have made clear exactly what was done by others and what I have contributed myself;\\
7. Either none of this work has been published before submission, or parts of this work have been published by :\\
\\
Stefan Collier\\
April 2016
}
\tableofcontents
\listoffigures
\listoftables

\mainmatter
%% ----------------------------------------------------------------
%\include{Introduction}
%\include{Conclusions}
\include{chapters/1Project/main}
\include{chapters/2Lit/main}
\include{chapters/3Design/HighLevel}
\include{chapters/3Design/InDepth}
\include{chapters/4Impl/main}

\include{chapters/5Experiments/1/main}
\include{chapters/5Experiments/2/main}
\include{chapters/5Experiments/3/main}
\include{chapters/5Experiments/4/main}

\include{chapters/6Conclusion/main}

\appendix
\include{appendix/AppendixB}
\include{appendix/D/main}
\include{appendix/AppendixC}

\backmatter
\bibliographystyle{ecs}
\bibliography{ECS}
\end{document}
%% ----------------------------------------------------------------


 %% ----------------------------------------------------------------
%% Progress.tex
%% ---------------------------------------------------------------- 
\documentclass{ecsprogress}    % Use the progress Style
\graphicspath{{../figs/}}   % Location of your graphics files
    \usepackage{natbib}            % Use Natbib style for the refs.
\hypersetup{colorlinks=true}   % Set to false for black/white printing
\input{Definitions}            % Include your abbreviations



\usepackage{enumitem}% http://ctan.org/pkg/enumitem
\usepackage{multirow}
\usepackage{float}
\usepackage{amsmath}
\usepackage{multicol}
\usepackage{amssymb}
\usepackage[normalem]{ulem}
\useunder{\uline}{\ul}{}
\usepackage{wrapfig}


\usepackage[table,xcdraw]{xcolor}


%% ----------------------------------------------------------------
\begin{document}
\frontmatter
\title      {Heterogeneous Agent-based Model for Supermarket Competition}
\authors    {\texorpdfstring
             {\href{mailto:sc22g13@ecs.soton.ac.uk}{Stefan J. Collier}}
             {Stefan J. Collier}
            }
\addresses  {\groupname\\\deptname\\\univname}
\date       {\today}
\subject    {}
\keywords   {}
\supervisor {Dr. Maria Polukarov}
\examiner   {Professor Sheng Chen}

\maketitle
\begin{abstract}
This project aim was to model and analyse the effects of competitive pricing behaviors of grocery retailers on the British market. 

This was achieved by creating a multi-agent model, containing retailer and consumer agents. The heterogeneous crowd of retailers employs either a uniform pricing strategy or a ‘local price flexing’ strategy. The actions of these retailers are chosen by predicting the profit of each action, using a perceptron. Following on from the consideration of different economic models, a discrete model was developed so that software agents have a discrete environment to operate within. Within the model, it has been observed how supermarkets with differing behaviors affect a heterogeneous crowd of consumer agents. The model was implemented in Java with Python used to evaluate the results. 

The simulation displays good acceptance with real grocery market behavior, i.e. captures the performance of British retailers thus can be used to determine the impact of changes in their behavior on their competitors and consumers.Furthermore it can be used to provide insight into sustainability of volatile pricing strategies, providing a useful insight in volatility of British supermarket retail industry. 
\end{abstract}
\acknowledgements{
I would like to express my sincere gratitude to Dr Maria Polukarov for her guidance and support which provided me the freedom to take this research in the direction of my interest.\\
\\
I would also like to thank my family and friends for their encouragement and support. To those who quietly listened to my software complaints. To those who worked throughout the nights with me. To those who helped me write what I couldn't say. I cannot thank you enough.
}

\declaration{
I, Stefan Collier, declare that this dissertation and the work presented in it are my own and has been generated by me as the result of my own original research.\\
I confirm that:\\
1. This work was done wholly or mainly while in candidature for a degree at this University;\\
2. Where any part of this dissertation has previously been submitted for any other qualification at this University or any other institution, this has been clearly stated;\\
3. Where I have consulted the published work of others, this is always clearly attributed;\\
4. Where I have quoted from the work of others, the source is always given. With the exception of such quotations, this dissertation is entirely my own work;\\
5. I have acknowledged all main sources of help;\\
6. Where the thesis is based on work done by myself jointly with others, I have made clear exactly what was done by others and what I have contributed myself;\\
7. Either none of this work has been published before submission, or parts of this work have been published by :\\
\\
Stefan Collier\\
April 2016
}
\tableofcontents
\listoffigures
\listoftables

\mainmatter
%% ----------------------------------------------------------------
%\include{Introduction}
%\include{Conclusions}
\include{chapters/1Project/main}
\include{chapters/2Lit/main}
\include{chapters/3Design/HighLevel}
\include{chapters/3Design/InDepth}
\include{chapters/4Impl/main}

\include{chapters/5Experiments/1/main}
\include{chapters/5Experiments/2/main}
\include{chapters/5Experiments/3/main}
\include{chapters/5Experiments/4/main}

\include{chapters/6Conclusion/main}

\appendix
\include{appendix/AppendixB}
\include{appendix/D/main}
\include{appendix/AppendixC}

\backmatter
\bibliographystyle{ecs}
\bibliography{ECS}
\end{document}
%% ----------------------------------------------------------------

 %% ----------------------------------------------------------------
%% Progress.tex
%% ---------------------------------------------------------------- 
\documentclass{ecsprogress}    % Use the progress Style
\graphicspath{{../figs/}}   % Location of your graphics files
    \usepackage{natbib}            % Use Natbib style for the refs.
\hypersetup{colorlinks=true}   % Set to false for black/white printing
\input{Definitions}            % Include your abbreviations



\usepackage{enumitem}% http://ctan.org/pkg/enumitem
\usepackage{multirow}
\usepackage{float}
\usepackage{amsmath}
\usepackage{multicol}
\usepackage{amssymb}
\usepackage[normalem]{ulem}
\useunder{\uline}{\ul}{}
\usepackage{wrapfig}


\usepackage[table,xcdraw]{xcolor}


%% ----------------------------------------------------------------
\begin{document}
\frontmatter
\title      {Heterogeneous Agent-based Model for Supermarket Competition}
\authors    {\texorpdfstring
             {\href{mailto:sc22g13@ecs.soton.ac.uk}{Stefan J. Collier}}
             {Stefan J. Collier}
            }
\addresses  {\groupname\\\deptname\\\univname}
\date       {\today}
\subject    {}
\keywords   {}
\supervisor {Dr. Maria Polukarov}
\examiner   {Professor Sheng Chen}

\maketitle
\begin{abstract}
This project aim was to model and analyse the effects of competitive pricing behaviors of grocery retailers on the British market. 

This was achieved by creating a multi-agent model, containing retailer and consumer agents. The heterogeneous crowd of retailers employs either a uniform pricing strategy or a ‘local price flexing’ strategy. The actions of these retailers are chosen by predicting the profit of each action, using a perceptron. Following on from the consideration of different economic models, a discrete model was developed so that software agents have a discrete environment to operate within. Within the model, it has been observed how supermarkets with differing behaviors affect a heterogeneous crowd of consumer agents. The model was implemented in Java with Python used to evaluate the results. 

The simulation displays good acceptance with real grocery market behavior, i.e. captures the performance of British retailers thus can be used to determine the impact of changes in their behavior on their competitors and consumers.Furthermore it can be used to provide insight into sustainability of volatile pricing strategies, providing a useful insight in volatility of British supermarket retail industry. 
\end{abstract}
\acknowledgements{
I would like to express my sincere gratitude to Dr Maria Polukarov for her guidance and support which provided me the freedom to take this research in the direction of my interest.\\
\\
I would also like to thank my family and friends for their encouragement and support. To those who quietly listened to my software complaints. To those who worked throughout the nights with me. To those who helped me write what I couldn't say. I cannot thank you enough.
}

\declaration{
I, Stefan Collier, declare that this dissertation and the work presented in it are my own and has been generated by me as the result of my own original research.\\
I confirm that:\\
1. This work was done wholly or mainly while in candidature for a degree at this University;\\
2. Where any part of this dissertation has previously been submitted for any other qualification at this University or any other institution, this has been clearly stated;\\
3. Where I have consulted the published work of others, this is always clearly attributed;\\
4. Where I have quoted from the work of others, the source is always given. With the exception of such quotations, this dissertation is entirely my own work;\\
5. I have acknowledged all main sources of help;\\
6. Where the thesis is based on work done by myself jointly with others, I have made clear exactly what was done by others and what I have contributed myself;\\
7. Either none of this work has been published before submission, or parts of this work have been published by :\\
\\
Stefan Collier\\
April 2016
}
\tableofcontents
\listoffigures
\listoftables

\mainmatter
%% ----------------------------------------------------------------
%\include{Introduction}
%\include{Conclusions}
\include{chapters/1Project/main}
\include{chapters/2Lit/main}
\include{chapters/3Design/HighLevel}
\include{chapters/3Design/InDepth}
\include{chapters/4Impl/main}

\include{chapters/5Experiments/1/main}
\include{chapters/5Experiments/2/main}
\include{chapters/5Experiments/3/main}
\include{chapters/5Experiments/4/main}

\include{chapters/6Conclusion/main}

\appendix
\include{appendix/AppendixB}
\include{appendix/D/main}
\include{appendix/AppendixC}

\backmatter
\bibliographystyle{ecs}
\bibliography{ECS}
\end{document}
%% ----------------------------------------------------------------

 %% ----------------------------------------------------------------
%% Progress.tex
%% ---------------------------------------------------------------- 
\documentclass{ecsprogress}    % Use the progress Style
\graphicspath{{../figs/}}   % Location of your graphics files
    \usepackage{natbib}            % Use Natbib style for the refs.
\hypersetup{colorlinks=true}   % Set to false for black/white printing
\input{Definitions}            % Include your abbreviations



\usepackage{enumitem}% http://ctan.org/pkg/enumitem
\usepackage{multirow}
\usepackage{float}
\usepackage{amsmath}
\usepackage{multicol}
\usepackage{amssymb}
\usepackage[normalem]{ulem}
\useunder{\uline}{\ul}{}
\usepackage{wrapfig}


\usepackage[table,xcdraw]{xcolor}


%% ----------------------------------------------------------------
\begin{document}
\frontmatter
\title      {Heterogeneous Agent-based Model for Supermarket Competition}
\authors    {\texorpdfstring
             {\href{mailto:sc22g13@ecs.soton.ac.uk}{Stefan J. Collier}}
             {Stefan J. Collier}
            }
\addresses  {\groupname\\\deptname\\\univname}
\date       {\today}
\subject    {}
\keywords   {}
\supervisor {Dr. Maria Polukarov}
\examiner   {Professor Sheng Chen}

\maketitle
\begin{abstract}
This project aim was to model and analyse the effects of competitive pricing behaviors of grocery retailers on the British market. 

This was achieved by creating a multi-agent model, containing retailer and consumer agents. The heterogeneous crowd of retailers employs either a uniform pricing strategy or a ‘local price flexing’ strategy. The actions of these retailers are chosen by predicting the profit of each action, using a perceptron. Following on from the consideration of different economic models, a discrete model was developed so that software agents have a discrete environment to operate within. Within the model, it has been observed how supermarkets with differing behaviors affect a heterogeneous crowd of consumer agents. The model was implemented in Java with Python used to evaluate the results. 

The simulation displays good acceptance with real grocery market behavior, i.e. captures the performance of British retailers thus can be used to determine the impact of changes in their behavior on their competitors and consumers.Furthermore it can be used to provide insight into sustainability of volatile pricing strategies, providing a useful insight in volatility of British supermarket retail industry. 
\end{abstract}
\acknowledgements{
I would like to express my sincere gratitude to Dr Maria Polukarov for her guidance and support which provided me the freedom to take this research in the direction of my interest.\\
\\
I would also like to thank my family and friends for their encouragement and support. To those who quietly listened to my software complaints. To those who worked throughout the nights with me. To those who helped me write what I couldn't say. I cannot thank you enough.
}

\declaration{
I, Stefan Collier, declare that this dissertation and the work presented in it are my own and has been generated by me as the result of my own original research.\\
I confirm that:\\
1. This work was done wholly or mainly while in candidature for a degree at this University;\\
2. Where any part of this dissertation has previously been submitted for any other qualification at this University or any other institution, this has been clearly stated;\\
3. Where I have consulted the published work of others, this is always clearly attributed;\\
4. Where I have quoted from the work of others, the source is always given. With the exception of such quotations, this dissertation is entirely my own work;\\
5. I have acknowledged all main sources of help;\\
6. Where the thesis is based on work done by myself jointly with others, I have made clear exactly what was done by others and what I have contributed myself;\\
7. Either none of this work has been published before submission, or parts of this work have been published by :\\
\\
Stefan Collier\\
April 2016
}
\tableofcontents
\listoffigures
\listoftables

\mainmatter
%% ----------------------------------------------------------------
%\include{Introduction}
%\include{Conclusions}
\include{chapters/1Project/main}
\include{chapters/2Lit/main}
\include{chapters/3Design/HighLevel}
\include{chapters/3Design/InDepth}
\include{chapters/4Impl/main}

\include{chapters/5Experiments/1/main}
\include{chapters/5Experiments/2/main}
\include{chapters/5Experiments/3/main}
\include{chapters/5Experiments/4/main}

\include{chapters/6Conclusion/main}

\appendix
\include{appendix/AppendixB}
\include{appendix/D/main}
\include{appendix/AppendixC}

\backmatter
\bibliographystyle{ecs}
\bibliography{ECS}
\end{document}
%% ----------------------------------------------------------------

 %% ----------------------------------------------------------------
%% Progress.tex
%% ---------------------------------------------------------------- 
\documentclass{ecsprogress}    % Use the progress Style
\graphicspath{{../figs/}}   % Location of your graphics files
    \usepackage{natbib}            % Use Natbib style for the refs.
\hypersetup{colorlinks=true}   % Set to false for black/white printing
\input{Definitions}            % Include your abbreviations



\usepackage{enumitem}% http://ctan.org/pkg/enumitem
\usepackage{multirow}
\usepackage{float}
\usepackage{amsmath}
\usepackage{multicol}
\usepackage{amssymb}
\usepackage[normalem]{ulem}
\useunder{\uline}{\ul}{}
\usepackage{wrapfig}


\usepackage[table,xcdraw]{xcolor}


%% ----------------------------------------------------------------
\begin{document}
\frontmatter
\title      {Heterogeneous Agent-based Model for Supermarket Competition}
\authors    {\texorpdfstring
             {\href{mailto:sc22g13@ecs.soton.ac.uk}{Stefan J. Collier}}
             {Stefan J. Collier}
            }
\addresses  {\groupname\\\deptname\\\univname}
\date       {\today}
\subject    {}
\keywords   {}
\supervisor {Dr. Maria Polukarov}
\examiner   {Professor Sheng Chen}

\maketitle
\begin{abstract}
This project aim was to model and analyse the effects of competitive pricing behaviors of grocery retailers on the British market. 

This was achieved by creating a multi-agent model, containing retailer and consumer agents. The heterogeneous crowd of retailers employs either a uniform pricing strategy or a ‘local price flexing’ strategy. The actions of these retailers are chosen by predicting the profit of each action, using a perceptron. Following on from the consideration of different economic models, a discrete model was developed so that software agents have a discrete environment to operate within. Within the model, it has been observed how supermarkets with differing behaviors affect a heterogeneous crowd of consumer agents. The model was implemented in Java with Python used to evaluate the results. 

The simulation displays good acceptance with real grocery market behavior, i.e. captures the performance of British retailers thus can be used to determine the impact of changes in their behavior on their competitors and consumers.Furthermore it can be used to provide insight into sustainability of volatile pricing strategies, providing a useful insight in volatility of British supermarket retail industry. 
\end{abstract}
\acknowledgements{
I would like to express my sincere gratitude to Dr Maria Polukarov for her guidance and support which provided me the freedom to take this research in the direction of my interest.\\
\\
I would also like to thank my family and friends for their encouragement and support. To those who quietly listened to my software complaints. To those who worked throughout the nights with me. To those who helped me write what I couldn't say. I cannot thank you enough.
}

\declaration{
I, Stefan Collier, declare that this dissertation and the work presented in it are my own and has been generated by me as the result of my own original research.\\
I confirm that:\\
1. This work was done wholly or mainly while in candidature for a degree at this University;\\
2. Where any part of this dissertation has previously been submitted for any other qualification at this University or any other institution, this has been clearly stated;\\
3. Where I have consulted the published work of others, this is always clearly attributed;\\
4. Where I have quoted from the work of others, the source is always given. With the exception of such quotations, this dissertation is entirely my own work;\\
5. I have acknowledged all main sources of help;\\
6. Where the thesis is based on work done by myself jointly with others, I have made clear exactly what was done by others and what I have contributed myself;\\
7. Either none of this work has been published before submission, or parts of this work have been published by :\\
\\
Stefan Collier\\
April 2016
}
\tableofcontents
\listoffigures
\listoftables

\mainmatter
%% ----------------------------------------------------------------
%\include{Introduction}
%\include{Conclusions}
\include{chapters/1Project/main}
\include{chapters/2Lit/main}
\include{chapters/3Design/HighLevel}
\include{chapters/3Design/InDepth}
\include{chapters/4Impl/main}

\include{chapters/5Experiments/1/main}
\include{chapters/5Experiments/2/main}
\include{chapters/5Experiments/3/main}
\include{chapters/5Experiments/4/main}

\include{chapters/6Conclusion/main}

\appendix
\include{appendix/AppendixB}
\include{appendix/D/main}
\include{appendix/AppendixC}

\backmatter
\bibliographystyle{ecs}
\bibliography{ECS}
\end{document}
%% ----------------------------------------------------------------


 %% ----------------------------------------------------------------
%% Progress.tex
%% ---------------------------------------------------------------- 
\documentclass{ecsprogress}    % Use the progress Style
\graphicspath{{../figs/}}   % Location of your graphics files
    \usepackage{natbib}            % Use Natbib style for the refs.
\hypersetup{colorlinks=true}   % Set to false for black/white printing
\input{Definitions}            % Include your abbreviations



\usepackage{enumitem}% http://ctan.org/pkg/enumitem
\usepackage{multirow}
\usepackage{float}
\usepackage{amsmath}
\usepackage{multicol}
\usepackage{amssymb}
\usepackage[normalem]{ulem}
\useunder{\uline}{\ul}{}
\usepackage{wrapfig}


\usepackage[table,xcdraw]{xcolor}


%% ----------------------------------------------------------------
\begin{document}
\frontmatter
\title      {Heterogeneous Agent-based Model for Supermarket Competition}
\authors    {\texorpdfstring
             {\href{mailto:sc22g13@ecs.soton.ac.uk}{Stefan J. Collier}}
             {Stefan J. Collier}
            }
\addresses  {\groupname\\\deptname\\\univname}
\date       {\today}
\subject    {}
\keywords   {}
\supervisor {Dr. Maria Polukarov}
\examiner   {Professor Sheng Chen}

\maketitle
\begin{abstract}
This project aim was to model and analyse the effects of competitive pricing behaviors of grocery retailers on the British market. 

This was achieved by creating a multi-agent model, containing retailer and consumer agents. The heterogeneous crowd of retailers employs either a uniform pricing strategy or a ‘local price flexing’ strategy. The actions of these retailers are chosen by predicting the profit of each action, using a perceptron. Following on from the consideration of different economic models, a discrete model was developed so that software agents have a discrete environment to operate within. Within the model, it has been observed how supermarkets with differing behaviors affect a heterogeneous crowd of consumer agents. The model was implemented in Java with Python used to evaluate the results. 

The simulation displays good acceptance with real grocery market behavior, i.e. captures the performance of British retailers thus can be used to determine the impact of changes in their behavior on their competitors and consumers.Furthermore it can be used to provide insight into sustainability of volatile pricing strategies, providing a useful insight in volatility of British supermarket retail industry. 
\end{abstract}
\acknowledgements{
I would like to express my sincere gratitude to Dr Maria Polukarov for her guidance and support which provided me the freedom to take this research in the direction of my interest.\\
\\
I would also like to thank my family and friends for their encouragement and support. To those who quietly listened to my software complaints. To those who worked throughout the nights with me. To those who helped me write what I couldn't say. I cannot thank you enough.
}

\declaration{
I, Stefan Collier, declare that this dissertation and the work presented in it are my own and has been generated by me as the result of my own original research.\\
I confirm that:\\
1. This work was done wholly or mainly while in candidature for a degree at this University;\\
2. Where any part of this dissertation has previously been submitted for any other qualification at this University or any other institution, this has been clearly stated;\\
3. Where I have consulted the published work of others, this is always clearly attributed;\\
4. Where I have quoted from the work of others, the source is always given. With the exception of such quotations, this dissertation is entirely my own work;\\
5. I have acknowledged all main sources of help;\\
6. Where the thesis is based on work done by myself jointly with others, I have made clear exactly what was done by others and what I have contributed myself;\\
7. Either none of this work has been published before submission, or parts of this work have been published by :\\
\\
Stefan Collier\\
April 2016
}
\tableofcontents
\listoffigures
\listoftables

\mainmatter
%% ----------------------------------------------------------------
%\include{Introduction}
%\include{Conclusions}
\include{chapters/1Project/main}
\include{chapters/2Lit/main}
\include{chapters/3Design/HighLevel}
\include{chapters/3Design/InDepth}
\include{chapters/4Impl/main}

\include{chapters/5Experiments/1/main}
\include{chapters/5Experiments/2/main}
\include{chapters/5Experiments/3/main}
\include{chapters/5Experiments/4/main}

\include{chapters/6Conclusion/main}

\appendix
\include{appendix/AppendixB}
\include{appendix/D/main}
\include{appendix/AppendixC}

\backmatter
\bibliographystyle{ecs}
\bibliography{ECS}
\end{document}
%% ----------------------------------------------------------------


\appendix
\include{appendix/AppendixB}
 %% ----------------------------------------------------------------
%% Progress.tex
%% ---------------------------------------------------------------- 
\documentclass{ecsprogress}    % Use the progress Style
\graphicspath{{../figs/}}   % Location of your graphics files
    \usepackage{natbib}            % Use Natbib style for the refs.
\hypersetup{colorlinks=true}   % Set to false for black/white printing
\input{Definitions}            % Include your abbreviations



\usepackage{enumitem}% http://ctan.org/pkg/enumitem
\usepackage{multirow}
\usepackage{float}
\usepackage{amsmath}
\usepackage{multicol}
\usepackage{amssymb}
\usepackage[normalem]{ulem}
\useunder{\uline}{\ul}{}
\usepackage{wrapfig}


\usepackage[table,xcdraw]{xcolor}


%% ----------------------------------------------------------------
\begin{document}
\frontmatter
\title      {Heterogeneous Agent-based Model for Supermarket Competition}
\authors    {\texorpdfstring
             {\href{mailto:sc22g13@ecs.soton.ac.uk}{Stefan J. Collier}}
             {Stefan J. Collier}
            }
\addresses  {\groupname\\\deptname\\\univname}
\date       {\today}
\subject    {}
\keywords   {}
\supervisor {Dr. Maria Polukarov}
\examiner   {Professor Sheng Chen}

\maketitle
\begin{abstract}
This project aim was to model and analyse the effects of competitive pricing behaviors of grocery retailers on the British market. 

This was achieved by creating a multi-agent model, containing retailer and consumer agents. The heterogeneous crowd of retailers employs either a uniform pricing strategy or a ‘local price flexing’ strategy. The actions of these retailers are chosen by predicting the profit of each action, using a perceptron. Following on from the consideration of different economic models, a discrete model was developed so that software agents have a discrete environment to operate within. Within the model, it has been observed how supermarkets with differing behaviors affect a heterogeneous crowd of consumer agents. The model was implemented in Java with Python used to evaluate the results. 

The simulation displays good acceptance with real grocery market behavior, i.e. captures the performance of British retailers thus can be used to determine the impact of changes in their behavior on their competitors and consumers.Furthermore it can be used to provide insight into sustainability of volatile pricing strategies, providing a useful insight in volatility of British supermarket retail industry. 
\end{abstract}
\acknowledgements{
I would like to express my sincere gratitude to Dr Maria Polukarov for her guidance and support which provided me the freedom to take this research in the direction of my interest.\\
\\
I would also like to thank my family and friends for their encouragement and support. To those who quietly listened to my software complaints. To those who worked throughout the nights with me. To those who helped me write what I couldn't say. I cannot thank you enough.
}

\declaration{
I, Stefan Collier, declare that this dissertation and the work presented in it are my own and has been generated by me as the result of my own original research.\\
I confirm that:\\
1. This work was done wholly or mainly while in candidature for a degree at this University;\\
2. Where any part of this dissertation has previously been submitted for any other qualification at this University or any other institution, this has been clearly stated;\\
3. Where I have consulted the published work of others, this is always clearly attributed;\\
4. Where I have quoted from the work of others, the source is always given. With the exception of such quotations, this dissertation is entirely my own work;\\
5. I have acknowledged all main sources of help;\\
6. Where the thesis is based on work done by myself jointly with others, I have made clear exactly what was done by others and what I have contributed myself;\\
7. Either none of this work has been published before submission, or parts of this work have been published by :\\
\\
Stefan Collier\\
April 2016
}
\tableofcontents
\listoffigures
\listoftables

\mainmatter
%% ----------------------------------------------------------------
%\include{Introduction}
%\include{Conclusions}
\include{chapters/1Project/main}
\include{chapters/2Lit/main}
\include{chapters/3Design/HighLevel}
\include{chapters/3Design/InDepth}
\include{chapters/4Impl/main}

\include{chapters/5Experiments/1/main}
\include{chapters/5Experiments/2/main}
\include{chapters/5Experiments/3/main}
\include{chapters/5Experiments/4/main}

\include{chapters/6Conclusion/main}

\appendix
\include{appendix/AppendixB}
\include{appendix/D/main}
\include{appendix/AppendixC}

\backmatter
\bibliographystyle{ecs}
\bibliography{ECS}
\end{document}
%% ----------------------------------------------------------------

\include{appendix/AppendixC}

\backmatter
\bibliographystyle{ecs}
\bibliography{ECS}
\end{document}
%% ----------------------------------------------------------------

\include{chapters/3Design/HighLevel}
\include{chapters/3Design/InDepth}
 %% ----------------------------------------------------------------
%% Progress.tex
%% ---------------------------------------------------------------- 
\documentclass{ecsprogress}    % Use the progress Style
\graphicspath{{../figs/}}   % Location of your graphics files
    \usepackage{natbib}            % Use Natbib style for the refs.
\hypersetup{colorlinks=true}   % Set to false for black/white printing
\input{Definitions}            % Include your abbreviations



\usepackage{enumitem}% http://ctan.org/pkg/enumitem
\usepackage{multirow}
\usepackage{float}
\usepackage{amsmath}
\usepackage{multicol}
\usepackage{amssymb}
\usepackage[normalem]{ulem}
\useunder{\uline}{\ul}{}
\usepackage{wrapfig}


\usepackage[table,xcdraw]{xcolor}


%% ----------------------------------------------------------------
\begin{document}
\frontmatter
\title      {Heterogeneous Agent-based Model for Supermarket Competition}
\authors    {\texorpdfstring
             {\href{mailto:sc22g13@ecs.soton.ac.uk}{Stefan J. Collier}}
             {Stefan J. Collier}
            }
\addresses  {\groupname\\\deptname\\\univname}
\date       {\today}
\subject    {}
\keywords   {}
\supervisor {Dr. Maria Polukarov}
\examiner   {Professor Sheng Chen}

\maketitle
\begin{abstract}
This project aim was to model and analyse the effects of competitive pricing behaviors of grocery retailers on the British market. 

This was achieved by creating a multi-agent model, containing retailer and consumer agents. The heterogeneous crowd of retailers employs either a uniform pricing strategy or a ‘local price flexing’ strategy. The actions of these retailers are chosen by predicting the profit of each action, using a perceptron. Following on from the consideration of different economic models, a discrete model was developed so that software agents have a discrete environment to operate within. Within the model, it has been observed how supermarkets with differing behaviors affect a heterogeneous crowd of consumer agents. The model was implemented in Java with Python used to evaluate the results. 

The simulation displays good acceptance with real grocery market behavior, i.e. captures the performance of British retailers thus can be used to determine the impact of changes in their behavior on their competitors and consumers.Furthermore it can be used to provide insight into sustainability of volatile pricing strategies, providing a useful insight in volatility of British supermarket retail industry. 
\end{abstract}
\acknowledgements{
I would like to express my sincere gratitude to Dr Maria Polukarov for her guidance and support which provided me the freedom to take this research in the direction of my interest.\\
\\
I would also like to thank my family and friends for their encouragement and support. To those who quietly listened to my software complaints. To those who worked throughout the nights with me. To those who helped me write what I couldn't say. I cannot thank you enough.
}

\declaration{
I, Stefan Collier, declare that this dissertation and the work presented in it are my own and has been generated by me as the result of my own original research.\\
I confirm that:\\
1. This work was done wholly or mainly while in candidature for a degree at this University;\\
2. Where any part of this dissertation has previously been submitted for any other qualification at this University or any other institution, this has been clearly stated;\\
3. Where I have consulted the published work of others, this is always clearly attributed;\\
4. Where I have quoted from the work of others, the source is always given. With the exception of such quotations, this dissertation is entirely my own work;\\
5. I have acknowledged all main sources of help;\\
6. Where the thesis is based on work done by myself jointly with others, I have made clear exactly what was done by others and what I have contributed myself;\\
7. Either none of this work has been published before submission, or parts of this work have been published by :\\
\\
Stefan Collier\\
April 2016
}
\tableofcontents
\listoffigures
\listoftables

\mainmatter
%% ----------------------------------------------------------------
%\include{Introduction}
%\include{Conclusions}
 %% ----------------------------------------------------------------
%% Progress.tex
%% ---------------------------------------------------------------- 
\documentclass{ecsprogress}    % Use the progress Style
\graphicspath{{../figs/}}   % Location of your graphics files
    \usepackage{natbib}            % Use Natbib style for the refs.
\hypersetup{colorlinks=true}   % Set to false for black/white printing
\input{Definitions}            % Include your abbreviations



\usepackage{enumitem}% http://ctan.org/pkg/enumitem
\usepackage{multirow}
\usepackage{float}
\usepackage{amsmath}
\usepackage{multicol}
\usepackage{amssymb}
\usepackage[normalem]{ulem}
\useunder{\uline}{\ul}{}
\usepackage{wrapfig}


\usepackage[table,xcdraw]{xcolor}


%% ----------------------------------------------------------------
\begin{document}
\frontmatter
\title      {Heterogeneous Agent-based Model for Supermarket Competition}
\authors    {\texorpdfstring
             {\href{mailto:sc22g13@ecs.soton.ac.uk}{Stefan J. Collier}}
             {Stefan J. Collier}
            }
\addresses  {\groupname\\\deptname\\\univname}
\date       {\today}
\subject    {}
\keywords   {}
\supervisor {Dr. Maria Polukarov}
\examiner   {Professor Sheng Chen}

\maketitle
\begin{abstract}
This project aim was to model and analyse the effects of competitive pricing behaviors of grocery retailers on the British market. 

This was achieved by creating a multi-agent model, containing retailer and consumer agents. The heterogeneous crowd of retailers employs either a uniform pricing strategy or a ‘local price flexing’ strategy. The actions of these retailers are chosen by predicting the profit of each action, using a perceptron. Following on from the consideration of different economic models, a discrete model was developed so that software agents have a discrete environment to operate within. Within the model, it has been observed how supermarkets with differing behaviors affect a heterogeneous crowd of consumer agents. The model was implemented in Java with Python used to evaluate the results. 

The simulation displays good acceptance with real grocery market behavior, i.e. captures the performance of British retailers thus can be used to determine the impact of changes in their behavior on their competitors and consumers.Furthermore it can be used to provide insight into sustainability of volatile pricing strategies, providing a useful insight in volatility of British supermarket retail industry. 
\end{abstract}
\acknowledgements{
I would like to express my sincere gratitude to Dr Maria Polukarov for her guidance and support which provided me the freedom to take this research in the direction of my interest.\\
\\
I would also like to thank my family and friends for their encouragement and support. To those who quietly listened to my software complaints. To those who worked throughout the nights with me. To those who helped me write what I couldn't say. I cannot thank you enough.
}

\declaration{
I, Stefan Collier, declare that this dissertation and the work presented in it are my own and has been generated by me as the result of my own original research.\\
I confirm that:\\
1. This work was done wholly or mainly while in candidature for a degree at this University;\\
2. Where any part of this dissertation has previously been submitted for any other qualification at this University or any other institution, this has been clearly stated;\\
3. Where I have consulted the published work of others, this is always clearly attributed;\\
4. Where I have quoted from the work of others, the source is always given. With the exception of such quotations, this dissertation is entirely my own work;\\
5. I have acknowledged all main sources of help;\\
6. Where the thesis is based on work done by myself jointly with others, I have made clear exactly what was done by others and what I have contributed myself;\\
7. Either none of this work has been published before submission, or parts of this work have been published by :\\
\\
Stefan Collier\\
April 2016
}
\tableofcontents
\listoffigures
\listoftables

\mainmatter
%% ----------------------------------------------------------------
%\include{Introduction}
%\include{Conclusions}
\include{chapters/1Project/main}
\include{chapters/2Lit/main}
\include{chapters/3Design/HighLevel}
\include{chapters/3Design/InDepth}
\include{chapters/4Impl/main}

\include{chapters/5Experiments/1/main}
\include{chapters/5Experiments/2/main}
\include{chapters/5Experiments/3/main}
\include{chapters/5Experiments/4/main}

\include{chapters/6Conclusion/main}

\appendix
\include{appendix/AppendixB}
\include{appendix/D/main}
\include{appendix/AppendixC}

\backmatter
\bibliographystyle{ecs}
\bibliography{ECS}
\end{document}
%% ----------------------------------------------------------------

 %% ----------------------------------------------------------------
%% Progress.tex
%% ---------------------------------------------------------------- 
\documentclass{ecsprogress}    % Use the progress Style
\graphicspath{{../figs/}}   % Location of your graphics files
    \usepackage{natbib}            % Use Natbib style for the refs.
\hypersetup{colorlinks=true}   % Set to false for black/white printing
\input{Definitions}            % Include your abbreviations



\usepackage{enumitem}% http://ctan.org/pkg/enumitem
\usepackage{multirow}
\usepackage{float}
\usepackage{amsmath}
\usepackage{multicol}
\usepackage{amssymb}
\usepackage[normalem]{ulem}
\useunder{\uline}{\ul}{}
\usepackage{wrapfig}


\usepackage[table,xcdraw]{xcolor}


%% ----------------------------------------------------------------
\begin{document}
\frontmatter
\title      {Heterogeneous Agent-based Model for Supermarket Competition}
\authors    {\texorpdfstring
             {\href{mailto:sc22g13@ecs.soton.ac.uk}{Stefan J. Collier}}
             {Stefan J. Collier}
            }
\addresses  {\groupname\\\deptname\\\univname}
\date       {\today}
\subject    {}
\keywords   {}
\supervisor {Dr. Maria Polukarov}
\examiner   {Professor Sheng Chen}

\maketitle
\begin{abstract}
This project aim was to model and analyse the effects of competitive pricing behaviors of grocery retailers on the British market. 

This was achieved by creating a multi-agent model, containing retailer and consumer agents. The heterogeneous crowd of retailers employs either a uniform pricing strategy or a ‘local price flexing’ strategy. The actions of these retailers are chosen by predicting the profit of each action, using a perceptron. Following on from the consideration of different economic models, a discrete model was developed so that software agents have a discrete environment to operate within. Within the model, it has been observed how supermarkets with differing behaviors affect a heterogeneous crowd of consumer agents. The model was implemented in Java with Python used to evaluate the results. 

The simulation displays good acceptance with real grocery market behavior, i.e. captures the performance of British retailers thus can be used to determine the impact of changes in their behavior on their competitors and consumers.Furthermore it can be used to provide insight into sustainability of volatile pricing strategies, providing a useful insight in volatility of British supermarket retail industry. 
\end{abstract}
\acknowledgements{
I would like to express my sincere gratitude to Dr Maria Polukarov for her guidance and support which provided me the freedom to take this research in the direction of my interest.\\
\\
I would also like to thank my family and friends for their encouragement and support. To those who quietly listened to my software complaints. To those who worked throughout the nights with me. To those who helped me write what I couldn't say. I cannot thank you enough.
}

\declaration{
I, Stefan Collier, declare that this dissertation and the work presented in it are my own and has been generated by me as the result of my own original research.\\
I confirm that:\\
1. This work was done wholly or mainly while in candidature for a degree at this University;\\
2. Where any part of this dissertation has previously been submitted for any other qualification at this University or any other institution, this has been clearly stated;\\
3. Where I have consulted the published work of others, this is always clearly attributed;\\
4. Where I have quoted from the work of others, the source is always given. With the exception of such quotations, this dissertation is entirely my own work;\\
5. I have acknowledged all main sources of help;\\
6. Where the thesis is based on work done by myself jointly with others, I have made clear exactly what was done by others and what I have contributed myself;\\
7. Either none of this work has been published before submission, or parts of this work have been published by :\\
\\
Stefan Collier\\
April 2016
}
\tableofcontents
\listoffigures
\listoftables

\mainmatter
%% ----------------------------------------------------------------
%\include{Introduction}
%\include{Conclusions}
\include{chapters/1Project/main}
\include{chapters/2Lit/main}
\include{chapters/3Design/HighLevel}
\include{chapters/3Design/InDepth}
\include{chapters/4Impl/main}

\include{chapters/5Experiments/1/main}
\include{chapters/5Experiments/2/main}
\include{chapters/5Experiments/3/main}
\include{chapters/5Experiments/4/main}

\include{chapters/6Conclusion/main}

\appendix
\include{appendix/AppendixB}
\include{appendix/D/main}
\include{appendix/AppendixC}

\backmatter
\bibliographystyle{ecs}
\bibliography{ECS}
\end{document}
%% ----------------------------------------------------------------

\include{chapters/3Design/HighLevel}
\include{chapters/3Design/InDepth}
 %% ----------------------------------------------------------------
%% Progress.tex
%% ---------------------------------------------------------------- 
\documentclass{ecsprogress}    % Use the progress Style
\graphicspath{{../figs/}}   % Location of your graphics files
    \usepackage{natbib}            % Use Natbib style for the refs.
\hypersetup{colorlinks=true}   % Set to false for black/white printing
\input{Definitions}            % Include your abbreviations



\usepackage{enumitem}% http://ctan.org/pkg/enumitem
\usepackage{multirow}
\usepackage{float}
\usepackage{amsmath}
\usepackage{multicol}
\usepackage{amssymb}
\usepackage[normalem]{ulem}
\useunder{\uline}{\ul}{}
\usepackage{wrapfig}


\usepackage[table,xcdraw]{xcolor}


%% ----------------------------------------------------------------
\begin{document}
\frontmatter
\title      {Heterogeneous Agent-based Model for Supermarket Competition}
\authors    {\texorpdfstring
             {\href{mailto:sc22g13@ecs.soton.ac.uk}{Stefan J. Collier}}
             {Stefan J. Collier}
            }
\addresses  {\groupname\\\deptname\\\univname}
\date       {\today}
\subject    {}
\keywords   {}
\supervisor {Dr. Maria Polukarov}
\examiner   {Professor Sheng Chen}

\maketitle
\begin{abstract}
This project aim was to model and analyse the effects of competitive pricing behaviors of grocery retailers on the British market. 

This was achieved by creating a multi-agent model, containing retailer and consumer agents. The heterogeneous crowd of retailers employs either a uniform pricing strategy or a ‘local price flexing’ strategy. The actions of these retailers are chosen by predicting the profit of each action, using a perceptron. Following on from the consideration of different economic models, a discrete model was developed so that software agents have a discrete environment to operate within. Within the model, it has been observed how supermarkets with differing behaviors affect a heterogeneous crowd of consumer agents. The model was implemented in Java with Python used to evaluate the results. 

The simulation displays good acceptance with real grocery market behavior, i.e. captures the performance of British retailers thus can be used to determine the impact of changes in their behavior on their competitors and consumers.Furthermore it can be used to provide insight into sustainability of volatile pricing strategies, providing a useful insight in volatility of British supermarket retail industry. 
\end{abstract}
\acknowledgements{
I would like to express my sincere gratitude to Dr Maria Polukarov for her guidance and support which provided me the freedom to take this research in the direction of my interest.\\
\\
I would also like to thank my family and friends for their encouragement and support. To those who quietly listened to my software complaints. To those who worked throughout the nights with me. To those who helped me write what I couldn't say. I cannot thank you enough.
}

\declaration{
I, Stefan Collier, declare that this dissertation and the work presented in it are my own and has been generated by me as the result of my own original research.\\
I confirm that:\\
1. This work was done wholly or mainly while in candidature for a degree at this University;\\
2. Where any part of this dissertation has previously been submitted for any other qualification at this University or any other institution, this has been clearly stated;\\
3. Where I have consulted the published work of others, this is always clearly attributed;\\
4. Where I have quoted from the work of others, the source is always given. With the exception of such quotations, this dissertation is entirely my own work;\\
5. I have acknowledged all main sources of help;\\
6. Where the thesis is based on work done by myself jointly with others, I have made clear exactly what was done by others and what I have contributed myself;\\
7. Either none of this work has been published before submission, or parts of this work have been published by :\\
\\
Stefan Collier\\
April 2016
}
\tableofcontents
\listoffigures
\listoftables

\mainmatter
%% ----------------------------------------------------------------
%\include{Introduction}
%\include{Conclusions}
\include{chapters/1Project/main}
\include{chapters/2Lit/main}
\include{chapters/3Design/HighLevel}
\include{chapters/3Design/InDepth}
\include{chapters/4Impl/main}

\include{chapters/5Experiments/1/main}
\include{chapters/5Experiments/2/main}
\include{chapters/5Experiments/3/main}
\include{chapters/5Experiments/4/main}

\include{chapters/6Conclusion/main}

\appendix
\include{appendix/AppendixB}
\include{appendix/D/main}
\include{appendix/AppendixC}

\backmatter
\bibliographystyle{ecs}
\bibliography{ECS}
\end{document}
%% ----------------------------------------------------------------


 %% ----------------------------------------------------------------
%% Progress.tex
%% ---------------------------------------------------------------- 
\documentclass{ecsprogress}    % Use the progress Style
\graphicspath{{../figs/}}   % Location of your graphics files
    \usepackage{natbib}            % Use Natbib style for the refs.
\hypersetup{colorlinks=true}   % Set to false for black/white printing
\input{Definitions}            % Include your abbreviations



\usepackage{enumitem}% http://ctan.org/pkg/enumitem
\usepackage{multirow}
\usepackage{float}
\usepackage{amsmath}
\usepackage{multicol}
\usepackage{amssymb}
\usepackage[normalem]{ulem}
\useunder{\uline}{\ul}{}
\usepackage{wrapfig}


\usepackage[table,xcdraw]{xcolor}


%% ----------------------------------------------------------------
\begin{document}
\frontmatter
\title      {Heterogeneous Agent-based Model for Supermarket Competition}
\authors    {\texorpdfstring
             {\href{mailto:sc22g13@ecs.soton.ac.uk}{Stefan J. Collier}}
             {Stefan J. Collier}
            }
\addresses  {\groupname\\\deptname\\\univname}
\date       {\today}
\subject    {}
\keywords   {}
\supervisor {Dr. Maria Polukarov}
\examiner   {Professor Sheng Chen}

\maketitle
\begin{abstract}
This project aim was to model and analyse the effects of competitive pricing behaviors of grocery retailers on the British market. 

This was achieved by creating a multi-agent model, containing retailer and consumer agents. The heterogeneous crowd of retailers employs either a uniform pricing strategy or a ‘local price flexing’ strategy. The actions of these retailers are chosen by predicting the profit of each action, using a perceptron. Following on from the consideration of different economic models, a discrete model was developed so that software agents have a discrete environment to operate within. Within the model, it has been observed how supermarkets with differing behaviors affect a heterogeneous crowd of consumer agents. The model was implemented in Java with Python used to evaluate the results. 

The simulation displays good acceptance with real grocery market behavior, i.e. captures the performance of British retailers thus can be used to determine the impact of changes in their behavior on their competitors and consumers.Furthermore it can be used to provide insight into sustainability of volatile pricing strategies, providing a useful insight in volatility of British supermarket retail industry. 
\end{abstract}
\acknowledgements{
I would like to express my sincere gratitude to Dr Maria Polukarov for her guidance and support which provided me the freedom to take this research in the direction of my interest.\\
\\
I would also like to thank my family and friends for their encouragement and support. To those who quietly listened to my software complaints. To those who worked throughout the nights with me. To those who helped me write what I couldn't say. I cannot thank you enough.
}

\declaration{
I, Stefan Collier, declare that this dissertation and the work presented in it are my own and has been generated by me as the result of my own original research.\\
I confirm that:\\
1. This work was done wholly or mainly while in candidature for a degree at this University;\\
2. Where any part of this dissertation has previously been submitted for any other qualification at this University or any other institution, this has been clearly stated;\\
3. Where I have consulted the published work of others, this is always clearly attributed;\\
4. Where I have quoted from the work of others, the source is always given. With the exception of such quotations, this dissertation is entirely my own work;\\
5. I have acknowledged all main sources of help;\\
6. Where the thesis is based on work done by myself jointly with others, I have made clear exactly what was done by others and what I have contributed myself;\\
7. Either none of this work has been published before submission, or parts of this work have been published by :\\
\\
Stefan Collier\\
April 2016
}
\tableofcontents
\listoffigures
\listoftables

\mainmatter
%% ----------------------------------------------------------------
%\include{Introduction}
%\include{Conclusions}
\include{chapters/1Project/main}
\include{chapters/2Lit/main}
\include{chapters/3Design/HighLevel}
\include{chapters/3Design/InDepth}
\include{chapters/4Impl/main}

\include{chapters/5Experiments/1/main}
\include{chapters/5Experiments/2/main}
\include{chapters/5Experiments/3/main}
\include{chapters/5Experiments/4/main}

\include{chapters/6Conclusion/main}

\appendix
\include{appendix/AppendixB}
\include{appendix/D/main}
\include{appendix/AppendixC}

\backmatter
\bibliographystyle{ecs}
\bibliography{ECS}
\end{document}
%% ----------------------------------------------------------------

 %% ----------------------------------------------------------------
%% Progress.tex
%% ---------------------------------------------------------------- 
\documentclass{ecsprogress}    % Use the progress Style
\graphicspath{{../figs/}}   % Location of your graphics files
    \usepackage{natbib}            % Use Natbib style for the refs.
\hypersetup{colorlinks=true}   % Set to false for black/white printing
\input{Definitions}            % Include your abbreviations



\usepackage{enumitem}% http://ctan.org/pkg/enumitem
\usepackage{multirow}
\usepackage{float}
\usepackage{amsmath}
\usepackage{multicol}
\usepackage{amssymb}
\usepackage[normalem]{ulem}
\useunder{\uline}{\ul}{}
\usepackage{wrapfig}


\usepackage[table,xcdraw]{xcolor}


%% ----------------------------------------------------------------
\begin{document}
\frontmatter
\title      {Heterogeneous Agent-based Model for Supermarket Competition}
\authors    {\texorpdfstring
             {\href{mailto:sc22g13@ecs.soton.ac.uk}{Stefan J. Collier}}
             {Stefan J. Collier}
            }
\addresses  {\groupname\\\deptname\\\univname}
\date       {\today}
\subject    {}
\keywords   {}
\supervisor {Dr. Maria Polukarov}
\examiner   {Professor Sheng Chen}

\maketitle
\begin{abstract}
This project aim was to model and analyse the effects of competitive pricing behaviors of grocery retailers on the British market. 

This was achieved by creating a multi-agent model, containing retailer and consumer agents. The heterogeneous crowd of retailers employs either a uniform pricing strategy or a ‘local price flexing’ strategy. The actions of these retailers are chosen by predicting the profit of each action, using a perceptron. Following on from the consideration of different economic models, a discrete model was developed so that software agents have a discrete environment to operate within. Within the model, it has been observed how supermarkets with differing behaviors affect a heterogeneous crowd of consumer agents. The model was implemented in Java with Python used to evaluate the results. 

The simulation displays good acceptance with real grocery market behavior, i.e. captures the performance of British retailers thus can be used to determine the impact of changes in their behavior on their competitors and consumers.Furthermore it can be used to provide insight into sustainability of volatile pricing strategies, providing a useful insight in volatility of British supermarket retail industry. 
\end{abstract}
\acknowledgements{
I would like to express my sincere gratitude to Dr Maria Polukarov for her guidance and support which provided me the freedom to take this research in the direction of my interest.\\
\\
I would also like to thank my family and friends for their encouragement and support. To those who quietly listened to my software complaints. To those who worked throughout the nights with me. To those who helped me write what I couldn't say. I cannot thank you enough.
}

\declaration{
I, Stefan Collier, declare that this dissertation and the work presented in it are my own and has been generated by me as the result of my own original research.\\
I confirm that:\\
1. This work was done wholly or mainly while in candidature for a degree at this University;\\
2. Where any part of this dissertation has previously been submitted for any other qualification at this University or any other institution, this has been clearly stated;\\
3. Where I have consulted the published work of others, this is always clearly attributed;\\
4. Where I have quoted from the work of others, the source is always given. With the exception of such quotations, this dissertation is entirely my own work;\\
5. I have acknowledged all main sources of help;\\
6. Where the thesis is based on work done by myself jointly with others, I have made clear exactly what was done by others and what I have contributed myself;\\
7. Either none of this work has been published before submission, or parts of this work have been published by :\\
\\
Stefan Collier\\
April 2016
}
\tableofcontents
\listoffigures
\listoftables

\mainmatter
%% ----------------------------------------------------------------
%\include{Introduction}
%\include{Conclusions}
\include{chapters/1Project/main}
\include{chapters/2Lit/main}
\include{chapters/3Design/HighLevel}
\include{chapters/3Design/InDepth}
\include{chapters/4Impl/main}

\include{chapters/5Experiments/1/main}
\include{chapters/5Experiments/2/main}
\include{chapters/5Experiments/3/main}
\include{chapters/5Experiments/4/main}

\include{chapters/6Conclusion/main}

\appendix
\include{appendix/AppendixB}
\include{appendix/D/main}
\include{appendix/AppendixC}

\backmatter
\bibliographystyle{ecs}
\bibliography{ECS}
\end{document}
%% ----------------------------------------------------------------

 %% ----------------------------------------------------------------
%% Progress.tex
%% ---------------------------------------------------------------- 
\documentclass{ecsprogress}    % Use the progress Style
\graphicspath{{../figs/}}   % Location of your graphics files
    \usepackage{natbib}            % Use Natbib style for the refs.
\hypersetup{colorlinks=true}   % Set to false for black/white printing
\input{Definitions}            % Include your abbreviations



\usepackage{enumitem}% http://ctan.org/pkg/enumitem
\usepackage{multirow}
\usepackage{float}
\usepackage{amsmath}
\usepackage{multicol}
\usepackage{amssymb}
\usepackage[normalem]{ulem}
\useunder{\uline}{\ul}{}
\usepackage{wrapfig}


\usepackage[table,xcdraw]{xcolor}


%% ----------------------------------------------------------------
\begin{document}
\frontmatter
\title      {Heterogeneous Agent-based Model for Supermarket Competition}
\authors    {\texorpdfstring
             {\href{mailto:sc22g13@ecs.soton.ac.uk}{Stefan J. Collier}}
             {Stefan J. Collier}
            }
\addresses  {\groupname\\\deptname\\\univname}
\date       {\today}
\subject    {}
\keywords   {}
\supervisor {Dr. Maria Polukarov}
\examiner   {Professor Sheng Chen}

\maketitle
\begin{abstract}
This project aim was to model and analyse the effects of competitive pricing behaviors of grocery retailers on the British market. 

This was achieved by creating a multi-agent model, containing retailer and consumer agents. The heterogeneous crowd of retailers employs either a uniform pricing strategy or a ‘local price flexing’ strategy. The actions of these retailers are chosen by predicting the profit of each action, using a perceptron. Following on from the consideration of different economic models, a discrete model was developed so that software agents have a discrete environment to operate within. Within the model, it has been observed how supermarkets with differing behaviors affect a heterogeneous crowd of consumer agents. The model was implemented in Java with Python used to evaluate the results. 

The simulation displays good acceptance with real grocery market behavior, i.e. captures the performance of British retailers thus can be used to determine the impact of changes in their behavior on their competitors and consumers.Furthermore it can be used to provide insight into sustainability of volatile pricing strategies, providing a useful insight in volatility of British supermarket retail industry. 
\end{abstract}
\acknowledgements{
I would like to express my sincere gratitude to Dr Maria Polukarov for her guidance and support which provided me the freedom to take this research in the direction of my interest.\\
\\
I would also like to thank my family and friends for their encouragement and support. To those who quietly listened to my software complaints. To those who worked throughout the nights with me. To those who helped me write what I couldn't say. I cannot thank you enough.
}

\declaration{
I, Stefan Collier, declare that this dissertation and the work presented in it are my own and has been generated by me as the result of my own original research.\\
I confirm that:\\
1. This work was done wholly or mainly while in candidature for a degree at this University;\\
2. Where any part of this dissertation has previously been submitted for any other qualification at this University or any other institution, this has been clearly stated;\\
3. Where I have consulted the published work of others, this is always clearly attributed;\\
4. Where I have quoted from the work of others, the source is always given. With the exception of such quotations, this dissertation is entirely my own work;\\
5. I have acknowledged all main sources of help;\\
6. Where the thesis is based on work done by myself jointly with others, I have made clear exactly what was done by others and what I have contributed myself;\\
7. Either none of this work has been published before submission, or parts of this work have been published by :\\
\\
Stefan Collier\\
April 2016
}
\tableofcontents
\listoffigures
\listoftables

\mainmatter
%% ----------------------------------------------------------------
%\include{Introduction}
%\include{Conclusions}
\include{chapters/1Project/main}
\include{chapters/2Lit/main}
\include{chapters/3Design/HighLevel}
\include{chapters/3Design/InDepth}
\include{chapters/4Impl/main}

\include{chapters/5Experiments/1/main}
\include{chapters/5Experiments/2/main}
\include{chapters/5Experiments/3/main}
\include{chapters/5Experiments/4/main}

\include{chapters/6Conclusion/main}

\appendix
\include{appendix/AppendixB}
\include{appendix/D/main}
\include{appendix/AppendixC}

\backmatter
\bibliographystyle{ecs}
\bibliography{ECS}
\end{document}
%% ----------------------------------------------------------------

 %% ----------------------------------------------------------------
%% Progress.tex
%% ---------------------------------------------------------------- 
\documentclass{ecsprogress}    % Use the progress Style
\graphicspath{{../figs/}}   % Location of your graphics files
    \usepackage{natbib}            % Use Natbib style for the refs.
\hypersetup{colorlinks=true}   % Set to false for black/white printing
\input{Definitions}            % Include your abbreviations



\usepackage{enumitem}% http://ctan.org/pkg/enumitem
\usepackage{multirow}
\usepackage{float}
\usepackage{amsmath}
\usepackage{multicol}
\usepackage{amssymb}
\usepackage[normalem]{ulem}
\useunder{\uline}{\ul}{}
\usepackage{wrapfig}


\usepackage[table,xcdraw]{xcolor}


%% ----------------------------------------------------------------
\begin{document}
\frontmatter
\title      {Heterogeneous Agent-based Model for Supermarket Competition}
\authors    {\texorpdfstring
             {\href{mailto:sc22g13@ecs.soton.ac.uk}{Stefan J. Collier}}
             {Stefan J. Collier}
            }
\addresses  {\groupname\\\deptname\\\univname}
\date       {\today}
\subject    {}
\keywords   {}
\supervisor {Dr. Maria Polukarov}
\examiner   {Professor Sheng Chen}

\maketitle
\begin{abstract}
This project aim was to model and analyse the effects of competitive pricing behaviors of grocery retailers on the British market. 

This was achieved by creating a multi-agent model, containing retailer and consumer agents. The heterogeneous crowd of retailers employs either a uniform pricing strategy or a ‘local price flexing’ strategy. The actions of these retailers are chosen by predicting the profit of each action, using a perceptron. Following on from the consideration of different economic models, a discrete model was developed so that software agents have a discrete environment to operate within. Within the model, it has been observed how supermarkets with differing behaviors affect a heterogeneous crowd of consumer agents. The model was implemented in Java with Python used to evaluate the results. 

The simulation displays good acceptance with real grocery market behavior, i.e. captures the performance of British retailers thus can be used to determine the impact of changes in their behavior on their competitors and consumers.Furthermore it can be used to provide insight into sustainability of volatile pricing strategies, providing a useful insight in volatility of British supermarket retail industry. 
\end{abstract}
\acknowledgements{
I would like to express my sincere gratitude to Dr Maria Polukarov for her guidance and support which provided me the freedom to take this research in the direction of my interest.\\
\\
I would also like to thank my family and friends for their encouragement and support. To those who quietly listened to my software complaints. To those who worked throughout the nights with me. To those who helped me write what I couldn't say. I cannot thank you enough.
}

\declaration{
I, Stefan Collier, declare that this dissertation and the work presented in it are my own and has been generated by me as the result of my own original research.\\
I confirm that:\\
1. This work was done wholly or mainly while in candidature for a degree at this University;\\
2. Where any part of this dissertation has previously been submitted for any other qualification at this University or any other institution, this has been clearly stated;\\
3. Where I have consulted the published work of others, this is always clearly attributed;\\
4. Where I have quoted from the work of others, the source is always given. With the exception of such quotations, this dissertation is entirely my own work;\\
5. I have acknowledged all main sources of help;\\
6. Where the thesis is based on work done by myself jointly with others, I have made clear exactly what was done by others and what I have contributed myself;\\
7. Either none of this work has been published before submission, or parts of this work have been published by :\\
\\
Stefan Collier\\
April 2016
}
\tableofcontents
\listoffigures
\listoftables

\mainmatter
%% ----------------------------------------------------------------
%\include{Introduction}
%\include{Conclusions}
\include{chapters/1Project/main}
\include{chapters/2Lit/main}
\include{chapters/3Design/HighLevel}
\include{chapters/3Design/InDepth}
\include{chapters/4Impl/main}

\include{chapters/5Experiments/1/main}
\include{chapters/5Experiments/2/main}
\include{chapters/5Experiments/3/main}
\include{chapters/5Experiments/4/main}

\include{chapters/6Conclusion/main}

\appendix
\include{appendix/AppendixB}
\include{appendix/D/main}
\include{appendix/AppendixC}

\backmatter
\bibliographystyle{ecs}
\bibliography{ECS}
\end{document}
%% ----------------------------------------------------------------


 %% ----------------------------------------------------------------
%% Progress.tex
%% ---------------------------------------------------------------- 
\documentclass{ecsprogress}    % Use the progress Style
\graphicspath{{../figs/}}   % Location of your graphics files
    \usepackage{natbib}            % Use Natbib style for the refs.
\hypersetup{colorlinks=true}   % Set to false for black/white printing
\input{Definitions}            % Include your abbreviations



\usepackage{enumitem}% http://ctan.org/pkg/enumitem
\usepackage{multirow}
\usepackage{float}
\usepackage{amsmath}
\usepackage{multicol}
\usepackage{amssymb}
\usepackage[normalem]{ulem}
\useunder{\uline}{\ul}{}
\usepackage{wrapfig}


\usepackage[table,xcdraw]{xcolor}


%% ----------------------------------------------------------------
\begin{document}
\frontmatter
\title      {Heterogeneous Agent-based Model for Supermarket Competition}
\authors    {\texorpdfstring
             {\href{mailto:sc22g13@ecs.soton.ac.uk}{Stefan J. Collier}}
             {Stefan J. Collier}
            }
\addresses  {\groupname\\\deptname\\\univname}
\date       {\today}
\subject    {}
\keywords   {}
\supervisor {Dr. Maria Polukarov}
\examiner   {Professor Sheng Chen}

\maketitle
\begin{abstract}
This project aim was to model and analyse the effects of competitive pricing behaviors of grocery retailers on the British market. 

This was achieved by creating a multi-agent model, containing retailer and consumer agents. The heterogeneous crowd of retailers employs either a uniform pricing strategy or a ‘local price flexing’ strategy. The actions of these retailers are chosen by predicting the profit of each action, using a perceptron. Following on from the consideration of different economic models, a discrete model was developed so that software agents have a discrete environment to operate within. Within the model, it has been observed how supermarkets with differing behaviors affect a heterogeneous crowd of consumer agents. The model was implemented in Java with Python used to evaluate the results. 

The simulation displays good acceptance with real grocery market behavior, i.e. captures the performance of British retailers thus can be used to determine the impact of changes in their behavior on their competitors and consumers.Furthermore it can be used to provide insight into sustainability of volatile pricing strategies, providing a useful insight in volatility of British supermarket retail industry. 
\end{abstract}
\acknowledgements{
I would like to express my sincere gratitude to Dr Maria Polukarov for her guidance and support which provided me the freedom to take this research in the direction of my interest.\\
\\
I would also like to thank my family and friends for their encouragement and support. To those who quietly listened to my software complaints. To those who worked throughout the nights with me. To those who helped me write what I couldn't say. I cannot thank you enough.
}

\declaration{
I, Stefan Collier, declare that this dissertation and the work presented in it are my own and has been generated by me as the result of my own original research.\\
I confirm that:\\
1. This work was done wholly or mainly while in candidature for a degree at this University;\\
2. Where any part of this dissertation has previously been submitted for any other qualification at this University or any other institution, this has been clearly stated;\\
3. Where I have consulted the published work of others, this is always clearly attributed;\\
4. Where I have quoted from the work of others, the source is always given. With the exception of such quotations, this dissertation is entirely my own work;\\
5. I have acknowledged all main sources of help;\\
6. Where the thesis is based on work done by myself jointly with others, I have made clear exactly what was done by others and what I have contributed myself;\\
7. Either none of this work has been published before submission, or parts of this work have been published by :\\
\\
Stefan Collier\\
April 2016
}
\tableofcontents
\listoffigures
\listoftables

\mainmatter
%% ----------------------------------------------------------------
%\include{Introduction}
%\include{Conclusions}
\include{chapters/1Project/main}
\include{chapters/2Lit/main}
\include{chapters/3Design/HighLevel}
\include{chapters/3Design/InDepth}
\include{chapters/4Impl/main}

\include{chapters/5Experiments/1/main}
\include{chapters/5Experiments/2/main}
\include{chapters/5Experiments/3/main}
\include{chapters/5Experiments/4/main}

\include{chapters/6Conclusion/main}

\appendix
\include{appendix/AppendixB}
\include{appendix/D/main}
\include{appendix/AppendixC}

\backmatter
\bibliographystyle{ecs}
\bibliography{ECS}
\end{document}
%% ----------------------------------------------------------------


\appendix
\include{appendix/AppendixB}
 %% ----------------------------------------------------------------
%% Progress.tex
%% ---------------------------------------------------------------- 
\documentclass{ecsprogress}    % Use the progress Style
\graphicspath{{../figs/}}   % Location of your graphics files
    \usepackage{natbib}            % Use Natbib style for the refs.
\hypersetup{colorlinks=true}   % Set to false for black/white printing
\input{Definitions}            % Include your abbreviations



\usepackage{enumitem}% http://ctan.org/pkg/enumitem
\usepackage{multirow}
\usepackage{float}
\usepackage{amsmath}
\usepackage{multicol}
\usepackage{amssymb}
\usepackage[normalem]{ulem}
\useunder{\uline}{\ul}{}
\usepackage{wrapfig}


\usepackage[table,xcdraw]{xcolor}


%% ----------------------------------------------------------------
\begin{document}
\frontmatter
\title      {Heterogeneous Agent-based Model for Supermarket Competition}
\authors    {\texorpdfstring
             {\href{mailto:sc22g13@ecs.soton.ac.uk}{Stefan J. Collier}}
             {Stefan J. Collier}
            }
\addresses  {\groupname\\\deptname\\\univname}
\date       {\today}
\subject    {}
\keywords   {}
\supervisor {Dr. Maria Polukarov}
\examiner   {Professor Sheng Chen}

\maketitle
\begin{abstract}
This project aim was to model and analyse the effects of competitive pricing behaviors of grocery retailers on the British market. 

This was achieved by creating a multi-agent model, containing retailer and consumer agents. The heterogeneous crowd of retailers employs either a uniform pricing strategy or a ‘local price flexing’ strategy. The actions of these retailers are chosen by predicting the profit of each action, using a perceptron. Following on from the consideration of different economic models, a discrete model was developed so that software agents have a discrete environment to operate within. Within the model, it has been observed how supermarkets with differing behaviors affect a heterogeneous crowd of consumer agents. The model was implemented in Java with Python used to evaluate the results. 

The simulation displays good acceptance with real grocery market behavior, i.e. captures the performance of British retailers thus can be used to determine the impact of changes in their behavior on their competitors and consumers.Furthermore it can be used to provide insight into sustainability of volatile pricing strategies, providing a useful insight in volatility of British supermarket retail industry. 
\end{abstract}
\acknowledgements{
I would like to express my sincere gratitude to Dr Maria Polukarov for her guidance and support which provided me the freedom to take this research in the direction of my interest.\\
\\
I would also like to thank my family and friends for their encouragement and support. To those who quietly listened to my software complaints. To those who worked throughout the nights with me. To those who helped me write what I couldn't say. I cannot thank you enough.
}

\declaration{
I, Stefan Collier, declare that this dissertation and the work presented in it are my own and has been generated by me as the result of my own original research.\\
I confirm that:\\
1. This work was done wholly or mainly while in candidature for a degree at this University;\\
2. Where any part of this dissertation has previously been submitted for any other qualification at this University or any other institution, this has been clearly stated;\\
3. Where I have consulted the published work of others, this is always clearly attributed;\\
4. Where I have quoted from the work of others, the source is always given. With the exception of such quotations, this dissertation is entirely my own work;\\
5. I have acknowledged all main sources of help;\\
6. Where the thesis is based on work done by myself jointly with others, I have made clear exactly what was done by others and what I have contributed myself;\\
7. Either none of this work has been published before submission, or parts of this work have been published by :\\
\\
Stefan Collier\\
April 2016
}
\tableofcontents
\listoffigures
\listoftables

\mainmatter
%% ----------------------------------------------------------------
%\include{Introduction}
%\include{Conclusions}
\include{chapters/1Project/main}
\include{chapters/2Lit/main}
\include{chapters/3Design/HighLevel}
\include{chapters/3Design/InDepth}
\include{chapters/4Impl/main}

\include{chapters/5Experiments/1/main}
\include{chapters/5Experiments/2/main}
\include{chapters/5Experiments/3/main}
\include{chapters/5Experiments/4/main}

\include{chapters/6Conclusion/main}

\appendix
\include{appendix/AppendixB}
\include{appendix/D/main}
\include{appendix/AppendixC}

\backmatter
\bibliographystyle{ecs}
\bibliography{ECS}
\end{document}
%% ----------------------------------------------------------------

\include{appendix/AppendixC}

\backmatter
\bibliographystyle{ecs}
\bibliography{ECS}
\end{document}
%% ----------------------------------------------------------------


 %% ----------------------------------------------------------------
%% Progress.tex
%% ---------------------------------------------------------------- 
\documentclass{ecsprogress}    % Use the progress Style
\graphicspath{{../figs/}}   % Location of your graphics files
    \usepackage{natbib}            % Use Natbib style for the refs.
\hypersetup{colorlinks=true}   % Set to false for black/white printing
\input{Definitions}            % Include your abbreviations



\usepackage{enumitem}% http://ctan.org/pkg/enumitem
\usepackage{multirow}
\usepackage{float}
\usepackage{amsmath}
\usepackage{multicol}
\usepackage{amssymb}
\usepackage[normalem]{ulem}
\useunder{\uline}{\ul}{}
\usepackage{wrapfig}


\usepackage[table,xcdraw]{xcolor}


%% ----------------------------------------------------------------
\begin{document}
\frontmatter
\title      {Heterogeneous Agent-based Model for Supermarket Competition}
\authors    {\texorpdfstring
             {\href{mailto:sc22g13@ecs.soton.ac.uk}{Stefan J. Collier}}
             {Stefan J. Collier}
            }
\addresses  {\groupname\\\deptname\\\univname}
\date       {\today}
\subject    {}
\keywords   {}
\supervisor {Dr. Maria Polukarov}
\examiner   {Professor Sheng Chen}

\maketitle
\begin{abstract}
This project aim was to model and analyse the effects of competitive pricing behaviors of grocery retailers on the British market. 

This was achieved by creating a multi-agent model, containing retailer and consumer agents. The heterogeneous crowd of retailers employs either a uniform pricing strategy or a ‘local price flexing’ strategy. The actions of these retailers are chosen by predicting the profit of each action, using a perceptron. Following on from the consideration of different economic models, a discrete model was developed so that software agents have a discrete environment to operate within. Within the model, it has been observed how supermarkets with differing behaviors affect a heterogeneous crowd of consumer agents. The model was implemented in Java with Python used to evaluate the results. 

The simulation displays good acceptance with real grocery market behavior, i.e. captures the performance of British retailers thus can be used to determine the impact of changes in their behavior on their competitors and consumers.Furthermore it can be used to provide insight into sustainability of volatile pricing strategies, providing a useful insight in volatility of British supermarket retail industry. 
\end{abstract}
\acknowledgements{
I would like to express my sincere gratitude to Dr Maria Polukarov for her guidance and support which provided me the freedom to take this research in the direction of my interest.\\
\\
I would also like to thank my family and friends for their encouragement and support. To those who quietly listened to my software complaints. To those who worked throughout the nights with me. To those who helped me write what I couldn't say. I cannot thank you enough.
}

\declaration{
I, Stefan Collier, declare that this dissertation and the work presented in it are my own and has been generated by me as the result of my own original research.\\
I confirm that:\\
1. This work was done wholly or mainly while in candidature for a degree at this University;\\
2. Where any part of this dissertation has previously been submitted for any other qualification at this University or any other institution, this has been clearly stated;\\
3. Where I have consulted the published work of others, this is always clearly attributed;\\
4. Where I have quoted from the work of others, the source is always given. With the exception of such quotations, this dissertation is entirely my own work;\\
5. I have acknowledged all main sources of help;\\
6. Where the thesis is based on work done by myself jointly with others, I have made clear exactly what was done by others and what I have contributed myself;\\
7. Either none of this work has been published before submission, or parts of this work have been published by :\\
\\
Stefan Collier\\
April 2016
}
\tableofcontents
\listoffigures
\listoftables

\mainmatter
%% ----------------------------------------------------------------
%\include{Introduction}
%\include{Conclusions}
 %% ----------------------------------------------------------------
%% Progress.tex
%% ---------------------------------------------------------------- 
\documentclass{ecsprogress}    % Use the progress Style
\graphicspath{{../figs/}}   % Location of your graphics files
    \usepackage{natbib}            % Use Natbib style for the refs.
\hypersetup{colorlinks=true}   % Set to false for black/white printing
\input{Definitions}            % Include your abbreviations



\usepackage{enumitem}% http://ctan.org/pkg/enumitem
\usepackage{multirow}
\usepackage{float}
\usepackage{amsmath}
\usepackage{multicol}
\usepackage{amssymb}
\usepackage[normalem]{ulem}
\useunder{\uline}{\ul}{}
\usepackage{wrapfig}


\usepackage[table,xcdraw]{xcolor}


%% ----------------------------------------------------------------
\begin{document}
\frontmatter
\title      {Heterogeneous Agent-based Model for Supermarket Competition}
\authors    {\texorpdfstring
             {\href{mailto:sc22g13@ecs.soton.ac.uk}{Stefan J. Collier}}
             {Stefan J. Collier}
            }
\addresses  {\groupname\\\deptname\\\univname}
\date       {\today}
\subject    {}
\keywords   {}
\supervisor {Dr. Maria Polukarov}
\examiner   {Professor Sheng Chen}

\maketitle
\begin{abstract}
This project aim was to model and analyse the effects of competitive pricing behaviors of grocery retailers on the British market. 

This was achieved by creating a multi-agent model, containing retailer and consumer agents. The heterogeneous crowd of retailers employs either a uniform pricing strategy or a ‘local price flexing’ strategy. The actions of these retailers are chosen by predicting the profit of each action, using a perceptron. Following on from the consideration of different economic models, a discrete model was developed so that software agents have a discrete environment to operate within. Within the model, it has been observed how supermarkets with differing behaviors affect a heterogeneous crowd of consumer agents. The model was implemented in Java with Python used to evaluate the results. 

The simulation displays good acceptance with real grocery market behavior, i.e. captures the performance of British retailers thus can be used to determine the impact of changes in their behavior on their competitors and consumers.Furthermore it can be used to provide insight into sustainability of volatile pricing strategies, providing a useful insight in volatility of British supermarket retail industry. 
\end{abstract}
\acknowledgements{
I would like to express my sincere gratitude to Dr Maria Polukarov for her guidance and support which provided me the freedom to take this research in the direction of my interest.\\
\\
I would also like to thank my family and friends for their encouragement and support. To those who quietly listened to my software complaints. To those who worked throughout the nights with me. To those who helped me write what I couldn't say. I cannot thank you enough.
}

\declaration{
I, Stefan Collier, declare that this dissertation and the work presented in it are my own and has been generated by me as the result of my own original research.\\
I confirm that:\\
1. This work was done wholly or mainly while in candidature for a degree at this University;\\
2. Where any part of this dissertation has previously been submitted for any other qualification at this University or any other institution, this has been clearly stated;\\
3. Where I have consulted the published work of others, this is always clearly attributed;\\
4. Where I have quoted from the work of others, the source is always given. With the exception of such quotations, this dissertation is entirely my own work;\\
5. I have acknowledged all main sources of help;\\
6. Where the thesis is based on work done by myself jointly with others, I have made clear exactly what was done by others and what I have contributed myself;\\
7. Either none of this work has been published before submission, or parts of this work have been published by :\\
\\
Stefan Collier\\
April 2016
}
\tableofcontents
\listoffigures
\listoftables

\mainmatter
%% ----------------------------------------------------------------
%\include{Introduction}
%\include{Conclusions}
\include{chapters/1Project/main}
\include{chapters/2Lit/main}
\include{chapters/3Design/HighLevel}
\include{chapters/3Design/InDepth}
\include{chapters/4Impl/main}

\include{chapters/5Experiments/1/main}
\include{chapters/5Experiments/2/main}
\include{chapters/5Experiments/3/main}
\include{chapters/5Experiments/4/main}

\include{chapters/6Conclusion/main}

\appendix
\include{appendix/AppendixB}
\include{appendix/D/main}
\include{appendix/AppendixC}

\backmatter
\bibliographystyle{ecs}
\bibliography{ECS}
\end{document}
%% ----------------------------------------------------------------

 %% ----------------------------------------------------------------
%% Progress.tex
%% ---------------------------------------------------------------- 
\documentclass{ecsprogress}    % Use the progress Style
\graphicspath{{../figs/}}   % Location of your graphics files
    \usepackage{natbib}            % Use Natbib style for the refs.
\hypersetup{colorlinks=true}   % Set to false for black/white printing
\input{Definitions}            % Include your abbreviations



\usepackage{enumitem}% http://ctan.org/pkg/enumitem
\usepackage{multirow}
\usepackage{float}
\usepackage{amsmath}
\usepackage{multicol}
\usepackage{amssymb}
\usepackage[normalem]{ulem}
\useunder{\uline}{\ul}{}
\usepackage{wrapfig}


\usepackage[table,xcdraw]{xcolor}


%% ----------------------------------------------------------------
\begin{document}
\frontmatter
\title      {Heterogeneous Agent-based Model for Supermarket Competition}
\authors    {\texorpdfstring
             {\href{mailto:sc22g13@ecs.soton.ac.uk}{Stefan J. Collier}}
             {Stefan J. Collier}
            }
\addresses  {\groupname\\\deptname\\\univname}
\date       {\today}
\subject    {}
\keywords   {}
\supervisor {Dr. Maria Polukarov}
\examiner   {Professor Sheng Chen}

\maketitle
\begin{abstract}
This project aim was to model and analyse the effects of competitive pricing behaviors of grocery retailers on the British market. 

This was achieved by creating a multi-agent model, containing retailer and consumer agents. The heterogeneous crowd of retailers employs either a uniform pricing strategy or a ‘local price flexing’ strategy. The actions of these retailers are chosen by predicting the profit of each action, using a perceptron. Following on from the consideration of different economic models, a discrete model was developed so that software agents have a discrete environment to operate within. Within the model, it has been observed how supermarkets with differing behaviors affect a heterogeneous crowd of consumer agents. The model was implemented in Java with Python used to evaluate the results. 

The simulation displays good acceptance with real grocery market behavior, i.e. captures the performance of British retailers thus can be used to determine the impact of changes in their behavior on their competitors and consumers.Furthermore it can be used to provide insight into sustainability of volatile pricing strategies, providing a useful insight in volatility of British supermarket retail industry. 
\end{abstract}
\acknowledgements{
I would like to express my sincere gratitude to Dr Maria Polukarov for her guidance and support which provided me the freedom to take this research in the direction of my interest.\\
\\
I would also like to thank my family and friends for their encouragement and support. To those who quietly listened to my software complaints. To those who worked throughout the nights with me. To those who helped me write what I couldn't say. I cannot thank you enough.
}

\declaration{
I, Stefan Collier, declare that this dissertation and the work presented in it are my own and has been generated by me as the result of my own original research.\\
I confirm that:\\
1. This work was done wholly or mainly while in candidature for a degree at this University;\\
2. Where any part of this dissertation has previously been submitted for any other qualification at this University or any other institution, this has been clearly stated;\\
3. Where I have consulted the published work of others, this is always clearly attributed;\\
4. Where I have quoted from the work of others, the source is always given. With the exception of such quotations, this dissertation is entirely my own work;\\
5. I have acknowledged all main sources of help;\\
6. Where the thesis is based on work done by myself jointly with others, I have made clear exactly what was done by others and what I have contributed myself;\\
7. Either none of this work has been published before submission, or parts of this work have been published by :\\
\\
Stefan Collier\\
April 2016
}
\tableofcontents
\listoffigures
\listoftables

\mainmatter
%% ----------------------------------------------------------------
%\include{Introduction}
%\include{Conclusions}
\include{chapters/1Project/main}
\include{chapters/2Lit/main}
\include{chapters/3Design/HighLevel}
\include{chapters/3Design/InDepth}
\include{chapters/4Impl/main}

\include{chapters/5Experiments/1/main}
\include{chapters/5Experiments/2/main}
\include{chapters/5Experiments/3/main}
\include{chapters/5Experiments/4/main}

\include{chapters/6Conclusion/main}

\appendix
\include{appendix/AppendixB}
\include{appendix/D/main}
\include{appendix/AppendixC}

\backmatter
\bibliographystyle{ecs}
\bibliography{ECS}
\end{document}
%% ----------------------------------------------------------------

\include{chapters/3Design/HighLevel}
\include{chapters/3Design/InDepth}
 %% ----------------------------------------------------------------
%% Progress.tex
%% ---------------------------------------------------------------- 
\documentclass{ecsprogress}    % Use the progress Style
\graphicspath{{../figs/}}   % Location of your graphics files
    \usepackage{natbib}            % Use Natbib style for the refs.
\hypersetup{colorlinks=true}   % Set to false for black/white printing
\input{Definitions}            % Include your abbreviations



\usepackage{enumitem}% http://ctan.org/pkg/enumitem
\usepackage{multirow}
\usepackage{float}
\usepackage{amsmath}
\usepackage{multicol}
\usepackage{amssymb}
\usepackage[normalem]{ulem}
\useunder{\uline}{\ul}{}
\usepackage{wrapfig}


\usepackage[table,xcdraw]{xcolor}


%% ----------------------------------------------------------------
\begin{document}
\frontmatter
\title      {Heterogeneous Agent-based Model for Supermarket Competition}
\authors    {\texorpdfstring
             {\href{mailto:sc22g13@ecs.soton.ac.uk}{Stefan J. Collier}}
             {Stefan J. Collier}
            }
\addresses  {\groupname\\\deptname\\\univname}
\date       {\today}
\subject    {}
\keywords   {}
\supervisor {Dr. Maria Polukarov}
\examiner   {Professor Sheng Chen}

\maketitle
\begin{abstract}
This project aim was to model and analyse the effects of competitive pricing behaviors of grocery retailers on the British market. 

This was achieved by creating a multi-agent model, containing retailer and consumer agents. The heterogeneous crowd of retailers employs either a uniform pricing strategy or a ‘local price flexing’ strategy. The actions of these retailers are chosen by predicting the profit of each action, using a perceptron. Following on from the consideration of different economic models, a discrete model was developed so that software agents have a discrete environment to operate within. Within the model, it has been observed how supermarkets with differing behaviors affect a heterogeneous crowd of consumer agents. The model was implemented in Java with Python used to evaluate the results. 

The simulation displays good acceptance with real grocery market behavior, i.e. captures the performance of British retailers thus can be used to determine the impact of changes in their behavior on their competitors and consumers.Furthermore it can be used to provide insight into sustainability of volatile pricing strategies, providing a useful insight in volatility of British supermarket retail industry. 
\end{abstract}
\acknowledgements{
I would like to express my sincere gratitude to Dr Maria Polukarov for her guidance and support which provided me the freedom to take this research in the direction of my interest.\\
\\
I would also like to thank my family and friends for their encouragement and support. To those who quietly listened to my software complaints. To those who worked throughout the nights with me. To those who helped me write what I couldn't say. I cannot thank you enough.
}

\declaration{
I, Stefan Collier, declare that this dissertation and the work presented in it are my own and has been generated by me as the result of my own original research.\\
I confirm that:\\
1. This work was done wholly or mainly while in candidature for a degree at this University;\\
2. Where any part of this dissertation has previously been submitted for any other qualification at this University or any other institution, this has been clearly stated;\\
3. Where I have consulted the published work of others, this is always clearly attributed;\\
4. Where I have quoted from the work of others, the source is always given. With the exception of such quotations, this dissertation is entirely my own work;\\
5. I have acknowledged all main sources of help;\\
6. Where the thesis is based on work done by myself jointly with others, I have made clear exactly what was done by others and what I have contributed myself;\\
7. Either none of this work has been published before submission, or parts of this work have been published by :\\
\\
Stefan Collier\\
April 2016
}
\tableofcontents
\listoffigures
\listoftables

\mainmatter
%% ----------------------------------------------------------------
%\include{Introduction}
%\include{Conclusions}
\include{chapters/1Project/main}
\include{chapters/2Lit/main}
\include{chapters/3Design/HighLevel}
\include{chapters/3Design/InDepth}
\include{chapters/4Impl/main}

\include{chapters/5Experiments/1/main}
\include{chapters/5Experiments/2/main}
\include{chapters/5Experiments/3/main}
\include{chapters/5Experiments/4/main}

\include{chapters/6Conclusion/main}

\appendix
\include{appendix/AppendixB}
\include{appendix/D/main}
\include{appendix/AppendixC}

\backmatter
\bibliographystyle{ecs}
\bibliography{ECS}
\end{document}
%% ----------------------------------------------------------------


 %% ----------------------------------------------------------------
%% Progress.tex
%% ---------------------------------------------------------------- 
\documentclass{ecsprogress}    % Use the progress Style
\graphicspath{{../figs/}}   % Location of your graphics files
    \usepackage{natbib}            % Use Natbib style for the refs.
\hypersetup{colorlinks=true}   % Set to false for black/white printing
\input{Definitions}            % Include your abbreviations



\usepackage{enumitem}% http://ctan.org/pkg/enumitem
\usepackage{multirow}
\usepackage{float}
\usepackage{amsmath}
\usepackage{multicol}
\usepackage{amssymb}
\usepackage[normalem]{ulem}
\useunder{\uline}{\ul}{}
\usepackage{wrapfig}


\usepackage[table,xcdraw]{xcolor}


%% ----------------------------------------------------------------
\begin{document}
\frontmatter
\title      {Heterogeneous Agent-based Model for Supermarket Competition}
\authors    {\texorpdfstring
             {\href{mailto:sc22g13@ecs.soton.ac.uk}{Stefan J. Collier}}
             {Stefan J. Collier}
            }
\addresses  {\groupname\\\deptname\\\univname}
\date       {\today}
\subject    {}
\keywords   {}
\supervisor {Dr. Maria Polukarov}
\examiner   {Professor Sheng Chen}

\maketitle
\begin{abstract}
This project aim was to model and analyse the effects of competitive pricing behaviors of grocery retailers on the British market. 

This was achieved by creating a multi-agent model, containing retailer and consumer agents. The heterogeneous crowd of retailers employs either a uniform pricing strategy or a ‘local price flexing’ strategy. The actions of these retailers are chosen by predicting the profit of each action, using a perceptron. Following on from the consideration of different economic models, a discrete model was developed so that software agents have a discrete environment to operate within. Within the model, it has been observed how supermarkets with differing behaviors affect a heterogeneous crowd of consumer agents. The model was implemented in Java with Python used to evaluate the results. 

The simulation displays good acceptance with real grocery market behavior, i.e. captures the performance of British retailers thus can be used to determine the impact of changes in their behavior on their competitors and consumers.Furthermore it can be used to provide insight into sustainability of volatile pricing strategies, providing a useful insight in volatility of British supermarket retail industry. 
\end{abstract}
\acknowledgements{
I would like to express my sincere gratitude to Dr Maria Polukarov for her guidance and support which provided me the freedom to take this research in the direction of my interest.\\
\\
I would also like to thank my family and friends for their encouragement and support. To those who quietly listened to my software complaints. To those who worked throughout the nights with me. To those who helped me write what I couldn't say. I cannot thank you enough.
}

\declaration{
I, Stefan Collier, declare that this dissertation and the work presented in it are my own and has been generated by me as the result of my own original research.\\
I confirm that:\\
1. This work was done wholly or mainly while in candidature for a degree at this University;\\
2. Where any part of this dissertation has previously been submitted for any other qualification at this University or any other institution, this has been clearly stated;\\
3. Where I have consulted the published work of others, this is always clearly attributed;\\
4. Where I have quoted from the work of others, the source is always given. With the exception of such quotations, this dissertation is entirely my own work;\\
5. I have acknowledged all main sources of help;\\
6. Where the thesis is based on work done by myself jointly with others, I have made clear exactly what was done by others and what I have contributed myself;\\
7. Either none of this work has been published before submission, or parts of this work have been published by :\\
\\
Stefan Collier\\
April 2016
}
\tableofcontents
\listoffigures
\listoftables

\mainmatter
%% ----------------------------------------------------------------
%\include{Introduction}
%\include{Conclusions}
\include{chapters/1Project/main}
\include{chapters/2Lit/main}
\include{chapters/3Design/HighLevel}
\include{chapters/3Design/InDepth}
\include{chapters/4Impl/main}

\include{chapters/5Experiments/1/main}
\include{chapters/5Experiments/2/main}
\include{chapters/5Experiments/3/main}
\include{chapters/5Experiments/4/main}

\include{chapters/6Conclusion/main}

\appendix
\include{appendix/AppendixB}
\include{appendix/D/main}
\include{appendix/AppendixC}

\backmatter
\bibliographystyle{ecs}
\bibliography{ECS}
\end{document}
%% ----------------------------------------------------------------

 %% ----------------------------------------------------------------
%% Progress.tex
%% ---------------------------------------------------------------- 
\documentclass{ecsprogress}    % Use the progress Style
\graphicspath{{../figs/}}   % Location of your graphics files
    \usepackage{natbib}            % Use Natbib style for the refs.
\hypersetup{colorlinks=true}   % Set to false for black/white printing
\input{Definitions}            % Include your abbreviations



\usepackage{enumitem}% http://ctan.org/pkg/enumitem
\usepackage{multirow}
\usepackage{float}
\usepackage{amsmath}
\usepackage{multicol}
\usepackage{amssymb}
\usepackage[normalem]{ulem}
\useunder{\uline}{\ul}{}
\usepackage{wrapfig}


\usepackage[table,xcdraw]{xcolor}


%% ----------------------------------------------------------------
\begin{document}
\frontmatter
\title      {Heterogeneous Agent-based Model for Supermarket Competition}
\authors    {\texorpdfstring
             {\href{mailto:sc22g13@ecs.soton.ac.uk}{Stefan J. Collier}}
             {Stefan J. Collier}
            }
\addresses  {\groupname\\\deptname\\\univname}
\date       {\today}
\subject    {}
\keywords   {}
\supervisor {Dr. Maria Polukarov}
\examiner   {Professor Sheng Chen}

\maketitle
\begin{abstract}
This project aim was to model and analyse the effects of competitive pricing behaviors of grocery retailers on the British market. 

This was achieved by creating a multi-agent model, containing retailer and consumer agents. The heterogeneous crowd of retailers employs either a uniform pricing strategy or a ‘local price flexing’ strategy. The actions of these retailers are chosen by predicting the profit of each action, using a perceptron. Following on from the consideration of different economic models, a discrete model was developed so that software agents have a discrete environment to operate within. Within the model, it has been observed how supermarkets with differing behaviors affect a heterogeneous crowd of consumer agents. The model was implemented in Java with Python used to evaluate the results. 

The simulation displays good acceptance with real grocery market behavior, i.e. captures the performance of British retailers thus can be used to determine the impact of changes in their behavior on their competitors and consumers.Furthermore it can be used to provide insight into sustainability of volatile pricing strategies, providing a useful insight in volatility of British supermarket retail industry. 
\end{abstract}
\acknowledgements{
I would like to express my sincere gratitude to Dr Maria Polukarov for her guidance and support which provided me the freedom to take this research in the direction of my interest.\\
\\
I would also like to thank my family and friends for their encouragement and support. To those who quietly listened to my software complaints. To those who worked throughout the nights with me. To those who helped me write what I couldn't say. I cannot thank you enough.
}

\declaration{
I, Stefan Collier, declare that this dissertation and the work presented in it are my own and has been generated by me as the result of my own original research.\\
I confirm that:\\
1. This work was done wholly or mainly while in candidature for a degree at this University;\\
2. Where any part of this dissertation has previously been submitted for any other qualification at this University or any other institution, this has been clearly stated;\\
3. Where I have consulted the published work of others, this is always clearly attributed;\\
4. Where I have quoted from the work of others, the source is always given. With the exception of such quotations, this dissertation is entirely my own work;\\
5. I have acknowledged all main sources of help;\\
6. Where the thesis is based on work done by myself jointly with others, I have made clear exactly what was done by others and what I have contributed myself;\\
7. Either none of this work has been published before submission, or parts of this work have been published by :\\
\\
Stefan Collier\\
April 2016
}
\tableofcontents
\listoffigures
\listoftables

\mainmatter
%% ----------------------------------------------------------------
%\include{Introduction}
%\include{Conclusions}
\include{chapters/1Project/main}
\include{chapters/2Lit/main}
\include{chapters/3Design/HighLevel}
\include{chapters/3Design/InDepth}
\include{chapters/4Impl/main}

\include{chapters/5Experiments/1/main}
\include{chapters/5Experiments/2/main}
\include{chapters/5Experiments/3/main}
\include{chapters/5Experiments/4/main}

\include{chapters/6Conclusion/main}

\appendix
\include{appendix/AppendixB}
\include{appendix/D/main}
\include{appendix/AppendixC}

\backmatter
\bibliographystyle{ecs}
\bibliography{ECS}
\end{document}
%% ----------------------------------------------------------------

 %% ----------------------------------------------------------------
%% Progress.tex
%% ---------------------------------------------------------------- 
\documentclass{ecsprogress}    % Use the progress Style
\graphicspath{{../figs/}}   % Location of your graphics files
    \usepackage{natbib}            % Use Natbib style for the refs.
\hypersetup{colorlinks=true}   % Set to false for black/white printing
\input{Definitions}            % Include your abbreviations



\usepackage{enumitem}% http://ctan.org/pkg/enumitem
\usepackage{multirow}
\usepackage{float}
\usepackage{amsmath}
\usepackage{multicol}
\usepackage{amssymb}
\usepackage[normalem]{ulem}
\useunder{\uline}{\ul}{}
\usepackage{wrapfig}


\usepackage[table,xcdraw]{xcolor}


%% ----------------------------------------------------------------
\begin{document}
\frontmatter
\title      {Heterogeneous Agent-based Model for Supermarket Competition}
\authors    {\texorpdfstring
             {\href{mailto:sc22g13@ecs.soton.ac.uk}{Stefan J. Collier}}
             {Stefan J. Collier}
            }
\addresses  {\groupname\\\deptname\\\univname}
\date       {\today}
\subject    {}
\keywords   {}
\supervisor {Dr. Maria Polukarov}
\examiner   {Professor Sheng Chen}

\maketitle
\begin{abstract}
This project aim was to model and analyse the effects of competitive pricing behaviors of grocery retailers on the British market. 

This was achieved by creating a multi-agent model, containing retailer and consumer agents. The heterogeneous crowd of retailers employs either a uniform pricing strategy or a ‘local price flexing’ strategy. The actions of these retailers are chosen by predicting the profit of each action, using a perceptron. Following on from the consideration of different economic models, a discrete model was developed so that software agents have a discrete environment to operate within. Within the model, it has been observed how supermarkets with differing behaviors affect a heterogeneous crowd of consumer agents. The model was implemented in Java with Python used to evaluate the results. 

The simulation displays good acceptance with real grocery market behavior, i.e. captures the performance of British retailers thus can be used to determine the impact of changes in their behavior on their competitors and consumers.Furthermore it can be used to provide insight into sustainability of volatile pricing strategies, providing a useful insight in volatility of British supermarket retail industry. 
\end{abstract}
\acknowledgements{
I would like to express my sincere gratitude to Dr Maria Polukarov for her guidance and support which provided me the freedom to take this research in the direction of my interest.\\
\\
I would also like to thank my family and friends for their encouragement and support. To those who quietly listened to my software complaints. To those who worked throughout the nights with me. To those who helped me write what I couldn't say. I cannot thank you enough.
}

\declaration{
I, Stefan Collier, declare that this dissertation and the work presented in it are my own and has been generated by me as the result of my own original research.\\
I confirm that:\\
1. This work was done wholly or mainly while in candidature for a degree at this University;\\
2. Where any part of this dissertation has previously been submitted for any other qualification at this University or any other institution, this has been clearly stated;\\
3. Where I have consulted the published work of others, this is always clearly attributed;\\
4. Where I have quoted from the work of others, the source is always given. With the exception of such quotations, this dissertation is entirely my own work;\\
5. I have acknowledged all main sources of help;\\
6. Where the thesis is based on work done by myself jointly with others, I have made clear exactly what was done by others and what I have contributed myself;\\
7. Either none of this work has been published before submission, or parts of this work have been published by :\\
\\
Stefan Collier\\
April 2016
}
\tableofcontents
\listoffigures
\listoftables

\mainmatter
%% ----------------------------------------------------------------
%\include{Introduction}
%\include{Conclusions}
\include{chapters/1Project/main}
\include{chapters/2Lit/main}
\include{chapters/3Design/HighLevel}
\include{chapters/3Design/InDepth}
\include{chapters/4Impl/main}

\include{chapters/5Experiments/1/main}
\include{chapters/5Experiments/2/main}
\include{chapters/5Experiments/3/main}
\include{chapters/5Experiments/4/main}

\include{chapters/6Conclusion/main}

\appendix
\include{appendix/AppendixB}
\include{appendix/D/main}
\include{appendix/AppendixC}

\backmatter
\bibliographystyle{ecs}
\bibliography{ECS}
\end{document}
%% ----------------------------------------------------------------

 %% ----------------------------------------------------------------
%% Progress.tex
%% ---------------------------------------------------------------- 
\documentclass{ecsprogress}    % Use the progress Style
\graphicspath{{../figs/}}   % Location of your graphics files
    \usepackage{natbib}            % Use Natbib style for the refs.
\hypersetup{colorlinks=true}   % Set to false for black/white printing
\input{Definitions}            % Include your abbreviations



\usepackage{enumitem}% http://ctan.org/pkg/enumitem
\usepackage{multirow}
\usepackage{float}
\usepackage{amsmath}
\usepackage{multicol}
\usepackage{amssymb}
\usepackage[normalem]{ulem}
\useunder{\uline}{\ul}{}
\usepackage{wrapfig}


\usepackage[table,xcdraw]{xcolor}


%% ----------------------------------------------------------------
\begin{document}
\frontmatter
\title      {Heterogeneous Agent-based Model for Supermarket Competition}
\authors    {\texorpdfstring
             {\href{mailto:sc22g13@ecs.soton.ac.uk}{Stefan J. Collier}}
             {Stefan J. Collier}
            }
\addresses  {\groupname\\\deptname\\\univname}
\date       {\today}
\subject    {}
\keywords   {}
\supervisor {Dr. Maria Polukarov}
\examiner   {Professor Sheng Chen}

\maketitle
\begin{abstract}
This project aim was to model and analyse the effects of competitive pricing behaviors of grocery retailers on the British market. 

This was achieved by creating a multi-agent model, containing retailer and consumer agents. The heterogeneous crowd of retailers employs either a uniform pricing strategy or a ‘local price flexing’ strategy. The actions of these retailers are chosen by predicting the profit of each action, using a perceptron. Following on from the consideration of different economic models, a discrete model was developed so that software agents have a discrete environment to operate within. Within the model, it has been observed how supermarkets with differing behaviors affect a heterogeneous crowd of consumer agents. The model was implemented in Java with Python used to evaluate the results. 

The simulation displays good acceptance with real grocery market behavior, i.e. captures the performance of British retailers thus can be used to determine the impact of changes in their behavior on their competitors and consumers.Furthermore it can be used to provide insight into sustainability of volatile pricing strategies, providing a useful insight in volatility of British supermarket retail industry. 
\end{abstract}
\acknowledgements{
I would like to express my sincere gratitude to Dr Maria Polukarov for her guidance and support which provided me the freedom to take this research in the direction of my interest.\\
\\
I would also like to thank my family and friends for their encouragement and support. To those who quietly listened to my software complaints. To those who worked throughout the nights with me. To those who helped me write what I couldn't say. I cannot thank you enough.
}

\declaration{
I, Stefan Collier, declare that this dissertation and the work presented in it are my own and has been generated by me as the result of my own original research.\\
I confirm that:\\
1. This work was done wholly or mainly while in candidature for a degree at this University;\\
2. Where any part of this dissertation has previously been submitted for any other qualification at this University or any other institution, this has been clearly stated;\\
3. Where I have consulted the published work of others, this is always clearly attributed;\\
4. Where I have quoted from the work of others, the source is always given. With the exception of such quotations, this dissertation is entirely my own work;\\
5. I have acknowledged all main sources of help;\\
6. Where the thesis is based on work done by myself jointly with others, I have made clear exactly what was done by others and what I have contributed myself;\\
7. Either none of this work has been published before submission, or parts of this work have been published by :\\
\\
Stefan Collier\\
April 2016
}
\tableofcontents
\listoffigures
\listoftables

\mainmatter
%% ----------------------------------------------------------------
%\include{Introduction}
%\include{Conclusions}
\include{chapters/1Project/main}
\include{chapters/2Lit/main}
\include{chapters/3Design/HighLevel}
\include{chapters/3Design/InDepth}
\include{chapters/4Impl/main}

\include{chapters/5Experiments/1/main}
\include{chapters/5Experiments/2/main}
\include{chapters/5Experiments/3/main}
\include{chapters/5Experiments/4/main}

\include{chapters/6Conclusion/main}

\appendix
\include{appendix/AppendixB}
\include{appendix/D/main}
\include{appendix/AppendixC}

\backmatter
\bibliographystyle{ecs}
\bibliography{ECS}
\end{document}
%% ----------------------------------------------------------------


 %% ----------------------------------------------------------------
%% Progress.tex
%% ---------------------------------------------------------------- 
\documentclass{ecsprogress}    % Use the progress Style
\graphicspath{{../figs/}}   % Location of your graphics files
    \usepackage{natbib}            % Use Natbib style for the refs.
\hypersetup{colorlinks=true}   % Set to false for black/white printing
\input{Definitions}            % Include your abbreviations



\usepackage{enumitem}% http://ctan.org/pkg/enumitem
\usepackage{multirow}
\usepackage{float}
\usepackage{amsmath}
\usepackage{multicol}
\usepackage{amssymb}
\usepackage[normalem]{ulem}
\useunder{\uline}{\ul}{}
\usepackage{wrapfig}


\usepackage[table,xcdraw]{xcolor}


%% ----------------------------------------------------------------
\begin{document}
\frontmatter
\title      {Heterogeneous Agent-based Model for Supermarket Competition}
\authors    {\texorpdfstring
             {\href{mailto:sc22g13@ecs.soton.ac.uk}{Stefan J. Collier}}
             {Stefan J. Collier}
            }
\addresses  {\groupname\\\deptname\\\univname}
\date       {\today}
\subject    {}
\keywords   {}
\supervisor {Dr. Maria Polukarov}
\examiner   {Professor Sheng Chen}

\maketitle
\begin{abstract}
This project aim was to model and analyse the effects of competitive pricing behaviors of grocery retailers on the British market. 

This was achieved by creating a multi-agent model, containing retailer and consumer agents. The heterogeneous crowd of retailers employs either a uniform pricing strategy or a ‘local price flexing’ strategy. The actions of these retailers are chosen by predicting the profit of each action, using a perceptron. Following on from the consideration of different economic models, a discrete model was developed so that software agents have a discrete environment to operate within. Within the model, it has been observed how supermarkets with differing behaviors affect a heterogeneous crowd of consumer agents. The model was implemented in Java with Python used to evaluate the results. 

The simulation displays good acceptance with real grocery market behavior, i.e. captures the performance of British retailers thus can be used to determine the impact of changes in their behavior on their competitors and consumers.Furthermore it can be used to provide insight into sustainability of volatile pricing strategies, providing a useful insight in volatility of British supermarket retail industry. 
\end{abstract}
\acknowledgements{
I would like to express my sincere gratitude to Dr Maria Polukarov for her guidance and support which provided me the freedom to take this research in the direction of my interest.\\
\\
I would also like to thank my family and friends for their encouragement and support. To those who quietly listened to my software complaints. To those who worked throughout the nights with me. To those who helped me write what I couldn't say. I cannot thank you enough.
}

\declaration{
I, Stefan Collier, declare that this dissertation and the work presented in it are my own and has been generated by me as the result of my own original research.\\
I confirm that:\\
1. This work was done wholly or mainly while in candidature for a degree at this University;\\
2. Where any part of this dissertation has previously been submitted for any other qualification at this University or any other institution, this has been clearly stated;\\
3. Where I have consulted the published work of others, this is always clearly attributed;\\
4. Where I have quoted from the work of others, the source is always given. With the exception of such quotations, this dissertation is entirely my own work;\\
5. I have acknowledged all main sources of help;\\
6. Where the thesis is based on work done by myself jointly with others, I have made clear exactly what was done by others and what I have contributed myself;\\
7. Either none of this work has been published before submission, or parts of this work have been published by :\\
\\
Stefan Collier\\
April 2016
}
\tableofcontents
\listoffigures
\listoftables

\mainmatter
%% ----------------------------------------------------------------
%\include{Introduction}
%\include{Conclusions}
\include{chapters/1Project/main}
\include{chapters/2Lit/main}
\include{chapters/3Design/HighLevel}
\include{chapters/3Design/InDepth}
\include{chapters/4Impl/main}

\include{chapters/5Experiments/1/main}
\include{chapters/5Experiments/2/main}
\include{chapters/5Experiments/3/main}
\include{chapters/5Experiments/4/main}

\include{chapters/6Conclusion/main}

\appendix
\include{appendix/AppendixB}
\include{appendix/D/main}
\include{appendix/AppendixC}

\backmatter
\bibliographystyle{ecs}
\bibliography{ECS}
\end{document}
%% ----------------------------------------------------------------


\appendix
\include{appendix/AppendixB}
 %% ----------------------------------------------------------------
%% Progress.tex
%% ---------------------------------------------------------------- 
\documentclass{ecsprogress}    % Use the progress Style
\graphicspath{{../figs/}}   % Location of your graphics files
    \usepackage{natbib}            % Use Natbib style for the refs.
\hypersetup{colorlinks=true}   % Set to false for black/white printing
\input{Definitions}            % Include your abbreviations



\usepackage{enumitem}% http://ctan.org/pkg/enumitem
\usepackage{multirow}
\usepackage{float}
\usepackage{amsmath}
\usepackage{multicol}
\usepackage{amssymb}
\usepackage[normalem]{ulem}
\useunder{\uline}{\ul}{}
\usepackage{wrapfig}


\usepackage[table,xcdraw]{xcolor}


%% ----------------------------------------------------------------
\begin{document}
\frontmatter
\title      {Heterogeneous Agent-based Model for Supermarket Competition}
\authors    {\texorpdfstring
             {\href{mailto:sc22g13@ecs.soton.ac.uk}{Stefan J. Collier}}
             {Stefan J. Collier}
            }
\addresses  {\groupname\\\deptname\\\univname}
\date       {\today}
\subject    {}
\keywords   {}
\supervisor {Dr. Maria Polukarov}
\examiner   {Professor Sheng Chen}

\maketitle
\begin{abstract}
This project aim was to model and analyse the effects of competitive pricing behaviors of grocery retailers on the British market. 

This was achieved by creating a multi-agent model, containing retailer and consumer agents. The heterogeneous crowd of retailers employs either a uniform pricing strategy or a ‘local price flexing’ strategy. The actions of these retailers are chosen by predicting the profit of each action, using a perceptron. Following on from the consideration of different economic models, a discrete model was developed so that software agents have a discrete environment to operate within. Within the model, it has been observed how supermarkets with differing behaviors affect a heterogeneous crowd of consumer agents. The model was implemented in Java with Python used to evaluate the results. 

The simulation displays good acceptance with real grocery market behavior, i.e. captures the performance of British retailers thus can be used to determine the impact of changes in their behavior on their competitors and consumers.Furthermore it can be used to provide insight into sustainability of volatile pricing strategies, providing a useful insight in volatility of British supermarket retail industry. 
\end{abstract}
\acknowledgements{
I would like to express my sincere gratitude to Dr Maria Polukarov for her guidance and support which provided me the freedom to take this research in the direction of my interest.\\
\\
I would also like to thank my family and friends for their encouragement and support. To those who quietly listened to my software complaints. To those who worked throughout the nights with me. To those who helped me write what I couldn't say. I cannot thank you enough.
}

\declaration{
I, Stefan Collier, declare that this dissertation and the work presented in it are my own and has been generated by me as the result of my own original research.\\
I confirm that:\\
1. This work was done wholly or mainly while in candidature for a degree at this University;\\
2. Where any part of this dissertation has previously been submitted for any other qualification at this University or any other institution, this has been clearly stated;\\
3. Where I have consulted the published work of others, this is always clearly attributed;\\
4. Where I have quoted from the work of others, the source is always given. With the exception of such quotations, this dissertation is entirely my own work;\\
5. I have acknowledged all main sources of help;\\
6. Where the thesis is based on work done by myself jointly with others, I have made clear exactly what was done by others and what I have contributed myself;\\
7. Either none of this work has been published before submission, or parts of this work have been published by :\\
\\
Stefan Collier\\
April 2016
}
\tableofcontents
\listoffigures
\listoftables

\mainmatter
%% ----------------------------------------------------------------
%\include{Introduction}
%\include{Conclusions}
\include{chapters/1Project/main}
\include{chapters/2Lit/main}
\include{chapters/3Design/HighLevel}
\include{chapters/3Design/InDepth}
\include{chapters/4Impl/main}

\include{chapters/5Experiments/1/main}
\include{chapters/5Experiments/2/main}
\include{chapters/5Experiments/3/main}
\include{chapters/5Experiments/4/main}

\include{chapters/6Conclusion/main}

\appendix
\include{appendix/AppendixB}
\include{appendix/D/main}
\include{appendix/AppendixC}

\backmatter
\bibliographystyle{ecs}
\bibliography{ECS}
\end{document}
%% ----------------------------------------------------------------

\include{appendix/AppendixC}

\backmatter
\bibliographystyle{ecs}
\bibliography{ECS}
\end{document}
%% ----------------------------------------------------------------

 %% ----------------------------------------------------------------
%% Progress.tex
%% ---------------------------------------------------------------- 
\documentclass{ecsprogress}    % Use the progress Style
\graphicspath{{../figs/}}   % Location of your graphics files
    \usepackage{natbib}            % Use Natbib style for the refs.
\hypersetup{colorlinks=true}   % Set to false for black/white printing
\input{Definitions}            % Include your abbreviations



\usepackage{enumitem}% http://ctan.org/pkg/enumitem
\usepackage{multirow}
\usepackage{float}
\usepackage{amsmath}
\usepackage{multicol}
\usepackage{amssymb}
\usepackage[normalem]{ulem}
\useunder{\uline}{\ul}{}
\usepackage{wrapfig}


\usepackage[table,xcdraw]{xcolor}


%% ----------------------------------------------------------------
\begin{document}
\frontmatter
\title      {Heterogeneous Agent-based Model for Supermarket Competition}
\authors    {\texorpdfstring
             {\href{mailto:sc22g13@ecs.soton.ac.uk}{Stefan J. Collier}}
             {Stefan J. Collier}
            }
\addresses  {\groupname\\\deptname\\\univname}
\date       {\today}
\subject    {}
\keywords   {}
\supervisor {Dr. Maria Polukarov}
\examiner   {Professor Sheng Chen}

\maketitle
\begin{abstract}
This project aim was to model and analyse the effects of competitive pricing behaviors of grocery retailers on the British market. 

This was achieved by creating a multi-agent model, containing retailer and consumer agents. The heterogeneous crowd of retailers employs either a uniform pricing strategy or a ‘local price flexing’ strategy. The actions of these retailers are chosen by predicting the profit of each action, using a perceptron. Following on from the consideration of different economic models, a discrete model was developed so that software agents have a discrete environment to operate within. Within the model, it has been observed how supermarkets with differing behaviors affect a heterogeneous crowd of consumer agents. The model was implemented in Java with Python used to evaluate the results. 

The simulation displays good acceptance with real grocery market behavior, i.e. captures the performance of British retailers thus can be used to determine the impact of changes in their behavior on their competitors and consumers.Furthermore it can be used to provide insight into sustainability of volatile pricing strategies, providing a useful insight in volatility of British supermarket retail industry. 
\end{abstract}
\acknowledgements{
I would like to express my sincere gratitude to Dr Maria Polukarov for her guidance and support which provided me the freedom to take this research in the direction of my interest.\\
\\
I would also like to thank my family and friends for their encouragement and support. To those who quietly listened to my software complaints. To those who worked throughout the nights with me. To those who helped me write what I couldn't say. I cannot thank you enough.
}

\declaration{
I, Stefan Collier, declare that this dissertation and the work presented in it are my own and has been generated by me as the result of my own original research.\\
I confirm that:\\
1. This work was done wholly or mainly while in candidature for a degree at this University;\\
2. Where any part of this dissertation has previously been submitted for any other qualification at this University or any other institution, this has been clearly stated;\\
3. Where I have consulted the published work of others, this is always clearly attributed;\\
4. Where I have quoted from the work of others, the source is always given. With the exception of such quotations, this dissertation is entirely my own work;\\
5. I have acknowledged all main sources of help;\\
6. Where the thesis is based on work done by myself jointly with others, I have made clear exactly what was done by others and what I have contributed myself;\\
7. Either none of this work has been published before submission, or parts of this work have been published by :\\
\\
Stefan Collier\\
April 2016
}
\tableofcontents
\listoffigures
\listoftables

\mainmatter
%% ----------------------------------------------------------------
%\include{Introduction}
%\include{Conclusions}
 %% ----------------------------------------------------------------
%% Progress.tex
%% ---------------------------------------------------------------- 
\documentclass{ecsprogress}    % Use the progress Style
\graphicspath{{../figs/}}   % Location of your graphics files
    \usepackage{natbib}            % Use Natbib style for the refs.
\hypersetup{colorlinks=true}   % Set to false for black/white printing
\input{Definitions}            % Include your abbreviations



\usepackage{enumitem}% http://ctan.org/pkg/enumitem
\usepackage{multirow}
\usepackage{float}
\usepackage{amsmath}
\usepackage{multicol}
\usepackage{amssymb}
\usepackage[normalem]{ulem}
\useunder{\uline}{\ul}{}
\usepackage{wrapfig}


\usepackage[table,xcdraw]{xcolor}


%% ----------------------------------------------------------------
\begin{document}
\frontmatter
\title      {Heterogeneous Agent-based Model for Supermarket Competition}
\authors    {\texorpdfstring
             {\href{mailto:sc22g13@ecs.soton.ac.uk}{Stefan J. Collier}}
             {Stefan J. Collier}
            }
\addresses  {\groupname\\\deptname\\\univname}
\date       {\today}
\subject    {}
\keywords   {}
\supervisor {Dr. Maria Polukarov}
\examiner   {Professor Sheng Chen}

\maketitle
\begin{abstract}
This project aim was to model and analyse the effects of competitive pricing behaviors of grocery retailers on the British market. 

This was achieved by creating a multi-agent model, containing retailer and consumer agents. The heterogeneous crowd of retailers employs either a uniform pricing strategy or a ‘local price flexing’ strategy. The actions of these retailers are chosen by predicting the profit of each action, using a perceptron. Following on from the consideration of different economic models, a discrete model was developed so that software agents have a discrete environment to operate within. Within the model, it has been observed how supermarkets with differing behaviors affect a heterogeneous crowd of consumer agents. The model was implemented in Java with Python used to evaluate the results. 

The simulation displays good acceptance with real grocery market behavior, i.e. captures the performance of British retailers thus can be used to determine the impact of changes in their behavior on their competitors and consumers.Furthermore it can be used to provide insight into sustainability of volatile pricing strategies, providing a useful insight in volatility of British supermarket retail industry. 
\end{abstract}
\acknowledgements{
I would like to express my sincere gratitude to Dr Maria Polukarov for her guidance and support which provided me the freedom to take this research in the direction of my interest.\\
\\
I would also like to thank my family and friends for their encouragement and support. To those who quietly listened to my software complaints. To those who worked throughout the nights with me. To those who helped me write what I couldn't say. I cannot thank you enough.
}

\declaration{
I, Stefan Collier, declare that this dissertation and the work presented in it are my own and has been generated by me as the result of my own original research.\\
I confirm that:\\
1. This work was done wholly or mainly while in candidature for a degree at this University;\\
2. Where any part of this dissertation has previously been submitted for any other qualification at this University or any other institution, this has been clearly stated;\\
3. Where I have consulted the published work of others, this is always clearly attributed;\\
4. Where I have quoted from the work of others, the source is always given. With the exception of such quotations, this dissertation is entirely my own work;\\
5. I have acknowledged all main sources of help;\\
6. Where the thesis is based on work done by myself jointly with others, I have made clear exactly what was done by others and what I have contributed myself;\\
7. Either none of this work has been published before submission, or parts of this work have been published by :\\
\\
Stefan Collier\\
April 2016
}
\tableofcontents
\listoffigures
\listoftables

\mainmatter
%% ----------------------------------------------------------------
%\include{Introduction}
%\include{Conclusions}
\include{chapters/1Project/main}
\include{chapters/2Lit/main}
\include{chapters/3Design/HighLevel}
\include{chapters/3Design/InDepth}
\include{chapters/4Impl/main}

\include{chapters/5Experiments/1/main}
\include{chapters/5Experiments/2/main}
\include{chapters/5Experiments/3/main}
\include{chapters/5Experiments/4/main}

\include{chapters/6Conclusion/main}

\appendix
\include{appendix/AppendixB}
\include{appendix/D/main}
\include{appendix/AppendixC}

\backmatter
\bibliographystyle{ecs}
\bibliography{ECS}
\end{document}
%% ----------------------------------------------------------------

 %% ----------------------------------------------------------------
%% Progress.tex
%% ---------------------------------------------------------------- 
\documentclass{ecsprogress}    % Use the progress Style
\graphicspath{{../figs/}}   % Location of your graphics files
    \usepackage{natbib}            % Use Natbib style for the refs.
\hypersetup{colorlinks=true}   % Set to false for black/white printing
\input{Definitions}            % Include your abbreviations



\usepackage{enumitem}% http://ctan.org/pkg/enumitem
\usepackage{multirow}
\usepackage{float}
\usepackage{amsmath}
\usepackage{multicol}
\usepackage{amssymb}
\usepackage[normalem]{ulem}
\useunder{\uline}{\ul}{}
\usepackage{wrapfig}


\usepackage[table,xcdraw]{xcolor}


%% ----------------------------------------------------------------
\begin{document}
\frontmatter
\title      {Heterogeneous Agent-based Model for Supermarket Competition}
\authors    {\texorpdfstring
             {\href{mailto:sc22g13@ecs.soton.ac.uk}{Stefan J. Collier}}
             {Stefan J. Collier}
            }
\addresses  {\groupname\\\deptname\\\univname}
\date       {\today}
\subject    {}
\keywords   {}
\supervisor {Dr. Maria Polukarov}
\examiner   {Professor Sheng Chen}

\maketitle
\begin{abstract}
This project aim was to model and analyse the effects of competitive pricing behaviors of grocery retailers on the British market. 

This was achieved by creating a multi-agent model, containing retailer and consumer agents. The heterogeneous crowd of retailers employs either a uniform pricing strategy or a ‘local price flexing’ strategy. The actions of these retailers are chosen by predicting the profit of each action, using a perceptron. Following on from the consideration of different economic models, a discrete model was developed so that software agents have a discrete environment to operate within. Within the model, it has been observed how supermarkets with differing behaviors affect a heterogeneous crowd of consumer agents. The model was implemented in Java with Python used to evaluate the results. 

The simulation displays good acceptance with real grocery market behavior, i.e. captures the performance of British retailers thus can be used to determine the impact of changes in their behavior on their competitors and consumers.Furthermore it can be used to provide insight into sustainability of volatile pricing strategies, providing a useful insight in volatility of British supermarket retail industry. 
\end{abstract}
\acknowledgements{
I would like to express my sincere gratitude to Dr Maria Polukarov for her guidance and support which provided me the freedom to take this research in the direction of my interest.\\
\\
I would also like to thank my family and friends for their encouragement and support. To those who quietly listened to my software complaints. To those who worked throughout the nights with me. To those who helped me write what I couldn't say. I cannot thank you enough.
}

\declaration{
I, Stefan Collier, declare that this dissertation and the work presented in it are my own and has been generated by me as the result of my own original research.\\
I confirm that:\\
1. This work was done wholly or mainly while in candidature for a degree at this University;\\
2. Where any part of this dissertation has previously been submitted for any other qualification at this University or any other institution, this has been clearly stated;\\
3. Where I have consulted the published work of others, this is always clearly attributed;\\
4. Where I have quoted from the work of others, the source is always given. With the exception of such quotations, this dissertation is entirely my own work;\\
5. I have acknowledged all main sources of help;\\
6. Where the thesis is based on work done by myself jointly with others, I have made clear exactly what was done by others and what I have contributed myself;\\
7. Either none of this work has been published before submission, or parts of this work have been published by :\\
\\
Stefan Collier\\
April 2016
}
\tableofcontents
\listoffigures
\listoftables

\mainmatter
%% ----------------------------------------------------------------
%\include{Introduction}
%\include{Conclusions}
\include{chapters/1Project/main}
\include{chapters/2Lit/main}
\include{chapters/3Design/HighLevel}
\include{chapters/3Design/InDepth}
\include{chapters/4Impl/main}

\include{chapters/5Experiments/1/main}
\include{chapters/5Experiments/2/main}
\include{chapters/5Experiments/3/main}
\include{chapters/5Experiments/4/main}

\include{chapters/6Conclusion/main}

\appendix
\include{appendix/AppendixB}
\include{appendix/D/main}
\include{appendix/AppendixC}

\backmatter
\bibliographystyle{ecs}
\bibliography{ECS}
\end{document}
%% ----------------------------------------------------------------

\include{chapters/3Design/HighLevel}
\include{chapters/3Design/InDepth}
 %% ----------------------------------------------------------------
%% Progress.tex
%% ---------------------------------------------------------------- 
\documentclass{ecsprogress}    % Use the progress Style
\graphicspath{{../figs/}}   % Location of your graphics files
    \usepackage{natbib}            % Use Natbib style for the refs.
\hypersetup{colorlinks=true}   % Set to false for black/white printing
\input{Definitions}            % Include your abbreviations



\usepackage{enumitem}% http://ctan.org/pkg/enumitem
\usepackage{multirow}
\usepackage{float}
\usepackage{amsmath}
\usepackage{multicol}
\usepackage{amssymb}
\usepackage[normalem]{ulem}
\useunder{\uline}{\ul}{}
\usepackage{wrapfig}


\usepackage[table,xcdraw]{xcolor}


%% ----------------------------------------------------------------
\begin{document}
\frontmatter
\title      {Heterogeneous Agent-based Model for Supermarket Competition}
\authors    {\texorpdfstring
             {\href{mailto:sc22g13@ecs.soton.ac.uk}{Stefan J. Collier}}
             {Stefan J. Collier}
            }
\addresses  {\groupname\\\deptname\\\univname}
\date       {\today}
\subject    {}
\keywords   {}
\supervisor {Dr. Maria Polukarov}
\examiner   {Professor Sheng Chen}

\maketitle
\begin{abstract}
This project aim was to model and analyse the effects of competitive pricing behaviors of grocery retailers on the British market. 

This was achieved by creating a multi-agent model, containing retailer and consumer agents. The heterogeneous crowd of retailers employs either a uniform pricing strategy or a ‘local price flexing’ strategy. The actions of these retailers are chosen by predicting the profit of each action, using a perceptron. Following on from the consideration of different economic models, a discrete model was developed so that software agents have a discrete environment to operate within. Within the model, it has been observed how supermarkets with differing behaviors affect a heterogeneous crowd of consumer agents. The model was implemented in Java with Python used to evaluate the results. 

The simulation displays good acceptance with real grocery market behavior, i.e. captures the performance of British retailers thus can be used to determine the impact of changes in their behavior on their competitors and consumers.Furthermore it can be used to provide insight into sustainability of volatile pricing strategies, providing a useful insight in volatility of British supermarket retail industry. 
\end{abstract}
\acknowledgements{
I would like to express my sincere gratitude to Dr Maria Polukarov for her guidance and support which provided me the freedom to take this research in the direction of my interest.\\
\\
I would also like to thank my family and friends for their encouragement and support. To those who quietly listened to my software complaints. To those who worked throughout the nights with me. To those who helped me write what I couldn't say. I cannot thank you enough.
}

\declaration{
I, Stefan Collier, declare that this dissertation and the work presented in it are my own and has been generated by me as the result of my own original research.\\
I confirm that:\\
1. This work was done wholly or mainly while in candidature for a degree at this University;\\
2. Where any part of this dissertation has previously been submitted for any other qualification at this University or any other institution, this has been clearly stated;\\
3. Where I have consulted the published work of others, this is always clearly attributed;\\
4. Where I have quoted from the work of others, the source is always given. With the exception of such quotations, this dissertation is entirely my own work;\\
5. I have acknowledged all main sources of help;\\
6. Where the thesis is based on work done by myself jointly with others, I have made clear exactly what was done by others and what I have contributed myself;\\
7. Either none of this work has been published before submission, or parts of this work have been published by :\\
\\
Stefan Collier\\
April 2016
}
\tableofcontents
\listoffigures
\listoftables

\mainmatter
%% ----------------------------------------------------------------
%\include{Introduction}
%\include{Conclusions}
\include{chapters/1Project/main}
\include{chapters/2Lit/main}
\include{chapters/3Design/HighLevel}
\include{chapters/3Design/InDepth}
\include{chapters/4Impl/main}

\include{chapters/5Experiments/1/main}
\include{chapters/5Experiments/2/main}
\include{chapters/5Experiments/3/main}
\include{chapters/5Experiments/4/main}

\include{chapters/6Conclusion/main}

\appendix
\include{appendix/AppendixB}
\include{appendix/D/main}
\include{appendix/AppendixC}

\backmatter
\bibliographystyle{ecs}
\bibliography{ECS}
\end{document}
%% ----------------------------------------------------------------


 %% ----------------------------------------------------------------
%% Progress.tex
%% ---------------------------------------------------------------- 
\documentclass{ecsprogress}    % Use the progress Style
\graphicspath{{../figs/}}   % Location of your graphics files
    \usepackage{natbib}            % Use Natbib style for the refs.
\hypersetup{colorlinks=true}   % Set to false for black/white printing
\input{Definitions}            % Include your abbreviations



\usepackage{enumitem}% http://ctan.org/pkg/enumitem
\usepackage{multirow}
\usepackage{float}
\usepackage{amsmath}
\usepackage{multicol}
\usepackage{amssymb}
\usepackage[normalem]{ulem}
\useunder{\uline}{\ul}{}
\usepackage{wrapfig}


\usepackage[table,xcdraw]{xcolor}


%% ----------------------------------------------------------------
\begin{document}
\frontmatter
\title      {Heterogeneous Agent-based Model for Supermarket Competition}
\authors    {\texorpdfstring
             {\href{mailto:sc22g13@ecs.soton.ac.uk}{Stefan J. Collier}}
             {Stefan J. Collier}
            }
\addresses  {\groupname\\\deptname\\\univname}
\date       {\today}
\subject    {}
\keywords   {}
\supervisor {Dr. Maria Polukarov}
\examiner   {Professor Sheng Chen}

\maketitle
\begin{abstract}
This project aim was to model and analyse the effects of competitive pricing behaviors of grocery retailers on the British market. 

This was achieved by creating a multi-agent model, containing retailer and consumer agents. The heterogeneous crowd of retailers employs either a uniform pricing strategy or a ‘local price flexing’ strategy. The actions of these retailers are chosen by predicting the profit of each action, using a perceptron. Following on from the consideration of different economic models, a discrete model was developed so that software agents have a discrete environment to operate within. Within the model, it has been observed how supermarkets with differing behaviors affect a heterogeneous crowd of consumer agents. The model was implemented in Java with Python used to evaluate the results. 

The simulation displays good acceptance with real grocery market behavior, i.e. captures the performance of British retailers thus can be used to determine the impact of changes in their behavior on their competitors and consumers.Furthermore it can be used to provide insight into sustainability of volatile pricing strategies, providing a useful insight in volatility of British supermarket retail industry. 
\end{abstract}
\acknowledgements{
I would like to express my sincere gratitude to Dr Maria Polukarov for her guidance and support which provided me the freedom to take this research in the direction of my interest.\\
\\
I would also like to thank my family and friends for their encouragement and support. To those who quietly listened to my software complaints. To those who worked throughout the nights with me. To those who helped me write what I couldn't say. I cannot thank you enough.
}

\declaration{
I, Stefan Collier, declare that this dissertation and the work presented in it are my own and has been generated by me as the result of my own original research.\\
I confirm that:\\
1. This work was done wholly or mainly while in candidature for a degree at this University;\\
2. Where any part of this dissertation has previously been submitted for any other qualification at this University or any other institution, this has been clearly stated;\\
3. Where I have consulted the published work of others, this is always clearly attributed;\\
4. Where I have quoted from the work of others, the source is always given. With the exception of such quotations, this dissertation is entirely my own work;\\
5. I have acknowledged all main sources of help;\\
6. Where the thesis is based on work done by myself jointly with others, I have made clear exactly what was done by others and what I have contributed myself;\\
7. Either none of this work has been published before submission, or parts of this work have been published by :\\
\\
Stefan Collier\\
April 2016
}
\tableofcontents
\listoffigures
\listoftables

\mainmatter
%% ----------------------------------------------------------------
%\include{Introduction}
%\include{Conclusions}
\include{chapters/1Project/main}
\include{chapters/2Lit/main}
\include{chapters/3Design/HighLevel}
\include{chapters/3Design/InDepth}
\include{chapters/4Impl/main}

\include{chapters/5Experiments/1/main}
\include{chapters/5Experiments/2/main}
\include{chapters/5Experiments/3/main}
\include{chapters/5Experiments/4/main}

\include{chapters/6Conclusion/main}

\appendix
\include{appendix/AppendixB}
\include{appendix/D/main}
\include{appendix/AppendixC}

\backmatter
\bibliographystyle{ecs}
\bibliography{ECS}
\end{document}
%% ----------------------------------------------------------------

 %% ----------------------------------------------------------------
%% Progress.tex
%% ---------------------------------------------------------------- 
\documentclass{ecsprogress}    % Use the progress Style
\graphicspath{{../figs/}}   % Location of your graphics files
    \usepackage{natbib}            % Use Natbib style for the refs.
\hypersetup{colorlinks=true}   % Set to false for black/white printing
\input{Definitions}            % Include your abbreviations



\usepackage{enumitem}% http://ctan.org/pkg/enumitem
\usepackage{multirow}
\usepackage{float}
\usepackage{amsmath}
\usepackage{multicol}
\usepackage{amssymb}
\usepackage[normalem]{ulem}
\useunder{\uline}{\ul}{}
\usepackage{wrapfig}


\usepackage[table,xcdraw]{xcolor}


%% ----------------------------------------------------------------
\begin{document}
\frontmatter
\title      {Heterogeneous Agent-based Model for Supermarket Competition}
\authors    {\texorpdfstring
             {\href{mailto:sc22g13@ecs.soton.ac.uk}{Stefan J. Collier}}
             {Stefan J. Collier}
            }
\addresses  {\groupname\\\deptname\\\univname}
\date       {\today}
\subject    {}
\keywords   {}
\supervisor {Dr. Maria Polukarov}
\examiner   {Professor Sheng Chen}

\maketitle
\begin{abstract}
This project aim was to model and analyse the effects of competitive pricing behaviors of grocery retailers on the British market. 

This was achieved by creating a multi-agent model, containing retailer and consumer agents. The heterogeneous crowd of retailers employs either a uniform pricing strategy or a ‘local price flexing’ strategy. The actions of these retailers are chosen by predicting the profit of each action, using a perceptron. Following on from the consideration of different economic models, a discrete model was developed so that software agents have a discrete environment to operate within. Within the model, it has been observed how supermarkets with differing behaviors affect a heterogeneous crowd of consumer agents. The model was implemented in Java with Python used to evaluate the results. 

The simulation displays good acceptance with real grocery market behavior, i.e. captures the performance of British retailers thus can be used to determine the impact of changes in their behavior on their competitors and consumers.Furthermore it can be used to provide insight into sustainability of volatile pricing strategies, providing a useful insight in volatility of British supermarket retail industry. 
\end{abstract}
\acknowledgements{
I would like to express my sincere gratitude to Dr Maria Polukarov for her guidance and support which provided me the freedom to take this research in the direction of my interest.\\
\\
I would also like to thank my family and friends for their encouragement and support. To those who quietly listened to my software complaints. To those who worked throughout the nights with me. To those who helped me write what I couldn't say. I cannot thank you enough.
}

\declaration{
I, Stefan Collier, declare that this dissertation and the work presented in it are my own and has been generated by me as the result of my own original research.\\
I confirm that:\\
1. This work was done wholly or mainly while in candidature for a degree at this University;\\
2. Where any part of this dissertation has previously been submitted for any other qualification at this University or any other institution, this has been clearly stated;\\
3. Where I have consulted the published work of others, this is always clearly attributed;\\
4. Where I have quoted from the work of others, the source is always given. With the exception of such quotations, this dissertation is entirely my own work;\\
5. I have acknowledged all main sources of help;\\
6. Where the thesis is based on work done by myself jointly with others, I have made clear exactly what was done by others and what I have contributed myself;\\
7. Either none of this work has been published before submission, or parts of this work have been published by :\\
\\
Stefan Collier\\
April 2016
}
\tableofcontents
\listoffigures
\listoftables

\mainmatter
%% ----------------------------------------------------------------
%\include{Introduction}
%\include{Conclusions}
\include{chapters/1Project/main}
\include{chapters/2Lit/main}
\include{chapters/3Design/HighLevel}
\include{chapters/3Design/InDepth}
\include{chapters/4Impl/main}

\include{chapters/5Experiments/1/main}
\include{chapters/5Experiments/2/main}
\include{chapters/5Experiments/3/main}
\include{chapters/5Experiments/4/main}

\include{chapters/6Conclusion/main}

\appendix
\include{appendix/AppendixB}
\include{appendix/D/main}
\include{appendix/AppendixC}

\backmatter
\bibliographystyle{ecs}
\bibliography{ECS}
\end{document}
%% ----------------------------------------------------------------

 %% ----------------------------------------------------------------
%% Progress.tex
%% ---------------------------------------------------------------- 
\documentclass{ecsprogress}    % Use the progress Style
\graphicspath{{../figs/}}   % Location of your graphics files
    \usepackage{natbib}            % Use Natbib style for the refs.
\hypersetup{colorlinks=true}   % Set to false for black/white printing
\input{Definitions}            % Include your abbreviations



\usepackage{enumitem}% http://ctan.org/pkg/enumitem
\usepackage{multirow}
\usepackage{float}
\usepackage{amsmath}
\usepackage{multicol}
\usepackage{amssymb}
\usepackage[normalem]{ulem}
\useunder{\uline}{\ul}{}
\usepackage{wrapfig}


\usepackage[table,xcdraw]{xcolor}


%% ----------------------------------------------------------------
\begin{document}
\frontmatter
\title      {Heterogeneous Agent-based Model for Supermarket Competition}
\authors    {\texorpdfstring
             {\href{mailto:sc22g13@ecs.soton.ac.uk}{Stefan J. Collier}}
             {Stefan J. Collier}
            }
\addresses  {\groupname\\\deptname\\\univname}
\date       {\today}
\subject    {}
\keywords   {}
\supervisor {Dr. Maria Polukarov}
\examiner   {Professor Sheng Chen}

\maketitle
\begin{abstract}
This project aim was to model and analyse the effects of competitive pricing behaviors of grocery retailers on the British market. 

This was achieved by creating a multi-agent model, containing retailer and consumer agents. The heterogeneous crowd of retailers employs either a uniform pricing strategy or a ‘local price flexing’ strategy. The actions of these retailers are chosen by predicting the profit of each action, using a perceptron. Following on from the consideration of different economic models, a discrete model was developed so that software agents have a discrete environment to operate within. Within the model, it has been observed how supermarkets with differing behaviors affect a heterogeneous crowd of consumer agents. The model was implemented in Java with Python used to evaluate the results. 

The simulation displays good acceptance with real grocery market behavior, i.e. captures the performance of British retailers thus can be used to determine the impact of changes in their behavior on their competitors and consumers.Furthermore it can be used to provide insight into sustainability of volatile pricing strategies, providing a useful insight in volatility of British supermarket retail industry. 
\end{abstract}
\acknowledgements{
I would like to express my sincere gratitude to Dr Maria Polukarov for her guidance and support which provided me the freedom to take this research in the direction of my interest.\\
\\
I would also like to thank my family and friends for their encouragement and support. To those who quietly listened to my software complaints. To those who worked throughout the nights with me. To those who helped me write what I couldn't say. I cannot thank you enough.
}

\declaration{
I, Stefan Collier, declare that this dissertation and the work presented in it are my own and has been generated by me as the result of my own original research.\\
I confirm that:\\
1. This work was done wholly or mainly while in candidature for a degree at this University;\\
2. Where any part of this dissertation has previously been submitted for any other qualification at this University or any other institution, this has been clearly stated;\\
3. Where I have consulted the published work of others, this is always clearly attributed;\\
4. Where I have quoted from the work of others, the source is always given. With the exception of such quotations, this dissertation is entirely my own work;\\
5. I have acknowledged all main sources of help;\\
6. Where the thesis is based on work done by myself jointly with others, I have made clear exactly what was done by others and what I have contributed myself;\\
7. Either none of this work has been published before submission, or parts of this work have been published by :\\
\\
Stefan Collier\\
April 2016
}
\tableofcontents
\listoffigures
\listoftables

\mainmatter
%% ----------------------------------------------------------------
%\include{Introduction}
%\include{Conclusions}
\include{chapters/1Project/main}
\include{chapters/2Lit/main}
\include{chapters/3Design/HighLevel}
\include{chapters/3Design/InDepth}
\include{chapters/4Impl/main}

\include{chapters/5Experiments/1/main}
\include{chapters/5Experiments/2/main}
\include{chapters/5Experiments/3/main}
\include{chapters/5Experiments/4/main}

\include{chapters/6Conclusion/main}

\appendix
\include{appendix/AppendixB}
\include{appendix/D/main}
\include{appendix/AppendixC}

\backmatter
\bibliographystyle{ecs}
\bibliography{ECS}
\end{document}
%% ----------------------------------------------------------------

 %% ----------------------------------------------------------------
%% Progress.tex
%% ---------------------------------------------------------------- 
\documentclass{ecsprogress}    % Use the progress Style
\graphicspath{{../figs/}}   % Location of your graphics files
    \usepackage{natbib}            % Use Natbib style for the refs.
\hypersetup{colorlinks=true}   % Set to false for black/white printing
\input{Definitions}            % Include your abbreviations



\usepackage{enumitem}% http://ctan.org/pkg/enumitem
\usepackage{multirow}
\usepackage{float}
\usepackage{amsmath}
\usepackage{multicol}
\usepackage{amssymb}
\usepackage[normalem]{ulem}
\useunder{\uline}{\ul}{}
\usepackage{wrapfig}


\usepackage[table,xcdraw]{xcolor}


%% ----------------------------------------------------------------
\begin{document}
\frontmatter
\title      {Heterogeneous Agent-based Model for Supermarket Competition}
\authors    {\texorpdfstring
             {\href{mailto:sc22g13@ecs.soton.ac.uk}{Stefan J. Collier}}
             {Stefan J. Collier}
            }
\addresses  {\groupname\\\deptname\\\univname}
\date       {\today}
\subject    {}
\keywords   {}
\supervisor {Dr. Maria Polukarov}
\examiner   {Professor Sheng Chen}

\maketitle
\begin{abstract}
This project aim was to model and analyse the effects of competitive pricing behaviors of grocery retailers on the British market. 

This was achieved by creating a multi-agent model, containing retailer and consumer agents. The heterogeneous crowd of retailers employs either a uniform pricing strategy or a ‘local price flexing’ strategy. The actions of these retailers are chosen by predicting the profit of each action, using a perceptron. Following on from the consideration of different economic models, a discrete model was developed so that software agents have a discrete environment to operate within. Within the model, it has been observed how supermarkets with differing behaviors affect a heterogeneous crowd of consumer agents. The model was implemented in Java with Python used to evaluate the results. 

The simulation displays good acceptance with real grocery market behavior, i.e. captures the performance of British retailers thus can be used to determine the impact of changes in their behavior on their competitors and consumers.Furthermore it can be used to provide insight into sustainability of volatile pricing strategies, providing a useful insight in volatility of British supermarket retail industry. 
\end{abstract}
\acknowledgements{
I would like to express my sincere gratitude to Dr Maria Polukarov for her guidance and support which provided me the freedom to take this research in the direction of my interest.\\
\\
I would also like to thank my family and friends for their encouragement and support. To those who quietly listened to my software complaints. To those who worked throughout the nights with me. To those who helped me write what I couldn't say. I cannot thank you enough.
}

\declaration{
I, Stefan Collier, declare that this dissertation and the work presented in it are my own and has been generated by me as the result of my own original research.\\
I confirm that:\\
1. This work was done wholly or mainly while in candidature for a degree at this University;\\
2. Where any part of this dissertation has previously been submitted for any other qualification at this University or any other institution, this has been clearly stated;\\
3. Where I have consulted the published work of others, this is always clearly attributed;\\
4. Where I have quoted from the work of others, the source is always given. With the exception of such quotations, this dissertation is entirely my own work;\\
5. I have acknowledged all main sources of help;\\
6. Where the thesis is based on work done by myself jointly with others, I have made clear exactly what was done by others and what I have contributed myself;\\
7. Either none of this work has been published before submission, or parts of this work have been published by :\\
\\
Stefan Collier\\
April 2016
}
\tableofcontents
\listoffigures
\listoftables

\mainmatter
%% ----------------------------------------------------------------
%\include{Introduction}
%\include{Conclusions}
\include{chapters/1Project/main}
\include{chapters/2Lit/main}
\include{chapters/3Design/HighLevel}
\include{chapters/3Design/InDepth}
\include{chapters/4Impl/main}

\include{chapters/5Experiments/1/main}
\include{chapters/5Experiments/2/main}
\include{chapters/5Experiments/3/main}
\include{chapters/5Experiments/4/main}

\include{chapters/6Conclusion/main}

\appendix
\include{appendix/AppendixB}
\include{appendix/D/main}
\include{appendix/AppendixC}

\backmatter
\bibliographystyle{ecs}
\bibliography{ECS}
\end{document}
%% ----------------------------------------------------------------


 %% ----------------------------------------------------------------
%% Progress.tex
%% ---------------------------------------------------------------- 
\documentclass{ecsprogress}    % Use the progress Style
\graphicspath{{../figs/}}   % Location of your graphics files
    \usepackage{natbib}            % Use Natbib style for the refs.
\hypersetup{colorlinks=true}   % Set to false for black/white printing
\input{Definitions}            % Include your abbreviations



\usepackage{enumitem}% http://ctan.org/pkg/enumitem
\usepackage{multirow}
\usepackage{float}
\usepackage{amsmath}
\usepackage{multicol}
\usepackage{amssymb}
\usepackage[normalem]{ulem}
\useunder{\uline}{\ul}{}
\usepackage{wrapfig}


\usepackage[table,xcdraw]{xcolor}


%% ----------------------------------------------------------------
\begin{document}
\frontmatter
\title      {Heterogeneous Agent-based Model for Supermarket Competition}
\authors    {\texorpdfstring
             {\href{mailto:sc22g13@ecs.soton.ac.uk}{Stefan J. Collier}}
             {Stefan J. Collier}
            }
\addresses  {\groupname\\\deptname\\\univname}
\date       {\today}
\subject    {}
\keywords   {}
\supervisor {Dr. Maria Polukarov}
\examiner   {Professor Sheng Chen}

\maketitle
\begin{abstract}
This project aim was to model and analyse the effects of competitive pricing behaviors of grocery retailers on the British market. 

This was achieved by creating a multi-agent model, containing retailer and consumer agents. The heterogeneous crowd of retailers employs either a uniform pricing strategy or a ‘local price flexing’ strategy. The actions of these retailers are chosen by predicting the profit of each action, using a perceptron. Following on from the consideration of different economic models, a discrete model was developed so that software agents have a discrete environment to operate within. Within the model, it has been observed how supermarkets with differing behaviors affect a heterogeneous crowd of consumer agents. The model was implemented in Java with Python used to evaluate the results. 

The simulation displays good acceptance with real grocery market behavior, i.e. captures the performance of British retailers thus can be used to determine the impact of changes in their behavior on their competitors and consumers.Furthermore it can be used to provide insight into sustainability of volatile pricing strategies, providing a useful insight in volatility of British supermarket retail industry. 
\end{abstract}
\acknowledgements{
I would like to express my sincere gratitude to Dr Maria Polukarov for her guidance and support which provided me the freedom to take this research in the direction of my interest.\\
\\
I would also like to thank my family and friends for their encouragement and support. To those who quietly listened to my software complaints. To those who worked throughout the nights with me. To those who helped me write what I couldn't say. I cannot thank you enough.
}

\declaration{
I, Stefan Collier, declare that this dissertation and the work presented in it are my own and has been generated by me as the result of my own original research.\\
I confirm that:\\
1. This work was done wholly or mainly while in candidature for a degree at this University;\\
2. Where any part of this dissertation has previously been submitted for any other qualification at this University or any other institution, this has been clearly stated;\\
3. Where I have consulted the published work of others, this is always clearly attributed;\\
4. Where I have quoted from the work of others, the source is always given. With the exception of such quotations, this dissertation is entirely my own work;\\
5. I have acknowledged all main sources of help;\\
6. Where the thesis is based on work done by myself jointly with others, I have made clear exactly what was done by others and what I have contributed myself;\\
7. Either none of this work has been published before submission, or parts of this work have been published by :\\
\\
Stefan Collier\\
April 2016
}
\tableofcontents
\listoffigures
\listoftables

\mainmatter
%% ----------------------------------------------------------------
%\include{Introduction}
%\include{Conclusions}
\include{chapters/1Project/main}
\include{chapters/2Lit/main}
\include{chapters/3Design/HighLevel}
\include{chapters/3Design/InDepth}
\include{chapters/4Impl/main}

\include{chapters/5Experiments/1/main}
\include{chapters/5Experiments/2/main}
\include{chapters/5Experiments/3/main}
\include{chapters/5Experiments/4/main}

\include{chapters/6Conclusion/main}

\appendix
\include{appendix/AppendixB}
\include{appendix/D/main}
\include{appendix/AppendixC}

\backmatter
\bibliographystyle{ecs}
\bibliography{ECS}
\end{document}
%% ----------------------------------------------------------------


\appendix
\include{appendix/AppendixB}
 %% ----------------------------------------------------------------
%% Progress.tex
%% ---------------------------------------------------------------- 
\documentclass{ecsprogress}    % Use the progress Style
\graphicspath{{../figs/}}   % Location of your graphics files
    \usepackage{natbib}            % Use Natbib style for the refs.
\hypersetup{colorlinks=true}   % Set to false for black/white printing
\input{Definitions}            % Include your abbreviations



\usepackage{enumitem}% http://ctan.org/pkg/enumitem
\usepackage{multirow}
\usepackage{float}
\usepackage{amsmath}
\usepackage{multicol}
\usepackage{amssymb}
\usepackage[normalem]{ulem}
\useunder{\uline}{\ul}{}
\usepackage{wrapfig}


\usepackage[table,xcdraw]{xcolor}


%% ----------------------------------------------------------------
\begin{document}
\frontmatter
\title      {Heterogeneous Agent-based Model for Supermarket Competition}
\authors    {\texorpdfstring
             {\href{mailto:sc22g13@ecs.soton.ac.uk}{Stefan J. Collier}}
             {Stefan J. Collier}
            }
\addresses  {\groupname\\\deptname\\\univname}
\date       {\today}
\subject    {}
\keywords   {}
\supervisor {Dr. Maria Polukarov}
\examiner   {Professor Sheng Chen}

\maketitle
\begin{abstract}
This project aim was to model and analyse the effects of competitive pricing behaviors of grocery retailers on the British market. 

This was achieved by creating a multi-agent model, containing retailer and consumer agents. The heterogeneous crowd of retailers employs either a uniform pricing strategy or a ‘local price flexing’ strategy. The actions of these retailers are chosen by predicting the profit of each action, using a perceptron. Following on from the consideration of different economic models, a discrete model was developed so that software agents have a discrete environment to operate within. Within the model, it has been observed how supermarkets with differing behaviors affect a heterogeneous crowd of consumer agents. The model was implemented in Java with Python used to evaluate the results. 

The simulation displays good acceptance with real grocery market behavior, i.e. captures the performance of British retailers thus can be used to determine the impact of changes in their behavior on their competitors and consumers.Furthermore it can be used to provide insight into sustainability of volatile pricing strategies, providing a useful insight in volatility of British supermarket retail industry. 
\end{abstract}
\acknowledgements{
I would like to express my sincere gratitude to Dr Maria Polukarov for her guidance and support which provided me the freedom to take this research in the direction of my interest.\\
\\
I would also like to thank my family and friends for their encouragement and support. To those who quietly listened to my software complaints. To those who worked throughout the nights with me. To those who helped me write what I couldn't say. I cannot thank you enough.
}

\declaration{
I, Stefan Collier, declare that this dissertation and the work presented in it are my own and has been generated by me as the result of my own original research.\\
I confirm that:\\
1. This work was done wholly or mainly while in candidature for a degree at this University;\\
2. Where any part of this dissertation has previously been submitted for any other qualification at this University or any other institution, this has been clearly stated;\\
3. Where I have consulted the published work of others, this is always clearly attributed;\\
4. Where I have quoted from the work of others, the source is always given. With the exception of such quotations, this dissertation is entirely my own work;\\
5. I have acknowledged all main sources of help;\\
6. Where the thesis is based on work done by myself jointly with others, I have made clear exactly what was done by others and what I have contributed myself;\\
7. Either none of this work has been published before submission, or parts of this work have been published by :\\
\\
Stefan Collier\\
April 2016
}
\tableofcontents
\listoffigures
\listoftables

\mainmatter
%% ----------------------------------------------------------------
%\include{Introduction}
%\include{Conclusions}
\include{chapters/1Project/main}
\include{chapters/2Lit/main}
\include{chapters/3Design/HighLevel}
\include{chapters/3Design/InDepth}
\include{chapters/4Impl/main}

\include{chapters/5Experiments/1/main}
\include{chapters/5Experiments/2/main}
\include{chapters/5Experiments/3/main}
\include{chapters/5Experiments/4/main}

\include{chapters/6Conclusion/main}

\appendix
\include{appendix/AppendixB}
\include{appendix/D/main}
\include{appendix/AppendixC}

\backmatter
\bibliographystyle{ecs}
\bibliography{ECS}
\end{document}
%% ----------------------------------------------------------------

\include{appendix/AppendixC}

\backmatter
\bibliographystyle{ecs}
\bibliography{ECS}
\end{document}
%% ----------------------------------------------------------------

 %% ----------------------------------------------------------------
%% Progress.tex
%% ---------------------------------------------------------------- 
\documentclass{ecsprogress}    % Use the progress Style
\graphicspath{{../figs/}}   % Location of your graphics files
    \usepackage{natbib}            % Use Natbib style for the refs.
\hypersetup{colorlinks=true}   % Set to false for black/white printing
\input{Definitions}            % Include your abbreviations



\usepackage{enumitem}% http://ctan.org/pkg/enumitem
\usepackage{multirow}
\usepackage{float}
\usepackage{amsmath}
\usepackage{multicol}
\usepackage{amssymb}
\usepackage[normalem]{ulem}
\useunder{\uline}{\ul}{}
\usepackage{wrapfig}


\usepackage[table,xcdraw]{xcolor}


%% ----------------------------------------------------------------
\begin{document}
\frontmatter
\title      {Heterogeneous Agent-based Model for Supermarket Competition}
\authors    {\texorpdfstring
             {\href{mailto:sc22g13@ecs.soton.ac.uk}{Stefan J. Collier}}
             {Stefan J. Collier}
            }
\addresses  {\groupname\\\deptname\\\univname}
\date       {\today}
\subject    {}
\keywords   {}
\supervisor {Dr. Maria Polukarov}
\examiner   {Professor Sheng Chen}

\maketitle
\begin{abstract}
This project aim was to model and analyse the effects of competitive pricing behaviors of grocery retailers on the British market. 

This was achieved by creating a multi-agent model, containing retailer and consumer agents. The heterogeneous crowd of retailers employs either a uniform pricing strategy or a ‘local price flexing’ strategy. The actions of these retailers are chosen by predicting the profit of each action, using a perceptron. Following on from the consideration of different economic models, a discrete model was developed so that software agents have a discrete environment to operate within. Within the model, it has been observed how supermarkets with differing behaviors affect a heterogeneous crowd of consumer agents. The model was implemented in Java with Python used to evaluate the results. 

The simulation displays good acceptance with real grocery market behavior, i.e. captures the performance of British retailers thus can be used to determine the impact of changes in their behavior on their competitors and consumers.Furthermore it can be used to provide insight into sustainability of volatile pricing strategies, providing a useful insight in volatility of British supermarket retail industry. 
\end{abstract}
\acknowledgements{
I would like to express my sincere gratitude to Dr Maria Polukarov for her guidance and support which provided me the freedom to take this research in the direction of my interest.\\
\\
I would also like to thank my family and friends for their encouragement and support. To those who quietly listened to my software complaints. To those who worked throughout the nights with me. To those who helped me write what I couldn't say. I cannot thank you enough.
}

\declaration{
I, Stefan Collier, declare that this dissertation and the work presented in it are my own and has been generated by me as the result of my own original research.\\
I confirm that:\\
1. This work was done wholly or mainly while in candidature for a degree at this University;\\
2. Where any part of this dissertation has previously been submitted for any other qualification at this University or any other institution, this has been clearly stated;\\
3. Where I have consulted the published work of others, this is always clearly attributed;\\
4. Where I have quoted from the work of others, the source is always given. With the exception of such quotations, this dissertation is entirely my own work;\\
5. I have acknowledged all main sources of help;\\
6. Where the thesis is based on work done by myself jointly with others, I have made clear exactly what was done by others and what I have contributed myself;\\
7. Either none of this work has been published before submission, or parts of this work have been published by :\\
\\
Stefan Collier\\
April 2016
}
\tableofcontents
\listoffigures
\listoftables

\mainmatter
%% ----------------------------------------------------------------
%\include{Introduction}
%\include{Conclusions}
 %% ----------------------------------------------------------------
%% Progress.tex
%% ---------------------------------------------------------------- 
\documentclass{ecsprogress}    % Use the progress Style
\graphicspath{{../figs/}}   % Location of your graphics files
    \usepackage{natbib}            % Use Natbib style for the refs.
\hypersetup{colorlinks=true}   % Set to false for black/white printing
\input{Definitions}            % Include your abbreviations



\usepackage{enumitem}% http://ctan.org/pkg/enumitem
\usepackage{multirow}
\usepackage{float}
\usepackage{amsmath}
\usepackage{multicol}
\usepackage{amssymb}
\usepackage[normalem]{ulem}
\useunder{\uline}{\ul}{}
\usepackage{wrapfig}


\usepackage[table,xcdraw]{xcolor}


%% ----------------------------------------------------------------
\begin{document}
\frontmatter
\title      {Heterogeneous Agent-based Model for Supermarket Competition}
\authors    {\texorpdfstring
             {\href{mailto:sc22g13@ecs.soton.ac.uk}{Stefan J. Collier}}
             {Stefan J. Collier}
            }
\addresses  {\groupname\\\deptname\\\univname}
\date       {\today}
\subject    {}
\keywords   {}
\supervisor {Dr. Maria Polukarov}
\examiner   {Professor Sheng Chen}

\maketitle
\begin{abstract}
This project aim was to model and analyse the effects of competitive pricing behaviors of grocery retailers on the British market. 

This was achieved by creating a multi-agent model, containing retailer and consumer agents. The heterogeneous crowd of retailers employs either a uniform pricing strategy or a ‘local price flexing’ strategy. The actions of these retailers are chosen by predicting the profit of each action, using a perceptron. Following on from the consideration of different economic models, a discrete model was developed so that software agents have a discrete environment to operate within. Within the model, it has been observed how supermarkets with differing behaviors affect a heterogeneous crowd of consumer agents. The model was implemented in Java with Python used to evaluate the results. 

The simulation displays good acceptance with real grocery market behavior, i.e. captures the performance of British retailers thus can be used to determine the impact of changes in their behavior on their competitors and consumers.Furthermore it can be used to provide insight into sustainability of volatile pricing strategies, providing a useful insight in volatility of British supermarket retail industry. 
\end{abstract}
\acknowledgements{
I would like to express my sincere gratitude to Dr Maria Polukarov for her guidance and support which provided me the freedom to take this research in the direction of my interest.\\
\\
I would also like to thank my family and friends for their encouragement and support. To those who quietly listened to my software complaints. To those who worked throughout the nights with me. To those who helped me write what I couldn't say. I cannot thank you enough.
}

\declaration{
I, Stefan Collier, declare that this dissertation and the work presented in it are my own and has been generated by me as the result of my own original research.\\
I confirm that:\\
1. This work was done wholly or mainly while in candidature for a degree at this University;\\
2. Where any part of this dissertation has previously been submitted for any other qualification at this University or any other institution, this has been clearly stated;\\
3. Where I have consulted the published work of others, this is always clearly attributed;\\
4. Where I have quoted from the work of others, the source is always given. With the exception of such quotations, this dissertation is entirely my own work;\\
5. I have acknowledged all main sources of help;\\
6. Where the thesis is based on work done by myself jointly with others, I have made clear exactly what was done by others and what I have contributed myself;\\
7. Either none of this work has been published before submission, or parts of this work have been published by :\\
\\
Stefan Collier\\
April 2016
}
\tableofcontents
\listoffigures
\listoftables

\mainmatter
%% ----------------------------------------------------------------
%\include{Introduction}
%\include{Conclusions}
\include{chapters/1Project/main}
\include{chapters/2Lit/main}
\include{chapters/3Design/HighLevel}
\include{chapters/3Design/InDepth}
\include{chapters/4Impl/main}

\include{chapters/5Experiments/1/main}
\include{chapters/5Experiments/2/main}
\include{chapters/5Experiments/3/main}
\include{chapters/5Experiments/4/main}

\include{chapters/6Conclusion/main}

\appendix
\include{appendix/AppendixB}
\include{appendix/D/main}
\include{appendix/AppendixC}

\backmatter
\bibliographystyle{ecs}
\bibliography{ECS}
\end{document}
%% ----------------------------------------------------------------

 %% ----------------------------------------------------------------
%% Progress.tex
%% ---------------------------------------------------------------- 
\documentclass{ecsprogress}    % Use the progress Style
\graphicspath{{../figs/}}   % Location of your graphics files
    \usepackage{natbib}            % Use Natbib style for the refs.
\hypersetup{colorlinks=true}   % Set to false for black/white printing
\input{Definitions}            % Include your abbreviations



\usepackage{enumitem}% http://ctan.org/pkg/enumitem
\usepackage{multirow}
\usepackage{float}
\usepackage{amsmath}
\usepackage{multicol}
\usepackage{amssymb}
\usepackage[normalem]{ulem}
\useunder{\uline}{\ul}{}
\usepackage{wrapfig}


\usepackage[table,xcdraw]{xcolor}


%% ----------------------------------------------------------------
\begin{document}
\frontmatter
\title      {Heterogeneous Agent-based Model for Supermarket Competition}
\authors    {\texorpdfstring
             {\href{mailto:sc22g13@ecs.soton.ac.uk}{Stefan J. Collier}}
             {Stefan J. Collier}
            }
\addresses  {\groupname\\\deptname\\\univname}
\date       {\today}
\subject    {}
\keywords   {}
\supervisor {Dr. Maria Polukarov}
\examiner   {Professor Sheng Chen}

\maketitle
\begin{abstract}
This project aim was to model and analyse the effects of competitive pricing behaviors of grocery retailers on the British market. 

This was achieved by creating a multi-agent model, containing retailer and consumer agents. The heterogeneous crowd of retailers employs either a uniform pricing strategy or a ‘local price flexing’ strategy. The actions of these retailers are chosen by predicting the profit of each action, using a perceptron. Following on from the consideration of different economic models, a discrete model was developed so that software agents have a discrete environment to operate within. Within the model, it has been observed how supermarkets with differing behaviors affect a heterogeneous crowd of consumer agents. The model was implemented in Java with Python used to evaluate the results. 

The simulation displays good acceptance with real grocery market behavior, i.e. captures the performance of British retailers thus can be used to determine the impact of changes in their behavior on their competitors and consumers.Furthermore it can be used to provide insight into sustainability of volatile pricing strategies, providing a useful insight in volatility of British supermarket retail industry. 
\end{abstract}
\acknowledgements{
I would like to express my sincere gratitude to Dr Maria Polukarov for her guidance and support which provided me the freedom to take this research in the direction of my interest.\\
\\
I would also like to thank my family and friends for their encouragement and support. To those who quietly listened to my software complaints. To those who worked throughout the nights with me. To those who helped me write what I couldn't say. I cannot thank you enough.
}

\declaration{
I, Stefan Collier, declare that this dissertation and the work presented in it are my own and has been generated by me as the result of my own original research.\\
I confirm that:\\
1. This work was done wholly or mainly while in candidature for a degree at this University;\\
2. Where any part of this dissertation has previously been submitted for any other qualification at this University or any other institution, this has been clearly stated;\\
3. Where I have consulted the published work of others, this is always clearly attributed;\\
4. Where I have quoted from the work of others, the source is always given. With the exception of such quotations, this dissertation is entirely my own work;\\
5. I have acknowledged all main sources of help;\\
6. Where the thesis is based on work done by myself jointly with others, I have made clear exactly what was done by others and what I have contributed myself;\\
7. Either none of this work has been published before submission, or parts of this work have been published by :\\
\\
Stefan Collier\\
April 2016
}
\tableofcontents
\listoffigures
\listoftables

\mainmatter
%% ----------------------------------------------------------------
%\include{Introduction}
%\include{Conclusions}
\include{chapters/1Project/main}
\include{chapters/2Lit/main}
\include{chapters/3Design/HighLevel}
\include{chapters/3Design/InDepth}
\include{chapters/4Impl/main}

\include{chapters/5Experiments/1/main}
\include{chapters/5Experiments/2/main}
\include{chapters/5Experiments/3/main}
\include{chapters/5Experiments/4/main}

\include{chapters/6Conclusion/main}

\appendix
\include{appendix/AppendixB}
\include{appendix/D/main}
\include{appendix/AppendixC}

\backmatter
\bibliographystyle{ecs}
\bibliography{ECS}
\end{document}
%% ----------------------------------------------------------------

\include{chapters/3Design/HighLevel}
\include{chapters/3Design/InDepth}
 %% ----------------------------------------------------------------
%% Progress.tex
%% ---------------------------------------------------------------- 
\documentclass{ecsprogress}    % Use the progress Style
\graphicspath{{../figs/}}   % Location of your graphics files
    \usepackage{natbib}            % Use Natbib style for the refs.
\hypersetup{colorlinks=true}   % Set to false for black/white printing
\input{Definitions}            % Include your abbreviations



\usepackage{enumitem}% http://ctan.org/pkg/enumitem
\usepackage{multirow}
\usepackage{float}
\usepackage{amsmath}
\usepackage{multicol}
\usepackage{amssymb}
\usepackage[normalem]{ulem}
\useunder{\uline}{\ul}{}
\usepackage{wrapfig}


\usepackage[table,xcdraw]{xcolor}


%% ----------------------------------------------------------------
\begin{document}
\frontmatter
\title      {Heterogeneous Agent-based Model for Supermarket Competition}
\authors    {\texorpdfstring
             {\href{mailto:sc22g13@ecs.soton.ac.uk}{Stefan J. Collier}}
             {Stefan J. Collier}
            }
\addresses  {\groupname\\\deptname\\\univname}
\date       {\today}
\subject    {}
\keywords   {}
\supervisor {Dr. Maria Polukarov}
\examiner   {Professor Sheng Chen}

\maketitle
\begin{abstract}
This project aim was to model and analyse the effects of competitive pricing behaviors of grocery retailers on the British market. 

This was achieved by creating a multi-agent model, containing retailer and consumer agents. The heterogeneous crowd of retailers employs either a uniform pricing strategy or a ‘local price flexing’ strategy. The actions of these retailers are chosen by predicting the profit of each action, using a perceptron. Following on from the consideration of different economic models, a discrete model was developed so that software agents have a discrete environment to operate within. Within the model, it has been observed how supermarkets with differing behaviors affect a heterogeneous crowd of consumer agents. The model was implemented in Java with Python used to evaluate the results. 

The simulation displays good acceptance with real grocery market behavior, i.e. captures the performance of British retailers thus can be used to determine the impact of changes in their behavior on their competitors and consumers.Furthermore it can be used to provide insight into sustainability of volatile pricing strategies, providing a useful insight in volatility of British supermarket retail industry. 
\end{abstract}
\acknowledgements{
I would like to express my sincere gratitude to Dr Maria Polukarov for her guidance and support which provided me the freedom to take this research in the direction of my interest.\\
\\
I would also like to thank my family and friends for their encouragement and support. To those who quietly listened to my software complaints. To those who worked throughout the nights with me. To those who helped me write what I couldn't say. I cannot thank you enough.
}

\declaration{
I, Stefan Collier, declare that this dissertation and the work presented in it are my own and has been generated by me as the result of my own original research.\\
I confirm that:\\
1. This work was done wholly or mainly while in candidature for a degree at this University;\\
2. Where any part of this dissertation has previously been submitted for any other qualification at this University or any other institution, this has been clearly stated;\\
3. Where I have consulted the published work of others, this is always clearly attributed;\\
4. Where I have quoted from the work of others, the source is always given. With the exception of such quotations, this dissertation is entirely my own work;\\
5. I have acknowledged all main sources of help;\\
6. Where the thesis is based on work done by myself jointly with others, I have made clear exactly what was done by others and what I have contributed myself;\\
7. Either none of this work has been published before submission, or parts of this work have been published by :\\
\\
Stefan Collier\\
April 2016
}
\tableofcontents
\listoffigures
\listoftables

\mainmatter
%% ----------------------------------------------------------------
%\include{Introduction}
%\include{Conclusions}
\include{chapters/1Project/main}
\include{chapters/2Lit/main}
\include{chapters/3Design/HighLevel}
\include{chapters/3Design/InDepth}
\include{chapters/4Impl/main}

\include{chapters/5Experiments/1/main}
\include{chapters/5Experiments/2/main}
\include{chapters/5Experiments/3/main}
\include{chapters/5Experiments/4/main}

\include{chapters/6Conclusion/main}

\appendix
\include{appendix/AppendixB}
\include{appendix/D/main}
\include{appendix/AppendixC}

\backmatter
\bibliographystyle{ecs}
\bibliography{ECS}
\end{document}
%% ----------------------------------------------------------------


 %% ----------------------------------------------------------------
%% Progress.tex
%% ---------------------------------------------------------------- 
\documentclass{ecsprogress}    % Use the progress Style
\graphicspath{{../figs/}}   % Location of your graphics files
    \usepackage{natbib}            % Use Natbib style for the refs.
\hypersetup{colorlinks=true}   % Set to false for black/white printing
\input{Definitions}            % Include your abbreviations



\usepackage{enumitem}% http://ctan.org/pkg/enumitem
\usepackage{multirow}
\usepackage{float}
\usepackage{amsmath}
\usepackage{multicol}
\usepackage{amssymb}
\usepackage[normalem]{ulem}
\useunder{\uline}{\ul}{}
\usepackage{wrapfig}


\usepackage[table,xcdraw]{xcolor}


%% ----------------------------------------------------------------
\begin{document}
\frontmatter
\title      {Heterogeneous Agent-based Model for Supermarket Competition}
\authors    {\texorpdfstring
             {\href{mailto:sc22g13@ecs.soton.ac.uk}{Stefan J. Collier}}
             {Stefan J. Collier}
            }
\addresses  {\groupname\\\deptname\\\univname}
\date       {\today}
\subject    {}
\keywords   {}
\supervisor {Dr. Maria Polukarov}
\examiner   {Professor Sheng Chen}

\maketitle
\begin{abstract}
This project aim was to model and analyse the effects of competitive pricing behaviors of grocery retailers on the British market. 

This was achieved by creating a multi-agent model, containing retailer and consumer agents. The heterogeneous crowd of retailers employs either a uniform pricing strategy or a ‘local price flexing’ strategy. The actions of these retailers are chosen by predicting the profit of each action, using a perceptron. Following on from the consideration of different economic models, a discrete model was developed so that software agents have a discrete environment to operate within. Within the model, it has been observed how supermarkets with differing behaviors affect a heterogeneous crowd of consumer agents. The model was implemented in Java with Python used to evaluate the results. 

The simulation displays good acceptance with real grocery market behavior, i.e. captures the performance of British retailers thus can be used to determine the impact of changes in their behavior on their competitors and consumers.Furthermore it can be used to provide insight into sustainability of volatile pricing strategies, providing a useful insight in volatility of British supermarket retail industry. 
\end{abstract}
\acknowledgements{
I would like to express my sincere gratitude to Dr Maria Polukarov for her guidance and support which provided me the freedom to take this research in the direction of my interest.\\
\\
I would also like to thank my family and friends for their encouragement and support. To those who quietly listened to my software complaints. To those who worked throughout the nights with me. To those who helped me write what I couldn't say. I cannot thank you enough.
}

\declaration{
I, Stefan Collier, declare that this dissertation and the work presented in it are my own and has been generated by me as the result of my own original research.\\
I confirm that:\\
1. This work was done wholly or mainly while in candidature for a degree at this University;\\
2. Where any part of this dissertation has previously been submitted for any other qualification at this University or any other institution, this has been clearly stated;\\
3. Where I have consulted the published work of others, this is always clearly attributed;\\
4. Where I have quoted from the work of others, the source is always given. With the exception of such quotations, this dissertation is entirely my own work;\\
5. I have acknowledged all main sources of help;\\
6. Where the thesis is based on work done by myself jointly with others, I have made clear exactly what was done by others and what I have contributed myself;\\
7. Either none of this work has been published before submission, or parts of this work have been published by :\\
\\
Stefan Collier\\
April 2016
}
\tableofcontents
\listoffigures
\listoftables

\mainmatter
%% ----------------------------------------------------------------
%\include{Introduction}
%\include{Conclusions}
\include{chapters/1Project/main}
\include{chapters/2Lit/main}
\include{chapters/3Design/HighLevel}
\include{chapters/3Design/InDepth}
\include{chapters/4Impl/main}

\include{chapters/5Experiments/1/main}
\include{chapters/5Experiments/2/main}
\include{chapters/5Experiments/3/main}
\include{chapters/5Experiments/4/main}

\include{chapters/6Conclusion/main}

\appendix
\include{appendix/AppendixB}
\include{appendix/D/main}
\include{appendix/AppendixC}

\backmatter
\bibliographystyle{ecs}
\bibliography{ECS}
\end{document}
%% ----------------------------------------------------------------

 %% ----------------------------------------------------------------
%% Progress.tex
%% ---------------------------------------------------------------- 
\documentclass{ecsprogress}    % Use the progress Style
\graphicspath{{../figs/}}   % Location of your graphics files
    \usepackage{natbib}            % Use Natbib style for the refs.
\hypersetup{colorlinks=true}   % Set to false for black/white printing
\input{Definitions}            % Include your abbreviations



\usepackage{enumitem}% http://ctan.org/pkg/enumitem
\usepackage{multirow}
\usepackage{float}
\usepackage{amsmath}
\usepackage{multicol}
\usepackage{amssymb}
\usepackage[normalem]{ulem}
\useunder{\uline}{\ul}{}
\usepackage{wrapfig}


\usepackage[table,xcdraw]{xcolor}


%% ----------------------------------------------------------------
\begin{document}
\frontmatter
\title      {Heterogeneous Agent-based Model for Supermarket Competition}
\authors    {\texorpdfstring
             {\href{mailto:sc22g13@ecs.soton.ac.uk}{Stefan J. Collier}}
             {Stefan J. Collier}
            }
\addresses  {\groupname\\\deptname\\\univname}
\date       {\today}
\subject    {}
\keywords   {}
\supervisor {Dr. Maria Polukarov}
\examiner   {Professor Sheng Chen}

\maketitle
\begin{abstract}
This project aim was to model and analyse the effects of competitive pricing behaviors of grocery retailers on the British market. 

This was achieved by creating a multi-agent model, containing retailer and consumer agents. The heterogeneous crowd of retailers employs either a uniform pricing strategy or a ‘local price flexing’ strategy. The actions of these retailers are chosen by predicting the profit of each action, using a perceptron. Following on from the consideration of different economic models, a discrete model was developed so that software agents have a discrete environment to operate within. Within the model, it has been observed how supermarkets with differing behaviors affect a heterogeneous crowd of consumer agents. The model was implemented in Java with Python used to evaluate the results. 

The simulation displays good acceptance with real grocery market behavior, i.e. captures the performance of British retailers thus can be used to determine the impact of changes in their behavior on their competitors and consumers.Furthermore it can be used to provide insight into sustainability of volatile pricing strategies, providing a useful insight in volatility of British supermarket retail industry. 
\end{abstract}
\acknowledgements{
I would like to express my sincere gratitude to Dr Maria Polukarov for her guidance and support which provided me the freedom to take this research in the direction of my interest.\\
\\
I would also like to thank my family and friends for their encouragement and support. To those who quietly listened to my software complaints. To those who worked throughout the nights with me. To those who helped me write what I couldn't say. I cannot thank you enough.
}

\declaration{
I, Stefan Collier, declare that this dissertation and the work presented in it are my own and has been generated by me as the result of my own original research.\\
I confirm that:\\
1. This work was done wholly or mainly while in candidature for a degree at this University;\\
2. Where any part of this dissertation has previously been submitted for any other qualification at this University or any other institution, this has been clearly stated;\\
3. Where I have consulted the published work of others, this is always clearly attributed;\\
4. Where I have quoted from the work of others, the source is always given. With the exception of such quotations, this dissertation is entirely my own work;\\
5. I have acknowledged all main sources of help;\\
6. Where the thesis is based on work done by myself jointly with others, I have made clear exactly what was done by others and what I have contributed myself;\\
7. Either none of this work has been published before submission, or parts of this work have been published by :\\
\\
Stefan Collier\\
April 2016
}
\tableofcontents
\listoffigures
\listoftables

\mainmatter
%% ----------------------------------------------------------------
%\include{Introduction}
%\include{Conclusions}
\include{chapters/1Project/main}
\include{chapters/2Lit/main}
\include{chapters/3Design/HighLevel}
\include{chapters/3Design/InDepth}
\include{chapters/4Impl/main}

\include{chapters/5Experiments/1/main}
\include{chapters/5Experiments/2/main}
\include{chapters/5Experiments/3/main}
\include{chapters/5Experiments/4/main}

\include{chapters/6Conclusion/main}

\appendix
\include{appendix/AppendixB}
\include{appendix/D/main}
\include{appendix/AppendixC}

\backmatter
\bibliographystyle{ecs}
\bibliography{ECS}
\end{document}
%% ----------------------------------------------------------------

 %% ----------------------------------------------------------------
%% Progress.tex
%% ---------------------------------------------------------------- 
\documentclass{ecsprogress}    % Use the progress Style
\graphicspath{{../figs/}}   % Location of your graphics files
    \usepackage{natbib}            % Use Natbib style for the refs.
\hypersetup{colorlinks=true}   % Set to false for black/white printing
\input{Definitions}            % Include your abbreviations



\usepackage{enumitem}% http://ctan.org/pkg/enumitem
\usepackage{multirow}
\usepackage{float}
\usepackage{amsmath}
\usepackage{multicol}
\usepackage{amssymb}
\usepackage[normalem]{ulem}
\useunder{\uline}{\ul}{}
\usepackage{wrapfig}


\usepackage[table,xcdraw]{xcolor}


%% ----------------------------------------------------------------
\begin{document}
\frontmatter
\title      {Heterogeneous Agent-based Model for Supermarket Competition}
\authors    {\texorpdfstring
             {\href{mailto:sc22g13@ecs.soton.ac.uk}{Stefan J. Collier}}
             {Stefan J. Collier}
            }
\addresses  {\groupname\\\deptname\\\univname}
\date       {\today}
\subject    {}
\keywords   {}
\supervisor {Dr. Maria Polukarov}
\examiner   {Professor Sheng Chen}

\maketitle
\begin{abstract}
This project aim was to model and analyse the effects of competitive pricing behaviors of grocery retailers on the British market. 

This was achieved by creating a multi-agent model, containing retailer and consumer agents. The heterogeneous crowd of retailers employs either a uniform pricing strategy or a ‘local price flexing’ strategy. The actions of these retailers are chosen by predicting the profit of each action, using a perceptron. Following on from the consideration of different economic models, a discrete model was developed so that software agents have a discrete environment to operate within. Within the model, it has been observed how supermarkets with differing behaviors affect a heterogeneous crowd of consumer agents. The model was implemented in Java with Python used to evaluate the results. 

The simulation displays good acceptance with real grocery market behavior, i.e. captures the performance of British retailers thus can be used to determine the impact of changes in their behavior on their competitors and consumers.Furthermore it can be used to provide insight into sustainability of volatile pricing strategies, providing a useful insight in volatility of British supermarket retail industry. 
\end{abstract}
\acknowledgements{
I would like to express my sincere gratitude to Dr Maria Polukarov for her guidance and support which provided me the freedom to take this research in the direction of my interest.\\
\\
I would also like to thank my family and friends for their encouragement and support. To those who quietly listened to my software complaints. To those who worked throughout the nights with me. To those who helped me write what I couldn't say. I cannot thank you enough.
}

\declaration{
I, Stefan Collier, declare that this dissertation and the work presented in it are my own and has been generated by me as the result of my own original research.\\
I confirm that:\\
1. This work was done wholly or mainly while in candidature for a degree at this University;\\
2. Where any part of this dissertation has previously been submitted for any other qualification at this University or any other institution, this has been clearly stated;\\
3. Where I have consulted the published work of others, this is always clearly attributed;\\
4. Where I have quoted from the work of others, the source is always given. With the exception of such quotations, this dissertation is entirely my own work;\\
5. I have acknowledged all main sources of help;\\
6. Where the thesis is based on work done by myself jointly with others, I have made clear exactly what was done by others and what I have contributed myself;\\
7. Either none of this work has been published before submission, or parts of this work have been published by :\\
\\
Stefan Collier\\
April 2016
}
\tableofcontents
\listoffigures
\listoftables

\mainmatter
%% ----------------------------------------------------------------
%\include{Introduction}
%\include{Conclusions}
\include{chapters/1Project/main}
\include{chapters/2Lit/main}
\include{chapters/3Design/HighLevel}
\include{chapters/3Design/InDepth}
\include{chapters/4Impl/main}

\include{chapters/5Experiments/1/main}
\include{chapters/5Experiments/2/main}
\include{chapters/5Experiments/3/main}
\include{chapters/5Experiments/4/main}

\include{chapters/6Conclusion/main}

\appendix
\include{appendix/AppendixB}
\include{appendix/D/main}
\include{appendix/AppendixC}

\backmatter
\bibliographystyle{ecs}
\bibliography{ECS}
\end{document}
%% ----------------------------------------------------------------

 %% ----------------------------------------------------------------
%% Progress.tex
%% ---------------------------------------------------------------- 
\documentclass{ecsprogress}    % Use the progress Style
\graphicspath{{../figs/}}   % Location of your graphics files
    \usepackage{natbib}            % Use Natbib style for the refs.
\hypersetup{colorlinks=true}   % Set to false for black/white printing
\input{Definitions}            % Include your abbreviations



\usepackage{enumitem}% http://ctan.org/pkg/enumitem
\usepackage{multirow}
\usepackage{float}
\usepackage{amsmath}
\usepackage{multicol}
\usepackage{amssymb}
\usepackage[normalem]{ulem}
\useunder{\uline}{\ul}{}
\usepackage{wrapfig}


\usepackage[table,xcdraw]{xcolor}


%% ----------------------------------------------------------------
\begin{document}
\frontmatter
\title      {Heterogeneous Agent-based Model for Supermarket Competition}
\authors    {\texorpdfstring
             {\href{mailto:sc22g13@ecs.soton.ac.uk}{Stefan J. Collier}}
             {Stefan J. Collier}
            }
\addresses  {\groupname\\\deptname\\\univname}
\date       {\today}
\subject    {}
\keywords   {}
\supervisor {Dr. Maria Polukarov}
\examiner   {Professor Sheng Chen}

\maketitle
\begin{abstract}
This project aim was to model and analyse the effects of competitive pricing behaviors of grocery retailers on the British market. 

This was achieved by creating a multi-agent model, containing retailer and consumer agents. The heterogeneous crowd of retailers employs either a uniform pricing strategy or a ‘local price flexing’ strategy. The actions of these retailers are chosen by predicting the profit of each action, using a perceptron. Following on from the consideration of different economic models, a discrete model was developed so that software agents have a discrete environment to operate within. Within the model, it has been observed how supermarkets with differing behaviors affect a heterogeneous crowd of consumer agents. The model was implemented in Java with Python used to evaluate the results. 

The simulation displays good acceptance with real grocery market behavior, i.e. captures the performance of British retailers thus can be used to determine the impact of changes in their behavior on their competitors and consumers.Furthermore it can be used to provide insight into sustainability of volatile pricing strategies, providing a useful insight in volatility of British supermarket retail industry. 
\end{abstract}
\acknowledgements{
I would like to express my sincere gratitude to Dr Maria Polukarov for her guidance and support which provided me the freedom to take this research in the direction of my interest.\\
\\
I would also like to thank my family and friends for their encouragement and support. To those who quietly listened to my software complaints. To those who worked throughout the nights with me. To those who helped me write what I couldn't say. I cannot thank you enough.
}

\declaration{
I, Stefan Collier, declare that this dissertation and the work presented in it are my own and has been generated by me as the result of my own original research.\\
I confirm that:\\
1. This work was done wholly or mainly while in candidature for a degree at this University;\\
2. Where any part of this dissertation has previously been submitted for any other qualification at this University or any other institution, this has been clearly stated;\\
3. Where I have consulted the published work of others, this is always clearly attributed;\\
4. Where I have quoted from the work of others, the source is always given. With the exception of such quotations, this dissertation is entirely my own work;\\
5. I have acknowledged all main sources of help;\\
6. Where the thesis is based on work done by myself jointly with others, I have made clear exactly what was done by others and what I have contributed myself;\\
7. Either none of this work has been published before submission, or parts of this work have been published by :\\
\\
Stefan Collier\\
April 2016
}
\tableofcontents
\listoffigures
\listoftables

\mainmatter
%% ----------------------------------------------------------------
%\include{Introduction}
%\include{Conclusions}
\include{chapters/1Project/main}
\include{chapters/2Lit/main}
\include{chapters/3Design/HighLevel}
\include{chapters/3Design/InDepth}
\include{chapters/4Impl/main}

\include{chapters/5Experiments/1/main}
\include{chapters/5Experiments/2/main}
\include{chapters/5Experiments/3/main}
\include{chapters/5Experiments/4/main}

\include{chapters/6Conclusion/main}

\appendix
\include{appendix/AppendixB}
\include{appendix/D/main}
\include{appendix/AppendixC}

\backmatter
\bibliographystyle{ecs}
\bibliography{ECS}
\end{document}
%% ----------------------------------------------------------------


 %% ----------------------------------------------------------------
%% Progress.tex
%% ---------------------------------------------------------------- 
\documentclass{ecsprogress}    % Use the progress Style
\graphicspath{{../figs/}}   % Location of your graphics files
    \usepackage{natbib}            % Use Natbib style for the refs.
\hypersetup{colorlinks=true}   % Set to false for black/white printing
\input{Definitions}            % Include your abbreviations



\usepackage{enumitem}% http://ctan.org/pkg/enumitem
\usepackage{multirow}
\usepackage{float}
\usepackage{amsmath}
\usepackage{multicol}
\usepackage{amssymb}
\usepackage[normalem]{ulem}
\useunder{\uline}{\ul}{}
\usepackage{wrapfig}


\usepackage[table,xcdraw]{xcolor}


%% ----------------------------------------------------------------
\begin{document}
\frontmatter
\title      {Heterogeneous Agent-based Model for Supermarket Competition}
\authors    {\texorpdfstring
             {\href{mailto:sc22g13@ecs.soton.ac.uk}{Stefan J. Collier}}
             {Stefan J. Collier}
            }
\addresses  {\groupname\\\deptname\\\univname}
\date       {\today}
\subject    {}
\keywords   {}
\supervisor {Dr. Maria Polukarov}
\examiner   {Professor Sheng Chen}

\maketitle
\begin{abstract}
This project aim was to model and analyse the effects of competitive pricing behaviors of grocery retailers on the British market. 

This was achieved by creating a multi-agent model, containing retailer and consumer agents. The heterogeneous crowd of retailers employs either a uniform pricing strategy or a ‘local price flexing’ strategy. The actions of these retailers are chosen by predicting the profit of each action, using a perceptron. Following on from the consideration of different economic models, a discrete model was developed so that software agents have a discrete environment to operate within. Within the model, it has been observed how supermarkets with differing behaviors affect a heterogeneous crowd of consumer agents. The model was implemented in Java with Python used to evaluate the results. 

The simulation displays good acceptance with real grocery market behavior, i.e. captures the performance of British retailers thus can be used to determine the impact of changes in their behavior on their competitors and consumers.Furthermore it can be used to provide insight into sustainability of volatile pricing strategies, providing a useful insight in volatility of British supermarket retail industry. 
\end{abstract}
\acknowledgements{
I would like to express my sincere gratitude to Dr Maria Polukarov for her guidance and support which provided me the freedom to take this research in the direction of my interest.\\
\\
I would also like to thank my family and friends for their encouragement and support. To those who quietly listened to my software complaints. To those who worked throughout the nights with me. To those who helped me write what I couldn't say. I cannot thank you enough.
}

\declaration{
I, Stefan Collier, declare that this dissertation and the work presented in it are my own and has been generated by me as the result of my own original research.\\
I confirm that:\\
1. This work was done wholly or mainly while in candidature for a degree at this University;\\
2. Where any part of this dissertation has previously been submitted for any other qualification at this University or any other institution, this has been clearly stated;\\
3. Where I have consulted the published work of others, this is always clearly attributed;\\
4. Where I have quoted from the work of others, the source is always given. With the exception of such quotations, this dissertation is entirely my own work;\\
5. I have acknowledged all main sources of help;\\
6. Where the thesis is based on work done by myself jointly with others, I have made clear exactly what was done by others and what I have contributed myself;\\
7. Either none of this work has been published before submission, or parts of this work have been published by :\\
\\
Stefan Collier\\
April 2016
}
\tableofcontents
\listoffigures
\listoftables

\mainmatter
%% ----------------------------------------------------------------
%\include{Introduction}
%\include{Conclusions}
\include{chapters/1Project/main}
\include{chapters/2Lit/main}
\include{chapters/3Design/HighLevel}
\include{chapters/3Design/InDepth}
\include{chapters/4Impl/main}

\include{chapters/5Experiments/1/main}
\include{chapters/5Experiments/2/main}
\include{chapters/5Experiments/3/main}
\include{chapters/5Experiments/4/main}

\include{chapters/6Conclusion/main}

\appendix
\include{appendix/AppendixB}
\include{appendix/D/main}
\include{appendix/AppendixC}

\backmatter
\bibliographystyle{ecs}
\bibliography{ECS}
\end{document}
%% ----------------------------------------------------------------


\appendix
\include{appendix/AppendixB}
 %% ----------------------------------------------------------------
%% Progress.tex
%% ---------------------------------------------------------------- 
\documentclass{ecsprogress}    % Use the progress Style
\graphicspath{{../figs/}}   % Location of your graphics files
    \usepackage{natbib}            % Use Natbib style for the refs.
\hypersetup{colorlinks=true}   % Set to false for black/white printing
\input{Definitions}            % Include your abbreviations



\usepackage{enumitem}% http://ctan.org/pkg/enumitem
\usepackage{multirow}
\usepackage{float}
\usepackage{amsmath}
\usepackage{multicol}
\usepackage{amssymb}
\usepackage[normalem]{ulem}
\useunder{\uline}{\ul}{}
\usepackage{wrapfig}


\usepackage[table,xcdraw]{xcolor}


%% ----------------------------------------------------------------
\begin{document}
\frontmatter
\title      {Heterogeneous Agent-based Model for Supermarket Competition}
\authors    {\texorpdfstring
             {\href{mailto:sc22g13@ecs.soton.ac.uk}{Stefan J. Collier}}
             {Stefan J. Collier}
            }
\addresses  {\groupname\\\deptname\\\univname}
\date       {\today}
\subject    {}
\keywords   {}
\supervisor {Dr. Maria Polukarov}
\examiner   {Professor Sheng Chen}

\maketitle
\begin{abstract}
This project aim was to model and analyse the effects of competitive pricing behaviors of grocery retailers on the British market. 

This was achieved by creating a multi-agent model, containing retailer and consumer agents. The heterogeneous crowd of retailers employs either a uniform pricing strategy or a ‘local price flexing’ strategy. The actions of these retailers are chosen by predicting the profit of each action, using a perceptron. Following on from the consideration of different economic models, a discrete model was developed so that software agents have a discrete environment to operate within. Within the model, it has been observed how supermarkets with differing behaviors affect a heterogeneous crowd of consumer agents. The model was implemented in Java with Python used to evaluate the results. 

The simulation displays good acceptance with real grocery market behavior, i.e. captures the performance of British retailers thus can be used to determine the impact of changes in their behavior on their competitors and consumers.Furthermore it can be used to provide insight into sustainability of volatile pricing strategies, providing a useful insight in volatility of British supermarket retail industry. 
\end{abstract}
\acknowledgements{
I would like to express my sincere gratitude to Dr Maria Polukarov for her guidance and support which provided me the freedom to take this research in the direction of my interest.\\
\\
I would also like to thank my family and friends for their encouragement and support. To those who quietly listened to my software complaints. To those who worked throughout the nights with me. To those who helped me write what I couldn't say. I cannot thank you enough.
}

\declaration{
I, Stefan Collier, declare that this dissertation and the work presented in it are my own and has been generated by me as the result of my own original research.\\
I confirm that:\\
1. This work was done wholly or mainly while in candidature for a degree at this University;\\
2. Where any part of this dissertation has previously been submitted for any other qualification at this University or any other institution, this has been clearly stated;\\
3. Where I have consulted the published work of others, this is always clearly attributed;\\
4. Where I have quoted from the work of others, the source is always given. With the exception of such quotations, this dissertation is entirely my own work;\\
5. I have acknowledged all main sources of help;\\
6. Where the thesis is based on work done by myself jointly with others, I have made clear exactly what was done by others and what I have contributed myself;\\
7. Either none of this work has been published before submission, or parts of this work have been published by :\\
\\
Stefan Collier\\
April 2016
}
\tableofcontents
\listoffigures
\listoftables

\mainmatter
%% ----------------------------------------------------------------
%\include{Introduction}
%\include{Conclusions}
\include{chapters/1Project/main}
\include{chapters/2Lit/main}
\include{chapters/3Design/HighLevel}
\include{chapters/3Design/InDepth}
\include{chapters/4Impl/main}

\include{chapters/5Experiments/1/main}
\include{chapters/5Experiments/2/main}
\include{chapters/5Experiments/3/main}
\include{chapters/5Experiments/4/main}

\include{chapters/6Conclusion/main}

\appendix
\include{appendix/AppendixB}
\include{appendix/D/main}
\include{appendix/AppendixC}

\backmatter
\bibliographystyle{ecs}
\bibliography{ECS}
\end{document}
%% ----------------------------------------------------------------

\include{appendix/AppendixC}

\backmatter
\bibliographystyle{ecs}
\bibliography{ECS}
\end{document}
%% ----------------------------------------------------------------

 %% ----------------------------------------------------------------
%% Progress.tex
%% ---------------------------------------------------------------- 
\documentclass{ecsprogress}    % Use the progress Style
\graphicspath{{../figs/}}   % Location of your graphics files
    \usepackage{natbib}            % Use Natbib style for the refs.
\hypersetup{colorlinks=true}   % Set to false for black/white printing
\input{Definitions}            % Include your abbreviations



\usepackage{enumitem}% http://ctan.org/pkg/enumitem
\usepackage{multirow}
\usepackage{float}
\usepackage{amsmath}
\usepackage{multicol}
\usepackage{amssymb}
\usepackage[normalem]{ulem}
\useunder{\uline}{\ul}{}
\usepackage{wrapfig}


\usepackage[table,xcdraw]{xcolor}


%% ----------------------------------------------------------------
\begin{document}
\frontmatter
\title      {Heterogeneous Agent-based Model for Supermarket Competition}
\authors    {\texorpdfstring
             {\href{mailto:sc22g13@ecs.soton.ac.uk}{Stefan J. Collier}}
             {Stefan J. Collier}
            }
\addresses  {\groupname\\\deptname\\\univname}
\date       {\today}
\subject    {}
\keywords   {}
\supervisor {Dr. Maria Polukarov}
\examiner   {Professor Sheng Chen}

\maketitle
\begin{abstract}
This project aim was to model and analyse the effects of competitive pricing behaviors of grocery retailers on the British market. 

This was achieved by creating a multi-agent model, containing retailer and consumer agents. The heterogeneous crowd of retailers employs either a uniform pricing strategy or a ‘local price flexing’ strategy. The actions of these retailers are chosen by predicting the profit of each action, using a perceptron. Following on from the consideration of different economic models, a discrete model was developed so that software agents have a discrete environment to operate within. Within the model, it has been observed how supermarkets with differing behaviors affect a heterogeneous crowd of consumer agents. The model was implemented in Java with Python used to evaluate the results. 

The simulation displays good acceptance with real grocery market behavior, i.e. captures the performance of British retailers thus can be used to determine the impact of changes in their behavior on their competitors and consumers.Furthermore it can be used to provide insight into sustainability of volatile pricing strategies, providing a useful insight in volatility of British supermarket retail industry. 
\end{abstract}
\acknowledgements{
I would like to express my sincere gratitude to Dr Maria Polukarov for her guidance and support which provided me the freedom to take this research in the direction of my interest.\\
\\
I would also like to thank my family and friends for their encouragement and support. To those who quietly listened to my software complaints. To those who worked throughout the nights with me. To those who helped me write what I couldn't say. I cannot thank you enough.
}

\declaration{
I, Stefan Collier, declare that this dissertation and the work presented in it are my own and has been generated by me as the result of my own original research.\\
I confirm that:\\
1. This work was done wholly or mainly while in candidature for a degree at this University;\\
2. Where any part of this dissertation has previously been submitted for any other qualification at this University or any other institution, this has been clearly stated;\\
3. Where I have consulted the published work of others, this is always clearly attributed;\\
4. Where I have quoted from the work of others, the source is always given. With the exception of such quotations, this dissertation is entirely my own work;\\
5. I have acknowledged all main sources of help;\\
6. Where the thesis is based on work done by myself jointly with others, I have made clear exactly what was done by others and what I have contributed myself;\\
7. Either none of this work has been published before submission, or parts of this work have been published by :\\
\\
Stefan Collier\\
April 2016
}
\tableofcontents
\listoffigures
\listoftables

\mainmatter
%% ----------------------------------------------------------------
%\include{Introduction}
%\include{Conclusions}
 %% ----------------------------------------------------------------
%% Progress.tex
%% ---------------------------------------------------------------- 
\documentclass{ecsprogress}    % Use the progress Style
\graphicspath{{../figs/}}   % Location of your graphics files
    \usepackage{natbib}            % Use Natbib style for the refs.
\hypersetup{colorlinks=true}   % Set to false for black/white printing
\input{Definitions}            % Include your abbreviations



\usepackage{enumitem}% http://ctan.org/pkg/enumitem
\usepackage{multirow}
\usepackage{float}
\usepackage{amsmath}
\usepackage{multicol}
\usepackage{amssymb}
\usepackage[normalem]{ulem}
\useunder{\uline}{\ul}{}
\usepackage{wrapfig}


\usepackage[table,xcdraw]{xcolor}


%% ----------------------------------------------------------------
\begin{document}
\frontmatter
\title      {Heterogeneous Agent-based Model for Supermarket Competition}
\authors    {\texorpdfstring
             {\href{mailto:sc22g13@ecs.soton.ac.uk}{Stefan J. Collier}}
             {Stefan J. Collier}
            }
\addresses  {\groupname\\\deptname\\\univname}
\date       {\today}
\subject    {}
\keywords   {}
\supervisor {Dr. Maria Polukarov}
\examiner   {Professor Sheng Chen}

\maketitle
\begin{abstract}
This project aim was to model and analyse the effects of competitive pricing behaviors of grocery retailers on the British market. 

This was achieved by creating a multi-agent model, containing retailer and consumer agents. The heterogeneous crowd of retailers employs either a uniform pricing strategy or a ‘local price flexing’ strategy. The actions of these retailers are chosen by predicting the profit of each action, using a perceptron. Following on from the consideration of different economic models, a discrete model was developed so that software agents have a discrete environment to operate within. Within the model, it has been observed how supermarkets with differing behaviors affect a heterogeneous crowd of consumer agents. The model was implemented in Java with Python used to evaluate the results. 

The simulation displays good acceptance with real grocery market behavior, i.e. captures the performance of British retailers thus can be used to determine the impact of changes in their behavior on their competitors and consumers.Furthermore it can be used to provide insight into sustainability of volatile pricing strategies, providing a useful insight in volatility of British supermarket retail industry. 
\end{abstract}
\acknowledgements{
I would like to express my sincere gratitude to Dr Maria Polukarov for her guidance and support which provided me the freedom to take this research in the direction of my interest.\\
\\
I would also like to thank my family and friends for their encouragement and support. To those who quietly listened to my software complaints. To those who worked throughout the nights with me. To those who helped me write what I couldn't say. I cannot thank you enough.
}

\declaration{
I, Stefan Collier, declare that this dissertation and the work presented in it are my own and has been generated by me as the result of my own original research.\\
I confirm that:\\
1. This work was done wholly or mainly while in candidature for a degree at this University;\\
2. Where any part of this dissertation has previously been submitted for any other qualification at this University or any other institution, this has been clearly stated;\\
3. Where I have consulted the published work of others, this is always clearly attributed;\\
4. Where I have quoted from the work of others, the source is always given. With the exception of such quotations, this dissertation is entirely my own work;\\
5. I have acknowledged all main sources of help;\\
6. Where the thesis is based on work done by myself jointly with others, I have made clear exactly what was done by others and what I have contributed myself;\\
7. Either none of this work has been published before submission, or parts of this work have been published by :\\
\\
Stefan Collier\\
April 2016
}
\tableofcontents
\listoffigures
\listoftables

\mainmatter
%% ----------------------------------------------------------------
%\include{Introduction}
%\include{Conclusions}
\include{chapters/1Project/main}
\include{chapters/2Lit/main}
\include{chapters/3Design/HighLevel}
\include{chapters/3Design/InDepth}
\include{chapters/4Impl/main}

\include{chapters/5Experiments/1/main}
\include{chapters/5Experiments/2/main}
\include{chapters/5Experiments/3/main}
\include{chapters/5Experiments/4/main}

\include{chapters/6Conclusion/main}

\appendix
\include{appendix/AppendixB}
\include{appendix/D/main}
\include{appendix/AppendixC}

\backmatter
\bibliographystyle{ecs}
\bibliography{ECS}
\end{document}
%% ----------------------------------------------------------------

 %% ----------------------------------------------------------------
%% Progress.tex
%% ---------------------------------------------------------------- 
\documentclass{ecsprogress}    % Use the progress Style
\graphicspath{{../figs/}}   % Location of your graphics files
    \usepackage{natbib}            % Use Natbib style for the refs.
\hypersetup{colorlinks=true}   % Set to false for black/white printing
\input{Definitions}            % Include your abbreviations



\usepackage{enumitem}% http://ctan.org/pkg/enumitem
\usepackage{multirow}
\usepackage{float}
\usepackage{amsmath}
\usepackage{multicol}
\usepackage{amssymb}
\usepackage[normalem]{ulem}
\useunder{\uline}{\ul}{}
\usepackage{wrapfig}


\usepackage[table,xcdraw]{xcolor}


%% ----------------------------------------------------------------
\begin{document}
\frontmatter
\title      {Heterogeneous Agent-based Model for Supermarket Competition}
\authors    {\texorpdfstring
             {\href{mailto:sc22g13@ecs.soton.ac.uk}{Stefan J. Collier}}
             {Stefan J. Collier}
            }
\addresses  {\groupname\\\deptname\\\univname}
\date       {\today}
\subject    {}
\keywords   {}
\supervisor {Dr. Maria Polukarov}
\examiner   {Professor Sheng Chen}

\maketitle
\begin{abstract}
This project aim was to model and analyse the effects of competitive pricing behaviors of grocery retailers on the British market. 

This was achieved by creating a multi-agent model, containing retailer and consumer agents. The heterogeneous crowd of retailers employs either a uniform pricing strategy or a ‘local price flexing’ strategy. The actions of these retailers are chosen by predicting the profit of each action, using a perceptron. Following on from the consideration of different economic models, a discrete model was developed so that software agents have a discrete environment to operate within. Within the model, it has been observed how supermarkets with differing behaviors affect a heterogeneous crowd of consumer agents. The model was implemented in Java with Python used to evaluate the results. 

The simulation displays good acceptance with real grocery market behavior, i.e. captures the performance of British retailers thus can be used to determine the impact of changes in their behavior on their competitors and consumers.Furthermore it can be used to provide insight into sustainability of volatile pricing strategies, providing a useful insight in volatility of British supermarket retail industry. 
\end{abstract}
\acknowledgements{
I would like to express my sincere gratitude to Dr Maria Polukarov for her guidance and support which provided me the freedom to take this research in the direction of my interest.\\
\\
I would also like to thank my family and friends for their encouragement and support. To those who quietly listened to my software complaints. To those who worked throughout the nights with me. To those who helped me write what I couldn't say. I cannot thank you enough.
}

\declaration{
I, Stefan Collier, declare that this dissertation and the work presented in it are my own and has been generated by me as the result of my own original research.\\
I confirm that:\\
1. This work was done wholly or mainly while in candidature for a degree at this University;\\
2. Where any part of this dissertation has previously been submitted for any other qualification at this University or any other institution, this has been clearly stated;\\
3. Where I have consulted the published work of others, this is always clearly attributed;\\
4. Where I have quoted from the work of others, the source is always given. With the exception of such quotations, this dissertation is entirely my own work;\\
5. I have acknowledged all main sources of help;\\
6. Where the thesis is based on work done by myself jointly with others, I have made clear exactly what was done by others and what I have contributed myself;\\
7. Either none of this work has been published before submission, or parts of this work have been published by :\\
\\
Stefan Collier\\
April 2016
}
\tableofcontents
\listoffigures
\listoftables

\mainmatter
%% ----------------------------------------------------------------
%\include{Introduction}
%\include{Conclusions}
\include{chapters/1Project/main}
\include{chapters/2Lit/main}
\include{chapters/3Design/HighLevel}
\include{chapters/3Design/InDepth}
\include{chapters/4Impl/main}

\include{chapters/5Experiments/1/main}
\include{chapters/5Experiments/2/main}
\include{chapters/5Experiments/3/main}
\include{chapters/5Experiments/4/main}

\include{chapters/6Conclusion/main}

\appendix
\include{appendix/AppendixB}
\include{appendix/D/main}
\include{appendix/AppendixC}

\backmatter
\bibliographystyle{ecs}
\bibliography{ECS}
\end{document}
%% ----------------------------------------------------------------

\include{chapters/3Design/HighLevel}
\include{chapters/3Design/InDepth}
 %% ----------------------------------------------------------------
%% Progress.tex
%% ---------------------------------------------------------------- 
\documentclass{ecsprogress}    % Use the progress Style
\graphicspath{{../figs/}}   % Location of your graphics files
    \usepackage{natbib}            % Use Natbib style for the refs.
\hypersetup{colorlinks=true}   % Set to false for black/white printing
\input{Definitions}            % Include your abbreviations



\usepackage{enumitem}% http://ctan.org/pkg/enumitem
\usepackage{multirow}
\usepackage{float}
\usepackage{amsmath}
\usepackage{multicol}
\usepackage{amssymb}
\usepackage[normalem]{ulem}
\useunder{\uline}{\ul}{}
\usepackage{wrapfig}


\usepackage[table,xcdraw]{xcolor}


%% ----------------------------------------------------------------
\begin{document}
\frontmatter
\title      {Heterogeneous Agent-based Model for Supermarket Competition}
\authors    {\texorpdfstring
             {\href{mailto:sc22g13@ecs.soton.ac.uk}{Stefan J. Collier}}
             {Stefan J. Collier}
            }
\addresses  {\groupname\\\deptname\\\univname}
\date       {\today}
\subject    {}
\keywords   {}
\supervisor {Dr. Maria Polukarov}
\examiner   {Professor Sheng Chen}

\maketitle
\begin{abstract}
This project aim was to model and analyse the effects of competitive pricing behaviors of grocery retailers on the British market. 

This was achieved by creating a multi-agent model, containing retailer and consumer agents. The heterogeneous crowd of retailers employs either a uniform pricing strategy or a ‘local price flexing’ strategy. The actions of these retailers are chosen by predicting the profit of each action, using a perceptron. Following on from the consideration of different economic models, a discrete model was developed so that software agents have a discrete environment to operate within. Within the model, it has been observed how supermarkets with differing behaviors affect a heterogeneous crowd of consumer agents. The model was implemented in Java with Python used to evaluate the results. 

The simulation displays good acceptance with real grocery market behavior, i.e. captures the performance of British retailers thus can be used to determine the impact of changes in their behavior on their competitors and consumers.Furthermore it can be used to provide insight into sustainability of volatile pricing strategies, providing a useful insight in volatility of British supermarket retail industry. 
\end{abstract}
\acknowledgements{
I would like to express my sincere gratitude to Dr Maria Polukarov for her guidance and support which provided me the freedom to take this research in the direction of my interest.\\
\\
I would also like to thank my family and friends for their encouragement and support. To those who quietly listened to my software complaints. To those who worked throughout the nights with me. To those who helped me write what I couldn't say. I cannot thank you enough.
}

\declaration{
I, Stefan Collier, declare that this dissertation and the work presented in it are my own and has been generated by me as the result of my own original research.\\
I confirm that:\\
1. This work was done wholly or mainly while in candidature for a degree at this University;\\
2. Where any part of this dissertation has previously been submitted for any other qualification at this University or any other institution, this has been clearly stated;\\
3. Where I have consulted the published work of others, this is always clearly attributed;\\
4. Where I have quoted from the work of others, the source is always given. With the exception of such quotations, this dissertation is entirely my own work;\\
5. I have acknowledged all main sources of help;\\
6. Where the thesis is based on work done by myself jointly with others, I have made clear exactly what was done by others and what I have contributed myself;\\
7. Either none of this work has been published before submission, or parts of this work have been published by :\\
\\
Stefan Collier\\
April 2016
}
\tableofcontents
\listoffigures
\listoftables

\mainmatter
%% ----------------------------------------------------------------
%\include{Introduction}
%\include{Conclusions}
\include{chapters/1Project/main}
\include{chapters/2Lit/main}
\include{chapters/3Design/HighLevel}
\include{chapters/3Design/InDepth}
\include{chapters/4Impl/main}

\include{chapters/5Experiments/1/main}
\include{chapters/5Experiments/2/main}
\include{chapters/5Experiments/3/main}
\include{chapters/5Experiments/4/main}

\include{chapters/6Conclusion/main}

\appendix
\include{appendix/AppendixB}
\include{appendix/D/main}
\include{appendix/AppendixC}

\backmatter
\bibliographystyle{ecs}
\bibliography{ECS}
\end{document}
%% ----------------------------------------------------------------


 %% ----------------------------------------------------------------
%% Progress.tex
%% ---------------------------------------------------------------- 
\documentclass{ecsprogress}    % Use the progress Style
\graphicspath{{../figs/}}   % Location of your graphics files
    \usepackage{natbib}            % Use Natbib style for the refs.
\hypersetup{colorlinks=true}   % Set to false for black/white printing
\input{Definitions}            % Include your abbreviations



\usepackage{enumitem}% http://ctan.org/pkg/enumitem
\usepackage{multirow}
\usepackage{float}
\usepackage{amsmath}
\usepackage{multicol}
\usepackage{amssymb}
\usepackage[normalem]{ulem}
\useunder{\uline}{\ul}{}
\usepackage{wrapfig}


\usepackage[table,xcdraw]{xcolor}


%% ----------------------------------------------------------------
\begin{document}
\frontmatter
\title      {Heterogeneous Agent-based Model for Supermarket Competition}
\authors    {\texorpdfstring
             {\href{mailto:sc22g13@ecs.soton.ac.uk}{Stefan J. Collier}}
             {Stefan J. Collier}
            }
\addresses  {\groupname\\\deptname\\\univname}
\date       {\today}
\subject    {}
\keywords   {}
\supervisor {Dr. Maria Polukarov}
\examiner   {Professor Sheng Chen}

\maketitle
\begin{abstract}
This project aim was to model and analyse the effects of competitive pricing behaviors of grocery retailers on the British market. 

This was achieved by creating a multi-agent model, containing retailer and consumer agents. The heterogeneous crowd of retailers employs either a uniform pricing strategy or a ‘local price flexing’ strategy. The actions of these retailers are chosen by predicting the profit of each action, using a perceptron. Following on from the consideration of different economic models, a discrete model was developed so that software agents have a discrete environment to operate within. Within the model, it has been observed how supermarkets with differing behaviors affect a heterogeneous crowd of consumer agents. The model was implemented in Java with Python used to evaluate the results. 

The simulation displays good acceptance with real grocery market behavior, i.e. captures the performance of British retailers thus can be used to determine the impact of changes in their behavior on their competitors and consumers.Furthermore it can be used to provide insight into sustainability of volatile pricing strategies, providing a useful insight in volatility of British supermarket retail industry. 
\end{abstract}
\acknowledgements{
I would like to express my sincere gratitude to Dr Maria Polukarov for her guidance and support which provided me the freedom to take this research in the direction of my interest.\\
\\
I would also like to thank my family and friends for their encouragement and support. To those who quietly listened to my software complaints. To those who worked throughout the nights with me. To those who helped me write what I couldn't say. I cannot thank you enough.
}

\declaration{
I, Stefan Collier, declare that this dissertation and the work presented in it are my own and has been generated by me as the result of my own original research.\\
I confirm that:\\
1. This work was done wholly or mainly while in candidature for a degree at this University;\\
2. Where any part of this dissertation has previously been submitted for any other qualification at this University or any other institution, this has been clearly stated;\\
3. Where I have consulted the published work of others, this is always clearly attributed;\\
4. Where I have quoted from the work of others, the source is always given. With the exception of such quotations, this dissertation is entirely my own work;\\
5. I have acknowledged all main sources of help;\\
6. Where the thesis is based on work done by myself jointly with others, I have made clear exactly what was done by others and what I have contributed myself;\\
7. Either none of this work has been published before submission, or parts of this work have been published by :\\
\\
Stefan Collier\\
April 2016
}
\tableofcontents
\listoffigures
\listoftables

\mainmatter
%% ----------------------------------------------------------------
%\include{Introduction}
%\include{Conclusions}
\include{chapters/1Project/main}
\include{chapters/2Lit/main}
\include{chapters/3Design/HighLevel}
\include{chapters/3Design/InDepth}
\include{chapters/4Impl/main}

\include{chapters/5Experiments/1/main}
\include{chapters/5Experiments/2/main}
\include{chapters/5Experiments/3/main}
\include{chapters/5Experiments/4/main}

\include{chapters/6Conclusion/main}

\appendix
\include{appendix/AppendixB}
\include{appendix/D/main}
\include{appendix/AppendixC}

\backmatter
\bibliographystyle{ecs}
\bibliography{ECS}
\end{document}
%% ----------------------------------------------------------------

 %% ----------------------------------------------------------------
%% Progress.tex
%% ---------------------------------------------------------------- 
\documentclass{ecsprogress}    % Use the progress Style
\graphicspath{{../figs/}}   % Location of your graphics files
    \usepackage{natbib}            % Use Natbib style for the refs.
\hypersetup{colorlinks=true}   % Set to false for black/white printing
\input{Definitions}            % Include your abbreviations



\usepackage{enumitem}% http://ctan.org/pkg/enumitem
\usepackage{multirow}
\usepackage{float}
\usepackage{amsmath}
\usepackage{multicol}
\usepackage{amssymb}
\usepackage[normalem]{ulem}
\useunder{\uline}{\ul}{}
\usepackage{wrapfig}


\usepackage[table,xcdraw]{xcolor}


%% ----------------------------------------------------------------
\begin{document}
\frontmatter
\title      {Heterogeneous Agent-based Model for Supermarket Competition}
\authors    {\texorpdfstring
             {\href{mailto:sc22g13@ecs.soton.ac.uk}{Stefan J. Collier}}
             {Stefan J. Collier}
            }
\addresses  {\groupname\\\deptname\\\univname}
\date       {\today}
\subject    {}
\keywords   {}
\supervisor {Dr. Maria Polukarov}
\examiner   {Professor Sheng Chen}

\maketitle
\begin{abstract}
This project aim was to model and analyse the effects of competitive pricing behaviors of grocery retailers on the British market. 

This was achieved by creating a multi-agent model, containing retailer and consumer agents. The heterogeneous crowd of retailers employs either a uniform pricing strategy or a ‘local price flexing’ strategy. The actions of these retailers are chosen by predicting the profit of each action, using a perceptron. Following on from the consideration of different economic models, a discrete model was developed so that software agents have a discrete environment to operate within. Within the model, it has been observed how supermarkets with differing behaviors affect a heterogeneous crowd of consumer agents. The model was implemented in Java with Python used to evaluate the results. 

The simulation displays good acceptance with real grocery market behavior, i.e. captures the performance of British retailers thus can be used to determine the impact of changes in their behavior on their competitors and consumers.Furthermore it can be used to provide insight into sustainability of volatile pricing strategies, providing a useful insight in volatility of British supermarket retail industry. 
\end{abstract}
\acknowledgements{
I would like to express my sincere gratitude to Dr Maria Polukarov for her guidance and support which provided me the freedom to take this research in the direction of my interest.\\
\\
I would also like to thank my family and friends for their encouragement and support. To those who quietly listened to my software complaints. To those who worked throughout the nights with me. To those who helped me write what I couldn't say. I cannot thank you enough.
}

\declaration{
I, Stefan Collier, declare that this dissertation and the work presented in it are my own and has been generated by me as the result of my own original research.\\
I confirm that:\\
1. This work was done wholly or mainly while in candidature for a degree at this University;\\
2. Where any part of this dissertation has previously been submitted for any other qualification at this University or any other institution, this has been clearly stated;\\
3. Where I have consulted the published work of others, this is always clearly attributed;\\
4. Where I have quoted from the work of others, the source is always given. With the exception of such quotations, this dissertation is entirely my own work;\\
5. I have acknowledged all main sources of help;\\
6. Where the thesis is based on work done by myself jointly with others, I have made clear exactly what was done by others and what I have contributed myself;\\
7. Either none of this work has been published before submission, or parts of this work have been published by :\\
\\
Stefan Collier\\
April 2016
}
\tableofcontents
\listoffigures
\listoftables

\mainmatter
%% ----------------------------------------------------------------
%\include{Introduction}
%\include{Conclusions}
\include{chapters/1Project/main}
\include{chapters/2Lit/main}
\include{chapters/3Design/HighLevel}
\include{chapters/3Design/InDepth}
\include{chapters/4Impl/main}

\include{chapters/5Experiments/1/main}
\include{chapters/5Experiments/2/main}
\include{chapters/5Experiments/3/main}
\include{chapters/5Experiments/4/main}

\include{chapters/6Conclusion/main}

\appendix
\include{appendix/AppendixB}
\include{appendix/D/main}
\include{appendix/AppendixC}

\backmatter
\bibliographystyle{ecs}
\bibliography{ECS}
\end{document}
%% ----------------------------------------------------------------

 %% ----------------------------------------------------------------
%% Progress.tex
%% ---------------------------------------------------------------- 
\documentclass{ecsprogress}    % Use the progress Style
\graphicspath{{../figs/}}   % Location of your graphics files
    \usepackage{natbib}            % Use Natbib style for the refs.
\hypersetup{colorlinks=true}   % Set to false for black/white printing
\input{Definitions}            % Include your abbreviations



\usepackage{enumitem}% http://ctan.org/pkg/enumitem
\usepackage{multirow}
\usepackage{float}
\usepackage{amsmath}
\usepackage{multicol}
\usepackage{amssymb}
\usepackage[normalem]{ulem}
\useunder{\uline}{\ul}{}
\usepackage{wrapfig}


\usepackage[table,xcdraw]{xcolor}


%% ----------------------------------------------------------------
\begin{document}
\frontmatter
\title      {Heterogeneous Agent-based Model for Supermarket Competition}
\authors    {\texorpdfstring
             {\href{mailto:sc22g13@ecs.soton.ac.uk}{Stefan J. Collier}}
             {Stefan J. Collier}
            }
\addresses  {\groupname\\\deptname\\\univname}
\date       {\today}
\subject    {}
\keywords   {}
\supervisor {Dr. Maria Polukarov}
\examiner   {Professor Sheng Chen}

\maketitle
\begin{abstract}
This project aim was to model and analyse the effects of competitive pricing behaviors of grocery retailers on the British market. 

This was achieved by creating a multi-agent model, containing retailer and consumer agents. The heterogeneous crowd of retailers employs either a uniform pricing strategy or a ‘local price flexing’ strategy. The actions of these retailers are chosen by predicting the profit of each action, using a perceptron. Following on from the consideration of different economic models, a discrete model was developed so that software agents have a discrete environment to operate within. Within the model, it has been observed how supermarkets with differing behaviors affect a heterogeneous crowd of consumer agents. The model was implemented in Java with Python used to evaluate the results. 

The simulation displays good acceptance with real grocery market behavior, i.e. captures the performance of British retailers thus can be used to determine the impact of changes in their behavior on their competitors and consumers.Furthermore it can be used to provide insight into sustainability of volatile pricing strategies, providing a useful insight in volatility of British supermarket retail industry. 
\end{abstract}
\acknowledgements{
I would like to express my sincere gratitude to Dr Maria Polukarov for her guidance and support which provided me the freedom to take this research in the direction of my interest.\\
\\
I would also like to thank my family and friends for their encouragement and support. To those who quietly listened to my software complaints. To those who worked throughout the nights with me. To those who helped me write what I couldn't say. I cannot thank you enough.
}

\declaration{
I, Stefan Collier, declare that this dissertation and the work presented in it are my own and has been generated by me as the result of my own original research.\\
I confirm that:\\
1. This work was done wholly or mainly while in candidature for a degree at this University;\\
2. Where any part of this dissertation has previously been submitted for any other qualification at this University or any other institution, this has been clearly stated;\\
3. Where I have consulted the published work of others, this is always clearly attributed;\\
4. Where I have quoted from the work of others, the source is always given. With the exception of such quotations, this dissertation is entirely my own work;\\
5. I have acknowledged all main sources of help;\\
6. Where the thesis is based on work done by myself jointly with others, I have made clear exactly what was done by others and what I have contributed myself;\\
7. Either none of this work has been published before submission, or parts of this work have been published by :\\
\\
Stefan Collier\\
April 2016
}
\tableofcontents
\listoffigures
\listoftables

\mainmatter
%% ----------------------------------------------------------------
%\include{Introduction}
%\include{Conclusions}
\include{chapters/1Project/main}
\include{chapters/2Lit/main}
\include{chapters/3Design/HighLevel}
\include{chapters/3Design/InDepth}
\include{chapters/4Impl/main}

\include{chapters/5Experiments/1/main}
\include{chapters/5Experiments/2/main}
\include{chapters/5Experiments/3/main}
\include{chapters/5Experiments/4/main}

\include{chapters/6Conclusion/main}

\appendix
\include{appendix/AppendixB}
\include{appendix/D/main}
\include{appendix/AppendixC}

\backmatter
\bibliographystyle{ecs}
\bibliography{ECS}
\end{document}
%% ----------------------------------------------------------------

 %% ----------------------------------------------------------------
%% Progress.tex
%% ---------------------------------------------------------------- 
\documentclass{ecsprogress}    % Use the progress Style
\graphicspath{{../figs/}}   % Location of your graphics files
    \usepackage{natbib}            % Use Natbib style for the refs.
\hypersetup{colorlinks=true}   % Set to false for black/white printing
\input{Definitions}            % Include your abbreviations



\usepackage{enumitem}% http://ctan.org/pkg/enumitem
\usepackage{multirow}
\usepackage{float}
\usepackage{amsmath}
\usepackage{multicol}
\usepackage{amssymb}
\usepackage[normalem]{ulem}
\useunder{\uline}{\ul}{}
\usepackage{wrapfig}


\usepackage[table,xcdraw]{xcolor}


%% ----------------------------------------------------------------
\begin{document}
\frontmatter
\title      {Heterogeneous Agent-based Model for Supermarket Competition}
\authors    {\texorpdfstring
             {\href{mailto:sc22g13@ecs.soton.ac.uk}{Stefan J. Collier}}
             {Stefan J. Collier}
            }
\addresses  {\groupname\\\deptname\\\univname}
\date       {\today}
\subject    {}
\keywords   {}
\supervisor {Dr. Maria Polukarov}
\examiner   {Professor Sheng Chen}

\maketitle
\begin{abstract}
This project aim was to model and analyse the effects of competitive pricing behaviors of grocery retailers on the British market. 

This was achieved by creating a multi-agent model, containing retailer and consumer agents. The heterogeneous crowd of retailers employs either a uniform pricing strategy or a ‘local price flexing’ strategy. The actions of these retailers are chosen by predicting the profit of each action, using a perceptron. Following on from the consideration of different economic models, a discrete model was developed so that software agents have a discrete environment to operate within. Within the model, it has been observed how supermarkets with differing behaviors affect a heterogeneous crowd of consumer agents. The model was implemented in Java with Python used to evaluate the results. 

The simulation displays good acceptance with real grocery market behavior, i.e. captures the performance of British retailers thus can be used to determine the impact of changes in their behavior on their competitors and consumers.Furthermore it can be used to provide insight into sustainability of volatile pricing strategies, providing a useful insight in volatility of British supermarket retail industry. 
\end{abstract}
\acknowledgements{
I would like to express my sincere gratitude to Dr Maria Polukarov for her guidance and support which provided me the freedom to take this research in the direction of my interest.\\
\\
I would also like to thank my family and friends for their encouragement and support. To those who quietly listened to my software complaints. To those who worked throughout the nights with me. To those who helped me write what I couldn't say. I cannot thank you enough.
}

\declaration{
I, Stefan Collier, declare that this dissertation and the work presented in it are my own and has been generated by me as the result of my own original research.\\
I confirm that:\\
1. This work was done wholly or mainly while in candidature for a degree at this University;\\
2. Where any part of this dissertation has previously been submitted for any other qualification at this University or any other institution, this has been clearly stated;\\
3. Where I have consulted the published work of others, this is always clearly attributed;\\
4. Where I have quoted from the work of others, the source is always given. With the exception of such quotations, this dissertation is entirely my own work;\\
5. I have acknowledged all main sources of help;\\
6. Where the thesis is based on work done by myself jointly with others, I have made clear exactly what was done by others and what I have contributed myself;\\
7. Either none of this work has been published before submission, or parts of this work have been published by :\\
\\
Stefan Collier\\
April 2016
}
\tableofcontents
\listoffigures
\listoftables

\mainmatter
%% ----------------------------------------------------------------
%\include{Introduction}
%\include{Conclusions}
\include{chapters/1Project/main}
\include{chapters/2Lit/main}
\include{chapters/3Design/HighLevel}
\include{chapters/3Design/InDepth}
\include{chapters/4Impl/main}

\include{chapters/5Experiments/1/main}
\include{chapters/5Experiments/2/main}
\include{chapters/5Experiments/3/main}
\include{chapters/5Experiments/4/main}

\include{chapters/6Conclusion/main}

\appendix
\include{appendix/AppendixB}
\include{appendix/D/main}
\include{appendix/AppendixC}

\backmatter
\bibliographystyle{ecs}
\bibliography{ECS}
\end{document}
%% ----------------------------------------------------------------


 %% ----------------------------------------------------------------
%% Progress.tex
%% ---------------------------------------------------------------- 
\documentclass{ecsprogress}    % Use the progress Style
\graphicspath{{../figs/}}   % Location of your graphics files
    \usepackage{natbib}            % Use Natbib style for the refs.
\hypersetup{colorlinks=true}   % Set to false for black/white printing
\input{Definitions}            % Include your abbreviations



\usepackage{enumitem}% http://ctan.org/pkg/enumitem
\usepackage{multirow}
\usepackage{float}
\usepackage{amsmath}
\usepackage{multicol}
\usepackage{amssymb}
\usepackage[normalem]{ulem}
\useunder{\uline}{\ul}{}
\usepackage{wrapfig}


\usepackage[table,xcdraw]{xcolor}


%% ----------------------------------------------------------------
\begin{document}
\frontmatter
\title      {Heterogeneous Agent-based Model for Supermarket Competition}
\authors    {\texorpdfstring
             {\href{mailto:sc22g13@ecs.soton.ac.uk}{Stefan J. Collier}}
             {Stefan J. Collier}
            }
\addresses  {\groupname\\\deptname\\\univname}
\date       {\today}
\subject    {}
\keywords   {}
\supervisor {Dr. Maria Polukarov}
\examiner   {Professor Sheng Chen}

\maketitle
\begin{abstract}
This project aim was to model and analyse the effects of competitive pricing behaviors of grocery retailers on the British market. 

This was achieved by creating a multi-agent model, containing retailer and consumer agents. The heterogeneous crowd of retailers employs either a uniform pricing strategy or a ‘local price flexing’ strategy. The actions of these retailers are chosen by predicting the profit of each action, using a perceptron. Following on from the consideration of different economic models, a discrete model was developed so that software agents have a discrete environment to operate within. Within the model, it has been observed how supermarkets with differing behaviors affect a heterogeneous crowd of consumer agents. The model was implemented in Java with Python used to evaluate the results. 

The simulation displays good acceptance with real grocery market behavior, i.e. captures the performance of British retailers thus can be used to determine the impact of changes in their behavior on their competitors and consumers.Furthermore it can be used to provide insight into sustainability of volatile pricing strategies, providing a useful insight in volatility of British supermarket retail industry. 
\end{abstract}
\acknowledgements{
I would like to express my sincere gratitude to Dr Maria Polukarov for her guidance and support which provided me the freedom to take this research in the direction of my interest.\\
\\
I would also like to thank my family and friends for their encouragement and support. To those who quietly listened to my software complaints. To those who worked throughout the nights with me. To those who helped me write what I couldn't say. I cannot thank you enough.
}

\declaration{
I, Stefan Collier, declare that this dissertation and the work presented in it are my own and has been generated by me as the result of my own original research.\\
I confirm that:\\
1. This work was done wholly or mainly while in candidature for a degree at this University;\\
2. Where any part of this dissertation has previously been submitted for any other qualification at this University or any other institution, this has been clearly stated;\\
3. Where I have consulted the published work of others, this is always clearly attributed;\\
4. Where I have quoted from the work of others, the source is always given. With the exception of such quotations, this dissertation is entirely my own work;\\
5. I have acknowledged all main sources of help;\\
6. Where the thesis is based on work done by myself jointly with others, I have made clear exactly what was done by others and what I have contributed myself;\\
7. Either none of this work has been published before submission, or parts of this work have been published by :\\
\\
Stefan Collier\\
April 2016
}
\tableofcontents
\listoffigures
\listoftables

\mainmatter
%% ----------------------------------------------------------------
%\include{Introduction}
%\include{Conclusions}
\include{chapters/1Project/main}
\include{chapters/2Lit/main}
\include{chapters/3Design/HighLevel}
\include{chapters/3Design/InDepth}
\include{chapters/4Impl/main}

\include{chapters/5Experiments/1/main}
\include{chapters/5Experiments/2/main}
\include{chapters/5Experiments/3/main}
\include{chapters/5Experiments/4/main}

\include{chapters/6Conclusion/main}

\appendix
\include{appendix/AppendixB}
\include{appendix/D/main}
\include{appendix/AppendixC}

\backmatter
\bibliographystyle{ecs}
\bibliography{ECS}
\end{document}
%% ----------------------------------------------------------------


\appendix
\include{appendix/AppendixB}
 %% ----------------------------------------------------------------
%% Progress.tex
%% ---------------------------------------------------------------- 
\documentclass{ecsprogress}    % Use the progress Style
\graphicspath{{../figs/}}   % Location of your graphics files
    \usepackage{natbib}            % Use Natbib style for the refs.
\hypersetup{colorlinks=true}   % Set to false for black/white printing
\input{Definitions}            % Include your abbreviations



\usepackage{enumitem}% http://ctan.org/pkg/enumitem
\usepackage{multirow}
\usepackage{float}
\usepackage{amsmath}
\usepackage{multicol}
\usepackage{amssymb}
\usepackage[normalem]{ulem}
\useunder{\uline}{\ul}{}
\usepackage{wrapfig}


\usepackage[table,xcdraw]{xcolor}


%% ----------------------------------------------------------------
\begin{document}
\frontmatter
\title      {Heterogeneous Agent-based Model for Supermarket Competition}
\authors    {\texorpdfstring
             {\href{mailto:sc22g13@ecs.soton.ac.uk}{Stefan J. Collier}}
             {Stefan J. Collier}
            }
\addresses  {\groupname\\\deptname\\\univname}
\date       {\today}
\subject    {}
\keywords   {}
\supervisor {Dr. Maria Polukarov}
\examiner   {Professor Sheng Chen}

\maketitle
\begin{abstract}
This project aim was to model and analyse the effects of competitive pricing behaviors of grocery retailers on the British market. 

This was achieved by creating a multi-agent model, containing retailer and consumer agents. The heterogeneous crowd of retailers employs either a uniform pricing strategy or a ‘local price flexing’ strategy. The actions of these retailers are chosen by predicting the profit of each action, using a perceptron. Following on from the consideration of different economic models, a discrete model was developed so that software agents have a discrete environment to operate within. Within the model, it has been observed how supermarkets with differing behaviors affect a heterogeneous crowd of consumer agents. The model was implemented in Java with Python used to evaluate the results. 

The simulation displays good acceptance with real grocery market behavior, i.e. captures the performance of British retailers thus can be used to determine the impact of changes in their behavior on their competitors and consumers.Furthermore it can be used to provide insight into sustainability of volatile pricing strategies, providing a useful insight in volatility of British supermarket retail industry. 
\end{abstract}
\acknowledgements{
I would like to express my sincere gratitude to Dr Maria Polukarov for her guidance and support which provided me the freedom to take this research in the direction of my interest.\\
\\
I would also like to thank my family and friends for their encouragement and support. To those who quietly listened to my software complaints. To those who worked throughout the nights with me. To those who helped me write what I couldn't say. I cannot thank you enough.
}

\declaration{
I, Stefan Collier, declare that this dissertation and the work presented in it are my own and has been generated by me as the result of my own original research.\\
I confirm that:\\
1. This work was done wholly or mainly while in candidature for a degree at this University;\\
2. Where any part of this dissertation has previously been submitted for any other qualification at this University or any other institution, this has been clearly stated;\\
3. Where I have consulted the published work of others, this is always clearly attributed;\\
4. Where I have quoted from the work of others, the source is always given. With the exception of such quotations, this dissertation is entirely my own work;\\
5. I have acknowledged all main sources of help;\\
6. Where the thesis is based on work done by myself jointly with others, I have made clear exactly what was done by others and what I have contributed myself;\\
7. Either none of this work has been published before submission, or parts of this work have been published by :\\
\\
Stefan Collier\\
April 2016
}
\tableofcontents
\listoffigures
\listoftables

\mainmatter
%% ----------------------------------------------------------------
%\include{Introduction}
%\include{Conclusions}
\include{chapters/1Project/main}
\include{chapters/2Lit/main}
\include{chapters/3Design/HighLevel}
\include{chapters/3Design/InDepth}
\include{chapters/4Impl/main}

\include{chapters/5Experiments/1/main}
\include{chapters/5Experiments/2/main}
\include{chapters/5Experiments/3/main}
\include{chapters/5Experiments/4/main}

\include{chapters/6Conclusion/main}

\appendix
\include{appendix/AppendixB}
\include{appendix/D/main}
\include{appendix/AppendixC}

\backmatter
\bibliographystyle{ecs}
\bibliography{ECS}
\end{document}
%% ----------------------------------------------------------------

\include{appendix/AppendixC}

\backmatter
\bibliographystyle{ecs}
\bibliography{ECS}
\end{document}
%% ----------------------------------------------------------------


 %% ----------------------------------------------------------------
%% Progress.tex
%% ---------------------------------------------------------------- 
\documentclass{ecsprogress}    % Use the progress Style
\graphicspath{{../figs/}}   % Location of your graphics files
    \usepackage{natbib}            % Use Natbib style for the refs.
\hypersetup{colorlinks=true}   % Set to false for black/white printing
\input{Definitions}            % Include your abbreviations



\usepackage{enumitem}% http://ctan.org/pkg/enumitem
\usepackage{multirow}
\usepackage{float}
\usepackage{amsmath}
\usepackage{multicol}
\usepackage{amssymb}
\usepackage[normalem]{ulem}
\useunder{\uline}{\ul}{}
\usepackage{wrapfig}


\usepackage[table,xcdraw]{xcolor}


%% ----------------------------------------------------------------
\begin{document}
\frontmatter
\title      {Heterogeneous Agent-based Model for Supermarket Competition}
\authors    {\texorpdfstring
             {\href{mailto:sc22g13@ecs.soton.ac.uk}{Stefan J. Collier}}
             {Stefan J. Collier}
            }
\addresses  {\groupname\\\deptname\\\univname}
\date       {\today}
\subject    {}
\keywords   {}
\supervisor {Dr. Maria Polukarov}
\examiner   {Professor Sheng Chen}

\maketitle
\begin{abstract}
This project aim was to model and analyse the effects of competitive pricing behaviors of grocery retailers on the British market. 

This was achieved by creating a multi-agent model, containing retailer and consumer agents. The heterogeneous crowd of retailers employs either a uniform pricing strategy or a ‘local price flexing’ strategy. The actions of these retailers are chosen by predicting the profit of each action, using a perceptron. Following on from the consideration of different economic models, a discrete model was developed so that software agents have a discrete environment to operate within. Within the model, it has been observed how supermarkets with differing behaviors affect a heterogeneous crowd of consumer agents. The model was implemented in Java with Python used to evaluate the results. 

The simulation displays good acceptance with real grocery market behavior, i.e. captures the performance of British retailers thus can be used to determine the impact of changes in their behavior on their competitors and consumers.Furthermore it can be used to provide insight into sustainability of volatile pricing strategies, providing a useful insight in volatility of British supermarket retail industry. 
\end{abstract}
\acknowledgements{
I would like to express my sincere gratitude to Dr Maria Polukarov for her guidance and support which provided me the freedom to take this research in the direction of my interest.\\
\\
I would also like to thank my family and friends for their encouragement and support. To those who quietly listened to my software complaints. To those who worked throughout the nights with me. To those who helped me write what I couldn't say. I cannot thank you enough.
}

\declaration{
I, Stefan Collier, declare that this dissertation and the work presented in it are my own and has been generated by me as the result of my own original research.\\
I confirm that:\\
1. This work was done wholly or mainly while in candidature for a degree at this University;\\
2. Where any part of this dissertation has previously been submitted for any other qualification at this University or any other institution, this has been clearly stated;\\
3. Where I have consulted the published work of others, this is always clearly attributed;\\
4. Where I have quoted from the work of others, the source is always given. With the exception of such quotations, this dissertation is entirely my own work;\\
5. I have acknowledged all main sources of help;\\
6. Where the thesis is based on work done by myself jointly with others, I have made clear exactly what was done by others and what I have contributed myself;\\
7. Either none of this work has been published before submission, or parts of this work have been published by :\\
\\
Stefan Collier\\
April 2016
}
\tableofcontents
\listoffigures
\listoftables

\mainmatter
%% ----------------------------------------------------------------
%\include{Introduction}
%\include{Conclusions}
 %% ----------------------------------------------------------------
%% Progress.tex
%% ---------------------------------------------------------------- 
\documentclass{ecsprogress}    % Use the progress Style
\graphicspath{{../figs/}}   % Location of your graphics files
    \usepackage{natbib}            % Use Natbib style for the refs.
\hypersetup{colorlinks=true}   % Set to false for black/white printing
\input{Definitions}            % Include your abbreviations



\usepackage{enumitem}% http://ctan.org/pkg/enumitem
\usepackage{multirow}
\usepackage{float}
\usepackage{amsmath}
\usepackage{multicol}
\usepackage{amssymb}
\usepackage[normalem]{ulem}
\useunder{\uline}{\ul}{}
\usepackage{wrapfig}


\usepackage[table,xcdraw]{xcolor}


%% ----------------------------------------------------------------
\begin{document}
\frontmatter
\title      {Heterogeneous Agent-based Model for Supermarket Competition}
\authors    {\texorpdfstring
             {\href{mailto:sc22g13@ecs.soton.ac.uk}{Stefan J. Collier}}
             {Stefan J. Collier}
            }
\addresses  {\groupname\\\deptname\\\univname}
\date       {\today}
\subject    {}
\keywords   {}
\supervisor {Dr. Maria Polukarov}
\examiner   {Professor Sheng Chen}

\maketitle
\begin{abstract}
This project aim was to model and analyse the effects of competitive pricing behaviors of grocery retailers on the British market. 

This was achieved by creating a multi-agent model, containing retailer and consumer agents. The heterogeneous crowd of retailers employs either a uniform pricing strategy or a ‘local price flexing’ strategy. The actions of these retailers are chosen by predicting the profit of each action, using a perceptron. Following on from the consideration of different economic models, a discrete model was developed so that software agents have a discrete environment to operate within. Within the model, it has been observed how supermarkets with differing behaviors affect a heterogeneous crowd of consumer agents. The model was implemented in Java with Python used to evaluate the results. 

The simulation displays good acceptance with real grocery market behavior, i.e. captures the performance of British retailers thus can be used to determine the impact of changes in their behavior on their competitors and consumers.Furthermore it can be used to provide insight into sustainability of volatile pricing strategies, providing a useful insight in volatility of British supermarket retail industry. 
\end{abstract}
\acknowledgements{
I would like to express my sincere gratitude to Dr Maria Polukarov for her guidance and support which provided me the freedom to take this research in the direction of my interest.\\
\\
I would also like to thank my family and friends for their encouragement and support. To those who quietly listened to my software complaints. To those who worked throughout the nights with me. To those who helped me write what I couldn't say. I cannot thank you enough.
}

\declaration{
I, Stefan Collier, declare that this dissertation and the work presented in it are my own and has been generated by me as the result of my own original research.\\
I confirm that:\\
1. This work was done wholly or mainly while in candidature for a degree at this University;\\
2. Where any part of this dissertation has previously been submitted for any other qualification at this University or any other institution, this has been clearly stated;\\
3. Where I have consulted the published work of others, this is always clearly attributed;\\
4. Where I have quoted from the work of others, the source is always given. With the exception of such quotations, this dissertation is entirely my own work;\\
5. I have acknowledged all main sources of help;\\
6. Where the thesis is based on work done by myself jointly with others, I have made clear exactly what was done by others and what I have contributed myself;\\
7. Either none of this work has been published before submission, or parts of this work have been published by :\\
\\
Stefan Collier\\
April 2016
}
\tableofcontents
\listoffigures
\listoftables

\mainmatter
%% ----------------------------------------------------------------
%\include{Introduction}
%\include{Conclusions}
\include{chapters/1Project/main}
\include{chapters/2Lit/main}
\include{chapters/3Design/HighLevel}
\include{chapters/3Design/InDepth}
\include{chapters/4Impl/main}

\include{chapters/5Experiments/1/main}
\include{chapters/5Experiments/2/main}
\include{chapters/5Experiments/3/main}
\include{chapters/5Experiments/4/main}

\include{chapters/6Conclusion/main}

\appendix
\include{appendix/AppendixB}
\include{appendix/D/main}
\include{appendix/AppendixC}

\backmatter
\bibliographystyle{ecs}
\bibliography{ECS}
\end{document}
%% ----------------------------------------------------------------

 %% ----------------------------------------------------------------
%% Progress.tex
%% ---------------------------------------------------------------- 
\documentclass{ecsprogress}    % Use the progress Style
\graphicspath{{../figs/}}   % Location of your graphics files
    \usepackage{natbib}            % Use Natbib style for the refs.
\hypersetup{colorlinks=true}   % Set to false for black/white printing
\input{Definitions}            % Include your abbreviations



\usepackage{enumitem}% http://ctan.org/pkg/enumitem
\usepackage{multirow}
\usepackage{float}
\usepackage{amsmath}
\usepackage{multicol}
\usepackage{amssymb}
\usepackage[normalem]{ulem}
\useunder{\uline}{\ul}{}
\usepackage{wrapfig}


\usepackage[table,xcdraw]{xcolor}


%% ----------------------------------------------------------------
\begin{document}
\frontmatter
\title      {Heterogeneous Agent-based Model for Supermarket Competition}
\authors    {\texorpdfstring
             {\href{mailto:sc22g13@ecs.soton.ac.uk}{Stefan J. Collier}}
             {Stefan J. Collier}
            }
\addresses  {\groupname\\\deptname\\\univname}
\date       {\today}
\subject    {}
\keywords   {}
\supervisor {Dr. Maria Polukarov}
\examiner   {Professor Sheng Chen}

\maketitle
\begin{abstract}
This project aim was to model and analyse the effects of competitive pricing behaviors of grocery retailers on the British market. 

This was achieved by creating a multi-agent model, containing retailer and consumer agents. The heterogeneous crowd of retailers employs either a uniform pricing strategy or a ‘local price flexing’ strategy. The actions of these retailers are chosen by predicting the profit of each action, using a perceptron. Following on from the consideration of different economic models, a discrete model was developed so that software agents have a discrete environment to operate within. Within the model, it has been observed how supermarkets with differing behaviors affect a heterogeneous crowd of consumer agents. The model was implemented in Java with Python used to evaluate the results. 

The simulation displays good acceptance with real grocery market behavior, i.e. captures the performance of British retailers thus can be used to determine the impact of changes in their behavior on their competitors and consumers.Furthermore it can be used to provide insight into sustainability of volatile pricing strategies, providing a useful insight in volatility of British supermarket retail industry. 
\end{abstract}
\acknowledgements{
I would like to express my sincere gratitude to Dr Maria Polukarov for her guidance and support which provided me the freedom to take this research in the direction of my interest.\\
\\
I would also like to thank my family and friends for their encouragement and support. To those who quietly listened to my software complaints. To those who worked throughout the nights with me. To those who helped me write what I couldn't say. I cannot thank you enough.
}

\declaration{
I, Stefan Collier, declare that this dissertation and the work presented in it are my own and has been generated by me as the result of my own original research.\\
I confirm that:\\
1. This work was done wholly or mainly while in candidature for a degree at this University;\\
2. Where any part of this dissertation has previously been submitted for any other qualification at this University or any other institution, this has been clearly stated;\\
3. Where I have consulted the published work of others, this is always clearly attributed;\\
4. Where I have quoted from the work of others, the source is always given. With the exception of such quotations, this dissertation is entirely my own work;\\
5. I have acknowledged all main sources of help;\\
6. Where the thesis is based on work done by myself jointly with others, I have made clear exactly what was done by others and what I have contributed myself;\\
7. Either none of this work has been published before submission, or parts of this work have been published by :\\
\\
Stefan Collier\\
April 2016
}
\tableofcontents
\listoffigures
\listoftables

\mainmatter
%% ----------------------------------------------------------------
%\include{Introduction}
%\include{Conclusions}
\include{chapters/1Project/main}
\include{chapters/2Lit/main}
\include{chapters/3Design/HighLevel}
\include{chapters/3Design/InDepth}
\include{chapters/4Impl/main}

\include{chapters/5Experiments/1/main}
\include{chapters/5Experiments/2/main}
\include{chapters/5Experiments/3/main}
\include{chapters/5Experiments/4/main}

\include{chapters/6Conclusion/main}

\appendix
\include{appendix/AppendixB}
\include{appendix/D/main}
\include{appendix/AppendixC}

\backmatter
\bibliographystyle{ecs}
\bibliography{ECS}
\end{document}
%% ----------------------------------------------------------------

\include{chapters/3Design/HighLevel}
\include{chapters/3Design/InDepth}
 %% ----------------------------------------------------------------
%% Progress.tex
%% ---------------------------------------------------------------- 
\documentclass{ecsprogress}    % Use the progress Style
\graphicspath{{../figs/}}   % Location of your graphics files
    \usepackage{natbib}            % Use Natbib style for the refs.
\hypersetup{colorlinks=true}   % Set to false for black/white printing
\input{Definitions}            % Include your abbreviations



\usepackage{enumitem}% http://ctan.org/pkg/enumitem
\usepackage{multirow}
\usepackage{float}
\usepackage{amsmath}
\usepackage{multicol}
\usepackage{amssymb}
\usepackage[normalem]{ulem}
\useunder{\uline}{\ul}{}
\usepackage{wrapfig}


\usepackage[table,xcdraw]{xcolor}


%% ----------------------------------------------------------------
\begin{document}
\frontmatter
\title      {Heterogeneous Agent-based Model for Supermarket Competition}
\authors    {\texorpdfstring
             {\href{mailto:sc22g13@ecs.soton.ac.uk}{Stefan J. Collier}}
             {Stefan J. Collier}
            }
\addresses  {\groupname\\\deptname\\\univname}
\date       {\today}
\subject    {}
\keywords   {}
\supervisor {Dr. Maria Polukarov}
\examiner   {Professor Sheng Chen}

\maketitle
\begin{abstract}
This project aim was to model and analyse the effects of competitive pricing behaviors of grocery retailers on the British market. 

This was achieved by creating a multi-agent model, containing retailer and consumer agents. The heterogeneous crowd of retailers employs either a uniform pricing strategy or a ‘local price flexing’ strategy. The actions of these retailers are chosen by predicting the profit of each action, using a perceptron. Following on from the consideration of different economic models, a discrete model was developed so that software agents have a discrete environment to operate within. Within the model, it has been observed how supermarkets with differing behaviors affect a heterogeneous crowd of consumer agents. The model was implemented in Java with Python used to evaluate the results. 

The simulation displays good acceptance with real grocery market behavior, i.e. captures the performance of British retailers thus can be used to determine the impact of changes in their behavior on their competitors and consumers.Furthermore it can be used to provide insight into sustainability of volatile pricing strategies, providing a useful insight in volatility of British supermarket retail industry. 
\end{abstract}
\acknowledgements{
I would like to express my sincere gratitude to Dr Maria Polukarov for her guidance and support which provided me the freedom to take this research in the direction of my interest.\\
\\
I would also like to thank my family and friends for their encouragement and support. To those who quietly listened to my software complaints. To those who worked throughout the nights with me. To those who helped me write what I couldn't say. I cannot thank you enough.
}

\declaration{
I, Stefan Collier, declare that this dissertation and the work presented in it are my own and has been generated by me as the result of my own original research.\\
I confirm that:\\
1. This work was done wholly or mainly while in candidature for a degree at this University;\\
2. Where any part of this dissertation has previously been submitted for any other qualification at this University or any other institution, this has been clearly stated;\\
3. Where I have consulted the published work of others, this is always clearly attributed;\\
4. Where I have quoted from the work of others, the source is always given. With the exception of such quotations, this dissertation is entirely my own work;\\
5. I have acknowledged all main sources of help;\\
6. Where the thesis is based on work done by myself jointly with others, I have made clear exactly what was done by others and what I have contributed myself;\\
7. Either none of this work has been published before submission, or parts of this work have been published by :\\
\\
Stefan Collier\\
April 2016
}
\tableofcontents
\listoffigures
\listoftables

\mainmatter
%% ----------------------------------------------------------------
%\include{Introduction}
%\include{Conclusions}
\include{chapters/1Project/main}
\include{chapters/2Lit/main}
\include{chapters/3Design/HighLevel}
\include{chapters/3Design/InDepth}
\include{chapters/4Impl/main}

\include{chapters/5Experiments/1/main}
\include{chapters/5Experiments/2/main}
\include{chapters/5Experiments/3/main}
\include{chapters/5Experiments/4/main}

\include{chapters/6Conclusion/main}

\appendix
\include{appendix/AppendixB}
\include{appendix/D/main}
\include{appendix/AppendixC}

\backmatter
\bibliographystyle{ecs}
\bibliography{ECS}
\end{document}
%% ----------------------------------------------------------------


 %% ----------------------------------------------------------------
%% Progress.tex
%% ---------------------------------------------------------------- 
\documentclass{ecsprogress}    % Use the progress Style
\graphicspath{{../figs/}}   % Location of your graphics files
    \usepackage{natbib}            % Use Natbib style for the refs.
\hypersetup{colorlinks=true}   % Set to false for black/white printing
\input{Definitions}            % Include your abbreviations



\usepackage{enumitem}% http://ctan.org/pkg/enumitem
\usepackage{multirow}
\usepackage{float}
\usepackage{amsmath}
\usepackage{multicol}
\usepackage{amssymb}
\usepackage[normalem]{ulem}
\useunder{\uline}{\ul}{}
\usepackage{wrapfig}


\usepackage[table,xcdraw]{xcolor}


%% ----------------------------------------------------------------
\begin{document}
\frontmatter
\title      {Heterogeneous Agent-based Model for Supermarket Competition}
\authors    {\texorpdfstring
             {\href{mailto:sc22g13@ecs.soton.ac.uk}{Stefan J. Collier}}
             {Stefan J. Collier}
            }
\addresses  {\groupname\\\deptname\\\univname}
\date       {\today}
\subject    {}
\keywords   {}
\supervisor {Dr. Maria Polukarov}
\examiner   {Professor Sheng Chen}

\maketitle
\begin{abstract}
This project aim was to model and analyse the effects of competitive pricing behaviors of grocery retailers on the British market. 

This was achieved by creating a multi-agent model, containing retailer and consumer agents. The heterogeneous crowd of retailers employs either a uniform pricing strategy or a ‘local price flexing’ strategy. The actions of these retailers are chosen by predicting the profit of each action, using a perceptron. Following on from the consideration of different economic models, a discrete model was developed so that software agents have a discrete environment to operate within. Within the model, it has been observed how supermarkets with differing behaviors affect a heterogeneous crowd of consumer agents. The model was implemented in Java with Python used to evaluate the results. 

The simulation displays good acceptance with real grocery market behavior, i.e. captures the performance of British retailers thus can be used to determine the impact of changes in their behavior on their competitors and consumers.Furthermore it can be used to provide insight into sustainability of volatile pricing strategies, providing a useful insight in volatility of British supermarket retail industry. 
\end{abstract}
\acknowledgements{
I would like to express my sincere gratitude to Dr Maria Polukarov for her guidance and support which provided me the freedom to take this research in the direction of my interest.\\
\\
I would also like to thank my family and friends for their encouragement and support. To those who quietly listened to my software complaints. To those who worked throughout the nights with me. To those who helped me write what I couldn't say. I cannot thank you enough.
}

\declaration{
I, Stefan Collier, declare that this dissertation and the work presented in it are my own and has been generated by me as the result of my own original research.\\
I confirm that:\\
1. This work was done wholly or mainly while in candidature for a degree at this University;\\
2. Where any part of this dissertation has previously been submitted for any other qualification at this University or any other institution, this has been clearly stated;\\
3. Where I have consulted the published work of others, this is always clearly attributed;\\
4. Where I have quoted from the work of others, the source is always given. With the exception of such quotations, this dissertation is entirely my own work;\\
5. I have acknowledged all main sources of help;\\
6. Where the thesis is based on work done by myself jointly with others, I have made clear exactly what was done by others and what I have contributed myself;\\
7. Either none of this work has been published before submission, or parts of this work have been published by :\\
\\
Stefan Collier\\
April 2016
}
\tableofcontents
\listoffigures
\listoftables

\mainmatter
%% ----------------------------------------------------------------
%\include{Introduction}
%\include{Conclusions}
\include{chapters/1Project/main}
\include{chapters/2Lit/main}
\include{chapters/3Design/HighLevel}
\include{chapters/3Design/InDepth}
\include{chapters/4Impl/main}

\include{chapters/5Experiments/1/main}
\include{chapters/5Experiments/2/main}
\include{chapters/5Experiments/3/main}
\include{chapters/5Experiments/4/main}

\include{chapters/6Conclusion/main}

\appendix
\include{appendix/AppendixB}
\include{appendix/D/main}
\include{appendix/AppendixC}

\backmatter
\bibliographystyle{ecs}
\bibliography{ECS}
\end{document}
%% ----------------------------------------------------------------

 %% ----------------------------------------------------------------
%% Progress.tex
%% ---------------------------------------------------------------- 
\documentclass{ecsprogress}    % Use the progress Style
\graphicspath{{../figs/}}   % Location of your graphics files
    \usepackage{natbib}            % Use Natbib style for the refs.
\hypersetup{colorlinks=true}   % Set to false for black/white printing
\input{Definitions}            % Include your abbreviations



\usepackage{enumitem}% http://ctan.org/pkg/enumitem
\usepackage{multirow}
\usepackage{float}
\usepackage{amsmath}
\usepackage{multicol}
\usepackage{amssymb}
\usepackage[normalem]{ulem}
\useunder{\uline}{\ul}{}
\usepackage{wrapfig}


\usepackage[table,xcdraw]{xcolor}


%% ----------------------------------------------------------------
\begin{document}
\frontmatter
\title      {Heterogeneous Agent-based Model for Supermarket Competition}
\authors    {\texorpdfstring
             {\href{mailto:sc22g13@ecs.soton.ac.uk}{Stefan J. Collier}}
             {Stefan J. Collier}
            }
\addresses  {\groupname\\\deptname\\\univname}
\date       {\today}
\subject    {}
\keywords   {}
\supervisor {Dr. Maria Polukarov}
\examiner   {Professor Sheng Chen}

\maketitle
\begin{abstract}
This project aim was to model and analyse the effects of competitive pricing behaviors of grocery retailers on the British market. 

This was achieved by creating a multi-agent model, containing retailer and consumer agents. The heterogeneous crowd of retailers employs either a uniform pricing strategy or a ‘local price flexing’ strategy. The actions of these retailers are chosen by predicting the profit of each action, using a perceptron. Following on from the consideration of different economic models, a discrete model was developed so that software agents have a discrete environment to operate within. Within the model, it has been observed how supermarkets with differing behaviors affect a heterogeneous crowd of consumer agents. The model was implemented in Java with Python used to evaluate the results. 

The simulation displays good acceptance with real grocery market behavior, i.e. captures the performance of British retailers thus can be used to determine the impact of changes in their behavior on their competitors and consumers.Furthermore it can be used to provide insight into sustainability of volatile pricing strategies, providing a useful insight in volatility of British supermarket retail industry. 
\end{abstract}
\acknowledgements{
I would like to express my sincere gratitude to Dr Maria Polukarov for her guidance and support which provided me the freedom to take this research in the direction of my interest.\\
\\
I would also like to thank my family and friends for their encouragement and support. To those who quietly listened to my software complaints. To those who worked throughout the nights with me. To those who helped me write what I couldn't say. I cannot thank you enough.
}

\declaration{
I, Stefan Collier, declare that this dissertation and the work presented in it are my own and has been generated by me as the result of my own original research.\\
I confirm that:\\
1. This work was done wholly or mainly while in candidature for a degree at this University;\\
2. Where any part of this dissertation has previously been submitted for any other qualification at this University or any other institution, this has been clearly stated;\\
3. Where I have consulted the published work of others, this is always clearly attributed;\\
4. Where I have quoted from the work of others, the source is always given. With the exception of such quotations, this dissertation is entirely my own work;\\
5. I have acknowledged all main sources of help;\\
6. Where the thesis is based on work done by myself jointly with others, I have made clear exactly what was done by others and what I have contributed myself;\\
7. Either none of this work has been published before submission, or parts of this work have been published by :\\
\\
Stefan Collier\\
April 2016
}
\tableofcontents
\listoffigures
\listoftables

\mainmatter
%% ----------------------------------------------------------------
%\include{Introduction}
%\include{Conclusions}
\include{chapters/1Project/main}
\include{chapters/2Lit/main}
\include{chapters/3Design/HighLevel}
\include{chapters/3Design/InDepth}
\include{chapters/4Impl/main}

\include{chapters/5Experiments/1/main}
\include{chapters/5Experiments/2/main}
\include{chapters/5Experiments/3/main}
\include{chapters/5Experiments/4/main}

\include{chapters/6Conclusion/main}

\appendix
\include{appendix/AppendixB}
\include{appendix/D/main}
\include{appendix/AppendixC}

\backmatter
\bibliographystyle{ecs}
\bibliography{ECS}
\end{document}
%% ----------------------------------------------------------------

 %% ----------------------------------------------------------------
%% Progress.tex
%% ---------------------------------------------------------------- 
\documentclass{ecsprogress}    % Use the progress Style
\graphicspath{{../figs/}}   % Location of your graphics files
    \usepackage{natbib}            % Use Natbib style for the refs.
\hypersetup{colorlinks=true}   % Set to false for black/white printing
\input{Definitions}            % Include your abbreviations



\usepackage{enumitem}% http://ctan.org/pkg/enumitem
\usepackage{multirow}
\usepackage{float}
\usepackage{amsmath}
\usepackage{multicol}
\usepackage{amssymb}
\usepackage[normalem]{ulem}
\useunder{\uline}{\ul}{}
\usepackage{wrapfig}


\usepackage[table,xcdraw]{xcolor}


%% ----------------------------------------------------------------
\begin{document}
\frontmatter
\title      {Heterogeneous Agent-based Model for Supermarket Competition}
\authors    {\texorpdfstring
             {\href{mailto:sc22g13@ecs.soton.ac.uk}{Stefan J. Collier}}
             {Stefan J. Collier}
            }
\addresses  {\groupname\\\deptname\\\univname}
\date       {\today}
\subject    {}
\keywords   {}
\supervisor {Dr. Maria Polukarov}
\examiner   {Professor Sheng Chen}

\maketitle
\begin{abstract}
This project aim was to model and analyse the effects of competitive pricing behaviors of grocery retailers on the British market. 

This was achieved by creating a multi-agent model, containing retailer and consumer agents. The heterogeneous crowd of retailers employs either a uniform pricing strategy or a ‘local price flexing’ strategy. The actions of these retailers are chosen by predicting the profit of each action, using a perceptron. Following on from the consideration of different economic models, a discrete model was developed so that software agents have a discrete environment to operate within. Within the model, it has been observed how supermarkets with differing behaviors affect a heterogeneous crowd of consumer agents. The model was implemented in Java with Python used to evaluate the results. 

The simulation displays good acceptance with real grocery market behavior, i.e. captures the performance of British retailers thus can be used to determine the impact of changes in their behavior on their competitors and consumers.Furthermore it can be used to provide insight into sustainability of volatile pricing strategies, providing a useful insight in volatility of British supermarket retail industry. 
\end{abstract}
\acknowledgements{
I would like to express my sincere gratitude to Dr Maria Polukarov for her guidance and support which provided me the freedom to take this research in the direction of my interest.\\
\\
I would also like to thank my family and friends for their encouragement and support. To those who quietly listened to my software complaints. To those who worked throughout the nights with me. To those who helped me write what I couldn't say. I cannot thank you enough.
}

\declaration{
I, Stefan Collier, declare that this dissertation and the work presented in it are my own and has been generated by me as the result of my own original research.\\
I confirm that:\\
1. This work was done wholly or mainly while in candidature for a degree at this University;\\
2. Where any part of this dissertation has previously been submitted for any other qualification at this University or any other institution, this has been clearly stated;\\
3. Where I have consulted the published work of others, this is always clearly attributed;\\
4. Where I have quoted from the work of others, the source is always given. With the exception of such quotations, this dissertation is entirely my own work;\\
5. I have acknowledged all main sources of help;\\
6. Where the thesis is based on work done by myself jointly with others, I have made clear exactly what was done by others and what I have contributed myself;\\
7. Either none of this work has been published before submission, or parts of this work have been published by :\\
\\
Stefan Collier\\
April 2016
}
\tableofcontents
\listoffigures
\listoftables

\mainmatter
%% ----------------------------------------------------------------
%\include{Introduction}
%\include{Conclusions}
\include{chapters/1Project/main}
\include{chapters/2Lit/main}
\include{chapters/3Design/HighLevel}
\include{chapters/3Design/InDepth}
\include{chapters/4Impl/main}

\include{chapters/5Experiments/1/main}
\include{chapters/5Experiments/2/main}
\include{chapters/5Experiments/3/main}
\include{chapters/5Experiments/4/main}

\include{chapters/6Conclusion/main}

\appendix
\include{appendix/AppendixB}
\include{appendix/D/main}
\include{appendix/AppendixC}

\backmatter
\bibliographystyle{ecs}
\bibliography{ECS}
\end{document}
%% ----------------------------------------------------------------

 %% ----------------------------------------------------------------
%% Progress.tex
%% ---------------------------------------------------------------- 
\documentclass{ecsprogress}    % Use the progress Style
\graphicspath{{../figs/}}   % Location of your graphics files
    \usepackage{natbib}            % Use Natbib style for the refs.
\hypersetup{colorlinks=true}   % Set to false for black/white printing
\input{Definitions}            % Include your abbreviations



\usepackage{enumitem}% http://ctan.org/pkg/enumitem
\usepackage{multirow}
\usepackage{float}
\usepackage{amsmath}
\usepackage{multicol}
\usepackage{amssymb}
\usepackage[normalem]{ulem}
\useunder{\uline}{\ul}{}
\usepackage{wrapfig}


\usepackage[table,xcdraw]{xcolor}


%% ----------------------------------------------------------------
\begin{document}
\frontmatter
\title      {Heterogeneous Agent-based Model for Supermarket Competition}
\authors    {\texorpdfstring
             {\href{mailto:sc22g13@ecs.soton.ac.uk}{Stefan J. Collier}}
             {Stefan J. Collier}
            }
\addresses  {\groupname\\\deptname\\\univname}
\date       {\today}
\subject    {}
\keywords   {}
\supervisor {Dr. Maria Polukarov}
\examiner   {Professor Sheng Chen}

\maketitle
\begin{abstract}
This project aim was to model and analyse the effects of competitive pricing behaviors of grocery retailers on the British market. 

This was achieved by creating a multi-agent model, containing retailer and consumer agents. The heterogeneous crowd of retailers employs either a uniform pricing strategy or a ‘local price flexing’ strategy. The actions of these retailers are chosen by predicting the profit of each action, using a perceptron. Following on from the consideration of different economic models, a discrete model was developed so that software agents have a discrete environment to operate within. Within the model, it has been observed how supermarkets with differing behaviors affect a heterogeneous crowd of consumer agents. The model was implemented in Java with Python used to evaluate the results. 

The simulation displays good acceptance with real grocery market behavior, i.e. captures the performance of British retailers thus can be used to determine the impact of changes in their behavior on their competitors and consumers.Furthermore it can be used to provide insight into sustainability of volatile pricing strategies, providing a useful insight in volatility of British supermarket retail industry. 
\end{abstract}
\acknowledgements{
I would like to express my sincere gratitude to Dr Maria Polukarov for her guidance and support which provided me the freedom to take this research in the direction of my interest.\\
\\
I would also like to thank my family and friends for their encouragement and support. To those who quietly listened to my software complaints. To those who worked throughout the nights with me. To those who helped me write what I couldn't say. I cannot thank you enough.
}

\declaration{
I, Stefan Collier, declare that this dissertation and the work presented in it are my own and has been generated by me as the result of my own original research.\\
I confirm that:\\
1. This work was done wholly or mainly while in candidature for a degree at this University;\\
2. Where any part of this dissertation has previously been submitted for any other qualification at this University or any other institution, this has been clearly stated;\\
3. Where I have consulted the published work of others, this is always clearly attributed;\\
4. Where I have quoted from the work of others, the source is always given. With the exception of such quotations, this dissertation is entirely my own work;\\
5. I have acknowledged all main sources of help;\\
6. Where the thesis is based on work done by myself jointly with others, I have made clear exactly what was done by others and what I have contributed myself;\\
7. Either none of this work has been published before submission, or parts of this work have been published by :\\
\\
Stefan Collier\\
April 2016
}
\tableofcontents
\listoffigures
\listoftables

\mainmatter
%% ----------------------------------------------------------------
%\include{Introduction}
%\include{Conclusions}
\include{chapters/1Project/main}
\include{chapters/2Lit/main}
\include{chapters/3Design/HighLevel}
\include{chapters/3Design/InDepth}
\include{chapters/4Impl/main}

\include{chapters/5Experiments/1/main}
\include{chapters/5Experiments/2/main}
\include{chapters/5Experiments/3/main}
\include{chapters/5Experiments/4/main}

\include{chapters/6Conclusion/main}

\appendix
\include{appendix/AppendixB}
\include{appendix/D/main}
\include{appendix/AppendixC}

\backmatter
\bibliographystyle{ecs}
\bibliography{ECS}
\end{document}
%% ----------------------------------------------------------------


 %% ----------------------------------------------------------------
%% Progress.tex
%% ---------------------------------------------------------------- 
\documentclass{ecsprogress}    % Use the progress Style
\graphicspath{{../figs/}}   % Location of your graphics files
    \usepackage{natbib}            % Use Natbib style for the refs.
\hypersetup{colorlinks=true}   % Set to false for black/white printing
\input{Definitions}            % Include your abbreviations



\usepackage{enumitem}% http://ctan.org/pkg/enumitem
\usepackage{multirow}
\usepackage{float}
\usepackage{amsmath}
\usepackage{multicol}
\usepackage{amssymb}
\usepackage[normalem]{ulem}
\useunder{\uline}{\ul}{}
\usepackage{wrapfig}


\usepackage[table,xcdraw]{xcolor}


%% ----------------------------------------------------------------
\begin{document}
\frontmatter
\title      {Heterogeneous Agent-based Model for Supermarket Competition}
\authors    {\texorpdfstring
             {\href{mailto:sc22g13@ecs.soton.ac.uk}{Stefan J. Collier}}
             {Stefan J. Collier}
            }
\addresses  {\groupname\\\deptname\\\univname}
\date       {\today}
\subject    {}
\keywords   {}
\supervisor {Dr. Maria Polukarov}
\examiner   {Professor Sheng Chen}

\maketitle
\begin{abstract}
This project aim was to model and analyse the effects of competitive pricing behaviors of grocery retailers on the British market. 

This was achieved by creating a multi-agent model, containing retailer and consumer agents. The heterogeneous crowd of retailers employs either a uniform pricing strategy or a ‘local price flexing’ strategy. The actions of these retailers are chosen by predicting the profit of each action, using a perceptron. Following on from the consideration of different economic models, a discrete model was developed so that software agents have a discrete environment to operate within. Within the model, it has been observed how supermarkets with differing behaviors affect a heterogeneous crowd of consumer agents. The model was implemented in Java with Python used to evaluate the results. 

The simulation displays good acceptance with real grocery market behavior, i.e. captures the performance of British retailers thus can be used to determine the impact of changes in their behavior on their competitors and consumers.Furthermore it can be used to provide insight into sustainability of volatile pricing strategies, providing a useful insight in volatility of British supermarket retail industry. 
\end{abstract}
\acknowledgements{
I would like to express my sincere gratitude to Dr Maria Polukarov for her guidance and support which provided me the freedom to take this research in the direction of my interest.\\
\\
I would also like to thank my family and friends for their encouragement and support. To those who quietly listened to my software complaints. To those who worked throughout the nights with me. To those who helped me write what I couldn't say. I cannot thank you enough.
}

\declaration{
I, Stefan Collier, declare that this dissertation and the work presented in it are my own and has been generated by me as the result of my own original research.\\
I confirm that:\\
1. This work was done wholly or mainly while in candidature for a degree at this University;\\
2. Where any part of this dissertation has previously been submitted for any other qualification at this University or any other institution, this has been clearly stated;\\
3. Where I have consulted the published work of others, this is always clearly attributed;\\
4. Where I have quoted from the work of others, the source is always given. With the exception of such quotations, this dissertation is entirely my own work;\\
5. I have acknowledged all main sources of help;\\
6. Where the thesis is based on work done by myself jointly with others, I have made clear exactly what was done by others and what I have contributed myself;\\
7. Either none of this work has been published before submission, or parts of this work have been published by :\\
\\
Stefan Collier\\
April 2016
}
\tableofcontents
\listoffigures
\listoftables

\mainmatter
%% ----------------------------------------------------------------
%\include{Introduction}
%\include{Conclusions}
\include{chapters/1Project/main}
\include{chapters/2Lit/main}
\include{chapters/3Design/HighLevel}
\include{chapters/3Design/InDepth}
\include{chapters/4Impl/main}

\include{chapters/5Experiments/1/main}
\include{chapters/5Experiments/2/main}
\include{chapters/5Experiments/3/main}
\include{chapters/5Experiments/4/main}

\include{chapters/6Conclusion/main}

\appendix
\include{appendix/AppendixB}
\include{appendix/D/main}
\include{appendix/AppendixC}

\backmatter
\bibliographystyle{ecs}
\bibliography{ECS}
\end{document}
%% ----------------------------------------------------------------


\appendix
\include{appendix/AppendixB}
 %% ----------------------------------------------------------------
%% Progress.tex
%% ---------------------------------------------------------------- 
\documentclass{ecsprogress}    % Use the progress Style
\graphicspath{{../figs/}}   % Location of your graphics files
    \usepackage{natbib}            % Use Natbib style for the refs.
\hypersetup{colorlinks=true}   % Set to false for black/white printing
\input{Definitions}            % Include your abbreviations



\usepackage{enumitem}% http://ctan.org/pkg/enumitem
\usepackage{multirow}
\usepackage{float}
\usepackage{amsmath}
\usepackage{multicol}
\usepackage{amssymb}
\usepackage[normalem]{ulem}
\useunder{\uline}{\ul}{}
\usepackage{wrapfig}


\usepackage[table,xcdraw]{xcolor}


%% ----------------------------------------------------------------
\begin{document}
\frontmatter
\title      {Heterogeneous Agent-based Model for Supermarket Competition}
\authors    {\texorpdfstring
             {\href{mailto:sc22g13@ecs.soton.ac.uk}{Stefan J. Collier}}
             {Stefan J. Collier}
            }
\addresses  {\groupname\\\deptname\\\univname}
\date       {\today}
\subject    {}
\keywords   {}
\supervisor {Dr. Maria Polukarov}
\examiner   {Professor Sheng Chen}

\maketitle
\begin{abstract}
This project aim was to model and analyse the effects of competitive pricing behaviors of grocery retailers on the British market. 

This was achieved by creating a multi-agent model, containing retailer and consumer agents. The heterogeneous crowd of retailers employs either a uniform pricing strategy or a ‘local price flexing’ strategy. The actions of these retailers are chosen by predicting the profit of each action, using a perceptron. Following on from the consideration of different economic models, a discrete model was developed so that software agents have a discrete environment to operate within. Within the model, it has been observed how supermarkets with differing behaviors affect a heterogeneous crowd of consumer agents. The model was implemented in Java with Python used to evaluate the results. 

The simulation displays good acceptance with real grocery market behavior, i.e. captures the performance of British retailers thus can be used to determine the impact of changes in their behavior on their competitors and consumers.Furthermore it can be used to provide insight into sustainability of volatile pricing strategies, providing a useful insight in volatility of British supermarket retail industry. 
\end{abstract}
\acknowledgements{
I would like to express my sincere gratitude to Dr Maria Polukarov for her guidance and support which provided me the freedom to take this research in the direction of my interest.\\
\\
I would also like to thank my family and friends for their encouragement and support. To those who quietly listened to my software complaints. To those who worked throughout the nights with me. To those who helped me write what I couldn't say. I cannot thank you enough.
}

\declaration{
I, Stefan Collier, declare that this dissertation and the work presented in it are my own and has been generated by me as the result of my own original research.\\
I confirm that:\\
1. This work was done wholly or mainly while in candidature for a degree at this University;\\
2. Where any part of this dissertation has previously been submitted for any other qualification at this University or any other institution, this has been clearly stated;\\
3. Where I have consulted the published work of others, this is always clearly attributed;\\
4. Where I have quoted from the work of others, the source is always given. With the exception of such quotations, this dissertation is entirely my own work;\\
5. I have acknowledged all main sources of help;\\
6. Where the thesis is based on work done by myself jointly with others, I have made clear exactly what was done by others and what I have contributed myself;\\
7. Either none of this work has been published before submission, or parts of this work have been published by :\\
\\
Stefan Collier\\
April 2016
}
\tableofcontents
\listoffigures
\listoftables

\mainmatter
%% ----------------------------------------------------------------
%\include{Introduction}
%\include{Conclusions}
\include{chapters/1Project/main}
\include{chapters/2Lit/main}
\include{chapters/3Design/HighLevel}
\include{chapters/3Design/InDepth}
\include{chapters/4Impl/main}

\include{chapters/5Experiments/1/main}
\include{chapters/5Experiments/2/main}
\include{chapters/5Experiments/3/main}
\include{chapters/5Experiments/4/main}

\include{chapters/6Conclusion/main}

\appendix
\include{appendix/AppendixB}
\include{appendix/D/main}
\include{appendix/AppendixC}

\backmatter
\bibliographystyle{ecs}
\bibliography{ECS}
\end{document}
%% ----------------------------------------------------------------

\include{appendix/AppendixC}

\backmatter
\bibliographystyle{ecs}
\bibliography{ECS}
\end{document}
%% ----------------------------------------------------------------


\appendix
\include{appendix/AppendixB}
 %% ----------------------------------------------------------------
%% Progress.tex
%% ---------------------------------------------------------------- 
\documentclass{ecsprogress}    % Use the progress Style
\graphicspath{{../figs/}}   % Location of your graphics files
    \usepackage{natbib}            % Use Natbib style for the refs.
\hypersetup{colorlinks=true}   % Set to false for black/white printing
\input{Definitions}            % Include your abbreviations



\usepackage{enumitem}% http://ctan.org/pkg/enumitem
\usepackage{multirow}
\usepackage{float}
\usepackage{amsmath}
\usepackage{multicol}
\usepackage{amssymb}
\usepackage[normalem]{ulem}
\useunder{\uline}{\ul}{}
\usepackage{wrapfig}


\usepackage[table,xcdraw]{xcolor}


%% ----------------------------------------------------------------
\begin{document}
\frontmatter
\title      {Heterogeneous Agent-based Model for Supermarket Competition}
\authors    {\texorpdfstring
             {\href{mailto:sc22g13@ecs.soton.ac.uk}{Stefan J. Collier}}
             {Stefan J. Collier}
            }
\addresses  {\groupname\\\deptname\\\univname}
\date       {\today}
\subject    {}
\keywords   {}
\supervisor {Dr. Maria Polukarov}
\examiner   {Professor Sheng Chen}

\maketitle
\begin{abstract}
This project aim was to model and analyse the effects of competitive pricing behaviors of grocery retailers on the British market. 

This was achieved by creating a multi-agent model, containing retailer and consumer agents. The heterogeneous crowd of retailers employs either a uniform pricing strategy or a ‘local price flexing’ strategy. The actions of these retailers are chosen by predicting the profit of each action, using a perceptron. Following on from the consideration of different economic models, a discrete model was developed so that software agents have a discrete environment to operate within. Within the model, it has been observed how supermarkets with differing behaviors affect a heterogeneous crowd of consumer agents. The model was implemented in Java with Python used to evaluate the results. 

The simulation displays good acceptance with real grocery market behavior, i.e. captures the performance of British retailers thus can be used to determine the impact of changes in their behavior on their competitors and consumers.Furthermore it can be used to provide insight into sustainability of volatile pricing strategies, providing a useful insight in volatility of British supermarket retail industry. 
\end{abstract}
\acknowledgements{
I would like to express my sincere gratitude to Dr Maria Polukarov for her guidance and support which provided me the freedom to take this research in the direction of my interest.\\
\\
I would also like to thank my family and friends for their encouragement and support. To those who quietly listened to my software complaints. To those who worked throughout the nights with me. To those who helped me write what I couldn't say. I cannot thank you enough.
}

\declaration{
I, Stefan Collier, declare that this dissertation and the work presented in it are my own and has been generated by me as the result of my own original research.\\
I confirm that:\\
1. This work was done wholly or mainly while in candidature for a degree at this University;\\
2. Where any part of this dissertation has previously been submitted for any other qualification at this University or any other institution, this has been clearly stated;\\
3. Where I have consulted the published work of others, this is always clearly attributed;\\
4. Where I have quoted from the work of others, the source is always given. With the exception of such quotations, this dissertation is entirely my own work;\\
5. I have acknowledged all main sources of help;\\
6. Where the thesis is based on work done by myself jointly with others, I have made clear exactly what was done by others and what I have contributed myself;\\
7. Either none of this work has been published before submission, or parts of this work have been published by :\\
\\
Stefan Collier\\
April 2016
}
\tableofcontents
\listoffigures
\listoftables

\mainmatter
%% ----------------------------------------------------------------
%\include{Introduction}
%\include{Conclusions}
 %% ----------------------------------------------------------------
%% Progress.tex
%% ---------------------------------------------------------------- 
\documentclass{ecsprogress}    % Use the progress Style
\graphicspath{{../figs/}}   % Location of your graphics files
    \usepackage{natbib}            % Use Natbib style for the refs.
\hypersetup{colorlinks=true}   % Set to false for black/white printing
\input{Definitions}            % Include your abbreviations



\usepackage{enumitem}% http://ctan.org/pkg/enumitem
\usepackage{multirow}
\usepackage{float}
\usepackage{amsmath}
\usepackage{multicol}
\usepackage{amssymb}
\usepackage[normalem]{ulem}
\useunder{\uline}{\ul}{}
\usepackage{wrapfig}


\usepackage[table,xcdraw]{xcolor}


%% ----------------------------------------------------------------
\begin{document}
\frontmatter
\title      {Heterogeneous Agent-based Model for Supermarket Competition}
\authors    {\texorpdfstring
             {\href{mailto:sc22g13@ecs.soton.ac.uk}{Stefan J. Collier}}
             {Stefan J. Collier}
            }
\addresses  {\groupname\\\deptname\\\univname}
\date       {\today}
\subject    {}
\keywords   {}
\supervisor {Dr. Maria Polukarov}
\examiner   {Professor Sheng Chen}

\maketitle
\begin{abstract}
This project aim was to model and analyse the effects of competitive pricing behaviors of grocery retailers on the British market. 

This was achieved by creating a multi-agent model, containing retailer and consumer agents. The heterogeneous crowd of retailers employs either a uniform pricing strategy or a ‘local price flexing’ strategy. The actions of these retailers are chosen by predicting the profit of each action, using a perceptron. Following on from the consideration of different economic models, a discrete model was developed so that software agents have a discrete environment to operate within. Within the model, it has been observed how supermarkets with differing behaviors affect a heterogeneous crowd of consumer agents. The model was implemented in Java with Python used to evaluate the results. 

The simulation displays good acceptance with real grocery market behavior, i.e. captures the performance of British retailers thus can be used to determine the impact of changes in their behavior on their competitors and consumers.Furthermore it can be used to provide insight into sustainability of volatile pricing strategies, providing a useful insight in volatility of British supermarket retail industry. 
\end{abstract}
\acknowledgements{
I would like to express my sincere gratitude to Dr Maria Polukarov for her guidance and support which provided me the freedom to take this research in the direction of my interest.\\
\\
I would also like to thank my family and friends for their encouragement and support. To those who quietly listened to my software complaints. To those who worked throughout the nights with me. To those who helped me write what I couldn't say. I cannot thank you enough.
}

\declaration{
I, Stefan Collier, declare that this dissertation and the work presented in it are my own and has been generated by me as the result of my own original research.\\
I confirm that:\\
1. This work was done wholly or mainly while in candidature for a degree at this University;\\
2. Where any part of this dissertation has previously been submitted for any other qualification at this University or any other institution, this has been clearly stated;\\
3. Where I have consulted the published work of others, this is always clearly attributed;\\
4. Where I have quoted from the work of others, the source is always given. With the exception of such quotations, this dissertation is entirely my own work;\\
5. I have acknowledged all main sources of help;\\
6. Where the thesis is based on work done by myself jointly with others, I have made clear exactly what was done by others and what I have contributed myself;\\
7. Either none of this work has been published before submission, or parts of this work have been published by :\\
\\
Stefan Collier\\
April 2016
}
\tableofcontents
\listoffigures
\listoftables

\mainmatter
%% ----------------------------------------------------------------
%\include{Introduction}
%\include{Conclusions}
\include{chapters/1Project/main}
\include{chapters/2Lit/main}
\include{chapters/3Design/HighLevel}
\include{chapters/3Design/InDepth}
\include{chapters/4Impl/main}

\include{chapters/5Experiments/1/main}
\include{chapters/5Experiments/2/main}
\include{chapters/5Experiments/3/main}
\include{chapters/5Experiments/4/main}

\include{chapters/6Conclusion/main}

\appendix
\include{appendix/AppendixB}
\include{appendix/D/main}
\include{appendix/AppendixC}

\backmatter
\bibliographystyle{ecs}
\bibliography{ECS}
\end{document}
%% ----------------------------------------------------------------

 %% ----------------------------------------------------------------
%% Progress.tex
%% ---------------------------------------------------------------- 
\documentclass{ecsprogress}    % Use the progress Style
\graphicspath{{../figs/}}   % Location of your graphics files
    \usepackage{natbib}            % Use Natbib style for the refs.
\hypersetup{colorlinks=true}   % Set to false for black/white printing
\input{Definitions}            % Include your abbreviations



\usepackage{enumitem}% http://ctan.org/pkg/enumitem
\usepackage{multirow}
\usepackage{float}
\usepackage{amsmath}
\usepackage{multicol}
\usepackage{amssymb}
\usepackage[normalem]{ulem}
\useunder{\uline}{\ul}{}
\usepackage{wrapfig}


\usepackage[table,xcdraw]{xcolor}


%% ----------------------------------------------------------------
\begin{document}
\frontmatter
\title      {Heterogeneous Agent-based Model for Supermarket Competition}
\authors    {\texorpdfstring
             {\href{mailto:sc22g13@ecs.soton.ac.uk}{Stefan J. Collier}}
             {Stefan J. Collier}
            }
\addresses  {\groupname\\\deptname\\\univname}
\date       {\today}
\subject    {}
\keywords   {}
\supervisor {Dr. Maria Polukarov}
\examiner   {Professor Sheng Chen}

\maketitle
\begin{abstract}
This project aim was to model and analyse the effects of competitive pricing behaviors of grocery retailers on the British market. 

This was achieved by creating a multi-agent model, containing retailer and consumer agents. The heterogeneous crowd of retailers employs either a uniform pricing strategy or a ‘local price flexing’ strategy. The actions of these retailers are chosen by predicting the profit of each action, using a perceptron. Following on from the consideration of different economic models, a discrete model was developed so that software agents have a discrete environment to operate within. Within the model, it has been observed how supermarkets with differing behaviors affect a heterogeneous crowd of consumer agents. The model was implemented in Java with Python used to evaluate the results. 

The simulation displays good acceptance with real grocery market behavior, i.e. captures the performance of British retailers thus can be used to determine the impact of changes in their behavior on their competitors and consumers.Furthermore it can be used to provide insight into sustainability of volatile pricing strategies, providing a useful insight in volatility of British supermarket retail industry. 
\end{abstract}
\acknowledgements{
I would like to express my sincere gratitude to Dr Maria Polukarov for her guidance and support which provided me the freedom to take this research in the direction of my interest.\\
\\
I would also like to thank my family and friends for their encouragement and support. To those who quietly listened to my software complaints. To those who worked throughout the nights with me. To those who helped me write what I couldn't say. I cannot thank you enough.
}

\declaration{
I, Stefan Collier, declare that this dissertation and the work presented in it are my own and has been generated by me as the result of my own original research.\\
I confirm that:\\
1. This work was done wholly or mainly while in candidature for a degree at this University;\\
2. Where any part of this dissertation has previously been submitted for any other qualification at this University or any other institution, this has been clearly stated;\\
3. Where I have consulted the published work of others, this is always clearly attributed;\\
4. Where I have quoted from the work of others, the source is always given. With the exception of such quotations, this dissertation is entirely my own work;\\
5. I have acknowledged all main sources of help;\\
6. Where the thesis is based on work done by myself jointly with others, I have made clear exactly what was done by others and what I have contributed myself;\\
7. Either none of this work has been published before submission, or parts of this work have been published by :\\
\\
Stefan Collier\\
April 2016
}
\tableofcontents
\listoffigures
\listoftables

\mainmatter
%% ----------------------------------------------------------------
%\include{Introduction}
%\include{Conclusions}
\include{chapters/1Project/main}
\include{chapters/2Lit/main}
\include{chapters/3Design/HighLevel}
\include{chapters/3Design/InDepth}
\include{chapters/4Impl/main}

\include{chapters/5Experiments/1/main}
\include{chapters/5Experiments/2/main}
\include{chapters/5Experiments/3/main}
\include{chapters/5Experiments/4/main}

\include{chapters/6Conclusion/main}

\appendix
\include{appendix/AppendixB}
\include{appendix/D/main}
\include{appendix/AppendixC}

\backmatter
\bibliographystyle{ecs}
\bibliography{ECS}
\end{document}
%% ----------------------------------------------------------------

\include{chapters/3Design/HighLevel}
\include{chapters/3Design/InDepth}
 %% ----------------------------------------------------------------
%% Progress.tex
%% ---------------------------------------------------------------- 
\documentclass{ecsprogress}    % Use the progress Style
\graphicspath{{../figs/}}   % Location of your graphics files
    \usepackage{natbib}            % Use Natbib style for the refs.
\hypersetup{colorlinks=true}   % Set to false for black/white printing
\input{Definitions}            % Include your abbreviations



\usepackage{enumitem}% http://ctan.org/pkg/enumitem
\usepackage{multirow}
\usepackage{float}
\usepackage{amsmath}
\usepackage{multicol}
\usepackage{amssymb}
\usepackage[normalem]{ulem}
\useunder{\uline}{\ul}{}
\usepackage{wrapfig}


\usepackage[table,xcdraw]{xcolor}


%% ----------------------------------------------------------------
\begin{document}
\frontmatter
\title      {Heterogeneous Agent-based Model for Supermarket Competition}
\authors    {\texorpdfstring
             {\href{mailto:sc22g13@ecs.soton.ac.uk}{Stefan J. Collier}}
             {Stefan J. Collier}
            }
\addresses  {\groupname\\\deptname\\\univname}
\date       {\today}
\subject    {}
\keywords   {}
\supervisor {Dr. Maria Polukarov}
\examiner   {Professor Sheng Chen}

\maketitle
\begin{abstract}
This project aim was to model and analyse the effects of competitive pricing behaviors of grocery retailers on the British market. 

This was achieved by creating a multi-agent model, containing retailer and consumer agents. The heterogeneous crowd of retailers employs either a uniform pricing strategy or a ‘local price flexing’ strategy. The actions of these retailers are chosen by predicting the profit of each action, using a perceptron. Following on from the consideration of different economic models, a discrete model was developed so that software agents have a discrete environment to operate within. Within the model, it has been observed how supermarkets with differing behaviors affect a heterogeneous crowd of consumer agents. The model was implemented in Java with Python used to evaluate the results. 

The simulation displays good acceptance with real grocery market behavior, i.e. captures the performance of British retailers thus can be used to determine the impact of changes in their behavior on their competitors and consumers.Furthermore it can be used to provide insight into sustainability of volatile pricing strategies, providing a useful insight in volatility of British supermarket retail industry. 
\end{abstract}
\acknowledgements{
I would like to express my sincere gratitude to Dr Maria Polukarov for her guidance and support which provided me the freedom to take this research in the direction of my interest.\\
\\
I would also like to thank my family and friends for their encouragement and support. To those who quietly listened to my software complaints. To those who worked throughout the nights with me. To those who helped me write what I couldn't say. I cannot thank you enough.
}

\declaration{
I, Stefan Collier, declare that this dissertation and the work presented in it are my own and has been generated by me as the result of my own original research.\\
I confirm that:\\
1. This work was done wholly or mainly while in candidature for a degree at this University;\\
2. Where any part of this dissertation has previously been submitted for any other qualification at this University or any other institution, this has been clearly stated;\\
3. Where I have consulted the published work of others, this is always clearly attributed;\\
4. Where I have quoted from the work of others, the source is always given. With the exception of such quotations, this dissertation is entirely my own work;\\
5. I have acknowledged all main sources of help;\\
6. Where the thesis is based on work done by myself jointly with others, I have made clear exactly what was done by others and what I have contributed myself;\\
7. Either none of this work has been published before submission, or parts of this work have been published by :\\
\\
Stefan Collier\\
April 2016
}
\tableofcontents
\listoffigures
\listoftables

\mainmatter
%% ----------------------------------------------------------------
%\include{Introduction}
%\include{Conclusions}
\include{chapters/1Project/main}
\include{chapters/2Lit/main}
\include{chapters/3Design/HighLevel}
\include{chapters/3Design/InDepth}
\include{chapters/4Impl/main}

\include{chapters/5Experiments/1/main}
\include{chapters/5Experiments/2/main}
\include{chapters/5Experiments/3/main}
\include{chapters/5Experiments/4/main}

\include{chapters/6Conclusion/main}

\appendix
\include{appendix/AppendixB}
\include{appendix/D/main}
\include{appendix/AppendixC}

\backmatter
\bibliographystyle{ecs}
\bibliography{ECS}
\end{document}
%% ----------------------------------------------------------------


 %% ----------------------------------------------------------------
%% Progress.tex
%% ---------------------------------------------------------------- 
\documentclass{ecsprogress}    % Use the progress Style
\graphicspath{{../figs/}}   % Location of your graphics files
    \usepackage{natbib}            % Use Natbib style for the refs.
\hypersetup{colorlinks=true}   % Set to false for black/white printing
\input{Definitions}            % Include your abbreviations



\usepackage{enumitem}% http://ctan.org/pkg/enumitem
\usepackage{multirow}
\usepackage{float}
\usepackage{amsmath}
\usepackage{multicol}
\usepackage{amssymb}
\usepackage[normalem]{ulem}
\useunder{\uline}{\ul}{}
\usepackage{wrapfig}


\usepackage[table,xcdraw]{xcolor}


%% ----------------------------------------------------------------
\begin{document}
\frontmatter
\title      {Heterogeneous Agent-based Model for Supermarket Competition}
\authors    {\texorpdfstring
             {\href{mailto:sc22g13@ecs.soton.ac.uk}{Stefan J. Collier}}
             {Stefan J. Collier}
            }
\addresses  {\groupname\\\deptname\\\univname}
\date       {\today}
\subject    {}
\keywords   {}
\supervisor {Dr. Maria Polukarov}
\examiner   {Professor Sheng Chen}

\maketitle
\begin{abstract}
This project aim was to model and analyse the effects of competitive pricing behaviors of grocery retailers on the British market. 

This was achieved by creating a multi-agent model, containing retailer and consumer agents. The heterogeneous crowd of retailers employs either a uniform pricing strategy or a ‘local price flexing’ strategy. The actions of these retailers are chosen by predicting the profit of each action, using a perceptron. Following on from the consideration of different economic models, a discrete model was developed so that software agents have a discrete environment to operate within. Within the model, it has been observed how supermarkets with differing behaviors affect a heterogeneous crowd of consumer agents. The model was implemented in Java with Python used to evaluate the results. 

The simulation displays good acceptance with real grocery market behavior, i.e. captures the performance of British retailers thus can be used to determine the impact of changes in their behavior on their competitors and consumers.Furthermore it can be used to provide insight into sustainability of volatile pricing strategies, providing a useful insight in volatility of British supermarket retail industry. 
\end{abstract}
\acknowledgements{
I would like to express my sincere gratitude to Dr Maria Polukarov for her guidance and support which provided me the freedom to take this research in the direction of my interest.\\
\\
I would also like to thank my family and friends for their encouragement and support. To those who quietly listened to my software complaints. To those who worked throughout the nights with me. To those who helped me write what I couldn't say. I cannot thank you enough.
}

\declaration{
I, Stefan Collier, declare that this dissertation and the work presented in it are my own and has been generated by me as the result of my own original research.\\
I confirm that:\\
1. This work was done wholly or mainly while in candidature for a degree at this University;\\
2. Where any part of this dissertation has previously been submitted for any other qualification at this University or any other institution, this has been clearly stated;\\
3. Where I have consulted the published work of others, this is always clearly attributed;\\
4. Where I have quoted from the work of others, the source is always given. With the exception of such quotations, this dissertation is entirely my own work;\\
5. I have acknowledged all main sources of help;\\
6. Where the thesis is based on work done by myself jointly with others, I have made clear exactly what was done by others and what I have contributed myself;\\
7. Either none of this work has been published before submission, or parts of this work have been published by :\\
\\
Stefan Collier\\
April 2016
}
\tableofcontents
\listoffigures
\listoftables

\mainmatter
%% ----------------------------------------------------------------
%\include{Introduction}
%\include{Conclusions}
\include{chapters/1Project/main}
\include{chapters/2Lit/main}
\include{chapters/3Design/HighLevel}
\include{chapters/3Design/InDepth}
\include{chapters/4Impl/main}

\include{chapters/5Experiments/1/main}
\include{chapters/5Experiments/2/main}
\include{chapters/5Experiments/3/main}
\include{chapters/5Experiments/4/main}

\include{chapters/6Conclusion/main}

\appendix
\include{appendix/AppendixB}
\include{appendix/D/main}
\include{appendix/AppendixC}

\backmatter
\bibliographystyle{ecs}
\bibliography{ECS}
\end{document}
%% ----------------------------------------------------------------

 %% ----------------------------------------------------------------
%% Progress.tex
%% ---------------------------------------------------------------- 
\documentclass{ecsprogress}    % Use the progress Style
\graphicspath{{../figs/}}   % Location of your graphics files
    \usepackage{natbib}            % Use Natbib style for the refs.
\hypersetup{colorlinks=true}   % Set to false for black/white printing
\input{Definitions}            % Include your abbreviations



\usepackage{enumitem}% http://ctan.org/pkg/enumitem
\usepackage{multirow}
\usepackage{float}
\usepackage{amsmath}
\usepackage{multicol}
\usepackage{amssymb}
\usepackage[normalem]{ulem}
\useunder{\uline}{\ul}{}
\usepackage{wrapfig}


\usepackage[table,xcdraw]{xcolor}


%% ----------------------------------------------------------------
\begin{document}
\frontmatter
\title      {Heterogeneous Agent-based Model for Supermarket Competition}
\authors    {\texorpdfstring
             {\href{mailto:sc22g13@ecs.soton.ac.uk}{Stefan J. Collier}}
             {Stefan J. Collier}
            }
\addresses  {\groupname\\\deptname\\\univname}
\date       {\today}
\subject    {}
\keywords   {}
\supervisor {Dr. Maria Polukarov}
\examiner   {Professor Sheng Chen}

\maketitle
\begin{abstract}
This project aim was to model and analyse the effects of competitive pricing behaviors of grocery retailers on the British market. 

This was achieved by creating a multi-agent model, containing retailer and consumer agents. The heterogeneous crowd of retailers employs either a uniform pricing strategy or a ‘local price flexing’ strategy. The actions of these retailers are chosen by predicting the profit of each action, using a perceptron. Following on from the consideration of different economic models, a discrete model was developed so that software agents have a discrete environment to operate within. Within the model, it has been observed how supermarkets with differing behaviors affect a heterogeneous crowd of consumer agents. The model was implemented in Java with Python used to evaluate the results. 

The simulation displays good acceptance with real grocery market behavior, i.e. captures the performance of British retailers thus can be used to determine the impact of changes in their behavior on their competitors and consumers.Furthermore it can be used to provide insight into sustainability of volatile pricing strategies, providing a useful insight in volatility of British supermarket retail industry. 
\end{abstract}
\acknowledgements{
I would like to express my sincere gratitude to Dr Maria Polukarov for her guidance and support which provided me the freedom to take this research in the direction of my interest.\\
\\
I would also like to thank my family and friends for their encouragement and support. To those who quietly listened to my software complaints. To those who worked throughout the nights with me. To those who helped me write what I couldn't say. I cannot thank you enough.
}

\declaration{
I, Stefan Collier, declare that this dissertation and the work presented in it are my own and has been generated by me as the result of my own original research.\\
I confirm that:\\
1. This work was done wholly or mainly while in candidature for a degree at this University;\\
2. Where any part of this dissertation has previously been submitted for any other qualification at this University or any other institution, this has been clearly stated;\\
3. Where I have consulted the published work of others, this is always clearly attributed;\\
4. Where I have quoted from the work of others, the source is always given. With the exception of such quotations, this dissertation is entirely my own work;\\
5. I have acknowledged all main sources of help;\\
6. Where the thesis is based on work done by myself jointly with others, I have made clear exactly what was done by others and what I have contributed myself;\\
7. Either none of this work has been published before submission, or parts of this work have been published by :\\
\\
Stefan Collier\\
April 2016
}
\tableofcontents
\listoffigures
\listoftables

\mainmatter
%% ----------------------------------------------------------------
%\include{Introduction}
%\include{Conclusions}
\include{chapters/1Project/main}
\include{chapters/2Lit/main}
\include{chapters/3Design/HighLevel}
\include{chapters/3Design/InDepth}
\include{chapters/4Impl/main}

\include{chapters/5Experiments/1/main}
\include{chapters/5Experiments/2/main}
\include{chapters/5Experiments/3/main}
\include{chapters/5Experiments/4/main}

\include{chapters/6Conclusion/main}

\appendix
\include{appendix/AppendixB}
\include{appendix/D/main}
\include{appendix/AppendixC}

\backmatter
\bibliographystyle{ecs}
\bibliography{ECS}
\end{document}
%% ----------------------------------------------------------------

 %% ----------------------------------------------------------------
%% Progress.tex
%% ---------------------------------------------------------------- 
\documentclass{ecsprogress}    % Use the progress Style
\graphicspath{{../figs/}}   % Location of your graphics files
    \usepackage{natbib}            % Use Natbib style for the refs.
\hypersetup{colorlinks=true}   % Set to false for black/white printing
\input{Definitions}            % Include your abbreviations



\usepackage{enumitem}% http://ctan.org/pkg/enumitem
\usepackage{multirow}
\usepackage{float}
\usepackage{amsmath}
\usepackage{multicol}
\usepackage{amssymb}
\usepackage[normalem]{ulem}
\useunder{\uline}{\ul}{}
\usepackage{wrapfig}


\usepackage[table,xcdraw]{xcolor}


%% ----------------------------------------------------------------
\begin{document}
\frontmatter
\title      {Heterogeneous Agent-based Model for Supermarket Competition}
\authors    {\texorpdfstring
             {\href{mailto:sc22g13@ecs.soton.ac.uk}{Stefan J. Collier}}
             {Stefan J. Collier}
            }
\addresses  {\groupname\\\deptname\\\univname}
\date       {\today}
\subject    {}
\keywords   {}
\supervisor {Dr. Maria Polukarov}
\examiner   {Professor Sheng Chen}

\maketitle
\begin{abstract}
This project aim was to model and analyse the effects of competitive pricing behaviors of grocery retailers on the British market. 

This was achieved by creating a multi-agent model, containing retailer and consumer agents. The heterogeneous crowd of retailers employs either a uniform pricing strategy or a ‘local price flexing’ strategy. The actions of these retailers are chosen by predicting the profit of each action, using a perceptron. Following on from the consideration of different economic models, a discrete model was developed so that software agents have a discrete environment to operate within. Within the model, it has been observed how supermarkets with differing behaviors affect a heterogeneous crowd of consumer agents. The model was implemented in Java with Python used to evaluate the results. 

The simulation displays good acceptance with real grocery market behavior, i.e. captures the performance of British retailers thus can be used to determine the impact of changes in their behavior on their competitors and consumers.Furthermore it can be used to provide insight into sustainability of volatile pricing strategies, providing a useful insight in volatility of British supermarket retail industry. 
\end{abstract}
\acknowledgements{
I would like to express my sincere gratitude to Dr Maria Polukarov for her guidance and support which provided me the freedom to take this research in the direction of my interest.\\
\\
I would also like to thank my family and friends for their encouragement and support. To those who quietly listened to my software complaints. To those who worked throughout the nights with me. To those who helped me write what I couldn't say. I cannot thank you enough.
}

\declaration{
I, Stefan Collier, declare that this dissertation and the work presented in it are my own and has been generated by me as the result of my own original research.\\
I confirm that:\\
1. This work was done wholly or mainly while in candidature for a degree at this University;\\
2. Where any part of this dissertation has previously been submitted for any other qualification at this University or any other institution, this has been clearly stated;\\
3. Where I have consulted the published work of others, this is always clearly attributed;\\
4. Where I have quoted from the work of others, the source is always given. With the exception of such quotations, this dissertation is entirely my own work;\\
5. I have acknowledged all main sources of help;\\
6. Where the thesis is based on work done by myself jointly with others, I have made clear exactly what was done by others and what I have contributed myself;\\
7. Either none of this work has been published before submission, or parts of this work have been published by :\\
\\
Stefan Collier\\
April 2016
}
\tableofcontents
\listoffigures
\listoftables

\mainmatter
%% ----------------------------------------------------------------
%\include{Introduction}
%\include{Conclusions}
\include{chapters/1Project/main}
\include{chapters/2Lit/main}
\include{chapters/3Design/HighLevel}
\include{chapters/3Design/InDepth}
\include{chapters/4Impl/main}

\include{chapters/5Experiments/1/main}
\include{chapters/5Experiments/2/main}
\include{chapters/5Experiments/3/main}
\include{chapters/5Experiments/4/main}

\include{chapters/6Conclusion/main}

\appendix
\include{appendix/AppendixB}
\include{appendix/D/main}
\include{appendix/AppendixC}

\backmatter
\bibliographystyle{ecs}
\bibliography{ECS}
\end{document}
%% ----------------------------------------------------------------

 %% ----------------------------------------------------------------
%% Progress.tex
%% ---------------------------------------------------------------- 
\documentclass{ecsprogress}    % Use the progress Style
\graphicspath{{../figs/}}   % Location of your graphics files
    \usepackage{natbib}            % Use Natbib style for the refs.
\hypersetup{colorlinks=true}   % Set to false for black/white printing
\input{Definitions}            % Include your abbreviations



\usepackage{enumitem}% http://ctan.org/pkg/enumitem
\usepackage{multirow}
\usepackage{float}
\usepackage{amsmath}
\usepackage{multicol}
\usepackage{amssymb}
\usepackage[normalem]{ulem}
\useunder{\uline}{\ul}{}
\usepackage{wrapfig}


\usepackage[table,xcdraw]{xcolor}


%% ----------------------------------------------------------------
\begin{document}
\frontmatter
\title      {Heterogeneous Agent-based Model for Supermarket Competition}
\authors    {\texorpdfstring
             {\href{mailto:sc22g13@ecs.soton.ac.uk}{Stefan J. Collier}}
             {Stefan J. Collier}
            }
\addresses  {\groupname\\\deptname\\\univname}
\date       {\today}
\subject    {}
\keywords   {}
\supervisor {Dr. Maria Polukarov}
\examiner   {Professor Sheng Chen}

\maketitle
\begin{abstract}
This project aim was to model and analyse the effects of competitive pricing behaviors of grocery retailers on the British market. 

This was achieved by creating a multi-agent model, containing retailer and consumer agents. The heterogeneous crowd of retailers employs either a uniform pricing strategy or a ‘local price flexing’ strategy. The actions of these retailers are chosen by predicting the profit of each action, using a perceptron. Following on from the consideration of different economic models, a discrete model was developed so that software agents have a discrete environment to operate within. Within the model, it has been observed how supermarkets with differing behaviors affect a heterogeneous crowd of consumer agents. The model was implemented in Java with Python used to evaluate the results. 

The simulation displays good acceptance with real grocery market behavior, i.e. captures the performance of British retailers thus can be used to determine the impact of changes in their behavior on their competitors and consumers.Furthermore it can be used to provide insight into sustainability of volatile pricing strategies, providing a useful insight in volatility of British supermarket retail industry. 
\end{abstract}
\acknowledgements{
I would like to express my sincere gratitude to Dr Maria Polukarov for her guidance and support which provided me the freedom to take this research in the direction of my interest.\\
\\
I would also like to thank my family and friends for their encouragement and support. To those who quietly listened to my software complaints. To those who worked throughout the nights with me. To those who helped me write what I couldn't say. I cannot thank you enough.
}

\declaration{
I, Stefan Collier, declare that this dissertation and the work presented in it are my own and has been generated by me as the result of my own original research.\\
I confirm that:\\
1. This work was done wholly or mainly while in candidature for a degree at this University;\\
2. Where any part of this dissertation has previously been submitted for any other qualification at this University or any other institution, this has been clearly stated;\\
3. Where I have consulted the published work of others, this is always clearly attributed;\\
4. Where I have quoted from the work of others, the source is always given. With the exception of such quotations, this dissertation is entirely my own work;\\
5. I have acknowledged all main sources of help;\\
6. Where the thesis is based on work done by myself jointly with others, I have made clear exactly what was done by others and what I have contributed myself;\\
7. Either none of this work has been published before submission, or parts of this work have been published by :\\
\\
Stefan Collier\\
April 2016
}
\tableofcontents
\listoffigures
\listoftables

\mainmatter
%% ----------------------------------------------------------------
%\include{Introduction}
%\include{Conclusions}
\include{chapters/1Project/main}
\include{chapters/2Lit/main}
\include{chapters/3Design/HighLevel}
\include{chapters/3Design/InDepth}
\include{chapters/4Impl/main}

\include{chapters/5Experiments/1/main}
\include{chapters/5Experiments/2/main}
\include{chapters/5Experiments/3/main}
\include{chapters/5Experiments/4/main}

\include{chapters/6Conclusion/main}

\appendix
\include{appendix/AppendixB}
\include{appendix/D/main}
\include{appendix/AppendixC}

\backmatter
\bibliographystyle{ecs}
\bibliography{ECS}
\end{document}
%% ----------------------------------------------------------------


 %% ----------------------------------------------------------------
%% Progress.tex
%% ---------------------------------------------------------------- 
\documentclass{ecsprogress}    % Use the progress Style
\graphicspath{{../figs/}}   % Location of your graphics files
    \usepackage{natbib}            % Use Natbib style for the refs.
\hypersetup{colorlinks=true}   % Set to false for black/white printing
\input{Definitions}            % Include your abbreviations



\usepackage{enumitem}% http://ctan.org/pkg/enumitem
\usepackage{multirow}
\usepackage{float}
\usepackage{amsmath}
\usepackage{multicol}
\usepackage{amssymb}
\usepackage[normalem]{ulem}
\useunder{\uline}{\ul}{}
\usepackage{wrapfig}


\usepackage[table,xcdraw]{xcolor}


%% ----------------------------------------------------------------
\begin{document}
\frontmatter
\title      {Heterogeneous Agent-based Model for Supermarket Competition}
\authors    {\texorpdfstring
             {\href{mailto:sc22g13@ecs.soton.ac.uk}{Stefan J. Collier}}
             {Stefan J. Collier}
            }
\addresses  {\groupname\\\deptname\\\univname}
\date       {\today}
\subject    {}
\keywords   {}
\supervisor {Dr. Maria Polukarov}
\examiner   {Professor Sheng Chen}

\maketitle
\begin{abstract}
This project aim was to model and analyse the effects of competitive pricing behaviors of grocery retailers on the British market. 

This was achieved by creating a multi-agent model, containing retailer and consumer agents. The heterogeneous crowd of retailers employs either a uniform pricing strategy or a ‘local price flexing’ strategy. The actions of these retailers are chosen by predicting the profit of each action, using a perceptron. Following on from the consideration of different economic models, a discrete model was developed so that software agents have a discrete environment to operate within. Within the model, it has been observed how supermarkets with differing behaviors affect a heterogeneous crowd of consumer agents. The model was implemented in Java with Python used to evaluate the results. 

The simulation displays good acceptance with real grocery market behavior, i.e. captures the performance of British retailers thus can be used to determine the impact of changes in their behavior on their competitors and consumers.Furthermore it can be used to provide insight into sustainability of volatile pricing strategies, providing a useful insight in volatility of British supermarket retail industry. 
\end{abstract}
\acknowledgements{
I would like to express my sincere gratitude to Dr Maria Polukarov for her guidance and support which provided me the freedom to take this research in the direction of my interest.\\
\\
I would also like to thank my family and friends for their encouragement and support. To those who quietly listened to my software complaints. To those who worked throughout the nights with me. To those who helped me write what I couldn't say. I cannot thank you enough.
}

\declaration{
I, Stefan Collier, declare that this dissertation and the work presented in it are my own and has been generated by me as the result of my own original research.\\
I confirm that:\\
1. This work was done wholly or mainly while in candidature for a degree at this University;\\
2. Where any part of this dissertation has previously been submitted for any other qualification at this University or any other institution, this has been clearly stated;\\
3. Where I have consulted the published work of others, this is always clearly attributed;\\
4. Where I have quoted from the work of others, the source is always given. With the exception of such quotations, this dissertation is entirely my own work;\\
5. I have acknowledged all main sources of help;\\
6. Where the thesis is based on work done by myself jointly with others, I have made clear exactly what was done by others and what I have contributed myself;\\
7. Either none of this work has been published before submission, or parts of this work have been published by :\\
\\
Stefan Collier\\
April 2016
}
\tableofcontents
\listoffigures
\listoftables

\mainmatter
%% ----------------------------------------------------------------
%\include{Introduction}
%\include{Conclusions}
\include{chapters/1Project/main}
\include{chapters/2Lit/main}
\include{chapters/3Design/HighLevel}
\include{chapters/3Design/InDepth}
\include{chapters/4Impl/main}

\include{chapters/5Experiments/1/main}
\include{chapters/5Experiments/2/main}
\include{chapters/5Experiments/3/main}
\include{chapters/5Experiments/4/main}

\include{chapters/6Conclusion/main}

\appendix
\include{appendix/AppendixB}
\include{appendix/D/main}
\include{appendix/AppendixC}

\backmatter
\bibliographystyle{ecs}
\bibliography{ECS}
\end{document}
%% ----------------------------------------------------------------


\appendix
\include{appendix/AppendixB}
 %% ----------------------------------------------------------------
%% Progress.tex
%% ---------------------------------------------------------------- 
\documentclass{ecsprogress}    % Use the progress Style
\graphicspath{{../figs/}}   % Location of your graphics files
    \usepackage{natbib}            % Use Natbib style for the refs.
\hypersetup{colorlinks=true}   % Set to false for black/white printing
\input{Definitions}            % Include your abbreviations



\usepackage{enumitem}% http://ctan.org/pkg/enumitem
\usepackage{multirow}
\usepackage{float}
\usepackage{amsmath}
\usepackage{multicol}
\usepackage{amssymb}
\usepackage[normalem]{ulem}
\useunder{\uline}{\ul}{}
\usepackage{wrapfig}


\usepackage[table,xcdraw]{xcolor}


%% ----------------------------------------------------------------
\begin{document}
\frontmatter
\title      {Heterogeneous Agent-based Model for Supermarket Competition}
\authors    {\texorpdfstring
             {\href{mailto:sc22g13@ecs.soton.ac.uk}{Stefan J. Collier}}
             {Stefan J. Collier}
            }
\addresses  {\groupname\\\deptname\\\univname}
\date       {\today}
\subject    {}
\keywords   {}
\supervisor {Dr. Maria Polukarov}
\examiner   {Professor Sheng Chen}

\maketitle
\begin{abstract}
This project aim was to model and analyse the effects of competitive pricing behaviors of grocery retailers on the British market. 

This was achieved by creating a multi-agent model, containing retailer and consumer agents. The heterogeneous crowd of retailers employs either a uniform pricing strategy or a ‘local price flexing’ strategy. The actions of these retailers are chosen by predicting the profit of each action, using a perceptron. Following on from the consideration of different economic models, a discrete model was developed so that software agents have a discrete environment to operate within. Within the model, it has been observed how supermarkets with differing behaviors affect a heterogeneous crowd of consumer agents. The model was implemented in Java with Python used to evaluate the results. 

The simulation displays good acceptance with real grocery market behavior, i.e. captures the performance of British retailers thus can be used to determine the impact of changes in their behavior on their competitors and consumers.Furthermore it can be used to provide insight into sustainability of volatile pricing strategies, providing a useful insight in volatility of British supermarket retail industry. 
\end{abstract}
\acknowledgements{
I would like to express my sincere gratitude to Dr Maria Polukarov for her guidance and support which provided me the freedom to take this research in the direction of my interest.\\
\\
I would also like to thank my family and friends for their encouragement and support. To those who quietly listened to my software complaints. To those who worked throughout the nights with me. To those who helped me write what I couldn't say. I cannot thank you enough.
}

\declaration{
I, Stefan Collier, declare that this dissertation and the work presented in it are my own and has been generated by me as the result of my own original research.\\
I confirm that:\\
1. This work was done wholly or mainly while in candidature for a degree at this University;\\
2. Where any part of this dissertation has previously been submitted for any other qualification at this University or any other institution, this has been clearly stated;\\
3. Where I have consulted the published work of others, this is always clearly attributed;\\
4. Where I have quoted from the work of others, the source is always given. With the exception of such quotations, this dissertation is entirely my own work;\\
5. I have acknowledged all main sources of help;\\
6. Where the thesis is based on work done by myself jointly with others, I have made clear exactly what was done by others and what I have contributed myself;\\
7. Either none of this work has been published before submission, or parts of this work have been published by :\\
\\
Stefan Collier\\
April 2016
}
\tableofcontents
\listoffigures
\listoftables

\mainmatter
%% ----------------------------------------------------------------
%\include{Introduction}
%\include{Conclusions}
\include{chapters/1Project/main}
\include{chapters/2Lit/main}
\include{chapters/3Design/HighLevel}
\include{chapters/3Design/InDepth}
\include{chapters/4Impl/main}

\include{chapters/5Experiments/1/main}
\include{chapters/5Experiments/2/main}
\include{chapters/5Experiments/3/main}
\include{chapters/5Experiments/4/main}

\include{chapters/6Conclusion/main}

\appendix
\include{appendix/AppendixB}
\include{appendix/D/main}
\include{appendix/AppendixC}

\backmatter
\bibliographystyle{ecs}
\bibliography{ECS}
\end{document}
%% ----------------------------------------------------------------

\include{appendix/AppendixC}

\backmatter
\bibliographystyle{ecs}
\bibliography{ECS}
\end{document}
%% ----------------------------------------------------------------

\include{appendix/AppendixC}

\backmatter
\bibliographystyle{ecs}
\bibliography{ECS}
\end{document}
%% ----------------------------------------------------------------

 %% ----------------------------------------------------------------
%% Progress.tex
%% ---------------------------------------------------------------- 
\documentclass{ecsprogress}    % Use the progress Style
\graphicspath{{../figs/}}   % Location of your graphics files
    \usepackage{natbib}            % Use Natbib style for the refs.
\hypersetup{colorlinks=true}   % Set to false for black/white printing
\input{Definitions}            % Include your abbreviations



\usepackage{enumitem}% http://ctan.org/pkg/enumitem
\usepackage{multirow}
\usepackage{float}
\usepackage{amsmath}
\usepackage{multicol}
\usepackage{amssymb}
\usepackage[normalem]{ulem}
\useunder{\uline}{\ul}{}
\usepackage{wrapfig}


\usepackage[table,xcdraw]{xcolor}


%% ----------------------------------------------------------------
\begin{document}
\frontmatter
\title      {Heterogeneous Agent-based Model for Supermarket Competition}
\authors    {\texorpdfstring
             {\href{mailto:sc22g13@ecs.soton.ac.uk}{Stefan J. Collier}}
             {Stefan J. Collier}
            }
\addresses  {\groupname\\\deptname\\\univname}
\date       {\today}
\subject    {}
\keywords   {}
\supervisor {Dr. Maria Polukarov}
\examiner   {Professor Sheng Chen}

\maketitle
\begin{abstract}
This project aim was to model and analyse the effects of competitive pricing behaviors of grocery retailers on the British market. 

This was achieved by creating a multi-agent model, containing retailer and consumer agents. The heterogeneous crowd of retailers employs either a uniform pricing strategy or a ‘local price flexing’ strategy. The actions of these retailers are chosen by predicting the profit of each action, using a perceptron. Following on from the consideration of different economic models, a discrete model was developed so that software agents have a discrete environment to operate within. Within the model, it has been observed how supermarkets with differing behaviors affect a heterogeneous crowd of consumer agents. The model was implemented in Java with Python used to evaluate the results. 

The simulation displays good acceptance with real grocery market behavior, i.e. captures the performance of British retailers thus can be used to determine the impact of changes in their behavior on their competitors and consumers.Furthermore it can be used to provide insight into sustainability of volatile pricing strategies, providing a useful insight in volatility of British supermarket retail industry. 
\end{abstract}
\acknowledgements{
I would like to express my sincere gratitude to Dr Maria Polukarov for her guidance and support which provided me the freedom to take this research in the direction of my interest.\\
\\
I would also like to thank my family and friends for their encouragement and support. To those who quietly listened to my software complaints. To those who worked throughout the nights with me. To those who helped me write what I couldn't say. I cannot thank you enough.
}

\declaration{
I, Stefan Collier, declare that this dissertation and the work presented in it are my own and has been generated by me as the result of my own original research.\\
I confirm that:\\
1. This work was done wholly or mainly while in candidature for a degree at this University;\\
2. Where any part of this dissertation has previously been submitted for any other qualification at this University or any other institution, this has been clearly stated;\\
3. Where I have consulted the published work of others, this is always clearly attributed;\\
4. Where I have quoted from the work of others, the source is always given. With the exception of such quotations, this dissertation is entirely my own work;\\
5. I have acknowledged all main sources of help;\\
6. Where the thesis is based on work done by myself jointly with others, I have made clear exactly what was done by others and what I have contributed myself;\\
7. Either none of this work has been published before submission, or parts of this work have been published by :\\
\\
Stefan Collier\\
April 2016
}
\tableofcontents
\listoffigures
\listoftables

\mainmatter
%% ----------------------------------------------------------------
%\include{Introduction}
%\include{Conclusions}
 %% ----------------------------------------------------------------
%% Progress.tex
%% ---------------------------------------------------------------- 
\documentclass{ecsprogress}    % Use the progress Style
\graphicspath{{../figs/}}   % Location of your graphics files
    \usepackage{natbib}            % Use Natbib style for the refs.
\hypersetup{colorlinks=true}   % Set to false for black/white printing
\input{Definitions}            % Include your abbreviations



\usepackage{enumitem}% http://ctan.org/pkg/enumitem
\usepackage{multirow}
\usepackage{float}
\usepackage{amsmath}
\usepackage{multicol}
\usepackage{amssymb}
\usepackage[normalem]{ulem}
\useunder{\uline}{\ul}{}
\usepackage{wrapfig}


\usepackage[table,xcdraw]{xcolor}


%% ----------------------------------------------------------------
\begin{document}
\frontmatter
\title      {Heterogeneous Agent-based Model for Supermarket Competition}
\authors    {\texorpdfstring
             {\href{mailto:sc22g13@ecs.soton.ac.uk}{Stefan J. Collier}}
             {Stefan J. Collier}
            }
\addresses  {\groupname\\\deptname\\\univname}
\date       {\today}
\subject    {}
\keywords   {}
\supervisor {Dr. Maria Polukarov}
\examiner   {Professor Sheng Chen}

\maketitle
\begin{abstract}
This project aim was to model and analyse the effects of competitive pricing behaviors of grocery retailers on the British market. 

This was achieved by creating a multi-agent model, containing retailer and consumer agents. The heterogeneous crowd of retailers employs either a uniform pricing strategy or a ‘local price flexing’ strategy. The actions of these retailers are chosen by predicting the profit of each action, using a perceptron. Following on from the consideration of different economic models, a discrete model was developed so that software agents have a discrete environment to operate within. Within the model, it has been observed how supermarkets with differing behaviors affect a heterogeneous crowd of consumer agents. The model was implemented in Java with Python used to evaluate the results. 

The simulation displays good acceptance with real grocery market behavior, i.e. captures the performance of British retailers thus can be used to determine the impact of changes in their behavior on their competitors and consumers.Furthermore it can be used to provide insight into sustainability of volatile pricing strategies, providing a useful insight in volatility of British supermarket retail industry. 
\end{abstract}
\acknowledgements{
I would like to express my sincere gratitude to Dr Maria Polukarov for her guidance and support which provided me the freedom to take this research in the direction of my interest.\\
\\
I would also like to thank my family and friends for their encouragement and support. To those who quietly listened to my software complaints. To those who worked throughout the nights with me. To those who helped me write what I couldn't say. I cannot thank you enough.
}

\declaration{
I, Stefan Collier, declare that this dissertation and the work presented in it are my own and has been generated by me as the result of my own original research.\\
I confirm that:\\
1. This work was done wholly or mainly while in candidature for a degree at this University;\\
2. Where any part of this dissertation has previously been submitted for any other qualification at this University or any other institution, this has been clearly stated;\\
3. Where I have consulted the published work of others, this is always clearly attributed;\\
4. Where I have quoted from the work of others, the source is always given. With the exception of such quotations, this dissertation is entirely my own work;\\
5. I have acknowledged all main sources of help;\\
6. Where the thesis is based on work done by myself jointly with others, I have made clear exactly what was done by others and what I have contributed myself;\\
7. Either none of this work has been published before submission, or parts of this work have been published by :\\
\\
Stefan Collier\\
April 2016
}
\tableofcontents
\listoffigures
\listoftables

\mainmatter
%% ----------------------------------------------------------------
%\include{Introduction}
%\include{Conclusions}
 %% ----------------------------------------------------------------
%% Progress.tex
%% ---------------------------------------------------------------- 
\documentclass{ecsprogress}    % Use the progress Style
\graphicspath{{../figs/}}   % Location of your graphics files
    \usepackage{natbib}            % Use Natbib style for the refs.
\hypersetup{colorlinks=true}   % Set to false for black/white printing
\input{Definitions}            % Include your abbreviations



\usepackage{enumitem}% http://ctan.org/pkg/enumitem
\usepackage{multirow}
\usepackage{float}
\usepackage{amsmath}
\usepackage{multicol}
\usepackage{amssymb}
\usepackage[normalem]{ulem}
\useunder{\uline}{\ul}{}
\usepackage{wrapfig}


\usepackage[table,xcdraw]{xcolor}


%% ----------------------------------------------------------------
\begin{document}
\frontmatter
\title      {Heterogeneous Agent-based Model for Supermarket Competition}
\authors    {\texorpdfstring
             {\href{mailto:sc22g13@ecs.soton.ac.uk}{Stefan J. Collier}}
             {Stefan J. Collier}
            }
\addresses  {\groupname\\\deptname\\\univname}
\date       {\today}
\subject    {}
\keywords   {}
\supervisor {Dr. Maria Polukarov}
\examiner   {Professor Sheng Chen}

\maketitle
\begin{abstract}
This project aim was to model and analyse the effects of competitive pricing behaviors of grocery retailers on the British market. 

This was achieved by creating a multi-agent model, containing retailer and consumer agents. The heterogeneous crowd of retailers employs either a uniform pricing strategy or a ‘local price flexing’ strategy. The actions of these retailers are chosen by predicting the profit of each action, using a perceptron. Following on from the consideration of different economic models, a discrete model was developed so that software agents have a discrete environment to operate within. Within the model, it has been observed how supermarkets with differing behaviors affect a heterogeneous crowd of consumer agents. The model was implemented in Java with Python used to evaluate the results. 

The simulation displays good acceptance with real grocery market behavior, i.e. captures the performance of British retailers thus can be used to determine the impact of changes in their behavior on their competitors and consumers.Furthermore it can be used to provide insight into sustainability of volatile pricing strategies, providing a useful insight in volatility of British supermarket retail industry. 
\end{abstract}
\acknowledgements{
I would like to express my sincere gratitude to Dr Maria Polukarov for her guidance and support which provided me the freedom to take this research in the direction of my interest.\\
\\
I would also like to thank my family and friends for their encouragement and support. To those who quietly listened to my software complaints. To those who worked throughout the nights with me. To those who helped me write what I couldn't say. I cannot thank you enough.
}

\declaration{
I, Stefan Collier, declare that this dissertation and the work presented in it are my own and has been generated by me as the result of my own original research.\\
I confirm that:\\
1. This work was done wholly or mainly while in candidature for a degree at this University;\\
2. Where any part of this dissertation has previously been submitted for any other qualification at this University or any other institution, this has been clearly stated;\\
3. Where I have consulted the published work of others, this is always clearly attributed;\\
4. Where I have quoted from the work of others, the source is always given. With the exception of such quotations, this dissertation is entirely my own work;\\
5. I have acknowledged all main sources of help;\\
6. Where the thesis is based on work done by myself jointly with others, I have made clear exactly what was done by others and what I have contributed myself;\\
7. Either none of this work has been published before submission, or parts of this work have been published by :\\
\\
Stefan Collier\\
April 2016
}
\tableofcontents
\listoffigures
\listoftables

\mainmatter
%% ----------------------------------------------------------------
%\include{Introduction}
%\include{Conclusions}
\include{chapters/1Project/main}
\include{chapters/2Lit/main}
\include{chapters/3Design/HighLevel}
\include{chapters/3Design/InDepth}
\include{chapters/4Impl/main}

\include{chapters/5Experiments/1/main}
\include{chapters/5Experiments/2/main}
\include{chapters/5Experiments/3/main}
\include{chapters/5Experiments/4/main}

\include{chapters/6Conclusion/main}

\appendix
\include{appendix/AppendixB}
\include{appendix/D/main}
\include{appendix/AppendixC}

\backmatter
\bibliographystyle{ecs}
\bibliography{ECS}
\end{document}
%% ----------------------------------------------------------------

 %% ----------------------------------------------------------------
%% Progress.tex
%% ---------------------------------------------------------------- 
\documentclass{ecsprogress}    % Use the progress Style
\graphicspath{{../figs/}}   % Location of your graphics files
    \usepackage{natbib}            % Use Natbib style for the refs.
\hypersetup{colorlinks=true}   % Set to false for black/white printing
\input{Definitions}            % Include your abbreviations



\usepackage{enumitem}% http://ctan.org/pkg/enumitem
\usepackage{multirow}
\usepackage{float}
\usepackage{amsmath}
\usepackage{multicol}
\usepackage{amssymb}
\usepackage[normalem]{ulem}
\useunder{\uline}{\ul}{}
\usepackage{wrapfig}


\usepackage[table,xcdraw]{xcolor}


%% ----------------------------------------------------------------
\begin{document}
\frontmatter
\title      {Heterogeneous Agent-based Model for Supermarket Competition}
\authors    {\texorpdfstring
             {\href{mailto:sc22g13@ecs.soton.ac.uk}{Stefan J. Collier}}
             {Stefan J. Collier}
            }
\addresses  {\groupname\\\deptname\\\univname}
\date       {\today}
\subject    {}
\keywords   {}
\supervisor {Dr. Maria Polukarov}
\examiner   {Professor Sheng Chen}

\maketitle
\begin{abstract}
This project aim was to model and analyse the effects of competitive pricing behaviors of grocery retailers on the British market. 

This was achieved by creating a multi-agent model, containing retailer and consumer agents. The heterogeneous crowd of retailers employs either a uniform pricing strategy or a ‘local price flexing’ strategy. The actions of these retailers are chosen by predicting the profit of each action, using a perceptron. Following on from the consideration of different economic models, a discrete model was developed so that software agents have a discrete environment to operate within. Within the model, it has been observed how supermarkets with differing behaviors affect a heterogeneous crowd of consumer agents. The model was implemented in Java with Python used to evaluate the results. 

The simulation displays good acceptance with real grocery market behavior, i.e. captures the performance of British retailers thus can be used to determine the impact of changes in their behavior on their competitors and consumers.Furthermore it can be used to provide insight into sustainability of volatile pricing strategies, providing a useful insight in volatility of British supermarket retail industry. 
\end{abstract}
\acknowledgements{
I would like to express my sincere gratitude to Dr Maria Polukarov for her guidance and support which provided me the freedom to take this research in the direction of my interest.\\
\\
I would also like to thank my family and friends for their encouragement and support. To those who quietly listened to my software complaints. To those who worked throughout the nights with me. To those who helped me write what I couldn't say. I cannot thank you enough.
}

\declaration{
I, Stefan Collier, declare that this dissertation and the work presented in it are my own and has been generated by me as the result of my own original research.\\
I confirm that:\\
1. This work was done wholly or mainly while in candidature for a degree at this University;\\
2. Where any part of this dissertation has previously been submitted for any other qualification at this University or any other institution, this has been clearly stated;\\
3. Where I have consulted the published work of others, this is always clearly attributed;\\
4. Where I have quoted from the work of others, the source is always given. With the exception of such quotations, this dissertation is entirely my own work;\\
5. I have acknowledged all main sources of help;\\
6. Where the thesis is based on work done by myself jointly with others, I have made clear exactly what was done by others and what I have contributed myself;\\
7. Either none of this work has been published before submission, or parts of this work have been published by :\\
\\
Stefan Collier\\
April 2016
}
\tableofcontents
\listoffigures
\listoftables

\mainmatter
%% ----------------------------------------------------------------
%\include{Introduction}
%\include{Conclusions}
\include{chapters/1Project/main}
\include{chapters/2Lit/main}
\include{chapters/3Design/HighLevel}
\include{chapters/3Design/InDepth}
\include{chapters/4Impl/main}

\include{chapters/5Experiments/1/main}
\include{chapters/5Experiments/2/main}
\include{chapters/5Experiments/3/main}
\include{chapters/5Experiments/4/main}

\include{chapters/6Conclusion/main}

\appendix
\include{appendix/AppendixB}
\include{appendix/D/main}
\include{appendix/AppendixC}

\backmatter
\bibliographystyle{ecs}
\bibliography{ECS}
\end{document}
%% ----------------------------------------------------------------

\include{chapters/3Design/HighLevel}
\include{chapters/3Design/InDepth}
 %% ----------------------------------------------------------------
%% Progress.tex
%% ---------------------------------------------------------------- 
\documentclass{ecsprogress}    % Use the progress Style
\graphicspath{{../figs/}}   % Location of your graphics files
    \usepackage{natbib}            % Use Natbib style for the refs.
\hypersetup{colorlinks=true}   % Set to false for black/white printing
\input{Definitions}            % Include your abbreviations



\usepackage{enumitem}% http://ctan.org/pkg/enumitem
\usepackage{multirow}
\usepackage{float}
\usepackage{amsmath}
\usepackage{multicol}
\usepackage{amssymb}
\usepackage[normalem]{ulem}
\useunder{\uline}{\ul}{}
\usepackage{wrapfig}


\usepackage[table,xcdraw]{xcolor}


%% ----------------------------------------------------------------
\begin{document}
\frontmatter
\title      {Heterogeneous Agent-based Model for Supermarket Competition}
\authors    {\texorpdfstring
             {\href{mailto:sc22g13@ecs.soton.ac.uk}{Stefan J. Collier}}
             {Stefan J. Collier}
            }
\addresses  {\groupname\\\deptname\\\univname}
\date       {\today}
\subject    {}
\keywords   {}
\supervisor {Dr. Maria Polukarov}
\examiner   {Professor Sheng Chen}

\maketitle
\begin{abstract}
This project aim was to model and analyse the effects of competitive pricing behaviors of grocery retailers on the British market. 

This was achieved by creating a multi-agent model, containing retailer and consumer agents. The heterogeneous crowd of retailers employs either a uniform pricing strategy or a ‘local price flexing’ strategy. The actions of these retailers are chosen by predicting the profit of each action, using a perceptron. Following on from the consideration of different economic models, a discrete model was developed so that software agents have a discrete environment to operate within. Within the model, it has been observed how supermarkets with differing behaviors affect a heterogeneous crowd of consumer agents. The model was implemented in Java with Python used to evaluate the results. 

The simulation displays good acceptance with real grocery market behavior, i.e. captures the performance of British retailers thus can be used to determine the impact of changes in their behavior on their competitors and consumers.Furthermore it can be used to provide insight into sustainability of volatile pricing strategies, providing a useful insight in volatility of British supermarket retail industry. 
\end{abstract}
\acknowledgements{
I would like to express my sincere gratitude to Dr Maria Polukarov for her guidance and support which provided me the freedom to take this research in the direction of my interest.\\
\\
I would also like to thank my family and friends for their encouragement and support. To those who quietly listened to my software complaints. To those who worked throughout the nights with me. To those who helped me write what I couldn't say. I cannot thank you enough.
}

\declaration{
I, Stefan Collier, declare that this dissertation and the work presented in it are my own and has been generated by me as the result of my own original research.\\
I confirm that:\\
1. This work was done wholly or mainly while in candidature for a degree at this University;\\
2. Where any part of this dissertation has previously been submitted for any other qualification at this University or any other institution, this has been clearly stated;\\
3. Where I have consulted the published work of others, this is always clearly attributed;\\
4. Where I have quoted from the work of others, the source is always given. With the exception of such quotations, this dissertation is entirely my own work;\\
5. I have acknowledged all main sources of help;\\
6. Where the thesis is based on work done by myself jointly with others, I have made clear exactly what was done by others and what I have contributed myself;\\
7. Either none of this work has been published before submission, or parts of this work have been published by :\\
\\
Stefan Collier\\
April 2016
}
\tableofcontents
\listoffigures
\listoftables

\mainmatter
%% ----------------------------------------------------------------
%\include{Introduction}
%\include{Conclusions}
\include{chapters/1Project/main}
\include{chapters/2Lit/main}
\include{chapters/3Design/HighLevel}
\include{chapters/3Design/InDepth}
\include{chapters/4Impl/main}

\include{chapters/5Experiments/1/main}
\include{chapters/5Experiments/2/main}
\include{chapters/5Experiments/3/main}
\include{chapters/5Experiments/4/main}

\include{chapters/6Conclusion/main}

\appendix
\include{appendix/AppendixB}
\include{appendix/D/main}
\include{appendix/AppendixC}

\backmatter
\bibliographystyle{ecs}
\bibliography{ECS}
\end{document}
%% ----------------------------------------------------------------


 %% ----------------------------------------------------------------
%% Progress.tex
%% ---------------------------------------------------------------- 
\documentclass{ecsprogress}    % Use the progress Style
\graphicspath{{../figs/}}   % Location of your graphics files
    \usepackage{natbib}            % Use Natbib style for the refs.
\hypersetup{colorlinks=true}   % Set to false for black/white printing
\input{Definitions}            % Include your abbreviations



\usepackage{enumitem}% http://ctan.org/pkg/enumitem
\usepackage{multirow}
\usepackage{float}
\usepackage{amsmath}
\usepackage{multicol}
\usepackage{amssymb}
\usepackage[normalem]{ulem}
\useunder{\uline}{\ul}{}
\usepackage{wrapfig}


\usepackage[table,xcdraw]{xcolor}


%% ----------------------------------------------------------------
\begin{document}
\frontmatter
\title      {Heterogeneous Agent-based Model for Supermarket Competition}
\authors    {\texorpdfstring
             {\href{mailto:sc22g13@ecs.soton.ac.uk}{Stefan J. Collier}}
             {Stefan J. Collier}
            }
\addresses  {\groupname\\\deptname\\\univname}
\date       {\today}
\subject    {}
\keywords   {}
\supervisor {Dr. Maria Polukarov}
\examiner   {Professor Sheng Chen}

\maketitle
\begin{abstract}
This project aim was to model and analyse the effects of competitive pricing behaviors of grocery retailers on the British market. 

This was achieved by creating a multi-agent model, containing retailer and consumer agents. The heterogeneous crowd of retailers employs either a uniform pricing strategy or a ‘local price flexing’ strategy. The actions of these retailers are chosen by predicting the profit of each action, using a perceptron. Following on from the consideration of different economic models, a discrete model was developed so that software agents have a discrete environment to operate within. Within the model, it has been observed how supermarkets with differing behaviors affect a heterogeneous crowd of consumer agents. The model was implemented in Java with Python used to evaluate the results. 

The simulation displays good acceptance with real grocery market behavior, i.e. captures the performance of British retailers thus can be used to determine the impact of changes in their behavior on their competitors and consumers.Furthermore it can be used to provide insight into sustainability of volatile pricing strategies, providing a useful insight in volatility of British supermarket retail industry. 
\end{abstract}
\acknowledgements{
I would like to express my sincere gratitude to Dr Maria Polukarov for her guidance and support which provided me the freedom to take this research in the direction of my interest.\\
\\
I would also like to thank my family and friends for their encouragement and support. To those who quietly listened to my software complaints. To those who worked throughout the nights with me. To those who helped me write what I couldn't say. I cannot thank you enough.
}

\declaration{
I, Stefan Collier, declare that this dissertation and the work presented in it are my own and has been generated by me as the result of my own original research.\\
I confirm that:\\
1. This work was done wholly or mainly while in candidature for a degree at this University;\\
2. Where any part of this dissertation has previously been submitted for any other qualification at this University or any other institution, this has been clearly stated;\\
3. Where I have consulted the published work of others, this is always clearly attributed;\\
4. Where I have quoted from the work of others, the source is always given. With the exception of such quotations, this dissertation is entirely my own work;\\
5. I have acknowledged all main sources of help;\\
6. Where the thesis is based on work done by myself jointly with others, I have made clear exactly what was done by others and what I have contributed myself;\\
7. Either none of this work has been published before submission, or parts of this work have been published by :\\
\\
Stefan Collier\\
April 2016
}
\tableofcontents
\listoffigures
\listoftables

\mainmatter
%% ----------------------------------------------------------------
%\include{Introduction}
%\include{Conclusions}
\include{chapters/1Project/main}
\include{chapters/2Lit/main}
\include{chapters/3Design/HighLevel}
\include{chapters/3Design/InDepth}
\include{chapters/4Impl/main}

\include{chapters/5Experiments/1/main}
\include{chapters/5Experiments/2/main}
\include{chapters/5Experiments/3/main}
\include{chapters/5Experiments/4/main}

\include{chapters/6Conclusion/main}

\appendix
\include{appendix/AppendixB}
\include{appendix/D/main}
\include{appendix/AppendixC}

\backmatter
\bibliographystyle{ecs}
\bibliography{ECS}
\end{document}
%% ----------------------------------------------------------------

 %% ----------------------------------------------------------------
%% Progress.tex
%% ---------------------------------------------------------------- 
\documentclass{ecsprogress}    % Use the progress Style
\graphicspath{{../figs/}}   % Location of your graphics files
    \usepackage{natbib}            % Use Natbib style for the refs.
\hypersetup{colorlinks=true}   % Set to false for black/white printing
\input{Definitions}            % Include your abbreviations



\usepackage{enumitem}% http://ctan.org/pkg/enumitem
\usepackage{multirow}
\usepackage{float}
\usepackage{amsmath}
\usepackage{multicol}
\usepackage{amssymb}
\usepackage[normalem]{ulem}
\useunder{\uline}{\ul}{}
\usepackage{wrapfig}


\usepackage[table,xcdraw]{xcolor}


%% ----------------------------------------------------------------
\begin{document}
\frontmatter
\title      {Heterogeneous Agent-based Model for Supermarket Competition}
\authors    {\texorpdfstring
             {\href{mailto:sc22g13@ecs.soton.ac.uk}{Stefan J. Collier}}
             {Stefan J. Collier}
            }
\addresses  {\groupname\\\deptname\\\univname}
\date       {\today}
\subject    {}
\keywords   {}
\supervisor {Dr. Maria Polukarov}
\examiner   {Professor Sheng Chen}

\maketitle
\begin{abstract}
This project aim was to model and analyse the effects of competitive pricing behaviors of grocery retailers on the British market. 

This was achieved by creating a multi-agent model, containing retailer and consumer agents. The heterogeneous crowd of retailers employs either a uniform pricing strategy or a ‘local price flexing’ strategy. The actions of these retailers are chosen by predicting the profit of each action, using a perceptron. Following on from the consideration of different economic models, a discrete model was developed so that software agents have a discrete environment to operate within. Within the model, it has been observed how supermarkets with differing behaviors affect a heterogeneous crowd of consumer agents. The model was implemented in Java with Python used to evaluate the results. 

The simulation displays good acceptance with real grocery market behavior, i.e. captures the performance of British retailers thus can be used to determine the impact of changes in their behavior on their competitors and consumers.Furthermore it can be used to provide insight into sustainability of volatile pricing strategies, providing a useful insight in volatility of British supermarket retail industry. 
\end{abstract}
\acknowledgements{
I would like to express my sincere gratitude to Dr Maria Polukarov for her guidance and support which provided me the freedom to take this research in the direction of my interest.\\
\\
I would also like to thank my family and friends for their encouragement and support. To those who quietly listened to my software complaints. To those who worked throughout the nights with me. To those who helped me write what I couldn't say. I cannot thank you enough.
}

\declaration{
I, Stefan Collier, declare that this dissertation and the work presented in it are my own and has been generated by me as the result of my own original research.\\
I confirm that:\\
1. This work was done wholly or mainly while in candidature for a degree at this University;\\
2. Where any part of this dissertation has previously been submitted for any other qualification at this University or any other institution, this has been clearly stated;\\
3. Where I have consulted the published work of others, this is always clearly attributed;\\
4. Where I have quoted from the work of others, the source is always given. With the exception of such quotations, this dissertation is entirely my own work;\\
5. I have acknowledged all main sources of help;\\
6. Where the thesis is based on work done by myself jointly with others, I have made clear exactly what was done by others and what I have contributed myself;\\
7. Either none of this work has been published before submission, or parts of this work have been published by :\\
\\
Stefan Collier\\
April 2016
}
\tableofcontents
\listoffigures
\listoftables

\mainmatter
%% ----------------------------------------------------------------
%\include{Introduction}
%\include{Conclusions}
\include{chapters/1Project/main}
\include{chapters/2Lit/main}
\include{chapters/3Design/HighLevel}
\include{chapters/3Design/InDepth}
\include{chapters/4Impl/main}

\include{chapters/5Experiments/1/main}
\include{chapters/5Experiments/2/main}
\include{chapters/5Experiments/3/main}
\include{chapters/5Experiments/4/main}

\include{chapters/6Conclusion/main}

\appendix
\include{appendix/AppendixB}
\include{appendix/D/main}
\include{appendix/AppendixC}

\backmatter
\bibliographystyle{ecs}
\bibliography{ECS}
\end{document}
%% ----------------------------------------------------------------

 %% ----------------------------------------------------------------
%% Progress.tex
%% ---------------------------------------------------------------- 
\documentclass{ecsprogress}    % Use the progress Style
\graphicspath{{../figs/}}   % Location of your graphics files
    \usepackage{natbib}            % Use Natbib style for the refs.
\hypersetup{colorlinks=true}   % Set to false for black/white printing
\input{Definitions}            % Include your abbreviations



\usepackage{enumitem}% http://ctan.org/pkg/enumitem
\usepackage{multirow}
\usepackage{float}
\usepackage{amsmath}
\usepackage{multicol}
\usepackage{amssymb}
\usepackage[normalem]{ulem}
\useunder{\uline}{\ul}{}
\usepackage{wrapfig}


\usepackage[table,xcdraw]{xcolor}


%% ----------------------------------------------------------------
\begin{document}
\frontmatter
\title      {Heterogeneous Agent-based Model for Supermarket Competition}
\authors    {\texorpdfstring
             {\href{mailto:sc22g13@ecs.soton.ac.uk}{Stefan J. Collier}}
             {Stefan J. Collier}
            }
\addresses  {\groupname\\\deptname\\\univname}
\date       {\today}
\subject    {}
\keywords   {}
\supervisor {Dr. Maria Polukarov}
\examiner   {Professor Sheng Chen}

\maketitle
\begin{abstract}
This project aim was to model and analyse the effects of competitive pricing behaviors of grocery retailers on the British market. 

This was achieved by creating a multi-agent model, containing retailer and consumer agents. The heterogeneous crowd of retailers employs either a uniform pricing strategy or a ‘local price flexing’ strategy. The actions of these retailers are chosen by predicting the profit of each action, using a perceptron. Following on from the consideration of different economic models, a discrete model was developed so that software agents have a discrete environment to operate within. Within the model, it has been observed how supermarkets with differing behaviors affect a heterogeneous crowd of consumer agents. The model was implemented in Java with Python used to evaluate the results. 

The simulation displays good acceptance with real grocery market behavior, i.e. captures the performance of British retailers thus can be used to determine the impact of changes in their behavior on their competitors and consumers.Furthermore it can be used to provide insight into sustainability of volatile pricing strategies, providing a useful insight in volatility of British supermarket retail industry. 
\end{abstract}
\acknowledgements{
I would like to express my sincere gratitude to Dr Maria Polukarov for her guidance and support which provided me the freedom to take this research in the direction of my interest.\\
\\
I would also like to thank my family and friends for their encouragement and support. To those who quietly listened to my software complaints. To those who worked throughout the nights with me. To those who helped me write what I couldn't say. I cannot thank you enough.
}

\declaration{
I, Stefan Collier, declare that this dissertation and the work presented in it are my own and has been generated by me as the result of my own original research.\\
I confirm that:\\
1. This work was done wholly or mainly while in candidature for a degree at this University;\\
2. Where any part of this dissertation has previously been submitted for any other qualification at this University or any other institution, this has been clearly stated;\\
3. Where I have consulted the published work of others, this is always clearly attributed;\\
4. Where I have quoted from the work of others, the source is always given. With the exception of such quotations, this dissertation is entirely my own work;\\
5. I have acknowledged all main sources of help;\\
6. Where the thesis is based on work done by myself jointly with others, I have made clear exactly what was done by others and what I have contributed myself;\\
7. Either none of this work has been published before submission, or parts of this work have been published by :\\
\\
Stefan Collier\\
April 2016
}
\tableofcontents
\listoffigures
\listoftables

\mainmatter
%% ----------------------------------------------------------------
%\include{Introduction}
%\include{Conclusions}
\include{chapters/1Project/main}
\include{chapters/2Lit/main}
\include{chapters/3Design/HighLevel}
\include{chapters/3Design/InDepth}
\include{chapters/4Impl/main}

\include{chapters/5Experiments/1/main}
\include{chapters/5Experiments/2/main}
\include{chapters/5Experiments/3/main}
\include{chapters/5Experiments/4/main}

\include{chapters/6Conclusion/main}

\appendix
\include{appendix/AppendixB}
\include{appendix/D/main}
\include{appendix/AppendixC}

\backmatter
\bibliographystyle{ecs}
\bibliography{ECS}
\end{document}
%% ----------------------------------------------------------------

 %% ----------------------------------------------------------------
%% Progress.tex
%% ---------------------------------------------------------------- 
\documentclass{ecsprogress}    % Use the progress Style
\graphicspath{{../figs/}}   % Location of your graphics files
    \usepackage{natbib}            % Use Natbib style for the refs.
\hypersetup{colorlinks=true}   % Set to false for black/white printing
\input{Definitions}            % Include your abbreviations



\usepackage{enumitem}% http://ctan.org/pkg/enumitem
\usepackage{multirow}
\usepackage{float}
\usepackage{amsmath}
\usepackage{multicol}
\usepackage{amssymb}
\usepackage[normalem]{ulem}
\useunder{\uline}{\ul}{}
\usepackage{wrapfig}


\usepackage[table,xcdraw]{xcolor}


%% ----------------------------------------------------------------
\begin{document}
\frontmatter
\title      {Heterogeneous Agent-based Model for Supermarket Competition}
\authors    {\texorpdfstring
             {\href{mailto:sc22g13@ecs.soton.ac.uk}{Stefan J. Collier}}
             {Stefan J. Collier}
            }
\addresses  {\groupname\\\deptname\\\univname}
\date       {\today}
\subject    {}
\keywords   {}
\supervisor {Dr. Maria Polukarov}
\examiner   {Professor Sheng Chen}

\maketitle
\begin{abstract}
This project aim was to model and analyse the effects of competitive pricing behaviors of grocery retailers on the British market. 

This was achieved by creating a multi-agent model, containing retailer and consumer agents. The heterogeneous crowd of retailers employs either a uniform pricing strategy or a ‘local price flexing’ strategy. The actions of these retailers are chosen by predicting the profit of each action, using a perceptron. Following on from the consideration of different economic models, a discrete model was developed so that software agents have a discrete environment to operate within. Within the model, it has been observed how supermarkets with differing behaviors affect a heterogeneous crowd of consumer agents. The model was implemented in Java with Python used to evaluate the results. 

The simulation displays good acceptance with real grocery market behavior, i.e. captures the performance of British retailers thus can be used to determine the impact of changes in their behavior on their competitors and consumers.Furthermore it can be used to provide insight into sustainability of volatile pricing strategies, providing a useful insight in volatility of British supermarket retail industry. 
\end{abstract}
\acknowledgements{
I would like to express my sincere gratitude to Dr Maria Polukarov for her guidance and support which provided me the freedom to take this research in the direction of my interest.\\
\\
I would also like to thank my family and friends for their encouragement and support. To those who quietly listened to my software complaints. To those who worked throughout the nights with me. To those who helped me write what I couldn't say. I cannot thank you enough.
}

\declaration{
I, Stefan Collier, declare that this dissertation and the work presented in it are my own and has been generated by me as the result of my own original research.\\
I confirm that:\\
1. This work was done wholly or mainly while in candidature for a degree at this University;\\
2. Where any part of this dissertation has previously been submitted for any other qualification at this University or any other institution, this has been clearly stated;\\
3. Where I have consulted the published work of others, this is always clearly attributed;\\
4. Where I have quoted from the work of others, the source is always given. With the exception of such quotations, this dissertation is entirely my own work;\\
5. I have acknowledged all main sources of help;\\
6. Where the thesis is based on work done by myself jointly with others, I have made clear exactly what was done by others and what I have contributed myself;\\
7. Either none of this work has been published before submission, or parts of this work have been published by :\\
\\
Stefan Collier\\
April 2016
}
\tableofcontents
\listoffigures
\listoftables

\mainmatter
%% ----------------------------------------------------------------
%\include{Introduction}
%\include{Conclusions}
\include{chapters/1Project/main}
\include{chapters/2Lit/main}
\include{chapters/3Design/HighLevel}
\include{chapters/3Design/InDepth}
\include{chapters/4Impl/main}

\include{chapters/5Experiments/1/main}
\include{chapters/5Experiments/2/main}
\include{chapters/5Experiments/3/main}
\include{chapters/5Experiments/4/main}

\include{chapters/6Conclusion/main}

\appendix
\include{appendix/AppendixB}
\include{appendix/D/main}
\include{appendix/AppendixC}

\backmatter
\bibliographystyle{ecs}
\bibliography{ECS}
\end{document}
%% ----------------------------------------------------------------


 %% ----------------------------------------------------------------
%% Progress.tex
%% ---------------------------------------------------------------- 
\documentclass{ecsprogress}    % Use the progress Style
\graphicspath{{../figs/}}   % Location of your graphics files
    \usepackage{natbib}            % Use Natbib style for the refs.
\hypersetup{colorlinks=true}   % Set to false for black/white printing
\input{Definitions}            % Include your abbreviations



\usepackage{enumitem}% http://ctan.org/pkg/enumitem
\usepackage{multirow}
\usepackage{float}
\usepackage{amsmath}
\usepackage{multicol}
\usepackage{amssymb}
\usepackage[normalem]{ulem}
\useunder{\uline}{\ul}{}
\usepackage{wrapfig}


\usepackage[table,xcdraw]{xcolor}


%% ----------------------------------------------------------------
\begin{document}
\frontmatter
\title      {Heterogeneous Agent-based Model for Supermarket Competition}
\authors    {\texorpdfstring
             {\href{mailto:sc22g13@ecs.soton.ac.uk}{Stefan J. Collier}}
             {Stefan J. Collier}
            }
\addresses  {\groupname\\\deptname\\\univname}
\date       {\today}
\subject    {}
\keywords   {}
\supervisor {Dr. Maria Polukarov}
\examiner   {Professor Sheng Chen}

\maketitle
\begin{abstract}
This project aim was to model and analyse the effects of competitive pricing behaviors of grocery retailers on the British market. 

This was achieved by creating a multi-agent model, containing retailer and consumer agents. The heterogeneous crowd of retailers employs either a uniform pricing strategy or a ‘local price flexing’ strategy. The actions of these retailers are chosen by predicting the profit of each action, using a perceptron. Following on from the consideration of different economic models, a discrete model was developed so that software agents have a discrete environment to operate within. Within the model, it has been observed how supermarkets with differing behaviors affect a heterogeneous crowd of consumer agents. The model was implemented in Java with Python used to evaluate the results. 

The simulation displays good acceptance with real grocery market behavior, i.e. captures the performance of British retailers thus can be used to determine the impact of changes in their behavior on their competitors and consumers.Furthermore it can be used to provide insight into sustainability of volatile pricing strategies, providing a useful insight in volatility of British supermarket retail industry. 
\end{abstract}
\acknowledgements{
I would like to express my sincere gratitude to Dr Maria Polukarov for her guidance and support which provided me the freedom to take this research in the direction of my interest.\\
\\
I would also like to thank my family and friends for their encouragement and support. To those who quietly listened to my software complaints. To those who worked throughout the nights with me. To those who helped me write what I couldn't say. I cannot thank you enough.
}

\declaration{
I, Stefan Collier, declare that this dissertation and the work presented in it are my own and has been generated by me as the result of my own original research.\\
I confirm that:\\
1. This work was done wholly or mainly while in candidature for a degree at this University;\\
2. Where any part of this dissertation has previously been submitted for any other qualification at this University or any other institution, this has been clearly stated;\\
3. Where I have consulted the published work of others, this is always clearly attributed;\\
4. Where I have quoted from the work of others, the source is always given. With the exception of such quotations, this dissertation is entirely my own work;\\
5. I have acknowledged all main sources of help;\\
6. Where the thesis is based on work done by myself jointly with others, I have made clear exactly what was done by others and what I have contributed myself;\\
7. Either none of this work has been published before submission, or parts of this work have been published by :\\
\\
Stefan Collier\\
April 2016
}
\tableofcontents
\listoffigures
\listoftables

\mainmatter
%% ----------------------------------------------------------------
%\include{Introduction}
%\include{Conclusions}
\include{chapters/1Project/main}
\include{chapters/2Lit/main}
\include{chapters/3Design/HighLevel}
\include{chapters/3Design/InDepth}
\include{chapters/4Impl/main}

\include{chapters/5Experiments/1/main}
\include{chapters/5Experiments/2/main}
\include{chapters/5Experiments/3/main}
\include{chapters/5Experiments/4/main}

\include{chapters/6Conclusion/main}

\appendix
\include{appendix/AppendixB}
\include{appendix/D/main}
\include{appendix/AppendixC}

\backmatter
\bibliographystyle{ecs}
\bibliography{ECS}
\end{document}
%% ----------------------------------------------------------------


\appendix
\include{appendix/AppendixB}
 %% ----------------------------------------------------------------
%% Progress.tex
%% ---------------------------------------------------------------- 
\documentclass{ecsprogress}    % Use the progress Style
\graphicspath{{../figs/}}   % Location of your graphics files
    \usepackage{natbib}            % Use Natbib style for the refs.
\hypersetup{colorlinks=true}   % Set to false for black/white printing
\input{Definitions}            % Include your abbreviations



\usepackage{enumitem}% http://ctan.org/pkg/enumitem
\usepackage{multirow}
\usepackage{float}
\usepackage{amsmath}
\usepackage{multicol}
\usepackage{amssymb}
\usepackage[normalem]{ulem}
\useunder{\uline}{\ul}{}
\usepackage{wrapfig}


\usepackage[table,xcdraw]{xcolor}


%% ----------------------------------------------------------------
\begin{document}
\frontmatter
\title      {Heterogeneous Agent-based Model for Supermarket Competition}
\authors    {\texorpdfstring
             {\href{mailto:sc22g13@ecs.soton.ac.uk}{Stefan J. Collier}}
             {Stefan J. Collier}
            }
\addresses  {\groupname\\\deptname\\\univname}
\date       {\today}
\subject    {}
\keywords   {}
\supervisor {Dr. Maria Polukarov}
\examiner   {Professor Sheng Chen}

\maketitle
\begin{abstract}
This project aim was to model and analyse the effects of competitive pricing behaviors of grocery retailers on the British market. 

This was achieved by creating a multi-agent model, containing retailer and consumer agents. The heterogeneous crowd of retailers employs either a uniform pricing strategy or a ‘local price flexing’ strategy. The actions of these retailers are chosen by predicting the profit of each action, using a perceptron. Following on from the consideration of different economic models, a discrete model was developed so that software agents have a discrete environment to operate within. Within the model, it has been observed how supermarkets with differing behaviors affect a heterogeneous crowd of consumer agents. The model was implemented in Java with Python used to evaluate the results. 

The simulation displays good acceptance with real grocery market behavior, i.e. captures the performance of British retailers thus can be used to determine the impact of changes in their behavior on their competitors and consumers.Furthermore it can be used to provide insight into sustainability of volatile pricing strategies, providing a useful insight in volatility of British supermarket retail industry. 
\end{abstract}
\acknowledgements{
I would like to express my sincere gratitude to Dr Maria Polukarov for her guidance and support which provided me the freedom to take this research in the direction of my interest.\\
\\
I would also like to thank my family and friends for their encouragement and support. To those who quietly listened to my software complaints. To those who worked throughout the nights with me. To those who helped me write what I couldn't say. I cannot thank you enough.
}

\declaration{
I, Stefan Collier, declare that this dissertation and the work presented in it are my own and has been generated by me as the result of my own original research.\\
I confirm that:\\
1. This work was done wholly or mainly while in candidature for a degree at this University;\\
2. Where any part of this dissertation has previously been submitted for any other qualification at this University or any other institution, this has been clearly stated;\\
3. Where I have consulted the published work of others, this is always clearly attributed;\\
4. Where I have quoted from the work of others, the source is always given. With the exception of such quotations, this dissertation is entirely my own work;\\
5. I have acknowledged all main sources of help;\\
6. Where the thesis is based on work done by myself jointly with others, I have made clear exactly what was done by others and what I have contributed myself;\\
7. Either none of this work has been published before submission, or parts of this work have been published by :\\
\\
Stefan Collier\\
April 2016
}
\tableofcontents
\listoffigures
\listoftables

\mainmatter
%% ----------------------------------------------------------------
%\include{Introduction}
%\include{Conclusions}
\include{chapters/1Project/main}
\include{chapters/2Lit/main}
\include{chapters/3Design/HighLevel}
\include{chapters/3Design/InDepth}
\include{chapters/4Impl/main}

\include{chapters/5Experiments/1/main}
\include{chapters/5Experiments/2/main}
\include{chapters/5Experiments/3/main}
\include{chapters/5Experiments/4/main}

\include{chapters/6Conclusion/main}

\appendix
\include{appendix/AppendixB}
\include{appendix/D/main}
\include{appendix/AppendixC}

\backmatter
\bibliographystyle{ecs}
\bibliography{ECS}
\end{document}
%% ----------------------------------------------------------------

\include{appendix/AppendixC}

\backmatter
\bibliographystyle{ecs}
\bibliography{ECS}
\end{document}
%% ----------------------------------------------------------------

 %% ----------------------------------------------------------------
%% Progress.tex
%% ---------------------------------------------------------------- 
\documentclass{ecsprogress}    % Use the progress Style
\graphicspath{{../figs/}}   % Location of your graphics files
    \usepackage{natbib}            % Use Natbib style for the refs.
\hypersetup{colorlinks=true}   % Set to false for black/white printing
\input{Definitions}            % Include your abbreviations



\usepackage{enumitem}% http://ctan.org/pkg/enumitem
\usepackage{multirow}
\usepackage{float}
\usepackage{amsmath}
\usepackage{multicol}
\usepackage{amssymb}
\usepackage[normalem]{ulem}
\useunder{\uline}{\ul}{}
\usepackage{wrapfig}


\usepackage[table,xcdraw]{xcolor}


%% ----------------------------------------------------------------
\begin{document}
\frontmatter
\title      {Heterogeneous Agent-based Model for Supermarket Competition}
\authors    {\texorpdfstring
             {\href{mailto:sc22g13@ecs.soton.ac.uk}{Stefan J. Collier}}
             {Stefan J. Collier}
            }
\addresses  {\groupname\\\deptname\\\univname}
\date       {\today}
\subject    {}
\keywords   {}
\supervisor {Dr. Maria Polukarov}
\examiner   {Professor Sheng Chen}

\maketitle
\begin{abstract}
This project aim was to model and analyse the effects of competitive pricing behaviors of grocery retailers on the British market. 

This was achieved by creating a multi-agent model, containing retailer and consumer agents. The heterogeneous crowd of retailers employs either a uniform pricing strategy or a ‘local price flexing’ strategy. The actions of these retailers are chosen by predicting the profit of each action, using a perceptron. Following on from the consideration of different economic models, a discrete model was developed so that software agents have a discrete environment to operate within. Within the model, it has been observed how supermarkets with differing behaviors affect a heterogeneous crowd of consumer agents. The model was implemented in Java with Python used to evaluate the results. 

The simulation displays good acceptance with real grocery market behavior, i.e. captures the performance of British retailers thus can be used to determine the impact of changes in their behavior on their competitors and consumers.Furthermore it can be used to provide insight into sustainability of volatile pricing strategies, providing a useful insight in volatility of British supermarket retail industry. 
\end{abstract}
\acknowledgements{
I would like to express my sincere gratitude to Dr Maria Polukarov for her guidance and support which provided me the freedom to take this research in the direction of my interest.\\
\\
I would also like to thank my family and friends for their encouragement and support. To those who quietly listened to my software complaints. To those who worked throughout the nights with me. To those who helped me write what I couldn't say. I cannot thank you enough.
}

\declaration{
I, Stefan Collier, declare that this dissertation and the work presented in it are my own and has been generated by me as the result of my own original research.\\
I confirm that:\\
1. This work was done wholly or mainly while in candidature for a degree at this University;\\
2. Where any part of this dissertation has previously been submitted for any other qualification at this University or any other institution, this has been clearly stated;\\
3. Where I have consulted the published work of others, this is always clearly attributed;\\
4. Where I have quoted from the work of others, the source is always given. With the exception of such quotations, this dissertation is entirely my own work;\\
5. I have acknowledged all main sources of help;\\
6. Where the thesis is based on work done by myself jointly with others, I have made clear exactly what was done by others and what I have contributed myself;\\
7. Either none of this work has been published before submission, or parts of this work have been published by :\\
\\
Stefan Collier\\
April 2016
}
\tableofcontents
\listoffigures
\listoftables

\mainmatter
%% ----------------------------------------------------------------
%\include{Introduction}
%\include{Conclusions}
 %% ----------------------------------------------------------------
%% Progress.tex
%% ---------------------------------------------------------------- 
\documentclass{ecsprogress}    % Use the progress Style
\graphicspath{{../figs/}}   % Location of your graphics files
    \usepackage{natbib}            % Use Natbib style for the refs.
\hypersetup{colorlinks=true}   % Set to false for black/white printing
\input{Definitions}            % Include your abbreviations



\usepackage{enumitem}% http://ctan.org/pkg/enumitem
\usepackage{multirow}
\usepackage{float}
\usepackage{amsmath}
\usepackage{multicol}
\usepackage{amssymb}
\usepackage[normalem]{ulem}
\useunder{\uline}{\ul}{}
\usepackage{wrapfig}


\usepackage[table,xcdraw]{xcolor}


%% ----------------------------------------------------------------
\begin{document}
\frontmatter
\title      {Heterogeneous Agent-based Model for Supermarket Competition}
\authors    {\texorpdfstring
             {\href{mailto:sc22g13@ecs.soton.ac.uk}{Stefan J. Collier}}
             {Stefan J. Collier}
            }
\addresses  {\groupname\\\deptname\\\univname}
\date       {\today}
\subject    {}
\keywords   {}
\supervisor {Dr. Maria Polukarov}
\examiner   {Professor Sheng Chen}

\maketitle
\begin{abstract}
This project aim was to model and analyse the effects of competitive pricing behaviors of grocery retailers on the British market. 

This was achieved by creating a multi-agent model, containing retailer and consumer agents. The heterogeneous crowd of retailers employs either a uniform pricing strategy or a ‘local price flexing’ strategy. The actions of these retailers are chosen by predicting the profit of each action, using a perceptron. Following on from the consideration of different economic models, a discrete model was developed so that software agents have a discrete environment to operate within. Within the model, it has been observed how supermarkets with differing behaviors affect a heterogeneous crowd of consumer agents. The model was implemented in Java with Python used to evaluate the results. 

The simulation displays good acceptance with real grocery market behavior, i.e. captures the performance of British retailers thus can be used to determine the impact of changes in their behavior on their competitors and consumers.Furthermore it can be used to provide insight into sustainability of volatile pricing strategies, providing a useful insight in volatility of British supermarket retail industry. 
\end{abstract}
\acknowledgements{
I would like to express my sincere gratitude to Dr Maria Polukarov for her guidance and support which provided me the freedom to take this research in the direction of my interest.\\
\\
I would also like to thank my family and friends for their encouragement and support. To those who quietly listened to my software complaints. To those who worked throughout the nights with me. To those who helped me write what I couldn't say. I cannot thank you enough.
}

\declaration{
I, Stefan Collier, declare that this dissertation and the work presented in it are my own and has been generated by me as the result of my own original research.\\
I confirm that:\\
1. This work was done wholly or mainly while in candidature for a degree at this University;\\
2. Where any part of this dissertation has previously been submitted for any other qualification at this University or any other institution, this has been clearly stated;\\
3. Where I have consulted the published work of others, this is always clearly attributed;\\
4. Where I have quoted from the work of others, the source is always given. With the exception of such quotations, this dissertation is entirely my own work;\\
5. I have acknowledged all main sources of help;\\
6. Where the thesis is based on work done by myself jointly with others, I have made clear exactly what was done by others and what I have contributed myself;\\
7. Either none of this work has been published before submission, or parts of this work have been published by :\\
\\
Stefan Collier\\
April 2016
}
\tableofcontents
\listoffigures
\listoftables

\mainmatter
%% ----------------------------------------------------------------
%\include{Introduction}
%\include{Conclusions}
\include{chapters/1Project/main}
\include{chapters/2Lit/main}
\include{chapters/3Design/HighLevel}
\include{chapters/3Design/InDepth}
\include{chapters/4Impl/main}

\include{chapters/5Experiments/1/main}
\include{chapters/5Experiments/2/main}
\include{chapters/5Experiments/3/main}
\include{chapters/5Experiments/4/main}

\include{chapters/6Conclusion/main}

\appendix
\include{appendix/AppendixB}
\include{appendix/D/main}
\include{appendix/AppendixC}

\backmatter
\bibliographystyle{ecs}
\bibliography{ECS}
\end{document}
%% ----------------------------------------------------------------

 %% ----------------------------------------------------------------
%% Progress.tex
%% ---------------------------------------------------------------- 
\documentclass{ecsprogress}    % Use the progress Style
\graphicspath{{../figs/}}   % Location of your graphics files
    \usepackage{natbib}            % Use Natbib style for the refs.
\hypersetup{colorlinks=true}   % Set to false for black/white printing
\input{Definitions}            % Include your abbreviations



\usepackage{enumitem}% http://ctan.org/pkg/enumitem
\usepackage{multirow}
\usepackage{float}
\usepackage{amsmath}
\usepackage{multicol}
\usepackage{amssymb}
\usepackage[normalem]{ulem}
\useunder{\uline}{\ul}{}
\usepackage{wrapfig}


\usepackage[table,xcdraw]{xcolor}


%% ----------------------------------------------------------------
\begin{document}
\frontmatter
\title      {Heterogeneous Agent-based Model for Supermarket Competition}
\authors    {\texorpdfstring
             {\href{mailto:sc22g13@ecs.soton.ac.uk}{Stefan J. Collier}}
             {Stefan J. Collier}
            }
\addresses  {\groupname\\\deptname\\\univname}
\date       {\today}
\subject    {}
\keywords   {}
\supervisor {Dr. Maria Polukarov}
\examiner   {Professor Sheng Chen}

\maketitle
\begin{abstract}
This project aim was to model and analyse the effects of competitive pricing behaviors of grocery retailers on the British market. 

This was achieved by creating a multi-agent model, containing retailer and consumer agents. The heterogeneous crowd of retailers employs either a uniform pricing strategy or a ‘local price flexing’ strategy. The actions of these retailers are chosen by predicting the profit of each action, using a perceptron. Following on from the consideration of different economic models, a discrete model was developed so that software agents have a discrete environment to operate within. Within the model, it has been observed how supermarkets with differing behaviors affect a heterogeneous crowd of consumer agents. The model was implemented in Java with Python used to evaluate the results. 

The simulation displays good acceptance with real grocery market behavior, i.e. captures the performance of British retailers thus can be used to determine the impact of changes in their behavior on their competitors and consumers.Furthermore it can be used to provide insight into sustainability of volatile pricing strategies, providing a useful insight in volatility of British supermarket retail industry. 
\end{abstract}
\acknowledgements{
I would like to express my sincere gratitude to Dr Maria Polukarov for her guidance and support which provided me the freedom to take this research in the direction of my interest.\\
\\
I would also like to thank my family and friends for their encouragement and support. To those who quietly listened to my software complaints. To those who worked throughout the nights with me. To those who helped me write what I couldn't say. I cannot thank you enough.
}

\declaration{
I, Stefan Collier, declare that this dissertation and the work presented in it are my own and has been generated by me as the result of my own original research.\\
I confirm that:\\
1. This work was done wholly or mainly while in candidature for a degree at this University;\\
2. Where any part of this dissertation has previously been submitted for any other qualification at this University or any other institution, this has been clearly stated;\\
3. Where I have consulted the published work of others, this is always clearly attributed;\\
4. Where I have quoted from the work of others, the source is always given. With the exception of such quotations, this dissertation is entirely my own work;\\
5. I have acknowledged all main sources of help;\\
6. Where the thesis is based on work done by myself jointly with others, I have made clear exactly what was done by others and what I have contributed myself;\\
7. Either none of this work has been published before submission, or parts of this work have been published by :\\
\\
Stefan Collier\\
April 2016
}
\tableofcontents
\listoffigures
\listoftables

\mainmatter
%% ----------------------------------------------------------------
%\include{Introduction}
%\include{Conclusions}
\include{chapters/1Project/main}
\include{chapters/2Lit/main}
\include{chapters/3Design/HighLevel}
\include{chapters/3Design/InDepth}
\include{chapters/4Impl/main}

\include{chapters/5Experiments/1/main}
\include{chapters/5Experiments/2/main}
\include{chapters/5Experiments/3/main}
\include{chapters/5Experiments/4/main}

\include{chapters/6Conclusion/main}

\appendix
\include{appendix/AppendixB}
\include{appendix/D/main}
\include{appendix/AppendixC}

\backmatter
\bibliographystyle{ecs}
\bibliography{ECS}
\end{document}
%% ----------------------------------------------------------------

\include{chapters/3Design/HighLevel}
\include{chapters/3Design/InDepth}
 %% ----------------------------------------------------------------
%% Progress.tex
%% ---------------------------------------------------------------- 
\documentclass{ecsprogress}    % Use the progress Style
\graphicspath{{../figs/}}   % Location of your graphics files
    \usepackage{natbib}            % Use Natbib style for the refs.
\hypersetup{colorlinks=true}   % Set to false for black/white printing
\input{Definitions}            % Include your abbreviations



\usepackage{enumitem}% http://ctan.org/pkg/enumitem
\usepackage{multirow}
\usepackage{float}
\usepackage{amsmath}
\usepackage{multicol}
\usepackage{amssymb}
\usepackage[normalem]{ulem}
\useunder{\uline}{\ul}{}
\usepackage{wrapfig}


\usepackage[table,xcdraw]{xcolor}


%% ----------------------------------------------------------------
\begin{document}
\frontmatter
\title      {Heterogeneous Agent-based Model for Supermarket Competition}
\authors    {\texorpdfstring
             {\href{mailto:sc22g13@ecs.soton.ac.uk}{Stefan J. Collier}}
             {Stefan J. Collier}
            }
\addresses  {\groupname\\\deptname\\\univname}
\date       {\today}
\subject    {}
\keywords   {}
\supervisor {Dr. Maria Polukarov}
\examiner   {Professor Sheng Chen}

\maketitle
\begin{abstract}
This project aim was to model and analyse the effects of competitive pricing behaviors of grocery retailers on the British market. 

This was achieved by creating a multi-agent model, containing retailer and consumer agents. The heterogeneous crowd of retailers employs either a uniform pricing strategy or a ‘local price flexing’ strategy. The actions of these retailers are chosen by predicting the profit of each action, using a perceptron. Following on from the consideration of different economic models, a discrete model was developed so that software agents have a discrete environment to operate within. Within the model, it has been observed how supermarkets with differing behaviors affect a heterogeneous crowd of consumer agents. The model was implemented in Java with Python used to evaluate the results. 

The simulation displays good acceptance with real grocery market behavior, i.e. captures the performance of British retailers thus can be used to determine the impact of changes in their behavior on their competitors and consumers.Furthermore it can be used to provide insight into sustainability of volatile pricing strategies, providing a useful insight in volatility of British supermarket retail industry. 
\end{abstract}
\acknowledgements{
I would like to express my sincere gratitude to Dr Maria Polukarov for her guidance and support which provided me the freedom to take this research in the direction of my interest.\\
\\
I would also like to thank my family and friends for their encouragement and support. To those who quietly listened to my software complaints. To those who worked throughout the nights with me. To those who helped me write what I couldn't say. I cannot thank you enough.
}

\declaration{
I, Stefan Collier, declare that this dissertation and the work presented in it are my own and has been generated by me as the result of my own original research.\\
I confirm that:\\
1. This work was done wholly or mainly while in candidature for a degree at this University;\\
2. Where any part of this dissertation has previously been submitted for any other qualification at this University or any other institution, this has been clearly stated;\\
3. Where I have consulted the published work of others, this is always clearly attributed;\\
4. Where I have quoted from the work of others, the source is always given. With the exception of such quotations, this dissertation is entirely my own work;\\
5. I have acknowledged all main sources of help;\\
6. Where the thesis is based on work done by myself jointly with others, I have made clear exactly what was done by others and what I have contributed myself;\\
7. Either none of this work has been published before submission, or parts of this work have been published by :\\
\\
Stefan Collier\\
April 2016
}
\tableofcontents
\listoffigures
\listoftables

\mainmatter
%% ----------------------------------------------------------------
%\include{Introduction}
%\include{Conclusions}
\include{chapters/1Project/main}
\include{chapters/2Lit/main}
\include{chapters/3Design/HighLevel}
\include{chapters/3Design/InDepth}
\include{chapters/4Impl/main}

\include{chapters/5Experiments/1/main}
\include{chapters/5Experiments/2/main}
\include{chapters/5Experiments/3/main}
\include{chapters/5Experiments/4/main}

\include{chapters/6Conclusion/main}

\appendix
\include{appendix/AppendixB}
\include{appendix/D/main}
\include{appendix/AppendixC}

\backmatter
\bibliographystyle{ecs}
\bibliography{ECS}
\end{document}
%% ----------------------------------------------------------------


 %% ----------------------------------------------------------------
%% Progress.tex
%% ---------------------------------------------------------------- 
\documentclass{ecsprogress}    % Use the progress Style
\graphicspath{{../figs/}}   % Location of your graphics files
    \usepackage{natbib}            % Use Natbib style for the refs.
\hypersetup{colorlinks=true}   % Set to false for black/white printing
\input{Definitions}            % Include your abbreviations



\usepackage{enumitem}% http://ctan.org/pkg/enumitem
\usepackage{multirow}
\usepackage{float}
\usepackage{amsmath}
\usepackage{multicol}
\usepackage{amssymb}
\usepackage[normalem]{ulem}
\useunder{\uline}{\ul}{}
\usepackage{wrapfig}


\usepackage[table,xcdraw]{xcolor}


%% ----------------------------------------------------------------
\begin{document}
\frontmatter
\title      {Heterogeneous Agent-based Model for Supermarket Competition}
\authors    {\texorpdfstring
             {\href{mailto:sc22g13@ecs.soton.ac.uk}{Stefan J. Collier}}
             {Stefan J. Collier}
            }
\addresses  {\groupname\\\deptname\\\univname}
\date       {\today}
\subject    {}
\keywords   {}
\supervisor {Dr. Maria Polukarov}
\examiner   {Professor Sheng Chen}

\maketitle
\begin{abstract}
This project aim was to model and analyse the effects of competitive pricing behaviors of grocery retailers on the British market. 

This was achieved by creating a multi-agent model, containing retailer and consumer agents. The heterogeneous crowd of retailers employs either a uniform pricing strategy or a ‘local price flexing’ strategy. The actions of these retailers are chosen by predicting the profit of each action, using a perceptron. Following on from the consideration of different economic models, a discrete model was developed so that software agents have a discrete environment to operate within. Within the model, it has been observed how supermarkets with differing behaviors affect a heterogeneous crowd of consumer agents. The model was implemented in Java with Python used to evaluate the results. 

The simulation displays good acceptance with real grocery market behavior, i.e. captures the performance of British retailers thus can be used to determine the impact of changes in their behavior on their competitors and consumers.Furthermore it can be used to provide insight into sustainability of volatile pricing strategies, providing a useful insight in volatility of British supermarket retail industry. 
\end{abstract}
\acknowledgements{
I would like to express my sincere gratitude to Dr Maria Polukarov for her guidance and support which provided me the freedom to take this research in the direction of my interest.\\
\\
I would also like to thank my family and friends for their encouragement and support. To those who quietly listened to my software complaints. To those who worked throughout the nights with me. To those who helped me write what I couldn't say. I cannot thank you enough.
}

\declaration{
I, Stefan Collier, declare that this dissertation and the work presented in it are my own and has been generated by me as the result of my own original research.\\
I confirm that:\\
1. This work was done wholly or mainly while in candidature for a degree at this University;\\
2. Where any part of this dissertation has previously been submitted for any other qualification at this University or any other institution, this has been clearly stated;\\
3. Where I have consulted the published work of others, this is always clearly attributed;\\
4. Where I have quoted from the work of others, the source is always given. With the exception of such quotations, this dissertation is entirely my own work;\\
5. I have acknowledged all main sources of help;\\
6. Where the thesis is based on work done by myself jointly with others, I have made clear exactly what was done by others and what I have contributed myself;\\
7. Either none of this work has been published before submission, or parts of this work have been published by :\\
\\
Stefan Collier\\
April 2016
}
\tableofcontents
\listoffigures
\listoftables

\mainmatter
%% ----------------------------------------------------------------
%\include{Introduction}
%\include{Conclusions}
\include{chapters/1Project/main}
\include{chapters/2Lit/main}
\include{chapters/3Design/HighLevel}
\include{chapters/3Design/InDepth}
\include{chapters/4Impl/main}

\include{chapters/5Experiments/1/main}
\include{chapters/5Experiments/2/main}
\include{chapters/5Experiments/3/main}
\include{chapters/5Experiments/4/main}

\include{chapters/6Conclusion/main}

\appendix
\include{appendix/AppendixB}
\include{appendix/D/main}
\include{appendix/AppendixC}

\backmatter
\bibliographystyle{ecs}
\bibliography{ECS}
\end{document}
%% ----------------------------------------------------------------

 %% ----------------------------------------------------------------
%% Progress.tex
%% ---------------------------------------------------------------- 
\documentclass{ecsprogress}    % Use the progress Style
\graphicspath{{../figs/}}   % Location of your graphics files
    \usepackage{natbib}            % Use Natbib style for the refs.
\hypersetup{colorlinks=true}   % Set to false for black/white printing
\input{Definitions}            % Include your abbreviations



\usepackage{enumitem}% http://ctan.org/pkg/enumitem
\usepackage{multirow}
\usepackage{float}
\usepackage{amsmath}
\usepackage{multicol}
\usepackage{amssymb}
\usepackage[normalem]{ulem}
\useunder{\uline}{\ul}{}
\usepackage{wrapfig}


\usepackage[table,xcdraw]{xcolor}


%% ----------------------------------------------------------------
\begin{document}
\frontmatter
\title      {Heterogeneous Agent-based Model for Supermarket Competition}
\authors    {\texorpdfstring
             {\href{mailto:sc22g13@ecs.soton.ac.uk}{Stefan J. Collier}}
             {Stefan J. Collier}
            }
\addresses  {\groupname\\\deptname\\\univname}
\date       {\today}
\subject    {}
\keywords   {}
\supervisor {Dr. Maria Polukarov}
\examiner   {Professor Sheng Chen}

\maketitle
\begin{abstract}
This project aim was to model and analyse the effects of competitive pricing behaviors of grocery retailers on the British market. 

This was achieved by creating a multi-agent model, containing retailer and consumer agents. The heterogeneous crowd of retailers employs either a uniform pricing strategy or a ‘local price flexing’ strategy. The actions of these retailers are chosen by predicting the profit of each action, using a perceptron. Following on from the consideration of different economic models, a discrete model was developed so that software agents have a discrete environment to operate within. Within the model, it has been observed how supermarkets with differing behaviors affect a heterogeneous crowd of consumer agents. The model was implemented in Java with Python used to evaluate the results. 

The simulation displays good acceptance with real grocery market behavior, i.e. captures the performance of British retailers thus can be used to determine the impact of changes in their behavior on their competitors and consumers.Furthermore it can be used to provide insight into sustainability of volatile pricing strategies, providing a useful insight in volatility of British supermarket retail industry. 
\end{abstract}
\acknowledgements{
I would like to express my sincere gratitude to Dr Maria Polukarov for her guidance and support which provided me the freedom to take this research in the direction of my interest.\\
\\
I would also like to thank my family and friends for their encouragement and support. To those who quietly listened to my software complaints. To those who worked throughout the nights with me. To those who helped me write what I couldn't say. I cannot thank you enough.
}

\declaration{
I, Stefan Collier, declare that this dissertation and the work presented in it are my own and has been generated by me as the result of my own original research.\\
I confirm that:\\
1. This work was done wholly or mainly while in candidature for a degree at this University;\\
2. Where any part of this dissertation has previously been submitted for any other qualification at this University or any other institution, this has been clearly stated;\\
3. Where I have consulted the published work of others, this is always clearly attributed;\\
4. Where I have quoted from the work of others, the source is always given. With the exception of such quotations, this dissertation is entirely my own work;\\
5. I have acknowledged all main sources of help;\\
6. Where the thesis is based on work done by myself jointly with others, I have made clear exactly what was done by others and what I have contributed myself;\\
7. Either none of this work has been published before submission, or parts of this work have been published by :\\
\\
Stefan Collier\\
April 2016
}
\tableofcontents
\listoffigures
\listoftables

\mainmatter
%% ----------------------------------------------------------------
%\include{Introduction}
%\include{Conclusions}
\include{chapters/1Project/main}
\include{chapters/2Lit/main}
\include{chapters/3Design/HighLevel}
\include{chapters/3Design/InDepth}
\include{chapters/4Impl/main}

\include{chapters/5Experiments/1/main}
\include{chapters/5Experiments/2/main}
\include{chapters/5Experiments/3/main}
\include{chapters/5Experiments/4/main}

\include{chapters/6Conclusion/main}

\appendix
\include{appendix/AppendixB}
\include{appendix/D/main}
\include{appendix/AppendixC}

\backmatter
\bibliographystyle{ecs}
\bibliography{ECS}
\end{document}
%% ----------------------------------------------------------------

 %% ----------------------------------------------------------------
%% Progress.tex
%% ---------------------------------------------------------------- 
\documentclass{ecsprogress}    % Use the progress Style
\graphicspath{{../figs/}}   % Location of your graphics files
    \usepackage{natbib}            % Use Natbib style for the refs.
\hypersetup{colorlinks=true}   % Set to false for black/white printing
\input{Definitions}            % Include your abbreviations



\usepackage{enumitem}% http://ctan.org/pkg/enumitem
\usepackage{multirow}
\usepackage{float}
\usepackage{amsmath}
\usepackage{multicol}
\usepackage{amssymb}
\usepackage[normalem]{ulem}
\useunder{\uline}{\ul}{}
\usepackage{wrapfig}


\usepackage[table,xcdraw]{xcolor}


%% ----------------------------------------------------------------
\begin{document}
\frontmatter
\title      {Heterogeneous Agent-based Model for Supermarket Competition}
\authors    {\texorpdfstring
             {\href{mailto:sc22g13@ecs.soton.ac.uk}{Stefan J. Collier}}
             {Stefan J. Collier}
            }
\addresses  {\groupname\\\deptname\\\univname}
\date       {\today}
\subject    {}
\keywords   {}
\supervisor {Dr. Maria Polukarov}
\examiner   {Professor Sheng Chen}

\maketitle
\begin{abstract}
This project aim was to model and analyse the effects of competitive pricing behaviors of grocery retailers on the British market. 

This was achieved by creating a multi-agent model, containing retailer and consumer agents. The heterogeneous crowd of retailers employs either a uniform pricing strategy or a ‘local price flexing’ strategy. The actions of these retailers are chosen by predicting the profit of each action, using a perceptron. Following on from the consideration of different economic models, a discrete model was developed so that software agents have a discrete environment to operate within. Within the model, it has been observed how supermarkets with differing behaviors affect a heterogeneous crowd of consumer agents. The model was implemented in Java with Python used to evaluate the results. 

The simulation displays good acceptance with real grocery market behavior, i.e. captures the performance of British retailers thus can be used to determine the impact of changes in their behavior on their competitors and consumers.Furthermore it can be used to provide insight into sustainability of volatile pricing strategies, providing a useful insight in volatility of British supermarket retail industry. 
\end{abstract}
\acknowledgements{
I would like to express my sincere gratitude to Dr Maria Polukarov for her guidance and support which provided me the freedom to take this research in the direction of my interest.\\
\\
I would also like to thank my family and friends for their encouragement and support. To those who quietly listened to my software complaints. To those who worked throughout the nights with me. To those who helped me write what I couldn't say. I cannot thank you enough.
}

\declaration{
I, Stefan Collier, declare that this dissertation and the work presented in it are my own and has been generated by me as the result of my own original research.\\
I confirm that:\\
1. This work was done wholly or mainly while in candidature for a degree at this University;\\
2. Where any part of this dissertation has previously been submitted for any other qualification at this University or any other institution, this has been clearly stated;\\
3. Where I have consulted the published work of others, this is always clearly attributed;\\
4. Where I have quoted from the work of others, the source is always given. With the exception of such quotations, this dissertation is entirely my own work;\\
5. I have acknowledged all main sources of help;\\
6. Where the thesis is based on work done by myself jointly with others, I have made clear exactly what was done by others and what I have contributed myself;\\
7. Either none of this work has been published before submission, or parts of this work have been published by :\\
\\
Stefan Collier\\
April 2016
}
\tableofcontents
\listoffigures
\listoftables

\mainmatter
%% ----------------------------------------------------------------
%\include{Introduction}
%\include{Conclusions}
\include{chapters/1Project/main}
\include{chapters/2Lit/main}
\include{chapters/3Design/HighLevel}
\include{chapters/3Design/InDepth}
\include{chapters/4Impl/main}

\include{chapters/5Experiments/1/main}
\include{chapters/5Experiments/2/main}
\include{chapters/5Experiments/3/main}
\include{chapters/5Experiments/4/main}

\include{chapters/6Conclusion/main}

\appendix
\include{appendix/AppendixB}
\include{appendix/D/main}
\include{appendix/AppendixC}

\backmatter
\bibliographystyle{ecs}
\bibliography{ECS}
\end{document}
%% ----------------------------------------------------------------

 %% ----------------------------------------------------------------
%% Progress.tex
%% ---------------------------------------------------------------- 
\documentclass{ecsprogress}    % Use the progress Style
\graphicspath{{../figs/}}   % Location of your graphics files
    \usepackage{natbib}            % Use Natbib style for the refs.
\hypersetup{colorlinks=true}   % Set to false for black/white printing
\input{Definitions}            % Include your abbreviations



\usepackage{enumitem}% http://ctan.org/pkg/enumitem
\usepackage{multirow}
\usepackage{float}
\usepackage{amsmath}
\usepackage{multicol}
\usepackage{amssymb}
\usepackage[normalem]{ulem}
\useunder{\uline}{\ul}{}
\usepackage{wrapfig}


\usepackage[table,xcdraw]{xcolor}


%% ----------------------------------------------------------------
\begin{document}
\frontmatter
\title      {Heterogeneous Agent-based Model for Supermarket Competition}
\authors    {\texorpdfstring
             {\href{mailto:sc22g13@ecs.soton.ac.uk}{Stefan J. Collier}}
             {Stefan J. Collier}
            }
\addresses  {\groupname\\\deptname\\\univname}
\date       {\today}
\subject    {}
\keywords   {}
\supervisor {Dr. Maria Polukarov}
\examiner   {Professor Sheng Chen}

\maketitle
\begin{abstract}
This project aim was to model and analyse the effects of competitive pricing behaviors of grocery retailers on the British market. 

This was achieved by creating a multi-agent model, containing retailer and consumer agents. The heterogeneous crowd of retailers employs either a uniform pricing strategy or a ‘local price flexing’ strategy. The actions of these retailers are chosen by predicting the profit of each action, using a perceptron. Following on from the consideration of different economic models, a discrete model was developed so that software agents have a discrete environment to operate within. Within the model, it has been observed how supermarkets with differing behaviors affect a heterogeneous crowd of consumer agents. The model was implemented in Java with Python used to evaluate the results. 

The simulation displays good acceptance with real grocery market behavior, i.e. captures the performance of British retailers thus can be used to determine the impact of changes in their behavior on their competitors and consumers.Furthermore it can be used to provide insight into sustainability of volatile pricing strategies, providing a useful insight in volatility of British supermarket retail industry. 
\end{abstract}
\acknowledgements{
I would like to express my sincere gratitude to Dr Maria Polukarov for her guidance and support which provided me the freedom to take this research in the direction of my interest.\\
\\
I would also like to thank my family and friends for their encouragement and support. To those who quietly listened to my software complaints. To those who worked throughout the nights with me. To those who helped me write what I couldn't say. I cannot thank you enough.
}

\declaration{
I, Stefan Collier, declare that this dissertation and the work presented in it are my own and has been generated by me as the result of my own original research.\\
I confirm that:\\
1. This work was done wholly or mainly while in candidature for a degree at this University;\\
2. Where any part of this dissertation has previously been submitted for any other qualification at this University or any other institution, this has been clearly stated;\\
3. Where I have consulted the published work of others, this is always clearly attributed;\\
4. Where I have quoted from the work of others, the source is always given. With the exception of such quotations, this dissertation is entirely my own work;\\
5. I have acknowledged all main sources of help;\\
6. Where the thesis is based on work done by myself jointly with others, I have made clear exactly what was done by others and what I have contributed myself;\\
7. Either none of this work has been published before submission, or parts of this work have been published by :\\
\\
Stefan Collier\\
April 2016
}
\tableofcontents
\listoffigures
\listoftables

\mainmatter
%% ----------------------------------------------------------------
%\include{Introduction}
%\include{Conclusions}
\include{chapters/1Project/main}
\include{chapters/2Lit/main}
\include{chapters/3Design/HighLevel}
\include{chapters/3Design/InDepth}
\include{chapters/4Impl/main}

\include{chapters/5Experiments/1/main}
\include{chapters/5Experiments/2/main}
\include{chapters/5Experiments/3/main}
\include{chapters/5Experiments/4/main}

\include{chapters/6Conclusion/main}

\appendix
\include{appendix/AppendixB}
\include{appendix/D/main}
\include{appendix/AppendixC}

\backmatter
\bibliographystyle{ecs}
\bibliography{ECS}
\end{document}
%% ----------------------------------------------------------------


 %% ----------------------------------------------------------------
%% Progress.tex
%% ---------------------------------------------------------------- 
\documentclass{ecsprogress}    % Use the progress Style
\graphicspath{{../figs/}}   % Location of your graphics files
    \usepackage{natbib}            % Use Natbib style for the refs.
\hypersetup{colorlinks=true}   % Set to false for black/white printing
\input{Definitions}            % Include your abbreviations



\usepackage{enumitem}% http://ctan.org/pkg/enumitem
\usepackage{multirow}
\usepackage{float}
\usepackage{amsmath}
\usepackage{multicol}
\usepackage{amssymb}
\usepackage[normalem]{ulem}
\useunder{\uline}{\ul}{}
\usepackage{wrapfig}


\usepackage[table,xcdraw]{xcolor}


%% ----------------------------------------------------------------
\begin{document}
\frontmatter
\title      {Heterogeneous Agent-based Model for Supermarket Competition}
\authors    {\texorpdfstring
             {\href{mailto:sc22g13@ecs.soton.ac.uk}{Stefan J. Collier}}
             {Stefan J. Collier}
            }
\addresses  {\groupname\\\deptname\\\univname}
\date       {\today}
\subject    {}
\keywords   {}
\supervisor {Dr. Maria Polukarov}
\examiner   {Professor Sheng Chen}

\maketitle
\begin{abstract}
This project aim was to model and analyse the effects of competitive pricing behaviors of grocery retailers on the British market. 

This was achieved by creating a multi-agent model, containing retailer and consumer agents. The heterogeneous crowd of retailers employs either a uniform pricing strategy or a ‘local price flexing’ strategy. The actions of these retailers are chosen by predicting the profit of each action, using a perceptron. Following on from the consideration of different economic models, a discrete model was developed so that software agents have a discrete environment to operate within. Within the model, it has been observed how supermarkets with differing behaviors affect a heterogeneous crowd of consumer agents. The model was implemented in Java with Python used to evaluate the results. 

The simulation displays good acceptance with real grocery market behavior, i.e. captures the performance of British retailers thus can be used to determine the impact of changes in their behavior on their competitors and consumers.Furthermore it can be used to provide insight into sustainability of volatile pricing strategies, providing a useful insight in volatility of British supermarket retail industry. 
\end{abstract}
\acknowledgements{
I would like to express my sincere gratitude to Dr Maria Polukarov for her guidance and support which provided me the freedom to take this research in the direction of my interest.\\
\\
I would also like to thank my family and friends for their encouragement and support. To those who quietly listened to my software complaints. To those who worked throughout the nights with me. To those who helped me write what I couldn't say. I cannot thank you enough.
}

\declaration{
I, Stefan Collier, declare that this dissertation and the work presented in it are my own and has been generated by me as the result of my own original research.\\
I confirm that:\\
1. This work was done wholly or mainly while in candidature for a degree at this University;\\
2. Where any part of this dissertation has previously been submitted for any other qualification at this University or any other institution, this has been clearly stated;\\
3. Where I have consulted the published work of others, this is always clearly attributed;\\
4. Where I have quoted from the work of others, the source is always given. With the exception of such quotations, this dissertation is entirely my own work;\\
5. I have acknowledged all main sources of help;\\
6. Where the thesis is based on work done by myself jointly with others, I have made clear exactly what was done by others and what I have contributed myself;\\
7. Either none of this work has been published before submission, or parts of this work have been published by :\\
\\
Stefan Collier\\
April 2016
}
\tableofcontents
\listoffigures
\listoftables

\mainmatter
%% ----------------------------------------------------------------
%\include{Introduction}
%\include{Conclusions}
\include{chapters/1Project/main}
\include{chapters/2Lit/main}
\include{chapters/3Design/HighLevel}
\include{chapters/3Design/InDepth}
\include{chapters/4Impl/main}

\include{chapters/5Experiments/1/main}
\include{chapters/5Experiments/2/main}
\include{chapters/5Experiments/3/main}
\include{chapters/5Experiments/4/main}

\include{chapters/6Conclusion/main}

\appendix
\include{appendix/AppendixB}
\include{appendix/D/main}
\include{appendix/AppendixC}

\backmatter
\bibliographystyle{ecs}
\bibliography{ECS}
\end{document}
%% ----------------------------------------------------------------


\appendix
\include{appendix/AppendixB}
 %% ----------------------------------------------------------------
%% Progress.tex
%% ---------------------------------------------------------------- 
\documentclass{ecsprogress}    % Use the progress Style
\graphicspath{{../figs/}}   % Location of your graphics files
    \usepackage{natbib}            % Use Natbib style for the refs.
\hypersetup{colorlinks=true}   % Set to false for black/white printing
\input{Definitions}            % Include your abbreviations



\usepackage{enumitem}% http://ctan.org/pkg/enumitem
\usepackage{multirow}
\usepackage{float}
\usepackage{amsmath}
\usepackage{multicol}
\usepackage{amssymb}
\usepackage[normalem]{ulem}
\useunder{\uline}{\ul}{}
\usepackage{wrapfig}


\usepackage[table,xcdraw]{xcolor}


%% ----------------------------------------------------------------
\begin{document}
\frontmatter
\title      {Heterogeneous Agent-based Model for Supermarket Competition}
\authors    {\texorpdfstring
             {\href{mailto:sc22g13@ecs.soton.ac.uk}{Stefan J. Collier}}
             {Stefan J. Collier}
            }
\addresses  {\groupname\\\deptname\\\univname}
\date       {\today}
\subject    {}
\keywords   {}
\supervisor {Dr. Maria Polukarov}
\examiner   {Professor Sheng Chen}

\maketitle
\begin{abstract}
This project aim was to model and analyse the effects of competitive pricing behaviors of grocery retailers on the British market. 

This was achieved by creating a multi-agent model, containing retailer and consumer agents. The heterogeneous crowd of retailers employs either a uniform pricing strategy or a ‘local price flexing’ strategy. The actions of these retailers are chosen by predicting the profit of each action, using a perceptron. Following on from the consideration of different economic models, a discrete model was developed so that software agents have a discrete environment to operate within. Within the model, it has been observed how supermarkets with differing behaviors affect a heterogeneous crowd of consumer agents. The model was implemented in Java with Python used to evaluate the results. 

The simulation displays good acceptance with real grocery market behavior, i.e. captures the performance of British retailers thus can be used to determine the impact of changes in their behavior on their competitors and consumers.Furthermore it can be used to provide insight into sustainability of volatile pricing strategies, providing a useful insight in volatility of British supermarket retail industry. 
\end{abstract}
\acknowledgements{
I would like to express my sincere gratitude to Dr Maria Polukarov for her guidance and support which provided me the freedom to take this research in the direction of my interest.\\
\\
I would also like to thank my family and friends for their encouragement and support. To those who quietly listened to my software complaints. To those who worked throughout the nights with me. To those who helped me write what I couldn't say. I cannot thank you enough.
}

\declaration{
I, Stefan Collier, declare that this dissertation and the work presented in it are my own and has been generated by me as the result of my own original research.\\
I confirm that:\\
1. This work was done wholly or mainly while in candidature for a degree at this University;\\
2. Where any part of this dissertation has previously been submitted for any other qualification at this University or any other institution, this has been clearly stated;\\
3. Where I have consulted the published work of others, this is always clearly attributed;\\
4. Where I have quoted from the work of others, the source is always given. With the exception of such quotations, this dissertation is entirely my own work;\\
5. I have acknowledged all main sources of help;\\
6. Where the thesis is based on work done by myself jointly with others, I have made clear exactly what was done by others and what I have contributed myself;\\
7. Either none of this work has been published before submission, or parts of this work have been published by :\\
\\
Stefan Collier\\
April 2016
}
\tableofcontents
\listoffigures
\listoftables

\mainmatter
%% ----------------------------------------------------------------
%\include{Introduction}
%\include{Conclusions}
\include{chapters/1Project/main}
\include{chapters/2Lit/main}
\include{chapters/3Design/HighLevel}
\include{chapters/3Design/InDepth}
\include{chapters/4Impl/main}

\include{chapters/5Experiments/1/main}
\include{chapters/5Experiments/2/main}
\include{chapters/5Experiments/3/main}
\include{chapters/5Experiments/4/main}

\include{chapters/6Conclusion/main}

\appendix
\include{appendix/AppendixB}
\include{appendix/D/main}
\include{appendix/AppendixC}

\backmatter
\bibliographystyle{ecs}
\bibliography{ECS}
\end{document}
%% ----------------------------------------------------------------

\include{appendix/AppendixC}

\backmatter
\bibliographystyle{ecs}
\bibliography{ECS}
\end{document}
%% ----------------------------------------------------------------

\include{chapters/3Design/HighLevel}
\include{chapters/3Design/InDepth}
 %% ----------------------------------------------------------------
%% Progress.tex
%% ---------------------------------------------------------------- 
\documentclass{ecsprogress}    % Use the progress Style
\graphicspath{{../figs/}}   % Location of your graphics files
    \usepackage{natbib}            % Use Natbib style for the refs.
\hypersetup{colorlinks=true}   % Set to false for black/white printing
\input{Definitions}            % Include your abbreviations



\usepackage{enumitem}% http://ctan.org/pkg/enumitem
\usepackage{multirow}
\usepackage{float}
\usepackage{amsmath}
\usepackage{multicol}
\usepackage{amssymb}
\usepackage[normalem]{ulem}
\useunder{\uline}{\ul}{}
\usepackage{wrapfig}


\usepackage[table,xcdraw]{xcolor}


%% ----------------------------------------------------------------
\begin{document}
\frontmatter
\title      {Heterogeneous Agent-based Model for Supermarket Competition}
\authors    {\texorpdfstring
             {\href{mailto:sc22g13@ecs.soton.ac.uk}{Stefan J. Collier}}
             {Stefan J. Collier}
            }
\addresses  {\groupname\\\deptname\\\univname}
\date       {\today}
\subject    {}
\keywords   {}
\supervisor {Dr. Maria Polukarov}
\examiner   {Professor Sheng Chen}

\maketitle
\begin{abstract}
This project aim was to model and analyse the effects of competitive pricing behaviors of grocery retailers on the British market. 

This was achieved by creating a multi-agent model, containing retailer and consumer agents. The heterogeneous crowd of retailers employs either a uniform pricing strategy or a ‘local price flexing’ strategy. The actions of these retailers are chosen by predicting the profit of each action, using a perceptron. Following on from the consideration of different economic models, a discrete model was developed so that software agents have a discrete environment to operate within. Within the model, it has been observed how supermarkets with differing behaviors affect a heterogeneous crowd of consumer agents. The model was implemented in Java with Python used to evaluate the results. 

The simulation displays good acceptance with real grocery market behavior, i.e. captures the performance of British retailers thus can be used to determine the impact of changes in their behavior on their competitors and consumers.Furthermore it can be used to provide insight into sustainability of volatile pricing strategies, providing a useful insight in volatility of British supermarket retail industry. 
\end{abstract}
\acknowledgements{
I would like to express my sincere gratitude to Dr Maria Polukarov for her guidance and support which provided me the freedom to take this research in the direction of my interest.\\
\\
I would also like to thank my family and friends for their encouragement and support. To those who quietly listened to my software complaints. To those who worked throughout the nights with me. To those who helped me write what I couldn't say. I cannot thank you enough.
}

\declaration{
I, Stefan Collier, declare that this dissertation and the work presented in it are my own and has been generated by me as the result of my own original research.\\
I confirm that:\\
1. This work was done wholly or mainly while in candidature for a degree at this University;\\
2. Where any part of this dissertation has previously been submitted for any other qualification at this University or any other institution, this has been clearly stated;\\
3. Where I have consulted the published work of others, this is always clearly attributed;\\
4. Where I have quoted from the work of others, the source is always given. With the exception of such quotations, this dissertation is entirely my own work;\\
5. I have acknowledged all main sources of help;\\
6. Where the thesis is based on work done by myself jointly with others, I have made clear exactly what was done by others and what I have contributed myself;\\
7. Either none of this work has been published before submission, or parts of this work have been published by :\\
\\
Stefan Collier\\
April 2016
}
\tableofcontents
\listoffigures
\listoftables

\mainmatter
%% ----------------------------------------------------------------
%\include{Introduction}
%\include{Conclusions}
 %% ----------------------------------------------------------------
%% Progress.tex
%% ---------------------------------------------------------------- 
\documentclass{ecsprogress}    % Use the progress Style
\graphicspath{{../figs/}}   % Location of your graphics files
    \usepackage{natbib}            % Use Natbib style for the refs.
\hypersetup{colorlinks=true}   % Set to false for black/white printing
\input{Definitions}            % Include your abbreviations



\usepackage{enumitem}% http://ctan.org/pkg/enumitem
\usepackage{multirow}
\usepackage{float}
\usepackage{amsmath}
\usepackage{multicol}
\usepackage{amssymb}
\usepackage[normalem]{ulem}
\useunder{\uline}{\ul}{}
\usepackage{wrapfig}


\usepackage[table,xcdraw]{xcolor}


%% ----------------------------------------------------------------
\begin{document}
\frontmatter
\title      {Heterogeneous Agent-based Model for Supermarket Competition}
\authors    {\texorpdfstring
             {\href{mailto:sc22g13@ecs.soton.ac.uk}{Stefan J. Collier}}
             {Stefan J. Collier}
            }
\addresses  {\groupname\\\deptname\\\univname}
\date       {\today}
\subject    {}
\keywords   {}
\supervisor {Dr. Maria Polukarov}
\examiner   {Professor Sheng Chen}

\maketitle
\begin{abstract}
This project aim was to model and analyse the effects of competitive pricing behaviors of grocery retailers on the British market. 

This was achieved by creating a multi-agent model, containing retailer and consumer agents. The heterogeneous crowd of retailers employs either a uniform pricing strategy or a ‘local price flexing’ strategy. The actions of these retailers are chosen by predicting the profit of each action, using a perceptron. Following on from the consideration of different economic models, a discrete model was developed so that software agents have a discrete environment to operate within. Within the model, it has been observed how supermarkets with differing behaviors affect a heterogeneous crowd of consumer agents. The model was implemented in Java with Python used to evaluate the results. 

The simulation displays good acceptance with real grocery market behavior, i.e. captures the performance of British retailers thus can be used to determine the impact of changes in their behavior on their competitors and consumers.Furthermore it can be used to provide insight into sustainability of volatile pricing strategies, providing a useful insight in volatility of British supermarket retail industry. 
\end{abstract}
\acknowledgements{
I would like to express my sincere gratitude to Dr Maria Polukarov for her guidance and support which provided me the freedom to take this research in the direction of my interest.\\
\\
I would also like to thank my family and friends for their encouragement and support. To those who quietly listened to my software complaints. To those who worked throughout the nights with me. To those who helped me write what I couldn't say. I cannot thank you enough.
}

\declaration{
I, Stefan Collier, declare that this dissertation and the work presented in it are my own and has been generated by me as the result of my own original research.\\
I confirm that:\\
1. This work was done wholly or mainly while in candidature for a degree at this University;\\
2. Where any part of this dissertation has previously been submitted for any other qualification at this University or any other institution, this has been clearly stated;\\
3. Where I have consulted the published work of others, this is always clearly attributed;\\
4. Where I have quoted from the work of others, the source is always given. With the exception of such quotations, this dissertation is entirely my own work;\\
5. I have acknowledged all main sources of help;\\
6. Where the thesis is based on work done by myself jointly with others, I have made clear exactly what was done by others and what I have contributed myself;\\
7. Either none of this work has been published before submission, or parts of this work have been published by :\\
\\
Stefan Collier\\
April 2016
}
\tableofcontents
\listoffigures
\listoftables

\mainmatter
%% ----------------------------------------------------------------
%\include{Introduction}
%\include{Conclusions}
\include{chapters/1Project/main}
\include{chapters/2Lit/main}
\include{chapters/3Design/HighLevel}
\include{chapters/3Design/InDepth}
\include{chapters/4Impl/main}

\include{chapters/5Experiments/1/main}
\include{chapters/5Experiments/2/main}
\include{chapters/5Experiments/3/main}
\include{chapters/5Experiments/4/main}

\include{chapters/6Conclusion/main}

\appendix
\include{appendix/AppendixB}
\include{appendix/D/main}
\include{appendix/AppendixC}

\backmatter
\bibliographystyle{ecs}
\bibliography{ECS}
\end{document}
%% ----------------------------------------------------------------

 %% ----------------------------------------------------------------
%% Progress.tex
%% ---------------------------------------------------------------- 
\documentclass{ecsprogress}    % Use the progress Style
\graphicspath{{../figs/}}   % Location of your graphics files
    \usepackage{natbib}            % Use Natbib style for the refs.
\hypersetup{colorlinks=true}   % Set to false for black/white printing
\input{Definitions}            % Include your abbreviations



\usepackage{enumitem}% http://ctan.org/pkg/enumitem
\usepackage{multirow}
\usepackage{float}
\usepackage{amsmath}
\usepackage{multicol}
\usepackage{amssymb}
\usepackage[normalem]{ulem}
\useunder{\uline}{\ul}{}
\usepackage{wrapfig}


\usepackage[table,xcdraw]{xcolor}


%% ----------------------------------------------------------------
\begin{document}
\frontmatter
\title      {Heterogeneous Agent-based Model for Supermarket Competition}
\authors    {\texorpdfstring
             {\href{mailto:sc22g13@ecs.soton.ac.uk}{Stefan J. Collier}}
             {Stefan J. Collier}
            }
\addresses  {\groupname\\\deptname\\\univname}
\date       {\today}
\subject    {}
\keywords   {}
\supervisor {Dr. Maria Polukarov}
\examiner   {Professor Sheng Chen}

\maketitle
\begin{abstract}
This project aim was to model and analyse the effects of competitive pricing behaviors of grocery retailers on the British market. 

This was achieved by creating a multi-agent model, containing retailer and consumer agents. The heterogeneous crowd of retailers employs either a uniform pricing strategy or a ‘local price flexing’ strategy. The actions of these retailers are chosen by predicting the profit of each action, using a perceptron. Following on from the consideration of different economic models, a discrete model was developed so that software agents have a discrete environment to operate within. Within the model, it has been observed how supermarkets with differing behaviors affect a heterogeneous crowd of consumer agents. The model was implemented in Java with Python used to evaluate the results. 

The simulation displays good acceptance with real grocery market behavior, i.e. captures the performance of British retailers thus can be used to determine the impact of changes in their behavior on their competitors and consumers.Furthermore it can be used to provide insight into sustainability of volatile pricing strategies, providing a useful insight in volatility of British supermarket retail industry. 
\end{abstract}
\acknowledgements{
I would like to express my sincere gratitude to Dr Maria Polukarov for her guidance and support which provided me the freedom to take this research in the direction of my interest.\\
\\
I would also like to thank my family and friends for their encouragement and support. To those who quietly listened to my software complaints. To those who worked throughout the nights with me. To those who helped me write what I couldn't say. I cannot thank you enough.
}

\declaration{
I, Stefan Collier, declare that this dissertation and the work presented in it are my own and has been generated by me as the result of my own original research.\\
I confirm that:\\
1. This work was done wholly or mainly while in candidature for a degree at this University;\\
2. Where any part of this dissertation has previously been submitted for any other qualification at this University or any other institution, this has been clearly stated;\\
3. Where I have consulted the published work of others, this is always clearly attributed;\\
4. Where I have quoted from the work of others, the source is always given. With the exception of such quotations, this dissertation is entirely my own work;\\
5. I have acknowledged all main sources of help;\\
6. Where the thesis is based on work done by myself jointly with others, I have made clear exactly what was done by others and what I have contributed myself;\\
7. Either none of this work has been published before submission, or parts of this work have been published by :\\
\\
Stefan Collier\\
April 2016
}
\tableofcontents
\listoffigures
\listoftables

\mainmatter
%% ----------------------------------------------------------------
%\include{Introduction}
%\include{Conclusions}
\include{chapters/1Project/main}
\include{chapters/2Lit/main}
\include{chapters/3Design/HighLevel}
\include{chapters/3Design/InDepth}
\include{chapters/4Impl/main}

\include{chapters/5Experiments/1/main}
\include{chapters/5Experiments/2/main}
\include{chapters/5Experiments/3/main}
\include{chapters/5Experiments/4/main}

\include{chapters/6Conclusion/main}

\appendix
\include{appendix/AppendixB}
\include{appendix/D/main}
\include{appendix/AppendixC}

\backmatter
\bibliographystyle{ecs}
\bibliography{ECS}
\end{document}
%% ----------------------------------------------------------------

\include{chapters/3Design/HighLevel}
\include{chapters/3Design/InDepth}
 %% ----------------------------------------------------------------
%% Progress.tex
%% ---------------------------------------------------------------- 
\documentclass{ecsprogress}    % Use the progress Style
\graphicspath{{../figs/}}   % Location of your graphics files
    \usepackage{natbib}            % Use Natbib style for the refs.
\hypersetup{colorlinks=true}   % Set to false for black/white printing
\input{Definitions}            % Include your abbreviations



\usepackage{enumitem}% http://ctan.org/pkg/enumitem
\usepackage{multirow}
\usepackage{float}
\usepackage{amsmath}
\usepackage{multicol}
\usepackage{amssymb}
\usepackage[normalem]{ulem}
\useunder{\uline}{\ul}{}
\usepackage{wrapfig}


\usepackage[table,xcdraw]{xcolor}


%% ----------------------------------------------------------------
\begin{document}
\frontmatter
\title      {Heterogeneous Agent-based Model for Supermarket Competition}
\authors    {\texorpdfstring
             {\href{mailto:sc22g13@ecs.soton.ac.uk}{Stefan J. Collier}}
             {Stefan J. Collier}
            }
\addresses  {\groupname\\\deptname\\\univname}
\date       {\today}
\subject    {}
\keywords   {}
\supervisor {Dr. Maria Polukarov}
\examiner   {Professor Sheng Chen}

\maketitle
\begin{abstract}
This project aim was to model and analyse the effects of competitive pricing behaviors of grocery retailers on the British market. 

This was achieved by creating a multi-agent model, containing retailer and consumer agents. The heterogeneous crowd of retailers employs either a uniform pricing strategy or a ‘local price flexing’ strategy. The actions of these retailers are chosen by predicting the profit of each action, using a perceptron. Following on from the consideration of different economic models, a discrete model was developed so that software agents have a discrete environment to operate within. Within the model, it has been observed how supermarkets with differing behaviors affect a heterogeneous crowd of consumer agents. The model was implemented in Java with Python used to evaluate the results. 

The simulation displays good acceptance with real grocery market behavior, i.e. captures the performance of British retailers thus can be used to determine the impact of changes in their behavior on their competitors and consumers.Furthermore it can be used to provide insight into sustainability of volatile pricing strategies, providing a useful insight in volatility of British supermarket retail industry. 
\end{abstract}
\acknowledgements{
I would like to express my sincere gratitude to Dr Maria Polukarov for her guidance and support which provided me the freedom to take this research in the direction of my interest.\\
\\
I would also like to thank my family and friends for their encouragement and support. To those who quietly listened to my software complaints. To those who worked throughout the nights with me. To those who helped me write what I couldn't say. I cannot thank you enough.
}

\declaration{
I, Stefan Collier, declare that this dissertation and the work presented in it are my own and has been generated by me as the result of my own original research.\\
I confirm that:\\
1. This work was done wholly or mainly while in candidature for a degree at this University;\\
2. Where any part of this dissertation has previously been submitted for any other qualification at this University or any other institution, this has been clearly stated;\\
3. Where I have consulted the published work of others, this is always clearly attributed;\\
4. Where I have quoted from the work of others, the source is always given. With the exception of such quotations, this dissertation is entirely my own work;\\
5. I have acknowledged all main sources of help;\\
6. Where the thesis is based on work done by myself jointly with others, I have made clear exactly what was done by others and what I have contributed myself;\\
7. Either none of this work has been published before submission, or parts of this work have been published by :\\
\\
Stefan Collier\\
April 2016
}
\tableofcontents
\listoffigures
\listoftables

\mainmatter
%% ----------------------------------------------------------------
%\include{Introduction}
%\include{Conclusions}
\include{chapters/1Project/main}
\include{chapters/2Lit/main}
\include{chapters/3Design/HighLevel}
\include{chapters/3Design/InDepth}
\include{chapters/4Impl/main}

\include{chapters/5Experiments/1/main}
\include{chapters/5Experiments/2/main}
\include{chapters/5Experiments/3/main}
\include{chapters/5Experiments/4/main}

\include{chapters/6Conclusion/main}

\appendix
\include{appendix/AppendixB}
\include{appendix/D/main}
\include{appendix/AppendixC}

\backmatter
\bibliographystyle{ecs}
\bibliography{ECS}
\end{document}
%% ----------------------------------------------------------------


 %% ----------------------------------------------------------------
%% Progress.tex
%% ---------------------------------------------------------------- 
\documentclass{ecsprogress}    % Use the progress Style
\graphicspath{{../figs/}}   % Location of your graphics files
    \usepackage{natbib}            % Use Natbib style for the refs.
\hypersetup{colorlinks=true}   % Set to false for black/white printing
\input{Definitions}            % Include your abbreviations



\usepackage{enumitem}% http://ctan.org/pkg/enumitem
\usepackage{multirow}
\usepackage{float}
\usepackage{amsmath}
\usepackage{multicol}
\usepackage{amssymb}
\usepackage[normalem]{ulem}
\useunder{\uline}{\ul}{}
\usepackage{wrapfig}


\usepackage[table,xcdraw]{xcolor}


%% ----------------------------------------------------------------
\begin{document}
\frontmatter
\title      {Heterogeneous Agent-based Model for Supermarket Competition}
\authors    {\texorpdfstring
             {\href{mailto:sc22g13@ecs.soton.ac.uk}{Stefan J. Collier}}
             {Stefan J. Collier}
            }
\addresses  {\groupname\\\deptname\\\univname}
\date       {\today}
\subject    {}
\keywords   {}
\supervisor {Dr. Maria Polukarov}
\examiner   {Professor Sheng Chen}

\maketitle
\begin{abstract}
This project aim was to model and analyse the effects of competitive pricing behaviors of grocery retailers on the British market. 

This was achieved by creating a multi-agent model, containing retailer and consumer agents. The heterogeneous crowd of retailers employs either a uniform pricing strategy or a ‘local price flexing’ strategy. The actions of these retailers are chosen by predicting the profit of each action, using a perceptron. Following on from the consideration of different economic models, a discrete model was developed so that software agents have a discrete environment to operate within. Within the model, it has been observed how supermarkets with differing behaviors affect a heterogeneous crowd of consumer agents. The model was implemented in Java with Python used to evaluate the results. 

The simulation displays good acceptance with real grocery market behavior, i.e. captures the performance of British retailers thus can be used to determine the impact of changes in their behavior on their competitors and consumers.Furthermore it can be used to provide insight into sustainability of volatile pricing strategies, providing a useful insight in volatility of British supermarket retail industry. 
\end{abstract}
\acknowledgements{
I would like to express my sincere gratitude to Dr Maria Polukarov for her guidance and support which provided me the freedom to take this research in the direction of my interest.\\
\\
I would also like to thank my family and friends for their encouragement and support. To those who quietly listened to my software complaints. To those who worked throughout the nights with me. To those who helped me write what I couldn't say. I cannot thank you enough.
}

\declaration{
I, Stefan Collier, declare that this dissertation and the work presented in it are my own and has been generated by me as the result of my own original research.\\
I confirm that:\\
1. This work was done wholly or mainly while in candidature for a degree at this University;\\
2. Where any part of this dissertation has previously been submitted for any other qualification at this University or any other institution, this has been clearly stated;\\
3. Where I have consulted the published work of others, this is always clearly attributed;\\
4. Where I have quoted from the work of others, the source is always given. With the exception of such quotations, this dissertation is entirely my own work;\\
5. I have acknowledged all main sources of help;\\
6. Where the thesis is based on work done by myself jointly with others, I have made clear exactly what was done by others and what I have contributed myself;\\
7. Either none of this work has been published before submission, or parts of this work have been published by :\\
\\
Stefan Collier\\
April 2016
}
\tableofcontents
\listoffigures
\listoftables

\mainmatter
%% ----------------------------------------------------------------
%\include{Introduction}
%\include{Conclusions}
\include{chapters/1Project/main}
\include{chapters/2Lit/main}
\include{chapters/3Design/HighLevel}
\include{chapters/3Design/InDepth}
\include{chapters/4Impl/main}

\include{chapters/5Experiments/1/main}
\include{chapters/5Experiments/2/main}
\include{chapters/5Experiments/3/main}
\include{chapters/5Experiments/4/main}

\include{chapters/6Conclusion/main}

\appendix
\include{appendix/AppendixB}
\include{appendix/D/main}
\include{appendix/AppendixC}

\backmatter
\bibliographystyle{ecs}
\bibliography{ECS}
\end{document}
%% ----------------------------------------------------------------

 %% ----------------------------------------------------------------
%% Progress.tex
%% ---------------------------------------------------------------- 
\documentclass{ecsprogress}    % Use the progress Style
\graphicspath{{../figs/}}   % Location of your graphics files
    \usepackage{natbib}            % Use Natbib style for the refs.
\hypersetup{colorlinks=true}   % Set to false for black/white printing
\input{Definitions}            % Include your abbreviations



\usepackage{enumitem}% http://ctan.org/pkg/enumitem
\usepackage{multirow}
\usepackage{float}
\usepackage{amsmath}
\usepackage{multicol}
\usepackage{amssymb}
\usepackage[normalem]{ulem}
\useunder{\uline}{\ul}{}
\usepackage{wrapfig}


\usepackage[table,xcdraw]{xcolor}


%% ----------------------------------------------------------------
\begin{document}
\frontmatter
\title      {Heterogeneous Agent-based Model for Supermarket Competition}
\authors    {\texorpdfstring
             {\href{mailto:sc22g13@ecs.soton.ac.uk}{Stefan J. Collier}}
             {Stefan J. Collier}
            }
\addresses  {\groupname\\\deptname\\\univname}
\date       {\today}
\subject    {}
\keywords   {}
\supervisor {Dr. Maria Polukarov}
\examiner   {Professor Sheng Chen}

\maketitle
\begin{abstract}
This project aim was to model and analyse the effects of competitive pricing behaviors of grocery retailers on the British market. 

This was achieved by creating a multi-agent model, containing retailer and consumer agents. The heterogeneous crowd of retailers employs either a uniform pricing strategy or a ‘local price flexing’ strategy. The actions of these retailers are chosen by predicting the profit of each action, using a perceptron. Following on from the consideration of different economic models, a discrete model was developed so that software agents have a discrete environment to operate within. Within the model, it has been observed how supermarkets with differing behaviors affect a heterogeneous crowd of consumer agents. The model was implemented in Java with Python used to evaluate the results. 

The simulation displays good acceptance with real grocery market behavior, i.e. captures the performance of British retailers thus can be used to determine the impact of changes in their behavior on their competitors and consumers.Furthermore it can be used to provide insight into sustainability of volatile pricing strategies, providing a useful insight in volatility of British supermarket retail industry. 
\end{abstract}
\acknowledgements{
I would like to express my sincere gratitude to Dr Maria Polukarov for her guidance and support which provided me the freedom to take this research in the direction of my interest.\\
\\
I would also like to thank my family and friends for their encouragement and support. To those who quietly listened to my software complaints. To those who worked throughout the nights with me. To those who helped me write what I couldn't say. I cannot thank you enough.
}

\declaration{
I, Stefan Collier, declare that this dissertation and the work presented in it are my own and has been generated by me as the result of my own original research.\\
I confirm that:\\
1. This work was done wholly or mainly while in candidature for a degree at this University;\\
2. Where any part of this dissertation has previously been submitted for any other qualification at this University or any other institution, this has been clearly stated;\\
3. Where I have consulted the published work of others, this is always clearly attributed;\\
4. Where I have quoted from the work of others, the source is always given. With the exception of such quotations, this dissertation is entirely my own work;\\
5. I have acknowledged all main sources of help;\\
6. Where the thesis is based on work done by myself jointly with others, I have made clear exactly what was done by others and what I have contributed myself;\\
7. Either none of this work has been published before submission, or parts of this work have been published by :\\
\\
Stefan Collier\\
April 2016
}
\tableofcontents
\listoffigures
\listoftables

\mainmatter
%% ----------------------------------------------------------------
%\include{Introduction}
%\include{Conclusions}
\include{chapters/1Project/main}
\include{chapters/2Lit/main}
\include{chapters/3Design/HighLevel}
\include{chapters/3Design/InDepth}
\include{chapters/4Impl/main}

\include{chapters/5Experiments/1/main}
\include{chapters/5Experiments/2/main}
\include{chapters/5Experiments/3/main}
\include{chapters/5Experiments/4/main}

\include{chapters/6Conclusion/main}

\appendix
\include{appendix/AppendixB}
\include{appendix/D/main}
\include{appendix/AppendixC}

\backmatter
\bibliographystyle{ecs}
\bibliography{ECS}
\end{document}
%% ----------------------------------------------------------------

 %% ----------------------------------------------------------------
%% Progress.tex
%% ---------------------------------------------------------------- 
\documentclass{ecsprogress}    % Use the progress Style
\graphicspath{{../figs/}}   % Location of your graphics files
    \usepackage{natbib}            % Use Natbib style for the refs.
\hypersetup{colorlinks=true}   % Set to false for black/white printing
\input{Definitions}            % Include your abbreviations



\usepackage{enumitem}% http://ctan.org/pkg/enumitem
\usepackage{multirow}
\usepackage{float}
\usepackage{amsmath}
\usepackage{multicol}
\usepackage{amssymb}
\usepackage[normalem]{ulem}
\useunder{\uline}{\ul}{}
\usepackage{wrapfig}


\usepackage[table,xcdraw]{xcolor}


%% ----------------------------------------------------------------
\begin{document}
\frontmatter
\title      {Heterogeneous Agent-based Model for Supermarket Competition}
\authors    {\texorpdfstring
             {\href{mailto:sc22g13@ecs.soton.ac.uk}{Stefan J. Collier}}
             {Stefan J. Collier}
            }
\addresses  {\groupname\\\deptname\\\univname}
\date       {\today}
\subject    {}
\keywords   {}
\supervisor {Dr. Maria Polukarov}
\examiner   {Professor Sheng Chen}

\maketitle
\begin{abstract}
This project aim was to model and analyse the effects of competitive pricing behaviors of grocery retailers on the British market. 

This was achieved by creating a multi-agent model, containing retailer and consumer agents. The heterogeneous crowd of retailers employs either a uniform pricing strategy or a ‘local price flexing’ strategy. The actions of these retailers are chosen by predicting the profit of each action, using a perceptron. Following on from the consideration of different economic models, a discrete model was developed so that software agents have a discrete environment to operate within. Within the model, it has been observed how supermarkets with differing behaviors affect a heterogeneous crowd of consumer agents. The model was implemented in Java with Python used to evaluate the results. 

The simulation displays good acceptance with real grocery market behavior, i.e. captures the performance of British retailers thus can be used to determine the impact of changes in their behavior on their competitors and consumers.Furthermore it can be used to provide insight into sustainability of volatile pricing strategies, providing a useful insight in volatility of British supermarket retail industry. 
\end{abstract}
\acknowledgements{
I would like to express my sincere gratitude to Dr Maria Polukarov for her guidance and support which provided me the freedom to take this research in the direction of my interest.\\
\\
I would also like to thank my family and friends for their encouragement and support. To those who quietly listened to my software complaints. To those who worked throughout the nights with me. To those who helped me write what I couldn't say. I cannot thank you enough.
}

\declaration{
I, Stefan Collier, declare that this dissertation and the work presented in it are my own and has been generated by me as the result of my own original research.\\
I confirm that:\\
1. This work was done wholly or mainly while in candidature for a degree at this University;\\
2. Where any part of this dissertation has previously been submitted for any other qualification at this University or any other institution, this has been clearly stated;\\
3. Where I have consulted the published work of others, this is always clearly attributed;\\
4. Where I have quoted from the work of others, the source is always given. With the exception of such quotations, this dissertation is entirely my own work;\\
5. I have acknowledged all main sources of help;\\
6. Where the thesis is based on work done by myself jointly with others, I have made clear exactly what was done by others and what I have contributed myself;\\
7. Either none of this work has been published before submission, or parts of this work have been published by :\\
\\
Stefan Collier\\
April 2016
}
\tableofcontents
\listoffigures
\listoftables

\mainmatter
%% ----------------------------------------------------------------
%\include{Introduction}
%\include{Conclusions}
\include{chapters/1Project/main}
\include{chapters/2Lit/main}
\include{chapters/3Design/HighLevel}
\include{chapters/3Design/InDepth}
\include{chapters/4Impl/main}

\include{chapters/5Experiments/1/main}
\include{chapters/5Experiments/2/main}
\include{chapters/5Experiments/3/main}
\include{chapters/5Experiments/4/main}

\include{chapters/6Conclusion/main}

\appendix
\include{appendix/AppendixB}
\include{appendix/D/main}
\include{appendix/AppendixC}

\backmatter
\bibliographystyle{ecs}
\bibliography{ECS}
\end{document}
%% ----------------------------------------------------------------

 %% ----------------------------------------------------------------
%% Progress.tex
%% ---------------------------------------------------------------- 
\documentclass{ecsprogress}    % Use the progress Style
\graphicspath{{../figs/}}   % Location of your graphics files
    \usepackage{natbib}            % Use Natbib style for the refs.
\hypersetup{colorlinks=true}   % Set to false for black/white printing
\input{Definitions}            % Include your abbreviations



\usepackage{enumitem}% http://ctan.org/pkg/enumitem
\usepackage{multirow}
\usepackage{float}
\usepackage{amsmath}
\usepackage{multicol}
\usepackage{amssymb}
\usepackage[normalem]{ulem}
\useunder{\uline}{\ul}{}
\usepackage{wrapfig}


\usepackage[table,xcdraw]{xcolor}


%% ----------------------------------------------------------------
\begin{document}
\frontmatter
\title      {Heterogeneous Agent-based Model for Supermarket Competition}
\authors    {\texorpdfstring
             {\href{mailto:sc22g13@ecs.soton.ac.uk}{Stefan J. Collier}}
             {Stefan J. Collier}
            }
\addresses  {\groupname\\\deptname\\\univname}
\date       {\today}
\subject    {}
\keywords   {}
\supervisor {Dr. Maria Polukarov}
\examiner   {Professor Sheng Chen}

\maketitle
\begin{abstract}
This project aim was to model and analyse the effects of competitive pricing behaviors of grocery retailers on the British market. 

This was achieved by creating a multi-agent model, containing retailer and consumer agents. The heterogeneous crowd of retailers employs either a uniform pricing strategy or a ‘local price flexing’ strategy. The actions of these retailers are chosen by predicting the profit of each action, using a perceptron. Following on from the consideration of different economic models, a discrete model was developed so that software agents have a discrete environment to operate within. Within the model, it has been observed how supermarkets with differing behaviors affect a heterogeneous crowd of consumer agents. The model was implemented in Java with Python used to evaluate the results. 

The simulation displays good acceptance with real grocery market behavior, i.e. captures the performance of British retailers thus can be used to determine the impact of changes in their behavior on their competitors and consumers.Furthermore it can be used to provide insight into sustainability of volatile pricing strategies, providing a useful insight in volatility of British supermarket retail industry. 
\end{abstract}
\acknowledgements{
I would like to express my sincere gratitude to Dr Maria Polukarov for her guidance and support which provided me the freedom to take this research in the direction of my interest.\\
\\
I would also like to thank my family and friends for their encouragement and support. To those who quietly listened to my software complaints. To those who worked throughout the nights with me. To those who helped me write what I couldn't say. I cannot thank you enough.
}

\declaration{
I, Stefan Collier, declare that this dissertation and the work presented in it are my own and has been generated by me as the result of my own original research.\\
I confirm that:\\
1. This work was done wholly or mainly while in candidature for a degree at this University;\\
2. Where any part of this dissertation has previously been submitted for any other qualification at this University or any other institution, this has been clearly stated;\\
3. Where I have consulted the published work of others, this is always clearly attributed;\\
4. Where I have quoted from the work of others, the source is always given. With the exception of such quotations, this dissertation is entirely my own work;\\
5. I have acknowledged all main sources of help;\\
6. Where the thesis is based on work done by myself jointly with others, I have made clear exactly what was done by others and what I have contributed myself;\\
7. Either none of this work has been published before submission, or parts of this work have been published by :\\
\\
Stefan Collier\\
April 2016
}
\tableofcontents
\listoffigures
\listoftables

\mainmatter
%% ----------------------------------------------------------------
%\include{Introduction}
%\include{Conclusions}
\include{chapters/1Project/main}
\include{chapters/2Lit/main}
\include{chapters/3Design/HighLevel}
\include{chapters/3Design/InDepth}
\include{chapters/4Impl/main}

\include{chapters/5Experiments/1/main}
\include{chapters/5Experiments/2/main}
\include{chapters/5Experiments/3/main}
\include{chapters/5Experiments/4/main}

\include{chapters/6Conclusion/main}

\appendix
\include{appendix/AppendixB}
\include{appendix/D/main}
\include{appendix/AppendixC}

\backmatter
\bibliographystyle{ecs}
\bibliography{ECS}
\end{document}
%% ----------------------------------------------------------------


 %% ----------------------------------------------------------------
%% Progress.tex
%% ---------------------------------------------------------------- 
\documentclass{ecsprogress}    % Use the progress Style
\graphicspath{{../figs/}}   % Location of your graphics files
    \usepackage{natbib}            % Use Natbib style for the refs.
\hypersetup{colorlinks=true}   % Set to false for black/white printing
\input{Definitions}            % Include your abbreviations



\usepackage{enumitem}% http://ctan.org/pkg/enumitem
\usepackage{multirow}
\usepackage{float}
\usepackage{amsmath}
\usepackage{multicol}
\usepackage{amssymb}
\usepackage[normalem]{ulem}
\useunder{\uline}{\ul}{}
\usepackage{wrapfig}


\usepackage[table,xcdraw]{xcolor}


%% ----------------------------------------------------------------
\begin{document}
\frontmatter
\title      {Heterogeneous Agent-based Model for Supermarket Competition}
\authors    {\texorpdfstring
             {\href{mailto:sc22g13@ecs.soton.ac.uk}{Stefan J. Collier}}
             {Stefan J. Collier}
            }
\addresses  {\groupname\\\deptname\\\univname}
\date       {\today}
\subject    {}
\keywords   {}
\supervisor {Dr. Maria Polukarov}
\examiner   {Professor Sheng Chen}

\maketitle
\begin{abstract}
This project aim was to model and analyse the effects of competitive pricing behaviors of grocery retailers on the British market. 

This was achieved by creating a multi-agent model, containing retailer and consumer agents. The heterogeneous crowd of retailers employs either a uniform pricing strategy or a ‘local price flexing’ strategy. The actions of these retailers are chosen by predicting the profit of each action, using a perceptron. Following on from the consideration of different economic models, a discrete model was developed so that software agents have a discrete environment to operate within. Within the model, it has been observed how supermarkets with differing behaviors affect a heterogeneous crowd of consumer agents. The model was implemented in Java with Python used to evaluate the results. 

The simulation displays good acceptance with real grocery market behavior, i.e. captures the performance of British retailers thus can be used to determine the impact of changes in their behavior on their competitors and consumers.Furthermore it can be used to provide insight into sustainability of volatile pricing strategies, providing a useful insight in volatility of British supermarket retail industry. 
\end{abstract}
\acknowledgements{
I would like to express my sincere gratitude to Dr Maria Polukarov for her guidance and support which provided me the freedom to take this research in the direction of my interest.\\
\\
I would also like to thank my family and friends for their encouragement and support. To those who quietly listened to my software complaints. To those who worked throughout the nights with me. To those who helped me write what I couldn't say. I cannot thank you enough.
}

\declaration{
I, Stefan Collier, declare that this dissertation and the work presented in it are my own and has been generated by me as the result of my own original research.\\
I confirm that:\\
1. This work was done wholly or mainly while in candidature for a degree at this University;\\
2. Where any part of this dissertation has previously been submitted for any other qualification at this University or any other institution, this has been clearly stated;\\
3. Where I have consulted the published work of others, this is always clearly attributed;\\
4. Where I have quoted from the work of others, the source is always given. With the exception of such quotations, this dissertation is entirely my own work;\\
5. I have acknowledged all main sources of help;\\
6. Where the thesis is based on work done by myself jointly with others, I have made clear exactly what was done by others and what I have contributed myself;\\
7. Either none of this work has been published before submission, or parts of this work have been published by :\\
\\
Stefan Collier\\
April 2016
}
\tableofcontents
\listoffigures
\listoftables

\mainmatter
%% ----------------------------------------------------------------
%\include{Introduction}
%\include{Conclusions}
\include{chapters/1Project/main}
\include{chapters/2Lit/main}
\include{chapters/3Design/HighLevel}
\include{chapters/3Design/InDepth}
\include{chapters/4Impl/main}

\include{chapters/5Experiments/1/main}
\include{chapters/5Experiments/2/main}
\include{chapters/5Experiments/3/main}
\include{chapters/5Experiments/4/main}

\include{chapters/6Conclusion/main}

\appendix
\include{appendix/AppendixB}
\include{appendix/D/main}
\include{appendix/AppendixC}

\backmatter
\bibliographystyle{ecs}
\bibliography{ECS}
\end{document}
%% ----------------------------------------------------------------


\appendix
\include{appendix/AppendixB}
 %% ----------------------------------------------------------------
%% Progress.tex
%% ---------------------------------------------------------------- 
\documentclass{ecsprogress}    % Use the progress Style
\graphicspath{{../figs/}}   % Location of your graphics files
    \usepackage{natbib}            % Use Natbib style for the refs.
\hypersetup{colorlinks=true}   % Set to false for black/white printing
\input{Definitions}            % Include your abbreviations



\usepackage{enumitem}% http://ctan.org/pkg/enumitem
\usepackage{multirow}
\usepackage{float}
\usepackage{amsmath}
\usepackage{multicol}
\usepackage{amssymb}
\usepackage[normalem]{ulem}
\useunder{\uline}{\ul}{}
\usepackage{wrapfig}


\usepackage[table,xcdraw]{xcolor}


%% ----------------------------------------------------------------
\begin{document}
\frontmatter
\title      {Heterogeneous Agent-based Model for Supermarket Competition}
\authors    {\texorpdfstring
             {\href{mailto:sc22g13@ecs.soton.ac.uk}{Stefan J. Collier}}
             {Stefan J. Collier}
            }
\addresses  {\groupname\\\deptname\\\univname}
\date       {\today}
\subject    {}
\keywords   {}
\supervisor {Dr. Maria Polukarov}
\examiner   {Professor Sheng Chen}

\maketitle
\begin{abstract}
This project aim was to model and analyse the effects of competitive pricing behaviors of grocery retailers on the British market. 

This was achieved by creating a multi-agent model, containing retailer and consumer agents. The heterogeneous crowd of retailers employs either a uniform pricing strategy or a ‘local price flexing’ strategy. The actions of these retailers are chosen by predicting the profit of each action, using a perceptron. Following on from the consideration of different economic models, a discrete model was developed so that software agents have a discrete environment to operate within. Within the model, it has been observed how supermarkets with differing behaviors affect a heterogeneous crowd of consumer agents. The model was implemented in Java with Python used to evaluate the results. 

The simulation displays good acceptance with real grocery market behavior, i.e. captures the performance of British retailers thus can be used to determine the impact of changes in their behavior on their competitors and consumers.Furthermore it can be used to provide insight into sustainability of volatile pricing strategies, providing a useful insight in volatility of British supermarket retail industry. 
\end{abstract}
\acknowledgements{
I would like to express my sincere gratitude to Dr Maria Polukarov for her guidance and support which provided me the freedom to take this research in the direction of my interest.\\
\\
I would also like to thank my family and friends for their encouragement and support. To those who quietly listened to my software complaints. To those who worked throughout the nights with me. To those who helped me write what I couldn't say. I cannot thank you enough.
}

\declaration{
I, Stefan Collier, declare that this dissertation and the work presented in it are my own and has been generated by me as the result of my own original research.\\
I confirm that:\\
1. This work was done wholly or mainly while in candidature for a degree at this University;\\
2. Where any part of this dissertation has previously been submitted for any other qualification at this University or any other institution, this has been clearly stated;\\
3. Where I have consulted the published work of others, this is always clearly attributed;\\
4. Where I have quoted from the work of others, the source is always given. With the exception of such quotations, this dissertation is entirely my own work;\\
5. I have acknowledged all main sources of help;\\
6. Where the thesis is based on work done by myself jointly with others, I have made clear exactly what was done by others and what I have contributed myself;\\
7. Either none of this work has been published before submission, or parts of this work have been published by :\\
\\
Stefan Collier\\
April 2016
}
\tableofcontents
\listoffigures
\listoftables

\mainmatter
%% ----------------------------------------------------------------
%\include{Introduction}
%\include{Conclusions}
\include{chapters/1Project/main}
\include{chapters/2Lit/main}
\include{chapters/3Design/HighLevel}
\include{chapters/3Design/InDepth}
\include{chapters/4Impl/main}

\include{chapters/5Experiments/1/main}
\include{chapters/5Experiments/2/main}
\include{chapters/5Experiments/3/main}
\include{chapters/5Experiments/4/main}

\include{chapters/6Conclusion/main}

\appendix
\include{appendix/AppendixB}
\include{appendix/D/main}
\include{appendix/AppendixC}

\backmatter
\bibliographystyle{ecs}
\bibliography{ECS}
\end{document}
%% ----------------------------------------------------------------

\include{appendix/AppendixC}

\backmatter
\bibliographystyle{ecs}
\bibliography{ECS}
\end{document}
%% ----------------------------------------------------------------


 %% ----------------------------------------------------------------
%% Progress.tex
%% ---------------------------------------------------------------- 
\documentclass{ecsprogress}    % Use the progress Style
\graphicspath{{../figs/}}   % Location of your graphics files
    \usepackage{natbib}            % Use Natbib style for the refs.
\hypersetup{colorlinks=true}   % Set to false for black/white printing
\input{Definitions}            % Include your abbreviations



\usepackage{enumitem}% http://ctan.org/pkg/enumitem
\usepackage{multirow}
\usepackage{float}
\usepackage{amsmath}
\usepackage{multicol}
\usepackage{amssymb}
\usepackage[normalem]{ulem}
\useunder{\uline}{\ul}{}
\usepackage{wrapfig}


\usepackage[table,xcdraw]{xcolor}


%% ----------------------------------------------------------------
\begin{document}
\frontmatter
\title      {Heterogeneous Agent-based Model for Supermarket Competition}
\authors    {\texorpdfstring
             {\href{mailto:sc22g13@ecs.soton.ac.uk}{Stefan J. Collier}}
             {Stefan J. Collier}
            }
\addresses  {\groupname\\\deptname\\\univname}
\date       {\today}
\subject    {}
\keywords   {}
\supervisor {Dr. Maria Polukarov}
\examiner   {Professor Sheng Chen}

\maketitle
\begin{abstract}
This project aim was to model and analyse the effects of competitive pricing behaviors of grocery retailers on the British market. 

This was achieved by creating a multi-agent model, containing retailer and consumer agents. The heterogeneous crowd of retailers employs either a uniform pricing strategy or a ‘local price flexing’ strategy. The actions of these retailers are chosen by predicting the profit of each action, using a perceptron. Following on from the consideration of different economic models, a discrete model was developed so that software agents have a discrete environment to operate within. Within the model, it has been observed how supermarkets with differing behaviors affect a heterogeneous crowd of consumer agents. The model was implemented in Java with Python used to evaluate the results. 

The simulation displays good acceptance with real grocery market behavior, i.e. captures the performance of British retailers thus can be used to determine the impact of changes in their behavior on their competitors and consumers.Furthermore it can be used to provide insight into sustainability of volatile pricing strategies, providing a useful insight in volatility of British supermarket retail industry. 
\end{abstract}
\acknowledgements{
I would like to express my sincere gratitude to Dr Maria Polukarov for her guidance and support which provided me the freedom to take this research in the direction of my interest.\\
\\
I would also like to thank my family and friends for their encouragement and support. To those who quietly listened to my software complaints. To those who worked throughout the nights with me. To those who helped me write what I couldn't say. I cannot thank you enough.
}

\declaration{
I, Stefan Collier, declare that this dissertation and the work presented in it are my own and has been generated by me as the result of my own original research.\\
I confirm that:\\
1. This work was done wholly or mainly while in candidature for a degree at this University;\\
2. Where any part of this dissertation has previously been submitted for any other qualification at this University or any other institution, this has been clearly stated;\\
3. Where I have consulted the published work of others, this is always clearly attributed;\\
4. Where I have quoted from the work of others, the source is always given. With the exception of such quotations, this dissertation is entirely my own work;\\
5. I have acknowledged all main sources of help;\\
6. Where the thesis is based on work done by myself jointly with others, I have made clear exactly what was done by others and what I have contributed myself;\\
7. Either none of this work has been published before submission, or parts of this work have been published by :\\
\\
Stefan Collier\\
April 2016
}
\tableofcontents
\listoffigures
\listoftables

\mainmatter
%% ----------------------------------------------------------------
%\include{Introduction}
%\include{Conclusions}
 %% ----------------------------------------------------------------
%% Progress.tex
%% ---------------------------------------------------------------- 
\documentclass{ecsprogress}    % Use the progress Style
\graphicspath{{../figs/}}   % Location of your graphics files
    \usepackage{natbib}            % Use Natbib style for the refs.
\hypersetup{colorlinks=true}   % Set to false for black/white printing
\input{Definitions}            % Include your abbreviations



\usepackage{enumitem}% http://ctan.org/pkg/enumitem
\usepackage{multirow}
\usepackage{float}
\usepackage{amsmath}
\usepackage{multicol}
\usepackage{amssymb}
\usepackage[normalem]{ulem}
\useunder{\uline}{\ul}{}
\usepackage{wrapfig}


\usepackage[table,xcdraw]{xcolor}


%% ----------------------------------------------------------------
\begin{document}
\frontmatter
\title      {Heterogeneous Agent-based Model for Supermarket Competition}
\authors    {\texorpdfstring
             {\href{mailto:sc22g13@ecs.soton.ac.uk}{Stefan J. Collier}}
             {Stefan J. Collier}
            }
\addresses  {\groupname\\\deptname\\\univname}
\date       {\today}
\subject    {}
\keywords   {}
\supervisor {Dr. Maria Polukarov}
\examiner   {Professor Sheng Chen}

\maketitle
\begin{abstract}
This project aim was to model and analyse the effects of competitive pricing behaviors of grocery retailers on the British market. 

This was achieved by creating a multi-agent model, containing retailer and consumer agents. The heterogeneous crowd of retailers employs either a uniform pricing strategy or a ‘local price flexing’ strategy. The actions of these retailers are chosen by predicting the profit of each action, using a perceptron. Following on from the consideration of different economic models, a discrete model was developed so that software agents have a discrete environment to operate within. Within the model, it has been observed how supermarkets with differing behaviors affect a heterogeneous crowd of consumer agents. The model was implemented in Java with Python used to evaluate the results. 

The simulation displays good acceptance with real grocery market behavior, i.e. captures the performance of British retailers thus can be used to determine the impact of changes in their behavior on their competitors and consumers.Furthermore it can be used to provide insight into sustainability of volatile pricing strategies, providing a useful insight in volatility of British supermarket retail industry. 
\end{abstract}
\acknowledgements{
I would like to express my sincere gratitude to Dr Maria Polukarov for her guidance and support which provided me the freedom to take this research in the direction of my interest.\\
\\
I would also like to thank my family and friends for their encouragement and support. To those who quietly listened to my software complaints. To those who worked throughout the nights with me. To those who helped me write what I couldn't say. I cannot thank you enough.
}

\declaration{
I, Stefan Collier, declare that this dissertation and the work presented in it are my own and has been generated by me as the result of my own original research.\\
I confirm that:\\
1. This work was done wholly or mainly while in candidature for a degree at this University;\\
2. Where any part of this dissertation has previously been submitted for any other qualification at this University or any other institution, this has been clearly stated;\\
3. Where I have consulted the published work of others, this is always clearly attributed;\\
4. Where I have quoted from the work of others, the source is always given. With the exception of such quotations, this dissertation is entirely my own work;\\
5. I have acknowledged all main sources of help;\\
6. Where the thesis is based on work done by myself jointly with others, I have made clear exactly what was done by others and what I have contributed myself;\\
7. Either none of this work has been published before submission, or parts of this work have been published by :\\
\\
Stefan Collier\\
April 2016
}
\tableofcontents
\listoffigures
\listoftables

\mainmatter
%% ----------------------------------------------------------------
%\include{Introduction}
%\include{Conclusions}
\include{chapters/1Project/main}
\include{chapters/2Lit/main}
\include{chapters/3Design/HighLevel}
\include{chapters/3Design/InDepth}
\include{chapters/4Impl/main}

\include{chapters/5Experiments/1/main}
\include{chapters/5Experiments/2/main}
\include{chapters/5Experiments/3/main}
\include{chapters/5Experiments/4/main}

\include{chapters/6Conclusion/main}

\appendix
\include{appendix/AppendixB}
\include{appendix/D/main}
\include{appendix/AppendixC}

\backmatter
\bibliographystyle{ecs}
\bibliography{ECS}
\end{document}
%% ----------------------------------------------------------------

 %% ----------------------------------------------------------------
%% Progress.tex
%% ---------------------------------------------------------------- 
\documentclass{ecsprogress}    % Use the progress Style
\graphicspath{{../figs/}}   % Location of your graphics files
    \usepackage{natbib}            % Use Natbib style for the refs.
\hypersetup{colorlinks=true}   % Set to false for black/white printing
\input{Definitions}            % Include your abbreviations



\usepackage{enumitem}% http://ctan.org/pkg/enumitem
\usepackage{multirow}
\usepackage{float}
\usepackage{amsmath}
\usepackage{multicol}
\usepackage{amssymb}
\usepackage[normalem]{ulem}
\useunder{\uline}{\ul}{}
\usepackage{wrapfig}


\usepackage[table,xcdraw]{xcolor}


%% ----------------------------------------------------------------
\begin{document}
\frontmatter
\title      {Heterogeneous Agent-based Model for Supermarket Competition}
\authors    {\texorpdfstring
             {\href{mailto:sc22g13@ecs.soton.ac.uk}{Stefan J. Collier}}
             {Stefan J. Collier}
            }
\addresses  {\groupname\\\deptname\\\univname}
\date       {\today}
\subject    {}
\keywords   {}
\supervisor {Dr. Maria Polukarov}
\examiner   {Professor Sheng Chen}

\maketitle
\begin{abstract}
This project aim was to model and analyse the effects of competitive pricing behaviors of grocery retailers on the British market. 

This was achieved by creating a multi-agent model, containing retailer and consumer agents. The heterogeneous crowd of retailers employs either a uniform pricing strategy or a ‘local price flexing’ strategy. The actions of these retailers are chosen by predicting the profit of each action, using a perceptron. Following on from the consideration of different economic models, a discrete model was developed so that software agents have a discrete environment to operate within. Within the model, it has been observed how supermarkets with differing behaviors affect a heterogeneous crowd of consumer agents. The model was implemented in Java with Python used to evaluate the results. 

The simulation displays good acceptance with real grocery market behavior, i.e. captures the performance of British retailers thus can be used to determine the impact of changes in their behavior on their competitors and consumers.Furthermore it can be used to provide insight into sustainability of volatile pricing strategies, providing a useful insight in volatility of British supermarket retail industry. 
\end{abstract}
\acknowledgements{
I would like to express my sincere gratitude to Dr Maria Polukarov for her guidance and support which provided me the freedom to take this research in the direction of my interest.\\
\\
I would also like to thank my family and friends for their encouragement and support. To those who quietly listened to my software complaints. To those who worked throughout the nights with me. To those who helped me write what I couldn't say. I cannot thank you enough.
}

\declaration{
I, Stefan Collier, declare that this dissertation and the work presented in it are my own and has been generated by me as the result of my own original research.\\
I confirm that:\\
1. This work was done wholly or mainly while in candidature for a degree at this University;\\
2. Where any part of this dissertation has previously been submitted for any other qualification at this University or any other institution, this has been clearly stated;\\
3. Where I have consulted the published work of others, this is always clearly attributed;\\
4. Where I have quoted from the work of others, the source is always given. With the exception of such quotations, this dissertation is entirely my own work;\\
5. I have acknowledged all main sources of help;\\
6. Where the thesis is based on work done by myself jointly with others, I have made clear exactly what was done by others and what I have contributed myself;\\
7. Either none of this work has been published before submission, or parts of this work have been published by :\\
\\
Stefan Collier\\
April 2016
}
\tableofcontents
\listoffigures
\listoftables

\mainmatter
%% ----------------------------------------------------------------
%\include{Introduction}
%\include{Conclusions}
\include{chapters/1Project/main}
\include{chapters/2Lit/main}
\include{chapters/3Design/HighLevel}
\include{chapters/3Design/InDepth}
\include{chapters/4Impl/main}

\include{chapters/5Experiments/1/main}
\include{chapters/5Experiments/2/main}
\include{chapters/5Experiments/3/main}
\include{chapters/5Experiments/4/main}

\include{chapters/6Conclusion/main}

\appendix
\include{appendix/AppendixB}
\include{appendix/D/main}
\include{appendix/AppendixC}

\backmatter
\bibliographystyle{ecs}
\bibliography{ECS}
\end{document}
%% ----------------------------------------------------------------

\include{chapters/3Design/HighLevel}
\include{chapters/3Design/InDepth}
 %% ----------------------------------------------------------------
%% Progress.tex
%% ---------------------------------------------------------------- 
\documentclass{ecsprogress}    % Use the progress Style
\graphicspath{{../figs/}}   % Location of your graphics files
    \usepackage{natbib}            % Use Natbib style for the refs.
\hypersetup{colorlinks=true}   % Set to false for black/white printing
\input{Definitions}            % Include your abbreviations



\usepackage{enumitem}% http://ctan.org/pkg/enumitem
\usepackage{multirow}
\usepackage{float}
\usepackage{amsmath}
\usepackage{multicol}
\usepackage{amssymb}
\usepackage[normalem]{ulem}
\useunder{\uline}{\ul}{}
\usepackage{wrapfig}


\usepackage[table,xcdraw]{xcolor}


%% ----------------------------------------------------------------
\begin{document}
\frontmatter
\title      {Heterogeneous Agent-based Model for Supermarket Competition}
\authors    {\texorpdfstring
             {\href{mailto:sc22g13@ecs.soton.ac.uk}{Stefan J. Collier}}
             {Stefan J. Collier}
            }
\addresses  {\groupname\\\deptname\\\univname}
\date       {\today}
\subject    {}
\keywords   {}
\supervisor {Dr. Maria Polukarov}
\examiner   {Professor Sheng Chen}

\maketitle
\begin{abstract}
This project aim was to model and analyse the effects of competitive pricing behaviors of grocery retailers on the British market. 

This was achieved by creating a multi-agent model, containing retailer and consumer agents. The heterogeneous crowd of retailers employs either a uniform pricing strategy or a ‘local price flexing’ strategy. The actions of these retailers are chosen by predicting the profit of each action, using a perceptron. Following on from the consideration of different economic models, a discrete model was developed so that software agents have a discrete environment to operate within. Within the model, it has been observed how supermarkets with differing behaviors affect a heterogeneous crowd of consumer agents. The model was implemented in Java with Python used to evaluate the results. 

The simulation displays good acceptance with real grocery market behavior, i.e. captures the performance of British retailers thus can be used to determine the impact of changes in their behavior on their competitors and consumers.Furthermore it can be used to provide insight into sustainability of volatile pricing strategies, providing a useful insight in volatility of British supermarket retail industry. 
\end{abstract}
\acknowledgements{
I would like to express my sincere gratitude to Dr Maria Polukarov for her guidance and support which provided me the freedom to take this research in the direction of my interest.\\
\\
I would also like to thank my family and friends for their encouragement and support. To those who quietly listened to my software complaints. To those who worked throughout the nights with me. To those who helped me write what I couldn't say. I cannot thank you enough.
}

\declaration{
I, Stefan Collier, declare that this dissertation and the work presented in it are my own and has been generated by me as the result of my own original research.\\
I confirm that:\\
1. This work was done wholly or mainly while in candidature for a degree at this University;\\
2. Where any part of this dissertation has previously been submitted for any other qualification at this University or any other institution, this has been clearly stated;\\
3. Where I have consulted the published work of others, this is always clearly attributed;\\
4. Where I have quoted from the work of others, the source is always given. With the exception of such quotations, this dissertation is entirely my own work;\\
5. I have acknowledged all main sources of help;\\
6. Where the thesis is based on work done by myself jointly with others, I have made clear exactly what was done by others and what I have contributed myself;\\
7. Either none of this work has been published before submission, or parts of this work have been published by :\\
\\
Stefan Collier\\
April 2016
}
\tableofcontents
\listoffigures
\listoftables

\mainmatter
%% ----------------------------------------------------------------
%\include{Introduction}
%\include{Conclusions}
\include{chapters/1Project/main}
\include{chapters/2Lit/main}
\include{chapters/3Design/HighLevel}
\include{chapters/3Design/InDepth}
\include{chapters/4Impl/main}

\include{chapters/5Experiments/1/main}
\include{chapters/5Experiments/2/main}
\include{chapters/5Experiments/3/main}
\include{chapters/5Experiments/4/main}

\include{chapters/6Conclusion/main}

\appendix
\include{appendix/AppendixB}
\include{appendix/D/main}
\include{appendix/AppendixC}

\backmatter
\bibliographystyle{ecs}
\bibliography{ECS}
\end{document}
%% ----------------------------------------------------------------


 %% ----------------------------------------------------------------
%% Progress.tex
%% ---------------------------------------------------------------- 
\documentclass{ecsprogress}    % Use the progress Style
\graphicspath{{../figs/}}   % Location of your graphics files
    \usepackage{natbib}            % Use Natbib style for the refs.
\hypersetup{colorlinks=true}   % Set to false for black/white printing
\input{Definitions}            % Include your abbreviations



\usepackage{enumitem}% http://ctan.org/pkg/enumitem
\usepackage{multirow}
\usepackage{float}
\usepackage{amsmath}
\usepackage{multicol}
\usepackage{amssymb}
\usepackage[normalem]{ulem}
\useunder{\uline}{\ul}{}
\usepackage{wrapfig}


\usepackage[table,xcdraw]{xcolor}


%% ----------------------------------------------------------------
\begin{document}
\frontmatter
\title      {Heterogeneous Agent-based Model for Supermarket Competition}
\authors    {\texorpdfstring
             {\href{mailto:sc22g13@ecs.soton.ac.uk}{Stefan J. Collier}}
             {Stefan J. Collier}
            }
\addresses  {\groupname\\\deptname\\\univname}
\date       {\today}
\subject    {}
\keywords   {}
\supervisor {Dr. Maria Polukarov}
\examiner   {Professor Sheng Chen}

\maketitle
\begin{abstract}
This project aim was to model and analyse the effects of competitive pricing behaviors of grocery retailers on the British market. 

This was achieved by creating a multi-agent model, containing retailer and consumer agents. The heterogeneous crowd of retailers employs either a uniform pricing strategy or a ‘local price flexing’ strategy. The actions of these retailers are chosen by predicting the profit of each action, using a perceptron. Following on from the consideration of different economic models, a discrete model was developed so that software agents have a discrete environment to operate within. Within the model, it has been observed how supermarkets with differing behaviors affect a heterogeneous crowd of consumer agents. The model was implemented in Java with Python used to evaluate the results. 

The simulation displays good acceptance with real grocery market behavior, i.e. captures the performance of British retailers thus can be used to determine the impact of changes in their behavior on their competitors and consumers.Furthermore it can be used to provide insight into sustainability of volatile pricing strategies, providing a useful insight in volatility of British supermarket retail industry. 
\end{abstract}
\acknowledgements{
I would like to express my sincere gratitude to Dr Maria Polukarov for her guidance and support which provided me the freedom to take this research in the direction of my interest.\\
\\
I would also like to thank my family and friends for their encouragement and support. To those who quietly listened to my software complaints. To those who worked throughout the nights with me. To those who helped me write what I couldn't say. I cannot thank you enough.
}

\declaration{
I, Stefan Collier, declare that this dissertation and the work presented in it are my own and has been generated by me as the result of my own original research.\\
I confirm that:\\
1. This work was done wholly or mainly while in candidature for a degree at this University;\\
2. Where any part of this dissertation has previously been submitted for any other qualification at this University or any other institution, this has been clearly stated;\\
3. Where I have consulted the published work of others, this is always clearly attributed;\\
4. Where I have quoted from the work of others, the source is always given. With the exception of such quotations, this dissertation is entirely my own work;\\
5. I have acknowledged all main sources of help;\\
6. Where the thesis is based on work done by myself jointly with others, I have made clear exactly what was done by others and what I have contributed myself;\\
7. Either none of this work has been published before submission, or parts of this work have been published by :\\
\\
Stefan Collier\\
April 2016
}
\tableofcontents
\listoffigures
\listoftables

\mainmatter
%% ----------------------------------------------------------------
%\include{Introduction}
%\include{Conclusions}
\include{chapters/1Project/main}
\include{chapters/2Lit/main}
\include{chapters/3Design/HighLevel}
\include{chapters/3Design/InDepth}
\include{chapters/4Impl/main}

\include{chapters/5Experiments/1/main}
\include{chapters/5Experiments/2/main}
\include{chapters/5Experiments/3/main}
\include{chapters/5Experiments/4/main}

\include{chapters/6Conclusion/main}

\appendix
\include{appendix/AppendixB}
\include{appendix/D/main}
\include{appendix/AppendixC}

\backmatter
\bibliographystyle{ecs}
\bibliography{ECS}
\end{document}
%% ----------------------------------------------------------------

 %% ----------------------------------------------------------------
%% Progress.tex
%% ---------------------------------------------------------------- 
\documentclass{ecsprogress}    % Use the progress Style
\graphicspath{{../figs/}}   % Location of your graphics files
    \usepackage{natbib}            % Use Natbib style for the refs.
\hypersetup{colorlinks=true}   % Set to false for black/white printing
\input{Definitions}            % Include your abbreviations



\usepackage{enumitem}% http://ctan.org/pkg/enumitem
\usepackage{multirow}
\usepackage{float}
\usepackage{amsmath}
\usepackage{multicol}
\usepackage{amssymb}
\usepackage[normalem]{ulem}
\useunder{\uline}{\ul}{}
\usepackage{wrapfig}


\usepackage[table,xcdraw]{xcolor}


%% ----------------------------------------------------------------
\begin{document}
\frontmatter
\title      {Heterogeneous Agent-based Model for Supermarket Competition}
\authors    {\texorpdfstring
             {\href{mailto:sc22g13@ecs.soton.ac.uk}{Stefan J. Collier}}
             {Stefan J. Collier}
            }
\addresses  {\groupname\\\deptname\\\univname}
\date       {\today}
\subject    {}
\keywords   {}
\supervisor {Dr. Maria Polukarov}
\examiner   {Professor Sheng Chen}

\maketitle
\begin{abstract}
This project aim was to model and analyse the effects of competitive pricing behaviors of grocery retailers on the British market. 

This was achieved by creating a multi-agent model, containing retailer and consumer agents. The heterogeneous crowd of retailers employs either a uniform pricing strategy or a ‘local price flexing’ strategy. The actions of these retailers are chosen by predicting the profit of each action, using a perceptron. Following on from the consideration of different economic models, a discrete model was developed so that software agents have a discrete environment to operate within. Within the model, it has been observed how supermarkets with differing behaviors affect a heterogeneous crowd of consumer agents. The model was implemented in Java with Python used to evaluate the results. 

The simulation displays good acceptance with real grocery market behavior, i.e. captures the performance of British retailers thus can be used to determine the impact of changes in their behavior on their competitors and consumers.Furthermore it can be used to provide insight into sustainability of volatile pricing strategies, providing a useful insight in volatility of British supermarket retail industry. 
\end{abstract}
\acknowledgements{
I would like to express my sincere gratitude to Dr Maria Polukarov for her guidance and support which provided me the freedom to take this research in the direction of my interest.\\
\\
I would also like to thank my family and friends for their encouragement and support. To those who quietly listened to my software complaints. To those who worked throughout the nights with me. To those who helped me write what I couldn't say. I cannot thank you enough.
}

\declaration{
I, Stefan Collier, declare that this dissertation and the work presented in it are my own and has been generated by me as the result of my own original research.\\
I confirm that:\\
1. This work was done wholly or mainly while in candidature for a degree at this University;\\
2. Where any part of this dissertation has previously been submitted for any other qualification at this University or any other institution, this has been clearly stated;\\
3. Where I have consulted the published work of others, this is always clearly attributed;\\
4. Where I have quoted from the work of others, the source is always given. With the exception of such quotations, this dissertation is entirely my own work;\\
5. I have acknowledged all main sources of help;\\
6. Where the thesis is based on work done by myself jointly with others, I have made clear exactly what was done by others and what I have contributed myself;\\
7. Either none of this work has been published before submission, or parts of this work have been published by :\\
\\
Stefan Collier\\
April 2016
}
\tableofcontents
\listoffigures
\listoftables

\mainmatter
%% ----------------------------------------------------------------
%\include{Introduction}
%\include{Conclusions}
\include{chapters/1Project/main}
\include{chapters/2Lit/main}
\include{chapters/3Design/HighLevel}
\include{chapters/3Design/InDepth}
\include{chapters/4Impl/main}

\include{chapters/5Experiments/1/main}
\include{chapters/5Experiments/2/main}
\include{chapters/5Experiments/3/main}
\include{chapters/5Experiments/4/main}

\include{chapters/6Conclusion/main}

\appendix
\include{appendix/AppendixB}
\include{appendix/D/main}
\include{appendix/AppendixC}

\backmatter
\bibliographystyle{ecs}
\bibliography{ECS}
\end{document}
%% ----------------------------------------------------------------

 %% ----------------------------------------------------------------
%% Progress.tex
%% ---------------------------------------------------------------- 
\documentclass{ecsprogress}    % Use the progress Style
\graphicspath{{../figs/}}   % Location of your graphics files
    \usepackage{natbib}            % Use Natbib style for the refs.
\hypersetup{colorlinks=true}   % Set to false for black/white printing
\input{Definitions}            % Include your abbreviations



\usepackage{enumitem}% http://ctan.org/pkg/enumitem
\usepackage{multirow}
\usepackage{float}
\usepackage{amsmath}
\usepackage{multicol}
\usepackage{amssymb}
\usepackage[normalem]{ulem}
\useunder{\uline}{\ul}{}
\usepackage{wrapfig}


\usepackage[table,xcdraw]{xcolor}


%% ----------------------------------------------------------------
\begin{document}
\frontmatter
\title      {Heterogeneous Agent-based Model for Supermarket Competition}
\authors    {\texorpdfstring
             {\href{mailto:sc22g13@ecs.soton.ac.uk}{Stefan J. Collier}}
             {Stefan J. Collier}
            }
\addresses  {\groupname\\\deptname\\\univname}
\date       {\today}
\subject    {}
\keywords   {}
\supervisor {Dr. Maria Polukarov}
\examiner   {Professor Sheng Chen}

\maketitle
\begin{abstract}
This project aim was to model and analyse the effects of competitive pricing behaviors of grocery retailers on the British market. 

This was achieved by creating a multi-agent model, containing retailer and consumer agents. The heterogeneous crowd of retailers employs either a uniform pricing strategy or a ‘local price flexing’ strategy. The actions of these retailers are chosen by predicting the profit of each action, using a perceptron. Following on from the consideration of different economic models, a discrete model was developed so that software agents have a discrete environment to operate within. Within the model, it has been observed how supermarkets with differing behaviors affect a heterogeneous crowd of consumer agents. The model was implemented in Java with Python used to evaluate the results. 

The simulation displays good acceptance with real grocery market behavior, i.e. captures the performance of British retailers thus can be used to determine the impact of changes in their behavior on their competitors and consumers.Furthermore it can be used to provide insight into sustainability of volatile pricing strategies, providing a useful insight in volatility of British supermarket retail industry. 
\end{abstract}
\acknowledgements{
I would like to express my sincere gratitude to Dr Maria Polukarov for her guidance and support which provided me the freedom to take this research in the direction of my interest.\\
\\
I would also like to thank my family and friends for their encouragement and support. To those who quietly listened to my software complaints. To those who worked throughout the nights with me. To those who helped me write what I couldn't say. I cannot thank you enough.
}

\declaration{
I, Stefan Collier, declare that this dissertation and the work presented in it are my own and has been generated by me as the result of my own original research.\\
I confirm that:\\
1. This work was done wholly or mainly while in candidature for a degree at this University;\\
2. Where any part of this dissertation has previously been submitted for any other qualification at this University or any other institution, this has been clearly stated;\\
3. Where I have consulted the published work of others, this is always clearly attributed;\\
4. Where I have quoted from the work of others, the source is always given. With the exception of such quotations, this dissertation is entirely my own work;\\
5. I have acknowledged all main sources of help;\\
6. Where the thesis is based on work done by myself jointly with others, I have made clear exactly what was done by others and what I have contributed myself;\\
7. Either none of this work has been published before submission, or parts of this work have been published by :\\
\\
Stefan Collier\\
April 2016
}
\tableofcontents
\listoffigures
\listoftables

\mainmatter
%% ----------------------------------------------------------------
%\include{Introduction}
%\include{Conclusions}
\include{chapters/1Project/main}
\include{chapters/2Lit/main}
\include{chapters/3Design/HighLevel}
\include{chapters/3Design/InDepth}
\include{chapters/4Impl/main}

\include{chapters/5Experiments/1/main}
\include{chapters/5Experiments/2/main}
\include{chapters/5Experiments/3/main}
\include{chapters/5Experiments/4/main}

\include{chapters/6Conclusion/main}

\appendix
\include{appendix/AppendixB}
\include{appendix/D/main}
\include{appendix/AppendixC}

\backmatter
\bibliographystyle{ecs}
\bibliography{ECS}
\end{document}
%% ----------------------------------------------------------------

 %% ----------------------------------------------------------------
%% Progress.tex
%% ---------------------------------------------------------------- 
\documentclass{ecsprogress}    % Use the progress Style
\graphicspath{{../figs/}}   % Location of your graphics files
    \usepackage{natbib}            % Use Natbib style for the refs.
\hypersetup{colorlinks=true}   % Set to false for black/white printing
\input{Definitions}            % Include your abbreviations



\usepackage{enumitem}% http://ctan.org/pkg/enumitem
\usepackage{multirow}
\usepackage{float}
\usepackage{amsmath}
\usepackage{multicol}
\usepackage{amssymb}
\usepackage[normalem]{ulem}
\useunder{\uline}{\ul}{}
\usepackage{wrapfig}


\usepackage[table,xcdraw]{xcolor}


%% ----------------------------------------------------------------
\begin{document}
\frontmatter
\title      {Heterogeneous Agent-based Model for Supermarket Competition}
\authors    {\texorpdfstring
             {\href{mailto:sc22g13@ecs.soton.ac.uk}{Stefan J. Collier}}
             {Stefan J. Collier}
            }
\addresses  {\groupname\\\deptname\\\univname}
\date       {\today}
\subject    {}
\keywords   {}
\supervisor {Dr. Maria Polukarov}
\examiner   {Professor Sheng Chen}

\maketitle
\begin{abstract}
This project aim was to model and analyse the effects of competitive pricing behaviors of grocery retailers on the British market. 

This was achieved by creating a multi-agent model, containing retailer and consumer agents. The heterogeneous crowd of retailers employs either a uniform pricing strategy or a ‘local price flexing’ strategy. The actions of these retailers are chosen by predicting the profit of each action, using a perceptron. Following on from the consideration of different economic models, a discrete model was developed so that software agents have a discrete environment to operate within. Within the model, it has been observed how supermarkets with differing behaviors affect a heterogeneous crowd of consumer agents. The model was implemented in Java with Python used to evaluate the results. 

The simulation displays good acceptance with real grocery market behavior, i.e. captures the performance of British retailers thus can be used to determine the impact of changes in their behavior on their competitors and consumers.Furthermore it can be used to provide insight into sustainability of volatile pricing strategies, providing a useful insight in volatility of British supermarket retail industry. 
\end{abstract}
\acknowledgements{
I would like to express my sincere gratitude to Dr Maria Polukarov for her guidance and support which provided me the freedom to take this research in the direction of my interest.\\
\\
I would also like to thank my family and friends for their encouragement and support. To those who quietly listened to my software complaints. To those who worked throughout the nights with me. To those who helped me write what I couldn't say. I cannot thank you enough.
}

\declaration{
I, Stefan Collier, declare that this dissertation and the work presented in it are my own and has been generated by me as the result of my own original research.\\
I confirm that:\\
1. This work was done wholly or mainly while in candidature for a degree at this University;\\
2. Where any part of this dissertation has previously been submitted for any other qualification at this University or any other institution, this has been clearly stated;\\
3. Where I have consulted the published work of others, this is always clearly attributed;\\
4. Where I have quoted from the work of others, the source is always given. With the exception of such quotations, this dissertation is entirely my own work;\\
5. I have acknowledged all main sources of help;\\
6. Where the thesis is based on work done by myself jointly with others, I have made clear exactly what was done by others and what I have contributed myself;\\
7. Either none of this work has been published before submission, or parts of this work have been published by :\\
\\
Stefan Collier\\
April 2016
}
\tableofcontents
\listoffigures
\listoftables

\mainmatter
%% ----------------------------------------------------------------
%\include{Introduction}
%\include{Conclusions}
\include{chapters/1Project/main}
\include{chapters/2Lit/main}
\include{chapters/3Design/HighLevel}
\include{chapters/3Design/InDepth}
\include{chapters/4Impl/main}

\include{chapters/5Experiments/1/main}
\include{chapters/5Experiments/2/main}
\include{chapters/5Experiments/3/main}
\include{chapters/5Experiments/4/main}

\include{chapters/6Conclusion/main}

\appendix
\include{appendix/AppendixB}
\include{appendix/D/main}
\include{appendix/AppendixC}

\backmatter
\bibliographystyle{ecs}
\bibliography{ECS}
\end{document}
%% ----------------------------------------------------------------


 %% ----------------------------------------------------------------
%% Progress.tex
%% ---------------------------------------------------------------- 
\documentclass{ecsprogress}    % Use the progress Style
\graphicspath{{../figs/}}   % Location of your graphics files
    \usepackage{natbib}            % Use Natbib style for the refs.
\hypersetup{colorlinks=true}   % Set to false for black/white printing
\input{Definitions}            % Include your abbreviations



\usepackage{enumitem}% http://ctan.org/pkg/enumitem
\usepackage{multirow}
\usepackage{float}
\usepackage{amsmath}
\usepackage{multicol}
\usepackage{amssymb}
\usepackage[normalem]{ulem}
\useunder{\uline}{\ul}{}
\usepackage{wrapfig}


\usepackage[table,xcdraw]{xcolor}


%% ----------------------------------------------------------------
\begin{document}
\frontmatter
\title      {Heterogeneous Agent-based Model for Supermarket Competition}
\authors    {\texorpdfstring
             {\href{mailto:sc22g13@ecs.soton.ac.uk}{Stefan J. Collier}}
             {Stefan J. Collier}
            }
\addresses  {\groupname\\\deptname\\\univname}
\date       {\today}
\subject    {}
\keywords   {}
\supervisor {Dr. Maria Polukarov}
\examiner   {Professor Sheng Chen}

\maketitle
\begin{abstract}
This project aim was to model and analyse the effects of competitive pricing behaviors of grocery retailers on the British market. 

This was achieved by creating a multi-agent model, containing retailer and consumer agents. The heterogeneous crowd of retailers employs either a uniform pricing strategy or a ‘local price flexing’ strategy. The actions of these retailers are chosen by predicting the profit of each action, using a perceptron. Following on from the consideration of different economic models, a discrete model was developed so that software agents have a discrete environment to operate within. Within the model, it has been observed how supermarkets with differing behaviors affect a heterogeneous crowd of consumer agents. The model was implemented in Java with Python used to evaluate the results. 

The simulation displays good acceptance with real grocery market behavior, i.e. captures the performance of British retailers thus can be used to determine the impact of changes in their behavior on their competitors and consumers.Furthermore it can be used to provide insight into sustainability of volatile pricing strategies, providing a useful insight in volatility of British supermarket retail industry. 
\end{abstract}
\acknowledgements{
I would like to express my sincere gratitude to Dr Maria Polukarov for her guidance and support which provided me the freedom to take this research in the direction of my interest.\\
\\
I would also like to thank my family and friends for their encouragement and support. To those who quietly listened to my software complaints. To those who worked throughout the nights with me. To those who helped me write what I couldn't say. I cannot thank you enough.
}

\declaration{
I, Stefan Collier, declare that this dissertation and the work presented in it are my own and has been generated by me as the result of my own original research.\\
I confirm that:\\
1. This work was done wholly or mainly while in candidature for a degree at this University;\\
2. Where any part of this dissertation has previously been submitted for any other qualification at this University or any other institution, this has been clearly stated;\\
3. Where I have consulted the published work of others, this is always clearly attributed;\\
4. Where I have quoted from the work of others, the source is always given. With the exception of such quotations, this dissertation is entirely my own work;\\
5. I have acknowledged all main sources of help;\\
6. Where the thesis is based on work done by myself jointly with others, I have made clear exactly what was done by others and what I have contributed myself;\\
7. Either none of this work has been published before submission, or parts of this work have been published by :\\
\\
Stefan Collier\\
April 2016
}
\tableofcontents
\listoffigures
\listoftables

\mainmatter
%% ----------------------------------------------------------------
%\include{Introduction}
%\include{Conclusions}
\include{chapters/1Project/main}
\include{chapters/2Lit/main}
\include{chapters/3Design/HighLevel}
\include{chapters/3Design/InDepth}
\include{chapters/4Impl/main}

\include{chapters/5Experiments/1/main}
\include{chapters/5Experiments/2/main}
\include{chapters/5Experiments/3/main}
\include{chapters/5Experiments/4/main}

\include{chapters/6Conclusion/main}

\appendix
\include{appendix/AppendixB}
\include{appendix/D/main}
\include{appendix/AppendixC}

\backmatter
\bibliographystyle{ecs}
\bibliography{ECS}
\end{document}
%% ----------------------------------------------------------------


\appendix
\include{appendix/AppendixB}
 %% ----------------------------------------------------------------
%% Progress.tex
%% ---------------------------------------------------------------- 
\documentclass{ecsprogress}    % Use the progress Style
\graphicspath{{../figs/}}   % Location of your graphics files
    \usepackage{natbib}            % Use Natbib style for the refs.
\hypersetup{colorlinks=true}   % Set to false for black/white printing
\input{Definitions}            % Include your abbreviations



\usepackage{enumitem}% http://ctan.org/pkg/enumitem
\usepackage{multirow}
\usepackage{float}
\usepackage{amsmath}
\usepackage{multicol}
\usepackage{amssymb}
\usepackage[normalem]{ulem}
\useunder{\uline}{\ul}{}
\usepackage{wrapfig}


\usepackage[table,xcdraw]{xcolor}


%% ----------------------------------------------------------------
\begin{document}
\frontmatter
\title      {Heterogeneous Agent-based Model for Supermarket Competition}
\authors    {\texorpdfstring
             {\href{mailto:sc22g13@ecs.soton.ac.uk}{Stefan J. Collier}}
             {Stefan J. Collier}
            }
\addresses  {\groupname\\\deptname\\\univname}
\date       {\today}
\subject    {}
\keywords   {}
\supervisor {Dr. Maria Polukarov}
\examiner   {Professor Sheng Chen}

\maketitle
\begin{abstract}
This project aim was to model and analyse the effects of competitive pricing behaviors of grocery retailers on the British market. 

This was achieved by creating a multi-agent model, containing retailer and consumer agents. The heterogeneous crowd of retailers employs either a uniform pricing strategy or a ‘local price flexing’ strategy. The actions of these retailers are chosen by predicting the profit of each action, using a perceptron. Following on from the consideration of different economic models, a discrete model was developed so that software agents have a discrete environment to operate within. Within the model, it has been observed how supermarkets with differing behaviors affect a heterogeneous crowd of consumer agents. The model was implemented in Java with Python used to evaluate the results. 

The simulation displays good acceptance with real grocery market behavior, i.e. captures the performance of British retailers thus can be used to determine the impact of changes in their behavior on their competitors and consumers.Furthermore it can be used to provide insight into sustainability of volatile pricing strategies, providing a useful insight in volatility of British supermarket retail industry. 
\end{abstract}
\acknowledgements{
I would like to express my sincere gratitude to Dr Maria Polukarov for her guidance and support which provided me the freedom to take this research in the direction of my interest.\\
\\
I would also like to thank my family and friends for their encouragement and support. To those who quietly listened to my software complaints. To those who worked throughout the nights with me. To those who helped me write what I couldn't say. I cannot thank you enough.
}

\declaration{
I, Stefan Collier, declare that this dissertation and the work presented in it are my own and has been generated by me as the result of my own original research.\\
I confirm that:\\
1. This work was done wholly or mainly while in candidature for a degree at this University;\\
2. Where any part of this dissertation has previously been submitted for any other qualification at this University or any other institution, this has been clearly stated;\\
3. Where I have consulted the published work of others, this is always clearly attributed;\\
4. Where I have quoted from the work of others, the source is always given. With the exception of such quotations, this dissertation is entirely my own work;\\
5. I have acknowledged all main sources of help;\\
6. Where the thesis is based on work done by myself jointly with others, I have made clear exactly what was done by others and what I have contributed myself;\\
7. Either none of this work has been published before submission, or parts of this work have been published by :\\
\\
Stefan Collier\\
April 2016
}
\tableofcontents
\listoffigures
\listoftables

\mainmatter
%% ----------------------------------------------------------------
%\include{Introduction}
%\include{Conclusions}
\include{chapters/1Project/main}
\include{chapters/2Lit/main}
\include{chapters/3Design/HighLevel}
\include{chapters/3Design/InDepth}
\include{chapters/4Impl/main}

\include{chapters/5Experiments/1/main}
\include{chapters/5Experiments/2/main}
\include{chapters/5Experiments/3/main}
\include{chapters/5Experiments/4/main}

\include{chapters/6Conclusion/main}

\appendix
\include{appendix/AppendixB}
\include{appendix/D/main}
\include{appendix/AppendixC}

\backmatter
\bibliographystyle{ecs}
\bibliography{ECS}
\end{document}
%% ----------------------------------------------------------------

\include{appendix/AppendixC}

\backmatter
\bibliographystyle{ecs}
\bibliography{ECS}
\end{document}
%% ----------------------------------------------------------------

 %% ----------------------------------------------------------------
%% Progress.tex
%% ---------------------------------------------------------------- 
\documentclass{ecsprogress}    % Use the progress Style
\graphicspath{{../figs/}}   % Location of your graphics files
    \usepackage{natbib}            % Use Natbib style for the refs.
\hypersetup{colorlinks=true}   % Set to false for black/white printing
\input{Definitions}            % Include your abbreviations



\usepackage{enumitem}% http://ctan.org/pkg/enumitem
\usepackage{multirow}
\usepackage{float}
\usepackage{amsmath}
\usepackage{multicol}
\usepackage{amssymb}
\usepackage[normalem]{ulem}
\useunder{\uline}{\ul}{}
\usepackage{wrapfig}


\usepackage[table,xcdraw]{xcolor}


%% ----------------------------------------------------------------
\begin{document}
\frontmatter
\title      {Heterogeneous Agent-based Model for Supermarket Competition}
\authors    {\texorpdfstring
             {\href{mailto:sc22g13@ecs.soton.ac.uk}{Stefan J. Collier}}
             {Stefan J. Collier}
            }
\addresses  {\groupname\\\deptname\\\univname}
\date       {\today}
\subject    {}
\keywords   {}
\supervisor {Dr. Maria Polukarov}
\examiner   {Professor Sheng Chen}

\maketitle
\begin{abstract}
This project aim was to model and analyse the effects of competitive pricing behaviors of grocery retailers on the British market. 

This was achieved by creating a multi-agent model, containing retailer and consumer agents. The heterogeneous crowd of retailers employs either a uniform pricing strategy or a ‘local price flexing’ strategy. The actions of these retailers are chosen by predicting the profit of each action, using a perceptron. Following on from the consideration of different economic models, a discrete model was developed so that software agents have a discrete environment to operate within. Within the model, it has been observed how supermarkets with differing behaviors affect a heterogeneous crowd of consumer agents. The model was implemented in Java with Python used to evaluate the results. 

The simulation displays good acceptance with real grocery market behavior, i.e. captures the performance of British retailers thus can be used to determine the impact of changes in their behavior on their competitors and consumers.Furthermore it can be used to provide insight into sustainability of volatile pricing strategies, providing a useful insight in volatility of British supermarket retail industry. 
\end{abstract}
\acknowledgements{
I would like to express my sincere gratitude to Dr Maria Polukarov for her guidance and support which provided me the freedom to take this research in the direction of my interest.\\
\\
I would also like to thank my family and friends for their encouragement and support. To those who quietly listened to my software complaints. To those who worked throughout the nights with me. To those who helped me write what I couldn't say. I cannot thank you enough.
}

\declaration{
I, Stefan Collier, declare that this dissertation and the work presented in it are my own and has been generated by me as the result of my own original research.\\
I confirm that:\\
1. This work was done wholly or mainly while in candidature for a degree at this University;\\
2. Where any part of this dissertation has previously been submitted for any other qualification at this University or any other institution, this has been clearly stated;\\
3. Where I have consulted the published work of others, this is always clearly attributed;\\
4. Where I have quoted from the work of others, the source is always given. With the exception of such quotations, this dissertation is entirely my own work;\\
5. I have acknowledged all main sources of help;\\
6. Where the thesis is based on work done by myself jointly with others, I have made clear exactly what was done by others and what I have contributed myself;\\
7. Either none of this work has been published before submission, or parts of this work have been published by :\\
\\
Stefan Collier\\
April 2016
}
\tableofcontents
\listoffigures
\listoftables

\mainmatter
%% ----------------------------------------------------------------
%\include{Introduction}
%\include{Conclusions}
 %% ----------------------------------------------------------------
%% Progress.tex
%% ---------------------------------------------------------------- 
\documentclass{ecsprogress}    % Use the progress Style
\graphicspath{{../figs/}}   % Location of your graphics files
    \usepackage{natbib}            % Use Natbib style for the refs.
\hypersetup{colorlinks=true}   % Set to false for black/white printing
\input{Definitions}            % Include your abbreviations



\usepackage{enumitem}% http://ctan.org/pkg/enumitem
\usepackage{multirow}
\usepackage{float}
\usepackage{amsmath}
\usepackage{multicol}
\usepackage{amssymb}
\usepackage[normalem]{ulem}
\useunder{\uline}{\ul}{}
\usepackage{wrapfig}


\usepackage[table,xcdraw]{xcolor}


%% ----------------------------------------------------------------
\begin{document}
\frontmatter
\title      {Heterogeneous Agent-based Model for Supermarket Competition}
\authors    {\texorpdfstring
             {\href{mailto:sc22g13@ecs.soton.ac.uk}{Stefan J. Collier}}
             {Stefan J. Collier}
            }
\addresses  {\groupname\\\deptname\\\univname}
\date       {\today}
\subject    {}
\keywords   {}
\supervisor {Dr. Maria Polukarov}
\examiner   {Professor Sheng Chen}

\maketitle
\begin{abstract}
This project aim was to model and analyse the effects of competitive pricing behaviors of grocery retailers on the British market. 

This was achieved by creating a multi-agent model, containing retailer and consumer agents. The heterogeneous crowd of retailers employs either a uniform pricing strategy or a ‘local price flexing’ strategy. The actions of these retailers are chosen by predicting the profit of each action, using a perceptron. Following on from the consideration of different economic models, a discrete model was developed so that software agents have a discrete environment to operate within. Within the model, it has been observed how supermarkets with differing behaviors affect a heterogeneous crowd of consumer agents. The model was implemented in Java with Python used to evaluate the results. 

The simulation displays good acceptance with real grocery market behavior, i.e. captures the performance of British retailers thus can be used to determine the impact of changes in their behavior on their competitors and consumers.Furthermore it can be used to provide insight into sustainability of volatile pricing strategies, providing a useful insight in volatility of British supermarket retail industry. 
\end{abstract}
\acknowledgements{
I would like to express my sincere gratitude to Dr Maria Polukarov for her guidance and support which provided me the freedom to take this research in the direction of my interest.\\
\\
I would also like to thank my family and friends for their encouragement and support. To those who quietly listened to my software complaints. To those who worked throughout the nights with me. To those who helped me write what I couldn't say. I cannot thank you enough.
}

\declaration{
I, Stefan Collier, declare that this dissertation and the work presented in it are my own and has been generated by me as the result of my own original research.\\
I confirm that:\\
1. This work was done wholly or mainly while in candidature for a degree at this University;\\
2. Where any part of this dissertation has previously been submitted for any other qualification at this University or any other institution, this has been clearly stated;\\
3. Where I have consulted the published work of others, this is always clearly attributed;\\
4. Where I have quoted from the work of others, the source is always given. With the exception of such quotations, this dissertation is entirely my own work;\\
5. I have acknowledged all main sources of help;\\
6. Where the thesis is based on work done by myself jointly with others, I have made clear exactly what was done by others and what I have contributed myself;\\
7. Either none of this work has been published before submission, or parts of this work have been published by :\\
\\
Stefan Collier\\
April 2016
}
\tableofcontents
\listoffigures
\listoftables

\mainmatter
%% ----------------------------------------------------------------
%\include{Introduction}
%\include{Conclusions}
\include{chapters/1Project/main}
\include{chapters/2Lit/main}
\include{chapters/3Design/HighLevel}
\include{chapters/3Design/InDepth}
\include{chapters/4Impl/main}

\include{chapters/5Experiments/1/main}
\include{chapters/5Experiments/2/main}
\include{chapters/5Experiments/3/main}
\include{chapters/5Experiments/4/main}

\include{chapters/6Conclusion/main}

\appendix
\include{appendix/AppendixB}
\include{appendix/D/main}
\include{appendix/AppendixC}

\backmatter
\bibliographystyle{ecs}
\bibliography{ECS}
\end{document}
%% ----------------------------------------------------------------

 %% ----------------------------------------------------------------
%% Progress.tex
%% ---------------------------------------------------------------- 
\documentclass{ecsprogress}    % Use the progress Style
\graphicspath{{../figs/}}   % Location of your graphics files
    \usepackage{natbib}            % Use Natbib style for the refs.
\hypersetup{colorlinks=true}   % Set to false for black/white printing
\input{Definitions}            % Include your abbreviations



\usepackage{enumitem}% http://ctan.org/pkg/enumitem
\usepackage{multirow}
\usepackage{float}
\usepackage{amsmath}
\usepackage{multicol}
\usepackage{amssymb}
\usepackage[normalem]{ulem}
\useunder{\uline}{\ul}{}
\usepackage{wrapfig}


\usepackage[table,xcdraw]{xcolor}


%% ----------------------------------------------------------------
\begin{document}
\frontmatter
\title      {Heterogeneous Agent-based Model for Supermarket Competition}
\authors    {\texorpdfstring
             {\href{mailto:sc22g13@ecs.soton.ac.uk}{Stefan J. Collier}}
             {Stefan J. Collier}
            }
\addresses  {\groupname\\\deptname\\\univname}
\date       {\today}
\subject    {}
\keywords   {}
\supervisor {Dr. Maria Polukarov}
\examiner   {Professor Sheng Chen}

\maketitle
\begin{abstract}
This project aim was to model and analyse the effects of competitive pricing behaviors of grocery retailers on the British market. 

This was achieved by creating a multi-agent model, containing retailer and consumer agents. The heterogeneous crowd of retailers employs either a uniform pricing strategy or a ‘local price flexing’ strategy. The actions of these retailers are chosen by predicting the profit of each action, using a perceptron. Following on from the consideration of different economic models, a discrete model was developed so that software agents have a discrete environment to operate within. Within the model, it has been observed how supermarkets with differing behaviors affect a heterogeneous crowd of consumer agents. The model was implemented in Java with Python used to evaluate the results. 

The simulation displays good acceptance with real grocery market behavior, i.e. captures the performance of British retailers thus can be used to determine the impact of changes in their behavior on their competitors and consumers.Furthermore it can be used to provide insight into sustainability of volatile pricing strategies, providing a useful insight in volatility of British supermarket retail industry. 
\end{abstract}
\acknowledgements{
I would like to express my sincere gratitude to Dr Maria Polukarov for her guidance and support which provided me the freedom to take this research in the direction of my interest.\\
\\
I would also like to thank my family and friends for their encouragement and support. To those who quietly listened to my software complaints. To those who worked throughout the nights with me. To those who helped me write what I couldn't say. I cannot thank you enough.
}

\declaration{
I, Stefan Collier, declare that this dissertation and the work presented in it are my own and has been generated by me as the result of my own original research.\\
I confirm that:\\
1. This work was done wholly or mainly while in candidature for a degree at this University;\\
2. Where any part of this dissertation has previously been submitted for any other qualification at this University or any other institution, this has been clearly stated;\\
3. Where I have consulted the published work of others, this is always clearly attributed;\\
4. Where I have quoted from the work of others, the source is always given. With the exception of such quotations, this dissertation is entirely my own work;\\
5. I have acknowledged all main sources of help;\\
6. Where the thesis is based on work done by myself jointly with others, I have made clear exactly what was done by others and what I have contributed myself;\\
7. Either none of this work has been published before submission, or parts of this work have been published by :\\
\\
Stefan Collier\\
April 2016
}
\tableofcontents
\listoffigures
\listoftables

\mainmatter
%% ----------------------------------------------------------------
%\include{Introduction}
%\include{Conclusions}
\include{chapters/1Project/main}
\include{chapters/2Lit/main}
\include{chapters/3Design/HighLevel}
\include{chapters/3Design/InDepth}
\include{chapters/4Impl/main}

\include{chapters/5Experiments/1/main}
\include{chapters/5Experiments/2/main}
\include{chapters/5Experiments/3/main}
\include{chapters/5Experiments/4/main}

\include{chapters/6Conclusion/main}

\appendix
\include{appendix/AppendixB}
\include{appendix/D/main}
\include{appendix/AppendixC}

\backmatter
\bibliographystyle{ecs}
\bibliography{ECS}
\end{document}
%% ----------------------------------------------------------------

\include{chapters/3Design/HighLevel}
\include{chapters/3Design/InDepth}
 %% ----------------------------------------------------------------
%% Progress.tex
%% ---------------------------------------------------------------- 
\documentclass{ecsprogress}    % Use the progress Style
\graphicspath{{../figs/}}   % Location of your graphics files
    \usepackage{natbib}            % Use Natbib style for the refs.
\hypersetup{colorlinks=true}   % Set to false for black/white printing
\input{Definitions}            % Include your abbreviations



\usepackage{enumitem}% http://ctan.org/pkg/enumitem
\usepackage{multirow}
\usepackage{float}
\usepackage{amsmath}
\usepackage{multicol}
\usepackage{amssymb}
\usepackage[normalem]{ulem}
\useunder{\uline}{\ul}{}
\usepackage{wrapfig}


\usepackage[table,xcdraw]{xcolor}


%% ----------------------------------------------------------------
\begin{document}
\frontmatter
\title      {Heterogeneous Agent-based Model for Supermarket Competition}
\authors    {\texorpdfstring
             {\href{mailto:sc22g13@ecs.soton.ac.uk}{Stefan J. Collier}}
             {Stefan J. Collier}
            }
\addresses  {\groupname\\\deptname\\\univname}
\date       {\today}
\subject    {}
\keywords   {}
\supervisor {Dr. Maria Polukarov}
\examiner   {Professor Sheng Chen}

\maketitle
\begin{abstract}
This project aim was to model and analyse the effects of competitive pricing behaviors of grocery retailers on the British market. 

This was achieved by creating a multi-agent model, containing retailer and consumer agents. The heterogeneous crowd of retailers employs either a uniform pricing strategy or a ‘local price flexing’ strategy. The actions of these retailers are chosen by predicting the profit of each action, using a perceptron. Following on from the consideration of different economic models, a discrete model was developed so that software agents have a discrete environment to operate within. Within the model, it has been observed how supermarkets with differing behaviors affect a heterogeneous crowd of consumer agents. The model was implemented in Java with Python used to evaluate the results. 

The simulation displays good acceptance with real grocery market behavior, i.e. captures the performance of British retailers thus can be used to determine the impact of changes in their behavior on their competitors and consumers.Furthermore it can be used to provide insight into sustainability of volatile pricing strategies, providing a useful insight in volatility of British supermarket retail industry. 
\end{abstract}
\acknowledgements{
I would like to express my sincere gratitude to Dr Maria Polukarov for her guidance and support which provided me the freedom to take this research in the direction of my interest.\\
\\
I would also like to thank my family and friends for their encouragement and support. To those who quietly listened to my software complaints. To those who worked throughout the nights with me. To those who helped me write what I couldn't say. I cannot thank you enough.
}

\declaration{
I, Stefan Collier, declare that this dissertation and the work presented in it are my own and has been generated by me as the result of my own original research.\\
I confirm that:\\
1. This work was done wholly or mainly while in candidature for a degree at this University;\\
2. Where any part of this dissertation has previously been submitted for any other qualification at this University or any other institution, this has been clearly stated;\\
3. Where I have consulted the published work of others, this is always clearly attributed;\\
4. Where I have quoted from the work of others, the source is always given. With the exception of such quotations, this dissertation is entirely my own work;\\
5. I have acknowledged all main sources of help;\\
6. Where the thesis is based on work done by myself jointly with others, I have made clear exactly what was done by others and what I have contributed myself;\\
7. Either none of this work has been published before submission, or parts of this work have been published by :\\
\\
Stefan Collier\\
April 2016
}
\tableofcontents
\listoffigures
\listoftables

\mainmatter
%% ----------------------------------------------------------------
%\include{Introduction}
%\include{Conclusions}
\include{chapters/1Project/main}
\include{chapters/2Lit/main}
\include{chapters/3Design/HighLevel}
\include{chapters/3Design/InDepth}
\include{chapters/4Impl/main}

\include{chapters/5Experiments/1/main}
\include{chapters/5Experiments/2/main}
\include{chapters/5Experiments/3/main}
\include{chapters/5Experiments/4/main}

\include{chapters/6Conclusion/main}

\appendix
\include{appendix/AppendixB}
\include{appendix/D/main}
\include{appendix/AppendixC}

\backmatter
\bibliographystyle{ecs}
\bibliography{ECS}
\end{document}
%% ----------------------------------------------------------------


 %% ----------------------------------------------------------------
%% Progress.tex
%% ---------------------------------------------------------------- 
\documentclass{ecsprogress}    % Use the progress Style
\graphicspath{{../figs/}}   % Location of your graphics files
    \usepackage{natbib}            % Use Natbib style for the refs.
\hypersetup{colorlinks=true}   % Set to false for black/white printing
\input{Definitions}            % Include your abbreviations



\usepackage{enumitem}% http://ctan.org/pkg/enumitem
\usepackage{multirow}
\usepackage{float}
\usepackage{amsmath}
\usepackage{multicol}
\usepackage{amssymb}
\usepackage[normalem]{ulem}
\useunder{\uline}{\ul}{}
\usepackage{wrapfig}


\usepackage[table,xcdraw]{xcolor}


%% ----------------------------------------------------------------
\begin{document}
\frontmatter
\title      {Heterogeneous Agent-based Model for Supermarket Competition}
\authors    {\texorpdfstring
             {\href{mailto:sc22g13@ecs.soton.ac.uk}{Stefan J. Collier}}
             {Stefan J. Collier}
            }
\addresses  {\groupname\\\deptname\\\univname}
\date       {\today}
\subject    {}
\keywords   {}
\supervisor {Dr. Maria Polukarov}
\examiner   {Professor Sheng Chen}

\maketitle
\begin{abstract}
This project aim was to model and analyse the effects of competitive pricing behaviors of grocery retailers on the British market. 

This was achieved by creating a multi-agent model, containing retailer and consumer agents. The heterogeneous crowd of retailers employs either a uniform pricing strategy or a ‘local price flexing’ strategy. The actions of these retailers are chosen by predicting the profit of each action, using a perceptron. Following on from the consideration of different economic models, a discrete model was developed so that software agents have a discrete environment to operate within. Within the model, it has been observed how supermarkets with differing behaviors affect a heterogeneous crowd of consumer agents. The model was implemented in Java with Python used to evaluate the results. 

The simulation displays good acceptance with real grocery market behavior, i.e. captures the performance of British retailers thus can be used to determine the impact of changes in their behavior on their competitors and consumers.Furthermore it can be used to provide insight into sustainability of volatile pricing strategies, providing a useful insight in volatility of British supermarket retail industry. 
\end{abstract}
\acknowledgements{
I would like to express my sincere gratitude to Dr Maria Polukarov for her guidance and support which provided me the freedom to take this research in the direction of my interest.\\
\\
I would also like to thank my family and friends for their encouragement and support. To those who quietly listened to my software complaints. To those who worked throughout the nights with me. To those who helped me write what I couldn't say. I cannot thank you enough.
}

\declaration{
I, Stefan Collier, declare that this dissertation and the work presented in it are my own and has been generated by me as the result of my own original research.\\
I confirm that:\\
1. This work was done wholly or mainly while in candidature for a degree at this University;\\
2. Where any part of this dissertation has previously been submitted for any other qualification at this University or any other institution, this has been clearly stated;\\
3. Where I have consulted the published work of others, this is always clearly attributed;\\
4. Where I have quoted from the work of others, the source is always given. With the exception of such quotations, this dissertation is entirely my own work;\\
5. I have acknowledged all main sources of help;\\
6. Where the thesis is based on work done by myself jointly with others, I have made clear exactly what was done by others and what I have contributed myself;\\
7. Either none of this work has been published before submission, or parts of this work have been published by :\\
\\
Stefan Collier\\
April 2016
}
\tableofcontents
\listoffigures
\listoftables

\mainmatter
%% ----------------------------------------------------------------
%\include{Introduction}
%\include{Conclusions}
\include{chapters/1Project/main}
\include{chapters/2Lit/main}
\include{chapters/3Design/HighLevel}
\include{chapters/3Design/InDepth}
\include{chapters/4Impl/main}

\include{chapters/5Experiments/1/main}
\include{chapters/5Experiments/2/main}
\include{chapters/5Experiments/3/main}
\include{chapters/5Experiments/4/main}

\include{chapters/6Conclusion/main}

\appendix
\include{appendix/AppendixB}
\include{appendix/D/main}
\include{appendix/AppendixC}

\backmatter
\bibliographystyle{ecs}
\bibliography{ECS}
\end{document}
%% ----------------------------------------------------------------

 %% ----------------------------------------------------------------
%% Progress.tex
%% ---------------------------------------------------------------- 
\documentclass{ecsprogress}    % Use the progress Style
\graphicspath{{../figs/}}   % Location of your graphics files
    \usepackage{natbib}            % Use Natbib style for the refs.
\hypersetup{colorlinks=true}   % Set to false for black/white printing
\input{Definitions}            % Include your abbreviations



\usepackage{enumitem}% http://ctan.org/pkg/enumitem
\usepackage{multirow}
\usepackage{float}
\usepackage{amsmath}
\usepackage{multicol}
\usepackage{amssymb}
\usepackage[normalem]{ulem}
\useunder{\uline}{\ul}{}
\usepackage{wrapfig}


\usepackage[table,xcdraw]{xcolor}


%% ----------------------------------------------------------------
\begin{document}
\frontmatter
\title      {Heterogeneous Agent-based Model for Supermarket Competition}
\authors    {\texorpdfstring
             {\href{mailto:sc22g13@ecs.soton.ac.uk}{Stefan J. Collier}}
             {Stefan J. Collier}
            }
\addresses  {\groupname\\\deptname\\\univname}
\date       {\today}
\subject    {}
\keywords   {}
\supervisor {Dr. Maria Polukarov}
\examiner   {Professor Sheng Chen}

\maketitle
\begin{abstract}
This project aim was to model and analyse the effects of competitive pricing behaviors of grocery retailers on the British market. 

This was achieved by creating a multi-agent model, containing retailer and consumer agents. The heterogeneous crowd of retailers employs either a uniform pricing strategy or a ‘local price flexing’ strategy. The actions of these retailers are chosen by predicting the profit of each action, using a perceptron. Following on from the consideration of different economic models, a discrete model was developed so that software agents have a discrete environment to operate within. Within the model, it has been observed how supermarkets with differing behaviors affect a heterogeneous crowd of consumer agents. The model was implemented in Java with Python used to evaluate the results. 

The simulation displays good acceptance with real grocery market behavior, i.e. captures the performance of British retailers thus can be used to determine the impact of changes in their behavior on their competitors and consumers.Furthermore it can be used to provide insight into sustainability of volatile pricing strategies, providing a useful insight in volatility of British supermarket retail industry. 
\end{abstract}
\acknowledgements{
I would like to express my sincere gratitude to Dr Maria Polukarov for her guidance and support which provided me the freedom to take this research in the direction of my interest.\\
\\
I would also like to thank my family and friends for their encouragement and support. To those who quietly listened to my software complaints. To those who worked throughout the nights with me. To those who helped me write what I couldn't say. I cannot thank you enough.
}

\declaration{
I, Stefan Collier, declare that this dissertation and the work presented in it are my own and has been generated by me as the result of my own original research.\\
I confirm that:\\
1. This work was done wholly or mainly while in candidature for a degree at this University;\\
2. Where any part of this dissertation has previously been submitted for any other qualification at this University or any other institution, this has been clearly stated;\\
3. Where I have consulted the published work of others, this is always clearly attributed;\\
4. Where I have quoted from the work of others, the source is always given. With the exception of such quotations, this dissertation is entirely my own work;\\
5. I have acknowledged all main sources of help;\\
6. Where the thesis is based on work done by myself jointly with others, I have made clear exactly what was done by others and what I have contributed myself;\\
7. Either none of this work has been published before submission, or parts of this work have been published by :\\
\\
Stefan Collier\\
April 2016
}
\tableofcontents
\listoffigures
\listoftables

\mainmatter
%% ----------------------------------------------------------------
%\include{Introduction}
%\include{Conclusions}
\include{chapters/1Project/main}
\include{chapters/2Lit/main}
\include{chapters/3Design/HighLevel}
\include{chapters/3Design/InDepth}
\include{chapters/4Impl/main}

\include{chapters/5Experiments/1/main}
\include{chapters/5Experiments/2/main}
\include{chapters/5Experiments/3/main}
\include{chapters/5Experiments/4/main}

\include{chapters/6Conclusion/main}

\appendix
\include{appendix/AppendixB}
\include{appendix/D/main}
\include{appendix/AppendixC}

\backmatter
\bibliographystyle{ecs}
\bibliography{ECS}
\end{document}
%% ----------------------------------------------------------------

 %% ----------------------------------------------------------------
%% Progress.tex
%% ---------------------------------------------------------------- 
\documentclass{ecsprogress}    % Use the progress Style
\graphicspath{{../figs/}}   % Location of your graphics files
    \usepackage{natbib}            % Use Natbib style for the refs.
\hypersetup{colorlinks=true}   % Set to false for black/white printing
\input{Definitions}            % Include your abbreviations



\usepackage{enumitem}% http://ctan.org/pkg/enumitem
\usepackage{multirow}
\usepackage{float}
\usepackage{amsmath}
\usepackage{multicol}
\usepackage{amssymb}
\usepackage[normalem]{ulem}
\useunder{\uline}{\ul}{}
\usepackage{wrapfig}


\usepackage[table,xcdraw]{xcolor}


%% ----------------------------------------------------------------
\begin{document}
\frontmatter
\title      {Heterogeneous Agent-based Model for Supermarket Competition}
\authors    {\texorpdfstring
             {\href{mailto:sc22g13@ecs.soton.ac.uk}{Stefan J. Collier}}
             {Stefan J. Collier}
            }
\addresses  {\groupname\\\deptname\\\univname}
\date       {\today}
\subject    {}
\keywords   {}
\supervisor {Dr. Maria Polukarov}
\examiner   {Professor Sheng Chen}

\maketitle
\begin{abstract}
This project aim was to model and analyse the effects of competitive pricing behaviors of grocery retailers on the British market. 

This was achieved by creating a multi-agent model, containing retailer and consumer agents. The heterogeneous crowd of retailers employs either a uniform pricing strategy or a ‘local price flexing’ strategy. The actions of these retailers are chosen by predicting the profit of each action, using a perceptron. Following on from the consideration of different economic models, a discrete model was developed so that software agents have a discrete environment to operate within. Within the model, it has been observed how supermarkets with differing behaviors affect a heterogeneous crowd of consumer agents. The model was implemented in Java with Python used to evaluate the results. 

The simulation displays good acceptance with real grocery market behavior, i.e. captures the performance of British retailers thus can be used to determine the impact of changes in their behavior on their competitors and consumers.Furthermore it can be used to provide insight into sustainability of volatile pricing strategies, providing a useful insight in volatility of British supermarket retail industry. 
\end{abstract}
\acknowledgements{
I would like to express my sincere gratitude to Dr Maria Polukarov for her guidance and support which provided me the freedom to take this research in the direction of my interest.\\
\\
I would also like to thank my family and friends for their encouragement and support. To those who quietly listened to my software complaints. To those who worked throughout the nights with me. To those who helped me write what I couldn't say. I cannot thank you enough.
}

\declaration{
I, Stefan Collier, declare that this dissertation and the work presented in it are my own and has been generated by me as the result of my own original research.\\
I confirm that:\\
1. This work was done wholly or mainly while in candidature for a degree at this University;\\
2. Where any part of this dissertation has previously been submitted for any other qualification at this University or any other institution, this has been clearly stated;\\
3. Where I have consulted the published work of others, this is always clearly attributed;\\
4. Where I have quoted from the work of others, the source is always given. With the exception of such quotations, this dissertation is entirely my own work;\\
5. I have acknowledged all main sources of help;\\
6. Where the thesis is based on work done by myself jointly with others, I have made clear exactly what was done by others and what I have contributed myself;\\
7. Either none of this work has been published before submission, or parts of this work have been published by :\\
\\
Stefan Collier\\
April 2016
}
\tableofcontents
\listoffigures
\listoftables

\mainmatter
%% ----------------------------------------------------------------
%\include{Introduction}
%\include{Conclusions}
\include{chapters/1Project/main}
\include{chapters/2Lit/main}
\include{chapters/3Design/HighLevel}
\include{chapters/3Design/InDepth}
\include{chapters/4Impl/main}

\include{chapters/5Experiments/1/main}
\include{chapters/5Experiments/2/main}
\include{chapters/5Experiments/3/main}
\include{chapters/5Experiments/4/main}

\include{chapters/6Conclusion/main}

\appendix
\include{appendix/AppendixB}
\include{appendix/D/main}
\include{appendix/AppendixC}

\backmatter
\bibliographystyle{ecs}
\bibliography{ECS}
\end{document}
%% ----------------------------------------------------------------

 %% ----------------------------------------------------------------
%% Progress.tex
%% ---------------------------------------------------------------- 
\documentclass{ecsprogress}    % Use the progress Style
\graphicspath{{../figs/}}   % Location of your graphics files
    \usepackage{natbib}            % Use Natbib style for the refs.
\hypersetup{colorlinks=true}   % Set to false for black/white printing
\input{Definitions}            % Include your abbreviations



\usepackage{enumitem}% http://ctan.org/pkg/enumitem
\usepackage{multirow}
\usepackage{float}
\usepackage{amsmath}
\usepackage{multicol}
\usepackage{amssymb}
\usepackage[normalem]{ulem}
\useunder{\uline}{\ul}{}
\usepackage{wrapfig}


\usepackage[table,xcdraw]{xcolor}


%% ----------------------------------------------------------------
\begin{document}
\frontmatter
\title      {Heterogeneous Agent-based Model for Supermarket Competition}
\authors    {\texorpdfstring
             {\href{mailto:sc22g13@ecs.soton.ac.uk}{Stefan J. Collier}}
             {Stefan J. Collier}
            }
\addresses  {\groupname\\\deptname\\\univname}
\date       {\today}
\subject    {}
\keywords   {}
\supervisor {Dr. Maria Polukarov}
\examiner   {Professor Sheng Chen}

\maketitle
\begin{abstract}
This project aim was to model and analyse the effects of competitive pricing behaviors of grocery retailers on the British market. 

This was achieved by creating a multi-agent model, containing retailer and consumer agents. The heterogeneous crowd of retailers employs either a uniform pricing strategy or a ‘local price flexing’ strategy. The actions of these retailers are chosen by predicting the profit of each action, using a perceptron. Following on from the consideration of different economic models, a discrete model was developed so that software agents have a discrete environment to operate within. Within the model, it has been observed how supermarkets with differing behaviors affect a heterogeneous crowd of consumer agents. The model was implemented in Java with Python used to evaluate the results. 

The simulation displays good acceptance with real grocery market behavior, i.e. captures the performance of British retailers thus can be used to determine the impact of changes in their behavior on their competitors and consumers.Furthermore it can be used to provide insight into sustainability of volatile pricing strategies, providing a useful insight in volatility of British supermarket retail industry. 
\end{abstract}
\acknowledgements{
I would like to express my sincere gratitude to Dr Maria Polukarov for her guidance and support which provided me the freedom to take this research in the direction of my interest.\\
\\
I would also like to thank my family and friends for their encouragement and support. To those who quietly listened to my software complaints. To those who worked throughout the nights with me. To those who helped me write what I couldn't say. I cannot thank you enough.
}

\declaration{
I, Stefan Collier, declare that this dissertation and the work presented in it are my own and has been generated by me as the result of my own original research.\\
I confirm that:\\
1. This work was done wholly or mainly while in candidature for a degree at this University;\\
2. Where any part of this dissertation has previously been submitted for any other qualification at this University or any other institution, this has been clearly stated;\\
3. Where I have consulted the published work of others, this is always clearly attributed;\\
4. Where I have quoted from the work of others, the source is always given. With the exception of such quotations, this dissertation is entirely my own work;\\
5. I have acknowledged all main sources of help;\\
6. Where the thesis is based on work done by myself jointly with others, I have made clear exactly what was done by others and what I have contributed myself;\\
7. Either none of this work has been published before submission, or parts of this work have been published by :\\
\\
Stefan Collier\\
April 2016
}
\tableofcontents
\listoffigures
\listoftables

\mainmatter
%% ----------------------------------------------------------------
%\include{Introduction}
%\include{Conclusions}
\include{chapters/1Project/main}
\include{chapters/2Lit/main}
\include{chapters/3Design/HighLevel}
\include{chapters/3Design/InDepth}
\include{chapters/4Impl/main}

\include{chapters/5Experiments/1/main}
\include{chapters/5Experiments/2/main}
\include{chapters/5Experiments/3/main}
\include{chapters/5Experiments/4/main}

\include{chapters/6Conclusion/main}

\appendix
\include{appendix/AppendixB}
\include{appendix/D/main}
\include{appendix/AppendixC}

\backmatter
\bibliographystyle{ecs}
\bibliography{ECS}
\end{document}
%% ----------------------------------------------------------------


 %% ----------------------------------------------------------------
%% Progress.tex
%% ---------------------------------------------------------------- 
\documentclass{ecsprogress}    % Use the progress Style
\graphicspath{{../figs/}}   % Location of your graphics files
    \usepackage{natbib}            % Use Natbib style for the refs.
\hypersetup{colorlinks=true}   % Set to false for black/white printing
\input{Definitions}            % Include your abbreviations



\usepackage{enumitem}% http://ctan.org/pkg/enumitem
\usepackage{multirow}
\usepackage{float}
\usepackage{amsmath}
\usepackage{multicol}
\usepackage{amssymb}
\usepackage[normalem]{ulem}
\useunder{\uline}{\ul}{}
\usepackage{wrapfig}


\usepackage[table,xcdraw]{xcolor}


%% ----------------------------------------------------------------
\begin{document}
\frontmatter
\title      {Heterogeneous Agent-based Model for Supermarket Competition}
\authors    {\texorpdfstring
             {\href{mailto:sc22g13@ecs.soton.ac.uk}{Stefan J. Collier}}
             {Stefan J. Collier}
            }
\addresses  {\groupname\\\deptname\\\univname}
\date       {\today}
\subject    {}
\keywords   {}
\supervisor {Dr. Maria Polukarov}
\examiner   {Professor Sheng Chen}

\maketitle
\begin{abstract}
This project aim was to model and analyse the effects of competitive pricing behaviors of grocery retailers on the British market. 

This was achieved by creating a multi-agent model, containing retailer and consumer agents. The heterogeneous crowd of retailers employs either a uniform pricing strategy or a ‘local price flexing’ strategy. The actions of these retailers are chosen by predicting the profit of each action, using a perceptron. Following on from the consideration of different economic models, a discrete model was developed so that software agents have a discrete environment to operate within. Within the model, it has been observed how supermarkets with differing behaviors affect a heterogeneous crowd of consumer agents. The model was implemented in Java with Python used to evaluate the results. 

The simulation displays good acceptance with real grocery market behavior, i.e. captures the performance of British retailers thus can be used to determine the impact of changes in their behavior on their competitors and consumers.Furthermore it can be used to provide insight into sustainability of volatile pricing strategies, providing a useful insight in volatility of British supermarket retail industry. 
\end{abstract}
\acknowledgements{
I would like to express my sincere gratitude to Dr Maria Polukarov for her guidance and support which provided me the freedom to take this research in the direction of my interest.\\
\\
I would also like to thank my family and friends for their encouragement and support. To those who quietly listened to my software complaints. To those who worked throughout the nights with me. To those who helped me write what I couldn't say. I cannot thank you enough.
}

\declaration{
I, Stefan Collier, declare that this dissertation and the work presented in it are my own and has been generated by me as the result of my own original research.\\
I confirm that:\\
1. This work was done wholly or mainly while in candidature for a degree at this University;\\
2. Where any part of this dissertation has previously been submitted for any other qualification at this University or any other institution, this has been clearly stated;\\
3. Where I have consulted the published work of others, this is always clearly attributed;\\
4. Where I have quoted from the work of others, the source is always given. With the exception of such quotations, this dissertation is entirely my own work;\\
5. I have acknowledged all main sources of help;\\
6. Where the thesis is based on work done by myself jointly with others, I have made clear exactly what was done by others and what I have contributed myself;\\
7. Either none of this work has been published before submission, or parts of this work have been published by :\\
\\
Stefan Collier\\
April 2016
}
\tableofcontents
\listoffigures
\listoftables

\mainmatter
%% ----------------------------------------------------------------
%\include{Introduction}
%\include{Conclusions}
\include{chapters/1Project/main}
\include{chapters/2Lit/main}
\include{chapters/3Design/HighLevel}
\include{chapters/3Design/InDepth}
\include{chapters/4Impl/main}

\include{chapters/5Experiments/1/main}
\include{chapters/5Experiments/2/main}
\include{chapters/5Experiments/3/main}
\include{chapters/5Experiments/4/main}

\include{chapters/6Conclusion/main}

\appendix
\include{appendix/AppendixB}
\include{appendix/D/main}
\include{appendix/AppendixC}

\backmatter
\bibliographystyle{ecs}
\bibliography{ECS}
\end{document}
%% ----------------------------------------------------------------


\appendix
\include{appendix/AppendixB}
 %% ----------------------------------------------------------------
%% Progress.tex
%% ---------------------------------------------------------------- 
\documentclass{ecsprogress}    % Use the progress Style
\graphicspath{{../figs/}}   % Location of your graphics files
    \usepackage{natbib}            % Use Natbib style for the refs.
\hypersetup{colorlinks=true}   % Set to false for black/white printing
\input{Definitions}            % Include your abbreviations



\usepackage{enumitem}% http://ctan.org/pkg/enumitem
\usepackage{multirow}
\usepackage{float}
\usepackage{amsmath}
\usepackage{multicol}
\usepackage{amssymb}
\usepackage[normalem]{ulem}
\useunder{\uline}{\ul}{}
\usepackage{wrapfig}


\usepackage[table,xcdraw]{xcolor}


%% ----------------------------------------------------------------
\begin{document}
\frontmatter
\title      {Heterogeneous Agent-based Model for Supermarket Competition}
\authors    {\texorpdfstring
             {\href{mailto:sc22g13@ecs.soton.ac.uk}{Stefan J. Collier}}
             {Stefan J. Collier}
            }
\addresses  {\groupname\\\deptname\\\univname}
\date       {\today}
\subject    {}
\keywords   {}
\supervisor {Dr. Maria Polukarov}
\examiner   {Professor Sheng Chen}

\maketitle
\begin{abstract}
This project aim was to model and analyse the effects of competitive pricing behaviors of grocery retailers on the British market. 

This was achieved by creating a multi-agent model, containing retailer and consumer agents. The heterogeneous crowd of retailers employs either a uniform pricing strategy or a ‘local price flexing’ strategy. The actions of these retailers are chosen by predicting the profit of each action, using a perceptron. Following on from the consideration of different economic models, a discrete model was developed so that software agents have a discrete environment to operate within. Within the model, it has been observed how supermarkets with differing behaviors affect a heterogeneous crowd of consumer agents. The model was implemented in Java with Python used to evaluate the results. 

The simulation displays good acceptance with real grocery market behavior, i.e. captures the performance of British retailers thus can be used to determine the impact of changes in their behavior on their competitors and consumers.Furthermore it can be used to provide insight into sustainability of volatile pricing strategies, providing a useful insight in volatility of British supermarket retail industry. 
\end{abstract}
\acknowledgements{
I would like to express my sincere gratitude to Dr Maria Polukarov for her guidance and support which provided me the freedom to take this research in the direction of my interest.\\
\\
I would also like to thank my family and friends for their encouragement and support. To those who quietly listened to my software complaints. To those who worked throughout the nights with me. To those who helped me write what I couldn't say. I cannot thank you enough.
}

\declaration{
I, Stefan Collier, declare that this dissertation and the work presented in it are my own and has been generated by me as the result of my own original research.\\
I confirm that:\\
1. This work was done wholly or mainly while in candidature for a degree at this University;\\
2. Where any part of this dissertation has previously been submitted for any other qualification at this University or any other institution, this has been clearly stated;\\
3. Where I have consulted the published work of others, this is always clearly attributed;\\
4. Where I have quoted from the work of others, the source is always given. With the exception of such quotations, this dissertation is entirely my own work;\\
5. I have acknowledged all main sources of help;\\
6. Where the thesis is based on work done by myself jointly with others, I have made clear exactly what was done by others and what I have contributed myself;\\
7. Either none of this work has been published before submission, or parts of this work have been published by :\\
\\
Stefan Collier\\
April 2016
}
\tableofcontents
\listoffigures
\listoftables

\mainmatter
%% ----------------------------------------------------------------
%\include{Introduction}
%\include{Conclusions}
\include{chapters/1Project/main}
\include{chapters/2Lit/main}
\include{chapters/3Design/HighLevel}
\include{chapters/3Design/InDepth}
\include{chapters/4Impl/main}

\include{chapters/5Experiments/1/main}
\include{chapters/5Experiments/2/main}
\include{chapters/5Experiments/3/main}
\include{chapters/5Experiments/4/main}

\include{chapters/6Conclusion/main}

\appendix
\include{appendix/AppendixB}
\include{appendix/D/main}
\include{appendix/AppendixC}

\backmatter
\bibliographystyle{ecs}
\bibliography{ECS}
\end{document}
%% ----------------------------------------------------------------

\include{appendix/AppendixC}

\backmatter
\bibliographystyle{ecs}
\bibliography{ECS}
\end{document}
%% ----------------------------------------------------------------

 %% ----------------------------------------------------------------
%% Progress.tex
%% ---------------------------------------------------------------- 
\documentclass{ecsprogress}    % Use the progress Style
\graphicspath{{../figs/}}   % Location of your graphics files
    \usepackage{natbib}            % Use Natbib style for the refs.
\hypersetup{colorlinks=true}   % Set to false for black/white printing
\input{Definitions}            % Include your abbreviations



\usepackage{enumitem}% http://ctan.org/pkg/enumitem
\usepackage{multirow}
\usepackage{float}
\usepackage{amsmath}
\usepackage{multicol}
\usepackage{amssymb}
\usepackage[normalem]{ulem}
\useunder{\uline}{\ul}{}
\usepackage{wrapfig}


\usepackage[table,xcdraw]{xcolor}


%% ----------------------------------------------------------------
\begin{document}
\frontmatter
\title      {Heterogeneous Agent-based Model for Supermarket Competition}
\authors    {\texorpdfstring
             {\href{mailto:sc22g13@ecs.soton.ac.uk}{Stefan J. Collier}}
             {Stefan J. Collier}
            }
\addresses  {\groupname\\\deptname\\\univname}
\date       {\today}
\subject    {}
\keywords   {}
\supervisor {Dr. Maria Polukarov}
\examiner   {Professor Sheng Chen}

\maketitle
\begin{abstract}
This project aim was to model and analyse the effects of competitive pricing behaviors of grocery retailers on the British market. 

This was achieved by creating a multi-agent model, containing retailer and consumer agents. The heterogeneous crowd of retailers employs either a uniform pricing strategy or a ‘local price flexing’ strategy. The actions of these retailers are chosen by predicting the profit of each action, using a perceptron. Following on from the consideration of different economic models, a discrete model was developed so that software agents have a discrete environment to operate within. Within the model, it has been observed how supermarkets with differing behaviors affect a heterogeneous crowd of consumer agents. The model was implemented in Java with Python used to evaluate the results. 

The simulation displays good acceptance with real grocery market behavior, i.e. captures the performance of British retailers thus can be used to determine the impact of changes in their behavior on their competitors and consumers.Furthermore it can be used to provide insight into sustainability of volatile pricing strategies, providing a useful insight in volatility of British supermarket retail industry. 
\end{abstract}
\acknowledgements{
I would like to express my sincere gratitude to Dr Maria Polukarov for her guidance and support which provided me the freedom to take this research in the direction of my interest.\\
\\
I would also like to thank my family and friends for their encouragement and support. To those who quietly listened to my software complaints. To those who worked throughout the nights with me. To those who helped me write what I couldn't say. I cannot thank you enough.
}

\declaration{
I, Stefan Collier, declare that this dissertation and the work presented in it are my own and has been generated by me as the result of my own original research.\\
I confirm that:\\
1. This work was done wholly or mainly while in candidature for a degree at this University;\\
2. Where any part of this dissertation has previously been submitted for any other qualification at this University or any other institution, this has been clearly stated;\\
3. Where I have consulted the published work of others, this is always clearly attributed;\\
4. Where I have quoted from the work of others, the source is always given. With the exception of such quotations, this dissertation is entirely my own work;\\
5. I have acknowledged all main sources of help;\\
6. Where the thesis is based on work done by myself jointly with others, I have made clear exactly what was done by others and what I have contributed myself;\\
7. Either none of this work has been published before submission, or parts of this work have been published by :\\
\\
Stefan Collier\\
April 2016
}
\tableofcontents
\listoffigures
\listoftables

\mainmatter
%% ----------------------------------------------------------------
%\include{Introduction}
%\include{Conclusions}
 %% ----------------------------------------------------------------
%% Progress.tex
%% ---------------------------------------------------------------- 
\documentclass{ecsprogress}    % Use the progress Style
\graphicspath{{../figs/}}   % Location of your graphics files
    \usepackage{natbib}            % Use Natbib style for the refs.
\hypersetup{colorlinks=true}   % Set to false for black/white printing
\input{Definitions}            % Include your abbreviations



\usepackage{enumitem}% http://ctan.org/pkg/enumitem
\usepackage{multirow}
\usepackage{float}
\usepackage{amsmath}
\usepackage{multicol}
\usepackage{amssymb}
\usepackage[normalem]{ulem}
\useunder{\uline}{\ul}{}
\usepackage{wrapfig}


\usepackage[table,xcdraw]{xcolor}


%% ----------------------------------------------------------------
\begin{document}
\frontmatter
\title      {Heterogeneous Agent-based Model for Supermarket Competition}
\authors    {\texorpdfstring
             {\href{mailto:sc22g13@ecs.soton.ac.uk}{Stefan J. Collier}}
             {Stefan J. Collier}
            }
\addresses  {\groupname\\\deptname\\\univname}
\date       {\today}
\subject    {}
\keywords   {}
\supervisor {Dr. Maria Polukarov}
\examiner   {Professor Sheng Chen}

\maketitle
\begin{abstract}
This project aim was to model and analyse the effects of competitive pricing behaviors of grocery retailers on the British market. 

This was achieved by creating a multi-agent model, containing retailer and consumer agents. The heterogeneous crowd of retailers employs either a uniform pricing strategy or a ‘local price flexing’ strategy. The actions of these retailers are chosen by predicting the profit of each action, using a perceptron. Following on from the consideration of different economic models, a discrete model was developed so that software agents have a discrete environment to operate within. Within the model, it has been observed how supermarkets with differing behaviors affect a heterogeneous crowd of consumer agents. The model was implemented in Java with Python used to evaluate the results. 

The simulation displays good acceptance with real grocery market behavior, i.e. captures the performance of British retailers thus can be used to determine the impact of changes in their behavior on their competitors and consumers.Furthermore it can be used to provide insight into sustainability of volatile pricing strategies, providing a useful insight in volatility of British supermarket retail industry. 
\end{abstract}
\acknowledgements{
I would like to express my sincere gratitude to Dr Maria Polukarov for her guidance and support which provided me the freedom to take this research in the direction of my interest.\\
\\
I would also like to thank my family and friends for their encouragement and support. To those who quietly listened to my software complaints. To those who worked throughout the nights with me. To those who helped me write what I couldn't say. I cannot thank you enough.
}

\declaration{
I, Stefan Collier, declare that this dissertation and the work presented in it are my own and has been generated by me as the result of my own original research.\\
I confirm that:\\
1. This work was done wholly or mainly while in candidature for a degree at this University;\\
2. Where any part of this dissertation has previously been submitted for any other qualification at this University or any other institution, this has been clearly stated;\\
3. Where I have consulted the published work of others, this is always clearly attributed;\\
4. Where I have quoted from the work of others, the source is always given. With the exception of such quotations, this dissertation is entirely my own work;\\
5. I have acknowledged all main sources of help;\\
6. Where the thesis is based on work done by myself jointly with others, I have made clear exactly what was done by others and what I have contributed myself;\\
7. Either none of this work has been published before submission, or parts of this work have been published by :\\
\\
Stefan Collier\\
April 2016
}
\tableofcontents
\listoffigures
\listoftables

\mainmatter
%% ----------------------------------------------------------------
%\include{Introduction}
%\include{Conclusions}
\include{chapters/1Project/main}
\include{chapters/2Lit/main}
\include{chapters/3Design/HighLevel}
\include{chapters/3Design/InDepth}
\include{chapters/4Impl/main}

\include{chapters/5Experiments/1/main}
\include{chapters/5Experiments/2/main}
\include{chapters/5Experiments/3/main}
\include{chapters/5Experiments/4/main}

\include{chapters/6Conclusion/main}

\appendix
\include{appendix/AppendixB}
\include{appendix/D/main}
\include{appendix/AppendixC}

\backmatter
\bibliographystyle{ecs}
\bibliography{ECS}
\end{document}
%% ----------------------------------------------------------------

 %% ----------------------------------------------------------------
%% Progress.tex
%% ---------------------------------------------------------------- 
\documentclass{ecsprogress}    % Use the progress Style
\graphicspath{{../figs/}}   % Location of your graphics files
    \usepackage{natbib}            % Use Natbib style for the refs.
\hypersetup{colorlinks=true}   % Set to false for black/white printing
\input{Definitions}            % Include your abbreviations



\usepackage{enumitem}% http://ctan.org/pkg/enumitem
\usepackage{multirow}
\usepackage{float}
\usepackage{amsmath}
\usepackage{multicol}
\usepackage{amssymb}
\usepackage[normalem]{ulem}
\useunder{\uline}{\ul}{}
\usepackage{wrapfig}


\usepackage[table,xcdraw]{xcolor}


%% ----------------------------------------------------------------
\begin{document}
\frontmatter
\title      {Heterogeneous Agent-based Model for Supermarket Competition}
\authors    {\texorpdfstring
             {\href{mailto:sc22g13@ecs.soton.ac.uk}{Stefan J. Collier}}
             {Stefan J. Collier}
            }
\addresses  {\groupname\\\deptname\\\univname}
\date       {\today}
\subject    {}
\keywords   {}
\supervisor {Dr. Maria Polukarov}
\examiner   {Professor Sheng Chen}

\maketitle
\begin{abstract}
This project aim was to model and analyse the effects of competitive pricing behaviors of grocery retailers on the British market. 

This was achieved by creating a multi-agent model, containing retailer and consumer agents. The heterogeneous crowd of retailers employs either a uniform pricing strategy or a ‘local price flexing’ strategy. The actions of these retailers are chosen by predicting the profit of each action, using a perceptron. Following on from the consideration of different economic models, a discrete model was developed so that software agents have a discrete environment to operate within. Within the model, it has been observed how supermarkets with differing behaviors affect a heterogeneous crowd of consumer agents. The model was implemented in Java with Python used to evaluate the results. 

The simulation displays good acceptance with real grocery market behavior, i.e. captures the performance of British retailers thus can be used to determine the impact of changes in their behavior on their competitors and consumers.Furthermore it can be used to provide insight into sustainability of volatile pricing strategies, providing a useful insight in volatility of British supermarket retail industry. 
\end{abstract}
\acknowledgements{
I would like to express my sincere gratitude to Dr Maria Polukarov for her guidance and support which provided me the freedom to take this research in the direction of my interest.\\
\\
I would also like to thank my family and friends for their encouragement and support. To those who quietly listened to my software complaints. To those who worked throughout the nights with me. To those who helped me write what I couldn't say. I cannot thank you enough.
}

\declaration{
I, Stefan Collier, declare that this dissertation and the work presented in it are my own and has been generated by me as the result of my own original research.\\
I confirm that:\\
1. This work was done wholly or mainly while in candidature for a degree at this University;\\
2. Where any part of this dissertation has previously been submitted for any other qualification at this University or any other institution, this has been clearly stated;\\
3. Where I have consulted the published work of others, this is always clearly attributed;\\
4. Where I have quoted from the work of others, the source is always given. With the exception of such quotations, this dissertation is entirely my own work;\\
5. I have acknowledged all main sources of help;\\
6. Where the thesis is based on work done by myself jointly with others, I have made clear exactly what was done by others and what I have contributed myself;\\
7. Either none of this work has been published before submission, or parts of this work have been published by :\\
\\
Stefan Collier\\
April 2016
}
\tableofcontents
\listoffigures
\listoftables

\mainmatter
%% ----------------------------------------------------------------
%\include{Introduction}
%\include{Conclusions}
\include{chapters/1Project/main}
\include{chapters/2Lit/main}
\include{chapters/3Design/HighLevel}
\include{chapters/3Design/InDepth}
\include{chapters/4Impl/main}

\include{chapters/5Experiments/1/main}
\include{chapters/5Experiments/2/main}
\include{chapters/5Experiments/3/main}
\include{chapters/5Experiments/4/main}

\include{chapters/6Conclusion/main}

\appendix
\include{appendix/AppendixB}
\include{appendix/D/main}
\include{appendix/AppendixC}

\backmatter
\bibliographystyle{ecs}
\bibliography{ECS}
\end{document}
%% ----------------------------------------------------------------

\include{chapters/3Design/HighLevel}
\include{chapters/3Design/InDepth}
 %% ----------------------------------------------------------------
%% Progress.tex
%% ---------------------------------------------------------------- 
\documentclass{ecsprogress}    % Use the progress Style
\graphicspath{{../figs/}}   % Location of your graphics files
    \usepackage{natbib}            % Use Natbib style for the refs.
\hypersetup{colorlinks=true}   % Set to false for black/white printing
\input{Definitions}            % Include your abbreviations



\usepackage{enumitem}% http://ctan.org/pkg/enumitem
\usepackage{multirow}
\usepackage{float}
\usepackage{amsmath}
\usepackage{multicol}
\usepackage{amssymb}
\usepackage[normalem]{ulem}
\useunder{\uline}{\ul}{}
\usepackage{wrapfig}


\usepackage[table,xcdraw]{xcolor}


%% ----------------------------------------------------------------
\begin{document}
\frontmatter
\title      {Heterogeneous Agent-based Model for Supermarket Competition}
\authors    {\texorpdfstring
             {\href{mailto:sc22g13@ecs.soton.ac.uk}{Stefan J. Collier}}
             {Stefan J. Collier}
            }
\addresses  {\groupname\\\deptname\\\univname}
\date       {\today}
\subject    {}
\keywords   {}
\supervisor {Dr. Maria Polukarov}
\examiner   {Professor Sheng Chen}

\maketitle
\begin{abstract}
This project aim was to model and analyse the effects of competitive pricing behaviors of grocery retailers on the British market. 

This was achieved by creating a multi-agent model, containing retailer and consumer agents. The heterogeneous crowd of retailers employs either a uniform pricing strategy or a ‘local price flexing’ strategy. The actions of these retailers are chosen by predicting the profit of each action, using a perceptron. Following on from the consideration of different economic models, a discrete model was developed so that software agents have a discrete environment to operate within. Within the model, it has been observed how supermarkets with differing behaviors affect a heterogeneous crowd of consumer agents. The model was implemented in Java with Python used to evaluate the results. 

The simulation displays good acceptance with real grocery market behavior, i.e. captures the performance of British retailers thus can be used to determine the impact of changes in their behavior on their competitors and consumers.Furthermore it can be used to provide insight into sustainability of volatile pricing strategies, providing a useful insight in volatility of British supermarket retail industry. 
\end{abstract}
\acknowledgements{
I would like to express my sincere gratitude to Dr Maria Polukarov for her guidance and support which provided me the freedom to take this research in the direction of my interest.\\
\\
I would also like to thank my family and friends for their encouragement and support. To those who quietly listened to my software complaints. To those who worked throughout the nights with me. To those who helped me write what I couldn't say. I cannot thank you enough.
}

\declaration{
I, Stefan Collier, declare that this dissertation and the work presented in it are my own and has been generated by me as the result of my own original research.\\
I confirm that:\\
1. This work was done wholly or mainly while in candidature for a degree at this University;\\
2. Where any part of this dissertation has previously been submitted for any other qualification at this University or any other institution, this has been clearly stated;\\
3. Where I have consulted the published work of others, this is always clearly attributed;\\
4. Where I have quoted from the work of others, the source is always given. With the exception of such quotations, this dissertation is entirely my own work;\\
5. I have acknowledged all main sources of help;\\
6. Where the thesis is based on work done by myself jointly with others, I have made clear exactly what was done by others and what I have contributed myself;\\
7. Either none of this work has been published before submission, or parts of this work have been published by :\\
\\
Stefan Collier\\
April 2016
}
\tableofcontents
\listoffigures
\listoftables

\mainmatter
%% ----------------------------------------------------------------
%\include{Introduction}
%\include{Conclusions}
\include{chapters/1Project/main}
\include{chapters/2Lit/main}
\include{chapters/3Design/HighLevel}
\include{chapters/3Design/InDepth}
\include{chapters/4Impl/main}

\include{chapters/5Experiments/1/main}
\include{chapters/5Experiments/2/main}
\include{chapters/5Experiments/3/main}
\include{chapters/5Experiments/4/main}

\include{chapters/6Conclusion/main}

\appendix
\include{appendix/AppendixB}
\include{appendix/D/main}
\include{appendix/AppendixC}

\backmatter
\bibliographystyle{ecs}
\bibliography{ECS}
\end{document}
%% ----------------------------------------------------------------


 %% ----------------------------------------------------------------
%% Progress.tex
%% ---------------------------------------------------------------- 
\documentclass{ecsprogress}    % Use the progress Style
\graphicspath{{../figs/}}   % Location of your graphics files
    \usepackage{natbib}            % Use Natbib style for the refs.
\hypersetup{colorlinks=true}   % Set to false for black/white printing
\input{Definitions}            % Include your abbreviations



\usepackage{enumitem}% http://ctan.org/pkg/enumitem
\usepackage{multirow}
\usepackage{float}
\usepackage{amsmath}
\usepackage{multicol}
\usepackage{amssymb}
\usepackage[normalem]{ulem}
\useunder{\uline}{\ul}{}
\usepackage{wrapfig}


\usepackage[table,xcdraw]{xcolor}


%% ----------------------------------------------------------------
\begin{document}
\frontmatter
\title      {Heterogeneous Agent-based Model for Supermarket Competition}
\authors    {\texorpdfstring
             {\href{mailto:sc22g13@ecs.soton.ac.uk}{Stefan J. Collier}}
             {Stefan J. Collier}
            }
\addresses  {\groupname\\\deptname\\\univname}
\date       {\today}
\subject    {}
\keywords   {}
\supervisor {Dr. Maria Polukarov}
\examiner   {Professor Sheng Chen}

\maketitle
\begin{abstract}
This project aim was to model and analyse the effects of competitive pricing behaviors of grocery retailers on the British market. 

This was achieved by creating a multi-agent model, containing retailer and consumer agents. The heterogeneous crowd of retailers employs either a uniform pricing strategy or a ‘local price flexing’ strategy. The actions of these retailers are chosen by predicting the profit of each action, using a perceptron. Following on from the consideration of different economic models, a discrete model was developed so that software agents have a discrete environment to operate within. Within the model, it has been observed how supermarkets with differing behaviors affect a heterogeneous crowd of consumer agents. The model was implemented in Java with Python used to evaluate the results. 

The simulation displays good acceptance with real grocery market behavior, i.e. captures the performance of British retailers thus can be used to determine the impact of changes in their behavior on their competitors and consumers.Furthermore it can be used to provide insight into sustainability of volatile pricing strategies, providing a useful insight in volatility of British supermarket retail industry. 
\end{abstract}
\acknowledgements{
I would like to express my sincere gratitude to Dr Maria Polukarov for her guidance and support which provided me the freedom to take this research in the direction of my interest.\\
\\
I would also like to thank my family and friends for their encouragement and support. To those who quietly listened to my software complaints. To those who worked throughout the nights with me. To those who helped me write what I couldn't say. I cannot thank you enough.
}

\declaration{
I, Stefan Collier, declare that this dissertation and the work presented in it are my own and has been generated by me as the result of my own original research.\\
I confirm that:\\
1. This work was done wholly or mainly while in candidature for a degree at this University;\\
2. Where any part of this dissertation has previously been submitted for any other qualification at this University or any other institution, this has been clearly stated;\\
3. Where I have consulted the published work of others, this is always clearly attributed;\\
4. Where I have quoted from the work of others, the source is always given. With the exception of such quotations, this dissertation is entirely my own work;\\
5. I have acknowledged all main sources of help;\\
6. Where the thesis is based on work done by myself jointly with others, I have made clear exactly what was done by others and what I have contributed myself;\\
7. Either none of this work has been published before submission, or parts of this work have been published by :\\
\\
Stefan Collier\\
April 2016
}
\tableofcontents
\listoffigures
\listoftables

\mainmatter
%% ----------------------------------------------------------------
%\include{Introduction}
%\include{Conclusions}
\include{chapters/1Project/main}
\include{chapters/2Lit/main}
\include{chapters/3Design/HighLevel}
\include{chapters/3Design/InDepth}
\include{chapters/4Impl/main}

\include{chapters/5Experiments/1/main}
\include{chapters/5Experiments/2/main}
\include{chapters/5Experiments/3/main}
\include{chapters/5Experiments/4/main}

\include{chapters/6Conclusion/main}

\appendix
\include{appendix/AppendixB}
\include{appendix/D/main}
\include{appendix/AppendixC}

\backmatter
\bibliographystyle{ecs}
\bibliography{ECS}
\end{document}
%% ----------------------------------------------------------------

 %% ----------------------------------------------------------------
%% Progress.tex
%% ---------------------------------------------------------------- 
\documentclass{ecsprogress}    % Use the progress Style
\graphicspath{{../figs/}}   % Location of your graphics files
    \usepackage{natbib}            % Use Natbib style for the refs.
\hypersetup{colorlinks=true}   % Set to false for black/white printing
\input{Definitions}            % Include your abbreviations



\usepackage{enumitem}% http://ctan.org/pkg/enumitem
\usepackage{multirow}
\usepackage{float}
\usepackage{amsmath}
\usepackage{multicol}
\usepackage{amssymb}
\usepackage[normalem]{ulem}
\useunder{\uline}{\ul}{}
\usepackage{wrapfig}


\usepackage[table,xcdraw]{xcolor}


%% ----------------------------------------------------------------
\begin{document}
\frontmatter
\title      {Heterogeneous Agent-based Model for Supermarket Competition}
\authors    {\texorpdfstring
             {\href{mailto:sc22g13@ecs.soton.ac.uk}{Stefan J. Collier}}
             {Stefan J. Collier}
            }
\addresses  {\groupname\\\deptname\\\univname}
\date       {\today}
\subject    {}
\keywords   {}
\supervisor {Dr. Maria Polukarov}
\examiner   {Professor Sheng Chen}

\maketitle
\begin{abstract}
This project aim was to model and analyse the effects of competitive pricing behaviors of grocery retailers on the British market. 

This was achieved by creating a multi-agent model, containing retailer and consumer agents. The heterogeneous crowd of retailers employs either a uniform pricing strategy or a ‘local price flexing’ strategy. The actions of these retailers are chosen by predicting the profit of each action, using a perceptron. Following on from the consideration of different economic models, a discrete model was developed so that software agents have a discrete environment to operate within. Within the model, it has been observed how supermarkets with differing behaviors affect a heterogeneous crowd of consumer agents. The model was implemented in Java with Python used to evaluate the results. 

The simulation displays good acceptance with real grocery market behavior, i.e. captures the performance of British retailers thus can be used to determine the impact of changes in their behavior on their competitors and consumers.Furthermore it can be used to provide insight into sustainability of volatile pricing strategies, providing a useful insight in volatility of British supermarket retail industry. 
\end{abstract}
\acknowledgements{
I would like to express my sincere gratitude to Dr Maria Polukarov for her guidance and support which provided me the freedom to take this research in the direction of my interest.\\
\\
I would also like to thank my family and friends for their encouragement and support. To those who quietly listened to my software complaints. To those who worked throughout the nights with me. To those who helped me write what I couldn't say. I cannot thank you enough.
}

\declaration{
I, Stefan Collier, declare that this dissertation and the work presented in it are my own and has been generated by me as the result of my own original research.\\
I confirm that:\\
1. This work was done wholly or mainly while in candidature for a degree at this University;\\
2. Where any part of this dissertation has previously been submitted for any other qualification at this University or any other institution, this has been clearly stated;\\
3. Where I have consulted the published work of others, this is always clearly attributed;\\
4. Where I have quoted from the work of others, the source is always given. With the exception of such quotations, this dissertation is entirely my own work;\\
5. I have acknowledged all main sources of help;\\
6. Where the thesis is based on work done by myself jointly with others, I have made clear exactly what was done by others and what I have contributed myself;\\
7. Either none of this work has been published before submission, or parts of this work have been published by :\\
\\
Stefan Collier\\
April 2016
}
\tableofcontents
\listoffigures
\listoftables

\mainmatter
%% ----------------------------------------------------------------
%\include{Introduction}
%\include{Conclusions}
\include{chapters/1Project/main}
\include{chapters/2Lit/main}
\include{chapters/3Design/HighLevel}
\include{chapters/3Design/InDepth}
\include{chapters/4Impl/main}

\include{chapters/5Experiments/1/main}
\include{chapters/5Experiments/2/main}
\include{chapters/5Experiments/3/main}
\include{chapters/5Experiments/4/main}

\include{chapters/6Conclusion/main}

\appendix
\include{appendix/AppendixB}
\include{appendix/D/main}
\include{appendix/AppendixC}

\backmatter
\bibliographystyle{ecs}
\bibliography{ECS}
\end{document}
%% ----------------------------------------------------------------

 %% ----------------------------------------------------------------
%% Progress.tex
%% ---------------------------------------------------------------- 
\documentclass{ecsprogress}    % Use the progress Style
\graphicspath{{../figs/}}   % Location of your graphics files
    \usepackage{natbib}            % Use Natbib style for the refs.
\hypersetup{colorlinks=true}   % Set to false for black/white printing
\input{Definitions}            % Include your abbreviations



\usepackage{enumitem}% http://ctan.org/pkg/enumitem
\usepackage{multirow}
\usepackage{float}
\usepackage{amsmath}
\usepackage{multicol}
\usepackage{amssymb}
\usepackage[normalem]{ulem}
\useunder{\uline}{\ul}{}
\usepackage{wrapfig}


\usepackage[table,xcdraw]{xcolor}


%% ----------------------------------------------------------------
\begin{document}
\frontmatter
\title      {Heterogeneous Agent-based Model for Supermarket Competition}
\authors    {\texorpdfstring
             {\href{mailto:sc22g13@ecs.soton.ac.uk}{Stefan J. Collier}}
             {Stefan J. Collier}
            }
\addresses  {\groupname\\\deptname\\\univname}
\date       {\today}
\subject    {}
\keywords   {}
\supervisor {Dr. Maria Polukarov}
\examiner   {Professor Sheng Chen}

\maketitle
\begin{abstract}
This project aim was to model and analyse the effects of competitive pricing behaviors of grocery retailers on the British market. 

This was achieved by creating a multi-agent model, containing retailer and consumer agents. The heterogeneous crowd of retailers employs either a uniform pricing strategy or a ‘local price flexing’ strategy. The actions of these retailers are chosen by predicting the profit of each action, using a perceptron. Following on from the consideration of different economic models, a discrete model was developed so that software agents have a discrete environment to operate within. Within the model, it has been observed how supermarkets with differing behaviors affect a heterogeneous crowd of consumer agents. The model was implemented in Java with Python used to evaluate the results. 

The simulation displays good acceptance with real grocery market behavior, i.e. captures the performance of British retailers thus can be used to determine the impact of changes in their behavior on their competitors and consumers.Furthermore it can be used to provide insight into sustainability of volatile pricing strategies, providing a useful insight in volatility of British supermarket retail industry. 
\end{abstract}
\acknowledgements{
I would like to express my sincere gratitude to Dr Maria Polukarov for her guidance and support which provided me the freedom to take this research in the direction of my interest.\\
\\
I would also like to thank my family and friends for their encouragement and support. To those who quietly listened to my software complaints. To those who worked throughout the nights with me. To those who helped me write what I couldn't say. I cannot thank you enough.
}

\declaration{
I, Stefan Collier, declare that this dissertation and the work presented in it are my own and has been generated by me as the result of my own original research.\\
I confirm that:\\
1. This work was done wholly or mainly while in candidature for a degree at this University;\\
2. Where any part of this dissertation has previously been submitted for any other qualification at this University or any other institution, this has been clearly stated;\\
3. Where I have consulted the published work of others, this is always clearly attributed;\\
4. Where I have quoted from the work of others, the source is always given. With the exception of such quotations, this dissertation is entirely my own work;\\
5. I have acknowledged all main sources of help;\\
6. Where the thesis is based on work done by myself jointly with others, I have made clear exactly what was done by others and what I have contributed myself;\\
7. Either none of this work has been published before submission, or parts of this work have been published by :\\
\\
Stefan Collier\\
April 2016
}
\tableofcontents
\listoffigures
\listoftables

\mainmatter
%% ----------------------------------------------------------------
%\include{Introduction}
%\include{Conclusions}
\include{chapters/1Project/main}
\include{chapters/2Lit/main}
\include{chapters/3Design/HighLevel}
\include{chapters/3Design/InDepth}
\include{chapters/4Impl/main}

\include{chapters/5Experiments/1/main}
\include{chapters/5Experiments/2/main}
\include{chapters/5Experiments/3/main}
\include{chapters/5Experiments/4/main}

\include{chapters/6Conclusion/main}

\appendix
\include{appendix/AppendixB}
\include{appendix/D/main}
\include{appendix/AppendixC}

\backmatter
\bibliographystyle{ecs}
\bibliography{ECS}
\end{document}
%% ----------------------------------------------------------------

 %% ----------------------------------------------------------------
%% Progress.tex
%% ---------------------------------------------------------------- 
\documentclass{ecsprogress}    % Use the progress Style
\graphicspath{{../figs/}}   % Location of your graphics files
    \usepackage{natbib}            % Use Natbib style for the refs.
\hypersetup{colorlinks=true}   % Set to false for black/white printing
\input{Definitions}            % Include your abbreviations



\usepackage{enumitem}% http://ctan.org/pkg/enumitem
\usepackage{multirow}
\usepackage{float}
\usepackage{amsmath}
\usepackage{multicol}
\usepackage{amssymb}
\usepackage[normalem]{ulem}
\useunder{\uline}{\ul}{}
\usepackage{wrapfig}


\usepackage[table,xcdraw]{xcolor}


%% ----------------------------------------------------------------
\begin{document}
\frontmatter
\title      {Heterogeneous Agent-based Model for Supermarket Competition}
\authors    {\texorpdfstring
             {\href{mailto:sc22g13@ecs.soton.ac.uk}{Stefan J. Collier}}
             {Stefan J. Collier}
            }
\addresses  {\groupname\\\deptname\\\univname}
\date       {\today}
\subject    {}
\keywords   {}
\supervisor {Dr. Maria Polukarov}
\examiner   {Professor Sheng Chen}

\maketitle
\begin{abstract}
This project aim was to model and analyse the effects of competitive pricing behaviors of grocery retailers on the British market. 

This was achieved by creating a multi-agent model, containing retailer and consumer agents. The heterogeneous crowd of retailers employs either a uniform pricing strategy or a ‘local price flexing’ strategy. The actions of these retailers are chosen by predicting the profit of each action, using a perceptron. Following on from the consideration of different economic models, a discrete model was developed so that software agents have a discrete environment to operate within. Within the model, it has been observed how supermarkets with differing behaviors affect a heterogeneous crowd of consumer agents. The model was implemented in Java with Python used to evaluate the results. 

The simulation displays good acceptance with real grocery market behavior, i.e. captures the performance of British retailers thus can be used to determine the impact of changes in their behavior on their competitors and consumers.Furthermore it can be used to provide insight into sustainability of volatile pricing strategies, providing a useful insight in volatility of British supermarket retail industry. 
\end{abstract}
\acknowledgements{
I would like to express my sincere gratitude to Dr Maria Polukarov for her guidance and support which provided me the freedom to take this research in the direction of my interest.\\
\\
I would also like to thank my family and friends for their encouragement and support. To those who quietly listened to my software complaints. To those who worked throughout the nights with me. To those who helped me write what I couldn't say. I cannot thank you enough.
}

\declaration{
I, Stefan Collier, declare that this dissertation and the work presented in it are my own and has been generated by me as the result of my own original research.\\
I confirm that:\\
1. This work was done wholly or mainly while in candidature for a degree at this University;\\
2. Where any part of this dissertation has previously been submitted for any other qualification at this University or any other institution, this has been clearly stated;\\
3. Where I have consulted the published work of others, this is always clearly attributed;\\
4. Where I have quoted from the work of others, the source is always given. With the exception of such quotations, this dissertation is entirely my own work;\\
5. I have acknowledged all main sources of help;\\
6. Where the thesis is based on work done by myself jointly with others, I have made clear exactly what was done by others and what I have contributed myself;\\
7. Either none of this work has been published before submission, or parts of this work have been published by :\\
\\
Stefan Collier\\
April 2016
}
\tableofcontents
\listoffigures
\listoftables

\mainmatter
%% ----------------------------------------------------------------
%\include{Introduction}
%\include{Conclusions}
\include{chapters/1Project/main}
\include{chapters/2Lit/main}
\include{chapters/3Design/HighLevel}
\include{chapters/3Design/InDepth}
\include{chapters/4Impl/main}

\include{chapters/5Experiments/1/main}
\include{chapters/5Experiments/2/main}
\include{chapters/5Experiments/3/main}
\include{chapters/5Experiments/4/main}

\include{chapters/6Conclusion/main}

\appendix
\include{appendix/AppendixB}
\include{appendix/D/main}
\include{appendix/AppendixC}

\backmatter
\bibliographystyle{ecs}
\bibliography{ECS}
\end{document}
%% ----------------------------------------------------------------


 %% ----------------------------------------------------------------
%% Progress.tex
%% ---------------------------------------------------------------- 
\documentclass{ecsprogress}    % Use the progress Style
\graphicspath{{../figs/}}   % Location of your graphics files
    \usepackage{natbib}            % Use Natbib style for the refs.
\hypersetup{colorlinks=true}   % Set to false for black/white printing
\input{Definitions}            % Include your abbreviations



\usepackage{enumitem}% http://ctan.org/pkg/enumitem
\usepackage{multirow}
\usepackage{float}
\usepackage{amsmath}
\usepackage{multicol}
\usepackage{amssymb}
\usepackage[normalem]{ulem}
\useunder{\uline}{\ul}{}
\usepackage{wrapfig}


\usepackage[table,xcdraw]{xcolor}


%% ----------------------------------------------------------------
\begin{document}
\frontmatter
\title      {Heterogeneous Agent-based Model for Supermarket Competition}
\authors    {\texorpdfstring
             {\href{mailto:sc22g13@ecs.soton.ac.uk}{Stefan J. Collier}}
             {Stefan J. Collier}
            }
\addresses  {\groupname\\\deptname\\\univname}
\date       {\today}
\subject    {}
\keywords   {}
\supervisor {Dr. Maria Polukarov}
\examiner   {Professor Sheng Chen}

\maketitle
\begin{abstract}
This project aim was to model and analyse the effects of competitive pricing behaviors of grocery retailers on the British market. 

This was achieved by creating a multi-agent model, containing retailer and consumer agents. The heterogeneous crowd of retailers employs either a uniform pricing strategy or a ‘local price flexing’ strategy. The actions of these retailers are chosen by predicting the profit of each action, using a perceptron. Following on from the consideration of different economic models, a discrete model was developed so that software agents have a discrete environment to operate within. Within the model, it has been observed how supermarkets with differing behaviors affect a heterogeneous crowd of consumer agents. The model was implemented in Java with Python used to evaluate the results. 

The simulation displays good acceptance with real grocery market behavior, i.e. captures the performance of British retailers thus can be used to determine the impact of changes in their behavior on their competitors and consumers.Furthermore it can be used to provide insight into sustainability of volatile pricing strategies, providing a useful insight in volatility of British supermarket retail industry. 
\end{abstract}
\acknowledgements{
I would like to express my sincere gratitude to Dr Maria Polukarov for her guidance and support which provided me the freedom to take this research in the direction of my interest.\\
\\
I would also like to thank my family and friends for their encouragement and support. To those who quietly listened to my software complaints. To those who worked throughout the nights with me. To those who helped me write what I couldn't say. I cannot thank you enough.
}

\declaration{
I, Stefan Collier, declare that this dissertation and the work presented in it are my own and has been generated by me as the result of my own original research.\\
I confirm that:\\
1. This work was done wholly or mainly while in candidature for a degree at this University;\\
2. Where any part of this dissertation has previously been submitted for any other qualification at this University or any other institution, this has been clearly stated;\\
3. Where I have consulted the published work of others, this is always clearly attributed;\\
4. Where I have quoted from the work of others, the source is always given. With the exception of such quotations, this dissertation is entirely my own work;\\
5. I have acknowledged all main sources of help;\\
6. Where the thesis is based on work done by myself jointly with others, I have made clear exactly what was done by others and what I have contributed myself;\\
7. Either none of this work has been published before submission, or parts of this work have been published by :\\
\\
Stefan Collier\\
April 2016
}
\tableofcontents
\listoffigures
\listoftables

\mainmatter
%% ----------------------------------------------------------------
%\include{Introduction}
%\include{Conclusions}
\include{chapters/1Project/main}
\include{chapters/2Lit/main}
\include{chapters/3Design/HighLevel}
\include{chapters/3Design/InDepth}
\include{chapters/4Impl/main}

\include{chapters/5Experiments/1/main}
\include{chapters/5Experiments/2/main}
\include{chapters/5Experiments/3/main}
\include{chapters/5Experiments/4/main}

\include{chapters/6Conclusion/main}

\appendix
\include{appendix/AppendixB}
\include{appendix/D/main}
\include{appendix/AppendixC}

\backmatter
\bibliographystyle{ecs}
\bibliography{ECS}
\end{document}
%% ----------------------------------------------------------------


\appendix
\include{appendix/AppendixB}
 %% ----------------------------------------------------------------
%% Progress.tex
%% ---------------------------------------------------------------- 
\documentclass{ecsprogress}    % Use the progress Style
\graphicspath{{../figs/}}   % Location of your graphics files
    \usepackage{natbib}            % Use Natbib style for the refs.
\hypersetup{colorlinks=true}   % Set to false for black/white printing
\input{Definitions}            % Include your abbreviations



\usepackage{enumitem}% http://ctan.org/pkg/enumitem
\usepackage{multirow}
\usepackage{float}
\usepackage{amsmath}
\usepackage{multicol}
\usepackage{amssymb}
\usepackage[normalem]{ulem}
\useunder{\uline}{\ul}{}
\usepackage{wrapfig}


\usepackage[table,xcdraw]{xcolor}


%% ----------------------------------------------------------------
\begin{document}
\frontmatter
\title      {Heterogeneous Agent-based Model for Supermarket Competition}
\authors    {\texorpdfstring
             {\href{mailto:sc22g13@ecs.soton.ac.uk}{Stefan J. Collier}}
             {Stefan J. Collier}
            }
\addresses  {\groupname\\\deptname\\\univname}
\date       {\today}
\subject    {}
\keywords   {}
\supervisor {Dr. Maria Polukarov}
\examiner   {Professor Sheng Chen}

\maketitle
\begin{abstract}
This project aim was to model and analyse the effects of competitive pricing behaviors of grocery retailers on the British market. 

This was achieved by creating a multi-agent model, containing retailer and consumer agents. The heterogeneous crowd of retailers employs either a uniform pricing strategy or a ‘local price flexing’ strategy. The actions of these retailers are chosen by predicting the profit of each action, using a perceptron. Following on from the consideration of different economic models, a discrete model was developed so that software agents have a discrete environment to operate within. Within the model, it has been observed how supermarkets with differing behaviors affect a heterogeneous crowd of consumer agents. The model was implemented in Java with Python used to evaluate the results. 

The simulation displays good acceptance with real grocery market behavior, i.e. captures the performance of British retailers thus can be used to determine the impact of changes in their behavior on their competitors and consumers.Furthermore it can be used to provide insight into sustainability of volatile pricing strategies, providing a useful insight in volatility of British supermarket retail industry. 
\end{abstract}
\acknowledgements{
I would like to express my sincere gratitude to Dr Maria Polukarov for her guidance and support which provided me the freedom to take this research in the direction of my interest.\\
\\
I would also like to thank my family and friends for their encouragement and support. To those who quietly listened to my software complaints. To those who worked throughout the nights with me. To those who helped me write what I couldn't say. I cannot thank you enough.
}

\declaration{
I, Stefan Collier, declare that this dissertation and the work presented in it are my own and has been generated by me as the result of my own original research.\\
I confirm that:\\
1. This work was done wholly or mainly while in candidature for a degree at this University;\\
2. Where any part of this dissertation has previously been submitted for any other qualification at this University or any other institution, this has been clearly stated;\\
3. Where I have consulted the published work of others, this is always clearly attributed;\\
4. Where I have quoted from the work of others, the source is always given. With the exception of such quotations, this dissertation is entirely my own work;\\
5. I have acknowledged all main sources of help;\\
6. Where the thesis is based on work done by myself jointly with others, I have made clear exactly what was done by others and what I have contributed myself;\\
7. Either none of this work has been published before submission, or parts of this work have been published by :\\
\\
Stefan Collier\\
April 2016
}
\tableofcontents
\listoffigures
\listoftables

\mainmatter
%% ----------------------------------------------------------------
%\include{Introduction}
%\include{Conclusions}
\include{chapters/1Project/main}
\include{chapters/2Lit/main}
\include{chapters/3Design/HighLevel}
\include{chapters/3Design/InDepth}
\include{chapters/4Impl/main}

\include{chapters/5Experiments/1/main}
\include{chapters/5Experiments/2/main}
\include{chapters/5Experiments/3/main}
\include{chapters/5Experiments/4/main}

\include{chapters/6Conclusion/main}

\appendix
\include{appendix/AppendixB}
\include{appendix/D/main}
\include{appendix/AppendixC}

\backmatter
\bibliographystyle{ecs}
\bibliography{ECS}
\end{document}
%% ----------------------------------------------------------------

\include{appendix/AppendixC}

\backmatter
\bibliographystyle{ecs}
\bibliography{ECS}
\end{document}
%% ----------------------------------------------------------------

 %% ----------------------------------------------------------------
%% Progress.tex
%% ---------------------------------------------------------------- 
\documentclass{ecsprogress}    % Use the progress Style
\graphicspath{{../figs/}}   % Location of your graphics files
    \usepackage{natbib}            % Use Natbib style for the refs.
\hypersetup{colorlinks=true}   % Set to false for black/white printing
\input{Definitions}            % Include your abbreviations



\usepackage{enumitem}% http://ctan.org/pkg/enumitem
\usepackage{multirow}
\usepackage{float}
\usepackage{amsmath}
\usepackage{multicol}
\usepackage{amssymb}
\usepackage[normalem]{ulem}
\useunder{\uline}{\ul}{}
\usepackage{wrapfig}


\usepackage[table,xcdraw]{xcolor}


%% ----------------------------------------------------------------
\begin{document}
\frontmatter
\title      {Heterogeneous Agent-based Model for Supermarket Competition}
\authors    {\texorpdfstring
             {\href{mailto:sc22g13@ecs.soton.ac.uk}{Stefan J. Collier}}
             {Stefan J. Collier}
            }
\addresses  {\groupname\\\deptname\\\univname}
\date       {\today}
\subject    {}
\keywords   {}
\supervisor {Dr. Maria Polukarov}
\examiner   {Professor Sheng Chen}

\maketitle
\begin{abstract}
This project aim was to model and analyse the effects of competitive pricing behaviors of grocery retailers on the British market. 

This was achieved by creating a multi-agent model, containing retailer and consumer agents. The heterogeneous crowd of retailers employs either a uniform pricing strategy or a ‘local price flexing’ strategy. The actions of these retailers are chosen by predicting the profit of each action, using a perceptron. Following on from the consideration of different economic models, a discrete model was developed so that software agents have a discrete environment to operate within. Within the model, it has been observed how supermarkets with differing behaviors affect a heterogeneous crowd of consumer agents. The model was implemented in Java with Python used to evaluate the results. 

The simulation displays good acceptance with real grocery market behavior, i.e. captures the performance of British retailers thus can be used to determine the impact of changes in their behavior on their competitors and consumers.Furthermore it can be used to provide insight into sustainability of volatile pricing strategies, providing a useful insight in volatility of British supermarket retail industry. 
\end{abstract}
\acknowledgements{
I would like to express my sincere gratitude to Dr Maria Polukarov for her guidance and support which provided me the freedom to take this research in the direction of my interest.\\
\\
I would also like to thank my family and friends for their encouragement and support. To those who quietly listened to my software complaints. To those who worked throughout the nights with me. To those who helped me write what I couldn't say. I cannot thank you enough.
}

\declaration{
I, Stefan Collier, declare that this dissertation and the work presented in it are my own and has been generated by me as the result of my own original research.\\
I confirm that:\\
1. This work was done wholly or mainly while in candidature for a degree at this University;\\
2. Where any part of this dissertation has previously been submitted for any other qualification at this University or any other institution, this has been clearly stated;\\
3. Where I have consulted the published work of others, this is always clearly attributed;\\
4. Where I have quoted from the work of others, the source is always given. With the exception of such quotations, this dissertation is entirely my own work;\\
5. I have acknowledged all main sources of help;\\
6. Where the thesis is based on work done by myself jointly with others, I have made clear exactly what was done by others and what I have contributed myself;\\
7. Either none of this work has been published before submission, or parts of this work have been published by :\\
\\
Stefan Collier\\
April 2016
}
\tableofcontents
\listoffigures
\listoftables

\mainmatter
%% ----------------------------------------------------------------
%\include{Introduction}
%\include{Conclusions}
 %% ----------------------------------------------------------------
%% Progress.tex
%% ---------------------------------------------------------------- 
\documentclass{ecsprogress}    % Use the progress Style
\graphicspath{{../figs/}}   % Location of your graphics files
    \usepackage{natbib}            % Use Natbib style for the refs.
\hypersetup{colorlinks=true}   % Set to false for black/white printing
\input{Definitions}            % Include your abbreviations



\usepackage{enumitem}% http://ctan.org/pkg/enumitem
\usepackage{multirow}
\usepackage{float}
\usepackage{amsmath}
\usepackage{multicol}
\usepackage{amssymb}
\usepackage[normalem]{ulem}
\useunder{\uline}{\ul}{}
\usepackage{wrapfig}


\usepackage[table,xcdraw]{xcolor}


%% ----------------------------------------------------------------
\begin{document}
\frontmatter
\title      {Heterogeneous Agent-based Model for Supermarket Competition}
\authors    {\texorpdfstring
             {\href{mailto:sc22g13@ecs.soton.ac.uk}{Stefan J. Collier}}
             {Stefan J. Collier}
            }
\addresses  {\groupname\\\deptname\\\univname}
\date       {\today}
\subject    {}
\keywords   {}
\supervisor {Dr. Maria Polukarov}
\examiner   {Professor Sheng Chen}

\maketitle
\begin{abstract}
This project aim was to model and analyse the effects of competitive pricing behaviors of grocery retailers on the British market. 

This was achieved by creating a multi-agent model, containing retailer and consumer agents. The heterogeneous crowd of retailers employs either a uniform pricing strategy or a ‘local price flexing’ strategy. The actions of these retailers are chosen by predicting the profit of each action, using a perceptron. Following on from the consideration of different economic models, a discrete model was developed so that software agents have a discrete environment to operate within. Within the model, it has been observed how supermarkets with differing behaviors affect a heterogeneous crowd of consumer agents. The model was implemented in Java with Python used to evaluate the results. 

The simulation displays good acceptance with real grocery market behavior, i.e. captures the performance of British retailers thus can be used to determine the impact of changes in their behavior on their competitors and consumers.Furthermore it can be used to provide insight into sustainability of volatile pricing strategies, providing a useful insight in volatility of British supermarket retail industry. 
\end{abstract}
\acknowledgements{
I would like to express my sincere gratitude to Dr Maria Polukarov for her guidance and support which provided me the freedom to take this research in the direction of my interest.\\
\\
I would also like to thank my family and friends for their encouragement and support. To those who quietly listened to my software complaints. To those who worked throughout the nights with me. To those who helped me write what I couldn't say. I cannot thank you enough.
}

\declaration{
I, Stefan Collier, declare that this dissertation and the work presented in it are my own and has been generated by me as the result of my own original research.\\
I confirm that:\\
1. This work was done wholly or mainly while in candidature for a degree at this University;\\
2. Where any part of this dissertation has previously been submitted for any other qualification at this University or any other institution, this has been clearly stated;\\
3. Where I have consulted the published work of others, this is always clearly attributed;\\
4. Where I have quoted from the work of others, the source is always given. With the exception of such quotations, this dissertation is entirely my own work;\\
5. I have acknowledged all main sources of help;\\
6. Where the thesis is based on work done by myself jointly with others, I have made clear exactly what was done by others and what I have contributed myself;\\
7. Either none of this work has been published before submission, or parts of this work have been published by :\\
\\
Stefan Collier\\
April 2016
}
\tableofcontents
\listoffigures
\listoftables

\mainmatter
%% ----------------------------------------------------------------
%\include{Introduction}
%\include{Conclusions}
\include{chapters/1Project/main}
\include{chapters/2Lit/main}
\include{chapters/3Design/HighLevel}
\include{chapters/3Design/InDepth}
\include{chapters/4Impl/main}

\include{chapters/5Experiments/1/main}
\include{chapters/5Experiments/2/main}
\include{chapters/5Experiments/3/main}
\include{chapters/5Experiments/4/main}

\include{chapters/6Conclusion/main}

\appendix
\include{appendix/AppendixB}
\include{appendix/D/main}
\include{appendix/AppendixC}

\backmatter
\bibliographystyle{ecs}
\bibliography{ECS}
\end{document}
%% ----------------------------------------------------------------

 %% ----------------------------------------------------------------
%% Progress.tex
%% ---------------------------------------------------------------- 
\documentclass{ecsprogress}    % Use the progress Style
\graphicspath{{../figs/}}   % Location of your graphics files
    \usepackage{natbib}            % Use Natbib style for the refs.
\hypersetup{colorlinks=true}   % Set to false for black/white printing
\input{Definitions}            % Include your abbreviations



\usepackage{enumitem}% http://ctan.org/pkg/enumitem
\usepackage{multirow}
\usepackage{float}
\usepackage{amsmath}
\usepackage{multicol}
\usepackage{amssymb}
\usepackage[normalem]{ulem}
\useunder{\uline}{\ul}{}
\usepackage{wrapfig}


\usepackage[table,xcdraw]{xcolor}


%% ----------------------------------------------------------------
\begin{document}
\frontmatter
\title      {Heterogeneous Agent-based Model for Supermarket Competition}
\authors    {\texorpdfstring
             {\href{mailto:sc22g13@ecs.soton.ac.uk}{Stefan J. Collier}}
             {Stefan J. Collier}
            }
\addresses  {\groupname\\\deptname\\\univname}
\date       {\today}
\subject    {}
\keywords   {}
\supervisor {Dr. Maria Polukarov}
\examiner   {Professor Sheng Chen}

\maketitle
\begin{abstract}
This project aim was to model and analyse the effects of competitive pricing behaviors of grocery retailers on the British market. 

This was achieved by creating a multi-agent model, containing retailer and consumer agents. The heterogeneous crowd of retailers employs either a uniform pricing strategy or a ‘local price flexing’ strategy. The actions of these retailers are chosen by predicting the profit of each action, using a perceptron. Following on from the consideration of different economic models, a discrete model was developed so that software agents have a discrete environment to operate within. Within the model, it has been observed how supermarkets with differing behaviors affect a heterogeneous crowd of consumer agents. The model was implemented in Java with Python used to evaluate the results. 

The simulation displays good acceptance with real grocery market behavior, i.e. captures the performance of British retailers thus can be used to determine the impact of changes in their behavior on their competitors and consumers.Furthermore it can be used to provide insight into sustainability of volatile pricing strategies, providing a useful insight in volatility of British supermarket retail industry. 
\end{abstract}
\acknowledgements{
I would like to express my sincere gratitude to Dr Maria Polukarov for her guidance and support which provided me the freedom to take this research in the direction of my interest.\\
\\
I would also like to thank my family and friends for their encouragement and support. To those who quietly listened to my software complaints. To those who worked throughout the nights with me. To those who helped me write what I couldn't say. I cannot thank you enough.
}

\declaration{
I, Stefan Collier, declare that this dissertation and the work presented in it are my own and has been generated by me as the result of my own original research.\\
I confirm that:\\
1. This work was done wholly or mainly while in candidature for a degree at this University;\\
2. Where any part of this dissertation has previously been submitted for any other qualification at this University or any other institution, this has been clearly stated;\\
3. Where I have consulted the published work of others, this is always clearly attributed;\\
4. Where I have quoted from the work of others, the source is always given. With the exception of such quotations, this dissertation is entirely my own work;\\
5. I have acknowledged all main sources of help;\\
6. Where the thesis is based on work done by myself jointly with others, I have made clear exactly what was done by others and what I have contributed myself;\\
7. Either none of this work has been published before submission, or parts of this work have been published by :\\
\\
Stefan Collier\\
April 2016
}
\tableofcontents
\listoffigures
\listoftables

\mainmatter
%% ----------------------------------------------------------------
%\include{Introduction}
%\include{Conclusions}
\include{chapters/1Project/main}
\include{chapters/2Lit/main}
\include{chapters/3Design/HighLevel}
\include{chapters/3Design/InDepth}
\include{chapters/4Impl/main}

\include{chapters/5Experiments/1/main}
\include{chapters/5Experiments/2/main}
\include{chapters/5Experiments/3/main}
\include{chapters/5Experiments/4/main}

\include{chapters/6Conclusion/main}

\appendix
\include{appendix/AppendixB}
\include{appendix/D/main}
\include{appendix/AppendixC}

\backmatter
\bibliographystyle{ecs}
\bibliography{ECS}
\end{document}
%% ----------------------------------------------------------------

\include{chapters/3Design/HighLevel}
\include{chapters/3Design/InDepth}
 %% ----------------------------------------------------------------
%% Progress.tex
%% ---------------------------------------------------------------- 
\documentclass{ecsprogress}    % Use the progress Style
\graphicspath{{../figs/}}   % Location of your graphics files
    \usepackage{natbib}            % Use Natbib style for the refs.
\hypersetup{colorlinks=true}   % Set to false for black/white printing
\input{Definitions}            % Include your abbreviations



\usepackage{enumitem}% http://ctan.org/pkg/enumitem
\usepackage{multirow}
\usepackage{float}
\usepackage{amsmath}
\usepackage{multicol}
\usepackage{amssymb}
\usepackage[normalem]{ulem}
\useunder{\uline}{\ul}{}
\usepackage{wrapfig}


\usepackage[table,xcdraw]{xcolor}


%% ----------------------------------------------------------------
\begin{document}
\frontmatter
\title      {Heterogeneous Agent-based Model for Supermarket Competition}
\authors    {\texorpdfstring
             {\href{mailto:sc22g13@ecs.soton.ac.uk}{Stefan J. Collier}}
             {Stefan J. Collier}
            }
\addresses  {\groupname\\\deptname\\\univname}
\date       {\today}
\subject    {}
\keywords   {}
\supervisor {Dr. Maria Polukarov}
\examiner   {Professor Sheng Chen}

\maketitle
\begin{abstract}
This project aim was to model and analyse the effects of competitive pricing behaviors of grocery retailers on the British market. 

This was achieved by creating a multi-agent model, containing retailer and consumer agents. The heterogeneous crowd of retailers employs either a uniform pricing strategy or a ‘local price flexing’ strategy. The actions of these retailers are chosen by predicting the profit of each action, using a perceptron. Following on from the consideration of different economic models, a discrete model was developed so that software agents have a discrete environment to operate within. Within the model, it has been observed how supermarkets with differing behaviors affect a heterogeneous crowd of consumer agents. The model was implemented in Java with Python used to evaluate the results. 

The simulation displays good acceptance with real grocery market behavior, i.e. captures the performance of British retailers thus can be used to determine the impact of changes in their behavior on their competitors and consumers.Furthermore it can be used to provide insight into sustainability of volatile pricing strategies, providing a useful insight in volatility of British supermarket retail industry. 
\end{abstract}
\acknowledgements{
I would like to express my sincere gratitude to Dr Maria Polukarov for her guidance and support which provided me the freedom to take this research in the direction of my interest.\\
\\
I would also like to thank my family and friends for their encouragement and support. To those who quietly listened to my software complaints. To those who worked throughout the nights with me. To those who helped me write what I couldn't say. I cannot thank you enough.
}

\declaration{
I, Stefan Collier, declare that this dissertation and the work presented in it are my own and has been generated by me as the result of my own original research.\\
I confirm that:\\
1. This work was done wholly or mainly while in candidature for a degree at this University;\\
2. Where any part of this dissertation has previously been submitted for any other qualification at this University or any other institution, this has been clearly stated;\\
3. Where I have consulted the published work of others, this is always clearly attributed;\\
4. Where I have quoted from the work of others, the source is always given. With the exception of such quotations, this dissertation is entirely my own work;\\
5. I have acknowledged all main sources of help;\\
6. Where the thesis is based on work done by myself jointly with others, I have made clear exactly what was done by others and what I have contributed myself;\\
7. Either none of this work has been published before submission, or parts of this work have been published by :\\
\\
Stefan Collier\\
April 2016
}
\tableofcontents
\listoffigures
\listoftables

\mainmatter
%% ----------------------------------------------------------------
%\include{Introduction}
%\include{Conclusions}
\include{chapters/1Project/main}
\include{chapters/2Lit/main}
\include{chapters/3Design/HighLevel}
\include{chapters/3Design/InDepth}
\include{chapters/4Impl/main}

\include{chapters/5Experiments/1/main}
\include{chapters/5Experiments/2/main}
\include{chapters/5Experiments/3/main}
\include{chapters/5Experiments/4/main}

\include{chapters/6Conclusion/main}

\appendix
\include{appendix/AppendixB}
\include{appendix/D/main}
\include{appendix/AppendixC}

\backmatter
\bibliographystyle{ecs}
\bibliography{ECS}
\end{document}
%% ----------------------------------------------------------------


 %% ----------------------------------------------------------------
%% Progress.tex
%% ---------------------------------------------------------------- 
\documentclass{ecsprogress}    % Use the progress Style
\graphicspath{{../figs/}}   % Location of your graphics files
    \usepackage{natbib}            % Use Natbib style for the refs.
\hypersetup{colorlinks=true}   % Set to false for black/white printing
\input{Definitions}            % Include your abbreviations



\usepackage{enumitem}% http://ctan.org/pkg/enumitem
\usepackage{multirow}
\usepackage{float}
\usepackage{amsmath}
\usepackage{multicol}
\usepackage{amssymb}
\usepackage[normalem]{ulem}
\useunder{\uline}{\ul}{}
\usepackage{wrapfig}


\usepackage[table,xcdraw]{xcolor}


%% ----------------------------------------------------------------
\begin{document}
\frontmatter
\title      {Heterogeneous Agent-based Model for Supermarket Competition}
\authors    {\texorpdfstring
             {\href{mailto:sc22g13@ecs.soton.ac.uk}{Stefan J. Collier}}
             {Stefan J. Collier}
            }
\addresses  {\groupname\\\deptname\\\univname}
\date       {\today}
\subject    {}
\keywords   {}
\supervisor {Dr. Maria Polukarov}
\examiner   {Professor Sheng Chen}

\maketitle
\begin{abstract}
This project aim was to model and analyse the effects of competitive pricing behaviors of grocery retailers on the British market. 

This was achieved by creating a multi-agent model, containing retailer and consumer agents. The heterogeneous crowd of retailers employs either a uniform pricing strategy or a ‘local price flexing’ strategy. The actions of these retailers are chosen by predicting the profit of each action, using a perceptron. Following on from the consideration of different economic models, a discrete model was developed so that software agents have a discrete environment to operate within. Within the model, it has been observed how supermarkets with differing behaviors affect a heterogeneous crowd of consumer agents. The model was implemented in Java with Python used to evaluate the results. 

The simulation displays good acceptance with real grocery market behavior, i.e. captures the performance of British retailers thus can be used to determine the impact of changes in their behavior on their competitors and consumers.Furthermore it can be used to provide insight into sustainability of volatile pricing strategies, providing a useful insight in volatility of British supermarket retail industry. 
\end{abstract}
\acknowledgements{
I would like to express my sincere gratitude to Dr Maria Polukarov for her guidance and support which provided me the freedom to take this research in the direction of my interest.\\
\\
I would also like to thank my family and friends for their encouragement and support. To those who quietly listened to my software complaints. To those who worked throughout the nights with me. To those who helped me write what I couldn't say. I cannot thank you enough.
}

\declaration{
I, Stefan Collier, declare that this dissertation and the work presented in it are my own and has been generated by me as the result of my own original research.\\
I confirm that:\\
1. This work was done wholly or mainly while in candidature for a degree at this University;\\
2. Where any part of this dissertation has previously been submitted for any other qualification at this University or any other institution, this has been clearly stated;\\
3. Where I have consulted the published work of others, this is always clearly attributed;\\
4. Where I have quoted from the work of others, the source is always given. With the exception of such quotations, this dissertation is entirely my own work;\\
5. I have acknowledged all main sources of help;\\
6. Where the thesis is based on work done by myself jointly with others, I have made clear exactly what was done by others and what I have contributed myself;\\
7. Either none of this work has been published before submission, or parts of this work have been published by :\\
\\
Stefan Collier\\
April 2016
}
\tableofcontents
\listoffigures
\listoftables

\mainmatter
%% ----------------------------------------------------------------
%\include{Introduction}
%\include{Conclusions}
\include{chapters/1Project/main}
\include{chapters/2Lit/main}
\include{chapters/3Design/HighLevel}
\include{chapters/3Design/InDepth}
\include{chapters/4Impl/main}

\include{chapters/5Experiments/1/main}
\include{chapters/5Experiments/2/main}
\include{chapters/5Experiments/3/main}
\include{chapters/5Experiments/4/main}

\include{chapters/6Conclusion/main}

\appendix
\include{appendix/AppendixB}
\include{appendix/D/main}
\include{appendix/AppendixC}

\backmatter
\bibliographystyle{ecs}
\bibliography{ECS}
\end{document}
%% ----------------------------------------------------------------

 %% ----------------------------------------------------------------
%% Progress.tex
%% ---------------------------------------------------------------- 
\documentclass{ecsprogress}    % Use the progress Style
\graphicspath{{../figs/}}   % Location of your graphics files
    \usepackage{natbib}            % Use Natbib style for the refs.
\hypersetup{colorlinks=true}   % Set to false for black/white printing
\input{Definitions}            % Include your abbreviations



\usepackage{enumitem}% http://ctan.org/pkg/enumitem
\usepackage{multirow}
\usepackage{float}
\usepackage{amsmath}
\usepackage{multicol}
\usepackage{amssymb}
\usepackage[normalem]{ulem}
\useunder{\uline}{\ul}{}
\usepackage{wrapfig}


\usepackage[table,xcdraw]{xcolor}


%% ----------------------------------------------------------------
\begin{document}
\frontmatter
\title      {Heterogeneous Agent-based Model for Supermarket Competition}
\authors    {\texorpdfstring
             {\href{mailto:sc22g13@ecs.soton.ac.uk}{Stefan J. Collier}}
             {Stefan J. Collier}
            }
\addresses  {\groupname\\\deptname\\\univname}
\date       {\today}
\subject    {}
\keywords   {}
\supervisor {Dr. Maria Polukarov}
\examiner   {Professor Sheng Chen}

\maketitle
\begin{abstract}
This project aim was to model and analyse the effects of competitive pricing behaviors of grocery retailers on the British market. 

This was achieved by creating a multi-agent model, containing retailer and consumer agents. The heterogeneous crowd of retailers employs either a uniform pricing strategy or a ‘local price flexing’ strategy. The actions of these retailers are chosen by predicting the profit of each action, using a perceptron. Following on from the consideration of different economic models, a discrete model was developed so that software agents have a discrete environment to operate within. Within the model, it has been observed how supermarkets with differing behaviors affect a heterogeneous crowd of consumer agents. The model was implemented in Java with Python used to evaluate the results. 

The simulation displays good acceptance with real grocery market behavior, i.e. captures the performance of British retailers thus can be used to determine the impact of changes in their behavior on their competitors and consumers.Furthermore it can be used to provide insight into sustainability of volatile pricing strategies, providing a useful insight in volatility of British supermarket retail industry. 
\end{abstract}
\acknowledgements{
I would like to express my sincere gratitude to Dr Maria Polukarov for her guidance and support which provided me the freedom to take this research in the direction of my interest.\\
\\
I would also like to thank my family and friends for their encouragement and support. To those who quietly listened to my software complaints. To those who worked throughout the nights with me. To those who helped me write what I couldn't say. I cannot thank you enough.
}

\declaration{
I, Stefan Collier, declare that this dissertation and the work presented in it are my own and has been generated by me as the result of my own original research.\\
I confirm that:\\
1. This work was done wholly or mainly while in candidature for a degree at this University;\\
2. Where any part of this dissertation has previously been submitted for any other qualification at this University or any other institution, this has been clearly stated;\\
3. Where I have consulted the published work of others, this is always clearly attributed;\\
4. Where I have quoted from the work of others, the source is always given. With the exception of such quotations, this dissertation is entirely my own work;\\
5. I have acknowledged all main sources of help;\\
6. Where the thesis is based on work done by myself jointly with others, I have made clear exactly what was done by others and what I have contributed myself;\\
7. Either none of this work has been published before submission, or parts of this work have been published by :\\
\\
Stefan Collier\\
April 2016
}
\tableofcontents
\listoffigures
\listoftables

\mainmatter
%% ----------------------------------------------------------------
%\include{Introduction}
%\include{Conclusions}
\include{chapters/1Project/main}
\include{chapters/2Lit/main}
\include{chapters/3Design/HighLevel}
\include{chapters/3Design/InDepth}
\include{chapters/4Impl/main}

\include{chapters/5Experiments/1/main}
\include{chapters/5Experiments/2/main}
\include{chapters/5Experiments/3/main}
\include{chapters/5Experiments/4/main}

\include{chapters/6Conclusion/main}

\appendix
\include{appendix/AppendixB}
\include{appendix/D/main}
\include{appendix/AppendixC}

\backmatter
\bibliographystyle{ecs}
\bibliography{ECS}
\end{document}
%% ----------------------------------------------------------------

 %% ----------------------------------------------------------------
%% Progress.tex
%% ---------------------------------------------------------------- 
\documentclass{ecsprogress}    % Use the progress Style
\graphicspath{{../figs/}}   % Location of your graphics files
    \usepackage{natbib}            % Use Natbib style for the refs.
\hypersetup{colorlinks=true}   % Set to false for black/white printing
\input{Definitions}            % Include your abbreviations



\usepackage{enumitem}% http://ctan.org/pkg/enumitem
\usepackage{multirow}
\usepackage{float}
\usepackage{amsmath}
\usepackage{multicol}
\usepackage{amssymb}
\usepackage[normalem]{ulem}
\useunder{\uline}{\ul}{}
\usepackage{wrapfig}


\usepackage[table,xcdraw]{xcolor}


%% ----------------------------------------------------------------
\begin{document}
\frontmatter
\title      {Heterogeneous Agent-based Model for Supermarket Competition}
\authors    {\texorpdfstring
             {\href{mailto:sc22g13@ecs.soton.ac.uk}{Stefan J. Collier}}
             {Stefan J. Collier}
            }
\addresses  {\groupname\\\deptname\\\univname}
\date       {\today}
\subject    {}
\keywords   {}
\supervisor {Dr. Maria Polukarov}
\examiner   {Professor Sheng Chen}

\maketitle
\begin{abstract}
This project aim was to model and analyse the effects of competitive pricing behaviors of grocery retailers on the British market. 

This was achieved by creating a multi-agent model, containing retailer and consumer agents. The heterogeneous crowd of retailers employs either a uniform pricing strategy or a ‘local price flexing’ strategy. The actions of these retailers are chosen by predicting the profit of each action, using a perceptron. Following on from the consideration of different economic models, a discrete model was developed so that software agents have a discrete environment to operate within. Within the model, it has been observed how supermarkets with differing behaviors affect a heterogeneous crowd of consumer agents. The model was implemented in Java with Python used to evaluate the results. 

The simulation displays good acceptance with real grocery market behavior, i.e. captures the performance of British retailers thus can be used to determine the impact of changes in their behavior on their competitors and consumers.Furthermore it can be used to provide insight into sustainability of volatile pricing strategies, providing a useful insight in volatility of British supermarket retail industry. 
\end{abstract}
\acknowledgements{
I would like to express my sincere gratitude to Dr Maria Polukarov for her guidance and support which provided me the freedom to take this research in the direction of my interest.\\
\\
I would also like to thank my family and friends for their encouragement and support. To those who quietly listened to my software complaints. To those who worked throughout the nights with me. To those who helped me write what I couldn't say. I cannot thank you enough.
}

\declaration{
I, Stefan Collier, declare that this dissertation and the work presented in it are my own and has been generated by me as the result of my own original research.\\
I confirm that:\\
1. This work was done wholly or mainly while in candidature for a degree at this University;\\
2. Where any part of this dissertation has previously been submitted for any other qualification at this University or any other institution, this has been clearly stated;\\
3. Where I have consulted the published work of others, this is always clearly attributed;\\
4. Where I have quoted from the work of others, the source is always given. With the exception of such quotations, this dissertation is entirely my own work;\\
5. I have acknowledged all main sources of help;\\
6. Where the thesis is based on work done by myself jointly with others, I have made clear exactly what was done by others and what I have contributed myself;\\
7. Either none of this work has been published before submission, or parts of this work have been published by :\\
\\
Stefan Collier\\
April 2016
}
\tableofcontents
\listoffigures
\listoftables

\mainmatter
%% ----------------------------------------------------------------
%\include{Introduction}
%\include{Conclusions}
\include{chapters/1Project/main}
\include{chapters/2Lit/main}
\include{chapters/3Design/HighLevel}
\include{chapters/3Design/InDepth}
\include{chapters/4Impl/main}

\include{chapters/5Experiments/1/main}
\include{chapters/5Experiments/2/main}
\include{chapters/5Experiments/3/main}
\include{chapters/5Experiments/4/main}

\include{chapters/6Conclusion/main}

\appendix
\include{appendix/AppendixB}
\include{appendix/D/main}
\include{appendix/AppendixC}

\backmatter
\bibliographystyle{ecs}
\bibliography{ECS}
\end{document}
%% ----------------------------------------------------------------

 %% ----------------------------------------------------------------
%% Progress.tex
%% ---------------------------------------------------------------- 
\documentclass{ecsprogress}    % Use the progress Style
\graphicspath{{../figs/}}   % Location of your graphics files
    \usepackage{natbib}            % Use Natbib style for the refs.
\hypersetup{colorlinks=true}   % Set to false for black/white printing
\input{Definitions}            % Include your abbreviations



\usepackage{enumitem}% http://ctan.org/pkg/enumitem
\usepackage{multirow}
\usepackage{float}
\usepackage{amsmath}
\usepackage{multicol}
\usepackage{amssymb}
\usepackage[normalem]{ulem}
\useunder{\uline}{\ul}{}
\usepackage{wrapfig}


\usepackage[table,xcdraw]{xcolor}


%% ----------------------------------------------------------------
\begin{document}
\frontmatter
\title      {Heterogeneous Agent-based Model for Supermarket Competition}
\authors    {\texorpdfstring
             {\href{mailto:sc22g13@ecs.soton.ac.uk}{Stefan J. Collier}}
             {Stefan J. Collier}
            }
\addresses  {\groupname\\\deptname\\\univname}
\date       {\today}
\subject    {}
\keywords   {}
\supervisor {Dr. Maria Polukarov}
\examiner   {Professor Sheng Chen}

\maketitle
\begin{abstract}
This project aim was to model and analyse the effects of competitive pricing behaviors of grocery retailers on the British market. 

This was achieved by creating a multi-agent model, containing retailer and consumer agents. The heterogeneous crowd of retailers employs either a uniform pricing strategy or a ‘local price flexing’ strategy. The actions of these retailers are chosen by predicting the profit of each action, using a perceptron. Following on from the consideration of different economic models, a discrete model was developed so that software agents have a discrete environment to operate within. Within the model, it has been observed how supermarkets with differing behaviors affect a heterogeneous crowd of consumer agents. The model was implemented in Java with Python used to evaluate the results. 

The simulation displays good acceptance with real grocery market behavior, i.e. captures the performance of British retailers thus can be used to determine the impact of changes in their behavior on their competitors and consumers.Furthermore it can be used to provide insight into sustainability of volatile pricing strategies, providing a useful insight in volatility of British supermarket retail industry. 
\end{abstract}
\acknowledgements{
I would like to express my sincere gratitude to Dr Maria Polukarov for her guidance and support which provided me the freedom to take this research in the direction of my interest.\\
\\
I would also like to thank my family and friends for their encouragement and support. To those who quietly listened to my software complaints. To those who worked throughout the nights with me. To those who helped me write what I couldn't say. I cannot thank you enough.
}

\declaration{
I, Stefan Collier, declare that this dissertation and the work presented in it are my own and has been generated by me as the result of my own original research.\\
I confirm that:\\
1. This work was done wholly or mainly while in candidature for a degree at this University;\\
2. Where any part of this dissertation has previously been submitted for any other qualification at this University or any other institution, this has been clearly stated;\\
3. Where I have consulted the published work of others, this is always clearly attributed;\\
4. Where I have quoted from the work of others, the source is always given. With the exception of such quotations, this dissertation is entirely my own work;\\
5. I have acknowledged all main sources of help;\\
6. Where the thesis is based on work done by myself jointly with others, I have made clear exactly what was done by others and what I have contributed myself;\\
7. Either none of this work has been published before submission, or parts of this work have been published by :\\
\\
Stefan Collier\\
April 2016
}
\tableofcontents
\listoffigures
\listoftables

\mainmatter
%% ----------------------------------------------------------------
%\include{Introduction}
%\include{Conclusions}
\include{chapters/1Project/main}
\include{chapters/2Lit/main}
\include{chapters/3Design/HighLevel}
\include{chapters/3Design/InDepth}
\include{chapters/4Impl/main}

\include{chapters/5Experiments/1/main}
\include{chapters/5Experiments/2/main}
\include{chapters/5Experiments/3/main}
\include{chapters/5Experiments/4/main}

\include{chapters/6Conclusion/main}

\appendix
\include{appendix/AppendixB}
\include{appendix/D/main}
\include{appendix/AppendixC}

\backmatter
\bibliographystyle{ecs}
\bibliography{ECS}
\end{document}
%% ----------------------------------------------------------------


 %% ----------------------------------------------------------------
%% Progress.tex
%% ---------------------------------------------------------------- 
\documentclass{ecsprogress}    % Use the progress Style
\graphicspath{{../figs/}}   % Location of your graphics files
    \usepackage{natbib}            % Use Natbib style for the refs.
\hypersetup{colorlinks=true}   % Set to false for black/white printing
\input{Definitions}            % Include your abbreviations



\usepackage{enumitem}% http://ctan.org/pkg/enumitem
\usepackage{multirow}
\usepackage{float}
\usepackage{amsmath}
\usepackage{multicol}
\usepackage{amssymb}
\usepackage[normalem]{ulem}
\useunder{\uline}{\ul}{}
\usepackage{wrapfig}


\usepackage[table,xcdraw]{xcolor}


%% ----------------------------------------------------------------
\begin{document}
\frontmatter
\title      {Heterogeneous Agent-based Model for Supermarket Competition}
\authors    {\texorpdfstring
             {\href{mailto:sc22g13@ecs.soton.ac.uk}{Stefan J. Collier}}
             {Stefan J. Collier}
            }
\addresses  {\groupname\\\deptname\\\univname}
\date       {\today}
\subject    {}
\keywords   {}
\supervisor {Dr. Maria Polukarov}
\examiner   {Professor Sheng Chen}

\maketitle
\begin{abstract}
This project aim was to model and analyse the effects of competitive pricing behaviors of grocery retailers on the British market. 

This was achieved by creating a multi-agent model, containing retailer and consumer agents. The heterogeneous crowd of retailers employs either a uniform pricing strategy or a ‘local price flexing’ strategy. The actions of these retailers are chosen by predicting the profit of each action, using a perceptron. Following on from the consideration of different economic models, a discrete model was developed so that software agents have a discrete environment to operate within. Within the model, it has been observed how supermarkets with differing behaviors affect a heterogeneous crowd of consumer agents. The model was implemented in Java with Python used to evaluate the results. 

The simulation displays good acceptance with real grocery market behavior, i.e. captures the performance of British retailers thus can be used to determine the impact of changes in their behavior on their competitors and consumers.Furthermore it can be used to provide insight into sustainability of volatile pricing strategies, providing a useful insight in volatility of British supermarket retail industry. 
\end{abstract}
\acknowledgements{
I would like to express my sincere gratitude to Dr Maria Polukarov for her guidance and support which provided me the freedom to take this research in the direction of my interest.\\
\\
I would also like to thank my family and friends for their encouragement and support. To those who quietly listened to my software complaints. To those who worked throughout the nights with me. To those who helped me write what I couldn't say. I cannot thank you enough.
}

\declaration{
I, Stefan Collier, declare that this dissertation and the work presented in it are my own and has been generated by me as the result of my own original research.\\
I confirm that:\\
1. This work was done wholly or mainly while in candidature for a degree at this University;\\
2. Where any part of this dissertation has previously been submitted for any other qualification at this University or any other institution, this has been clearly stated;\\
3. Where I have consulted the published work of others, this is always clearly attributed;\\
4. Where I have quoted from the work of others, the source is always given. With the exception of such quotations, this dissertation is entirely my own work;\\
5. I have acknowledged all main sources of help;\\
6. Where the thesis is based on work done by myself jointly with others, I have made clear exactly what was done by others and what I have contributed myself;\\
7. Either none of this work has been published before submission, or parts of this work have been published by :\\
\\
Stefan Collier\\
April 2016
}
\tableofcontents
\listoffigures
\listoftables

\mainmatter
%% ----------------------------------------------------------------
%\include{Introduction}
%\include{Conclusions}
\include{chapters/1Project/main}
\include{chapters/2Lit/main}
\include{chapters/3Design/HighLevel}
\include{chapters/3Design/InDepth}
\include{chapters/4Impl/main}

\include{chapters/5Experiments/1/main}
\include{chapters/5Experiments/2/main}
\include{chapters/5Experiments/3/main}
\include{chapters/5Experiments/4/main}

\include{chapters/6Conclusion/main}

\appendix
\include{appendix/AppendixB}
\include{appendix/D/main}
\include{appendix/AppendixC}

\backmatter
\bibliographystyle{ecs}
\bibliography{ECS}
\end{document}
%% ----------------------------------------------------------------


\appendix
\include{appendix/AppendixB}
 %% ----------------------------------------------------------------
%% Progress.tex
%% ---------------------------------------------------------------- 
\documentclass{ecsprogress}    % Use the progress Style
\graphicspath{{../figs/}}   % Location of your graphics files
    \usepackage{natbib}            % Use Natbib style for the refs.
\hypersetup{colorlinks=true}   % Set to false for black/white printing
\input{Definitions}            % Include your abbreviations



\usepackage{enumitem}% http://ctan.org/pkg/enumitem
\usepackage{multirow}
\usepackage{float}
\usepackage{amsmath}
\usepackage{multicol}
\usepackage{amssymb}
\usepackage[normalem]{ulem}
\useunder{\uline}{\ul}{}
\usepackage{wrapfig}


\usepackage[table,xcdraw]{xcolor}


%% ----------------------------------------------------------------
\begin{document}
\frontmatter
\title      {Heterogeneous Agent-based Model for Supermarket Competition}
\authors    {\texorpdfstring
             {\href{mailto:sc22g13@ecs.soton.ac.uk}{Stefan J. Collier}}
             {Stefan J. Collier}
            }
\addresses  {\groupname\\\deptname\\\univname}
\date       {\today}
\subject    {}
\keywords   {}
\supervisor {Dr. Maria Polukarov}
\examiner   {Professor Sheng Chen}

\maketitle
\begin{abstract}
This project aim was to model and analyse the effects of competitive pricing behaviors of grocery retailers on the British market. 

This was achieved by creating a multi-agent model, containing retailer and consumer agents. The heterogeneous crowd of retailers employs either a uniform pricing strategy or a ‘local price flexing’ strategy. The actions of these retailers are chosen by predicting the profit of each action, using a perceptron. Following on from the consideration of different economic models, a discrete model was developed so that software agents have a discrete environment to operate within. Within the model, it has been observed how supermarkets with differing behaviors affect a heterogeneous crowd of consumer agents. The model was implemented in Java with Python used to evaluate the results. 

The simulation displays good acceptance with real grocery market behavior, i.e. captures the performance of British retailers thus can be used to determine the impact of changes in their behavior on their competitors and consumers.Furthermore it can be used to provide insight into sustainability of volatile pricing strategies, providing a useful insight in volatility of British supermarket retail industry. 
\end{abstract}
\acknowledgements{
I would like to express my sincere gratitude to Dr Maria Polukarov for her guidance and support which provided me the freedom to take this research in the direction of my interest.\\
\\
I would also like to thank my family and friends for their encouragement and support. To those who quietly listened to my software complaints. To those who worked throughout the nights with me. To those who helped me write what I couldn't say. I cannot thank you enough.
}

\declaration{
I, Stefan Collier, declare that this dissertation and the work presented in it are my own and has been generated by me as the result of my own original research.\\
I confirm that:\\
1. This work was done wholly or mainly while in candidature for a degree at this University;\\
2. Where any part of this dissertation has previously been submitted for any other qualification at this University or any other institution, this has been clearly stated;\\
3. Where I have consulted the published work of others, this is always clearly attributed;\\
4. Where I have quoted from the work of others, the source is always given. With the exception of such quotations, this dissertation is entirely my own work;\\
5. I have acknowledged all main sources of help;\\
6. Where the thesis is based on work done by myself jointly with others, I have made clear exactly what was done by others and what I have contributed myself;\\
7. Either none of this work has been published before submission, or parts of this work have been published by :\\
\\
Stefan Collier\\
April 2016
}
\tableofcontents
\listoffigures
\listoftables

\mainmatter
%% ----------------------------------------------------------------
%\include{Introduction}
%\include{Conclusions}
\include{chapters/1Project/main}
\include{chapters/2Lit/main}
\include{chapters/3Design/HighLevel}
\include{chapters/3Design/InDepth}
\include{chapters/4Impl/main}

\include{chapters/5Experiments/1/main}
\include{chapters/5Experiments/2/main}
\include{chapters/5Experiments/3/main}
\include{chapters/5Experiments/4/main}

\include{chapters/6Conclusion/main}

\appendix
\include{appendix/AppendixB}
\include{appendix/D/main}
\include{appendix/AppendixC}

\backmatter
\bibliographystyle{ecs}
\bibliography{ECS}
\end{document}
%% ----------------------------------------------------------------

\include{appendix/AppendixC}

\backmatter
\bibliographystyle{ecs}
\bibliography{ECS}
\end{document}
%% ----------------------------------------------------------------


 %% ----------------------------------------------------------------
%% Progress.tex
%% ---------------------------------------------------------------- 
\documentclass{ecsprogress}    % Use the progress Style
\graphicspath{{../figs/}}   % Location of your graphics files
    \usepackage{natbib}            % Use Natbib style for the refs.
\hypersetup{colorlinks=true}   % Set to false for black/white printing
\input{Definitions}            % Include your abbreviations



\usepackage{enumitem}% http://ctan.org/pkg/enumitem
\usepackage{multirow}
\usepackage{float}
\usepackage{amsmath}
\usepackage{multicol}
\usepackage{amssymb}
\usepackage[normalem]{ulem}
\useunder{\uline}{\ul}{}
\usepackage{wrapfig}


\usepackage[table,xcdraw]{xcolor}


%% ----------------------------------------------------------------
\begin{document}
\frontmatter
\title      {Heterogeneous Agent-based Model for Supermarket Competition}
\authors    {\texorpdfstring
             {\href{mailto:sc22g13@ecs.soton.ac.uk}{Stefan J. Collier}}
             {Stefan J. Collier}
            }
\addresses  {\groupname\\\deptname\\\univname}
\date       {\today}
\subject    {}
\keywords   {}
\supervisor {Dr. Maria Polukarov}
\examiner   {Professor Sheng Chen}

\maketitle
\begin{abstract}
This project aim was to model and analyse the effects of competitive pricing behaviors of grocery retailers on the British market. 

This was achieved by creating a multi-agent model, containing retailer and consumer agents. The heterogeneous crowd of retailers employs either a uniform pricing strategy or a ‘local price flexing’ strategy. The actions of these retailers are chosen by predicting the profit of each action, using a perceptron. Following on from the consideration of different economic models, a discrete model was developed so that software agents have a discrete environment to operate within. Within the model, it has been observed how supermarkets with differing behaviors affect a heterogeneous crowd of consumer agents. The model was implemented in Java with Python used to evaluate the results. 

The simulation displays good acceptance with real grocery market behavior, i.e. captures the performance of British retailers thus can be used to determine the impact of changes in their behavior on their competitors and consumers.Furthermore it can be used to provide insight into sustainability of volatile pricing strategies, providing a useful insight in volatility of British supermarket retail industry. 
\end{abstract}
\acknowledgements{
I would like to express my sincere gratitude to Dr Maria Polukarov for her guidance and support which provided me the freedom to take this research in the direction of my interest.\\
\\
I would also like to thank my family and friends for their encouragement and support. To those who quietly listened to my software complaints. To those who worked throughout the nights with me. To those who helped me write what I couldn't say. I cannot thank you enough.
}

\declaration{
I, Stefan Collier, declare that this dissertation and the work presented in it are my own and has been generated by me as the result of my own original research.\\
I confirm that:\\
1. This work was done wholly or mainly while in candidature for a degree at this University;\\
2. Where any part of this dissertation has previously been submitted for any other qualification at this University or any other institution, this has been clearly stated;\\
3. Where I have consulted the published work of others, this is always clearly attributed;\\
4. Where I have quoted from the work of others, the source is always given. With the exception of such quotations, this dissertation is entirely my own work;\\
5. I have acknowledged all main sources of help;\\
6. Where the thesis is based on work done by myself jointly with others, I have made clear exactly what was done by others and what I have contributed myself;\\
7. Either none of this work has been published before submission, or parts of this work have been published by :\\
\\
Stefan Collier\\
April 2016
}
\tableofcontents
\listoffigures
\listoftables

\mainmatter
%% ----------------------------------------------------------------
%\include{Introduction}
%\include{Conclusions}
 %% ----------------------------------------------------------------
%% Progress.tex
%% ---------------------------------------------------------------- 
\documentclass{ecsprogress}    % Use the progress Style
\graphicspath{{../figs/}}   % Location of your graphics files
    \usepackage{natbib}            % Use Natbib style for the refs.
\hypersetup{colorlinks=true}   % Set to false for black/white printing
\input{Definitions}            % Include your abbreviations



\usepackage{enumitem}% http://ctan.org/pkg/enumitem
\usepackage{multirow}
\usepackage{float}
\usepackage{amsmath}
\usepackage{multicol}
\usepackage{amssymb}
\usepackage[normalem]{ulem}
\useunder{\uline}{\ul}{}
\usepackage{wrapfig}


\usepackage[table,xcdraw]{xcolor}


%% ----------------------------------------------------------------
\begin{document}
\frontmatter
\title      {Heterogeneous Agent-based Model for Supermarket Competition}
\authors    {\texorpdfstring
             {\href{mailto:sc22g13@ecs.soton.ac.uk}{Stefan J. Collier}}
             {Stefan J. Collier}
            }
\addresses  {\groupname\\\deptname\\\univname}
\date       {\today}
\subject    {}
\keywords   {}
\supervisor {Dr. Maria Polukarov}
\examiner   {Professor Sheng Chen}

\maketitle
\begin{abstract}
This project aim was to model and analyse the effects of competitive pricing behaviors of grocery retailers on the British market. 

This was achieved by creating a multi-agent model, containing retailer and consumer agents. The heterogeneous crowd of retailers employs either a uniform pricing strategy or a ‘local price flexing’ strategy. The actions of these retailers are chosen by predicting the profit of each action, using a perceptron. Following on from the consideration of different economic models, a discrete model was developed so that software agents have a discrete environment to operate within. Within the model, it has been observed how supermarkets with differing behaviors affect a heterogeneous crowd of consumer agents. The model was implemented in Java with Python used to evaluate the results. 

The simulation displays good acceptance with real grocery market behavior, i.e. captures the performance of British retailers thus can be used to determine the impact of changes in their behavior on their competitors and consumers.Furthermore it can be used to provide insight into sustainability of volatile pricing strategies, providing a useful insight in volatility of British supermarket retail industry. 
\end{abstract}
\acknowledgements{
I would like to express my sincere gratitude to Dr Maria Polukarov for her guidance and support which provided me the freedom to take this research in the direction of my interest.\\
\\
I would also like to thank my family and friends for their encouragement and support. To those who quietly listened to my software complaints. To those who worked throughout the nights with me. To those who helped me write what I couldn't say. I cannot thank you enough.
}

\declaration{
I, Stefan Collier, declare that this dissertation and the work presented in it are my own and has been generated by me as the result of my own original research.\\
I confirm that:\\
1. This work was done wholly or mainly while in candidature for a degree at this University;\\
2. Where any part of this dissertation has previously been submitted for any other qualification at this University or any other institution, this has been clearly stated;\\
3. Where I have consulted the published work of others, this is always clearly attributed;\\
4. Where I have quoted from the work of others, the source is always given. With the exception of such quotations, this dissertation is entirely my own work;\\
5. I have acknowledged all main sources of help;\\
6. Where the thesis is based on work done by myself jointly with others, I have made clear exactly what was done by others and what I have contributed myself;\\
7. Either none of this work has been published before submission, or parts of this work have been published by :\\
\\
Stefan Collier\\
April 2016
}
\tableofcontents
\listoffigures
\listoftables

\mainmatter
%% ----------------------------------------------------------------
%\include{Introduction}
%\include{Conclusions}
\include{chapters/1Project/main}
\include{chapters/2Lit/main}
\include{chapters/3Design/HighLevel}
\include{chapters/3Design/InDepth}
\include{chapters/4Impl/main}

\include{chapters/5Experiments/1/main}
\include{chapters/5Experiments/2/main}
\include{chapters/5Experiments/3/main}
\include{chapters/5Experiments/4/main}

\include{chapters/6Conclusion/main}

\appendix
\include{appendix/AppendixB}
\include{appendix/D/main}
\include{appendix/AppendixC}

\backmatter
\bibliographystyle{ecs}
\bibliography{ECS}
\end{document}
%% ----------------------------------------------------------------

 %% ----------------------------------------------------------------
%% Progress.tex
%% ---------------------------------------------------------------- 
\documentclass{ecsprogress}    % Use the progress Style
\graphicspath{{../figs/}}   % Location of your graphics files
    \usepackage{natbib}            % Use Natbib style for the refs.
\hypersetup{colorlinks=true}   % Set to false for black/white printing
\input{Definitions}            % Include your abbreviations



\usepackage{enumitem}% http://ctan.org/pkg/enumitem
\usepackage{multirow}
\usepackage{float}
\usepackage{amsmath}
\usepackage{multicol}
\usepackage{amssymb}
\usepackage[normalem]{ulem}
\useunder{\uline}{\ul}{}
\usepackage{wrapfig}


\usepackage[table,xcdraw]{xcolor}


%% ----------------------------------------------------------------
\begin{document}
\frontmatter
\title      {Heterogeneous Agent-based Model for Supermarket Competition}
\authors    {\texorpdfstring
             {\href{mailto:sc22g13@ecs.soton.ac.uk}{Stefan J. Collier}}
             {Stefan J. Collier}
            }
\addresses  {\groupname\\\deptname\\\univname}
\date       {\today}
\subject    {}
\keywords   {}
\supervisor {Dr. Maria Polukarov}
\examiner   {Professor Sheng Chen}

\maketitle
\begin{abstract}
This project aim was to model and analyse the effects of competitive pricing behaviors of grocery retailers on the British market. 

This was achieved by creating a multi-agent model, containing retailer and consumer agents. The heterogeneous crowd of retailers employs either a uniform pricing strategy or a ‘local price flexing’ strategy. The actions of these retailers are chosen by predicting the profit of each action, using a perceptron. Following on from the consideration of different economic models, a discrete model was developed so that software agents have a discrete environment to operate within. Within the model, it has been observed how supermarkets with differing behaviors affect a heterogeneous crowd of consumer agents. The model was implemented in Java with Python used to evaluate the results. 

The simulation displays good acceptance with real grocery market behavior, i.e. captures the performance of British retailers thus can be used to determine the impact of changes in their behavior on their competitors and consumers.Furthermore it can be used to provide insight into sustainability of volatile pricing strategies, providing a useful insight in volatility of British supermarket retail industry. 
\end{abstract}
\acknowledgements{
I would like to express my sincere gratitude to Dr Maria Polukarov for her guidance and support which provided me the freedom to take this research in the direction of my interest.\\
\\
I would also like to thank my family and friends for their encouragement and support. To those who quietly listened to my software complaints. To those who worked throughout the nights with me. To those who helped me write what I couldn't say. I cannot thank you enough.
}

\declaration{
I, Stefan Collier, declare that this dissertation and the work presented in it are my own and has been generated by me as the result of my own original research.\\
I confirm that:\\
1. This work was done wholly or mainly while in candidature for a degree at this University;\\
2. Where any part of this dissertation has previously been submitted for any other qualification at this University or any other institution, this has been clearly stated;\\
3. Where I have consulted the published work of others, this is always clearly attributed;\\
4. Where I have quoted from the work of others, the source is always given. With the exception of such quotations, this dissertation is entirely my own work;\\
5. I have acknowledged all main sources of help;\\
6. Where the thesis is based on work done by myself jointly with others, I have made clear exactly what was done by others and what I have contributed myself;\\
7. Either none of this work has been published before submission, or parts of this work have been published by :\\
\\
Stefan Collier\\
April 2016
}
\tableofcontents
\listoffigures
\listoftables

\mainmatter
%% ----------------------------------------------------------------
%\include{Introduction}
%\include{Conclusions}
\include{chapters/1Project/main}
\include{chapters/2Lit/main}
\include{chapters/3Design/HighLevel}
\include{chapters/3Design/InDepth}
\include{chapters/4Impl/main}

\include{chapters/5Experiments/1/main}
\include{chapters/5Experiments/2/main}
\include{chapters/5Experiments/3/main}
\include{chapters/5Experiments/4/main}

\include{chapters/6Conclusion/main}

\appendix
\include{appendix/AppendixB}
\include{appendix/D/main}
\include{appendix/AppendixC}

\backmatter
\bibliographystyle{ecs}
\bibliography{ECS}
\end{document}
%% ----------------------------------------------------------------

\include{chapters/3Design/HighLevel}
\include{chapters/3Design/InDepth}
 %% ----------------------------------------------------------------
%% Progress.tex
%% ---------------------------------------------------------------- 
\documentclass{ecsprogress}    % Use the progress Style
\graphicspath{{../figs/}}   % Location of your graphics files
    \usepackage{natbib}            % Use Natbib style for the refs.
\hypersetup{colorlinks=true}   % Set to false for black/white printing
\input{Definitions}            % Include your abbreviations



\usepackage{enumitem}% http://ctan.org/pkg/enumitem
\usepackage{multirow}
\usepackage{float}
\usepackage{amsmath}
\usepackage{multicol}
\usepackage{amssymb}
\usepackage[normalem]{ulem}
\useunder{\uline}{\ul}{}
\usepackage{wrapfig}


\usepackage[table,xcdraw]{xcolor}


%% ----------------------------------------------------------------
\begin{document}
\frontmatter
\title      {Heterogeneous Agent-based Model for Supermarket Competition}
\authors    {\texorpdfstring
             {\href{mailto:sc22g13@ecs.soton.ac.uk}{Stefan J. Collier}}
             {Stefan J. Collier}
            }
\addresses  {\groupname\\\deptname\\\univname}
\date       {\today}
\subject    {}
\keywords   {}
\supervisor {Dr. Maria Polukarov}
\examiner   {Professor Sheng Chen}

\maketitle
\begin{abstract}
This project aim was to model and analyse the effects of competitive pricing behaviors of grocery retailers on the British market. 

This was achieved by creating a multi-agent model, containing retailer and consumer agents. The heterogeneous crowd of retailers employs either a uniform pricing strategy or a ‘local price flexing’ strategy. The actions of these retailers are chosen by predicting the profit of each action, using a perceptron. Following on from the consideration of different economic models, a discrete model was developed so that software agents have a discrete environment to operate within. Within the model, it has been observed how supermarkets with differing behaviors affect a heterogeneous crowd of consumer agents. The model was implemented in Java with Python used to evaluate the results. 

The simulation displays good acceptance with real grocery market behavior, i.e. captures the performance of British retailers thus can be used to determine the impact of changes in their behavior on their competitors and consumers.Furthermore it can be used to provide insight into sustainability of volatile pricing strategies, providing a useful insight in volatility of British supermarket retail industry. 
\end{abstract}
\acknowledgements{
I would like to express my sincere gratitude to Dr Maria Polukarov for her guidance and support which provided me the freedom to take this research in the direction of my interest.\\
\\
I would also like to thank my family and friends for their encouragement and support. To those who quietly listened to my software complaints. To those who worked throughout the nights with me. To those who helped me write what I couldn't say. I cannot thank you enough.
}

\declaration{
I, Stefan Collier, declare that this dissertation and the work presented in it are my own and has been generated by me as the result of my own original research.\\
I confirm that:\\
1. This work was done wholly or mainly while in candidature for a degree at this University;\\
2. Where any part of this dissertation has previously been submitted for any other qualification at this University or any other institution, this has been clearly stated;\\
3. Where I have consulted the published work of others, this is always clearly attributed;\\
4. Where I have quoted from the work of others, the source is always given. With the exception of such quotations, this dissertation is entirely my own work;\\
5. I have acknowledged all main sources of help;\\
6. Where the thesis is based on work done by myself jointly with others, I have made clear exactly what was done by others and what I have contributed myself;\\
7. Either none of this work has been published before submission, or parts of this work have been published by :\\
\\
Stefan Collier\\
April 2016
}
\tableofcontents
\listoffigures
\listoftables

\mainmatter
%% ----------------------------------------------------------------
%\include{Introduction}
%\include{Conclusions}
\include{chapters/1Project/main}
\include{chapters/2Lit/main}
\include{chapters/3Design/HighLevel}
\include{chapters/3Design/InDepth}
\include{chapters/4Impl/main}

\include{chapters/5Experiments/1/main}
\include{chapters/5Experiments/2/main}
\include{chapters/5Experiments/3/main}
\include{chapters/5Experiments/4/main}

\include{chapters/6Conclusion/main}

\appendix
\include{appendix/AppendixB}
\include{appendix/D/main}
\include{appendix/AppendixC}

\backmatter
\bibliographystyle{ecs}
\bibliography{ECS}
\end{document}
%% ----------------------------------------------------------------


 %% ----------------------------------------------------------------
%% Progress.tex
%% ---------------------------------------------------------------- 
\documentclass{ecsprogress}    % Use the progress Style
\graphicspath{{../figs/}}   % Location of your graphics files
    \usepackage{natbib}            % Use Natbib style for the refs.
\hypersetup{colorlinks=true}   % Set to false for black/white printing
\input{Definitions}            % Include your abbreviations



\usepackage{enumitem}% http://ctan.org/pkg/enumitem
\usepackage{multirow}
\usepackage{float}
\usepackage{amsmath}
\usepackage{multicol}
\usepackage{amssymb}
\usepackage[normalem]{ulem}
\useunder{\uline}{\ul}{}
\usepackage{wrapfig}


\usepackage[table,xcdraw]{xcolor}


%% ----------------------------------------------------------------
\begin{document}
\frontmatter
\title      {Heterogeneous Agent-based Model for Supermarket Competition}
\authors    {\texorpdfstring
             {\href{mailto:sc22g13@ecs.soton.ac.uk}{Stefan J. Collier}}
             {Stefan J. Collier}
            }
\addresses  {\groupname\\\deptname\\\univname}
\date       {\today}
\subject    {}
\keywords   {}
\supervisor {Dr. Maria Polukarov}
\examiner   {Professor Sheng Chen}

\maketitle
\begin{abstract}
This project aim was to model and analyse the effects of competitive pricing behaviors of grocery retailers on the British market. 

This was achieved by creating a multi-agent model, containing retailer and consumer agents. The heterogeneous crowd of retailers employs either a uniform pricing strategy or a ‘local price flexing’ strategy. The actions of these retailers are chosen by predicting the profit of each action, using a perceptron. Following on from the consideration of different economic models, a discrete model was developed so that software agents have a discrete environment to operate within. Within the model, it has been observed how supermarkets with differing behaviors affect a heterogeneous crowd of consumer agents. The model was implemented in Java with Python used to evaluate the results. 

The simulation displays good acceptance with real grocery market behavior, i.e. captures the performance of British retailers thus can be used to determine the impact of changes in their behavior on their competitors and consumers.Furthermore it can be used to provide insight into sustainability of volatile pricing strategies, providing a useful insight in volatility of British supermarket retail industry. 
\end{abstract}
\acknowledgements{
I would like to express my sincere gratitude to Dr Maria Polukarov for her guidance and support which provided me the freedom to take this research in the direction of my interest.\\
\\
I would also like to thank my family and friends for their encouragement and support. To those who quietly listened to my software complaints. To those who worked throughout the nights with me. To those who helped me write what I couldn't say. I cannot thank you enough.
}

\declaration{
I, Stefan Collier, declare that this dissertation and the work presented in it are my own and has been generated by me as the result of my own original research.\\
I confirm that:\\
1. This work was done wholly or mainly while in candidature for a degree at this University;\\
2. Where any part of this dissertation has previously been submitted for any other qualification at this University or any other institution, this has been clearly stated;\\
3. Where I have consulted the published work of others, this is always clearly attributed;\\
4. Where I have quoted from the work of others, the source is always given. With the exception of such quotations, this dissertation is entirely my own work;\\
5. I have acknowledged all main sources of help;\\
6. Where the thesis is based on work done by myself jointly with others, I have made clear exactly what was done by others and what I have contributed myself;\\
7. Either none of this work has been published before submission, or parts of this work have been published by :\\
\\
Stefan Collier\\
April 2016
}
\tableofcontents
\listoffigures
\listoftables

\mainmatter
%% ----------------------------------------------------------------
%\include{Introduction}
%\include{Conclusions}
\include{chapters/1Project/main}
\include{chapters/2Lit/main}
\include{chapters/3Design/HighLevel}
\include{chapters/3Design/InDepth}
\include{chapters/4Impl/main}

\include{chapters/5Experiments/1/main}
\include{chapters/5Experiments/2/main}
\include{chapters/5Experiments/3/main}
\include{chapters/5Experiments/4/main}

\include{chapters/6Conclusion/main}

\appendix
\include{appendix/AppendixB}
\include{appendix/D/main}
\include{appendix/AppendixC}

\backmatter
\bibliographystyle{ecs}
\bibliography{ECS}
\end{document}
%% ----------------------------------------------------------------

 %% ----------------------------------------------------------------
%% Progress.tex
%% ---------------------------------------------------------------- 
\documentclass{ecsprogress}    % Use the progress Style
\graphicspath{{../figs/}}   % Location of your graphics files
    \usepackage{natbib}            % Use Natbib style for the refs.
\hypersetup{colorlinks=true}   % Set to false for black/white printing
\input{Definitions}            % Include your abbreviations



\usepackage{enumitem}% http://ctan.org/pkg/enumitem
\usepackage{multirow}
\usepackage{float}
\usepackage{amsmath}
\usepackage{multicol}
\usepackage{amssymb}
\usepackage[normalem]{ulem}
\useunder{\uline}{\ul}{}
\usepackage{wrapfig}


\usepackage[table,xcdraw]{xcolor}


%% ----------------------------------------------------------------
\begin{document}
\frontmatter
\title      {Heterogeneous Agent-based Model for Supermarket Competition}
\authors    {\texorpdfstring
             {\href{mailto:sc22g13@ecs.soton.ac.uk}{Stefan J. Collier}}
             {Stefan J. Collier}
            }
\addresses  {\groupname\\\deptname\\\univname}
\date       {\today}
\subject    {}
\keywords   {}
\supervisor {Dr. Maria Polukarov}
\examiner   {Professor Sheng Chen}

\maketitle
\begin{abstract}
This project aim was to model and analyse the effects of competitive pricing behaviors of grocery retailers on the British market. 

This was achieved by creating a multi-agent model, containing retailer and consumer agents. The heterogeneous crowd of retailers employs either a uniform pricing strategy or a ‘local price flexing’ strategy. The actions of these retailers are chosen by predicting the profit of each action, using a perceptron. Following on from the consideration of different economic models, a discrete model was developed so that software agents have a discrete environment to operate within. Within the model, it has been observed how supermarkets with differing behaviors affect a heterogeneous crowd of consumer agents. The model was implemented in Java with Python used to evaluate the results. 

The simulation displays good acceptance with real grocery market behavior, i.e. captures the performance of British retailers thus can be used to determine the impact of changes in their behavior on their competitors and consumers.Furthermore it can be used to provide insight into sustainability of volatile pricing strategies, providing a useful insight in volatility of British supermarket retail industry. 
\end{abstract}
\acknowledgements{
I would like to express my sincere gratitude to Dr Maria Polukarov for her guidance and support which provided me the freedom to take this research in the direction of my interest.\\
\\
I would also like to thank my family and friends for their encouragement and support. To those who quietly listened to my software complaints. To those who worked throughout the nights with me. To those who helped me write what I couldn't say. I cannot thank you enough.
}

\declaration{
I, Stefan Collier, declare that this dissertation and the work presented in it are my own and has been generated by me as the result of my own original research.\\
I confirm that:\\
1. This work was done wholly or mainly while in candidature for a degree at this University;\\
2. Where any part of this dissertation has previously been submitted for any other qualification at this University or any other institution, this has been clearly stated;\\
3. Where I have consulted the published work of others, this is always clearly attributed;\\
4. Where I have quoted from the work of others, the source is always given. With the exception of such quotations, this dissertation is entirely my own work;\\
5. I have acknowledged all main sources of help;\\
6. Where the thesis is based on work done by myself jointly with others, I have made clear exactly what was done by others and what I have contributed myself;\\
7. Either none of this work has been published before submission, or parts of this work have been published by :\\
\\
Stefan Collier\\
April 2016
}
\tableofcontents
\listoffigures
\listoftables

\mainmatter
%% ----------------------------------------------------------------
%\include{Introduction}
%\include{Conclusions}
\include{chapters/1Project/main}
\include{chapters/2Lit/main}
\include{chapters/3Design/HighLevel}
\include{chapters/3Design/InDepth}
\include{chapters/4Impl/main}

\include{chapters/5Experiments/1/main}
\include{chapters/5Experiments/2/main}
\include{chapters/5Experiments/3/main}
\include{chapters/5Experiments/4/main}

\include{chapters/6Conclusion/main}

\appendix
\include{appendix/AppendixB}
\include{appendix/D/main}
\include{appendix/AppendixC}

\backmatter
\bibliographystyle{ecs}
\bibliography{ECS}
\end{document}
%% ----------------------------------------------------------------

 %% ----------------------------------------------------------------
%% Progress.tex
%% ---------------------------------------------------------------- 
\documentclass{ecsprogress}    % Use the progress Style
\graphicspath{{../figs/}}   % Location of your graphics files
    \usepackage{natbib}            % Use Natbib style for the refs.
\hypersetup{colorlinks=true}   % Set to false for black/white printing
\input{Definitions}            % Include your abbreviations



\usepackage{enumitem}% http://ctan.org/pkg/enumitem
\usepackage{multirow}
\usepackage{float}
\usepackage{amsmath}
\usepackage{multicol}
\usepackage{amssymb}
\usepackage[normalem]{ulem}
\useunder{\uline}{\ul}{}
\usepackage{wrapfig}


\usepackage[table,xcdraw]{xcolor}


%% ----------------------------------------------------------------
\begin{document}
\frontmatter
\title      {Heterogeneous Agent-based Model for Supermarket Competition}
\authors    {\texorpdfstring
             {\href{mailto:sc22g13@ecs.soton.ac.uk}{Stefan J. Collier}}
             {Stefan J. Collier}
            }
\addresses  {\groupname\\\deptname\\\univname}
\date       {\today}
\subject    {}
\keywords   {}
\supervisor {Dr. Maria Polukarov}
\examiner   {Professor Sheng Chen}

\maketitle
\begin{abstract}
This project aim was to model and analyse the effects of competitive pricing behaviors of grocery retailers on the British market. 

This was achieved by creating a multi-agent model, containing retailer and consumer agents. The heterogeneous crowd of retailers employs either a uniform pricing strategy or a ‘local price flexing’ strategy. The actions of these retailers are chosen by predicting the profit of each action, using a perceptron. Following on from the consideration of different economic models, a discrete model was developed so that software agents have a discrete environment to operate within. Within the model, it has been observed how supermarkets with differing behaviors affect a heterogeneous crowd of consumer agents. The model was implemented in Java with Python used to evaluate the results. 

The simulation displays good acceptance with real grocery market behavior, i.e. captures the performance of British retailers thus can be used to determine the impact of changes in their behavior on their competitors and consumers.Furthermore it can be used to provide insight into sustainability of volatile pricing strategies, providing a useful insight in volatility of British supermarket retail industry. 
\end{abstract}
\acknowledgements{
I would like to express my sincere gratitude to Dr Maria Polukarov for her guidance and support which provided me the freedom to take this research in the direction of my interest.\\
\\
I would also like to thank my family and friends for their encouragement and support. To those who quietly listened to my software complaints. To those who worked throughout the nights with me. To those who helped me write what I couldn't say. I cannot thank you enough.
}

\declaration{
I, Stefan Collier, declare that this dissertation and the work presented in it are my own and has been generated by me as the result of my own original research.\\
I confirm that:\\
1. This work was done wholly or mainly while in candidature for a degree at this University;\\
2. Where any part of this dissertation has previously been submitted for any other qualification at this University or any other institution, this has been clearly stated;\\
3. Where I have consulted the published work of others, this is always clearly attributed;\\
4. Where I have quoted from the work of others, the source is always given. With the exception of such quotations, this dissertation is entirely my own work;\\
5. I have acknowledged all main sources of help;\\
6. Where the thesis is based on work done by myself jointly with others, I have made clear exactly what was done by others and what I have contributed myself;\\
7. Either none of this work has been published before submission, or parts of this work have been published by :\\
\\
Stefan Collier\\
April 2016
}
\tableofcontents
\listoffigures
\listoftables

\mainmatter
%% ----------------------------------------------------------------
%\include{Introduction}
%\include{Conclusions}
\include{chapters/1Project/main}
\include{chapters/2Lit/main}
\include{chapters/3Design/HighLevel}
\include{chapters/3Design/InDepth}
\include{chapters/4Impl/main}

\include{chapters/5Experiments/1/main}
\include{chapters/5Experiments/2/main}
\include{chapters/5Experiments/3/main}
\include{chapters/5Experiments/4/main}

\include{chapters/6Conclusion/main}

\appendix
\include{appendix/AppendixB}
\include{appendix/D/main}
\include{appendix/AppendixC}

\backmatter
\bibliographystyle{ecs}
\bibliography{ECS}
\end{document}
%% ----------------------------------------------------------------

 %% ----------------------------------------------------------------
%% Progress.tex
%% ---------------------------------------------------------------- 
\documentclass{ecsprogress}    % Use the progress Style
\graphicspath{{../figs/}}   % Location of your graphics files
    \usepackage{natbib}            % Use Natbib style for the refs.
\hypersetup{colorlinks=true}   % Set to false for black/white printing
\input{Definitions}            % Include your abbreviations



\usepackage{enumitem}% http://ctan.org/pkg/enumitem
\usepackage{multirow}
\usepackage{float}
\usepackage{amsmath}
\usepackage{multicol}
\usepackage{amssymb}
\usepackage[normalem]{ulem}
\useunder{\uline}{\ul}{}
\usepackage{wrapfig}


\usepackage[table,xcdraw]{xcolor}


%% ----------------------------------------------------------------
\begin{document}
\frontmatter
\title      {Heterogeneous Agent-based Model for Supermarket Competition}
\authors    {\texorpdfstring
             {\href{mailto:sc22g13@ecs.soton.ac.uk}{Stefan J. Collier}}
             {Stefan J. Collier}
            }
\addresses  {\groupname\\\deptname\\\univname}
\date       {\today}
\subject    {}
\keywords   {}
\supervisor {Dr. Maria Polukarov}
\examiner   {Professor Sheng Chen}

\maketitle
\begin{abstract}
This project aim was to model and analyse the effects of competitive pricing behaviors of grocery retailers on the British market. 

This was achieved by creating a multi-agent model, containing retailer and consumer agents. The heterogeneous crowd of retailers employs either a uniform pricing strategy or a ‘local price flexing’ strategy. The actions of these retailers are chosen by predicting the profit of each action, using a perceptron. Following on from the consideration of different economic models, a discrete model was developed so that software agents have a discrete environment to operate within. Within the model, it has been observed how supermarkets with differing behaviors affect a heterogeneous crowd of consumer agents. The model was implemented in Java with Python used to evaluate the results. 

The simulation displays good acceptance with real grocery market behavior, i.e. captures the performance of British retailers thus can be used to determine the impact of changes in their behavior on their competitors and consumers.Furthermore it can be used to provide insight into sustainability of volatile pricing strategies, providing a useful insight in volatility of British supermarket retail industry. 
\end{abstract}
\acknowledgements{
I would like to express my sincere gratitude to Dr Maria Polukarov for her guidance and support which provided me the freedom to take this research in the direction of my interest.\\
\\
I would also like to thank my family and friends for their encouragement and support. To those who quietly listened to my software complaints. To those who worked throughout the nights with me. To those who helped me write what I couldn't say. I cannot thank you enough.
}

\declaration{
I, Stefan Collier, declare that this dissertation and the work presented in it are my own and has been generated by me as the result of my own original research.\\
I confirm that:\\
1. This work was done wholly or mainly while in candidature for a degree at this University;\\
2. Where any part of this dissertation has previously been submitted for any other qualification at this University or any other institution, this has been clearly stated;\\
3. Where I have consulted the published work of others, this is always clearly attributed;\\
4. Where I have quoted from the work of others, the source is always given. With the exception of such quotations, this dissertation is entirely my own work;\\
5. I have acknowledged all main sources of help;\\
6. Where the thesis is based on work done by myself jointly with others, I have made clear exactly what was done by others and what I have contributed myself;\\
7. Either none of this work has been published before submission, or parts of this work have been published by :\\
\\
Stefan Collier\\
April 2016
}
\tableofcontents
\listoffigures
\listoftables

\mainmatter
%% ----------------------------------------------------------------
%\include{Introduction}
%\include{Conclusions}
\include{chapters/1Project/main}
\include{chapters/2Lit/main}
\include{chapters/3Design/HighLevel}
\include{chapters/3Design/InDepth}
\include{chapters/4Impl/main}

\include{chapters/5Experiments/1/main}
\include{chapters/5Experiments/2/main}
\include{chapters/5Experiments/3/main}
\include{chapters/5Experiments/4/main}

\include{chapters/6Conclusion/main}

\appendix
\include{appendix/AppendixB}
\include{appendix/D/main}
\include{appendix/AppendixC}

\backmatter
\bibliographystyle{ecs}
\bibliography{ECS}
\end{document}
%% ----------------------------------------------------------------


 %% ----------------------------------------------------------------
%% Progress.tex
%% ---------------------------------------------------------------- 
\documentclass{ecsprogress}    % Use the progress Style
\graphicspath{{../figs/}}   % Location of your graphics files
    \usepackage{natbib}            % Use Natbib style for the refs.
\hypersetup{colorlinks=true}   % Set to false for black/white printing
\input{Definitions}            % Include your abbreviations



\usepackage{enumitem}% http://ctan.org/pkg/enumitem
\usepackage{multirow}
\usepackage{float}
\usepackage{amsmath}
\usepackage{multicol}
\usepackage{amssymb}
\usepackage[normalem]{ulem}
\useunder{\uline}{\ul}{}
\usepackage{wrapfig}


\usepackage[table,xcdraw]{xcolor}


%% ----------------------------------------------------------------
\begin{document}
\frontmatter
\title      {Heterogeneous Agent-based Model for Supermarket Competition}
\authors    {\texorpdfstring
             {\href{mailto:sc22g13@ecs.soton.ac.uk}{Stefan J. Collier}}
             {Stefan J. Collier}
            }
\addresses  {\groupname\\\deptname\\\univname}
\date       {\today}
\subject    {}
\keywords   {}
\supervisor {Dr. Maria Polukarov}
\examiner   {Professor Sheng Chen}

\maketitle
\begin{abstract}
This project aim was to model and analyse the effects of competitive pricing behaviors of grocery retailers on the British market. 

This was achieved by creating a multi-agent model, containing retailer and consumer agents. The heterogeneous crowd of retailers employs either a uniform pricing strategy or a ‘local price flexing’ strategy. The actions of these retailers are chosen by predicting the profit of each action, using a perceptron. Following on from the consideration of different economic models, a discrete model was developed so that software agents have a discrete environment to operate within. Within the model, it has been observed how supermarkets with differing behaviors affect a heterogeneous crowd of consumer agents. The model was implemented in Java with Python used to evaluate the results. 

The simulation displays good acceptance with real grocery market behavior, i.e. captures the performance of British retailers thus can be used to determine the impact of changes in their behavior on their competitors and consumers.Furthermore it can be used to provide insight into sustainability of volatile pricing strategies, providing a useful insight in volatility of British supermarket retail industry. 
\end{abstract}
\acknowledgements{
I would like to express my sincere gratitude to Dr Maria Polukarov for her guidance and support which provided me the freedom to take this research in the direction of my interest.\\
\\
I would also like to thank my family and friends for their encouragement and support. To those who quietly listened to my software complaints. To those who worked throughout the nights with me. To those who helped me write what I couldn't say. I cannot thank you enough.
}

\declaration{
I, Stefan Collier, declare that this dissertation and the work presented in it are my own and has been generated by me as the result of my own original research.\\
I confirm that:\\
1. This work was done wholly or mainly while in candidature for a degree at this University;\\
2. Where any part of this dissertation has previously been submitted for any other qualification at this University or any other institution, this has been clearly stated;\\
3. Where I have consulted the published work of others, this is always clearly attributed;\\
4. Where I have quoted from the work of others, the source is always given. With the exception of such quotations, this dissertation is entirely my own work;\\
5. I have acknowledged all main sources of help;\\
6. Where the thesis is based on work done by myself jointly with others, I have made clear exactly what was done by others and what I have contributed myself;\\
7. Either none of this work has been published before submission, or parts of this work have been published by :\\
\\
Stefan Collier\\
April 2016
}
\tableofcontents
\listoffigures
\listoftables

\mainmatter
%% ----------------------------------------------------------------
%\include{Introduction}
%\include{Conclusions}
\include{chapters/1Project/main}
\include{chapters/2Lit/main}
\include{chapters/3Design/HighLevel}
\include{chapters/3Design/InDepth}
\include{chapters/4Impl/main}

\include{chapters/5Experiments/1/main}
\include{chapters/5Experiments/2/main}
\include{chapters/5Experiments/3/main}
\include{chapters/5Experiments/4/main}

\include{chapters/6Conclusion/main}

\appendix
\include{appendix/AppendixB}
\include{appendix/D/main}
\include{appendix/AppendixC}

\backmatter
\bibliographystyle{ecs}
\bibliography{ECS}
\end{document}
%% ----------------------------------------------------------------


\appendix
\include{appendix/AppendixB}
 %% ----------------------------------------------------------------
%% Progress.tex
%% ---------------------------------------------------------------- 
\documentclass{ecsprogress}    % Use the progress Style
\graphicspath{{../figs/}}   % Location of your graphics files
    \usepackage{natbib}            % Use Natbib style for the refs.
\hypersetup{colorlinks=true}   % Set to false for black/white printing
\input{Definitions}            % Include your abbreviations



\usepackage{enumitem}% http://ctan.org/pkg/enumitem
\usepackage{multirow}
\usepackage{float}
\usepackage{amsmath}
\usepackage{multicol}
\usepackage{amssymb}
\usepackage[normalem]{ulem}
\useunder{\uline}{\ul}{}
\usepackage{wrapfig}


\usepackage[table,xcdraw]{xcolor}


%% ----------------------------------------------------------------
\begin{document}
\frontmatter
\title      {Heterogeneous Agent-based Model for Supermarket Competition}
\authors    {\texorpdfstring
             {\href{mailto:sc22g13@ecs.soton.ac.uk}{Stefan J. Collier}}
             {Stefan J. Collier}
            }
\addresses  {\groupname\\\deptname\\\univname}
\date       {\today}
\subject    {}
\keywords   {}
\supervisor {Dr. Maria Polukarov}
\examiner   {Professor Sheng Chen}

\maketitle
\begin{abstract}
This project aim was to model and analyse the effects of competitive pricing behaviors of grocery retailers on the British market. 

This was achieved by creating a multi-agent model, containing retailer and consumer agents. The heterogeneous crowd of retailers employs either a uniform pricing strategy or a ‘local price flexing’ strategy. The actions of these retailers are chosen by predicting the profit of each action, using a perceptron. Following on from the consideration of different economic models, a discrete model was developed so that software agents have a discrete environment to operate within. Within the model, it has been observed how supermarkets with differing behaviors affect a heterogeneous crowd of consumer agents. The model was implemented in Java with Python used to evaluate the results. 

The simulation displays good acceptance with real grocery market behavior, i.e. captures the performance of British retailers thus can be used to determine the impact of changes in their behavior on their competitors and consumers.Furthermore it can be used to provide insight into sustainability of volatile pricing strategies, providing a useful insight in volatility of British supermarket retail industry. 
\end{abstract}
\acknowledgements{
I would like to express my sincere gratitude to Dr Maria Polukarov for her guidance and support which provided me the freedom to take this research in the direction of my interest.\\
\\
I would also like to thank my family and friends for their encouragement and support. To those who quietly listened to my software complaints. To those who worked throughout the nights with me. To those who helped me write what I couldn't say. I cannot thank you enough.
}

\declaration{
I, Stefan Collier, declare that this dissertation and the work presented in it are my own and has been generated by me as the result of my own original research.\\
I confirm that:\\
1. This work was done wholly or mainly while in candidature for a degree at this University;\\
2. Where any part of this dissertation has previously been submitted for any other qualification at this University or any other institution, this has been clearly stated;\\
3. Where I have consulted the published work of others, this is always clearly attributed;\\
4. Where I have quoted from the work of others, the source is always given. With the exception of such quotations, this dissertation is entirely my own work;\\
5. I have acknowledged all main sources of help;\\
6. Where the thesis is based on work done by myself jointly with others, I have made clear exactly what was done by others and what I have contributed myself;\\
7. Either none of this work has been published before submission, or parts of this work have been published by :\\
\\
Stefan Collier\\
April 2016
}
\tableofcontents
\listoffigures
\listoftables

\mainmatter
%% ----------------------------------------------------------------
%\include{Introduction}
%\include{Conclusions}
\include{chapters/1Project/main}
\include{chapters/2Lit/main}
\include{chapters/3Design/HighLevel}
\include{chapters/3Design/InDepth}
\include{chapters/4Impl/main}

\include{chapters/5Experiments/1/main}
\include{chapters/5Experiments/2/main}
\include{chapters/5Experiments/3/main}
\include{chapters/5Experiments/4/main}

\include{chapters/6Conclusion/main}

\appendix
\include{appendix/AppendixB}
\include{appendix/D/main}
\include{appendix/AppendixC}

\backmatter
\bibliographystyle{ecs}
\bibliography{ECS}
\end{document}
%% ----------------------------------------------------------------

\include{appendix/AppendixC}

\backmatter
\bibliographystyle{ecs}
\bibliography{ECS}
\end{document}
%% ----------------------------------------------------------------


\appendix
\include{appendix/AppendixB}
 %% ----------------------------------------------------------------
%% Progress.tex
%% ---------------------------------------------------------------- 
\documentclass{ecsprogress}    % Use the progress Style
\graphicspath{{../figs/}}   % Location of your graphics files
    \usepackage{natbib}            % Use Natbib style for the refs.
\hypersetup{colorlinks=true}   % Set to false for black/white printing
\input{Definitions}            % Include your abbreviations



\usepackage{enumitem}% http://ctan.org/pkg/enumitem
\usepackage{multirow}
\usepackage{float}
\usepackage{amsmath}
\usepackage{multicol}
\usepackage{amssymb}
\usepackage[normalem]{ulem}
\useunder{\uline}{\ul}{}
\usepackage{wrapfig}


\usepackage[table,xcdraw]{xcolor}


%% ----------------------------------------------------------------
\begin{document}
\frontmatter
\title      {Heterogeneous Agent-based Model for Supermarket Competition}
\authors    {\texorpdfstring
             {\href{mailto:sc22g13@ecs.soton.ac.uk}{Stefan J. Collier}}
             {Stefan J. Collier}
            }
\addresses  {\groupname\\\deptname\\\univname}
\date       {\today}
\subject    {}
\keywords   {}
\supervisor {Dr. Maria Polukarov}
\examiner   {Professor Sheng Chen}

\maketitle
\begin{abstract}
This project aim was to model and analyse the effects of competitive pricing behaviors of grocery retailers on the British market. 

This was achieved by creating a multi-agent model, containing retailer and consumer agents. The heterogeneous crowd of retailers employs either a uniform pricing strategy or a ‘local price flexing’ strategy. The actions of these retailers are chosen by predicting the profit of each action, using a perceptron. Following on from the consideration of different economic models, a discrete model was developed so that software agents have a discrete environment to operate within. Within the model, it has been observed how supermarkets with differing behaviors affect a heterogeneous crowd of consumer agents. The model was implemented in Java with Python used to evaluate the results. 

The simulation displays good acceptance with real grocery market behavior, i.e. captures the performance of British retailers thus can be used to determine the impact of changes in their behavior on their competitors and consumers.Furthermore it can be used to provide insight into sustainability of volatile pricing strategies, providing a useful insight in volatility of British supermarket retail industry. 
\end{abstract}
\acknowledgements{
I would like to express my sincere gratitude to Dr Maria Polukarov for her guidance and support which provided me the freedom to take this research in the direction of my interest.\\
\\
I would also like to thank my family and friends for their encouragement and support. To those who quietly listened to my software complaints. To those who worked throughout the nights with me. To those who helped me write what I couldn't say. I cannot thank you enough.
}

\declaration{
I, Stefan Collier, declare that this dissertation and the work presented in it are my own and has been generated by me as the result of my own original research.\\
I confirm that:\\
1. This work was done wholly or mainly while in candidature for a degree at this University;\\
2. Where any part of this dissertation has previously been submitted for any other qualification at this University or any other institution, this has been clearly stated;\\
3. Where I have consulted the published work of others, this is always clearly attributed;\\
4. Where I have quoted from the work of others, the source is always given. With the exception of such quotations, this dissertation is entirely my own work;\\
5. I have acknowledged all main sources of help;\\
6. Where the thesis is based on work done by myself jointly with others, I have made clear exactly what was done by others and what I have contributed myself;\\
7. Either none of this work has been published before submission, or parts of this work have been published by :\\
\\
Stefan Collier\\
April 2016
}
\tableofcontents
\listoffigures
\listoftables

\mainmatter
%% ----------------------------------------------------------------
%\include{Introduction}
%\include{Conclusions}
 %% ----------------------------------------------------------------
%% Progress.tex
%% ---------------------------------------------------------------- 
\documentclass{ecsprogress}    % Use the progress Style
\graphicspath{{../figs/}}   % Location of your graphics files
    \usepackage{natbib}            % Use Natbib style for the refs.
\hypersetup{colorlinks=true}   % Set to false for black/white printing
\input{Definitions}            % Include your abbreviations



\usepackage{enumitem}% http://ctan.org/pkg/enumitem
\usepackage{multirow}
\usepackage{float}
\usepackage{amsmath}
\usepackage{multicol}
\usepackage{amssymb}
\usepackage[normalem]{ulem}
\useunder{\uline}{\ul}{}
\usepackage{wrapfig}


\usepackage[table,xcdraw]{xcolor}


%% ----------------------------------------------------------------
\begin{document}
\frontmatter
\title      {Heterogeneous Agent-based Model for Supermarket Competition}
\authors    {\texorpdfstring
             {\href{mailto:sc22g13@ecs.soton.ac.uk}{Stefan J. Collier}}
             {Stefan J. Collier}
            }
\addresses  {\groupname\\\deptname\\\univname}
\date       {\today}
\subject    {}
\keywords   {}
\supervisor {Dr. Maria Polukarov}
\examiner   {Professor Sheng Chen}

\maketitle
\begin{abstract}
This project aim was to model and analyse the effects of competitive pricing behaviors of grocery retailers on the British market. 

This was achieved by creating a multi-agent model, containing retailer and consumer agents. The heterogeneous crowd of retailers employs either a uniform pricing strategy or a ‘local price flexing’ strategy. The actions of these retailers are chosen by predicting the profit of each action, using a perceptron. Following on from the consideration of different economic models, a discrete model was developed so that software agents have a discrete environment to operate within. Within the model, it has been observed how supermarkets with differing behaviors affect a heterogeneous crowd of consumer agents. The model was implemented in Java with Python used to evaluate the results. 

The simulation displays good acceptance with real grocery market behavior, i.e. captures the performance of British retailers thus can be used to determine the impact of changes in their behavior on their competitors and consumers.Furthermore it can be used to provide insight into sustainability of volatile pricing strategies, providing a useful insight in volatility of British supermarket retail industry. 
\end{abstract}
\acknowledgements{
I would like to express my sincere gratitude to Dr Maria Polukarov for her guidance and support which provided me the freedom to take this research in the direction of my interest.\\
\\
I would also like to thank my family and friends for their encouragement and support. To those who quietly listened to my software complaints. To those who worked throughout the nights with me. To those who helped me write what I couldn't say. I cannot thank you enough.
}

\declaration{
I, Stefan Collier, declare that this dissertation and the work presented in it are my own and has been generated by me as the result of my own original research.\\
I confirm that:\\
1. This work was done wholly or mainly while in candidature for a degree at this University;\\
2. Where any part of this dissertation has previously been submitted for any other qualification at this University or any other institution, this has been clearly stated;\\
3. Where I have consulted the published work of others, this is always clearly attributed;\\
4. Where I have quoted from the work of others, the source is always given. With the exception of such quotations, this dissertation is entirely my own work;\\
5. I have acknowledged all main sources of help;\\
6. Where the thesis is based on work done by myself jointly with others, I have made clear exactly what was done by others and what I have contributed myself;\\
7. Either none of this work has been published before submission, or parts of this work have been published by :\\
\\
Stefan Collier\\
April 2016
}
\tableofcontents
\listoffigures
\listoftables

\mainmatter
%% ----------------------------------------------------------------
%\include{Introduction}
%\include{Conclusions}
\include{chapters/1Project/main}
\include{chapters/2Lit/main}
\include{chapters/3Design/HighLevel}
\include{chapters/3Design/InDepth}
\include{chapters/4Impl/main}

\include{chapters/5Experiments/1/main}
\include{chapters/5Experiments/2/main}
\include{chapters/5Experiments/3/main}
\include{chapters/5Experiments/4/main}

\include{chapters/6Conclusion/main}

\appendix
\include{appendix/AppendixB}
\include{appendix/D/main}
\include{appendix/AppendixC}

\backmatter
\bibliographystyle{ecs}
\bibliography{ECS}
\end{document}
%% ----------------------------------------------------------------

 %% ----------------------------------------------------------------
%% Progress.tex
%% ---------------------------------------------------------------- 
\documentclass{ecsprogress}    % Use the progress Style
\graphicspath{{../figs/}}   % Location of your graphics files
    \usepackage{natbib}            % Use Natbib style for the refs.
\hypersetup{colorlinks=true}   % Set to false for black/white printing
\input{Definitions}            % Include your abbreviations



\usepackage{enumitem}% http://ctan.org/pkg/enumitem
\usepackage{multirow}
\usepackage{float}
\usepackage{amsmath}
\usepackage{multicol}
\usepackage{amssymb}
\usepackage[normalem]{ulem}
\useunder{\uline}{\ul}{}
\usepackage{wrapfig}


\usepackage[table,xcdraw]{xcolor}


%% ----------------------------------------------------------------
\begin{document}
\frontmatter
\title      {Heterogeneous Agent-based Model for Supermarket Competition}
\authors    {\texorpdfstring
             {\href{mailto:sc22g13@ecs.soton.ac.uk}{Stefan J. Collier}}
             {Stefan J. Collier}
            }
\addresses  {\groupname\\\deptname\\\univname}
\date       {\today}
\subject    {}
\keywords   {}
\supervisor {Dr. Maria Polukarov}
\examiner   {Professor Sheng Chen}

\maketitle
\begin{abstract}
This project aim was to model and analyse the effects of competitive pricing behaviors of grocery retailers on the British market. 

This was achieved by creating a multi-agent model, containing retailer and consumer agents. The heterogeneous crowd of retailers employs either a uniform pricing strategy or a ‘local price flexing’ strategy. The actions of these retailers are chosen by predicting the profit of each action, using a perceptron. Following on from the consideration of different economic models, a discrete model was developed so that software agents have a discrete environment to operate within. Within the model, it has been observed how supermarkets with differing behaviors affect a heterogeneous crowd of consumer agents. The model was implemented in Java with Python used to evaluate the results. 

The simulation displays good acceptance with real grocery market behavior, i.e. captures the performance of British retailers thus can be used to determine the impact of changes in their behavior on their competitors and consumers.Furthermore it can be used to provide insight into sustainability of volatile pricing strategies, providing a useful insight in volatility of British supermarket retail industry. 
\end{abstract}
\acknowledgements{
I would like to express my sincere gratitude to Dr Maria Polukarov for her guidance and support which provided me the freedom to take this research in the direction of my interest.\\
\\
I would also like to thank my family and friends for their encouragement and support. To those who quietly listened to my software complaints. To those who worked throughout the nights with me. To those who helped me write what I couldn't say. I cannot thank you enough.
}

\declaration{
I, Stefan Collier, declare that this dissertation and the work presented in it are my own and has been generated by me as the result of my own original research.\\
I confirm that:\\
1. This work was done wholly or mainly while in candidature for a degree at this University;\\
2. Where any part of this dissertation has previously been submitted for any other qualification at this University or any other institution, this has been clearly stated;\\
3. Where I have consulted the published work of others, this is always clearly attributed;\\
4. Where I have quoted from the work of others, the source is always given. With the exception of such quotations, this dissertation is entirely my own work;\\
5. I have acknowledged all main sources of help;\\
6. Where the thesis is based on work done by myself jointly with others, I have made clear exactly what was done by others and what I have contributed myself;\\
7. Either none of this work has been published before submission, or parts of this work have been published by :\\
\\
Stefan Collier\\
April 2016
}
\tableofcontents
\listoffigures
\listoftables

\mainmatter
%% ----------------------------------------------------------------
%\include{Introduction}
%\include{Conclusions}
\include{chapters/1Project/main}
\include{chapters/2Lit/main}
\include{chapters/3Design/HighLevel}
\include{chapters/3Design/InDepth}
\include{chapters/4Impl/main}

\include{chapters/5Experiments/1/main}
\include{chapters/5Experiments/2/main}
\include{chapters/5Experiments/3/main}
\include{chapters/5Experiments/4/main}

\include{chapters/6Conclusion/main}

\appendix
\include{appendix/AppendixB}
\include{appendix/D/main}
\include{appendix/AppendixC}

\backmatter
\bibliographystyle{ecs}
\bibliography{ECS}
\end{document}
%% ----------------------------------------------------------------

\include{chapters/3Design/HighLevel}
\include{chapters/3Design/InDepth}
 %% ----------------------------------------------------------------
%% Progress.tex
%% ---------------------------------------------------------------- 
\documentclass{ecsprogress}    % Use the progress Style
\graphicspath{{../figs/}}   % Location of your graphics files
    \usepackage{natbib}            % Use Natbib style for the refs.
\hypersetup{colorlinks=true}   % Set to false for black/white printing
\input{Definitions}            % Include your abbreviations



\usepackage{enumitem}% http://ctan.org/pkg/enumitem
\usepackage{multirow}
\usepackage{float}
\usepackage{amsmath}
\usepackage{multicol}
\usepackage{amssymb}
\usepackage[normalem]{ulem}
\useunder{\uline}{\ul}{}
\usepackage{wrapfig}


\usepackage[table,xcdraw]{xcolor}


%% ----------------------------------------------------------------
\begin{document}
\frontmatter
\title      {Heterogeneous Agent-based Model for Supermarket Competition}
\authors    {\texorpdfstring
             {\href{mailto:sc22g13@ecs.soton.ac.uk}{Stefan J. Collier}}
             {Stefan J. Collier}
            }
\addresses  {\groupname\\\deptname\\\univname}
\date       {\today}
\subject    {}
\keywords   {}
\supervisor {Dr. Maria Polukarov}
\examiner   {Professor Sheng Chen}

\maketitle
\begin{abstract}
This project aim was to model and analyse the effects of competitive pricing behaviors of grocery retailers on the British market. 

This was achieved by creating a multi-agent model, containing retailer and consumer agents. The heterogeneous crowd of retailers employs either a uniform pricing strategy or a ‘local price flexing’ strategy. The actions of these retailers are chosen by predicting the profit of each action, using a perceptron. Following on from the consideration of different economic models, a discrete model was developed so that software agents have a discrete environment to operate within. Within the model, it has been observed how supermarkets with differing behaviors affect a heterogeneous crowd of consumer agents. The model was implemented in Java with Python used to evaluate the results. 

The simulation displays good acceptance with real grocery market behavior, i.e. captures the performance of British retailers thus can be used to determine the impact of changes in their behavior on their competitors and consumers.Furthermore it can be used to provide insight into sustainability of volatile pricing strategies, providing a useful insight in volatility of British supermarket retail industry. 
\end{abstract}
\acknowledgements{
I would like to express my sincere gratitude to Dr Maria Polukarov for her guidance and support which provided me the freedom to take this research in the direction of my interest.\\
\\
I would also like to thank my family and friends for their encouragement and support. To those who quietly listened to my software complaints. To those who worked throughout the nights with me. To those who helped me write what I couldn't say. I cannot thank you enough.
}

\declaration{
I, Stefan Collier, declare that this dissertation and the work presented in it are my own and has been generated by me as the result of my own original research.\\
I confirm that:\\
1. This work was done wholly or mainly while in candidature for a degree at this University;\\
2. Where any part of this dissertation has previously been submitted for any other qualification at this University or any other institution, this has been clearly stated;\\
3. Where I have consulted the published work of others, this is always clearly attributed;\\
4. Where I have quoted from the work of others, the source is always given. With the exception of such quotations, this dissertation is entirely my own work;\\
5. I have acknowledged all main sources of help;\\
6. Where the thesis is based on work done by myself jointly with others, I have made clear exactly what was done by others and what I have contributed myself;\\
7. Either none of this work has been published before submission, or parts of this work have been published by :\\
\\
Stefan Collier\\
April 2016
}
\tableofcontents
\listoffigures
\listoftables

\mainmatter
%% ----------------------------------------------------------------
%\include{Introduction}
%\include{Conclusions}
\include{chapters/1Project/main}
\include{chapters/2Lit/main}
\include{chapters/3Design/HighLevel}
\include{chapters/3Design/InDepth}
\include{chapters/4Impl/main}

\include{chapters/5Experiments/1/main}
\include{chapters/5Experiments/2/main}
\include{chapters/5Experiments/3/main}
\include{chapters/5Experiments/4/main}

\include{chapters/6Conclusion/main}

\appendix
\include{appendix/AppendixB}
\include{appendix/D/main}
\include{appendix/AppendixC}

\backmatter
\bibliographystyle{ecs}
\bibliography{ECS}
\end{document}
%% ----------------------------------------------------------------


 %% ----------------------------------------------------------------
%% Progress.tex
%% ---------------------------------------------------------------- 
\documentclass{ecsprogress}    % Use the progress Style
\graphicspath{{../figs/}}   % Location of your graphics files
    \usepackage{natbib}            % Use Natbib style for the refs.
\hypersetup{colorlinks=true}   % Set to false for black/white printing
\input{Definitions}            % Include your abbreviations



\usepackage{enumitem}% http://ctan.org/pkg/enumitem
\usepackage{multirow}
\usepackage{float}
\usepackage{amsmath}
\usepackage{multicol}
\usepackage{amssymb}
\usepackage[normalem]{ulem}
\useunder{\uline}{\ul}{}
\usepackage{wrapfig}


\usepackage[table,xcdraw]{xcolor}


%% ----------------------------------------------------------------
\begin{document}
\frontmatter
\title      {Heterogeneous Agent-based Model for Supermarket Competition}
\authors    {\texorpdfstring
             {\href{mailto:sc22g13@ecs.soton.ac.uk}{Stefan J. Collier}}
             {Stefan J. Collier}
            }
\addresses  {\groupname\\\deptname\\\univname}
\date       {\today}
\subject    {}
\keywords   {}
\supervisor {Dr. Maria Polukarov}
\examiner   {Professor Sheng Chen}

\maketitle
\begin{abstract}
This project aim was to model and analyse the effects of competitive pricing behaviors of grocery retailers on the British market. 

This was achieved by creating a multi-agent model, containing retailer and consumer agents. The heterogeneous crowd of retailers employs either a uniform pricing strategy or a ‘local price flexing’ strategy. The actions of these retailers are chosen by predicting the profit of each action, using a perceptron. Following on from the consideration of different economic models, a discrete model was developed so that software agents have a discrete environment to operate within. Within the model, it has been observed how supermarkets with differing behaviors affect a heterogeneous crowd of consumer agents. The model was implemented in Java with Python used to evaluate the results. 

The simulation displays good acceptance with real grocery market behavior, i.e. captures the performance of British retailers thus can be used to determine the impact of changes in their behavior on their competitors and consumers.Furthermore it can be used to provide insight into sustainability of volatile pricing strategies, providing a useful insight in volatility of British supermarket retail industry. 
\end{abstract}
\acknowledgements{
I would like to express my sincere gratitude to Dr Maria Polukarov for her guidance and support which provided me the freedom to take this research in the direction of my interest.\\
\\
I would also like to thank my family and friends for their encouragement and support. To those who quietly listened to my software complaints. To those who worked throughout the nights with me. To those who helped me write what I couldn't say. I cannot thank you enough.
}

\declaration{
I, Stefan Collier, declare that this dissertation and the work presented in it are my own and has been generated by me as the result of my own original research.\\
I confirm that:\\
1. This work was done wholly or mainly while in candidature for a degree at this University;\\
2. Where any part of this dissertation has previously been submitted for any other qualification at this University or any other institution, this has been clearly stated;\\
3. Where I have consulted the published work of others, this is always clearly attributed;\\
4. Where I have quoted from the work of others, the source is always given. With the exception of such quotations, this dissertation is entirely my own work;\\
5. I have acknowledged all main sources of help;\\
6. Where the thesis is based on work done by myself jointly with others, I have made clear exactly what was done by others and what I have contributed myself;\\
7. Either none of this work has been published before submission, or parts of this work have been published by :\\
\\
Stefan Collier\\
April 2016
}
\tableofcontents
\listoffigures
\listoftables

\mainmatter
%% ----------------------------------------------------------------
%\include{Introduction}
%\include{Conclusions}
\include{chapters/1Project/main}
\include{chapters/2Lit/main}
\include{chapters/3Design/HighLevel}
\include{chapters/3Design/InDepth}
\include{chapters/4Impl/main}

\include{chapters/5Experiments/1/main}
\include{chapters/5Experiments/2/main}
\include{chapters/5Experiments/3/main}
\include{chapters/5Experiments/4/main}

\include{chapters/6Conclusion/main}

\appendix
\include{appendix/AppendixB}
\include{appendix/D/main}
\include{appendix/AppendixC}

\backmatter
\bibliographystyle{ecs}
\bibliography{ECS}
\end{document}
%% ----------------------------------------------------------------

 %% ----------------------------------------------------------------
%% Progress.tex
%% ---------------------------------------------------------------- 
\documentclass{ecsprogress}    % Use the progress Style
\graphicspath{{../figs/}}   % Location of your graphics files
    \usepackage{natbib}            % Use Natbib style for the refs.
\hypersetup{colorlinks=true}   % Set to false for black/white printing
\input{Definitions}            % Include your abbreviations



\usepackage{enumitem}% http://ctan.org/pkg/enumitem
\usepackage{multirow}
\usepackage{float}
\usepackage{amsmath}
\usepackage{multicol}
\usepackage{amssymb}
\usepackage[normalem]{ulem}
\useunder{\uline}{\ul}{}
\usepackage{wrapfig}


\usepackage[table,xcdraw]{xcolor}


%% ----------------------------------------------------------------
\begin{document}
\frontmatter
\title      {Heterogeneous Agent-based Model for Supermarket Competition}
\authors    {\texorpdfstring
             {\href{mailto:sc22g13@ecs.soton.ac.uk}{Stefan J. Collier}}
             {Stefan J. Collier}
            }
\addresses  {\groupname\\\deptname\\\univname}
\date       {\today}
\subject    {}
\keywords   {}
\supervisor {Dr. Maria Polukarov}
\examiner   {Professor Sheng Chen}

\maketitle
\begin{abstract}
This project aim was to model and analyse the effects of competitive pricing behaviors of grocery retailers on the British market. 

This was achieved by creating a multi-agent model, containing retailer and consumer agents. The heterogeneous crowd of retailers employs either a uniform pricing strategy or a ‘local price flexing’ strategy. The actions of these retailers are chosen by predicting the profit of each action, using a perceptron. Following on from the consideration of different economic models, a discrete model was developed so that software agents have a discrete environment to operate within. Within the model, it has been observed how supermarkets with differing behaviors affect a heterogeneous crowd of consumer agents. The model was implemented in Java with Python used to evaluate the results. 

The simulation displays good acceptance with real grocery market behavior, i.e. captures the performance of British retailers thus can be used to determine the impact of changes in their behavior on their competitors and consumers.Furthermore it can be used to provide insight into sustainability of volatile pricing strategies, providing a useful insight in volatility of British supermarket retail industry. 
\end{abstract}
\acknowledgements{
I would like to express my sincere gratitude to Dr Maria Polukarov for her guidance and support which provided me the freedom to take this research in the direction of my interest.\\
\\
I would also like to thank my family and friends for their encouragement and support. To those who quietly listened to my software complaints. To those who worked throughout the nights with me. To those who helped me write what I couldn't say. I cannot thank you enough.
}

\declaration{
I, Stefan Collier, declare that this dissertation and the work presented in it are my own and has been generated by me as the result of my own original research.\\
I confirm that:\\
1. This work was done wholly or mainly while in candidature for a degree at this University;\\
2. Where any part of this dissertation has previously been submitted for any other qualification at this University or any other institution, this has been clearly stated;\\
3. Where I have consulted the published work of others, this is always clearly attributed;\\
4. Where I have quoted from the work of others, the source is always given. With the exception of such quotations, this dissertation is entirely my own work;\\
5. I have acknowledged all main sources of help;\\
6. Where the thesis is based on work done by myself jointly with others, I have made clear exactly what was done by others and what I have contributed myself;\\
7. Either none of this work has been published before submission, or parts of this work have been published by :\\
\\
Stefan Collier\\
April 2016
}
\tableofcontents
\listoffigures
\listoftables

\mainmatter
%% ----------------------------------------------------------------
%\include{Introduction}
%\include{Conclusions}
\include{chapters/1Project/main}
\include{chapters/2Lit/main}
\include{chapters/3Design/HighLevel}
\include{chapters/3Design/InDepth}
\include{chapters/4Impl/main}

\include{chapters/5Experiments/1/main}
\include{chapters/5Experiments/2/main}
\include{chapters/5Experiments/3/main}
\include{chapters/5Experiments/4/main}

\include{chapters/6Conclusion/main}

\appendix
\include{appendix/AppendixB}
\include{appendix/D/main}
\include{appendix/AppendixC}

\backmatter
\bibliographystyle{ecs}
\bibliography{ECS}
\end{document}
%% ----------------------------------------------------------------

 %% ----------------------------------------------------------------
%% Progress.tex
%% ---------------------------------------------------------------- 
\documentclass{ecsprogress}    % Use the progress Style
\graphicspath{{../figs/}}   % Location of your graphics files
    \usepackage{natbib}            % Use Natbib style for the refs.
\hypersetup{colorlinks=true}   % Set to false for black/white printing
\input{Definitions}            % Include your abbreviations



\usepackage{enumitem}% http://ctan.org/pkg/enumitem
\usepackage{multirow}
\usepackage{float}
\usepackage{amsmath}
\usepackage{multicol}
\usepackage{amssymb}
\usepackage[normalem]{ulem}
\useunder{\uline}{\ul}{}
\usepackage{wrapfig}


\usepackage[table,xcdraw]{xcolor}


%% ----------------------------------------------------------------
\begin{document}
\frontmatter
\title      {Heterogeneous Agent-based Model for Supermarket Competition}
\authors    {\texorpdfstring
             {\href{mailto:sc22g13@ecs.soton.ac.uk}{Stefan J. Collier}}
             {Stefan J. Collier}
            }
\addresses  {\groupname\\\deptname\\\univname}
\date       {\today}
\subject    {}
\keywords   {}
\supervisor {Dr. Maria Polukarov}
\examiner   {Professor Sheng Chen}

\maketitle
\begin{abstract}
This project aim was to model and analyse the effects of competitive pricing behaviors of grocery retailers on the British market. 

This was achieved by creating a multi-agent model, containing retailer and consumer agents. The heterogeneous crowd of retailers employs either a uniform pricing strategy or a ‘local price flexing’ strategy. The actions of these retailers are chosen by predicting the profit of each action, using a perceptron. Following on from the consideration of different economic models, a discrete model was developed so that software agents have a discrete environment to operate within. Within the model, it has been observed how supermarkets with differing behaviors affect a heterogeneous crowd of consumer agents. The model was implemented in Java with Python used to evaluate the results. 

The simulation displays good acceptance with real grocery market behavior, i.e. captures the performance of British retailers thus can be used to determine the impact of changes in their behavior on their competitors and consumers.Furthermore it can be used to provide insight into sustainability of volatile pricing strategies, providing a useful insight in volatility of British supermarket retail industry. 
\end{abstract}
\acknowledgements{
I would like to express my sincere gratitude to Dr Maria Polukarov for her guidance and support which provided me the freedom to take this research in the direction of my interest.\\
\\
I would also like to thank my family and friends for their encouragement and support. To those who quietly listened to my software complaints. To those who worked throughout the nights with me. To those who helped me write what I couldn't say. I cannot thank you enough.
}

\declaration{
I, Stefan Collier, declare that this dissertation and the work presented in it are my own and has been generated by me as the result of my own original research.\\
I confirm that:\\
1. This work was done wholly or mainly while in candidature for a degree at this University;\\
2. Where any part of this dissertation has previously been submitted for any other qualification at this University or any other institution, this has been clearly stated;\\
3. Where I have consulted the published work of others, this is always clearly attributed;\\
4. Where I have quoted from the work of others, the source is always given. With the exception of such quotations, this dissertation is entirely my own work;\\
5. I have acknowledged all main sources of help;\\
6. Where the thesis is based on work done by myself jointly with others, I have made clear exactly what was done by others and what I have contributed myself;\\
7. Either none of this work has been published before submission, or parts of this work have been published by :\\
\\
Stefan Collier\\
April 2016
}
\tableofcontents
\listoffigures
\listoftables

\mainmatter
%% ----------------------------------------------------------------
%\include{Introduction}
%\include{Conclusions}
\include{chapters/1Project/main}
\include{chapters/2Lit/main}
\include{chapters/3Design/HighLevel}
\include{chapters/3Design/InDepth}
\include{chapters/4Impl/main}

\include{chapters/5Experiments/1/main}
\include{chapters/5Experiments/2/main}
\include{chapters/5Experiments/3/main}
\include{chapters/5Experiments/4/main}

\include{chapters/6Conclusion/main}

\appendix
\include{appendix/AppendixB}
\include{appendix/D/main}
\include{appendix/AppendixC}

\backmatter
\bibliographystyle{ecs}
\bibliography{ECS}
\end{document}
%% ----------------------------------------------------------------

 %% ----------------------------------------------------------------
%% Progress.tex
%% ---------------------------------------------------------------- 
\documentclass{ecsprogress}    % Use the progress Style
\graphicspath{{../figs/}}   % Location of your graphics files
    \usepackage{natbib}            % Use Natbib style for the refs.
\hypersetup{colorlinks=true}   % Set to false for black/white printing
\input{Definitions}            % Include your abbreviations



\usepackage{enumitem}% http://ctan.org/pkg/enumitem
\usepackage{multirow}
\usepackage{float}
\usepackage{amsmath}
\usepackage{multicol}
\usepackage{amssymb}
\usepackage[normalem]{ulem}
\useunder{\uline}{\ul}{}
\usepackage{wrapfig}


\usepackage[table,xcdraw]{xcolor}


%% ----------------------------------------------------------------
\begin{document}
\frontmatter
\title      {Heterogeneous Agent-based Model for Supermarket Competition}
\authors    {\texorpdfstring
             {\href{mailto:sc22g13@ecs.soton.ac.uk}{Stefan J. Collier}}
             {Stefan J. Collier}
            }
\addresses  {\groupname\\\deptname\\\univname}
\date       {\today}
\subject    {}
\keywords   {}
\supervisor {Dr. Maria Polukarov}
\examiner   {Professor Sheng Chen}

\maketitle
\begin{abstract}
This project aim was to model and analyse the effects of competitive pricing behaviors of grocery retailers on the British market. 

This was achieved by creating a multi-agent model, containing retailer and consumer agents. The heterogeneous crowd of retailers employs either a uniform pricing strategy or a ‘local price flexing’ strategy. The actions of these retailers are chosen by predicting the profit of each action, using a perceptron. Following on from the consideration of different economic models, a discrete model was developed so that software agents have a discrete environment to operate within. Within the model, it has been observed how supermarkets with differing behaviors affect a heterogeneous crowd of consumer agents. The model was implemented in Java with Python used to evaluate the results. 

The simulation displays good acceptance with real grocery market behavior, i.e. captures the performance of British retailers thus can be used to determine the impact of changes in their behavior on their competitors and consumers.Furthermore it can be used to provide insight into sustainability of volatile pricing strategies, providing a useful insight in volatility of British supermarket retail industry. 
\end{abstract}
\acknowledgements{
I would like to express my sincere gratitude to Dr Maria Polukarov for her guidance and support which provided me the freedom to take this research in the direction of my interest.\\
\\
I would also like to thank my family and friends for their encouragement and support. To those who quietly listened to my software complaints. To those who worked throughout the nights with me. To those who helped me write what I couldn't say. I cannot thank you enough.
}

\declaration{
I, Stefan Collier, declare that this dissertation and the work presented in it are my own and has been generated by me as the result of my own original research.\\
I confirm that:\\
1. This work was done wholly or mainly while in candidature for a degree at this University;\\
2. Where any part of this dissertation has previously been submitted for any other qualification at this University or any other institution, this has been clearly stated;\\
3. Where I have consulted the published work of others, this is always clearly attributed;\\
4. Where I have quoted from the work of others, the source is always given. With the exception of such quotations, this dissertation is entirely my own work;\\
5. I have acknowledged all main sources of help;\\
6. Where the thesis is based on work done by myself jointly with others, I have made clear exactly what was done by others and what I have contributed myself;\\
7. Either none of this work has been published before submission, or parts of this work have been published by :\\
\\
Stefan Collier\\
April 2016
}
\tableofcontents
\listoffigures
\listoftables

\mainmatter
%% ----------------------------------------------------------------
%\include{Introduction}
%\include{Conclusions}
\include{chapters/1Project/main}
\include{chapters/2Lit/main}
\include{chapters/3Design/HighLevel}
\include{chapters/3Design/InDepth}
\include{chapters/4Impl/main}

\include{chapters/5Experiments/1/main}
\include{chapters/5Experiments/2/main}
\include{chapters/5Experiments/3/main}
\include{chapters/5Experiments/4/main}

\include{chapters/6Conclusion/main}

\appendix
\include{appendix/AppendixB}
\include{appendix/D/main}
\include{appendix/AppendixC}

\backmatter
\bibliographystyle{ecs}
\bibliography{ECS}
\end{document}
%% ----------------------------------------------------------------


 %% ----------------------------------------------------------------
%% Progress.tex
%% ---------------------------------------------------------------- 
\documentclass{ecsprogress}    % Use the progress Style
\graphicspath{{../figs/}}   % Location of your graphics files
    \usepackage{natbib}            % Use Natbib style for the refs.
\hypersetup{colorlinks=true}   % Set to false for black/white printing
\input{Definitions}            % Include your abbreviations



\usepackage{enumitem}% http://ctan.org/pkg/enumitem
\usepackage{multirow}
\usepackage{float}
\usepackage{amsmath}
\usepackage{multicol}
\usepackage{amssymb}
\usepackage[normalem]{ulem}
\useunder{\uline}{\ul}{}
\usepackage{wrapfig}


\usepackage[table,xcdraw]{xcolor}


%% ----------------------------------------------------------------
\begin{document}
\frontmatter
\title      {Heterogeneous Agent-based Model for Supermarket Competition}
\authors    {\texorpdfstring
             {\href{mailto:sc22g13@ecs.soton.ac.uk}{Stefan J. Collier}}
             {Stefan J. Collier}
            }
\addresses  {\groupname\\\deptname\\\univname}
\date       {\today}
\subject    {}
\keywords   {}
\supervisor {Dr. Maria Polukarov}
\examiner   {Professor Sheng Chen}

\maketitle
\begin{abstract}
This project aim was to model and analyse the effects of competitive pricing behaviors of grocery retailers on the British market. 

This was achieved by creating a multi-agent model, containing retailer and consumer agents. The heterogeneous crowd of retailers employs either a uniform pricing strategy or a ‘local price flexing’ strategy. The actions of these retailers are chosen by predicting the profit of each action, using a perceptron. Following on from the consideration of different economic models, a discrete model was developed so that software agents have a discrete environment to operate within. Within the model, it has been observed how supermarkets with differing behaviors affect a heterogeneous crowd of consumer agents. The model was implemented in Java with Python used to evaluate the results. 

The simulation displays good acceptance with real grocery market behavior, i.e. captures the performance of British retailers thus can be used to determine the impact of changes in their behavior on their competitors and consumers.Furthermore it can be used to provide insight into sustainability of volatile pricing strategies, providing a useful insight in volatility of British supermarket retail industry. 
\end{abstract}
\acknowledgements{
I would like to express my sincere gratitude to Dr Maria Polukarov for her guidance and support which provided me the freedom to take this research in the direction of my interest.\\
\\
I would also like to thank my family and friends for their encouragement and support. To those who quietly listened to my software complaints. To those who worked throughout the nights with me. To those who helped me write what I couldn't say. I cannot thank you enough.
}

\declaration{
I, Stefan Collier, declare that this dissertation and the work presented in it are my own and has been generated by me as the result of my own original research.\\
I confirm that:\\
1. This work was done wholly or mainly while in candidature for a degree at this University;\\
2. Where any part of this dissertation has previously been submitted for any other qualification at this University or any other institution, this has been clearly stated;\\
3. Where I have consulted the published work of others, this is always clearly attributed;\\
4. Where I have quoted from the work of others, the source is always given. With the exception of such quotations, this dissertation is entirely my own work;\\
5. I have acknowledged all main sources of help;\\
6. Where the thesis is based on work done by myself jointly with others, I have made clear exactly what was done by others and what I have contributed myself;\\
7. Either none of this work has been published before submission, or parts of this work have been published by :\\
\\
Stefan Collier\\
April 2016
}
\tableofcontents
\listoffigures
\listoftables

\mainmatter
%% ----------------------------------------------------------------
%\include{Introduction}
%\include{Conclusions}
\include{chapters/1Project/main}
\include{chapters/2Lit/main}
\include{chapters/3Design/HighLevel}
\include{chapters/3Design/InDepth}
\include{chapters/4Impl/main}

\include{chapters/5Experiments/1/main}
\include{chapters/5Experiments/2/main}
\include{chapters/5Experiments/3/main}
\include{chapters/5Experiments/4/main}

\include{chapters/6Conclusion/main}

\appendix
\include{appendix/AppendixB}
\include{appendix/D/main}
\include{appendix/AppendixC}

\backmatter
\bibliographystyle{ecs}
\bibliography{ECS}
\end{document}
%% ----------------------------------------------------------------


\appendix
\include{appendix/AppendixB}
 %% ----------------------------------------------------------------
%% Progress.tex
%% ---------------------------------------------------------------- 
\documentclass{ecsprogress}    % Use the progress Style
\graphicspath{{../figs/}}   % Location of your graphics files
    \usepackage{natbib}            % Use Natbib style for the refs.
\hypersetup{colorlinks=true}   % Set to false for black/white printing
\input{Definitions}            % Include your abbreviations



\usepackage{enumitem}% http://ctan.org/pkg/enumitem
\usepackage{multirow}
\usepackage{float}
\usepackage{amsmath}
\usepackage{multicol}
\usepackage{amssymb}
\usepackage[normalem]{ulem}
\useunder{\uline}{\ul}{}
\usepackage{wrapfig}


\usepackage[table,xcdraw]{xcolor}


%% ----------------------------------------------------------------
\begin{document}
\frontmatter
\title      {Heterogeneous Agent-based Model for Supermarket Competition}
\authors    {\texorpdfstring
             {\href{mailto:sc22g13@ecs.soton.ac.uk}{Stefan J. Collier}}
             {Stefan J. Collier}
            }
\addresses  {\groupname\\\deptname\\\univname}
\date       {\today}
\subject    {}
\keywords   {}
\supervisor {Dr. Maria Polukarov}
\examiner   {Professor Sheng Chen}

\maketitle
\begin{abstract}
This project aim was to model and analyse the effects of competitive pricing behaviors of grocery retailers on the British market. 

This was achieved by creating a multi-agent model, containing retailer and consumer agents. The heterogeneous crowd of retailers employs either a uniform pricing strategy or a ‘local price flexing’ strategy. The actions of these retailers are chosen by predicting the profit of each action, using a perceptron. Following on from the consideration of different economic models, a discrete model was developed so that software agents have a discrete environment to operate within. Within the model, it has been observed how supermarkets with differing behaviors affect a heterogeneous crowd of consumer agents. The model was implemented in Java with Python used to evaluate the results. 

The simulation displays good acceptance with real grocery market behavior, i.e. captures the performance of British retailers thus can be used to determine the impact of changes in their behavior on their competitors and consumers.Furthermore it can be used to provide insight into sustainability of volatile pricing strategies, providing a useful insight in volatility of British supermarket retail industry. 
\end{abstract}
\acknowledgements{
I would like to express my sincere gratitude to Dr Maria Polukarov for her guidance and support which provided me the freedom to take this research in the direction of my interest.\\
\\
I would also like to thank my family and friends for their encouragement and support. To those who quietly listened to my software complaints. To those who worked throughout the nights with me. To those who helped me write what I couldn't say. I cannot thank you enough.
}

\declaration{
I, Stefan Collier, declare that this dissertation and the work presented in it are my own and has been generated by me as the result of my own original research.\\
I confirm that:\\
1. This work was done wholly or mainly while in candidature for a degree at this University;\\
2. Where any part of this dissertation has previously been submitted for any other qualification at this University or any other institution, this has been clearly stated;\\
3. Where I have consulted the published work of others, this is always clearly attributed;\\
4. Where I have quoted from the work of others, the source is always given. With the exception of such quotations, this dissertation is entirely my own work;\\
5. I have acknowledged all main sources of help;\\
6. Where the thesis is based on work done by myself jointly with others, I have made clear exactly what was done by others and what I have contributed myself;\\
7. Either none of this work has been published before submission, or parts of this work have been published by :\\
\\
Stefan Collier\\
April 2016
}
\tableofcontents
\listoffigures
\listoftables

\mainmatter
%% ----------------------------------------------------------------
%\include{Introduction}
%\include{Conclusions}
\include{chapters/1Project/main}
\include{chapters/2Lit/main}
\include{chapters/3Design/HighLevel}
\include{chapters/3Design/InDepth}
\include{chapters/4Impl/main}

\include{chapters/5Experiments/1/main}
\include{chapters/5Experiments/2/main}
\include{chapters/5Experiments/3/main}
\include{chapters/5Experiments/4/main}

\include{chapters/6Conclusion/main}

\appendix
\include{appendix/AppendixB}
\include{appendix/D/main}
\include{appendix/AppendixC}

\backmatter
\bibliographystyle{ecs}
\bibliography{ECS}
\end{document}
%% ----------------------------------------------------------------

\include{appendix/AppendixC}

\backmatter
\bibliographystyle{ecs}
\bibliography{ECS}
\end{document}
%% ----------------------------------------------------------------

\include{appendix/AppendixC}

\backmatter
\bibliographystyle{ecs}
\bibliography{ECS}
\end{document}
%% ----------------------------------------------------------------


 %% ----------------------------------------------------------------
%% Progress.tex
%% ---------------------------------------------------------------- 
\documentclass{ecsprogress}    % Use the progress Style
\graphicspath{{../figs/}}   % Location of your graphics files
    \usepackage{natbib}            % Use Natbib style for the refs.
\hypersetup{colorlinks=true}   % Set to false for black/white printing
\input{Definitions}            % Include your abbreviations



\usepackage{enumitem}% http://ctan.org/pkg/enumitem
\usepackage{multirow}
\usepackage{float}
\usepackage{amsmath}
\usepackage{multicol}
\usepackage{amssymb}
\usepackage[normalem]{ulem}
\useunder{\uline}{\ul}{}
\usepackage{wrapfig}


\usepackage[table,xcdraw]{xcolor}


%% ----------------------------------------------------------------
\begin{document}
\frontmatter
\title      {Heterogeneous Agent-based Model for Supermarket Competition}
\authors    {\texorpdfstring
             {\href{mailto:sc22g13@ecs.soton.ac.uk}{Stefan J. Collier}}
             {Stefan J. Collier}
            }
\addresses  {\groupname\\\deptname\\\univname}
\date       {\today}
\subject    {}
\keywords   {}
\supervisor {Dr. Maria Polukarov}
\examiner   {Professor Sheng Chen}

\maketitle
\begin{abstract}
This project aim was to model and analyse the effects of competitive pricing behaviors of grocery retailers on the British market. 

This was achieved by creating a multi-agent model, containing retailer and consumer agents. The heterogeneous crowd of retailers employs either a uniform pricing strategy or a ‘local price flexing’ strategy. The actions of these retailers are chosen by predicting the profit of each action, using a perceptron. Following on from the consideration of different economic models, a discrete model was developed so that software agents have a discrete environment to operate within. Within the model, it has been observed how supermarkets with differing behaviors affect a heterogeneous crowd of consumer agents. The model was implemented in Java with Python used to evaluate the results. 

The simulation displays good acceptance with real grocery market behavior, i.e. captures the performance of British retailers thus can be used to determine the impact of changes in their behavior on their competitors and consumers.Furthermore it can be used to provide insight into sustainability of volatile pricing strategies, providing a useful insight in volatility of British supermarket retail industry. 
\end{abstract}
\acknowledgements{
I would like to express my sincere gratitude to Dr Maria Polukarov for her guidance and support which provided me the freedom to take this research in the direction of my interest.\\
\\
I would also like to thank my family and friends for their encouragement and support. To those who quietly listened to my software complaints. To those who worked throughout the nights with me. To those who helped me write what I couldn't say. I cannot thank you enough.
}

\declaration{
I, Stefan Collier, declare that this dissertation and the work presented in it are my own and has been generated by me as the result of my own original research.\\
I confirm that:\\
1. This work was done wholly or mainly while in candidature for a degree at this University;\\
2. Where any part of this dissertation has previously been submitted for any other qualification at this University or any other institution, this has been clearly stated;\\
3. Where I have consulted the published work of others, this is always clearly attributed;\\
4. Where I have quoted from the work of others, the source is always given. With the exception of such quotations, this dissertation is entirely my own work;\\
5. I have acknowledged all main sources of help;\\
6. Where the thesis is based on work done by myself jointly with others, I have made clear exactly what was done by others and what I have contributed myself;\\
7. Either none of this work has been published before submission, or parts of this work have been published by :\\
\\
Stefan Collier\\
April 2016
}
\tableofcontents
\listoffigures
\listoftables

\mainmatter
%% ----------------------------------------------------------------
%\include{Introduction}
%\include{Conclusions}
 %% ----------------------------------------------------------------
%% Progress.tex
%% ---------------------------------------------------------------- 
\documentclass{ecsprogress}    % Use the progress Style
\graphicspath{{../figs/}}   % Location of your graphics files
    \usepackage{natbib}            % Use Natbib style for the refs.
\hypersetup{colorlinks=true}   % Set to false for black/white printing
\input{Definitions}            % Include your abbreviations



\usepackage{enumitem}% http://ctan.org/pkg/enumitem
\usepackage{multirow}
\usepackage{float}
\usepackage{amsmath}
\usepackage{multicol}
\usepackage{amssymb}
\usepackage[normalem]{ulem}
\useunder{\uline}{\ul}{}
\usepackage{wrapfig}


\usepackage[table,xcdraw]{xcolor}


%% ----------------------------------------------------------------
\begin{document}
\frontmatter
\title      {Heterogeneous Agent-based Model for Supermarket Competition}
\authors    {\texorpdfstring
             {\href{mailto:sc22g13@ecs.soton.ac.uk}{Stefan J. Collier}}
             {Stefan J. Collier}
            }
\addresses  {\groupname\\\deptname\\\univname}
\date       {\today}
\subject    {}
\keywords   {}
\supervisor {Dr. Maria Polukarov}
\examiner   {Professor Sheng Chen}

\maketitle
\begin{abstract}
This project aim was to model and analyse the effects of competitive pricing behaviors of grocery retailers on the British market. 

This was achieved by creating a multi-agent model, containing retailer and consumer agents. The heterogeneous crowd of retailers employs either a uniform pricing strategy or a ‘local price flexing’ strategy. The actions of these retailers are chosen by predicting the profit of each action, using a perceptron. Following on from the consideration of different economic models, a discrete model was developed so that software agents have a discrete environment to operate within. Within the model, it has been observed how supermarkets with differing behaviors affect a heterogeneous crowd of consumer agents. The model was implemented in Java with Python used to evaluate the results. 

The simulation displays good acceptance with real grocery market behavior, i.e. captures the performance of British retailers thus can be used to determine the impact of changes in their behavior on their competitors and consumers.Furthermore it can be used to provide insight into sustainability of volatile pricing strategies, providing a useful insight in volatility of British supermarket retail industry. 
\end{abstract}
\acknowledgements{
I would like to express my sincere gratitude to Dr Maria Polukarov for her guidance and support which provided me the freedom to take this research in the direction of my interest.\\
\\
I would also like to thank my family and friends for their encouragement and support. To those who quietly listened to my software complaints. To those who worked throughout the nights with me. To those who helped me write what I couldn't say. I cannot thank you enough.
}

\declaration{
I, Stefan Collier, declare that this dissertation and the work presented in it are my own and has been generated by me as the result of my own original research.\\
I confirm that:\\
1. This work was done wholly or mainly while in candidature for a degree at this University;\\
2. Where any part of this dissertation has previously been submitted for any other qualification at this University or any other institution, this has been clearly stated;\\
3. Where I have consulted the published work of others, this is always clearly attributed;\\
4. Where I have quoted from the work of others, the source is always given. With the exception of such quotations, this dissertation is entirely my own work;\\
5. I have acknowledged all main sources of help;\\
6. Where the thesis is based on work done by myself jointly with others, I have made clear exactly what was done by others and what I have contributed myself;\\
7. Either none of this work has been published before submission, or parts of this work have been published by :\\
\\
Stefan Collier\\
April 2016
}
\tableofcontents
\listoffigures
\listoftables

\mainmatter
%% ----------------------------------------------------------------
%\include{Introduction}
%\include{Conclusions}
 %% ----------------------------------------------------------------
%% Progress.tex
%% ---------------------------------------------------------------- 
\documentclass{ecsprogress}    % Use the progress Style
\graphicspath{{../figs/}}   % Location of your graphics files
    \usepackage{natbib}            % Use Natbib style for the refs.
\hypersetup{colorlinks=true}   % Set to false for black/white printing
\input{Definitions}            % Include your abbreviations



\usepackage{enumitem}% http://ctan.org/pkg/enumitem
\usepackage{multirow}
\usepackage{float}
\usepackage{amsmath}
\usepackage{multicol}
\usepackage{amssymb}
\usepackage[normalem]{ulem}
\useunder{\uline}{\ul}{}
\usepackage{wrapfig}


\usepackage[table,xcdraw]{xcolor}


%% ----------------------------------------------------------------
\begin{document}
\frontmatter
\title      {Heterogeneous Agent-based Model for Supermarket Competition}
\authors    {\texorpdfstring
             {\href{mailto:sc22g13@ecs.soton.ac.uk}{Stefan J. Collier}}
             {Stefan J. Collier}
            }
\addresses  {\groupname\\\deptname\\\univname}
\date       {\today}
\subject    {}
\keywords   {}
\supervisor {Dr. Maria Polukarov}
\examiner   {Professor Sheng Chen}

\maketitle
\begin{abstract}
This project aim was to model and analyse the effects of competitive pricing behaviors of grocery retailers on the British market. 

This was achieved by creating a multi-agent model, containing retailer and consumer agents. The heterogeneous crowd of retailers employs either a uniform pricing strategy or a ‘local price flexing’ strategy. The actions of these retailers are chosen by predicting the profit of each action, using a perceptron. Following on from the consideration of different economic models, a discrete model was developed so that software agents have a discrete environment to operate within. Within the model, it has been observed how supermarkets with differing behaviors affect a heterogeneous crowd of consumer agents. The model was implemented in Java with Python used to evaluate the results. 

The simulation displays good acceptance with real grocery market behavior, i.e. captures the performance of British retailers thus can be used to determine the impact of changes in their behavior on their competitors and consumers.Furthermore it can be used to provide insight into sustainability of volatile pricing strategies, providing a useful insight in volatility of British supermarket retail industry. 
\end{abstract}
\acknowledgements{
I would like to express my sincere gratitude to Dr Maria Polukarov for her guidance and support which provided me the freedom to take this research in the direction of my interest.\\
\\
I would also like to thank my family and friends for their encouragement and support. To those who quietly listened to my software complaints. To those who worked throughout the nights with me. To those who helped me write what I couldn't say. I cannot thank you enough.
}

\declaration{
I, Stefan Collier, declare that this dissertation and the work presented in it are my own and has been generated by me as the result of my own original research.\\
I confirm that:\\
1. This work was done wholly or mainly while in candidature for a degree at this University;\\
2. Where any part of this dissertation has previously been submitted for any other qualification at this University or any other institution, this has been clearly stated;\\
3. Where I have consulted the published work of others, this is always clearly attributed;\\
4. Where I have quoted from the work of others, the source is always given. With the exception of such quotations, this dissertation is entirely my own work;\\
5. I have acknowledged all main sources of help;\\
6. Where the thesis is based on work done by myself jointly with others, I have made clear exactly what was done by others and what I have contributed myself;\\
7. Either none of this work has been published before submission, or parts of this work have been published by :\\
\\
Stefan Collier\\
April 2016
}
\tableofcontents
\listoffigures
\listoftables

\mainmatter
%% ----------------------------------------------------------------
%\include{Introduction}
%\include{Conclusions}
\include{chapters/1Project/main}
\include{chapters/2Lit/main}
\include{chapters/3Design/HighLevel}
\include{chapters/3Design/InDepth}
\include{chapters/4Impl/main}

\include{chapters/5Experiments/1/main}
\include{chapters/5Experiments/2/main}
\include{chapters/5Experiments/3/main}
\include{chapters/5Experiments/4/main}

\include{chapters/6Conclusion/main}

\appendix
\include{appendix/AppendixB}
\include{appendix/D/main}
\include{appendix/AppendixC}

\backmatter
\bibliographystyle{ecs}
\bibliography{ECS}
\end{document}
%% ----------------------------------------------------------------

 %% ----------------------------------------------------------------
%% Progress.tex
%% ---------------------------------------------------------------- 
\documentclass{ecsprogress}    % Use the progress Style
\graphicspath{{../figs/}}   % Location of your graphics files
    \usepackage{natbib}            % Use Natbib style for the refs.
\hypersetup{colorlinks=true}   % Set to false for black/white printing
\input{Definitions}            % Include your abbreviations



\usepackage{enumitem}% http://ctan.org/pkg/enumitem
\usepackage{multirow}
\usepackage{float}
\usepackage{amsmath}
\usepackage{multicol}
\usepackage{amssymb}
\usepackage[normalem]{ulem}
\useunder{\uline}{\ul}{}
\usepackage{wrapfig}


\usepackage[table,xcdraw]{xcolor}


%% ----------------------------------------------------------------
\begin{document}
\frontmatter
\title      {Heterogeneous Agent-based Model for Supermarket Competition}
\authors    {\texorpdfstring
             {\href{mailto:sc22g13@ecs.soton.ac.uk}{Stefan J. Collier}}
             {Stefan J. Collier}
            }
\addresses  {\groupname\\\deptname\\\univname}
\date       {\today}
\subject    {}
\keywords   {}
\supervisor {Dr. Maria Polukarov}
\examiner   {Professor Sheng Chen}

\maketitle
\begin{abstract}
This project aim was to model and analyse the effects of competitive pricing behaviors of grocery retailers on the British market. 

This was achieved by creating a multi-agent model, containing retailer and consumer agents. The heterogeneous crowd of retailers employs either a uniform pricing strategy or a ‘local price flexing’ strategy. The actions of these retailers are chosen by predicting the profit of each action, using a perceptron. Following on from the consideration of different economic models, a discrete model was developed so that software agents have a discrete environment to operate within. Within the model, it has been observed how supermarkets with differing behaviors affect a heterogeneous crowd of consumer agents. The model was implemented in Java with Python used to evaluate the results. 

The simulation displays good acceptance with real grocery market behavior, i.e. captures the performance of British retailers thus can be used to determine the impact of changes in their behavior on their competitors and consumers.Furthermore it can be used to provide insight into sustainability of volatile pricing strategies, providing a useful insight in volatility of British supermarket retail industry. 
\end{abstract}
\acknowledgements{
I would like to express my sincere gratitude to Dr Maria Polukarov for her guidance and support which provided me the freedom to take this research in the direction of my interest.\\
\\
I would also like to thank my family and friends for their encouragement and support. To those who quietly listened to my software complaints. To those who worked throughout the nights with me. To those who helped me write what I couldn't say. I cannot thank you enough.
}

\declaration{
I, Stefan Collier, declare that this dissertation and the work presented in it are my own and has been generated by me as the result of my own original research.\\
I confirm that:\\
1. This work was done wholly or mainly while in candidature for a degree at this University;\\
2. Where any part of this dissertation has previously been submitted for any other qualification at this University or any other institution, this has been clearly stated;\\
3. Where I have consulted the published work of others, this is always clearly attributed;\\
4. Where I have quoted from the work of others, the source is always given. With the exception of such quotations, this dissertation is entirely my own work;\\
5. I have acknowledged all main sources of help;\\
6. Where the thesis is based on work done by myself jointly with others, I have made clear exactly what was done by others and what I have contributed myself;\\
7. Either none of this work has been published before submission, or parts of this work have been published by :\\
\\
Stefan Collier\\
April 2016
}
\tableofcontents
\listoffigures
\listoftables

\mainmatter
%% ----------------------------------------------------------------
%\include{Introduction}
%\include{Conclusions}
\include{chapters/1Project/main}
\include{chapters/2Lit/main}
\include{chapters/3Design/HighLevel}
\include{chapters/3Design/InDepth}
\include{chapters/4Impl/main}

\include{chapters/5Experiments/1/main}
\include{chapters/5Experiments/2/main}
\include{chapters/5Experiments/3/main}
\include{chapters/5Experiments/4/main}

\include{chapters/6Conclusion/main}

\appendix
\include{appendix/AppendixB}
\include{appendix/D/main}
\include{appendix/AppendixC}

\backmatter
\bibliographystyle{ecs}
\bibliography{ECS}
\end{document}
%% ----------------------------------------------------------------

\include{chapters/3Design/HighLevel}
\include{chapters/3Design/InDepth}
 %% ----------------------------------------------------------------
%% Progress.tex
%% ---------------------------------------------------------------- 
\documentclass{ecsprogress}    % Use the progress Style
\graphicspath{{../figs/}}   % Location of your graphics files
    \usepackage{natbib}            % Use Natbib style for the refs.
\hypersetup{colorlinks=true}   % Set to false for black/white printing
\input{Definitions}            % Include your abbreviations



\usepackage{enumitem}% http://ctan.org/pkg/enumitem
\usepackage{multirow}
\usepackage{float}
\usepackage{amsmath}
\usepackage{multicol}
\usepackage{amssymb}
\usepackage[normalem]{ulem}
\useunder{\uline}{\ul}{}
\usepackage{wrapfig}


\usepackage[table,xcdraw]{xcolor}


%% ----------------------------------------------------------------
\begin{document}
\frontmatter
\title      {Heterogeneous Agent-based Model for Supermarket Competition}
\authors    {\texorpdfstring
             {\href{mailto:sc22g13@ecs.soton.ac.uk}{Stefan J. Collier}}
             {Stefan J. Collier}
            }
\addresses  {\groupname\\\deptname\\\univname}
\date       {\today}
\subject    {}
\keywords   {}
\supervisor {Dr. Maria Polukarov}
\examiner   {Professor Sheng Chen}

\maketitle
\begin{abstract}
This project aim was to model and analyse the effects of competitive pricing behaviors of grocery retailers on the British market. 

This was achieved by creating a multi-agent model, containing retailer and consumer agents. The heterogeneous crowd of retailers employs either a uniform pricing strategy or a ‘local price flexing’ strategy. The actions of these retailers are chosen by predicting the profit of each action, using a perceptron. Following on from the consideration of different economic models, a discrete model was developed so that software agents have a discrete environment to operate within. Within the model, it has been observed how supermarkets with differing behaviors affect a heterogeneous crowd of consumer agents. The model was implemented in Java with Python used to evaluate the results. 

The simulation displays good acceptance with real grocery market behavior, i.e. captures the performance of British retailers thus can be used to determine the impact of changes in their behavior on their competitors and consumers.Furthermore it can be used to provide insight into sustainability of volatile pricing strategies, providing a useful insight in volatility of British supermarket retail industry. 
\end{abstract}
\acknowledgements{
I would like to express my sincere gratitude to Dr Maria Polukarov for her guidance and support which provided me the freedom to take this research in the direction of my interest.\\
\\
I would also like to thank my family and friends for their encouragement and support. To those who quietly listened to my software complaints. To those who worked throughout the nights with me. To those who helped me write what I couldn't say. I cannot thank you enough.
}

\declaration{
I, Stefan Collier, declare that this dissertation and the work presented in it are my own and has been generated by me as the result of my own original research.\\
I confirm that:\\
1. This work was done wholly or mainly while in candidature for a degree at this University;\\
2. Where any part of this dissertation has previously been submitted for any other qualification at this University or any other institution, this has been clearly stated;\\
3. Where I have consulted the published work of others, this is always clearly attributed;\\
4. Where I have quoted from the work of others, the source is always given. With the exception of such quotations, this dissertation is entirely my own work;\\
5. I have acknowledged all main sources of help;\\
6. Where the thesis is based on work done by myself jointly with others, I have made clear exactly what was done by others and what I have contributed myself;\\
7. Either none of this work has been published before submission, or parts of this work have been published by :\\
\\
Stefan Collier\\
April 2016
}
\tableofcontents
\listoffigures
\listoftables

\mainmatter
%% ----------------------------------------------------------------
%\include{Introduction}
%\include{Conclusions}
\include{chapters/1Project/main}
\include{chapters/2Lit/main}
\include{chapters/3Design/HighLevel}
\include{chapters/3Design/InDepth}
\include{chapters/4Impl/main}

\include{chapters/5Experiments/1/main}
\include{chapters/5Experiments/2/main}
\include{chapters/5Experiments/3/main}
\include{chapters/5Experiments/4/main}

\include{chapters/6Conclusion/main}

\appendix
\include{appendix/AppendixB}
\include{appendix/D/main}
\include{appendix/AppendixC}

\backmatter
\bibliographystyle{ecs}
\bibliography{ECS}
\end{document}
%% ----------------------------------------------------------------


 %% ----------------------------------------------------------------
%% Progress.tex
%% ---------------------------------------------------------------- 
\documentclass{ecsprogress}    % Use the progress Style
\graphicspath{{../figs/}}   % Location of your graphics files
    \usepackage{natbib}            % Use Natbib style for the refs.
\hypersetup{colorlinks=true}   % Set to false for black/white printing
\input{Definitions}            % Include your abbreviations



\usepackage{enumitem}% http://ctan.org/pkg/enumitem
\usepackage{multirow}
\usepackage{float}
\usepackage{amsmath}
\usepackage{multicol}
\usepackage{amssymb}
\usepackage[normalem]{ulem}
\useunder{\uline}{\ul}{}
\usepackage{wrapfig}


\usepackage[table,xcdraw]{xcolor}


%% ----------------------------------------------------------------
\begin{document}
\frontmatter
\title      {Heterogeneous Agent-based Model for Supermarket Competition}
\authors    {\texorpdfstring
             {\href{mailto:sc22g13@ecs.soton.ac.uk}{Stefan J. Collier}}
             {Stefan J. Collier}
            }
\addresses  {\groupname\\\deptname\\\univname}
\date       {\today}
\subject    {}
\keywords   {}
\supervisor {Dr. Maria Polukarov}
\examiner   {Professor Sheng Chen}

\maketitle
\begin{abstract}
This project aim was to model and analyse the effects of competitive pricing behaviors of grocery retailers on the British market. 

This was achieved by creating a multi-agent model, containing retailer and consumer agents. The heterogeneous crowd of retailers employs either a uniform pricing strategy or a ‘local price flexing’ strategy. The actions of these retailers are chosen by predicting the profit of each action, using a perceptron. Following on from the consideration of different economic models, a discrete model was developed so that software agents have a discrete environment to operate within. Within the model, it has been observed how supermarkets with differing behaviors affect a heterogeneous crowd of consumer agents. The model was implemented in Java with Python used to evaluate the results. 

The simulation displays good acceptance with real grocery market behavior, i.e. captures the performance of British retailers thus can be used to determine the impact of changes in their behavior on their competitors and consumers.Furthermore it can be used to provide insight into sustainability of volatile pricing strategies, providing a useful insight in volatility of British supermarket retail industry. 
\end{abstract}
\acknowledgements{
I would like to express my sincere gratitude to Dr Maria Polukarov for her guidance and support which provided me the freedom to take this research in the direction of my interest.\\
\\
I would also like to thank my family and friends for their encouragement and support. To those who quietly listened to my software complaints. To those who worked throughout the nights with me. To those who helped me write what I couldn't say. I cannot thank you enough.
}

\declaration{
I, Stefan Collier, declare that this dissertation and the work presented in it are my own and has been generated by me as the result of my own original research.\\
I confirm that:\\
1. This work was done wholly or mainly while in candidature for a degree at this University;\\
2. Where any part of this dissertation has previously been submitted for any other qualification at this University or any other institution, this has been clearly stated;\\
3. Where I have consulted the published work of others, this is always clearly attributed;\\
4. Where I have quoted from the work of others, the source is always given. With the exception of such quotations, this dissertation is entirely my own work;\\
5. I have acknowledged all main sources of help;\\
6. Where the thesis is based on work done by myself jointly with others, I have made clear exactly what was done by others and what I have contributed myself;\\
7. Either none of this work has been published before submission, or parts of this work have been published by :\\
\\
Stefan Collier\\
April 2016
}
\tableofcontents
\listoffigures
\listoftables

\mainmatter
%% ----------------------------------------------------------------
%\include{Introduction}
%\include{Conclusions}
\include{chapters/1Project/main}
\include{chapters/2Lit/main}
\include{chapters/3Design/HighLevel}
\include{chapters/3Design/InDepth}
\include{chapters/4Impl/main}

\include{chapters/5Experiments/1/main}
\include{chapters/5Experiments/2/main}
\include{chapters/5Experiments/3/main}
\include{chapters/5Experiments/4/main}

\include{chapters/6Conclusion/main}

\appendix
\include{appendix/AppendixB}
\include{appendix/D/main}
\include{appendix/AppendixC}

\backmatter
\bibliographystyle{ecs}
\bibliography{ECS}
\end{document}
%% ----------------------------------------------------------------

 %% ----------------------------------------------------------------
%% Progress.tex
%% ---------------------------------------------------------------- 
\documentclass{ecsprogress}    % Use the progress Style
\graphicspath{{../figs/}}   % Location of your graphics files
    \usepackage{natbib}            % Use Natbib style for the refs.
\hypersetup{colorlinks=true}   % Set to false for black/white printing
\input{Definitions}            % Include your abbreviations



\usepackage{enumitem}% http://ctan.org/pkg/enumitem
\usepackage{multirow}
\usepackage{float}
\usepackage{amsmath}
\usepackage{multicol}
\usepackage{amssymb}
\usepackage[normalem]{ulem}
\useunder{\uline}{\ul}{}
\usepackage{wrapfig}


\usepackage[table,xcdraw]{xcolor}


%% ----------------------------------------------------------------
\begin{document}
\frontmatter
\title      {Heterogeneous Agent-based Model for Supermarket Competition}
\authors    {\texorpdfstring
             {\href{mailto:sc22g13@ecs.soton.ac.uk}{Stefan J. Collier}}
             {Stefan J. Collier}
            }
\addresses  {\groupname\\\deptname\\\univname}
\date       {\today}
\subject    {}
\keywords   {}
\supervisor {Dr. Maria Polukarov}
\examiner   {Professor Sheng Chen}

\maketitle
\begin{abstract}
This project aim was to model and analyse the effects of competitive pricing behaviors of grocery retailers on the British market. 

This was achieved by creating a multi-agent model, containing retailer and consumer agents. The heterogeneous crowd of retailers employs either a uniform pricing strategy or a ‘local price flexing’ strategy. The actions of these retailers are chosen by predicting the profit of each action, using a perceptron. Following on from the consideration of different economic models, a discrete model was developed so that software agents have a discrete environment to operate within. Within the model, it has been observed how supermarkets with differing behaviors affect a heterogeneous crowd of consumer agents. The model was implemented in Java with Python used to evaluate the results. 

The simulation displays good acceptance with real grocery market behavior, i.e. captures the performance of British retailers thus can be used to determine the impact of changes in their behavior on their competitors and consumers.Furthermore it can be used to provide insight into sustainability of volatile pricing strategies, providing a useful insight in volatility of British supermarket retail industry. 
\end{abstract}
\acknowledgements{
I would like to express my sincere gratitude to Dr Maria Polukarov for her guidance and support which provided me the freedom to take this research in the direction of my interest.\\
\\
I would also like to thank my family and friends for their encouragement and support. To those who quietly listened to my software complaints. To those who worked throughout the nights with me. To those who helped me write what I couldn't say. I cannot thank you enough.
}

\declaration{
I, Stefan Collier, declare that this dissertation and the work presented in it are my own and has been generated by me as the result of my own original research.\\
I confirm that:\\
1. This work was done wholly or mainly while in candidature for a degree at this University;\\
2. Where any part of this dissertation has previously been submitted for any other qualification at this University or any other institution, this has been clearly stated;\\
3. Where I have consulted the published work of others, this is always clearly attributed;\\
4. Where I have quoted from the work of others, the source is always given. With the exception of such quotations, this dissertation is entirely my own work;\\
5. I have acknowledged all main sources of help;\\
6. Where the thesis is based on work done by myself jointly with others, I have made clear exactly what was done by others and what I have contributed myself;\\
7. Either none of this work has been published before submission, or parts of this work have been published by :\\
\\
Stefan Collier\\
April 2016
}
\tableofcontents
\listoffigures
\listoftables

\mainmatter
%% ----------------------------------------------------------------
%\include{Introduction}
%\include{Conclusions}
\include{chapters/1Project/main}
\include{chapters/2Lit/main}
\include{chapters/3Design/HighLevel}
\include{chapters/3Design/InDepth}
\include{chapters/4Impl/main}

\include{chapters/5Experiments/1/main}
\include{chapters/5Experiments/2/main}
\include{chapters/5Experiments/3/main}
\include{chapters/5Experiments/4/main}

\include{chapters/6Conclusion/main}

\appendix
\include{appendix/AppendixB}
\include{appendix/D/main}
\include{appendix/AppendixC}

\backmatter
\bibliographystyle{ecs}
\bibliography{ECS}
\end{document}
%% ----------------------------------------------------------------

 %% ----------------------------------------------------------------
%% Progress.tex
%% ---------------------------------------------------------------- 
\documentclass{ecsprogress}    % Use the progress Style
\graphicspath{{../figs/}}   % Location of your graphics files
    \usepackage{natbib}            % Use Natbib style for the refs.
\hypersetup{colorlinks=true}   % Set to false for black/white printing
\input{Definitions}            % Include your abbreviations



\usepackage{enumitem}% http://ctan.org/pkg/enumitem
\usepackage{multirow}
\usepackage{float}
\usepackage{amsmath}
\usepackage{multicol}
\usepackage{amssymb}
\usepackage[normalem]{ulem}
\useunder{\uline}{\ul}{}
\usepackage{wrapfig}


\usepackage[table,xcdraw]{xcolor}


%% ----------------------------------------------------------------
\begin{document}
\frontmatter
\title      {Heterogeneous Agent-based Model for Supermarket Competition}
\authors    {\texorpdfstring
             {\href{mailto:sc22g13@ecs.soton.ac.uk}{Stefan J. Collier}}
             {Stefan J. Collier}
            }
\addresses  {\groupname\\\deptname\\\univname}
\date       {\today}
\subject    {}
\keywords   {}
\supervisor {Dr. Maria Polukarov}
\examiner   {Professor Sheng Chen}

\maketitle
\begin{abstract}
This project aim was to model and analyse the effects of competitive pricing behaviors of grocery retailers on the British market. 

This was achieved by creating a multi-agent model, containing retailer and consumer agents. The heterogeneous crowd of retailers employs either a uniform pricing strategy or a ‘local price flexing’ strategy. The actions of these retailers are chosen by predicting the profit of each action, using a perceptron. Following on from the consideration of different economic models, a discrete model was developed so that software agents have a discrete environment to operate within. Within the model, it has been observed how supermarkets with differing behaviors affect a heterogeneous crowd of consumer agents. The model was implemented in Java with Python used to evaluate the results. 

The simulation displays good acceptance with real grocery market behavior, i.e. captures the performance of British retailers thus can be used to determine the impact of changes in their behavior on their competitors and consumers.Furthermore it can be used to provide insight into sustainability of volatile pricing strategies, providing a useful insight in volatility of British supermarket retail industry. 
\end{abstract}
\acknowledgements{
I would like to express my sincere gratitude to Dr Maria Polukarov for her guidance and support which provided me the freedom to take this research in the direction of my interest.\\
\\
I would also like to thank my family and friends for their encouragement and support. To those who quietly listened to my software complaints. To those who worked throughout the nights with me. To those who helped me write what I couldn't say. I cannot thank you enough.
}

\declaration{
I, Stefan Collier, declare that this dissertation and the work presented in it are my own and has been generated by me as the result of my own original research.\\
I confirm that:\\
1. This work was done wholly or mainly while in candidature for a degree at this University;\\
2. Where any part of this dissertation has previously been submitted for any other qualification at this University or any other institution, this has been clearly stated;\\
3. Where I have consulted the published work of others, this is always clearly attributed;\\
4. Where I have quoted from the work of others, the source is always given. With the exception of such quotations, this dissertation is entirely my own work;\\
5. I have acknowledged all main sources of help;\\
6. Where the thesis is based on work done by myself jointly with others, I have made clear exactly what was done by others and what I have contributed myself;\\
7. Either none of this work has been published before submission, or parts of this work have been published by :\\
\\
Stefan Collier\\
April 2016
}
\tableofcontents
\listoffigures
\listoftables

\mainmatter
%% ----------------------------------------------------------------
%\include{Introduction}
%\include{Conclusions}
\include{chapters/1Project/main}
\include{chapters/2Lit/main}
\include{chapters/3Design/HighLevel}
\include{chapters/3Design/InDepth}
\include{chapters/4Impl/main}

\include{chapters/5Experiments/1/main}
\include{chapters/5Experiments/2/main}
\include{chapters/5Experiments/3/main}
\include{chapters/5Experiments/4/main}

\include{chapters/6Conclusion/main}

\appendix
\include{appendix/AppendixB}
\include{appendix/D/main}
\include{appendix/AppendixC}

\backmatter
\bibliographystyle{ecs}
\bibliography{ECS}
\end{document}
%% ----------------------------------------------------------------

 %% ----------------------------------------------------------------
%% Progress.tex
%% ---------------------------------------------------------------- 
\documentclass{ecsprogress}    % Use the progress Style
\graphicspath{{../figs/}}   % Location of your graphics files
    \usepackage{natbib}            % Use Natbib style for the refs.
\hypersetup{colorlinks=true}   % Set to false for black/white printing
\input{Definitions}            % Include your abbreviations



\usepackage{enumitem}% http://ctan.org/pkg/enumitem
\usepackage{multirow}
\usepackage{float}
\usepackage{amsmath}
\usepackage{multicol}
\usepackage{amssymb}
\usepackage[normalem]{ulem}
\useunder{\uline}{\ul}{}
\usepackage{wrapfig}


\usepackage[table,xcdraw]{xcolor}


%% ----------------------------------------------------------------
\begin{document}
\frontmatter
\title      {Heterogeneous Agent-based Model for Supermarket Competition}
\authors    {\texorpdfstring
             {\href{mailto:sc22g13@ecs.soton.ac.uk}{Stefan J. Collier}}
             {Stefan J. Collier}
            }
\addresses  {\groupname\\\deptname\\\univname}
\date       {\today}
\subject    {}
\keywords   {}
\supervisor {Dr. Maria Polukarov}
\examiner   {Professor Sheng Chen}

\maketitle
\begin{abstract}
This project aim was to model and analyse the effects of competitive pricing behaviors of grocery retailers on the British market. 

This was achieved by creating a multi-agent model, containing retailer and consumer agents. The heterogeneous crowd of retailers employs either a uniform pricing strategy or a ‘local price flexing’ strategy. The actions of these retailers are chosen by predicting the profit of each action, using a perceptron. Following on from the consideration of different economic models, a discrete model was developed so that software agents have a discrete environment to operate within. Within the model, it has been observed how supermarkets with differing behaviors affect a heterogeneous crowd of consumer agents. The model was implemented in Java with Python used to evaluate the results. 

The simulation displays good acceptance with real grocery market behavior, i.e. captures the performance of British retailers thus can be used to determine the impact of changes in their behavior on their competitors and consumers.Furthermore it can be used to provide insight into sustainability of volatile pricing strategies, providing a useful insight in volatility of British supermarket retail industry. 
\end{abstract}
\acknowledgements{
I would like to express my sincere gratitude to Dr Maria Polukarov for her guidance and support which provided me the freedom to take this research in the direction of my interest.\\
\\
I would also like to thank my family and friends for their encouragement and support. To those who quietly listened to my software complaints. To those who worked throughout the nights with me. To those who helped me write what I couldn't say. I cannot thank you enough.
}

\declaration{
I, Stefan Collier, declare that this dissertation and the work presented in it are my own and has been generated by me as the result of my own original research.\\
I confirm that:\\
1. This work was done wholly or mainly while in candidature for a degree at this University;\\
2. Where any part of this dissertation has previously been submitted for any other qualification at this University or any other institution, this has been clearly stated;\\
3. Where I have consulted the published work of others, this is always clearly attributed;\\
4. Where I have quoted from the work of others, the source is always given. With the exception of such quotations, this dissertation is entirely my own work;\\
5. I have acknowledged all main sources of help;\\
6. Where the thesis is based on work done by myself jointly with others, I have made clear exactly what was done by others and what I have contributed myself;\\
7. Either none of this work has been published before submission, or parts of this work have been published by :\\
\\
Stefan Collier\\
April 2016
}
\tableofcontents
\listoffigures
\listoftables

\mainmatter
%% ----------------------------------------------------------------
%\include{Introduction}
%\include{Conclusions}
\include{chapters/1Project/main}
\include{chapters/2Lit/main}
\include{chapters/3Design/HighLevel}
\include{chapters/3Design/InDepth}
\include{chapters/4Impl/main}

\include{chapters/5Experiments/1/main}
\include{chapters/5Experiments/2/main}
\include{chapters/5Experiments/3/main}
\include{chapters/5Experiments/4/main}

\include{chapters/6Conclusion/main}

\appendix
\include{appendix/AppendixB}
\include{appendix/D/main}
\include{appendix/AppendixC}

\backmatter
\bibliographystyle{ecs}
\bibliography{ECS}
\end{document}
%% ----------------------------------------------------------------


 %% ----------------------------------------------------------------
%% Progress.tex
%% ---------------------------------------------------------------- 
\documentclass{ecsprogress}    % Use the progress Style
\graphicspath{{../figs/}}   % Location of your graphics files
    \usepackage{natbib}            % Use Natbib style for the refs.
\hypersetup{colorlinks=true}   % Set to false for black/white printing
\input{Definitions}            % Include your abbreviations



\usepackage{enumitem}% http://ctan.org/pkg/enumitem
\usepackage{multirow}
\usepackage{float}
\usepackage{amsmath}
\usepackage{multicol}
\usepackage{amssymb}
\usepackage[normalem]{ulem}
\useunder{\uline}{\ul}{}
\usepackage{wrapfig}


\usepackage[table,xcdraw]{xcolor}


%% ----------------------------------------------------------------
\begin{document}
\frontmatter
\title      {Heterogeneous Agent-based Model for Supermarket Competition}
\authors    {\texorpdfstring
             {\href{mailto:sc22g13@ecs.soton.ac.uk}{Stefan J. Collier}}
             {Stefan J. Collier}
            }
\addresses  {\groupname\\\deptname\\\univname}
\date       {\today}
\subject    {}
\keywords   {}
\supervisor {Dr. Maria Polukarov}
\examiner   {Professor Sheng Chen}

\maketitle
\begin{abstract}
This project aim was to model and analyse the effects of competitive pricing behaviors of grocery retailers on the British market. 

This was achieved by creating a multi-agent model, containing retailer and consumer agents. The heterogeneous crowd of retailers employs either a uniform pricing strategy or a ‘local price flexing’ strategy. The actions of these retailers are chosen by predicting the profit of each action, using a perceptron. Following on from the consideration of different economic models, a discrete model was developed so that software agents have a discrete environment to operate within. Within the model, it has been observed how supermarkets with differing behaviors affect a heterogeneous crowd of consumer agents. The model was implemented in Java with Python used to evaluate the results. 

The simulation displays good acceptance with real grocery market behavior, i.e. captures the performance of British retailers thus can be used to determine the impact of changes in their behavior on their competitors and consumers.Furthermore it can be used to provide insight into sustainability of volatile pricing strategies, providing a useful insight in volatility of British supermarket retail industry. 
\end{abstract}
\acknowledgements{
I would like to express my sincere gratitude to Dr Maria Polukarov for her guidance and support which provided me the freedom to take this research in the direction of my interest.\\
\\
I would also like to thank my family and friends for their encouragement and support. To those who quietly listened to my software complaints. To those who worked throughout the nights with me. To those who helped me write what I couldn't say. I cannot thank you enough.
}

\declaration{
I, Stefan Collier, declare that this dissertation and the work presented in it are my own and has been generated by me as the result of my own original research.\\
I confirm that:\\
1. This work was done wholly or mainly while in candidature for a degree at this University;\\
2. Where any part of this dissertation has previously been submitted for any other qualification at this University or any other institution, this has been clearly stated;\\
3. Where I have consulted the published work of others, this is always clearly attributed;\\
4. Where I have quoted from the work of others, the source is always given. With the exception of such quotations, this dissertation is entirely my own work;\\
5. I have acknowledged all main sources of help;\\
6. Where the thesis is based on work done by myself jointly with others, I have made clear exactly what was done by others and what I have contributed myself;\\
7. Either none of this work has been published before submission, or parts of this work have been published by :\\
\\
Stefan Collier\\
April 2016
}
\tableofcontents
\listoffigures
\listoftables

\mainmatter
%% ----------------------------------------------------------------
%\include{Introduction}
%\include{Conclusions}
\include{chapters/1Project/main}
\include{chapters/2Lit/main}
\include{chapters/3Design/HighLevel}
\include{chapters/3Design/InDepth}
\include{chapters/4Impl/main}

\include{chapters/5Experiments/1/main}
\include{chapters/5Experiments/2/main}
\include{chapters/5Experiments/3/main}
\include{chapters/5Experiments/4/main}

\include{chapters/6Conclusion/main}

\appendix
\include{appendix/AppendixB}
\include{appendix/D/main}
\include{appendix/AppendixC}

\backmatter
\bibliographystyle{ecs}
\bibliography{ECS}
\end{document}
%% ----------------------------------------------------------------


\appendix
\include{appendix/AppendixB}
 %% ----------------------------------------------------------------
%% Progress.tex
%% ---------------------------------------------------------------- 
\documentclass{ecsprogress}    % Use the progress Style
\graphicspath{{../figs/}}   % Location of your graphics files
    \usepackage{natbib}            % Use Natbib style for the refs.
\hypersetup{colorlinks=true}   % Set to false for black/white printing
\input{Definitions}            % Include your abbreviations



\usepackage{enumitem}% http://ctan.org/pkg/enumitem
\usepackage{multirow}
\usepackage{float}
\usepackage{amsmath}
\usepackage{multicol}
\usepackage{amssymb}
\usepackage[normalem]{ulem}
\useunder{\uline}{\ul}{}
\usepackage{wrapfig}


\usepackage[table,xcdraw]{xcolor}


%% ----------------------------------------------------------------
\begin{document}
\frontmatter
\title      {Heterogeneous Agent-based Model for Supermarket Competition}
\authors    {\texorpdfstring
             {\href{mailto:sc22g13@ecs.soton.ac.uk}{Stefan J. Collier}}
             {Stefan J. Collier}
            }
\addresses  {\groupname\\\deptname\\\univname}
\date       {\today}
\subject    {}
\keywords   {}
\supervisor {Dr. Maria Polukarov}
\examiner   {Professor Sheng Chen}

\maketitle
\begin{abstract}
This project aim was to model and analyse the effects of competitive pricing behaviors of grocery retailers on the British market. 

This was achieved by creating a multi-agent model, containing retailer and consumer agents. The heterogeneous crowd of retailers employs either a uniform pricing strategy or a ‘local price flexing’ strategy. The actions of these retailers are chosen by predicting the profit of each action, using a perceptron. Following on from the consideration of different economic models, a discrete model was developed so that software agents have a discrete environment to operate within. Within the model, it has been observed how supermarkets with differing behaviors affect a heterogeneous crowd of consumer agents. The model was implemented in Java with Python used to evaluate the results. 

The simulation displays good acceptance with real grocery market behavior, i.e. captures the performance of British retailers thus can be used to determine the impact of changes in their behavior on their competitors and consumers.Furthermore it can be used to provide insight into sustainability of volatile pricing strategies, providing a useful insight in volatility of British supermarket retail industry. 
\end{abstract}
\acknowledgements{
I would like to express my sincere gratitude to Dr Maria Polukarov for her guidance and support which provided me the freedom to take this research in the direction of my interest.\\
\\
I would also like to thank my family and friends for their encouragement and support. To those who quietly listened to my software complaints. To those who worked throughout the nights with me. To those who helped me write what I couldn't say. I cannot thank you enough.
}

\declaration{
I, Stefan Collier, declare that this dissertation and the work presented in it are my own and has been generated by me as the result of my own original research.\\
I confirm that:\\
1. This work was done wholly or mainly while in candidature for a degree at this University;\\
2. Where any part of this dissertation has previously been submitted for any other qualification at this University or any other institution, this has been clearly stated;\\
3. Where I have consulted the published work of others, this is always clearly attributed;\\
4. Where I have quoted from the work of others, the source is always given. With the exception of such quotations, this dissertation is entirely my own work;\\
5. I have acknowledged all main sources of help;\\
6. Where the thesis is based on work done by myself jointly with others, I have made clear exactly what was done by others and what I have contributed myself;\\
7. Either none of this work has been published before submission, or parts of this work have been published by :\\
\\
Stefan Collier\\
April 2016
}
\tableofcontents
\listoffigures
\listoftables

\mainmatter
%% ----------------------------------------------------------------
%\include{Introduction}
%\include{Conclusions}
\include{chapters/1Project/main}
\include{chapters/2Lit/main}
\include{chapters/3Design/HighLevel}
\include{chapters/3Design/InDepth}
\include{chapters/4Impl/main}

\include{chapters/5Experiments/1/main}
\include{chapters/5Experiments/2/main}
\include{chapters/5Experiments/3/main}
\include{chapters/5Experiments/4/main}

\include{chapters/6Conclusion/main}

\appendix
\include{appendix/AppendixB}
\include{appendix/D/main}
\include{appendix/AppendixC}

\backmatter
\bibliographystyle{ecs}
\bibliography{ECS}
\end{document}
%% ----------------------------------------------------------------

\include{appendix/AppendixC}

\backmatter
\bibliographystyle{ecs}
\bibliography{ECS}
\end{document}
%% ----------------------------------------------------------------

 %% ----------------------------------------------------------------
%% Progress.tex
%% ---------------------------------------------------------------- 
\documentclass{ecsprogress}    % Use the progress Style
\graphicspath{{../figs/}}   % Location of your graphics files
    \usepackage{natbib}            % Use Natbib style for the refs.
\hypersetup{colorlinks=true}   % Set to false for black/white printing
\input{Definitions}            % Include your abbreviations



\usepackage{enumitem}% http://ctan.org/pkg/enumitem
\usepackage{multirow}
\usepackage{float}
\usepackage{amsmath}
\usepackage{multicol}
\usepackage{amssymb}
\usepackage[normalem]{ulem}
\useunder{\uline}{\ul}{}
\usepackage{wrapfig}


\usepackage[table,xcdraw]{xcolor}


%% ----------------------------------------------------------------
\begin{document}
\frontmatter
\title      {Heterogeneous Agent-based Model for Supermarket Competition}
\authors    {\texorpdfstring
             {\href{mailto:sc22g13@ecs.soton.ac.uk}{Stefan J. Collier}}
             {Stefan J. Collier}
            }
\addresses  {\groupname\\\deptname\\\univname}
\date       {\today}
\subject    {}
\keywords   {}
\supervisor {Dr. Maria Polukarov}
\examiner   {Professor Sheng Chen}

\maketitle
\begin{abstract}
This project aim was to model and analyse the effects of competitive pricing behaviors of grocery retailers on the British market. 

This was achieved by creating a multi-agent model, containing retailer and consumer agents. The heterogeneous crowd of retailers employs either a uniform pricing strategy or a ‘local price flexing’ strategy. The actions of these retailers are chosen by predicting the profit of each action, using a perceptron. Following on from the consideration of different economic models, a discrete model was developed so that software agents have a discrete environment to operate within. Within the model, it has been observed how supermarkets with differing behaviors affect a heterogeneous crowd of consumer agents. The model was implemented in Java with Python used to evaluate the results. 

The simulation displays good acceptance with real grocery market behavior, i.e. captures the performance of British retailers thus can be used to determine the impact of changes in their behavior on their competitors and consumers.Furthermore it can be used to provide insight into sustainability of volatile pricing strategies, providing a useful insight in volatility of British supermarket retail industry. 
\end{abstract}
\acknowledgements{
I would like to express my sincere gratitude to Dr Maria Polukarov for her guidance and support which provided me the freedom to take this research in the direction of my interest.\\
\\
I would also like to thank my family and friends for their encouragement and support. To those who quietly listened to my software complaints. To those who worked throughout the nights with me. To those who helped me write what I couldn't say. I cannot thank you enough.
}

\declaration{
I, Stefan Collier, declare that this dissertation and the work presented in it are my own and has been generated by me as the result of my own original research.\\
I confirm that:\\
1. This work was done wholly or mainly while in candidature for a degree at this University;\\
2. Where any part of this dissertation has previously been submitted for any other qualification at this University or any other institution, this has been clearly stated;\\
3. Where I have consulted the published work of others, this is always clearly attributed;\\
4. Where I have quoted from the work of others, the source is always given. With the exception of such quotations, this dissertation is entirely my own work;\\
5. I have acknowledged all main sources of help;\\
6. Where the thesis is based on work done by myself jointly with others, I have made clear exactly what was done by others and what I have contributed myself;\\
7. Either none of this work has been published before submission, or parts of this work have been published by :\\
\\
Stefan Collier\\
April 2016
}
\tableofcontents
\listoffigures
\listoftables

\mainmatter
%% ----------------------------------------------------------------
%\include{Introduction}
%\include{Conclusions}
 %% ----------------------------------------------------------------
%% Progress.tex
%% ---------------------------------------------------------------- 
\documentclass{ecsprogress}    % Use the progress Style
\graphicspath{{../figs/}}   % Location of your graphics files
    \usepackage{natbib}            % Use Natbib style for the refs.
\hypersetup{colorlinks=true}   % Set to false for black/white printing
\input{Definitions}            % Include your abbreviations



\usepackage{enumitem}% http://ctan.org/pkg/enumitem
\usepackage{multirow}
\usepackage{float}
\usepackage{amsmath}
\usepackage{multicol}
\usepackage{amssymb}
\usepackage[normalem]{ulem}
\useunder{\uline}{\ul}{}
\usepackage{wrapfig}


\usepackage[table,xcdraw]{xcolor}


%% ----------------------------------------------------------------
\begin{document}
\frontmatter
\title      {Heterogeneous Agent-based Model for Supermarket Competition}
\authors    {\texorpdfstring
             {\href{mailto:sc22g13@ecs.soton.ac.uk}{Stefan J. Collier}}
             {Stefan J. Collier}
            }
\addresses  {\groupname\\\deptname\\\univname}
\date       {\today}
\subject    {}
\keywords   {}
\supervisor {Dr. Maria Polukarov}
\examiner   {Professor Sheng Chen}

\maketitle
\begin{abstract}
This project aim was to model and analyse the effects of competitive pricing behaviors of grocery retailers on the British market. 

This was achieved by creating a multi-agent model, containing retailer and consumer agents. The heterogeneous crowd of retailers employs either a uniform pricing strategy or a ‘local price flexing’ strategy. The actions of these retailers are chosen by predicting the profit of each action, using a perceptron. Following on from the consideration of different economic models, a discrete model was developed so that software agents have a discrete environment to operate within. Within the model, it has been observed how supermarkets with differing behaviors affect a heterogeneous crowd of consumer agents. The model was implemented in Java with Python used to evaluate the results. 

The simulation displays good acceptance with real grocery market behavior, i.e. captures the performance of British retailers thus can be used to determine the impact of changes in their behavior on their competitors and consumers.Furthermore it can be used to provide insight into sustainability of volatile pricing strategies, providing a useful insight in volatility of British supermarket retail industry. 
\end{abstract}
\acknowledgements{
I would like to express my sincere gratitude to Dr Maria Polukarov for her guidance and support which provided me the freedom to take this research in the direction of my interest.\\
\\
I would also like to thank my family and friends for their encouragement and support. To those who quietly listened to my software complaints. To those who worked throughout the nights with me. To those who helped me write what I couldn't say. I cannot thank you enough.
}

\declaration{
I, Stefan Collier, declare that this dissertation and the work presented in it are my own and has been generated by me as the result of my own original research.\\
I confirm that:\\
1. This work was done wholly or mainly while in candidature for a degree at this University;\\
2. Where any part of this dissertation has previously been submitted for any other qualification at this University or any other institution, this has been clearly stated;\\
3. Where I have consulted the published work of others, this is always clearly attributed;\\
4. Where I have quoted from the work of others, the source is always given. With the exception of such quotations, this dissertation is entirely my own work;\\
5. I have acknowledged all main sources of help;\\
6. Where the thesis is based on work done by myself jointly with others, I have made clear exactly what was done by others and what I have contributed myself;\\
7. Either none of this work has been published before submission, or parts of this work have been published by :\\
\\
Stefan Collier\\
April 2016
}
\tableofcontents
\listoffigures
\listoftables

\mainmatter
%% ----------------------------------------------------------------
%\include{Introduction}
%\include{Conclusions}
\include{chapters/1Project/main}
\include{chapters/2Lit/main}
\include{chapters/3Design/HighLevel}
\include{chapters/3Design/InDepth}
\include{chapters/4Impl/main}

\include{chapters/5Experiments/1/main}
\include{chapters/5Experiments/2/main}
\include{chapters/5Experiments/3/main}
\include{chapters/5Experiments/4/main}

\include{chapters/6Conclusion/main}

\appendix
\include{appendix/AppendixB}
\include{appendix/D/main}
\include{appendix/AppendixC}

\backmatter
\bibliographystyle{ecs}
\bibliography{ECS}
\end{document}
%% ----------------------------------------------------------------

 %% ----------------------------------------------------------------
%% Progress.tex
%% ---------------------------------------------------------------- 
\documentclass{ecsprogress}    % Use the progress Style
\graphicspath{{../figs/}}   % Location of your graphics files
    \usepackage{natbib}            % Use Natbib style for the refs.
\hypersetup{colorlinks=true}   % Set to false for black/white printing
\input{Definitions}            % Include your abbreviations



\usepackage{enumitem}% http://ctan.org/pkg/enumitem
\usepackage{multirow}
\usepackage{float}
\usepackage{amsmath}
\usepackage{multicol}
\usepackage{amssymb}
\usepackage[normalem]{ulem}
\useunder{\uline}{\ul}{}
\usepackage{wrapfig}


\usepackage[table,xcdraw]{xcolor}


%% ----------------------------------------------------------------
\begin{document}
\frontmatter
\title      {Heterogeneous Agent-based Model for Supermarket Competition}
\authors    {\texorpdfstring
             {\href{mailto:sc22g13@ecs.soton.ac.uk}{Stefan J. Collier}}
             {Stefan J. Collier}
            }
\addresses  {\groupname\\\deptname\\\univname}
\date       {\today}
\subject    {}
\keywords   {}
\supervisor {Dr. Maria Polukarov}
\examiner   {Professor Sheng Chen}

\maketitle
\begin{abstract}
This project aim was to model and analyse the effects of competitive pricing behaviors of grocery retailers on the British market. 

This was achieved by creating a multi-agent model, containing retailer and consumer agents. The heterogeneous crowd of retailers employs either a uniform pricing strategy or a ‘local price flexing’ strategy. The actions of these retailers are chosen by predicting the profit of each action, using a perceptron. Following on from the consideration of different economic models, a discrete model was developed so that software agents have a discrete environment to operate within. Within the model, it has been observed how supermarkets with differing behaviors affect a heterogeneous crowd of consumer agents. The model was implemented in Java with Python used to evaluate the results. 

The simulation displays good acceptance with real grocery market behavior, i.e. captures the performance of British retailers thus can be used to determine the impact of changes in their behavior on their competitors and consumers.Furthermore it can be used to provide insight into sustainability of volatile pricing strategies, providing a useful insight in volatility of British supermarket retail industry. 
\end{abstract}
\acknowledgements{
I would like to express my sincere gratitude to Dr Maria Polukarov for her guidance and support which provided me the freedom to take this research in the direction of my interest.\\
\\
I would also like to thank my family and friends for their encouragement and support. To those who quietly listened to my software complaints. To those who worked throughout the nights with me. To those who helped me write what I couldn't say. I cannot thank you enough.
}

\declaration{
I, Stefan Collier, declare that this dissertation and the work presented in it are my own and has been generated by me as the result of my own original research.\\
I confirm that:\\
1. This work was done wholly or mainly while in candidature for a degree at this University;\\
2. Where any part of this dissertation has previously been submitted for any other qualification at this University or any other institution, this has been clearly stated;\\
3. Where I have consulted the published work of others, this is always clearly attributed;\\
4. Where I have quoted from the work of others, the source is always given. With the exception of such quotations, this dissertation is entirely my own work;\\
5. I have acknowledged all main sources of help;\\
6. Where the thesis is based on work done by myself jointly with others, I have made clear exactly what was done by others and what I have contributed myself;\\
7. Either none of this work has been published before submission, or parts of this work have been published by :\\
\\
Stefan Collier\\
April 2016
}
\tableofcontents
\listoffigures
\listoftables

\mainmatter
%% ----------------------------------------------------------------
%\include{Introduction}
%\include{Conclusions}
\include{chapters/1Project/main}
\include{chapters/2Lit/main}
\include{chapters/3Design/HighLevel}
\include{chapters/3Design/InDepth}
\include{chapters/4Impl/main}

\include{chapters/5Experiments/1/main}
\include{chapters/5Experiments/2/main}
\include{chapters/5Experiments/3/main}
\include{chapters/5Experiments/4/main}

\include{chapters/6Conclusion/main}

\appendix
\include{appendix/AppendixB}
\include{appendix/D/main}
\include{appendix/AppendixC}

\backmatter
\bibliographystyle{ecs}
\bibliography{ECS}
\end{document}
%% ----------------------------------------------------------------

\include{chapters/3Design/HighLevel}
\include{chapters/3Design/InDepth}
 %% ----------------------------------------------------------------
%% Progress.tex
%% ---------------------------------------------------------------- 
\documentclass{ecsprogress}    % Use the progress Style
\graphicspath{{../figs/}}   % Location of your graphics files
    \usepackage{natbib}            % Use Natbib style for the refs.
\hypersetup{colorlinks=true}   % Set to false for black/white printing
\input{Definitions}            % Include your abbreviations



\usepackage{enumitem}% http://ctan.org/pkg/enumitem
\usepackage{multirow}
\usepackage{float}
\usepackage{amsmath}
\usepackage{multicol}
\usepackage{amssymb}
\usepackage[normalem]{ulem}
\useunder{\uline}{\ul}{}
\usepackage{wrapfig}


\usepackage[table,xcdraw]{xcolor}


%% ----------------------------------------------------------------
\begin{document}
\frontmatter
\title      {Heterogeneous Agent-based Model for Supermarket Competition}
\authors    {\texorpdfstring
             {\href{mailto:sc22g13@ecs.soton.ac.uk}{Stefan J. Collier}}
             {Stefan J. Collier}
            }
\addresses  {\groupname\\\deptname\\\univname}
\date       {\today}
\subject    {}
\keywords   {}
\supervisor {Dr. Maria Polukarov}
\examiner   {Professor Sheng Chen}

\maketitle
\begin{abstract}
This project aim was to model and analyse the effects of competitive pricing behaviors of grocery retailers on the British market. 

This was achieved by creating a multi-agent model, containing retailer and consumer agents. The heterogeneous crowd of retailers employs either a uniform pricing strategy or a ‘local price flexing’ strategy. The actions of these retailers are chosen by predicting the profit of each action, using a perceptron. Following on from the consideration of different economic models, a discrete model was developed so that software agents have a discrete environment to operate within. Within the model, it has been observed how supermarkets with differing behaviors affect a heterogeneous crowd of consumer agents. The model was implemented in Java with Python used to evaluate the results. 

The simulation displays good acceptance with real grocery market behavior, i.e. captures the performance of British retailers thus can be used to determine the impact of changes in their behavior on their competitors and consumers.Furthermore it can be used to provide insight into sustainability of volatile pricing strategies, providing a useful insight in volatility of British supermarket retail industry. 
\end{abstract}
\acknowledgements{
I would like to express my sincere gratitude to Dr Maria Polukarov for her guidance and support which provided me the freedom to take this research in the direction of my interest.\\
\\
I would also like to thank my family and friends for their encouragement and support. To those who quietly listened to my software complaints. To those who worked throughout the nights with me. To those who helped me write what I couldn't say. I cannot thank you enough.
}

\declaration{
I, Stefan Collier, declare that this dissertation and the work presented in it are my own and has been generated by me as the result of my own original research.\\
I confirm that:\\
1. This work was done wholly or mainly while in candidature for a degree at this University;\\
2. Where any part of this dissertation has previously been submitted for any other qualification at this University or any other institution, this has been clearly stated;\\
3. Where I have consulted the published work of others, this is always clearly attributed;\\
4. Where I have quoted from the work of others, the source is always given. With the exception of such quotations, this dissertation is entirely my own work;\\
5. I have acknowledged all main sources of help;\\
6. Where the thesis is based on work done by myself jointly with others, I have made clear exactly what was done by others and what I have contributed myself;\\
7. Either none of this work has been published before submission, or parts of this work have been published by :\\
\\
Stefan Collier\\
April 2016
}
\tableofcontents
\listoffigures
\listoftables

\mainmatter
%% ----------------------------------------------------------------
%\include{Introduction}
%\include{Conclusions}
\include{chapters/1Project/main}
\include{chapters/2Lit/main}
\include{chapters/3Design/HighLevel}
\include{chapters/3Design/InDepth}
\include{chapters/4Impl/main}

\include{chapters/5Experiments/1/main}
\include{chapters/5Experiments/2/main}
\include{chapters/5Experiments/3/main}
\include{chapters/5Experiments/4/main}

\include{chapters/6Conclusion/main}

\appendix
\include{appendix/AppendixB}
\include{appendix/D/main}
\include{appendix/AppendixC}

\backmatter
\bibliographystyle{ecs}
\bibliography{ECS}
\end{document}
%% ----------------------------------------------------------------


 %% ----------------------------------------------------------------
%% Progress.tex
%% ---------------------------------------------------------------- 
\documentclass{ecsprogress}    % Use the progress Style
\graphicspath{{../figs/}}   % Location of your graphics files
    \usepackage{natbib}            % Use Natbib style for the refs.
\hypersetup{colorlinks=true}   % Set to false for black/white printing
\input{Definitions}            % Include your abbreviations



\usepackage{enumitem}% http://ctan.org/pkg/enumitem
\usepackage{multirow}
\usepackage{float}
\usepackage{amsmath}
\usepackage{multicol}
\usepackage{amssymb}
\usepackage[normalem]{ulem}
\useunder{\uline}{\ul}{}
\usepackage{wrapfig}


\usepackage[table,xcdraw]{xcolor}


%% ----------------------------------------------------------------
\begin{document}
\frontmatter
\title      {Heterogeneous Agent-based Model for Supermarket Competition}
\authors    {\texorpdfstring
             {\href{mailto:sc22g13@ecs.soton.ac.uk}{Stefan J. Collier}}
             {Stefan J. Collier}
            }
\addresses  {\groupname\\\deptname\\\univname}
\date       {\today}
\subject    {}
\keywords   {}
\supervisor {Dr. Maria Polukarov}
\examiner   {Professor Sheng Chen}

\maketitle
\begin{abstract}
This project aim was to model and analyse the effects of competitive pricing behaviors of grocery retailers on the British market. 

This was achieved by creating a multi-agent model, containing retailer and consumer agents. The heterogeneous crowd of retailers employs either a uniform pricing strategy or a ‘local price flexing’ strategy. The actions of these retailers are chosen by predicting the profit of each action, using a perceptron. Following on from the consideration of different economic models, a discrete model was developed so that software agents have a discrete environment to operate within. Within the model, it has been observed how supermarkets with differing behaviors affect a heterogeneous crowd of consumer agents. The model was implemented in Java with Python used to evaluate the results. 

The simulation displays good acceptance with real grocery market behavior, i.e. captures the performance of British retailers thus can be used to determine the impact of changes in their behavior on their competitors and consumers.Furthermore it can be used to provide insight into sustainability of volatile pricing strategies, providing a useful insight in volatility of British supermarket retail industry. 
\end{abstract}
\acknowledgements{
I would like to express my sincere gratitude to Dr Maria Polukarov for her guidance and support which provided me the freedom to take this research in the direction of my interest.\\
\\
I would also like to thank my family and friends for their encouragement and support. To those who quietly listened to my software complaints. To those who worked throughout the nights with me. To those who helped me write what I couldn't say. I cannot thank you enough.
}

\declaration{
I, Stefan Collier, declare that this dissertation and the work presented in it are my own and has been generated by me as the result of my own original research.\\
I confirm that:\\
1. This work was done wholly or mainly while in candidature for a degree at this University;\\
2. Where any part of this dissertation has previously been submitted for any other qualification at this University or any other institution, this has been clearly stated;\\
3. Where I have consulted the published work of others, this is always clearly attributed;\\
4. Where I have quoted from the work of others, the source is always given. With the exception of such quotations, this dissertation is entirely my own work;\\
5. I have acknowledged all main sources of help;\\
6. Where the thesis is based on work done by myself jointly with others, I have made clear exactly what was done by others and what I have contributed myself;\\
7. Either none of this work has been published before submission, or parts of this work have been published by :\\
\\
Stefan Collier\\
April 2016
}
\tableofcontents
\listoffigures
\listoftables

\mainmatter
%% ----------------------------------------------------------------
%\include{Introduction}
%\include{Conclusions}
\include{chapters/1Project/main}
\include{chapters/2Lit/main}
\include{chapters/3Design/HighLevel}
\include{chapters/3Design/InDepth}
\include{chapters/4Impl/main}

\include{chapters/5Experiments/1/main}
\include{chapters/5Experiments/2/main}
\include{chapters/5Experiments/3/main}
\include{chapters/5Experiments/4/main}

\include{chapters/6Conclusion/main}

\appendix
\include{appendix/AppendixB}
\include{appendix/D/main}
\include{appendix/AppendixC}

\backmatter
\bibliographystyle{ecs}
\bibliography{ECS}
\end{document}
%% ----------------------------------------------------------------

 %% ----------------------------------------------------------------
%% Progress.tex
%% ---------------------------------------------------------------- 
\documentclass{ecsprogress}    % Use the progress Style
\graphicspath{{../figs/}}   % Location of your graphics files
    \usepackage{natbib}            % Use Natbib style for the refs.
\hypersetup{colorlinks=true}   % Set to false for black/white printing
\input{Definitions}            % Include your abbreviations



\usepackage{enumitem}% http://ctan.org/pkg/enumitem
\usepackage{multirow}
\usepackage{float}
\usepackage{amsmath}
\usepackage{multicol}
\usepackage{amssymb}
\usepackage[normalem]{ulem}
\useunder{\uline}{\ul}{}
\usepackage{wrapfig}


\usepackage[table,xcdraw]{xcolor}


%% ----------------------------------------------------------------
\begin{document}
\frontmatter
\title      {Heterogeneous Agent-based Model for Supermarket Competition}
\authors    {\texorpdfstring
             {\href{mailto:sc22g13@ecs.soton.ac.uk}{Stefan J. Collier}}
             {Stefan J. Collier}
            }
\addresses  {\groupname\\\deptname\\\univname}
\date       {\today}
\subject    {}
\keywords   {}
\supervisor {Dr. Maria Polukarov}
\examiner   {Professor Sheng Chen}

\maketitle
\begin{abstract}
This project aim was to model and analyse the effects of competitive pricing behaviors of grocery retailers on the British market. 

This was achieved by creating a multi-agent model, containing retailer and consumer agents. The heterogeneous crowd of retailers employs either a uniform pricing strategy or a ‘local price flexing’ strategy. The actions of these retailers are chosen by predicting the profit of each action, using a perceptron. Following on from the consideration of different economic models, a discrete model was developed so that software agents have a discrete environment to operate within. Within the model, it has been observed how supermarkets with differing behaviors affect a heterogeneous crowd of consumer agents. The model was implemented in Java with Python used to evaluate the results. 

The simulation displays good acceptance with real grocery market behavior, i.e. captures the performance of British retailers thus can be used to determine the impact of changes in their behavior on their competitors and consumers.Furthermore it can be used to provide insight into sustainability of volatile pricing strategies, providing a useful insight in volatility of British supermarket retail industry. 
\end{abstract}
\acknowledgements{
I would like to express my sincere gratitude to Dr Maria Polukarov for her guidance and support which provided me the freedom to take this research in the direction of my interest.\\
\\
I would also like to thank my family and friends for their encouragement and support. To those who quietly listened to my software complaints. To those who worked throughout the nights with me. To those who helped me write what I couldn't say. I cannot thank you enough.
}

\declaration{
I, Stefan Collier, declare that this dissertation and the work presented in it are my own and has been generated by me as the result of my own original research.\\
I confirm that:\\
1. This work was done wholly or mainly while in candidature for a degree at this University;\\
2. Where any part of this dissertation has previously been submitted for any other qualification at this University or any other institution, this has been clearly stated;\\
3. Where I have consulted the published work of others, this is always clearly attributed;\\
4. Where I have quoted from the work of others, the source is always given. With the exception of such quotations, this dissertation is entirely my own work;\\
5. I have acknowledged all main sources of help;\\
6. Where the thesis is based on work done by myself jointly with others, I have made clear exactly what was done by others and what I have contributed myself;\\
7. Either none of this work has been published before submission, or parts of this work have been published by :\\
\\
Stefan Collier\\
April 2016
}
\tableofcontents
\listoffigures
\listoftables

\mainmatter
%% ----------------------------------------------------------------
%\include{Introduction}
%\include{Conclusions}
\include{chapters/1Project/main}
\include{chapters/2Lit/main}
\include{chapters/3Design/HighLevel}
\include{chapters/3Design/InDepth}
\include{chapters/4Impl/main}

\include{chapters/5Experiments/1/main}
\include{chapters/5Experiments/2/main}
\include{chapters/5Experiments/3/main}
\include{chapters/5Experiments/4/main}

\include{chapters/6Conclusion/main}

\appendix
\include{appendix/AppendixB}
\include{appendix/D/main}
\include{appendix/AppendixC}

\backmatter
\bibliographystyle{ecs}
\bibliography{ECS}
\end{document}
%% ----------------------------------------------------------------

 %% ----------------------------------------------------------------
%% Progress.tex
%% ---------------------------------------------------------------- 
\documentclass{ecsprogress}    % Use the progress Style
\graphicspath{{../figs/}}   % Location of your graphics files
    \usepackage{natbib}            % Use Natbib style for the refs.
\hypersetup{colorlinks=true}   % Set to false for black/white printing
\input{Definitions}            % Include your abbreviations



\usepackage{enumitem}% http://ctan.org/pkg/enumitem
\usepackage{multirow}
\usepackage{float}
\usepackage{amsmath}
\usepackage{multicol}
\usepackage{amssymb}
\usepackage[normalem]{ulem}
\useunder{\uline}{\ul}{}
\usepackage{wrapfig}


\usepackage[table,xcdraw]{xcolor}


%% ----------------------------------------------------------------
\begin{document}
\frontmatter
\title      {Heterogeneous Agent-based Model for Supermarket Competition}
\authors    {\texorpdfstring
             {\href{mailto:sc22g13@ecs.soton.ac.uk}{Stefan J. Collier}}
             {Stefan J. Collier}
            }
\addresses  {\groupname\\\deptname\\\univname}
\date       {\today}
\subject    {}
\keywords   {}
\supervisor {Dr. Maria Polukarov}
\examiner   {Professor Sheng Chen}

\maketitle
\begin{abstract}
This project aim was to model and analyse the effects of competitive pricing behaviors of grocery retailers on the British market. 

This was achieved by creating a multi-agent model, containing retailer and consumer agents. The heterogeneous crowd of retailers employs either a uniform pricing strategy or a ‘local price flexing’ strategy. The actions of these retailers are chosen by predicting the profit of each action, using a perceptron. Following on from the consideration of different economic models, a discrete model was developed so that software agents have a discrete environment to operate within. Within the model, it has been observed how supermarkets with differing behaviors affect a heterogeneous crowd of consumer agents. The model was implemented in Java with Python used to evaluate the results. 

The simulation displays good acceptance with real grocery market behavior, i.e. captures the performance of British retailers thus can be used to determine the impact of changes in their behavior on their competitors and consumers.Furthermore it can be used to provide insight into sustainability of volatile pricing strategies, providing a useful insight in volatility of British supermarket retail industry. 
\end{abstract}
\acknowledgements{
I would like to express my sincere gratitude to Dr Maria Polukarov for her guidance and support which provided me the freedom to take this research in the direction of my interest.\\
\\
I would also like to thank my family and friends for their encouragement and support. To those who quietly listened to my software complaints. To those who worked throughout the nights with me. To those who helped me write what I couldn't say. I cannot thank you enough.
}

\declaration{
I, Stefan Collier, declare that this dissertation and the work presented in it are my own and has been generated by me as the result of my own original research.\\
I confirm that:\\
1. This work was done wholly or mainly while in candidature for a degree at this University;\\
2. Where any part of this dissertation has previously been submitted for any other qualification at this University or any other institution, this has been clearly stated;\\
3. Where I have consulted the published work of others, this is always clearly attributed;\\
4. Where I have quoted from the work of others, the source is always given. With the exception of such quotations, this dissertation is entirely my own work;\\
5. I have acknowledged all main sources of help;\\
6. Where the thesis is based on work done by myself jointly with others, I have made clear exactly what was done by others and what I have contributed myself;\\
7. Either none of this work has been published before submission, or parts of this work have been published by :\\
\\
Stefan Collier\\
April 2016
}
\tableofcontents
\listoffigures
\listoftables

\mainmatter
%% ----------------------------------------------------------------
%\include{Introduction}
%\include{Conclusions}
\include{chapters/1Project/main}
\include{chapters/2Lit/main}
\include{chapters/3Design/HighLevel}
\include{chapters/3Design/InDepth}
\include{chapters/4Impl/main}

\include{chapters/5Experiments/1/main}
\include{chapters/5Experiments/2/main}
\include{chapters/5Experiments/3/main}
\include{chapters/5Experiments/4/main}

\include{chapters/6Conclusion/main}

\appendix
\include{appendix/AppendixB}
\include{appendix/D/main}
\include{appendix/AppendixC}

\backmatter
\bibliographystyle{ecs}
\bibliography{ECS}
\end{document}
%% ----------------------------------------------------------------

 %% ----------------------------------------------------------------
%% Progress.tex
%% ---------------------------------------------------------------- 
\documentclass{ecsprogress}    % Use the progress Style
\graphicspath{{../figs/}}   % Location of your graphics files
    \usepackage{natbib}            % Use Natbib style for the refs.
\hypersetup{colorlinks=true}   % Set to false for black/white printing
\input{Definitions}            % Include your abbreviations



\usepackage{enumitem}% http://ctan.org/pkg/enumitem
\usepackage{multirow}
\usepackage{float}
\usepackage{amsmath}
\usepackage{multicol}
\usepackage{amssymb}
\usepackage[normalem]{ulem}
\useunder{\uline}{\ul}{}
\usepackage{wrapfig}


\usepackage[table,xcdraw]{xcolor}


%% ----------------------------------------------------------------
\begin{document}
\frontmatter
\title      {Heterogeneous Agent-based Model for Supermarket Competition}
\authors    {\texorpdfstring
             {\href{mailto:sc22g13@ecs.soton.ac.uk}{Stefan J. Collier}}
             {Stefan J. Collier}
            }
\addresses  {\groupname\\\deptname\\\univname}
\date       {\today}
\subject    {}
\keywords   {}
\supervisor {Dr. Maria Polukarov}
\examiner   {Professor Sheng Chen}

\maketitle
\begin{abstract}
This project aim was to model and analyse the effects of competitive pricing behaviors of grocery retailers on the British market. 

This was achieved by creating a multi-agent model, containing retailer and consumer agents. The heterogeneous crowd of retailers employs either a uniform pricing strategy or a ‘local price flexing’ strategy. The actions of these retailers are chosen by predicting the profit of each action, using a perceptron. Following on from the consideration of different economic models, a discrete model was developed so that software agents have a discrete environment to operate within. Within the model, it has been observed how supermarkets with differing behaviors affect a heterogeneous crowd of consumer agents. The model was implemented in Java with Python used to evaluate the results. 

The simulation displays good acceptance with real grocery market behavior, i.e. captures the performance of British retailers thus can be used to determine the impact of changes in their behavior on their competitors and consumers.Furthermore it can be used to provide insight into sustainability of volatile pricing strategies, providing a useful insight in volatility of British supermarket retail industry. 
\end{abstract}
\acknowledgements{
I would like to express my sincere gratitude to Dr Maria Polukarov for her guidance and support which provided me the freedom to take this research in the direction of my interest.\\
\\
I would also like to thank my family and friends for their encouragement and support. To those who quietly listened to my software complaints. To those who worked throughout the nights with me. To those who helped me write what I couldn't say. I cannot thank you enough.
}

\declaration{
I, Stefan Collier, declare that this dissertation and the work presented in it are my own and has been generated by me as the result of my own original research.\\
I confirm that:\\
1. This work was done wholly or mainly while in candidature for a degree at this University;\\
2. Where any part of this dissertation has previously been submitted for any other qualification at this University or any other institution, this has been clearly stated;\\
3. Where I have consulted the published work of others, this is always clearly attributed;\\
4. Where I have quoted from the work of others, the source is always given. With the exception of such quotations, this dissertation is entirely my own work;\\
5. I have acknowledged all main sources of help;\\
6. Where the thesis is based on work done by myself jointly with others, I have made clear exactly what was done by others and what I have contributed myself;\\
7. Either none of this work has been published before submission, or parts of this work have been published by :\\
\\
Stefan Collier\\
April 2016
}
\tableofcontents
\listoffigures
\listoftables

\mainmatter
%% ----------------------------------------------------------------
%\include{Introduction}
%\include{Conclusions}
\include{chapters/1Project/main}
\include{chapters/2Lit/main}
\include{chapters/3Design/HighLevel}
\include{chapters/3Design/InDepth}
\include{chapters/4Impl/main}

\include{chapters/5Experiments/1/main}
\include{chapters/5Experiments/2/main}
\include{chapters/5Experiments/3/main}
\include{chapters/5Experiments/4/main}

\include{chapters/6Conclusion/main}

\appendix
\include{appendix/AppendixB}
\include{appendix/D/main}
\include{appendix/AppendixC}

\backmatter
\bibliographystyle{ecs}
\bibliography{ECS}
\end{document}
%% ----------------------------------------------------------------


 %% ----------------------------------------------------------------
%% Progress.tex
%% ---------------------------------------------------------------- 
\documentclass{ecsprogress}    % Use the progress Style
\graphicspath{{../figs/}}   % Location of your graphics files
    \usepackage{natbib}            % Use Natbib style for the refs.
\hypersetup{colorlinks=true}   % Set to false for black/white printing
\input{Definitions}            % Include your abbreviations



\usepackage{enumitem}% http://ctan.org/pkg/enumitem
\usepackage{multirow}
\usepackage{float}
\usepackage{amsmath}
\usepackage{multicol}
\usepackage{amssymb}
\usepackage[normalem]{ulem}
\useunder{\uline}{\ul}{}
\usepackage{wrapfig}


\usepackage[table,xcdraw]{xcolor}


%% ----------------------------------------------------------------
\begin{document}
\frontmatter
\title      {Heterogeneous Agent-based Model for Supermarket Competition}
\authors    {\texorpdfstring
             {\href{mailto:sc22g13@ecs.soton.ac.uk}{Stefan J. Collier}}
             {Stefan J. Collier}
            }
\addresses  {\groupname\\\deptname\\\univname}
\date       {\today}
\subject    {}
\keywords   {}
\supervisor {Dr. Maria Polukarov}
\examiner   {Professor Sheng Chen}

\maketitle
\begin{abstract}
This project aim was to model and analyse the effects of competitive pricing behaviors of grocery retailers on the British market. 

This was achieved by creating a multi-agent model, containing retailer and consumer agents. The heterogeneous crowd of retailers employs either a uniform pricing strategy or a ‘local price flexing’ strategy. The actions of these retailers are chosen by predicting the profit of each action, using a perceptron. Following on from the consideration of different economic models, a discrete model was developed so that software agents have a discrete environment to operate within. Within the model, it has been observed how supermarkets with differing behaviors affect a heterogeneous crowd of consumer agents. The model was implemented in Java with Python used to evaluate the results. 

The simulation displays good acceptance with real grocery market behavior, i.e. captures the performance of British retailers thus can be used to determine the impact of changes in their behavior on their competitors and consumers.Furthermore it can be used to provide insight into sustainability of volatile pricing strategies, providing a useful insight in volatility of British supermarket retail industry. 
\end{abstract}
\acknowledgements{
I would like to express my sincere gratitude to Dr Maria Polukarov for her guidance and support which provided me the freedom to take this research in the direction of my interest.\\
\\
I would also like to thank my family and friends for their encouragement and support. To those who quietly listened to my software complaints. To those who worked throughout the nights with me. To those who helped me write what I couldn't say. I cannot thank you enough.
}

\declaration{
I, Stefan Collier, declare that this dissertation and the work presented in it are my own and has been generated by me as the result of my own original research.\\
I confirm that:\\
1. This work was done wholly or mainly while in candidature for a degree at this University;\\
2. Where any part of this dissertation has previously been submitted for any other qualification at this University or any other institution, this has been clearly stated;\\
3. Where I have consulted the published work of others, this is always clearly attributed;\\
4. Where I have quoted from the work of others, the source is always given. With the exception of such quotations, this dissertation is entirely my own work;\\
5. I have acknowledged all main sources of help;\\
6. Where the thesis is based on work done by myself jointly with others, I have made clear exactly what was done by others and what I have contributed myself;\\
7. Either none of this work has been published before submission, or parts of this work have been published by :\\
\\
Stefan Collier\\
April 2016
}
\tableofcontents
\listoffigures
\listoftables

\mainmatter
%% ----------------------------------------------------------------
%\include{Introduction}
%\include{Conclusions}
\include{chapters/1Project/main}
\include{chapters/2Lit/main}
\include{chapters/3Design/HighLevel}
\include{chapters/3Design/InDepth}
\include{chapters/4Impl/main}

\include{chapters/5Experiments/1/main}
\include{chapters/5Experiments/2/main}
\include{chapters/5Experiments/3/main}
\include{chapters/5Experiments/4/main}

\include{chapters/6Conclusion/main}

\appendix
\include{appendix/AppendixB}
\include{appendix/D/main}
\include{appendix/AppendixC}

\backmatter
\bibliographystyle{ecs}
\bibliography{ECS}
\end{document}
%% ----------------------------------------------------------------


\appendix
\include{appendix/AppendixB}
 %% ----------------------------------------------------------------
%% Progress.tex
%% ---------------------------------------------------------------- 
\documentclass{ecsprogress}    % Use the progress Style
\graphicspath{{../figs/}}   % Location of your graphics files
    \usepackage{natbib}            % Use Natbib style for the refs.
\hypersetup{colorlinks=true}   % Set to false for black/white printing
\input{Definitions}            % Include your abbreviations



\usepackage{enumitem}% http://ctan.org/pkg/enumitem
\usepackage{multirow}
\usepackage{float}
\usepackage{amsmath}
\usepackage{multicol}
\usepackage{amssymb}
\usepackage[normalem]{ulem}
\useunder{\uline}{\ul}{}
\usepackage{wrapfig}


\usepackage[table,xcdraw]{xcolor}


%% ----------------------------------------------------------------
\begin{document}
\frontmatter
\title      {Heterogeneous Agent-based Model for Supermarket Competition}
\authors    {\texorpdfstring
             {\href{mailto:sc22g13@ecs.soton.ac.uk}{Stefan J. Collier}}
             {Stefan J. Collier}
            }
\addresses  {\groupname\\\deptname\\\univname}
\date       {\today}
\subject    {}
\keywords   {}
\supervisor {Dr. Maria Polukarov}
\examiner   {Professor Sheng Chen}

\maketitle
\begin{abstract}
This project aim was to model and analyse the effects of competitive pricing behaviors of grocery retailers on the British market. 

This was achieved by creating a multi-agent model, containing retailer and consumer agents. The heterogeneous crowd of retailers employs either a uniform pricing strategy or a ‘local price flexing’ strategy. The actions of these retailers are chosen by predicting the profit of each action, using a perceptron. Following on from the consideration of different economic models, a discrete model was developed so that software agents have a discrete environment to operate within. Within the model, it has been observed how supermarkets with differing behaviors affect a heterogeneous crowd of consumer agents. The model was implemented in Java with Python used to evaluate the results. 

The simulation displays good acceptance with real grocery market behavior, i.e. captures the performance of British retailers thus can be used to determine the impact of changes in their behavior on their competitors and consumers.Furthermore it can be used to provide insight into sustainability of volatile pricing strategies, providing a useful insight in volatility of British supermarket retail industry. 
\end{abstract}
\acknowledgements{
I would like to express my sincere gratitude to Dr Maria Polukarov for her guidance and support which provided me the freedom to take this research in the direction of my interest.\\
\\
I would also like to thank my family and friends for their encouragement and support. To those who quietly listened to my software complaints. To those who worked throughout the nights with me. To those who helped me write what I couldn't say. I cannot thank you enough.
}

\declaration{
I, Stefan Collier, declare that this dissertation and the work presented in it are my own and has been generated by me as the result of my own original research.\\
I confirm that:\\
1. This work was done wholly or mainly while in candidature for a degree at this University;\\
2. Where any part of this dissertation has previously been submitted for any other qualification at this University or any other institution, this has been clearly stated;\\
3. Where I have consulted the published work of others, this is always clearly attributed;\\
4. Where I have quoted from the work of others, the source is always given. With the exception of such quotations, this dissertation is entirely my own work;\\
5. I have acknowledged all main sources of help;\\
6. Where the thesis is based on work done by myself jointly with others, I have made clear exactly what was done by others and what I have contributed myself;\\
7. Either none of this work has been published before submission, or parts of this work have been published by :\\
\\
Stefan Collier\\
April 2016
}
\tableofcontents
\listoffigures
\listoftables

\mainmatter
%% ----------------------------------------------------------------
%\include{Introduction}
%\include{Conclusions}
\include{chapters/1Project/main}
\include{chapters/2Lit/main}
\include{chapters/3Design/HighLevel}
\include{chapters/3Design/InDepth}
\include{chapters/4Impl/main}

\include{chapters/5Experiments/1/main}
\include{chapters/5Experiments/2/main}
\include{chapters/5Experiments/3/main}
\include{chapters/5Experiments/4/main}

\include{chapters/6Conclusion/main}

\appendix
\include{appendix/AppendixB}
\include{appendix/D/main}
\include{appendix/AppendixC}

\backmatter
\bibliographystyle{ecs}
\bibliography{ECS}
\end{document}
%% ----------------------------------------------------------------

\include{appendix/AppendixC}

\backmatter
\bibliographystyle{ecs}
\bibliography{ECS}
\end{document}
%% ----------------------------------------------------------------

\include{chapters/3Design/HighLevel}
\include{chapters/3Design/InDepth}
 %% ----------------------------------------------------------------
%% Progress.tex
%% ---------------------------------------------------------------- 
\documentclass{ecsprogress}    % Use the progress Style
\graphicspath{{../figs/}}   % Location of your graphics files
    \usepackage{natbib}            % Use Natbib style for the refs.
\hypersetup{colorlinks=true}   % Set to false for black/white printing
\input{Definitions}            % Include your abbreviations



\usepackage{enumitem}% http://ctan.org/pkg/enumitem
\usepackage{multirow}
\usepackage{float}
\usepackage{amsmath}
\usepackage{multicol}
\usepackage{amssymb}
\usepackage[normalem]{ulem}
\useunder{\uline}{\ul}{}
\usepackage{wrapfig}


\usepackage[table,xcdraw]{xcolor}


%% ----------------------------------------------------------------
\begin{document}
\frontmatter
\title      {Heterogeneous Agent-based Model for Supermarket Competition}
\authors    {\texorpdfstring
             {\href{mailto:sc22g13@ecs.soton.ac.uk}{Stefan J. Collier}}
             {Stefan J. Collier}
            }
\addresses  {\groupname\\\deptname\\\univname}
\date       {\today}
\subject    {}
\keywords   {}
\supervisor {Dr. Maria Polukarov}
\examiner   {Professor Sheng Chen}

\maketitle
\begin{abstract}
This project aim was to model and analyse the effects of competitive pricing behaviors of grocery retailers on the British market. 

This was achieved by creating a multi-agent model, containing retailer and consumer agents. The heterogeneous crowd of retailers employs either a uniform pricing strategy or a ‘local price flexing’ strategy. The actions of these retailers are chosen by predicting the profit of each action, using a perceptron. Following on from the consideration of different economic models, a discrete model was developed so that software agents have a discrete environment to operate within. Within the model, it has been observed how supermarkets with differing behaviors affect a heterogeneous crowd of consumer agents. The model was implemented in Java with Python used to evaluate the results. 

The simulation displays good acceptance with real grocery market behavior, i.e. captures the performance of British retailers thus can be used to determine the impact of changes in their behavior on their competitors and consumers.Furthermore it can be used to provide insight into sustainability of volatile pricing strategies, providing a useful insight in volatility of British supermarket retail industry. 
\end{abstract}
\acknowledgements{
I would like to express my sincere gratitude to Dr Maria Polukarov for her guidance and support which provided me the freedom to take this research in the direction of my interest.\\
\\
I would also like to thank my family and friends for their encouragement and support. To those who quietly listened to my software complaints. To those who worked throughout the nights with me. To those who helped me write what I couldn't say. I cannot thank you enough.
}

\declaration{
I, Stefan Collier, declare that this dissertation and the work presented in it are my own and has been generated by me as the result of my own original research.\\
I confirm that:\\
1. This work was done wholly or mainly while in candidature for a degree at this University;\\
2. Where any part of this dissertation has previously been submitted for any other qualification at this University or any other institution, this has been clearly stated;\\
3. Where I have consulted the published work of others, this is always clearly attributed;\\
4. Where I have quoted from the work of others, the source is always given. With the exception of such quotations, this dissertation is entirely my own work;\\
5. I have acknowledged all main sources of help;\\
6. Where the thesis is based on work done by myself jointly with others, I have made clear exactly what was done by others and what I have contributed myself;\\
7. Either none of this work has been published before submission, or parts of this work have been published by :\\
\\
Stefan Collier\\
April 2016
}
\tableofcontents
\listoffigures
\listoftables

\mainmatter
%% ----------------------------------------------------------------
%\include{Introduction}
%\include{Conclusions}
 %% ----------------------------------------------------------------
%% Progress.tex
%% ---------------------------------------------------------------- 
\documentclass{ecsprogress}    % Use the progress Style
\graphicspath{{../figs/}}   % Location of your graphics files
    \usepackage{natbib}            % Use Natbib style for the refs.
\hypersetup{colorlinks=true}   % Set to false for black/white printing
\input{Definitions}            % Include your abbreviations



\usepackage{enumitem}% http://ctan.org/pkg/enumitem
\usepackage{multirow}
\usepackage{float}
\usepackage{amsmath}
\usepackage{multicol}
\usepackage{amssymb}
\usepackage[normalem]{ulem}
\useunder{\uline}{\ul}{}
\usepackage{wrapfig}


\usepackage[table,xcdraw]{xcolor}


%% ----------------------------------------------------------------
\begin{document}
\frontmatter
\title      {Heterogeneous Agent-based Model for Supermarket Competition}
\authors    {\texorpdfstring
             {\href{mailto:sc22g13@ecs.soton.ac.uk}{Stefan J. Collier}}
             {Stefan J. Collier}
            }
\addresses  {\groupname\\\deptname\\\univname}
\date       {\today}
\subject    {}
\keywords   {}
\supervisor {Dr. Maria Polukarov}
\examiner   {Professor Sheng Chen}

\maketitle
\begin{abstract}
This project aim was to model and analyse the effects of competitive pricing behaviors of grocery retailers on the British market. 

This was achieved by creating a multi-agent model, containing retailer and consumer agents. The heterogeneous crowd of retailers employs either a uniform pricing strategy or a ‘local price flexing’ strategy. The actions of these retailers are chosen by predicting the profit of each action, using a perceptron. Following on from the consideration of different economic models, a discrete model was developed so that software agents have a discrete environment to operate within. Within the model, it has been observed how supermarkets with differing behaviors affect a heterogeneous crowd of consumer agents. The model was implemented in Java with Python used to evaluate the results. 

The simulation displays good acceptance with real grocery market behavior, i.e. captures the performance of British retailers thus can be used to determine the impact of changes in their behavior on their competitors and consumers.Furthermore it can be used to provide insight into sustainability of volatile pricing strategies, providing a useful insight in volatility of British supermarket retail industry. 
\end{abstract}
\acknowledgements{
I would like to express my sincere gratitude to Dr Maria Polukarov for her guidance and support which provided me the freedom to take this research in the direction of my interest.\\
\\
I would also like to thank my family and friends for their encouragement and support. To those who quietly listened to my software complaints. To those who worked throughout the nights with me. To those who helped me write what I couldn't say. I cannot thank you enough.
}

\declaration{
I, Stefan Collier, declare that this dissertation and the work presented in it are my own and has been generated by me as the result of my own original research.\\
I confirm that:\\
1. This work was done wholly or mainly while in candidature for a degree at this University;\\
2. Where any part of this dissertation has previously been submitted for any other qualification at this University or any other institution, this has been clearly stated;\\
3. Where I have consulted the published work of others, this is always clearly attributed;\\
4. Where I have quoted from the work of others, the source is always given. With the exception of such quotations, this dissertation is entirely my own work;\\
5. I have acknowledged all main sources of help;\\
6. Where the thesis is based on work done by myself jointly with others, I have made clear exactly what was done by others and what I have contributed myself;\\
7. Either none of this work has been published before submission, or parts of this work have been published by :\\
\\
Stefan Collier\\
April 2016
}
\tableofcontents
\listoffigures
\listoftables

\mainmatter
%% ----------------------------------------------------------------
%\include{Introduction}
%\include{Conclusions}
\include{chapters/1Project/main}
\include{chapters/2Lit/main}
\include{chapters/3Design/HighLevel}
\include{chapters/3Design/InDepth}
\include{chapters/4Impl/main}

\include{chapters/5Experiments/1/main}
\include{chapters/5Experiments/2/main}
\include{chapters/5Experiments/3/main}
\include{chapters/5Experiments/4/main}

\include{chapters/6Conclusion/main}

\appendix
\include{appendix/AppendixB}
\include{appendix/D/main}
\include{appendix/AppendixC}

\backmatter
\bibliographystyle{ecs}
\bibliography{ECS}
\end{document}
%% ----------------------------------------------------------------

 %% ----------------------------------------------------------------
%% Progress.tex
%% ---------------------------------------------------------------- 
\documentclass{ecsprogress}    % Use the progress Style
\graphicspath{{../figs/}}   % Location of your graphics files
    \usepackage{natbib}            % Use Natbib style for the refs.
\hypersetup{colorlinks=true}   % Set to false for black/white printing
\input{Definitions}            % Include your abbreviations



\usepackage{enumitem}% http://ctan.org/pkg/enumitem
\usepackage{multirow}
\usepackage{float}
\usepackage{amsmath}
\usepackage{multicol}
\usepackage{amssymb}
\usepackage[normalem]{ulem}
\useunder{\uline}{\ul}{}
\usepackage{wrapfig}


\usepackage[table,xcdraw]{xcolor}


%% ----------------------------------------------------------------
\begin{document}
\frontmatter
\title      {Heterogeneous Agent-based Model for Supermarket Competition}
\authors    {\texorpdfstring
             {\href{mailto:sc22g13@ecs.soton.ac.uk}{Stefan J. Collier}}
             {Stefan J. Collier}
            }
\addresses  {\groupname\\\deptname\\\univname}
\date       {\today}
\subject    {}
\keywords   {}
\supervisor {Dr. Maria Polukarov}
\examiner   {Professor Sheng Chen}

\maketitle
\begin{abstract}
This project aim was to model and analyse the effects of competitive pricing behaviors of grocery retailers on the British market. 

This was achieved by creating a multi-agent model, containing retailer and consumer agents. The heterogeneous crowd of retailers employs either a uniform pricing strategy or a ‘local price flexing’ strategy. The actions of these retailers are chosen by predicting the profit of each action, using a perceptron. Following on from the consideration of different economic models, a discrete model was developed so that software agents have a discrete environment to operate within. Within the model, it has been observed how supermarkets with differing behaviors affect a heterogeneous crowd of consumer agents. The model was implemented in Java with Python used to evaluate the results. 

The simulation displays good acceptance with real grocery market behavior, i.e. captures the performance of British retailers thus can be used to determine the impact of changes in their behavior on their competitors and consumers.Furthermore it can be used to provide insight into sustainability of volatile pricing strategies, providing a useful insight in volatility of British supermarket retail industry. 
\end{abstract}
\acknowledgements{
I would like to express my sincere gratitude to Dr Maria Polukarov for her guidance and support which provided me the freedom to take this research in the direction of my interest.\\
\\
I would also like to thank my family and friends for their encouragement and support. To those who quietly listened to my software complaints. To those who worked throughout the nights with me. To those who helped me write what I couldn't say. I cannot thank you enough.
}

\declaration{
I, Stefan Collier, declare that this dissertation and the work presented in it are my own and has been generated by me as the result of my own original research.\\
I confirm that:\\
1. This work was done wholly or mainly while in candidature for a degree at this University;\\
2. Where any part of this dissertation has previously been submitted for any other qualification at this University or any other institution, this has been clearly stated;\\
3. Where I have consulted the published work of others, this is always clearly attributed;\\
4. Where I have quoted from the work of others, the source is always given. With the exception of such quotations, this dissertation is entirely my own work;\\
5. I have acknowledged all main sources of help;\\
6. Where the thesis is based on work done by myself jointly with others, I have made clear exactly what was done by others and what I have contributed myself;\\
7. Either none of this work has been published before submission, or parts of this work have been published by :\\
\\
Stefan Collier\\
April 2016
}
\tableofcontents
\listoffigures
\listoftables

\mainmatter
%% ----------------------------------------------------------------
%\include{Introduction}
%\include{Conclusions}
\include{chapters/1Project/main}
\include{chapters/2Lit/main}
\include{chapters/3Design/HighLevel}
\include{chapters/3Design/InDepth}
\include{chapters/4Impl/main}

\include{chapters/5Experiments/1/main}
\include{chapters/5Experiments/2/main}
\include{chapters/5Experiments/3/main}
\include{chapters/5Experiments/4/main}

\include{chapters/6Conclusion/main}

\appendix
\include{appendix/AppendixB}
\include{appendix/D/main}
\include{appendix/AppendixC}

\backmatter
\bibliographystyle{ecs}
\bibliography{ECS}
\end{document}
%% ----------------------------------------------------------------

\include{chapters/3Design/HighLevel}
\include{chapters/3Design/InDepth}
 %% ----------------------------------------------------------------
%% Progress.tex
%% ---------------------------------------------------------------- 
\documentclass{ecsprogress}    % Use the progress Style
\graphicspath{{../figs/}}   % Location of your graphics files
    \usepackage{natbib}            % Use Natbib style for the refs.
\hypersetup{colorlinks=true}   % Set to false for black/white printing
\input{Definitions}            % Include your abbreviations



\usepackage{enumitem}% http://ctan.org/pkg/enumitem
\usepackage{multirow}
\usepackage{float}
\usepackage{amsmath}
\usepackage{multicol}
\usepackage{amssymb}
\usepackage[normalem]{ulem}
\useunder{\uline}{\ul}{}
\usepackage{wrapfig}


\usepackage[table,xcdraw]{xcolor}


%% ----------------------------------------------------------------
\begin{document}
\frontmatter
\title      {Heterogeneous Agent-based Model for Supermarket Competition}
\authors    {\texorpdfstring
             {\href{mailto:sc22g13@ecs.soton.ac.uk}{Stefan J. Collier}}
             {Stefan J. Collier}
            }
\addresses  {\groupname\\\deptname\\\univname}
\date       {\today}
\subject    {}
\keywords   {}
\supervisor {Dr. Maria Polukarov}
\examiner   {Professor Sheng Chen}

\maketitle
\begin{abstract}
This project aim was to model and analyse the effects of competitive pricing behaviors of grocery retailers on the British market. 

This was achieved by creating a multi-agent model, containing retailer and consumer agents. The heterogeneous crowd of retailers employs either a uniform pricing strategy or a ‘local price flexing’ strategy. The actions of these retailers are chosen by predicting the profit of each action, using a perceptron. Following on from the consideration of different economic models, a discrete model was developed so that software agents have a discrete environment to operate within. Within the model, it has been observed how supermarkets with differing behaviors affect a heterogeneous crowd of consumer agents. The model was implemented in Java with Python used to evaluate the results. 

The simulation displays good acceptance with real grocery market behavior, i.e. captures the performance of British retailers thus can be used to determine the impact of changes in their behavior on their competitors and consumers.Furthermore it can be used to provide insight into sustainability of volatile pricing strategies, providing a useful insight in volatility of British supermarket retail industry. 
\end{abstract}
\acknowledgements{
I would like to express my sincere gratitude to Dr Maria Polukarov for her guidance and support which provided me the freedom to take this research in the direction of my interest.\\
\\
I would also like to thank my family and friends for their encouragement and support. To those who quietly listened to my software complaints. To those who worked throughout the nights with me. To those who helped me write what I couldn't say. I cannot thank you enough.
}

\declaration{
I, Stefan Collier, declare that this dissertation and the work presented in it are my own and has been generated by me as the result of my own original research.\\
I confirm that:\\
1. This work was done wholly or mainly while in candidature for a degree at this University;\\
2. Where any part of this dissertation has previously been submitted for any other qualification at this University or any other institution, this has been clearly stated;\\
3. Where I have consulted the published work of others, this is always clearly attributed;\\
4. Where I have quoted from the work of others, the source is always given. With the exception of such quotations, this dissertation is entirely my own work;\\
5. I have acknowledged all main sources of help;\\
6. Where the thesis is based on work done by myself jointly with others, I have made clear exactly what was done by others and what I have contributed myself;\\
7. Either none of this work has been published before submission, or parts of this work have been published by :\\
\\
Stefan Collier\\
April 2016
}
\tableofcontents
\listoffigures
\listoftables

\mainmatter
%% ----------------------------------------------------------------
%\include{Introduction}
%\include{Conclusions}
\include{chapters/1Project/main}
\include{chapters/2Lit/main}
\include{chapters/3Design/HighLevel}
\include{chapters/3Design/InDepth}
\include{chapters/4Impl/main}

\include{chapters/5Experiments/1/main}
\include{chapters/5Experiments/2/main}
\include{chapters/5Experiments/3/main}
\include{chapters/5Experiments/4/main}

\include{chapters/6Conclusion/main}

\appendix
\include{appendix/AppendixB}
\include{appendix/D/main}
\include{appendix/AppendixC}

\backmatter
\bibliographystyle{ecs}
\bibliography{ECS}
\end{document}
%% ----------------------------------------------------------------


 %% ----------------------------------------------------------------
%% Progress.tex
%% ---------------------------------------------------------------- 
\documentclass{ecsprogress}    % Use the progress Style
\graphicspath{{../figs/}}   % Location of your graphics files
    \usepackage{natbib}            % Use Natbib style for the refs.
\hypersetup{colorlinks=true}   % Set to false for black/white printing
\input{Definitions}            % Include your abbreviations



\usepackage{enumitem}% http://ctan.org/pkg/enumitem
\usepackage{multirow}
\usepackage{float}
\usepackage{amsmath}
\usepackage{multicol}
\usepackage{amssymb}
\usepackage[normalem]{ulem}
\useunder{\uline}{\ul}{}
\usepackage{wrapfig}


\usepackage[table,xcdraw]{xcolor}


%% ----------------------------------------------------------------
\begin{document}
\frontmatter
\title      {Heterogeneous Agent-based Model for Supermarket Competition}
\authors    {\texorpdfstring
             {\href{mailto:sc22g13@ecs.soton.ac.uk}{Stefan J. Collier}}
             {Stefan J. Collier}
            }
\addresses  {\groupname\\\deptname\\\univname}
\date       {\today}
\subject    {}
\keywords   {}
\supervisor {Dr. Maria Polukarov}
\examiner   {Professor Sheng Chen}

\maketitle
\begin{abstract}
This project aim was to model and analyse the effects of competitive pricing behaviors of grocery retailers on the British market. 

This was achieved by creating a multi-agent model, containing retailer and consumer agents. The heterogeneous crowd of retailers employs either a uniform pricing strategy or a ‘local price flexing’ strategy. The actions of these retailers are chosen by predicting the profit of each action, using a perceptron. Following on from the consideration of different economic models, a discrete model was developed so that software agents have a discrete environment to operate within. Within the model, it has been observed how supermarkets with differing behaviors affect a heterogeneous crowd of consumer agents. The model was implemented in Java with Python used to evaluate the results. 

The simulation displays good acceptance with real grocery market behavior, i.e. captures the performance of British retailers thus can be used to determine the impact of changes in their behavior on their competitors and consumers.Furthermore it can be used to provide insight into sustainability of volatile pricing strategies, providing a useful insight in volatility of British supermarket retail industry. 
\end{abstract}
\acknowledgements{
I would like to express my sincere gratitude to Dr Maria Polukarov for her guidance and support which provided me the freedom to take this research in the direction of my interest.\\
\\
I would also like to thank my family and friends for their encouragement and support. To those who quietly listened to my software complaints. To those who worked throughout the nights with me. To those who helped me write what I couldn't say. I cannot thank you enough.
}

\declaration{
I, Stefan Collier, declare that this dissertation and the work presented in it are my own and has been generated by me as the result of my own original research.\\
I confirm that:\\
1. This work was done wholly or mainly while in candidature for a degree at this University;\\
2. Where any part of this dissertation has previously been submitted for any other qualification at this University or any other institution, this has been clearly stated;\\
3. Where I have consulted the published work of others, this is always clearly attributed;\\
4. Where I have quoted from the work of others, the source is always given. With the exception of such quotations, this dissertation is entirely my own work;\\
5. I have acknowledged all main sources of help;\\
6. Where the thesis is based on work done by myself jointly with others, I have made clear exactly what was done by others and what I have contributed myself;\\
7. Either none of this work has been published before submission, or parts of this work have been published by :\\
\\
Stefan Collier\\
April 2016
}
\tableofcontents
\listoffigures
\listoftables

\mainmatter
%% ----------------------------------------------------------------
%\include{Introduction}
%\include{Conclusions}
\include{chapters/1Project/main}
\include{chapters/2Lit/main}
\include{chapters/3Design/HighLevel}
\include{chapters/3Design/InDepth}
\include{chapters/4Impl/main}

\include{chapters/5Experiments/1/main}
\include{chapters/5Experiments/2/main}
\include{chapters/5Experiments/3/main}
\include{chapters/5Experiments/4/main}

\include{chapters/6Conclusion/main}

\appendix
\include{appendix/AppendixB}
\include{appendix/D/main}
\include{appendix/AppendixC}

\backmatter
\bibliographystyle{ecs}
\bibliography{ECS}
\end{document}
%% ----------------------------------------------------------------

 %% ----------------------------------------------------------------
%% Progress.tex
%% ---------------------------------------------------------------- 
\documentclass{ecsprogress}    % Use the progress Style
\graphicspath{{../figs/}}   % Location of your graphics files
    \usepackage{natbib}            % Use Natbib style for the refs.
\hypersetup{colorlinks=true}   % Set to false for black/white printing
\input{Definitions}            % Include your abbreviations



\usepackage{enumitem}% http://ctan.org/pkg/enumitem
\usepackage{multirow}
\usepackage{float}
\usepackage{amsmath}
\usepackage{multicol}
\usepackage{amssymb}
\usepackage[normalem]{ulem}
\useunder{\uline}{\ul}{}
\usepackage{wrapfig}


\usepackage[table,xcdraw]{xcolor}


%% ----------------------------------------------------------------
\begin{document}
\frontmatter
\title      {Heterogeneous Agent-based Model for Supermarket Competition}
\authors    {\texorpdfstring
             {\href{mailto:sc22g13@ecs.soton.ac.uk}{Stefan J. Collier}}
             {Stefan J. Collier}
            }
\addresses  {\groupname\\\deptname\\\univname}
\date       {\today}
\subject    {}
\keywords   {}
\supervisor {Dr. Maria Polukarov}
\examiner   {Professor Sheng Chen}

\maketitle
\begin{abstract}
This project aim was to model and analyse the effects of competitive pricing behaviors of grocery retailers on the British market. 

This was achieved by creating a multi-agent model, containing retailer and consumer agents. The heterogeneous crowd of retailers employs either a uniform pricing strategy or a ‘local price flexing’ strategy. The actions of these retailers are chosen by predicting the profit of each action, using a perceptron. Following on from the consideration of different economic models, a discrete model was developed so that software agents have a discrete environment to operate within. Within the model, it has been observed how supermarkets with differing behaviors affect a heterogeneous crowd of consumer agents. The model was implemented in Java with Python used to evaluate the results. 

The simulation displays good acceptance with real grocery market behavior, i.e. captures the performance of British retailers thus can be used to determine the impact of changes in their behavior on their competitors and consumers.Furthermore it can be used to provide insight into sustainability of volatile pricing strategies, providing a useful insight in volatility of British supermarket retail industry. 
\end{abstract}
\acknowledgements{
I would like to express my sincere gratitude to Dr Maria Polukarov for her guidance and support which provided me the freedom to take this research in the direction of my interest.\\
\\
I would also like to thank my family and friends for their encouragement and support. To those who quietly listened to my software complaints. To those who worked throughout the nights with me. To those who helped me write what I couldn't say. I cannot thank you enough.
}

\declaration{
I, Stefan Collier, declare that this dissertation and the work presented in it are my own and has been generated by me as the result of my own original research.\\
I confirm that:\\
1. This work was done wholly or mainly while in candidature for a degree at this University;\\
2. Where any part of this dissertation has previously been submitted for any other qualification at this University or any other institution, this has been clearly stated;\\
3. Where I have consulted the published work of others, this is always clearly attributed;\\
4. Where I have quoted from the work of others, the source is always given. With the exception of such quotations, this dissertation is entirely my own work;\\
5. I have acknowledged all main sources of help;\\
6. Where the thesis is based on work done by myself jointly with others, I have made clear exactly what was done by others and what I have contributed myself;\\
7. Either none of this work has been published before submission, or parts of this work have been published by :\\
\\
Stefan Collier\\
April 2016
}
\tableofcontents
\listoffigures
\listoftables

\mainmatter
%% ----------------------------------------------------------------
%\include{Introduction}
%\include{Conclusions}
\include{chapters/1Project/main}
\include{chapters/2Lit/main}
\include{chapters/3Design/HighLevel}
\include{chapters/3Design/InDepth}
\include{chapters/4Impl/main}

\include{chapters/5Experiments/1/main}
\include{chapters/5Experiments/2/main}
\include{chapters/5Experiments/3/main}
\include{chapters/5Experiments/4/main}

\include{chapters/6Conclusion/main}

\appendix
\include{appendix/AppendixB}
\include{appendix/D/main}
\include{appendix/AppendixC}

\backmatter
\bibliographystyle{ecs}
\bibliography{ECS}
\end{document}
%% ----------------------------------------------------------------

 %% ----------------------------------------------------------------
%% Progress.tex
%% ---------------------------------------------------------------- 
\documentclass{ecsprogress}    % Use the progress Style
\graphicspath{{../figs/}}   % Location of your graphics files
    \usepackage{natbib}            % Use Natbib style for the refs.
\hypersetup{colorlinks=true}   % Set to false for black/white printing
\input{Definitions}            % Include your abbreviations



\usepackage{enumitem}% http://ctan.org/pkg/enumitem
\usepackage{multirow}
\usepackage{float}
\usepackage{amsmath}
\usepackage{multicol}
\usepackage{amssymb}
\usepackage[normalem]{ulem}
\useunder{\uline}{\ul}{}
\usepackage{wrapfig}


\usepackage[table,xcdraw]{xcolor}


%% ----------------------------------------------------------------
\begin{document}
\frontmatter
\title      {Heterogeneous Agent-based Model for Supermarket Competition}
\authors    {\texorpdfstring
             {\href{mailto:sc22g13@ecs.soton.ac.uk}{Stefan J. Collier}}
             {Stefan J. Collier}
            }
\addresses  {\groupname\\\deptname\\\univname}
\date       {\today}
\subject    {}
\keywords   {}
\supervisor {Dr. Maria Polukarov}
\examiner   {Professor Sheng Chen}

\maketitle
\begin{abstract}
This project aim was to model and analyse the effects of competitive pricing behaviors of grocery retailers on the British market. 

This was achieved by creating a multi-agent model, containing retailer and consumer agents. The heterogeneous crowd of retailers employs either a uniform pricing strategy or a ‘local price flexing’ strategy. The actions of these retailers are chosen by predicting the profit of each action, using a perceptron. Following on from the consideration of different economic models, a discrete model was developed so that software agents have a discrete environment to operate within. Within the model, it has been observed how supermarkets with differing behaviors affect a heterogeneous crowd of consumer agents. The model was implemented in Java with Python used to evaluate the results. 

The simulation displays good acceptance with real grocery market behavior, i.e. captures the performance of British retailers thus can be used to determine the impact of changes in their behavior on their competitors and consumers.Furthermore it can be used to provide insight into sustainability of volatile pricing strategies, providing a useful insight in volatility of British supermarket retail industry. 
\end{abstract}
\acknowledgements{
I would like to express my sincere gratitude to Dr Maria Polukarov for her guidance and support which provided me the freedom to take this research in the direction of my interest.\\
\\
I would also like to thank my family and friends for their encouragement and support. To those who quietly listened to my software complaints. To those who worked throughout the nights with me. To those who helped me write what I couldn't say. I cannot thank you enough.
}

\declaration{
I, Stefan Collier, declare that this dissertation and the work presented in it are my own and has been generated by me as the result of my own original research.\\
I confirm that:\\
1. This work was done wholly or mainly while in candidature for a degree at this University;\\
2. Where any part of this dissertation has previously been submitted for any other qualification at this University or any other institution, this has been clearly stated;\\
3. Where I have consulted the published work of others, this is always clearly attributed;\\
4. Where I have quoted from the work of others, the source is always given. With the exception of such quotations, this dissertation is entirely my own work;\\
5. I have acknowledged all main sources of help;\\
6. Where the thesis is based on work done by myself jointly with others, I have made clear exactly what was done by others and what I have contributed myself;\\
7. Either none of this work has been published before submission, or parts of this work have been published by :\\
\\
Stefan Collier\\
April 2016
}
\tableofcontents
\listoffigures
\listoftables

\mainmatter
%% ----------------------------------------------------------------
%\include{Introduction}
%\include{Conclusions}
\include{chapters/1Project/main}
\include{chapters/2Lit/main}
\include{chapters/3Design/HighLevel}
\include{chapters/3Design/InDepth}
\include{chapters/4Impl/main}

\include{chapters/5Experiments/1/main}
\include{chapters/5Experiments/2/main}
\include{chapters/5Experiments/3/main}
\include{chapters/5Experiments/4/main}

\include{chapters/6Conclusion/main}

\appendix
\include{appendix/AppendixB}
\include{appendix/D/main}
\include{appendix/AppendixC}

\backmatter
\bibliographystyle{ecs}
\bibliography{ECS}
\end{document}
%% ----------------------------------------------------------------

 %% ----------------------------------------------------------------
%% Progress.tex
%% ---------------------------------------------------------------- 
\documentclass{ecsprogress}    % Use the progress Style
\graphicspath{{../figs/}}   % Location of your graphics files
    \usepackage{natbib}            % Use Natbib style for the refs.
\hypersetup{colorlinks=true}   % Set to false for black/white printing
\input{Definitions}            % Include your abbreviations



\usepackage{enumitem}% http://ctan.org/pkg/enumitem
\usepackage{multirow}
\usepackage{float}
\usepackage{amsmath}
\usepackage{multicol}
\usepackage{amssymb}
\usepackage[normalem]{ulem}
\useunder{\uline}{\ul}{}
\usepackage{wrapfig}


\usepackage[table,xcdraw]{xcolor}


%% ----------------------------------------------------------------
\begin{document}
\frontmatter
\title      {Heterogeneous Agent-based Model for Supermarket Competition}
\authors    {\texorpdfstring
             {\href{mailto:sc22g13@ecs.soton.ac.uk}{Stefan J. Collier}}
             {Stefan J. Collier}
            }
\addresses  {\groupname\\\deptname\\\univname}
\date       {\today}
\subject    {}
\keywords   {}
\supervisor {Dr. Maria Polukarov}
\examiner   {Professor Sheng Chen}

\maketitle
\begin{abstract}
This project aim was to model and analyse the effects of competitive pricing behaviors of grocery retailers on the British market. 

This was achieved by creating a multi-agent model, containing retailer and consumer agents. The heterogeneous crowd of retailers employs either a uniform pricing strategy or a ‘local price flexing’ strategy. The actions of these retailers are chosen by predicting the profit of each action, using a perceptron. Following on from the consideration of different economic models, a discrete model was developed so that software agents have a discrete environment to operate within. Within the model, it has been observed how supermarkets with differing behaviors affect a heterogeneous crowd of consumer agents. The model was implemented in Java with Python used to evaluate the results. 

The simulation displays good acceptance with real grocery market behavior, i.e. captures the performance of British retailers thus can be used to determine the impact of changes in their behavior on their competitors and consumers.Furthermore it can be used to provide insight into sustainability of volatile pricing strategies, providing a useful insight in volatility of British supermarket retail industry. 
\end{abstract}
\acknowledgements{
I would like to express my sincere gratitude to Dr Maria Polukarov for her guidance and support which provided me the freedom to take this research in the direction of my interest.\\
\\
I would also like to thank my family and friends for their encouragement and support. To those who quietly listened to my software complaints. To those who worked throughout the nights with me. To those who helped me write what I couldn't say. I cannot thank you enough.
}

\declaration{
I, Stefan Collier, declare that this dissertation and the work presented in it are my own and has been generated by me as the result of my own original research.\\
I confirm that:\\
1. This work was done wholly or mainly while in candidature for a degree at this University;\\
2. Where any part of this dissertation has previously been submitted for any other qualification at this University or any other institution, this has been clearly stated;\\
3. Where I have consulted the published work of others, this is always clearly attributed;\\
4. Where I have quoted from the work of others, the source is always given. With the exception of such quotations, this dissertation is entirely my own work;\\
5. I have acknowledged all main sources of help;\\
6. Where the thesis is based on work done by myself jointly with others, I have made clear exactly what was done by others and what I have contributed myself;\\
7. Either none of this work has been published before submission, or parts of this work have been published by :\\
\\
Stefan Collier\\
April 2016
}
\tableofcontents
\listoffigures
\listoftables

\mainmatter
%% ----------------------------------------------------------------
%\include{Introduction}
%\include{Conclusions}
\include{chapters/1Project/main}
\include{chapters/2Lit/main}
\include{chapters/3Design/HighLevel}
\include{chapters/3Design/InDepth}
\include{chapters/4Impl/main}

\include{chapters/5Experiments/1/main}
\include{chapters/5Experiments/2/main}
\include{chapters/5Experiments/3/main}
\include{chapters/5Experiments/4/main}

\include{chapters/6Conclusion/main}

\appendix
\include{appendix/AppendixB}
\include{appendix/D/main}
\include{appendix/AppendixC}

\backmatter
\bibliographystyle{ecs}
\bibliography{ECS}
\end{document}
%% ----------------------------------------------------------------


 %% ----------------------------------------------------------------
%% Progress.tex
%% ---------------------------------------------------------------- 
\documentclass{ecsprogress}    % Use the progress Style
\graphicspath{{../figs/}}   % Location of your graphics files
    \usepackage{natbib}            % Use Natbib style for the refs.
\hypersetup{colorlinks=true}   % Set to false for black/white printing
\input{Definitions}            % Include your abbreviations



\usepackage{enumitem}% http://ctan.org/pkg/enumitem
\usepackage{multirow}
\usepackage{float}
\usepackage{amsmath}
\usepackage{multicol}
\usepackage{amssymb}
\usepackage[normalem]{ulem}
\useunder{\uline}{\ul}{}
\usepackage{wrapfig}


\usepackage[table,xcdraw]{xcolor}


%% ----------------------------------------------------------------
\begin{document}
\frontmatter
\title      {Heterogeneous Agent-based Model for Supermarket Competition}
\authors    {\texorpdfstring
             {\href{mailto:sc22g13@ecs.soton.ac.uk}{Stefan J. Collier}}
             {Stefan J. Collier}
            }
\addresses  {\groupname\\\deptname\\\univname}
\date       {\today}
\subject    {}
\keywords   {}
\supervisor {Dr. Maria Polukarov}
\examiner   {Professor Sheng Chen}

\maketitle
\begin{abstract}
This project aim was to model and analyse the effects of competitive pricing behaviors of grocery retailers on the British market. 

This was achieved by creating a multi-agent model, containing retailer and consumer agents. The heterogeneous crowd of retailers employs either a uniform pricing strategy or a ‘local price flexing’ strategy. The actions of these retailers are chosen by predicting the profit of each action, using a perceptron. Following on from the consideration of different economic models, a discrete model was developed so that software agents have a discrete environment to operate within. Within the model, it has been observed how supermarkets with differing behaviors affect a heterogeneous crowd of consumer agents. The model was implemented in Java with Python used to evaluate the results. 

The simulation displays good acceptance with real grocery market behavior, i.e. captures the performance of British retailers thus can be used to determine the impact of changes in their behavior on their competitors and consumers.Furthermore it can be used to provide insight into sustainability of volatile pricing strategies, providing a useful insight in volatility of British supermarket retail industry. 
\end{abstract}
\acknowledgements{
I would like to express my sincere gratitude to Dr Maria Polukarov for her guidance and support which provided me the freedom to take this research in the direction of my interest.\\
\\
I would also like to thank my family and friends for their encouragement and support. To those who quietly listened to my software complaints. To those who worked throughout the nights with me. To those who helped me write what I couldn't say. I cannot thank you enough.
}

\declaration{
I, Stefan Collier, declare that this dissertation and the work presented in it are my own and has been generated by me as the result of my own original research.\\
I confirm that:\\
1. This work was done wholly or mainly while in candidature for a degree at this University;\\
2. Where any part of this dissertation has previously been submitted for any other qualification at this University or any other institution, this has been clearly stated;\\
3. Where I have consulted the published work of others, this is always clearly attributed;\\
4. Where I have quoted from the work of others, the source is always given. With the exception of such quotations, this dissertation is entirely my own work;\\
5. I have acknowledged all main sources of help;\\
6. Where the thesis is based on work done by myself jointly with others, I have made clear exactly what was done by others and what I have contributed myself;\\
7. Either none of this work has been published before submission, or parts of this work have been published by :\\
\\
Stefan Collier\\
April 2016
}
\tableofcontents
\listoffigures
\listoftables

\mainmatter
%% ----------------------------------------------------------------
%\include{Introduction}
%\include{Conclusions}
\include{chapters/1Project/main}
\include{chapters/2Lit/main}
\include{chapters/3Design/HighLevel}
\include{chapters/3Design/InDepth}
\include{chapters/4Impl/main}

\include{chapters/5Experiments/1/main}
\include{chapters/5Experiments/2/main}
\include{chapters/5Experiments/3/main}
\include{chapters/5Experiments/4/main}

\include{chapters/6Conclusion/main}

\appendix
\include{appendix/AppendixB}
\include{appendix/D/main}
\include{appendix/AppendixC}

\backmatter
\bibliographystyle{ecs}
\bibliography{ECS}
\end{document}
%% ----------------------------------------------------------------


\appendix
\include{appendix/AppendixB}
 %% ----------------------------------------------------------------
%% Progress.tex
%% ---------------------------------------------------------------- 
\documentclass{ecsprogress}    % Use the progress Style
\graphicspath{{../figs/}}   % Location of your graphics files
    \usepackage{natbib}            % Use Natbib style for the refs.
\hypersetup{colorlinks=true}   % Set to false for black/white printing
\input{Definitions}            % Include your abbreviations



\usepackage{enumitem}% http://ctan.org/pkg/enumitem
\usepackage{multirow}
\usepackage{float}
\usepackage{amsmath}
\usepackage{multicol}
\usepackage{amssymb}
\usepackage[normalem]{ulem}
\useunder{\uline}{\ul}{}
\usepackage{wrapfig}


\usepackage[table,xcdraw]{xcolor}


%% ----------------------------------------------------------------
\begin{document}
\frontmatter
\title      {Heterogeneous Agent-based Model for Supermarket Competition}
\authors    {\texorpdfstring
             {\href{mailto:sc22g13@ecs.soton.ac.uk}{Stefan J. Collier}}
             {Stefan J. Collier}
            }
\addresses  {\groupname\\\deptname\\\univname}
\date       {\today}
\subject    {}
\keywords   {}
\supervisor {Dr. Maria Polukarov}
\examiner   {Professor Sheng Chen}

\maketitle
\begin{abstract}
This project aim was to model and analyse the effects of competitive pricing behaviors of grocery retailers on the British market. 

This was achieved by creating a multi-agent model, containing retailer and consumer agents. The heterogeneous crowd of retailers employs either a uniform pricing strategy or a ‘local price flexing’ strategy. The actions of these retailers are chosen by predicting the profit of each action, using a perceptron. Following on from the consideration of different economic models, a discrete model was developed so that software agents have a discrete environment to operate within. Within the model, it has been observed how supermarkets with differing behaviors affect a heterogeneous crowd of consumer agents. The model was implemented in Java with Python used to evaluate the results. 

The simulation displays good acceptance with real grocery market behavior, i.e. captures the performance of British retailers thus can be used to determine the impact of changes in their behavior on their competitors and consumers.Furthermore it can be used to provide insight into sustainability of volatile pricing strategies, providing a useful insight in volatility of British supermarket retail industry. 
\end{abstract}
\acknowledgements{
I would like to express my sincere gratitude to Dr Maria Polukarov for her guidance and support which provided me the freedom to take this research in the direction of my interest.\\
\\
I would also like to thank my family and friends for their encouragement and support. To those who quietly listened to my software complaints. To those who worked throughout the nights with me. To those who helped me write what I couldn't say. I cannot thank you enough.
}

\declaration{
I, Stefan Collier, declare that this dissertation and the work presented in it are my own and has been generated by me as the result of my own original research.\\
I confirm that:\\
1. This work was done wholly or mainly while in candidature for a degree at this University;\\
2. Where any part of this dissertation has previously been submitted for any other qualification at this University or any other institution, this has been clearly stated;\\
3. Where I have consulted the published work of others, this is always clearly attributed;\\
4. Where I have quoted from the work of others, the source is always given. With the exception of such quotations, this dissertation is entirely my own work;\\
5. I have acknowledged all main sources of help;\\
6. Where the thesis is based on work done by myself jointly with others, I have made clear exactly what was done by others and what I have contributed myself;\\
7. Either none of this work has been published before submission, or parts of this work have been published by :\\
\\
Stefan Collier\\
April 2016
}
\tableofcontents
\listoffigures
\listoftables

\mainmatter
%% ----------------------------------------------------------------
%\include{Introduction}
%\include{Conclusions}
\include{chapters/1Project/main}
\include{chapters/2Lit/main}
\include{chapters/3Design/HighLevel}
\include{chapters/3Design/InDepth}
\include{chapters/4Impl/main}

\include{chapters/5Experiments/1/main}
\include{chapters/5Experiments/2/main}
\include{chapters/5Experiments/3/main}
\include{chapters/5Experiments/4/main}

\include{chapters/6Conclusion/main}

\appendix
\include{appendix/AppendixB}
\include{appendix/D/main}
\include{appendix/AppendixC}

\backmatter
\bibliographystyle{ecs}
\bibliography{ECS}
\end{document}
%% ----------------------------------------------------------------

\include{appendix/AppendixC}

\backmatter
\bibliographystyle{ecs}
\bibliography{ECS}
\end{document}
%% ----------------------------------------------------------------


 %% ----------------------------------------------------------------
%% Progress.tex
%% ---------------------------------------------------------------- 
\documentclass{ecsprogress}    % Use the progress Style
\graphicspath{{../figs/}}   % Location of your graphics files
    \usepackage{natbib}            % Use Natbib style for the refs.
\hypersetup{colorlinks=true}   % Set to false for black/white printing
\input{Definitions}            % Include your abbreviations



\usepackage{enumitem}% http://ctan.org/pkg/enumitem
\usepackage{multirow}
\usepackage{float}
\usepackage{amsmath}
\usepackage{multicol}
\usepackage{amssymb}
\usepackage[normalem]{ulem}
\useunder{\uline}{\ul}{}
\usepackage{wrapfig}


\usepackage[table,xcdraw]{xcolor}


%% ----------------------------------------------------------------
\begin{document}
\frontmatter
\title      {Heterogeneous Agent-based Model for Supermarket Competition}
\authors    {\texorpdfstring
             {\href{mailto:sc22g13@ecs.soton.ac.uk}{Stefan J. Collier}}
             {Stefan J. Collier}
            }
\addresses  {\groupname\\\deptname\\\univname}
\date       {\today}
\subject    {}
\keywords   {}
\supervisor {Dr. Maria Polukarov}
\examiner   {Professor Sheng Chen}

\maketitle
\begin{abstract}
This project aim was to model and analyse the effects of competitive pricing behaviors of grocery retailers on the British market. 

This was achieved by creating a multi-agent model, containing retailer and consumer agents. The heterogeneous crowd of retailers employs either a uniform pricing strategy or a ‘local price flexing’ strategy. The actions of these retailers are chosen by predicting the profit of each action, using a perceptron. Following on from the consideration of different economic models, a discrete model was developed so that software agents have a discrete environment to operate within. Within the model, it has been observed how supermarkets with differing behaviors affect a heterogeneous crowd of consumer agents. The model was implemented in Java with Python used to evaluate the results. 

The simulation displays good acceptance with real grocery market behavior, i.e. captures the performance of British retailers thus can be used to determine the impact of changes in their behavior on their competitors and consumers.Furthermore it can be used to provide insight into sustainability of volatile pricing strategies, providing a useful insight in volatility of British supermarket retail industry. 
\end{abstract}
\acknowledgements{
I would like to express my sincere gratitude to Dr Maria Polukarov for her guidance and support which provided me the freedom to take this research in the direction of my interest.\\
\\
I would also like to thank my family and friends for their encouragement and support. To those who quietly listened to my software complaints. To those who worked throughout the nights with me. To those who helped me write what I couldn't say. I cannot thank you enough.
}

\declaration{
I, Stefan Collier, declare that this dissertation and the work presented in it are my own and has been generated by me as the result of my own original research.\\
I confirm that:\\
1. This work was done wholly or mainly while in candidature for a degree at this University;\\
2. Where any part of this dissertation has previously been submitted for any other qualification at this University or any other institution, this has been clearly stated;\\
3. Where I have consulted the published work of others, this is always clearly attributed;\\
4. Where I have quoted from the work of others, the source is always given. With the exception of such quotations, this dissertation is entirely my own work;\\
5. I have acknowledged all main sources of help;\\
6. Where the thesis is based on work done by myself jointly with others, I have made clear exactly what was done by others and what I have contributed myself;\\
7. Either none of this work has been published before submission, or parts of this work have been published by :\\
\\
Stefan Collier\\
April 2016
}
\tableofcontents
\listoffigures
\listoftables

\mainmatter
%% ----------------------------------------------------------------
%\include{Introduction}
%\include{Conclusions}
 %% ----------------------------------------------------------------
%% Progress.tex
%% ---------------------------------------------------------------- 
\documentclass{ecsprogress}    % Use the progress Style
\graphicspath{{../figs/}}   % Location of your graphics files
    \usepackage{natbib}            % Use Natbib style for the refs.
\hypersetup{colorlinks=true}   % Set to false for black/white printing
\input{Definitions}            % Include your abbreviations



\usepackage{enumitem}% http://ctan.org/pkg/enumitem
\usepackage{multirow}
\usepackage{float}
\usepackage{amsmath}
\usepackage{multicol}
\usepackage{amssymb}
\usepackage[normalem]{ulem}
\useunder{\uline}{\ul}{}
\usepackage{wrapfig}


\usepackage[table,xcdraw]{xcolor}


%% ----------------------------------------------------------------
\begin{document}
\frontmatter
\title      {Heterogeneous Agent-based Model for Supermarket Competition}
\authors    {\texorpdfstring
             {\href{mailto:sc22g13@ecs.soton.ac.uk}{Stefan J. Collier}}
             {Stefan J. Collier}
            }
\addresses  {\groupname\\\deptname\\\univname}
\date       {\today}
\subject    {}
\keywords   {}
\supervisor {Dr. Maria Polukarov}
\examiner   {Professor Sheng Chen}

\maketitle
\begin{abstract}
This project aim was to model and analyse the effects of competitive pricing behaviors of grocery retailers on the British market. 

This was achieved by creating a multi-agent model, containing retailer and consumer agents. The heterogeneous crowd of retailers employs either a uniform pricing strategy or a ‘local price flexing’ strategy. The actions of these retailers are chosen by predicting the profit of each action, using a perceptron. Following on from the consideration of different economic models, a discrete model was developed so that software agents have a discrete environment to operate within. Within the model, it has been observed how supermarkets with differing behaviors affect a heterogeneous crowd of consumer agents. The model was implemented in Java with Python used to evaluate the results. 

The simulation displays good acceptance with real grocery market behavior, i.e. captures the performance of British retailers thus can be used to determine the impact of changes in their behavior on their competitors and consumers.Furthermore it can be used to provide insight into sustainability of volatile pricing strategies, providing a useful insight in volatility of British supermarket retail industry. 
\end{abstract}
\acknowledgements{
I would like to express my sincere gratitude to Dr Maria Polukarov for her guidance and support which provided me the freedom to take this research in the direction of my interest.\\
\\
I would also like to thank my family and friends for their encouragement and support. To those who quietly listened to my software complaints. To those who worked throughout the nights with me. To those who helped me write what I couldn't say. I cannot thank you enough.
}

\declaration{
I, Stefan Collier, declare that this dissertation and the work presented in it are my own and has been generated by me as the result of my own original research.\\
I confirm that:\\
1. This work was done wholly or mainly while in candidature for a degree at this University;\\
2. Where any part of this dissertation has previously been submitted for any other qualification at this University or any other institution, this has been clearly stated;\\
3. Where I have consulted the published work of others, this is always clearly attributed;\\
4. Where I have quoted from the work of others, the source is always given. With the exception of such quotations, this dissertation is entirely my own work;\\
5. I have acknowledged all main sources of help;\\
6. Where the thesis is based on work done by myself jointly with others, I have made clear exactly what was done by others and what I have contributed myself;\\
7. Either none of this work has been published before submission, or parts of this work have been published by :\\
\\
Stefan Collier\\
April 2016
}
\tableofcontents
\listoffigures
\listoftables

\mainmatter
%% ----------------------------------------------------------------
%\include{Introduction}
%\include{Conclusions}
\include{chapters/1Project/main}
\include{chapters/2Lit/main}
\include{chapters/3Design/HighLevel}
\include{chapters/3Design/InDepth}
\include{chapters/4Impl/main}

\include{chapters/5Experiments/1/main}
\include{chapters/5Experiments/2/main}
\include{chapters/5Experiments/3/main}
\include{chapters/5Experiments/4/main}

\include{chapters/6Conclusion/main}

\appendix
\include{appendix/AppendixB}
\include{appendix/D/main}
\include{appendix/AppendixC}

\backmatter
\bibliographystyle{ecs}
\bibliography{ECS}
\end{document}
%% ----------------------------------------------------------------

 %% ----------------------------------------------------------------
%% Progress.tex
%% ---------------------------------------------------------------- 
\documentclass{ecsprogress}    % Use the progress Style
\graphicspath{{../figs/}}   % Location of your graphics files
    \usepackage{natbib}            % Use Natbib style for the refs.
\hypersetup{colorlinks=true}   % Set to false for black/white printing
\input{Definitions}            % Include your abbreviations



\usepackage{enumitem}% http://ctan.org/pkg/enumitem
\usepackage{multirow}
\usepackage{float}
\usepackage{amsmath}
\usepackage{multicol}
\usepackage{amssymb}
\usepackage[normalem]{ulem}
\useunder{\uline}{\ul}{}
\usepackage{wrapfig}


\usepackage[table,xcdraw]{xcolor}


%% ----------------------------------------------------------------
\begin{document}
\frontmatter
\title      {Heterogeneous Agent-based Model for Supermarket Competition}
\authors    {\texorpdfstring
             {\href{mailto:sc22g13@ecs.soton.ac.uk}{Stefan J. Collier}}
             {Stefan J. Collier}
            }
\addresses  {\groupname\\\deptname\\\univname}
\date       {\today}
\subject    {}
\keywords   {}
\supervisor {Dr. Maria Polukarov}
\examiner   {Professor Sheng Chen}

\maketitle
\begin{abstract}
This project aim was to model and analyse the effects of competitive pricing behaviors of grocery retailers on the British market. 

This was achieved by creating a multi-agent model, containing retailer and consumer agents. The heterogeneous crowd of retailers employs either a uniform pricing strategy or a ‘local price flexing’ strategy. The actions of these retailers are chosen by predicting the profit of each action, using a perceptron. Following on from the consideration of different economic models, a discrete model was developed so that software agents have a discrete environment to operate within. Within the model, it has been observed how supermarkets with differing behaviors affect a heterogeneous crowd of consumer agents. The model was implemented in Java with Python used to evaluate the results. 

The simulation displays good acceptance with real grocery market behavior, i.e. captures the performance of British retailers thus can be used to determine the impact of changes in their behavior on their competitors and consumers.Furthermore it can be used to provide insight into sustainability of volatile pricing strategies, providing a useful insight in volatility of British supermarket retail industry. 
\end{abstract}
\acknowledgements{
I would like to express my sincere gratitude to Dr Maria Polukarov for her guidance and support which provided me the freedom to take this research in the direction of my interest.\\
\\
I would also like to thank my family and friends for their encouragement and support. To those who quietly listened to my software complaints. To those who worked throughout the nights with me. To those who helped me write what I couldn't say. I cannot thank you enough.
}

\declaration{
I, Stefan Collier, declare that this dissertation and the work presented in it are my own and has been generated by me as the result of my own original research.\\
I confirm that:\\
1. This work was done wholly or mainly while in candidature for a degree at this University;\\
2. Where any part of this dissertation has previously been submitted for any other qualification at this University or any other institution, this has been clearly stated;\\
3. Where I have consulted the published work of others, this is always clearly attributed;\\
4. Where I have quoted from the work of others, the source is always given. With the exception of such quotations, this dissertation is entirely my own work;\\
5. I have acknowledged all main sources of help;\\
6. Where the thesis is based on work done by myself jointly with others, I have made clear exactly what was done by others and what I have contributed myself;\\
7. Either none of this work has been published before submission, or parts of this work have been published by :\\
\\
Stefan Collier\\
April 2016
}
\tableofcontents
\listoffigures
\listoftables

\mainmatter
%% ----------------------------------------------------------------
%\include{Introduction}
%\include{Conclusions}
\include{chapters/1Project/main}
\include{chapters/2Lit/main}
\include{chapters/3Design/HighLevel}
\include{chapters/3Design/InDepth}
\include{chapters/4Impl/main}

\include{chapters/5Experiments/1/main}
\include{chapters/5Experiments/2/main}
\include{chapters/5Experiments/3/main}
\include{chapters/5Experiments/4/main}

\include{chapters/6Conclusion/main}

\appendix
\include{appendix/AppendixB}
\include{appendix/D/main}
\include{appendix/AppendixC}

\backmatter
\bibliographystyle{ecs}
\bibliography{ECS}
\end{document}
%% ----------------------------------------------------------------

\include{chapters/3Design/HighLevel}
\include{chapters/3Design/InDepth}
 %% ----------------------------------------------------------------
%% Progress.tex
%% ---------------------------------------------------------------- 
\documentclass{ecsprogress}    % Use the progress Style
\graphicspath{{../figs/}}   % Location of your graphics files
    \usepackage{natbib}            % Use Natbib style for the refs.
\hypersetup{colorlinks=true}   % Set to false for black/white printing
\input{Definitions}            % Include your abbreviations



\usepackage{enumitem}% http://ctan.org/pkg/enumitem
\usepackage{multirow}
\usepackage{float}
\usepackage{amsmath}
\usepackage{multicol}
\usepackage{amssymb}
\usepackage[normalem]{ulem}
\useunder{\uline}{\ul}{}
\usepackage{wrapfig}


\usepackage[table,xcdraw]{xcolor}


%% ----------------------------------------------------------------
\begin{document}
\frontmatter
\title      {Heterogeneous Agent-based Model for Supermarket Competition}
\authors    {\texorpdfstring
             {\href{mailto:sc22g13@ecs.soton.ac.uk}{Stefan J. Collier}}
             {Stefan J. Collier}
            }
\addresses  {\groupname\\\deptname\\\univname}
\date       {\today}
\subject    {}
\keywords   {}
\supervisor {Dr. Maria Polukarov}
\examiner   {Professor Sheng Chen}

\maketitle
\begin{abstract}
This project aim was to model and analyse the effects of competitive pricing behaviors of grocery retailers on the British market. 

This was achieved by creating a multi-agent model, containing retailer and consumer agents. The heterogeneous crowd of retailers employs either a uniform pricing strategy or a ‘local price flexing’ strategy. The actions of these retailers are chosen by predicting the profit of each action, using a perceptron. Following on from the consideration of different economic models, a discrete model was developed so that software agents have a discrete environment to operate within. Within the model, it has been observed how supermarkets with differing behaviors affect a heterogeneous crowd of consumer agents. The model was implemented in Java with Python used to evaluate the results. 

The simulation displays good acceptance with real grocery market behavior, i.e. captures the performance of British retailers thus can be used to determine the impact of changes in their behavior on their competitors and consumers.Furthermore it can be used to provide insight into sustainability of volatile pricing strategies, providing a useful insight in volatility of British supermarket retail industry. 
\end{abstract}
\acknowledgements{
I would like to express my sincere gratitude to Dr Maria Polukarov for her guidance and support which provided me the freedom to take this research in the direction of my interest.\\
\\
I would also like to thank my family and friends for their encouragement and support. To those who quietly listened to my software complaints. To those who worked throughout the nights with me. To those who helped me write what I couldn't say. I cannot thank you enough.
}

\declaration{
I, Stefan Collier, declare that this dissertation and the work presented in it are my own and has been generated by me as the result of my own original research.\\
I confirm that:\\
1. This work was done wholly or mainly while in candidature for a degree at this University;\\
2. Where any part of this dissertation has previously been submitted for any other qualification at this University or any other institution, this has been clearly stated;\\
3. Where I have consulted the published work of others, this is always clearly attributed;\\
4. Where I have quoted from the work of others, the source is always given. With the exception of such quotations, this dissertation is entirely my own work;\\
5. I have acknowledged all main sources of help;\\
6. Where the thesis is based on work done by myself jointly with others, I have made clear exactly what was done by others and what I have contributed myself;\\
7. Either none of this work has been published before submission, or parts of this work have been published by :\\
\\
Stefan Collier\\
April 2016
}
\tableofcontents
\listoffigures
\listoftables

\mainmatter
%% ----------------------------------------------------------------
%\include{Introduction}
%\include{Conclusions}
\include{chapters/1Project/main}
\include{chapters/2Lit/main}
\include{chapters/3Design/HighLevel}
\include{chapters/3Design/InDepth}
\include{chapters/4Impl/main}

\include{chapters/5Experiments/1/main}
\include{chapters/5Experiments/2/main}
\include{chapters/5Experiments/3/main}
\include{chapters/5Experiments/4/main}

\include{chapters/6Conclusion/main}

\appendix
\include{appendix/AppendixB}
\include{appendix/D/main}
\include{appendix/AppendixC}

\backmatter
\bibliographystyle{ecs}
\bibliography{ECS}
\end{document}
%% ----------------------------------------------------------------


 %% ----------------------------------------------------------------
%% Progress.tex
%% ---------------------------------------------------------------- 
\documentclass{ecsprogress}    % Use the progress Style
\graphicspath{{../figs/}}   % Location of your graphics files
    \usepackage{natbib}            % Use Natbib style for the refs.
\hypersetup{colorlinks=true}   % Set to false for black/white printing
\input{Definitions}            % Include your abbreviations



\usepackage{enumitem}% http://ctan.org/pkg/enumitem
\usepackage{multirow}
\usepackage{float}
\usepackage{amsmath}
\usepackage{multicol}
\usepackage{amssymb}
\usepackage[normalem]{ulem}
\useunder{\uline}{\ul}{}
\usepackage{wrapfig}


\usepackage[table,xcdraw]{xcolor}


%% ----------------------------------------------------------------
\begin{document}
\frontmatter
\title      {Heterogeneous Agent-based Model for Supermarket Competition}
\authors    {\texorpdfstring
             {\href{mailto:sc22g13@ecs.soton.ac.uk}{Stefan J. Collier}}
             {Stefan J. Collier}
            }
\addresses  {\groupname\\\deptname\\\univname}
\date       {\today}
\subject    {}
\keywords   {}
\supervisor {Dr. Maria Polukarov}
\examiner   {Professor Sheng Chen}

\maketitle
\begin{abstract}
This project aim was to model and analyse the effects of competitive pricing behaviors of grocery retailers on the British market. 

This was achieved by creating a multi-agent model, containing retailer and consumer agents. The heterogeneous crowd of retailers employs either a uniform pricing strategy or a ‘local price flexing’ strategy. The actions of these retailers are chosen by predicting the profit of each action, using a perceptron. Following on from the consideration of different economic models, a discrete model was developed so that software agents have a discrete environment to operate within. Within the model, it has been observed how supermarkets with differing behaviors affect a heterogeneous crowd of consumer agents. The model was implemented in Java with Python used to evaluate the results. 

The simulation displays good acceptance with real grocery market behavior, i.e. captures the performance of British retailers thus can be used to determine the impact of changes in their behavior on their competitors and consumers.Furthermore it can be used to provide insight into sustainability of volatile pricing strategies, providing a useful insight in volatility of British supermarket retail industry. 
\end{abstract}
\acknowledgements{
I would like to express my sincere gratitude to Dr Maria Polukarov for her guidance and support which provided me the freedom to take this research in the direction of my interest.\\
\\
I would also like to thank my family and friends for their encouragement and support. To those who quietly listened to my software complaints. To those who worked throughout the nights with me. To those who helped me write what I couldn't say. I cannot thank you enough.
}

\declaration{
I, Stefan Collier, declare that this dissertation and the work presented in it are my own and has been generated by me as the result of my own original research.\\
I confirm that:\\
1. This work was done wholly or mainly while in candidature for a degree at this University;\\
2. Where any part of this dissertation has previously been submitted for any other qualification at this University or any other institution, this has been clearly stated;\\
3. Where I have consulted the published work of others, this is always clearly attributed;\\
4. Where I have quoted from the work of others, the source is always given. With the exception of such quotations, this dissertation is entirely my own work;\\
5. I have acknowledged all main sources of help;\\
6. Where the thesis is based on work done by myself jointly with others, I have made clear exactly what was done by others and what I have contributed myself;\\
7. Either none of this work has been published before submission, or parts of this work have been published by :\\
\\
Stefan Collier\\
April 2016
}
\tableofcontents
\listoffigures
\listoftables

\mainmatter
%% ----------------------------------------------------------------
%\include{Introduction}
%\include{Conclusions}
\include{chapters/1Project/main}
\include{chapters/2Lit/main}
\include{chapters/3Design/HighLevel}
\include{chapters/3Design/InDepth}
\include{chapters/4Impl/main}

\include{chapters/5Experiments/1/main}
\include{chapters/5Experiments/2/main}
\include{chapters/5Experiments/3/main}
\include{chapters/5Experiments/4/main}

\include{chapters/6Conclusion/main}

\appendix
\include{appendix/AppendixB}
\include{appendix/D/main}
\include{appendix/AppendixC}

\backmatter
\bibliographystyle{ecs}
\bibliography{ECS}
\end{document}
%% ----------------------------------------------------------------

 %% ----------------------------------------------------------------
%% Progress.tex
%% ---------------------------------------------------------------- 
\documentclass{ecsprogress}    % Use the progress Style
\graphicspath{{../figs/}}   % Location of your graphics files
    \usepackage{natbib}            % Use Natbib style for the refs.
\hypersetup{colorlinks=true}   % Set to false for black/white printing
\input{Definitions}            % Include your abbreviations



\usepackage{enumitem}% http://ctan.org/pkg/enumitem
\usepackage{multirow}
\usepackage{float}
\usepackage{amsmath}
\usepackage{multicol}
\usepackage{amssymb}
\usepackage[normalem]{ulem}
\useunder{\uline}{\ul}{}
\usepackage{wrapfig}


\usepackage[table,xcdraw]{xcolor}


%% ----------------------------------------------------------------
\begin{document}
\frontmatter
\title      {Heterogeneous Agent-based Model for Supermarket Competition}
\authors    {\texorpdfstring
             {\href{mailto:sc22g13@ecs.soton.ac.uk}{Stefan J. Collier}}
             {Stefan J. Collier}
            }
\addresses  {\groupname\\\deptname\\\univname}
\date       {\today}
\subject    {}
\keywords   {}
\supervisor {Dr. Maria Polukarov}
\examiner   {Professor Sheng Chen}

\maketitle
\begin{abstract}
This project aim was to model and analyse the effects of competitive pricing behaviors of grocery retailers on the British market. 

This was achieved by creating a multi-agent model, containing retailer and consumer agents. The heterogeneous crowd of retailers employs either a uniform pricing strategy or a ‘local price flexing’ strategy. The actions of these retailers are chosen by predicting the profit of each action, using a perceptron. Following on from the consideration of different economic models, a discrete model was developed so that software agents have a discrete environment to operate within. Within the model, it has been observed how supermarkets with differing behaviors affect a heterogeneous crowd of consumer agents. The model was implemented in Java with Python used to evaluate the results. 

The simulation displays good acceptance with real grocery market behavior, i.e. captures the performance of British retailers thus can be used to determine the impact of changes in their behavior on their competitors and consumers.Furthermore it can be used to provide insight into sustainability of volatile pricing strategies, providing a useful insight in volatility of British supermarket retail industry. 
\end{abstract}
\acknowledgements{
I would like to express my sincere gratitude to Dr Maria Polukarov for her guidance and support which provided me the freedom to take this research in the direction of my interest.\\
\\
I would also like to thank my family and friends for their encouragement and support. To those who quietly listened to my software complaints. To those who worked throughout the nights with me. To those who helped me write what I couldn't say. I cannot thank you enough.
}

\declaration{
I, Stefan Collier, declare that this dissertation and the work presented in it are my own and has been generated by me as the result of my own original research.\\
I confirm that:\\
1. This work was done wholly or mainly while in candidature for a degree at this University;\\
2. Where any part of this dissertation has previously been submitted for any other qualification at this University or any other institution, this has been clearly stated;\\
3. Where I have consulted the published work of others, this is always clearly attributed;\\
4. Where I have quoted from the work of others, the source is always given. With the exception of such quotations, this dissertation is entirely my own work;\\
5. I have acknowledged all main sources of help;\\
6. Where the thesis is based on work done by myself jointly with others, I have made clear exactly what was done by others and what I have contributed myself;\\
7. Either none of this work has been published before submission, or parts of this work have been published by :\\
\\
Stefan Collier\\
April 2016
}
\tableofcontents
\listoffigures
\listoftables

\mainmatter
%% ----------------------------------------------------------------
%\include{Introduction}
%\include{Conclusions}
\include{chapters/1Project/main}
\include{chapters/2Lit/main}
\include{chapters/3Design/HighLevel}
\include{chapters/3Design/InDepth}
\include{chapters/4Impl/main}

\include{chapters/5Experiments/1/main}
\include{chapters/5Experiments/2/main}
\include{chapters/5Experiments/3/main}
\include{chapters/5Experiments/4/main}

\include{chapters/6Conclusion/main}

\appendix
\include{appendix/AppendixB}
\include{appendix/D/main}
\include{appendix/AppendixC}

\backmatter
\bibliographystyle{ecs}
\bibliography{ECS}
\end{document}
%% ----------------------------------------------------------------

 %% ----------------------------------------------------------------
%% Progress.tex
%% ---------------------------------------------------------------- 
\documentclass{ecsprogress}    % Use the progress Style
\graphicspath{{../figs/}}   % Location of your graphics files
    \usepackage{natbib}            % Use Natbib style for the refs.
\hypersetup{colorlinks=true}   % Set to false for black/white printing
\input{Definitions}            % Include your abbreviations



\usepackage{enumitem}% http://ctan.org/pkg/enumitem
\usepackage{multirow}
\usepackage{float}
\usepackage{amsmath}
\usepackage{multicol}
\usepackage{amssymb}
\usepackage[normalem]{ulem}
\useunder{\uline}{\ul}{}
\usepackage{wrapfig}


\usepackage[table,xcdraw]{xcolor}


%% ----------------------------------------------------------------
\begin{document}
\frontmatter
\title      {Heterogeneous Agent-based Model for Supermarket Competition}
\authors    {\texorpdfstring
             {\href{mailto:sc22g13@ecs.soton.ac.uk}{Stefan J. Collier}}
             {Stefan J. Collier}
            }
\addresses  {\groupname\\\deptname\\\univname}
\date       {\today}
\subject    {}
\keywords   {}
\supervisor {Dr. Maria Polukarov}
\examiner   {Professor Sheng Chen}

\maketitle
\begin{abstract}
This project aim was to model and analyse the effects of competitive pricing behaviors of grocery retailers on the British market. 

This was achieved by creating a multi-agent model, containing retailer and consumer agents. The heterogeneous crowd of retailers employs either a uniform pricing strategy or a ‘local price flexing’ strategy. The actions of these retailers are chosen by predicting the profit of each action, using a perceptron. Following on from the consideration of different economic models, a discrete model was developed so that software agents have a discrete environment to operate within. Within the model, it has been observed how supermarkets with differing behaviors affect a heterogeneous crowd of consumer agents. The model was implemented in Java with Python used to evaluate the results. 

The simulation displays good acceptance with real grocery market behavior, i.e. captures the performance of British retailers thus can be used to determine the impact of changes in their behavior on their competitors and consumers.Furthermore it can be used to provide insight into sustainability of volatile pricing strategies, providing a useful insight in volatility of British supermarket retail industry. 
\end{abstract}
\acknowledgements{
I would like to express my sincere gratitude to Dr Maria Polukarov for her guidance and support which provided me the freedom to take this research in the direction of my interest.\\
\\
I would also like to thank my family and friends for their encouragement and support. To those who quietly listened to my software complaints. To those who worked throughout the nights with me. To those who helped me write what I couldn't say. I cannot thank you enough.
}

\declaration{
I, Stefan Collier, declare that this dissertation and the work presented in it are my own and has been generated by me as the result of my own original research.\\
I confirm that:\\
1. This work was done wholly or mainly while in candidature for a degree at this University;\\
2. Where any part of this dissertation has previously been submitted for any other qualification at this University or any other institution, this has been clearly stated;\\
3. Where I have consulted the published work of others, this is always clearly attributed;\\
4. Where I have quoted from the work of others, the source is always given. With the exception of such quotations, this dissertation is entirely my own work;\\
5. I have acknowledged all main sources of help;\\
6. Where the thesis is based on work done by myself jointly with others, I have made clear exactly what was done by others and what I have contributed myself;\\
7. Either none of this work has been published before submission, or parts of this work have been published by :\\
\\
Stefan Collier\\
April 2016
}
\tableofcontents
\listoffigures
\listoftables

\mainmatter
%% ----------------------------------------------------------------
%\include{Introduction}
%\include{Conclusions}
\include{chapters/1Project/main}
\include{chapters/2Lit/main}
\include{chapters/3Design/HighLevel}
\include{chapters/3Design/InDepth}
\include{chapters/4Impl/main}

\include{chapters/5Experiments/1/main}
\include{chapters/5Experiments/2/main}
\include{chapters/5Experiments/3/main}
\include{chapters/5Experiments/4/main}

\include{chapters/6Conclusion/main}

\appendix
\include{appendix/AppendixB}
\include{appendix/D/main}
\include{appendix/AppendixC}

\backmatter
\bibliographystyle{ecs}
\bibliography{ECS}
\end{document}
%% ----------------------------------------------------------------

 %% ----------------------------------------------------------------
%% Progress.tex
%% ---------------------------------------------------------------- 
\documentclass{ecsprogress}    % Use the progress Style
\graphicspath{{../figs/}}   % Location of your graphics files
    \usepackage{natbib}            % Use Natbib style for the refs.
\hypersetup{colorlinks=true}   % Set to false for black/white printing
\input{Definitions}            % Include your abbreviations



\usepackage{enumitem}% http://ctan.org/pkg/enumitem
\usepackage{multirow}
\usepackage{float}
\usepackage{amsmath}
\usepackage{multicol}
\usepackage{amssymb}
\usepackage[normalem]{ulem}
\useunder{\uline}{\ul}{}
\usepackage{wrapfig}


\usepackage[table,xcdraw]{xcolor}


%% ----------------------------------------------------------------
\begin{document}
\frontmatter
\title      {Heterogeneous Agent-based Model for Supermarket Competition}
\authors    {\texorpdfstring
             {\href{mailto:sc22g13@ecs.soton.ac.uk}{Stefan J. Collier}}
             {Stefan J. Collier}
            }
\addresses  {\groupname\\\deptname\\\univname}
\date       {\today}
\subject    {}
\keywords   {}
\supervisor {Dr. Maria Polukarov}
\examiner   {Professor Sheng Chen}

\maketitle
\begin{abstract}
This project aim was to model and analyse the effects of competitive pricing behaviors of grocery retailers on the British market. 

This was achieved by creating a multi-agent model, containing retailer and consumer agents. The heterogeneous crowd of retailers employs either a uniform pricing strategy or a ‘local price flexing’ strategy. The actions of these retailers are chosen by predicting the profit of each action, using a perceptron. Following on from the consideration of different economic models, a discrete model was developed so that software agents have a discrete environment to operate within. Within the model, it has been observed how supermarkets with differing behaviors affect a heterogeneous crowd of consumer agents. The model was implemented in Java with Python used to evaluate the results. 

The simulation displays good acceptance with real grocery market behavior, i.e. captures the performance of British retailers thus can be used to determine the impact of changes in their behavior on their competitors and consumers.Furthermore it can be used to provide insight into sustainability of volatile pricing strategies, providing a useful insight in volatility of British supermarket retail industry. 
\end{abstract}
\acknowledgements{
I would like to express my sincere gratitude to Dr Maria Polukarov for her guidance and support which provided me the freedom to take this research in the direction of my interest.\\
\\
I would also like to thank my family and friends for their encouragement and support. To those who quietly listened to my software complaints. To those who worked throughout the nights with me. To those who helped me write what I couldn't say. I cannot thank you enough.
}

\declaration{
I, Stefan Collier, declare that this dissertation and the work presented in it are my own and has been generated by me as the result of my own original research.\\
I confirm that:\\
1. This work was done wholly or mainly while in candidature for a degree at this University;\\
2. Where any part of this dissertation has previously been submitted for any other qualification at this University or any other institution, this has been clearly stated;\\
3. Where I have consulted the published work of others, this is always clearly attributed;\\
4. Where I have quoted from the work of others, the source is always given. With the exception of such quotations, this dissertation is entirely my own work;\\
5. I have acknowledged all main sources of help;\\
6. Where the thesis is based on work done by myself jointly with others, I have made clear exactly what was done by others and what I have contributed myself;\\
7. Either none of this work has been published before submission, or parts of this work have been published by :\\
\\
Stefan Collier\\
April 2016
}
\tableofcontents
\listoffigures
\listoftables

\mainmatter
%% ----------------------------------------------------------------
%\include{Introduction}
%\include{Conclusions}
\include{chapters/1Project/main}
\include{chapters/2Lit/main}
\include{chapters/3Design/HighLevel}
\include{chapters/3Design/InDepth}
\include{chapters/4Impl/main}

\include{chapters/5Experiments/1/main}
\include{chapters/5Experiments/2/main}
\include{chapters/5Experiments/3/main}
\include{chapters/5Experiments/4/main}

\include{chapters/6Conclusion/main}

\appendix
\include{appendix/AppendixB}
\include{appendix/D/main}
\include{appendix/AppendixC}

\backmatter
\bibliographystyle{ecs}
\bibliography{ECS}
\end{document}
%% ----------------------------------------------------------------


 %% ----------------------------------------------------------------
%% Progress.tex
%% ---------------------------------------------------------------- 
\documentclass{ecsprogress}    % Use the progress Style
\graphicspath{{../figs/}}   % Location of your graphics files
    \usepackage{natbib}            % Use Natbib style for the refs.
\hypersetup{colorlinks=true}   % Set to false for black/white printing
\input{Definitions}            % Include your abbreviations



\usepackage{enumitem}% http://ctan.org/pkg/enumitem
\usepackage{multirow}
\usepackage{float}
\usepackage{amsmath}
\usepackage{multicol}
\usepackage{amssymb}
\usepackage[normalem]{ulem}
\useunder{\uline}{\ul}{}
\usepackage{wrapfig}


\usepackage[table,xcdraw]{xcolor}


%% ----------------------------------------------------------------
\begin{document}
\frontmatter
\title      {Heterogeneous Agent-based Model for Supermarket Competition}
\authors    {\texorpdfstring
             {\href{mailto:sc22g13@ecs.soton.ac.uk}{Stefan J. Collier}}
             {Stefan J. Collier}
            }
\addresses  {\groupname\\\deptname\\\univname}
\date       {\today}
\subject    {}
\keywords   {}
\supervisor {Dr. Maria Polukarov}
\examiner   {Professor Sheng Chen}

\maketitle
\begin{abstract}
This project aim was to model and analyse the effects of competitive pricing behaviors of grocery retailers on the British market. 

This was achieved by creating a multi-agent model, containing retailer and consumer agents. The heterogeneous crowd of retailers employs either a uniform pricing strategy or a ‘local price flexing’ strategy. The actions of these retailers are chosen by predicting the profit of each action, using a perceptron. Following on from the consideration of different economic models, a discrete model was developed so that software agents have a discrete environment to operate within. Within the model, it has been observed how supermarkets with differing behaviors affect a heterogeneous crowd of consumer agents. The model was implemented in Java with Python used to evaluate the results. 

The simulation displays good acceptance with real grocery market behavior, i.e. captures the performance of British retailers thus can be used to determine the impact of changes in their behavior on their competitors and consumers.Furthermore it can be used to provide insight into sustainability of volatile pricing strategies, providing a useful insight in volatility of British supermarket retail industry. 
\end{abstract}
\acknowledgements{
I would like to express my sincere gratitude to Dr Maria Polukarov for her guidance and support which provided me the freedom to take this research in the direction of my interest.\\
\\
I would also like to thank my family and friends for their encouragement and support. To those who quietly listened to my software complaints. To those who worked throughout the nights with me. To those who helped me write what I couldn't say. I cannot thank you enough.
}

\declaration{
I, Stefan Collier, declare that this dissertation and the work presented in it are my own and has been generated by me as the result of my own original research.\\
I confirm that:\\
1. This work was done wholly or mainly while in candidature for a degree at this University;\\
2. Where any part of this dissertation has previously been submitted for any other qualification at this University or any other institution, this has been clearly stated;\\
3. Where I have consulted the published work of others, this is always clearly attributed;\\
4. Where I have quoted from the work of others, the source is always given. With the exception of such quotations, this dissertation is entirely my own work;\\
5. I have acknowledged all main sources of help;\\
6. Where the thesis is based on work done by myself jointly with others, I have made clear exactly what was done by others and what I have contributed myself;\\
7. Either none of this work has been published before submission, or parts of this work have been published by :\\
\\
Stefan Collier\\
April 2016
}
\tableofcontents
\listoffigures
\listoftables

\mainmatter
%% ----------------------------------------------------------------
%\include{Introduction}
%\include{Conclusions}
\include{chapters/1Project/main}
\include{chapters/2Lit/main}
\include{chapters/3Design/HighLevel}
\include{chapters/3Design/InDepth}
\include{chapters/4Impl/main}

\include{chapters/5Experiments/1/main}
\include{chapters/5Experiments/2/main}
\include{chapters/5Experiments/3/main}
\include{chapters/5Experiments/4/main}

\include{chapters/6Conclusion/main}

\appendix
\include{appendix/AppendixB}
\include{appendix/D/main}
\include{appendix/AppendixC}

\backmatter
\bibliographystyle{ecs}
\bibliography{ECS}
\end{document}
%% ----------------------------------------------------------------


\appendix
\include{appendix/AppendixB}
 %% ----------------------------------------------------------------
%% Progress.tex
%% ---------------------------------------------------------------- 
\documentclass{ecsprogress}    % Use the progress Style
\graphicspath{{../figs/}}   % Location of your graphics files
    \usepackage{natbib}            % Use Natbib style for the refs.
\hypersetup{colorlinks=true}   % Set to false for black/white printing
\input{Definitions}            % Include your abbreviations



\usepackage{enumitem}% http://ctan.org/pkg/enumitem
\usepackage{multirow}
\usepackage{float}
\usepackage{amsmath}
\usepackage{multicol}
\usepackage{amssymb}
\usepackage[normalem]{ulem}
\useunder{\uline}{\ul}{}
\usepackage{wrapfig}


\usepackage[table,xcdraw]{xcolor}


%% ----------------------------------------------------------------
\begin{document}
\frontmatter
\title      {Heterogeneous Agent-based Model for Supermarket Competition}
\authors    {\texorpdfstring
             {\href{mailto:sc22g13@ecs.soton.ac.uk}{Stefan J. Collier}}
             {Stefan J. Collier}
            }
\addresses  {\groupname\\\deptname\\\univname}
\date       {\today}
\subject    {}
\keywords   {}
\supervisor {Dr. Maria Polukarov}
\examiner   {Professor Sheng Chen}

\maketitle
\begin{abstract}
This project aim was to model and analyse the effects of competitive pricing behaviors of grocery retailers on the British market. 

This was achieved by creating a multi-agent model, containing retailer and consumer agents. The heterogeneous crowd of retailers employs either a uniform pricing strategy or a ‘local price flexing’ strategy. The actions of these retailers are chosen by predicting the profit of each action, using a perceptron. Following on from the consideration of different economic models, a discrete model was developed so that software agents have a discrete environment to operate within. Within the model, it has been observed how supermarkets with differing behaviors affect a heterogeneous crowd of consumer agents. The model was implemented in Java with Python used to evaluate the results. 

The simulation displays good acceptance with real grocery market behavior, i.e. captures the performance of British retailers thus can be used to determine the impact of changes in their behavior on their competitors and consumers.Furthermore it can be used to provide insight into sustainability of volatile pricing strategies, providing a useful insight in volatility of British supermarket retail industry. 
\end{abstract}
\acknowledgements{
I would like to express my sincere gratitude to Dr Maria Polukarov for her guidance and support which provided me the freedom to take this research in the direction of my interest.\\
\\
I would also like to thank my family and friends for their encouragement and support. To those who quietly listened to my software complaints. To those who worked throughout the nights with me. To those who helped me write what I couldn't say. I cannot thank you enough.
}

\declaration{
I, Stefan Collier, declare that this dissertation and the work presented in it are my own and has been generated by me as the result of my own original research.\\
I confirm that:\\
1. This work was done wholly or mainly while in candidature for a degree at this University;\\
2. Where any part of this dissertation has previously been submitted for any other qualification at this University or any other institution, this has been clearly stated;\\
3. Where I have consulted the published work of others, this is always clearly attributed;\\
4. Where I have quoted from the work of others, the source is always given. With the exception of such quotations, this dissertation is entirely my own work;\\
5. I have acknowledged all main sources of help;\\
6. Where the thesis is based on work done by myself jointly with others, I have made clear exactly what was done by others and what I have contributed myself;\\
7. Either none of this work has been published before submission, or parts of this work have been published by :\\
\\
Stefan Collier\\
April 2016
}
\tableofcontents
\listoffigures
\listoftables

\mainmatter
%% ----------------------------------------------------------------
%\include{Introduction}
%\include{Conclusions}
\include{chapters/1Project/main}
\include{chapters/2Lit/main}
\include{chapters/3Design/HighLevel}
\include{chapters/3Design/InDepth}
\include{chapters/4Impl/main}

\include{chapters/5Experiments/1/main}
\include{chapters/5Experiments/2/main}
\include{chapters/5Experiments/3/main}
\include{chapters/5Experiments/4/main}

\include{chapters/6Conclusion/main}

\appendix
\include{appendix/AppendixB}
\include{appendix/D/main}
\include{appendix/AppendixC}

\backmatter
\bibliographystyle{ecs}
\bibliography{ECS}
\end{document}
%% ----------------------------------------------------------------

\include{appendix/AppendixC}

\backmatter
\bibliographystyle{ecs}
\bibliography{ECS}
\end{document}
%% ----------------------------------------------------------------

 %% ----------------------------------------------------------------
%% Progress.tex
%% ---------------------------------------------------------------- 
\documentclass{ecsprogress}    % Use the progress Style
\graphicspath{{../figs/}}   % Location of your graphics files
    \usepackage{natbib}            % Use Natbib style for the refs.
\hypersetup{colorlinks=true}   % Set to false for black/white printing
\input{Definitions}            % Include your abbreviations



\usepackage{enumitem}% http://ctan.org/pkg/enumitem
\usepackage{multirow}
\usepackage{float}
\usepackage{amsmath}
\usepackage{multicol}
\usepackage{amssymb}
\usepackage[normalem]{ulem}
\useunder{\uline}{\ul}{}
\usepackage{wrapfig}


\usepackage[table,xcdraw]{xcolor}


%% ----------------------------------------------------------------
\begin{document}
\frontmatter
\title      {Heterogeneous Agent-based Model for Supermarket Competition}
\authors    {\texorpdfstring
             {\href{mailto:sc22g13@ecs.soton.ac.uk}{Stefan J. Collier}}
             {Stefan J. Collier}
            }
\addresses  {\groupname\\\deptname\\\univname}
\date       {\today}
\subject    {}
\keywords   {}
\supervisor {Dr. Maria Polukarov}
\examiner   {Professor Sheng Chen}

\maketitle
\begin{abstract}
This project aim was to model and analyse the effects of competitive pricing behaviors of grocery retailers on the British market. 

This was achieved by creating a multi-agent model, containing retailer and consumer agents. The heterogeneous crowd of retailers employs either a uniform pricing strategy or a ‘local price flexing’ strategy. The actions of these retailers are chosen by predicting the profit of each action, using a perceptron. Following on from the consideration of different economic models, a discrete model was developed so that software agents have a discrete environment to operate within. Within the model, it has been observed how supermarkets with differing behaviors affect a heterogeneous crowd of consumer agents. The model was implemented in Java with Python used to evaluate the results. 

The simulation displays good acceptance with real grocery market behavior, i.e. captures the performance of British retailers thus can be used to determine the impact of changes in their behavior on their competitors and consumers.Furthermore it can be used to provide insight into sustainability of volatile pricing strategies, providing a useful insight in volatility of British supermarket retail industry. 
\end{abstract}
\acknowledgements{
I would like to express my sincere gratitude to Dr Maria Polukarov for her guidance and support which provided me the freedom to take this research in the direction of my interest.\\
\\
I would also like to thank my family and friends for their encouragement and support. To those who quietly listened to my software complaints. To those who worked throughout the nights with me. To those who helped me write what I couldn't say. I cannot thank you enough.
}

\declaration{
I, Stefan Collier, declare that this dissertation and the work presented in it are my own and has been generated by me as the result of my own original research.\\
I confirm that:\\
1. This work was done wholly or mainly while in candidature for a degree at this University;\\
2. Where any part of this dissertation has previously been submitted for any other qualification at this University or any other institution, this has been clearly stated;\\
3. Where I have consulted the published work of others, this is always clearly attributed;\\
4. Where I have quoted from the work of others, the source is always given. With the exception of such quotations, this dissertation is entirely my own work;\\
5. I have acknowledged all main sources of help;\\
6. Where the thesis is based on work done by myself jointly with others, I have made clear exactly what was done by others and what I have contributed myself;\\
7. Either none of this work has been published before submission, or parts of this work have been published by :\\
\\
Stefan Collier\\
April 2016
}
\tableofcontents
\listoffigures
\listoftables

\mainmatter
%% ----------------------------------------------------------------
%\include{Introduction}
%\include{Conclusions}
 %% ----------------------------------------------------------------
%% Progress.tex
%% ---------------------------------------------------------------- 
\documentclass{ecsprogress}    % Use the progress Style
\graphicspath{{../figs/}}   % Location of your graphics files
    \usepackage{natbib}            % Use Natbib style for the refs.
\hypersetup{colorlinks=true}   % Set to false for black/white printing
\input{Definitions}            % Include your abbreviations



\usepackage{enumitem}% http://ctan.org/pkg/enumitem
\usepackage{multirow}
\usepackage{float}
\usepackage{amsmath}
\usepackage{multicol}
\usepackage{amssymb}
\usepackage[normalem]{ulem}
\useunder{\uline}{\ul}{}
\usepackage{wrapfig}


\usepackage[table,xcdraw]{xcolor}


%% ----------------------------------------------------------------
\begin{document}
\frontmatter
\title      {Heterogeneous Agent-based Model for Supermarket Competition}
\authors    {\texorpdfstring
             {\href{mailto:sc22g13@ecs.soton.ac.uk}{Stefan J. Collier}}
             {Stefan J. Collier}
            }
\addresses  {\groupname\\\deptname\\\univname}
\date       {\today}
\subject    {}
\keywords   {}
\supervisor {Dr. Maria Polukarov}
\examiner   {Professor Sheng Chen}

\maketitle
\begin{abstract}
This project aim was to model and analyse the effects of competitive pricing behaviors of grocery retailers on the British market. 

This was achieved by creating a multi-agent model, containing retailer and consumer agents. The heterogeneous crowd of retailers employs either a uniform pricing strategy or a ‘local price flexing’ strategy. The actions of these retailers are chosen by predicting the profit of each action, using a perceptron. Following on from the consideration of different economic models, a discrete model was developed so that software agents have a discrete environment to operate within. Within the model, it has been observed how supermarkets with differing behaviors affect a heterogeneous crowd of consumer agents. The model was implemented in Java with Python used to evaluate the results. 

The simulation displays good acceptance with real grocery market behavior, i.e. captures the performance of British retailers thus can be used to determine the impact of changes in their behavior on their competitors and consumers.Furthermore it can be used to provide insight into sustainability of volatile pricing strategies, providing a useful insight in volatility of British supermarket retail industry. 
\end{abstract}
\acknowledgements{
I would like to express my sincere gratitude to Dr Maria Polukarov for her guidance and support which provided me the freedom to take this research in the direction of my interest.\\
\\
I would also like to thank my family and friends for their encouragement and support. To those who quietly listened to my software complaints. To those who worked throughout the nights with me. To those who helped me write what I couldn't say. I cannot thank you enough.
}

\declaration{
I, Stefan Collier, declare that this dissertation and the work presented in it are my own and has been generated by me as the result of my own original research.\\
I confirm that:\\
1. This work was done wholly or mainly while in candidature for a degree at this University;\\
2. Where any part of this dissertation has previously been submitted for any other qualification at this University or any other institution, this has been clearly stated;\\
3. Where I have consulted the published work of others, this is always clearly attributed;\\
4. Where I have quoted from the work of others, the source is always given. With the exception of such quotations, this dissertation is entirely my own work;\\
5. I have acknowledged all main sources of help;\\
6. Where the thesis is based on work done by myself jointly with others, I have made clear exactly what was done by others and what I have contributed myself;\\
7. Either none of this work has been published before submission, or parts of this work have been published by :\\
\\
Stefan Collier\\
April 2016
}
\tableofcontents
\listoffigures
\listoftables

\mainmatter
%% ----------------------------------------------------------------
%\include{Introduction}
%\include{Conclusions}
\include{chapters/1Project/main}
\include{chapters/2Lit/main}
\include{chapters/3Design/HighLevel}
\include{chapters/3Design/InDepth}
\include{chapters/4Impl/main}

\include{chapters/5Experiments/1/main}
\include{chapters/5Experiments/2/main}
\include{chapters/5Experiments/3/main}
\include{chapters/5Experiments/4/main}

\include{chapters/6Conclusion/main}

\appendix
\include{appendix/AppendixB}
\include{appendix/D/main}
\include{appendix/AppendixC}

\backmatter
\bibliographystyle{ecs}
\bibliography{ECS}
\end{document}
%% ----------------------------------------------------------------

 %% ----------------------------------------------------------------
%% Progress.tex
%% ---------------------------------------------------------------- 
\documentclass{ecsprogress}    % Use the progress Style
\graphicspath{{../figs/}}   % Location of your graphics files
    \usepackage{natbib}            % Use Natbib style for the refs.
\hypersetup{colorlinks=true}   % Set to false for black/white printing
\input{Definitions}            % Include your abbreviations



\usepackage{enumitem}% http://ctan.org/pkg/enumitem
\usepackage{multirow}
\usepackage{float}
\usepackage{amsmath}
\usepackage{multicol}
\usepackage{amssymb}
\usepackage[normalem]{ulem}
\useunder{\uline}{\ul}{}
\usepackage{wrapfig}


\usepackage[table,xcdraw]{xcolor}


%% ----------------------------------------------------------------
\begin{document}
\frontmatter
\title      {Heterogeneous Agent-based Model for Supermarket Competition}
\authors    {\texorpdfstring
             {\href{mailto:sc22g13@ecs.soton.ac.uk}{Stefan J. Collier}}
             {Stefan J. Collier}
            }
\addresses  {\groupname\\\deptname\\\univname}
\date       {\today}
\subject    {}
\keywords   {}
\supervisor {Dr. Maria Polukarov}
\examiner   {Professor Sheng Chen}

\maketitle
\begin{abstract}
This project aim was to model and analyse the effects of competitive pricing behaviors of grocery retailers on the British market. 

This was achieved by creating a multi-agent model, containing retailer and consumer agents. The heterogeneous crowd of retailers employs either a uniform pricing strategy or a ‘local price flexing’ strategy. The actions of these retailers are chosen by predicting the profit of each action, using a perceptron. Following on from the consideration of different economic models, a discrete model was developed so that software agents have a discrete environment to operate within. Within the model, it has been observed how supermarkets with differing behaviors affect a heterogeneous crowd of consumer agents. The model was implemented in Java with Python used to evaluate the results. 

The simulation displays good acceptance with real grocery market behavior, i.e. captures the performance of British retailers thus can be used to determine the impact of changes in their behavior on their competitors and consumers.Furthermore it can be used to provide insight into sustainability of volatile pricing strategies, providing a useful insight in volatility of British supermarket retail industry. 
\end{abstract}
\acknowledgements{
I would like to express my sincere gratitude to Dr Maria Polukarov for her guidance and support which provided me the freedom to take this research in the direction of my interest.\\
\\
I would also like to thank my family and friends for their encouragement and support. To those who quietly listened to my software complaints. To those who worked throughout the nights with me. To those who helped me write what I couldn't say. I cannot thank you enough.
}

\declaration{
I, Stefan Collier, declare that this dissertation and the work presented in it are my own and has been generated by me as the result of my own original research.\\
I confirm that:\\
1. This work was done wholly or mainly while in candidature for a degree at this University;\\
2. Where any part of this dissertation has previously been submitted for any other qualification at this University or any other institution, this has been clearly stated;\\
3. Where I have consulted the published work of others, this is always clearly attributed;\\
4. Where I have quoted from the work of others, the source is always given. With the exception of such quotations, this dissertation is entirely my own work;\\
5. I have acknowledged all main sources of help;\\
6. Where the thesis is based on work done by myself jointly with others, I have made clear exactly what was done by others and what I have contributed myself;\\
7. Either none of this work has been published before submission, or parts of this work have been published by :\\
\\
Stefan Collier\\
April 2016
}
\tableofcontents
\listoffigures
\listoftables

\mainmatter
%% ----------------------------------------------------------------
%\include{Introduction}
%\include{Conclusions}
\include{chapters/1Project/main}
\include{chapters/2Lit/main}
\include{chapters/3Design/HighLevel}
\include{chapters/3Design/InDepth}
\include{chapters/4Impl/main}

\include{chapters/5Experiments/1/main}
\include{chapters/5Experiments/2/main}
\include{chapters/5Experiments/3/main}
\include{chapters/5Experiments/4/main}

\include{chapters/6Conclusion/main}

\appendix
\include{appendix/AppendixB}
\include{appendix/D/main}
\include{appendix/AppendixC}

\backmatter
\bibliographystyle{ecs}
\bibliography{ECS}
\end{document}
%% ----------------------------------------------------------------

\include{chapters/3Design/HighLevel}
\include{chapters/3Design/InDepth}
 %% ----------------------------------------------------------------
%% Progress.tex
%% ---------------------------------------------------------------- 
\documentclass{ecsprogress}    % Use the progress Style
\graphicspath{{../figs/}}   % Location of your graphics files
    \usepackage{natbib}            % Use Natbib style for the refs.
\hypersetup{colorlinks=true}   % Set to false for black/white printing
\input{Definitions}            % Include your abbreviations



\usepackage{enumitem}% http://ctan.org/pkg/enumitem
\usepackage{multirow}
\usepackage{float}
\usepackage{amsmath}
\usepackage{multicol}
\usepackage{amssymb}
\usepackage[normalem]{ulem}
\useunder{\uline}{\ul}{}
\usepackage{wrapfig}


\usepackage[table,xcdraw]{xcolor}


%% ----------------------------------------------------------------
\begin{document}
\frontmatter
\title      {Heterogeneous Agent-based Model for Supermarket Competition}
\authors    {\texorpdfstring
             {\href{mailto:sc22g13@ecs.soton.ac.uk}{Stefan J. Collier}}
             {Stefan J. Collier}
            }
\addresses  {\groupname\\\deptname\\\univname}
\date       {\today}
\subject    {}
\keywords   {}
\supervisor {Dr. Maria Polukarov}
\examiner   {Professor Sheng Chen}

\maketitle
\begin{abstract}
This project aim was to model and analyse the effects of competitive pricing behaviors of grocery retailers on the British market. 

This was achieved by creating a multi-agent model, containing retailer and consumer agents. The heterogeneous crowd of retailers employs either a uniform pricing strategy or a ‘local price flexing’ strategy. The actions of these retailers are chosen by predicting the profit of each action, using a perceptron. Following on from the consideration of different economic models, a discrete model was developed so that software agents have a discrete environment to operate within. Within the model, it has been observed how supermarkets with differing behaviors affect a heterogeneous crowd of consumer agents. The model was implemented in Java with Python used to evaluate the results. 

The simulation displays good acceptance with real grocery market behavior, i.e. captures the performance of British retailers thus can be used to determine the impact of changes in their behavior on their competitors and consumers.Furthermore it can be used to provide insight into sustainability of volatile pricing strategies, providing a useful insight in volatility of British supermarket retail industry. 
\end{abstract}
\acknowledgements{
I would like to express my sincere gratitude to Dr Maria Polukarov for her guidance and support which provided me the freedom to take this research in the direction of my interest.\\
\\
I would also like to thank my family and friends for their encouragement and support. To those who quietly listened to my software complaints. To those who worked throughout the nights with me. To those who helped me write what I couldn't say. I cannot thank you enough.
}

\declaration{
I, Stefan Collier, declare that this dissertation and the work presented in it are my own and has been generated by me as the result of my own original research.\\
I confirm that:\\
1. This work was done wholly or mainly while in candidature for a degree at this University;\\
2. Where any part of this dissertation has previously been submitted for any other qualification at this University or any other institution, this has been clearly stated;\\
3. Where I have consulted the published work of others, this is always clearly attributed;\\
4. Where I have quoted from the work of others, the source is always given. With the exception of such quotations, this dissertation is entirely my own work;\\
5. I have acknowledged all main sources of help;\\
6. Where the thesis is based on work done by myself jointly with others, I have made clear exactly what was done by others and what I have contributed myself;\\
7. Either none of this work has been published before submission, or parts of this work have been published by :\\
\\
Stefan Collier\\
April 2016
}
\tableofcontents
\listoffigures
\listoftables

\mainmatter
%% ----------------------------------------------------------------
%\include{Introduction}
%\include{Conclusions}
\include{chapters/1Project/main}
\include{chapters/2Lit/main}
\include{chapters/3Design/HighLevel}
\include{chapters/3Design/InDepth}
\include{chapters/4Impl/main}

\include{chapters/5Experiments/1/main}
\include{chapters/5Experiments/2/main}
\include{chapters/5Experiments/3/main}
\include{chapters/5Experiments/4/main}

\include{chapters/6Conclusion/main}

\appendix
\include{appendix/AppendixB}
\include{appendix/D/main}
\include{appendix/AppendixC}

\backmatter
\bibliographystyle{ecs}
\bibliography{ECS}
\end{document}
%% ----------------------------------------------------------------


 %% ----------------------------------------------------------------
%% Progress.tex
%% ---------------------------------------------------------------- 
\documentclass{ecsprogress}    % Use the progress Style
\graphicspath{{../figs/}}   % Location of your graphics files
    \usepackage{natbib}            % Use Natbib style for the refs.
\hypersetup{colorlinks=true}   % Set to false for black/white printing
\input{Definitions}            % Include your abbreviations



\usepackage{enumitem}% http://ctan.org/pkg/enumitem
\usepackage{multirow}
\usepackage{float}
\usepackage{amsmath}
\usepackage{multicol}
\usepackage{amssymb}
\usepackage[normalem]{ulem}
\useunder{\uline}{\ul}{}
\usepackage{wrapfig}


\usepackage[table,xcdraw]{xcolor}


%% ----------------------------------------------------------------
\begin{document}
\frontmatter
\title      {Heterogeneous Agent-based Model for Supermarket Competition}
\authors    {\texorpdfstring
             {\href{mailto:sc22g13@ecs.soton.ac.uk}{Stefan J. Collier}}
             {Stefan J. Collier}
            }
\addresses  {\groupname\\\deptname\\\univname}
\date       {\today}
\subject    {}
\keywords   {}
\supervisor {Dr. Maria Polukarov}
\examiner   {Professor Sheng Chen}

\maketitle
\begin{abstract}
This project aim was to model and analyse the effects of competitive pricing behaviors of grocery retailers on the British market. 

This was achieved by creating a multi-agent model, containing retailer and consumer agents. The heterogeneous crowd of retailers employs either a uniform pricing strategy or a ‘local price flexing’ strategy. The actions of these retailers are chosen by predicting the profit of each action, using a perceptron. Following on from the consideration of different economic models, a discrete model was developed so that software agents have a discrete environment to operate within. Within the model, it has been observed how supermarkets with differing behaviors affect a heterogeneous crowd of consumer agents. The model was implemented in Java with Python used to evaluate the results. 

The simulation displays good acceptance with real grocery market behavior, i.e. captures the performance of British retailers thus can be used to determine the impact of changes in their behavior on their competitors and consumers.Furthermore it can be used to provide insight into sustainability of volatile pricing strategies, providing a useful insight in volatility of British supermarket retail industry. 
\end{abstract}
\acknowledgements{
I would like to express my sincere gratitude to Dr Maria Polukarov for her guidance and support which provided me the freedom to take this research in the direction of my interest.\\
\\
I would also like to thank my family and friends for their encouragement and support. To those who quietly listened to my software complaints. To those who worked throughout the nights with me. To those who helped me write what I couldn't say. I cannot thank you enough.
}

\declaration{
I, Stefan Collier, declare that this dissertation and the work presented in it are my own and has been generated by me as the result of my own original research.\\
I confirm that:\\
1. This work was done wholly or mainly while in candidature for a degree at this University;\\
2. Where any part of this dissertation has previously been submitted for any other qualification at this University or any other institution, this has been clearly stated;\\
3. Where I have consulted the published work of others, this is always clearly attributed;\\
4. Where I have quoted from the work of others, the source is always given. With the exception of such quotations, this dissertation is entirely my own work;\\
5. I have acknowledged all main sources of help;\\
6. Where the thesis is based on work done by myself jointly with others, I have made clear exactly what was done by others and what I have contributed myself;\\
7. Either none of this work has been published before submission, or parts of this work have been published by :\\
\\
Stefan Collier\\
April 2016
}
\tableofcontents
\listoffigures
\listoftables

\mainmatter
%% ----------------------------------------------------------------
%\include{Introduction}
%\include{Conclusions}
\include{chapters/1Project/main}
\include{chapters/2Lit/main}
\include{chapters/3Design/HighLevel}
\include{chapters/3Design/InDepth}
\include{chapters/4Impl/main}

\include{chapters/5Experiments/1/main}
\include{chapters/5Experiments/2/main}
\include{chapters/5Experiments/3/main}
\include{chapters/5Experiments/4/main}

\include{chapters/6Conclusion/main}

\appendix
\include{appendix/AppendixB}
\include{appendix/D/main}
\include{appendix/AppendixC}

\backmatter
\bibliographystyle{ecs}
\bibliography{ECS}
\end{document}
%% ----------------------------------------------------------------

 %% ----------------------------------------------------------------
%% Progress.tex
%% ---------------------------------------------------------------- 
\documentclass{ecsprogress}    % Use the progress Style
\graphicspath{{../figs/}}   % Location of your graphics files
    \usepackage{natbib}            % Use Natbib style for the refs.
\hypersetup{colorlinks=true}   % Set to false for black/white printing
\input{Definitions}            % Include your abbreviations



\usepackage{enumitem}% http://ctan.org/pkg/enumitem
\usepackage{multirow}
\usepackage{float}
\usepackage{amsmath}
\usepackage{multicol}
\usepackage{amssymb}
\usepackage[normalem]{ulem}
\useunder{\uline}{\ul}{}
\usepackage{wrapfig}


\usepackage[table,xcdraw]{xcolor}


%% ----------------------------------------------------------------
\begin{document}
\frontmatter
\title      {Heterogeneous Agent-based Model for Supermarket Competition}
\authors    {\texorpdfstring
             {\href{mailto:sc22g13@ecs.soton.ac.uk}{Stefan J. Collier}}
             {Stefan J. Collier}
            }
\addresses  {\groupname\\\deptname\\\univname}
\date       {\today}
\subject    {}
\keywords   {}
\supervisor {Dr. Maria Polukarov}
\examiner   {Professor Sheng Chen}

\maketitle
\begin{abstract}
This project aim was to model and analyse the effects of competitive pricing behaviors of grocery retailers on the British market. 

This was achieved by creating a multi-agent model, containing retailer and consumer agents. The heterogeneous crowd of retailers employs either a uniform pricing strategy or a ‘local price flexing’ strategy. The actions of these retailers are chosen by predicting the profit of each action, using a perceptron. Following on from the consideration of different economic models, a discrete model was developed so that software agents have a discrete environment to operate within. Within the model, it has been observed how supermarkets with differing behaviors affect a heterogeneous crowd of consumer agents. The model was implemented in Java with Python used to evaluate the results. 

The simulation displays good acceptance with real grocery market behavior, i.e. captures the performance of British retailers thus can be used to determine the impact of changes in their behavior on their competitors and consumers.Furthermore it can be used to provide insight into sustainability of volatile pricing strategies, providing a useful insight in volatility of British supermarket retail industry. 
\end{abstract}
\acknowledgements{
I would like to express my sincere gratitude to Dr Maria Polukarov for her guidance and support which provided me the freedom to take this research in the direction of my interest.\\
\\
I would also like to thank my family and friends for their encouragement and support. To those who quietly listened to my software complaints. To those who worked throughout the nights with me. To those who helped me write what I couldn't say. I cannot thank you enough.
}

\declaration{
I, Stefan Collier, declare that this dissertation and the work presented in it are my own and has been generated by me as the result of my own original research.\\
I confirm that:\\
1. This work was done wholly or mainly while in candidature for a degree at this University;\\
2. Where any part of this dissertation has previously been submitted for any other qualification at this University or any other institution, this has been clearly stated;\\
3. Where I have consulted the published work of others, this is always clearly attributed;\\
4. Where I have quoted from the work of others, the source is always given. With the exception of such quotations, this dissertation is entirely my own work;\\
5. I have acknowledged all main sources of help;\\
6. Where the thesis is based on work done by myself jointly with others, I have made clear exactly what was done by others and what I have contributed myself;\\
7. Either none of this work has been published before submission, or parts of this work have been published by :\\
\\
Stefan Collier\\
April 2016
}
\tableofcontents
\listoffigures
\listoftables

\mainmatter
%% ----------------------------------------------------------------
%\include{Introduction}
%\include{Conclusions}
\include{chapters/1Project/main}
\include{chapters/2Lit/main}
\include{chapters/3Design/HighLevel}
\include{chapters/3Design/InDepth}
\include{chapters/4Impl/main}

\include{chapters/5Experiments/1/main}
\include{chapters/5Experiments/2/main}
\include{chapters/5Experiments/3/main}
\include{chapters/5Experiments/4/main}

\include{chapters/6Conclusion/main}

\appendix
\include{appendix/AppendixB}
\include{appendix/D/main}
\include{appendix/AppendixC}

\backmatter
\bibliographystyle{ecs}
\bibliography{ECS}
\end{document}
%% ----------------------------------------------------------------

 %% ----------------------------------------------------------------
%% Progress.tex
%% ---------------------------------------------------------------- 
\documentclass{ecsprogress}    % Use the progress Style
\graphicspath{{../figs/}}   % Location of your graphics files
    \usepackage{natbib}            % Use Natbib style for the refs.
\hypersetup{colorlinks=true}   % Set to false for black/white printing
\input{Definitions}            % Include your abbreviations



\usepackage{enumitem}% http://ctan.org/pkg/enumitem
\usepackage{multirow}
\usepackage{float}
\usepackage{amsmath}
\usepackage{multicol}
\usepackage{amssymb}
\usepackage[normalem]{ulem}
\useunder{\uline}{\ul}{}
\usepackage{wrapfig}


\usepackage[table,xcdraw]{xcolor}


%% ----------------------------------------------------------------
\begin{document}
\frontmatter
\title      {Heterogeneous Agent-based Model for Supermarket Competition}
\authors    {\texorpdfstring
             {\href{mailto:sc22g13@ecs.soton.ac.uk}{Stefan J. Collier}}
             {Stefan J. Collier}
            }
\addresses  {\groupname\\\deptname\\\univname}
\date       {\today}
\subject    {}
\keywords   {}
\supervisor {Dr. Maria Polukarov}
\examiner   {Professor Sheng Chen}

\maketitle
\begin{abstract}
This project aim was to model and analyse the effects of competitive pricing behaviors of grocery retailers on the British market. 

This was achieved by creating a multi-agent model, containing retailer and consumer agents. The heterogeneous crowd of retailers employs either a uniform pricing strategy or a ‘local price flexing’ strategy. The actions of these retailers are chosen by predicting the profit of each action, using a perceptron. Following on from the consideration of different economic models, a discrete model was developed so that software agents have a discrete environment to operate within. Within the model, it has been observed how supermarkets with differing behaviors affect a heterogeneous crowd of consumer agents. The model was implemented in Java with Python used to evaluate the results. 

The simulation displays good acceptance with real grocery market behavior, i.e. captures the performance of British retailers thus can be used to determine the impact of changes in their behavior on their competitors and consumers.Furthermore it can be used to provide insight into sustainability of volatile pricing strategies, providing a useful insight in volatility of British supermarket retail industry. 
\end{abstract}
\acknowledgements{
I would like to express my sincere gratitude to Dr Maria Polukarov for her guidance and support which provided me the freedom to take this research in the direction of my interest.\\
\\
I would also like to thank my family and friends for their encouragement and support. To those who quietly listened to my software complaints. To those who worked throughout the nights with me. To those who helped me write what I couldn't say. I cannot thank you enough.
}

\declaration{
I, Stefan Collier, declare that this dissertation and the work presented in it are my own and has been generated by me as the result of my own original research.\\
I confirm that:\\
1. This work was done wholly or mainly while in candidature for a degree at this University;\\
2. Where any part of this dissertation has previously been submitted for any other qualification at this University or any other institution, this has been clearly stated;\\
3. Where I have consulted the published work of others, this is always clearly attributed;\\
4. Where I have quoted from the work of others, the source is always given. With the exception of such quotations, this dissertation is entirely my own work;\\
5. I have acknowledged all main sources of help;\\
6. Where the thesis is based on work done by myself jointly with others, I have made clear exactly what was done by others and what I have contributed myself;\\
7. Either none of this work has been published before submission, or parts of this work have been published by :\\
\\
Stefan Collier\\
April 2016
}
\tableofcontents
\listoffigures
\listoftables

\mainmatter
%% ----------------------------------------------------------------
%\include{Introduction}
%\include{Conclusions}
\include{chapters/1Project/main}
\include{chapters/2Lit/main}
\include{chapters/3Design/HighLevel}
\include{chapters/3Design/InDepth}
\include{chapters/4Impl/main}

\include{chapters/5Experiments/1/main}
\include{chapters/5Experiments/2/main}
\include{chapters/5Experiments/3/main}
\include{chapters/5Experiments/4/main}

\include{chapters/6Conclusion/main}

\appendix
\include{appendix/AppendixB}
\include{appendix/D/main}
\include{appendix/AppendixC}

\backmatter
\bibliographystyle{ecs}
\bibliography{ECS}
\end{document}
%% ----------------------------------------------------------------

 %% ----------------------------------------------------------------
%% Progress.tex
%% ---------------------------------------------------------------- 
\documentclass{ecsprogress}    % Use the progress Style
\graphicspath{{../figs/}}   % Location of your graphics files
    \usepackage{natbib}            % Use Natbib style for the refs.
\hypersetup{colorlinks=true}   % Set to false for black/white printing
\input{Definitions}            % Include your abbreviations



\usepackage{enumitem}% http://ctan.org/pkg/enumitem
\usepackage{multirow}
\usepackage{float}
\usepackage{amsmath}
\usepackage{multicol}
\usepackage{amssymb}
\usepackage[normalem]{ulem}
\useunder{\uline}{\ul}{}
\usepackage{wrapfig}


\usepackage[table,xcdraw]{xcolor}


%% ----------------------------------------------------------------
\begin{document}
\frontmatter
\title      {Heterogeneous Agent-based Model for Supermarket Competition}
\authors    {\texorpdfstring
             {\href{mailto:sc22g13@ecs.soton.ac.uk}{Stefan J. Collier}}
             {Stefan J. Collier}
            }
\addresses  {\groupname\\\deptname\\\univname}
\date       {\today}
\subject    {}
\keywords   {}
\supervisor {Dr. Maria Polukarov}
\examiner   {Professor Sheng Chen}

\maketitle
\begin{abstract}
This project aim was to model and analyse the effects of competitive pricing behaviors of grocery retailers on the British market. 

This was achieved by creating a multi-agent model, containing retailer and consumer agents. The heterogeneous crowd of retailers employs either a uniform pricing strategy or a ‘local price flexing’ strategy. The actions of these retailers are chosen by predicting the profit of each action, using a perceptron. Following on from the consideration of different economic models, a discrete model was developed so that software agents have a discrete environment to operate within. Within the model, it has been observed how supermarkets with differing behaviors affect a heterogeneous crowd of consumer agents. The model was implemented in Java with Python used to evaluate the results. 

The simulation displays good acceptance with real grocery market behavior, i.e. captures the performance of British retailers thus can be used to determine the impact of changes in their behavior on their competitors and consumers.Furthermore it can be used to provide insight into sustainability of volatile pricing strategies, providing a useful insight in volatility of British supermarket retail industry. 
\end{abstract}
\acknowledgements{
I would like to express my sincere gratitude to Dr Maria Polukarov for her guidance and support which provided me the freedom to take this research in the direction of my interest.\\
\\
I would also like to thank my family and friends for their encouragement and support. To those who quietly listened to my software complaints. To those who worked throughout the nights with me. To those who helped me write what I couldn't say. I cannot thank you enough.
}

\declaration{
I, Stefan Collier, declare that this dissertation and the work presented in it are my own and has been generated by me as the result of my own original research.\\
I confirm that:\\
1. This work was done wholly or mainly while in candidature for a degree at this University;\\
2. Where any part of this dissertation has previously been submitted for any other qualification at this University or any other institution, this has been clearly stated;\\
3. Where I have consulted the published work of others, this is always clearly attributed;\\
4. Where I have quoted from the work of others, the source is always given. With the exception of such quotations, this dissertation is entirely my own work;\\
5. I have acknowledged all main sources of help;\\
6. Where the thesis is based on work done by myself jointly with others, I have made clear exactly what was done by others and what I have contributed myself;\\
7. Either none of this work has been published before submission, or parts of this work have been published by :\\
\\
Stefan Collier\\
April 2016
}
\tableofcontents
\listoffigures
\listoftables

\mainmatter
%% ----------------------------------------------------------------
%\include{Introduction}
%\include{Conclusions}
\include{chapters/1Project/main}
\include{chapters/2Lit/main}
\include{chapters/3Design/HighLevel}
\include{chapters/3Design/InDepth}
\include{chapters/4Impl/main}

\include{chapters/5Experiments/1/main}
\include{chapters/5Experiments/2/main}
\include{chapters/5Experiments/3/main}
\include{chapters/5Experiments/4/main}

\include{chapters/6Conclusion/main}

\appendix
\include{appendix/AppendixB}
\include{appendix/D/main}
\include{appendix/AppendixC}

\backmatter
\bibliographystyle{ecs}
\bibliography{ECS}
\end{document}
%% ----------------------------------------------------------------


 %% ----------------------------------------------------------------
%% Progress.tex
%% ---------------------------------------------------------------- 
\documentclass{ecsprogress}    % Use the progress Style
\graphicspath{{../figs/}}   % Location of your graphics files
    \usepackage{natbib}            % Use Natbib style for the refs.
\hypersetup{colorlinks=true}   % Set to false for black/white printing
\input{Definitions}            % Include your abbreviations



\usepackage{enumitem}% http://ctan.org/pkg/enumitem
\usepackage{multirow}
\usepackage{float}
\usepackage{amsmath}
\usepackage{multicol}
\usepackage{amssymb}
\usepackage[normalem]{ulem}
\useunder{\uline}{\ul}{}
\usepackage{wrapfig}


\usepackage[table,xcdraw]{xcolor}


%% ----------------------------------------------------------------
\begin{document}
\frontmatter
\title      {Heterogeneous Agent-based Model for Supermarket Competition}
\authors    {\texorpdfstring
             {\href{mailto:sc22g13@ecs.soton.ac.uk}{Stefan J. Collier}}
             {Stefan J. Collier}
            }
\addresses  {\groupname\\\deptname\\\univname}
\date       {\today}
\subject    {}
\keywords   {}
\supervisor {Dr. Maria Polukarov}
\examiner   {Professor Sheng Chen}

\maketitle
\begin{abstract}
This project aim was to model and analyse the effects of competitive pricing behaviors of grocery retailers on the British market. 

This was achieved by creating a multi-agent model, containing retailer and consumer agents. The heterogeneous crowd of retailers employs either a uniform pricing strategy or a ‘local price flexing’ strategy. The actions of these retailers are chosen by predicting the profit of each action, using a perceptron. Following on from the consideration of different economic models, a discrete model was developed so that software agents have a discrete environment to operate within. Within the model, it has been observed how supermarkets with differing behaviors affect a heterogeneous crowd of consumer agents. The model was implemented in Java with Python used to evaluate the results. 

The simulation displays good acceptance with real grocery market behavior, i.e. captures the performance of British retailers thus can be used to determine the impact of changes in their behavior on their competitors and consumers.Furthermore it can be used to provide insight into sustainability of volatile pricing strategies, providing a useful insight in volatility of British supermarket retail industry. 
\end{abstract}
\acknowledgements{
I would like to express my sincere gratitude to Dr Maria Polukarov for her guidance and support which provided me the freedom to take this research in the direction of my interest.\\
\\
I would also like to thank my family and friends for their encouragement and support. To those who quietly listened to my software complaints. To those who worked throughout the nights with me. To those who helped me write what I couldn't say. I cannot thank you enough.
}

\declaration{
I, Stefan Collier, declare that this dissertation and the work presented in it are my own and has been generated by me as the result of my own original research.\\
I confirm that:\\
1. This work was done wholly or mainly while in candidature for a degree at this University;\\
2. Where any part of this dissertation has previously been submitted for any other qualification at this University or any other institution, this has been clearly stated;\\
3. Where I have consulted the published work of others, this is always clearly attributed;\\
4. Where I have quoted from the work of others, the source is always given. With the exception of such quotations, this dissertation is entirely my own work;\\
5. I have acknowledged all main sources of help;\\
6. Where the thesis is based on work done by myself jointly with others, I have made clear exactly what was done by others and what I have contributed myself;\\
7. Either none of this work has been published before submission, or parts of this work have been published by :\\
\\
Stefan Collier\\
April 2016
}
\tableofcontents
\listoffigures
\listoftables

\mainmatter
%% ----------------------------------------------------------------
%\include{Introduction}
%\include{Conclusions}
\include{chapters/1Project/main}
\include{chapters/2Lit/main}
\include{chapters/3Design/HighLevel}
\include{chapters/3Design/InDepth}
\include{chapters/4Impl/main}

\include{chapters/5Experiments/1/main}
\include{chapters/5Experiments/2/main}
\include{chapters/5Experiments/3/main}
\include{chapters/5Experiments/4/main}

\include{chapters/6Conclusion/main}

\appendix
\include{appendix/AppendixB}
\include{appendix/D/main}
\include{appendix/AppendixC}

\backmatter
\bibliographystyle{ecs}
\bibliography{ECS}
\end{document}
%% ----------------------------------------------------------------


\appendix
\include{appendix/AppendixB}
 %% ----------------------------------------------------------------
%% Progress.tex
%% ---------------------------------------------------------------- 
\documentclass{ecsprogress}    % Use the progress Style
\graphicspath{{../figs/}}   % Location of your graphics files
    \usepackage{natbib}            % Use Natbib style for the refs.
\hypersetup{colorlinks=true}   % Set to false for black/white printing
\input{Definitions}            % Include your abbreviations



\usepackage{enumitem}% http://ctan.org/pkg/enumitem
\usepackage{multirow}
\usepackage{float}
\usepackage{amsmath}
\usepackage{multicol}
\usepackage{amssymb}
\usepackage[normalem]{ulem}
\useunder{\uline}{\ul}{}
\usepackage{wrapfig}


\usepackage[table,xcdraw]{xcolor}


%% ----------------------------------------------------------------
\begin{document}
\frontmatter
\title      {Heterogeneous Agent-based Model for Supermarket Competition}
\authors    {\texorpdfstring
             {\href{mailto:sc22g13@ecs.soton.ac.uk}{Stefan J. Collier}}
             {Stefan J. Collier}
            }
\addresses  {\groupname\\\deptname\\\univname}
\date       {\today}
\subject    {}
\keywords   {}
\supervisor {Dr. Maria Polukarov}
\examiner   {Professor Sheng Chen}

\maketitle
\begin{abstract}
This project aim was to model and analyse the effects of competitive pricing behaviors of grocery retailers on the British market. 

This was achieved by creating a multi-agent model, containing retailer and consumer agents. The heterogeneous crowd of retailers employs either a uniform pricing strategy or a ‘local price flexing’ strategy. The actions of these retailers are chosen by predicting the profit of each action, using a perceptron. Following on from the consideration of different economic models, a discrete model was developed so that software agents have a discrete environment to operate within. Within the model, it has been observed how supermarkets with differing behaviors affect a heterogeneous crowd of consumer agents. The model was implemented in Java with Python used to evaluate the results. 

The simulation displays good acceptance with real grocery market behavior, i.e. captures the performance of British retailers thus can be used to determine the impact of changes in their behavior on their competitors and consumers.Furthermore it can be used to provide insight into sustainability of volatile pricing strategies, providing a useful insight in volatility of British supermarket retail industry. 
\end{abstract}
\acknowledgements{
I would like to express my sincere gratitude to Dr Maria Polukarov for her guidance and support which provided me the freedom to take this research in the direction of my interest.\\
\\
I would also like to thank my family and friends for their encouragement and support. To those who quietly listened to my software complaints. To those who worked throughout the nights with me. To those who helped me write what I couldn't say. I cannot thank you enough.
}

\declaration{
I, Stefan Collier, declare that this dissertation and the work presented in it are my own and has been generated by me as the result of my own original research.\\
I confirm that:\\
1. This work was done wholly or mainly while in candidature for a degree at this University;\\
2. Where any part of this dissertation has previously been submitted for any other qualification at this University or any other institution, this has been clearly stated;\\
3. Where I have consulted the published work of others, this is always clearly attributed;\\
4. Where I have quoted from the work of others, the source is always given. With the exception of such quotations, this dissertation is entirely my own work;\\
5. I have acknowledged all main sources of help;\\
6. Where the thesis is based on work done by myself jointly with others, I have made clear exactly what was done by others and what I have contributed myself;\\
7. Either none of this work has been published before submission, or parts of this work have been published by :\\
\\
Stefan Collier\\
April 2016
}
\tableofcontents
\listoffigures
\listoftables

\mainmatter
%% ----------------------------------------------------------------
%\include{Introduction}
%\include{Conclusions}
\include{chapters/1Project/main}
\include{chapters/2Lit/main}
\include{chapters/3Design/HighLevel}
\include{chapters/3Design/InDepth}
\include{chapters/4Impl/main}

\include{chapters/5Experiments/1/main}
\include{chapters/5Experiments/2/main}
\include{chapters/5Experiments/3/main}
\include{chapters/5Experiments/4/main}

\include{chapters/6Conclusion/main}

\appendix
\include{appendix/AppendixB}
\include{appendix/D/main}
\include{appendix/AppendixC}

\backmatter
\bibliographystyle{ecs}
\bibliography{ECS}
\end{document}
%% ----------------------------------------------------------------

\include{appendix/AppendixC}

\backmatter
\bibliographystyle{ecs}
\bibliography{ECS}
\end{document}
%% ----------------------------------------------------------------

 %% ----------------------------------------------------------------
%% Progress.tex
%% ---------------------------------------------------------------- 
\documentclass{ecsprogress}    % Use the progress Style
\graphicspath{{../figs/}}   % Location of your graphics files
    \usepackage{natbib}            % Use Natbib style for the refs.
\hypersetup{colorlinks=true}   % Set to false for black/white printing
\input{Definitions}            % Include your abbreviations



\usepackage{enumitem}% http://ctan.org/pkg/enumitem
\usepackage{multirow}
\usepackage{float}
\usepackage{amsmath}
\usepackage{multicol}
\usepackage{amssymb}
\usepackage[normalem]{ulem}
\useunder{\uline}{\ul}{}
\usepackage{wrapfig}


\usepackage[table,xcdraw]{xcolor}


%% ----------------------------------------------------------------
\begin{document}
\frontmatter
\title      {Heterogeneous Agent-based Model for Supermarket Competition}
\authors    {\texorpdfstring
             {\href{mailto:sc22g13@ecs.soton.ac.uk}{Stefan J. Collier}}
             {Stefan J. Collier}
            }
\addresses  {\groupname\\\deptname\\\univname}
\date       {\today}
\subject    {}
\keywords   {}
\supervisor {Dr. Maria Polukarov}
\examiner   {Professor Sheng Chen}

\maketitle
\begin{abstract}
This project aim was to model and analyse the effects of competitive pricing behaviors of grocery retailers on the British market. 

This was achieved by creating a multi-agent model, containing retailer and consumer agents. The heterogeneous crowd of retailers employs either a uniform pricing strategy or a ‘local price flexing’ strategy. The actions of these retailers are chosen by predicting the profit of each action, using a perceptron. Following on from the consideration of different economic models, a discrete model was developed so that software agents have a discrete environment to operate within. Within the model, it has been observed how supermarkets with differing behaviors affect a heterogeneous crowd of consumer agents. The model was implemented in Java with Python used to evaluate the results. 

The simulation displays good acceptance with real grocery market behavior, i.e. captures the performance of British retailers thus can be used to determine the impact of changes in their behavior on their competitors and consumers.Furthermore it can be used to provide insight into sustainability of volatile pricing strategies, providing a useful insight in volatility of British supermarket retail industry. 
\end{abstract}
\acknowledgements{
I would like to express my sincere gratitude to Dr Maria Polukarov for her guidance and support which provided me the freedom to take this research in the direction of my interest.\\
\\
I would also like to thank my family and friends for their encouragement and support. To those who quietly listened to my software complaints. To those who worked throughout the nights with me. To those who helped me write what I couldn't say. I cannot thank you enough.
}

\declaration{
I, Stefan Collier, declare that this dissertation and the work presented in it are my own and has been generated by me as the result of my own original research.\\
I confirm that:\\
1. This work was done wholly or mainly while in candidature for a degree at this University;\\
2. Where any part of this dissertation has previously been submitted for any other qualification at this University or any other institution, this has been clearly stated;\\
3. Where I have consulted the published work of others, this is always clearly attributed;\\
4. Where I have quoted from the work of others, the source is always given. With the exception of such quotations, this dissertation is entirely my own work;\\
5. I have acknowledged all main sources of help;\\
6. Where the thesis is based on work done by myself jointly with others, I have made clear exactly what was done by others and what I have contributed myself;\\
7. Either none of this work has been published before submission, or parts of this work have been published by :\\
\\
Stefan Collier\\
April 2016
}
\tableofcontents
\listoffigures
\listoftables

\mainmatter
%% ----------------------------------------------------------------
%\include{Introduction}
%\include{Conclusions}
 %% ----------------------------------------------------------------
%% Progress.tex
%% ---------------------------------------------------------------- 
\documentclass{ecsprogress}    % Use the progress Style
\graphicspath{{../figs/}}   % Location of your graphics files
    \usepackage{natbib}            % Use Natbib style for the refs.
\hypersetup{colorlinks=true}   % Set to false for black/white printing
\input{Definitions}            % Include your abbreviations



\usepackage{enumitem}% http://ctan.org/pkg/enumitem
\usepackage{multirow}
\usepackage{float}
\usepackage{amsmath}
\usepackage{multicol}
\usepackage{amssymb}
\usepackage[normalem]{ulem}
\useunder{\uline}{\ul}{}
\usepackage{wrapfig}


\usepackage[table,xcdraw]{xcolor}


%% ----------------------------------------------------------------
\begin{document}
\frontmatter
\title      {Heterogeneous Agent-based Model for Supermarket Competition}
\authors    {\texorpdfstring
             {\href{mailto:sc22g13@ecs.soton.ac.uk}{Stefan J. Collier}}
             {Stefan J. Collier}
            }
\addresses  {\groupname\\\deptname\\\univname}
\date       {\today}
\subject    {}
\keywords   {}
\supervisor {Dr. Maria Polukarov}
\examiner   {Professor Sheng Chen}

\maketitle
\begin{abstract}
This project aim was to model and analyse the effects of competitive pricing behaviors of grocery retailers on the British market. 

This was achieved by creating a multi-agent model, containing retailer and consumer agents. The heterogeneous crowd of retailers employs either a uniform pricing strategy or a ‘local price flexing’ strategy. The actions of these retailers are chosen by predicting the profit of each action, using a perceptron. Following on from the consideration of different economic models, a discrete model was developed so that software agents have a discrete environment to operate within. Within the model, it has been observed how supermarkets with differing behaviors affect a heterogeneous crowd of consumer agents. The model was implemented in Java with Python used to evaluate the results. 

The simulation displays good acceptance with real grocery market behavior, i.e. captures the performance of British retailers thus can be used to determine the impact of changes in their behavior on their competitors and consumers.Furthermore it can be used to provide insight into sustainability of volatile pricing strategies, providing a useful insight in volatility of British supermarket retail industry. 
\end{abstract}
\acknowledgements{
I would like to express my sincere gratitude to Dr Maria Polukarov for her guidance and support which provided me the freedom to take this research in the direction of my interest.\\
\\
I would also like to thank my family and friends for their encouragement and support. To those who quietly listened to my software complaints. To those who worked throughout the nights with me. To those who helped me write what I couldn't say. I cannot thank you enough.
}

\declaration{
I, Stefan Collier, declare that this dissertation and the work presented in it are my own and has been generated by me as the result of my own original research.\\
I confirm that:\\
1. This work was done wholly or mainly while in candidature for a degree at this University;\\
2. Where any part of this dissertation has previously been submitted for any other qualification at this University or any other institution, this has been clearly stated;\\
3. Where I have consulted the published work of others, this is always clearly attributed;\\
4. Where I have quoted from the work of others, the source is always given. With the exception of such quotations, this dissertation is entirely my own work;\\
5. I have acknowledged all main sources of help;\\
6. Where the thesis is based on work done by myself jointly with others, I have made clear exactly what was done by others and what I have contributed myself;\\
7. Either none of this work has been published before submission, or parts of this work have been published by :\\
\\
Stefan Collier\\
April 2016
}
\tableofcontents
\listoffigures
\listoftables

\mainmatter
%% ----------------------------------------------------------------
%\include{Introduction}
%\include{Conclusions}
\include{chapters/1Project/main}
\include{chapters/2Lit/main}
\include{chapters/3Design/HighLevel}
\include{chapters/3Design/InDepth}
\include{chapters/4Impl/main}

\include{chapters/5Experiments/1/main}
\include{chapters/5Experiments/2/main}
\include{chapters/5Experiments/3/main}
\include{chapters/5Experiments/4/main}

\include{chapters/6Conclusion/main}

\appendix
\include{appendix/AppendixB}
\include{appendix/D/main}
\include{appendix/AppendixC}

\backmatter
\bibliographystyle{ecs}
\bibliography{ECS}
\end{document}
%% ----------------------------------------------------------------

 %% ----------------------------------------------------------------
%% Progress.tex
%% ---------------------------------------------------------------- 
\documentclass{ecsprogress}    % Use the progress Style
\graphicspath{{../figs/}}   % Location of your graphics files
    \usepackage{natbib}            % Use Natbib style for the refs.
\hypersetup{colorlinks=true}   % Set to false for black/white printing
\input{Definitions}            % Include your abbreviations



\usepackage{enumitem}% http://ctan.org/pkg/enumitem
\usepackage{multirow}
\usepackage{float}
\usepackage{amsmath}
\usepackage{multicol}
\usepackage{amssymb}
\usepackage[normalem]{ulem}
\useunder{\uline}{\ul}{}
\usepackage{wrapfig}


\usepackage[table,xcdraw]{xcolor}


%% ----------------------------------------------------------------
\begin{document}
\frontmatter
\title      {Heterogeneous Agent-based Model for Supermarket Competition}
\authors    {\texorpdfstring
             {\href{mailto:sc22g13@ecs.soton.ac.uk}{Stefan J. Collier}}
             {Stefan J. Collier}
            }
\addresses  {\groupname\\\deptname\\\univname}
\date       {\today}
\subject    {}
\keywords   {}
\supervisor {Dr. Maria Polukarov}
\examiner   {Professor Sheng Chen}

\maketitle
\begin{abstract}
This project aim was to model and analyse the effects of competitive pricing behaviors of grocery retailers on the British market. 

This was achieved by creating a multi-agent model, containing retailer and consumer agents. The heterogeneous crowd of retailers employs either a uniform pricing strategy or a ‘local price flexing’ strategy. The actions of these retailers are chosen by predicting the profit of each action, using a perceptron. Following on from the consideration of different economic models, a discrete model was developed so that software agents have a discrete environment to operate within. Within the model, it has been observed how supermarkets with differing behaviors affect a heterogeneous crowd of consumer agents. The model was implemented in Java with Python used to evaluate the results. 

The simulation displays good acceptance with real grocery market behavior, i.e. captures the performance of British retailers thus can be used to determine the impact of changes in their behavior on their competitors and consumers.Furthermore it can be used to provide insight into sustainability of volatile pricing strategies, providing a useful insight in volatility of British supermarket retail industry. 
\end{abstract}
\acknowledgements{
I would like to express my sincere gratitude to Dr Maria Polukarov for her guidance and support which provided me the freedom to take this research in the direction of my interest.\\
\\
I would also like to thank my family and friends for their encouragement and support. To those who quietly listened to my software complaints. To those who worked throughout the nights with me. To those who helped me write what I couldn't say. I cannot thank you enough.
}

\declaration{
I, Stefan Collier, declare that this dissertation and the work presented in it are my own and has been generated by me as the result of my own original research.\\
I confirm that:\\
1. This work was done wholly or mainly while in candidature for a degree at this University;\\
2. Where any part of this dissertation has previously been submitted for any other qualification at this University or any other institution, this has been clearly stated;\\
3. Where I have consulted the published work of others, this is always clearly attributed;\\
4. Where I have quoted from the work of others, the source is always given. With the exception of such quotations, this dissertation is entirely my own work;\\
5. I have acknowledged all main sources of help;\\
6. Where the thesis is based on work done by myself jointly with others, I have made clear exactly what was done by others and what I have contributed myself;\\
7. Either none of this work has been published before submission, or parts of this work have been published by :\\
\\
Stefan Collier\\
April 2016
}
\tableofcontents
\listoffigures
\listoftables

\mainmatter
%% ----------------------------------------------------------------
%\include{Introduction}
%\include{Conclusions}
\include{chapters/1Project/main}
\include{chapters/2Lit/main}
\include{chapters/3Design/HighLevel}
\include{chapters/3Design/InDepth}
\include{chapters/4Impl/main}

\include{chapters/5Experiments/1/main}
\include{chapters/5Experiments/2/main}
\include{chapters/5Experiments/3/main}
\include{chapters/5Experiments/4/main}

\include{chapters/6Conclusion/main}

\appendix
\include{appendix/AppendixB}
\include{appendix/D/main}
\include{appendix/AppendixC}

\backmatter
\bibliographystyle{ecs}
\bibliography{ECS}
\end{document}
%% ----------------------------------------------------------------

\include{chapters/3Design/HighLevel}
\include{chapters/3Design/InDepth}
 %% ----------------------------------------------------------------
%% Progress.tex
%% ---------------------------------------------------------------- 
\documentclass{ecsprogress}    % Use the progress Style
\graphicspath{{../figs/}}   % Location of your graphics files
    \usepackage{natbib}            % Use Natbib style for the refs.
\hypersetup{colorlinks=true}   % Set to false for black/white printing
\input{Definitions}            % Include your abbreviations



\usepackage{enumitem}% http://ctan.org/pkg/enumitem
\usepackage{multirow}
\usepackage{float}
\usepackage{amsmath}
\usepackage{multicol}
\usepackage{amssymb}
\usepackage[normalem]{ulem}
\useunder{\uline}{\ul}{}
\usepackage{wrapfig}


\usepackage[table,xcdraw]{xcolor}


%% ----------------------------------------------------------------
\begin{document}
\frontmatter
\title      {Heterogeneous Agent-based Model for Supermarket Competition}
\authors    {\texorpdfstring
             {\href{mailto:sc22g13@ecs.soton.ac.uk}{Stefan J. Collier}}
             {Stefan J. Collier}
            }
\addresses  {\groupname\\\deptname\\\univname}
\date       {\today}
\subject    {}
\keywords   {}
\supervisor {Dr. Maria Polukarov}
\examiner   {Professor Sheng Chen}

\maketitle
\begin{abstract}
This project aim was to model and analyse the effects of competitive pricing behaviors of grocery retailers on the British market. 

This was achieved by creating a multi-agent model, containing retailer and consumer agents. The heterogeneous crowd of retailers employs either a uniform pricing strategy or a ‘local price flexing’ strategy. The actions of these retailers are chosen by predicting the profit of each action, using a perceptron. Following on from the consideration of different economic models, a discrete model was developed so that software agents have a discrete environment to operate within. Within the model, it has been observed how supermarkets with differing behaviors affect a heterogeneous crowd of consumer agents. The model was implemented in Java with Python used to evaluate the results. 

The simulation displays good acceptance with real grocery market behavior, i.e. captures the performance of British retailers thus can be used to determine the impact of changes in their behavior on their competitors and consumers.Furthermore it can be used to provide insight into sustainability of volatile pricing strategies, providing a useful insight in volatility of British supermarket retail industry. 
\end{abstract}
\acknowledgements{
I would like to express my sincere gratitude to Dr Maria Polukarov for her guidance and support which provided me the freedom to take this research in the direction of my interest.\\
\\
I would also like to thank my family and friends for their encouragement and support. To those who quietly listened to my software complaints. To those who worked throughout the nights with me. To those who helped me write what I couldn't say. I cannot thank you enough.
}

\declaration{
I, Stefan Collier, declare that this dissertation and the work presented in it are my own and has been generated by me as the result of my own original research.\\
I confirm that:\\
1. This work was done wholly or mainly while in candidature for a degree at this University;\\
2. Where any part of this dissertation has previously been submitted for any other qualification at this University or any other institution, this has been clearly stated;\\
3. Where I have consulted the published work of others, this is always clearly attributed;\\
4. Where I have quoted from the work of others, the source is always given. With the exception of such quotations, this dissertation is entirely my own work;\\
5. I have acknowledged all main sources of help;\\
6. Where the thesis is based on work done by myself jointly with others, I have made clear exactly what was done by others and what I have contributed myself;\\
7. Either none of this work has been published before submission, or parts of this work have been published by :\\
\\
Stefan Collier\\
April 2016
}
\tableofcontents
\listoffigures
\listoftables

\mainmatter
%% ----------------------------------------------------------------
%\include{Introduction}
%\include{Conclusions}
\include{chapters/1Project/main}
\include{chapters/2Lit/main}
\include{chapters/3Design/HighLevel}
\include{chapters/3Design/InDepth}
\include{chapters/4Impl/main}

\include{chapters/5Experiments/1/main}
\include{chapters/5Experiments/2/main}
\include{chapters/5Experiments/3/main}
\include{chapters/5Experiments/4/main}

\include{chapters/6Conclusion/main}

\appendix
\include{appendix/AppendixB}
\include{appendix/D/main}
\include{appendix/AppendixC}

\backmatter
\bibliographystyle{ecs}
\bibliography{ECS}
\end{document}
%% ----------------------------------------------------------------


 %% ----------------------------------------------------------------
%% Progress.tex
%% ---------------------------------------------------------------- 
\documentclass{ecsprogress}    % Use the progress Style
\graphicspath{{../figs/}}   % Location of your graphics files
    \usepackage{natbib}            % Use Natbib style for the refs.
\hypersetup{colorlinks=true}   % Set to false for black/white printing
\input{Definitions}            % Include your abbreviations



\usepackage{enumitem}% http://ctan.org/pkg/enumitem
\usepackage{multirow}
\usepackage{float}
\usepackage{amsmath}
\usepackage{multicol}
\usepackage{amssymb}
\usepackage[normalem]{ulem}
\useunder{\uline}{\ul}{}
\usepackage{wrapfig}


\usepackage[table,xcdraw]{xcolor}


%% ----------------------------------------------------------------
\begin{document}
\frontmatter
\title      {Heterogeneous Agent-based Model for Supermarket Competition}
\authors    {\texorpdfstring
             {\href{mailto:sc22g13@ecs.soton.ac.uk}{Stefan J. Collier}}
             {Stefan J. Collier}
            }
\addresses  {\groupname\\\deptname\\\univname}
\date       {\today}
\subject    {}
\keywords   {}
\supervisor {Dr. Maria Polukarov}
\examiner   {Professor Sheng Chen}

\maketitle
\begin{abstract}
This project aim was to model and analyse the effects of competitive pricing behaviors of grocery retailers on the British market. 

This was achieved by creating a multi-agent model, containing retailer and consumer agents. The heterogeneous crowd of retailers employs either a uniform pricing strategy or a ‘local price flexing’ strategy. The actions of these retailers are chosen by predicting the profit of each action, using a perceptron. Following on from the consideration of different economic models, a discrete model was developed so that software agents have a discrete environment to operate within. Within the model, it has been observed how supermarkets with differing behaviors affect a heterogeneous crowd of consumer agents. The model was implemented in Java with Python used to evaluate the results. 

The simulation displays good acceptance with real grocery market behavior, i.e. captures the performance of British retailers thus can be used to determine the impact of changes in their behavior on their competitors and consumers.Furthermore it can be used to provide insight into sustainability of volatile pricing strategies, providing a useful insight in volatility of British supermarket retail industry. 
\end{abstract}
\acknowledgements{
I would like to express my sincere gratitude to Dr Maria Polukarov for her guidance and support which provided me the freedom to take this research in the direction of my interest.\\
\\
I would also like to thank my family and friends for their encouragement and support. To those who quietly listened to my software complaints. To those who worked throughout the nights with me. To those who helped me write what I couldn't say. I cannot thank you enough.
}

\declaration{
I, Stefan Collier, declare that this dissertation and the work presented in it are my own and has been generated by me as the result of my own original research.\\
I confirm that:\\
1. This work was done wholly or mainly while in candidature for a degree at this University;\\
2. Where any part of this dissertation has previously been submitted for any other qualification at this University or any other institution, this has been clearly stated;\\
3. Where I have consulted the published work of others, this is always clearly attributed;\\
4. Where I have quoted from the work of others, the source is always given. With the exception of such quotations, this dissertation is entirely my own work;\\
5. I have acknowledged all main sources of help;\\
6. Where the thesis is based on work done by myself jointly with others, I have made clear exactly what was done by others and what I have contributed myself;\\
7. Either none of this work has been published before submission, or parts of this work have been published by :\\
\\
Stefan Collier\\
April 2016
}
\tableofcontents
\listoffigures
\listoftables

\mainmatter
%% ----------------------------------------------------------------
%\include{Introduction}
%\include{Conclusions}
\include{chapters/1Project/main}
\include{chapters/2Lit/main}
\include{chapters/3Design/HighLevel}
\include{chapters/3Design/InDepth}
\include{chapters/4Impl/main}

\include{chapters/5Experiments/1/main}
\include{chapters/5Experiments/2/main}
\include{chapters/5Experiments/3/main}
\include{chapters/5Experiments/4/main}

\include{chapters/6Conclusion/main}

\appendix
\include{appendix/AppendixB}
\include{appendix/D/main}
\include{appendix/AppendixC}

\backmatter
\bibliographystyle{ecs}
\bibliography{ECS}
\end{document}
%% ----------------------------------------------------------------

 %% ----------------------------------------------------------------
%% Progress.tex
%% ---------------------------------------------------------------- 
\documentclass{ecsprogress}    % Use the progress Style
\graphicspath{{../figs/}}   % Location of your graphics files
    \usepackage{natbib}            % Use Natbib style for the refs.
\hypersetup{colorlinks=true}   % Set to false for black/white printing
\input{Definitions}            % Include your abbreviations



\usepackage{enumitem}% http://ctan.org/pkg/enumitem
\usepackage{multirow}
\usepackage{float}
\usepackage{amsmath}
\usepackage{multicol}
\usepackage{amssymb}
\usepackage[normalem]{ulem}
\useunder{\uline}{\ul}{}
\usepackage{wrapfig}


\usepackage[table,xcdraw]{xcolor}


%% ----------------------------------------------------------------
\begin{document}
\frontmatter
\title      {Heterogeneous Agent-based Model for Supermarket Competition}
\authors    {\texorpdfstring
             {\href{mailto:sc22g13@ecs.soton.ac.uk}{Stefan J. Collier}}
             {Stefan J. Collier}
            }
\addresses  {\groupname\\\deptname\\\univname}
\date       {\today}
\subject    {}
\keywords   {}
\supervisor {Dr. Maria Polukarov}
\examiner   {Professor Sheng Chen}

\maketitle
\begin{abstract}
This project aim was to model and analyse the effects of competitive pricing behaviors of grocery retailers on the British market. 

This was achieved by creating a multi-agent model, containing retailer and consumer agents. The heterogeneous crowd of retailers employs either a uniform pricing strategy or a ‘local price flexing’ strategy. The actions of these retailers are chosen by predicting the profit of each action, using a perceptron. Following on from the consideration of different economic models, a discrete model was developed so that software agents have a discrete environment to operate within. Within the model, it has been observed how supermarkets with differing behaviors affect a heterogeneous crowd of consumer agents. The model was implemented in Java with Python used to evaluate the results. 

The simulation displays good acceptance with real grocery market behavior, i.e. captures the performance of British retailers thus can be used to determine the impact of changes in their behavior on their competitors and consumers.Furthermore it can be used to provide insight into sustainability of volatile pricing strategies, providing a useful insight in volatility of British supermarket retail industry. 
\end{abstract}
\acknowledgements{
I would like to express my sincere gratitude to Dr Maria Polukarov for her guidance and support which provided me the freedom to take this research in the direction of my interest.\\
\\
I would also like to thank my family and friends for their encouragement and support. To those who quietly listened to my software complaints. To those who worked throughout the nights with me. To those who helped me write what I couldn't say. I cannot thank you enough.
}

\declaration{
I, Stefan Collier, declare that this dissertation and the work presented in it are my own and has been generated by me as the result of my own original research.\\
I confirm that:\\
1. This work was done wholly or mainly while in candidature for a degree at this University;\\
2. Where any part of this dissertation has previously been submitted for any other qualification at this University or any other institution, this has been clearly stated;\\
3. Where I have consulted the published work of others, this is always clearly attributed;\\
4. Where I have quoted from the work of others, the source is always given. With the exception of such quotations, this dissertation is entirely my own work;\\
5. I have acknowledged all main sources of help;\\
6. Where the thesis is based on work done by myself jointly with others, I have made clear exactly what was done by others and what I have contributed myself;\\
7. Either none of this work has been published before submission, or parts of this work have been published by :\\
\\
Stefan Collier\\
April 2016
}
\tableofcontents
\listoffigures
\listoftables

\mainmatter
%% ----------------------------------------------------------------
%\include{Introduction}
%\include{Conclusions}
\include{chapters/1Project/main}
\include{chapters/2Lit/main}
\include{chapters/3Design/HighLevel}
\include{chapters/3Design/InDepth}
\include{chapters/4Impl/main}

\include{chapters/5Experiments/1/main}
\include{chapters/5Experiments/2/main}
\include{chapters/5Experiments/3/main}
\include{chapters/5Experiments/4/main}

\include{chapters/6Conclusion/main}

\appendix
\include{appendix/AppendixB}
\include{appendix/D/main}
\include{appendix/AppendixC}

\backmatter
\bibliographystyle{ecs}
\bibliography{ECS}
\end{document}
%% ----------------------------------------------------------------

 %% ----------------------------------------------------------------
%% Progress.tex
%% ---------------------------------------------------------------- 
\documentclass{ecsprogress}    % Use the progress Style
\graphicspath{{../figs/}}   % Location of your graphics files
    \usepackage{natbib}            % Use Natbib style for the refs.
\hypersetup{colorlinks=true}   % Set to false for black/white printing
\input{Definitions}            % Include your abbreviations



\usepackage{enumitem}% http://ctan.org/pkg/enumitem
\usepackage{multirow}
\usepackage{float}
\usepackage{amsmath}
\usepackage{multicol}
\usepackage{amssymb}
\usepackage[normalem]{ulem}
\useunder{\uline}{\ul}{}
\usepackage{wrapfig}


\usepackage[table,xcdraw]{xcolor}


%% ----------------------------------------------------------------
\begin{document}
\frontmatter
\title      {Heterogeneous Agent-based Model for Supermarket Competition}
\authors    {\texorpdfstring
             {\href{mailto:sc22g13@ecs.soton.ac.uk}{Stefan J. Collier}}
             {Stefan J. Collier}
            }
\addresses  {\groupname\\\deptname\\\univname}
\date       {\today}
\subject    {}
\keywords   {}
\supervisor {Dr. Maria Polukarov}
\examiner   {Professor Sheng Chen}

\maketitle
\begin{abstract}
This project aim was to model and analyse the effects of competitive pricing behaviors of grocery retailers on the British market. 

This was achieved by creating a multi-agent model, containing retailer and consumer agents. The heterogeneous crowd of retailers employs either a uniform pricing strategy or a ‘local price flexing’ strategy. The actions of these retailers are chosen by predicting the profit of each action, using a perceptron. Following on from the consideration of different economic models, a discrete model was developed so that software agents have a discrete environment to operate within. Within the model, it has been observed how supermarkets with differing behaviors affect a heterogeneous crowd of consumer agents. The model was implemented in Java with Python used to evaluate the results. 

The simulation displays good acceptance with real grocery market behavior, i.e. captures the performance of British retailers thus can be used to determine the impact of changes in their behavior on their competitors and consumers.Furthermore it can be used to provide insight into sustainability of volatile pricing strategies, providing a useful insight in volatility of British supermarket retail industry. 
\end{abstract}
\acknowledgements{
I would like to express my sincere gratitude to Dr Maria Polukarov for her guidance and support which provided me the freedom to take this research in the direction of my interest.\\
\\
I would also like to thank my family and friends for their encouragement and support. To those who quietly listened to my software complaints. To those who worked throughout the nights with me. To those who helped me write what I couldn't say. I cannot thank you enough.
}

\declaration{
I, Stefan Collier, declare that this dissertation and the work presented in it are my own and has been generated by me as the result of my own original research.\\
I confirm that:\\
1. This work was done wholly or mainly while in candidature for a degree at this University;\\
2. Where any part of this dissertation has previously been submitted for any other qualification at this University or any other institution, this has been clearly stated;\\
3. Where I have consulted the published work of others, this is always clearly attributed;\\
4. Where I have quoted from the work of others, the source is always given. With the exception of such quotations, this dissertation is entirely my own work;\\
5. I have acknowledged all main sources of help;\\
6. Where the thesis is based on work done by myself jointly with others, I have made clear exactly what was done by others and what I have contributed myself;\\
7. Either none of this work has been published before submission, or parts of this work have been published by :\\
\\
Stefan Collier\\
April 2016
}
\tableofcontents
\listoffigures
\listoftables

\mainmatter
%% ----------------------------------------------------------------
%\include{Introduction}
%\include{Conclusions}
\include{chapters/1Project/main}
\include{chapters/2Lit/main}
\include{chapters/3Design/HighLevel}
\include{chapters/3Design/InDepth}
\include{chapters/4Impl/main}

\include{chapters/5Experiments/1/main}
\include{chapters/5Experiments/2/main}
\include{chapters/5Experiments/3/main}
\include{chapters/5Experiments/4/main}

\include{chapters/6Conclusion/main}

\appendix
\include{appendix/AppendixB}
\include{appendix/D/main}
\include{appendix/AppendixC}

\backmatter
\bibliographystyle{ecs}
\bibliography{ECS}
\end{document}
%% ----------------------------------------------------------------

 %% ----------------------------------------------------------------
%% Progress.tex
%% ---------------------------------------------------------------- 
\documentclass{ecsprogress}    % Use the progress Style
\graphicspath{{../figs/}}   % Location of your graphics files
    \usepackage{natbib}            % Use Natbib style for the refs.
\hypersetup{colorlinks=true}   % Set to false for black/white printing
\input{Definitions}            % Include your abbreviations



\usepackage{enumitem}% http://ctan.org/pkg/enumitem
\usepackage{multirow}
\usepackage{float}
\usepackage{amsmath}
\usepackage{multicol}
\usepackage{amssymb}
\usepackage[normalem]{ulem}
\useunder{\uline}{\ul}{}
\usepackage{wrapfig}


\usepackage[table,xcdraw]{xcolor}


%% ----------------------------------------------------------------
\begin{document}
\frontmatter
\title      {Heterogeneous Agent-based Model for Supermarket Competition}
\authors    {\texorpdfstring
             {\href{mailto:sc22g13@ecs.soton.ac.uk}{Stefan J. Collier}}
             {Stefan J. Collier}
            }
\addresses  {\groupname\\\deptname\\\univname}
\date       {\today}
\subject    {}
\keywords   {}
\supervisor {Dr. Maria Polukarov}
\examiner   {Professor Sheng Chen}

\maketitle
\begin{abstract}
This project aim was to model and analyse the effects of competitive pricing behaviors of grocery retailers on the British market. 

This was achieved by creating a multi-agent model, containing retailer and consumer agents. The heterogeneous crowd of retailers employs either a uniform pricing strategy or a ‘local price flexing’ strategy. The actions of these retailers are chosen by predicting the profit of each action, using a perceptron. Following on from the consideration of different economic models, a discrete model was developed so that software agents have a discrete environment to operate within. Within the model, it has been observed how supermarkets with differing behaviors affect a heterogeneous crowd of consumer agents. The model was implemented in Java with Python used to evaluate the results. 

The simulation displays good acceptance with real grocery market behavior, i.e. captures the performance of British retailers thus can be used to determine the impact of changes in their behavior on their competitors and consumers.Furthermore it can be used to provide insight into sustainability of volatile pricing strategies, providing a useful insight in volatility of British supermarket retail industry. 
\end{abstract}
\acknowledgements{
I would like to express my sincere gratitude to Dr Maria Polukarov for her guidance and support which provided me the freedom to take this research in the direction of my interest.\\
\\
I would also like to thank my family and friends for their encouragement and support. To those who quietly listened to my software complaints. To those who worked throughout the nights with me. To those who helped me write what I couldn't say. I cannot thank you enough.
}

\declaration{
I, Stefan Collier, declare that this dissertation and the work presented in it are my own and has been generated by me as the result of my own original research.\\
I confirm that:\\
1. This work was done wholly or mainly while in candidature for a degree at this University;\\
2. Where any part of this dissertation has previously been submitted for any other qualification at this University or any other institution, this has been clearly stated;\\
3. Where I have consulted the published work of others, this is always clearly attributed;\\
4. Where I have quoted from the work of others, the source is always given. With the exception of such quotations, this dissertation is entirely my own work;\\
5. I have acknowledged all main sources of help;\\
6. Where the thesis is based on work done by myself jointly with others, I have made clear exactly what was done by others and what I have contributed myself;\\
7. Either none of this work has been published before submission, or parts of this work have been published by :\\
\\
Stefan Collier\\
April 2016
}
\tableofcontents
\listoffigures
\listoftables

\mainmatter
%% ----------------------------------------------------------------
%\include{Introduction}
%\include{Conclusions}
\include{chapters/1Project/main}
\include{chapters/2Lit/main}
\include{chapters/3Design/HighLevel}
\include{chapters/3Design/InDepth}
\include{chapters/4Impl/main}

\include{chapters/5Experiments/1/main}
\include{chapters/5Experiments/2/main}
\include{chapters/5Experiments/3/main}
\include{chapters/5Experiments/4/main}

\include{chapters/6Conclusion/main}

\appendix
\include{appendix/AppendixB}
\include{appendix/D/main}
\include{appendix/AppendixC}

\backmatter
\bibliographystyle{ecs}
\bibliography{ECS}
\end{document}
%% ----------------------------------------------------------------


 %% ----------------------------------------------------------------
%% Progress.tex
%% ---------------------------------------------------------------- 
\documentclass{ecsprogress}    % Use the progress Style
\graphicspath{{../figs/}}   % Location of your graphics files
    \usepackage{natbib}            % Use Natbib style for the refs.
\hypersetup{colorlinks=true}   % Set to false for black/white printing
\input{Definitions}            % Include your abbreviations



\usepackage{enumitem}% http://ctan.org/pkg/enumitem
\usepackage{multirow}
\usepackage{float}
\usepackage{amsmath}
\usepackage{multicol}
\usepackage{amssymb}
\usepackage[normalem]{ulem}
\useunder{\uline}{\ul}{}
\usepackage{wrapfig}


\usepackage[table,xcdraw]{xcolor}


%% ----------------------------------------------------------------
\begin{document}
\frontmatter
\title      {Heterogeneous Agent-based Model for Supermarket Competition}
\authors    {\texorpdfstring
             {\href{mailto:sc22g13@ecs.soton.ac.uk}{Stefan J. Collier}}
             {Stefan J. Collier}
            }
\addresses  {\groupname\\\deptname\\\univname}
\date       {\today}
\subject    {}
\keywords   {}
\supervisor {Dr. Maria Polukarov}
\examiner   {Professor Sheng Chen}

\maketitle
\begin{abstract}
This project aim was to model and analyse the effects of competitive pricing behaviors of grocery retailers on the British market. 

This was achieved by creating a multi-agent model, containing retailer and consumer agents. The heterogeneous crowd of retailers employs either a uniform pricing strategy or a ‘local price flexing’ strategy. The actions of these retailers are chosen by predicting the profit of each action, using a perceptron. Following on from the consideration of different economic models, a discrete model was developed so that software agents have a discrete environment to operate within. Within the model, it has been observed how supermarkets with differing behaviors affect a heterogeneous crowd of consumer agents. The model was implemented in Java with Python used to evaluate the results. 

The simulation displays good acceptance with real grocery market behavior, i.e. captures the performance of British retailers thus can be used to determine the impact of changes in their behavior on their competitors and consumers.Furthermore it can be used to provide insight into sustainability of volatile pricing strategies, providing a useful insight in volatility of British supermarket retail industry. 
\end{abstract}
\acknowledgements{
I would like to express my sincere gratitude to Dr Maria Polukarov for her guidance and support which provided me the freedom to take this research in the direction of my interest.\\
\\
I would also like to thank my family and friends for their encouragement and support. To those who quietly listened to my software complaints. To those who worked throughout the nights with me. To those who helped me write what I couldn't say. I cannot thank you enough.
}

\declaration{
I, Stefan Collier, declare that this dissertation and the work presented in it are my own and has been generated by me as the result of my own original research.\\
I confirm that:\\
1. This work was done wholly or mainly while in candidature for a degree at this University;\\
2. Where any part of this dissertation has previously been submitted for any other qualification at this University or any other institution, this has been clearly stated;\\
3. Where I have consulted the published work of others, this is always clearly attributed;\\
4. Where I have quoted from the work of others, the source is always given. With the exception of such quotations, this dissertation is entirely my own work;\\
5. I have acknowledged all main sources of help;\\
6. Where the thesis is based on work done by myself jointly with others, I have made clear exactly what was done by others and what I have contributed myself;\\
7. Either none of this work has been published before submission, or parts of this work have been published by :\\
\\
Stefan Collier\\
April 2016
}
\tableofcontents
\listoffigures
\listoftables

\mainmatter
%% ----------------------------------------------------------------
%\include{Introduction}
%\include{Conclusions}
\include{chapters/1Project/main}
\include{chapters/2Lit/main}
\include{chapters/3Design/HighLevel}
\include{chapters/3Design/InDepth}
\include{chapters/4Impl/main}

\include{chapters/5Experiments/1/main}
\include{chapters/5Experiments/2/main}
\include{chapters/5Experiments/3/main}
\include{chapters/5Experiments/4/main}

\include{chapters/6Conclusion/main}

\appendix
\include{appendix/AppendixB}
\include{appendix/D/main}
\include{appendix/AppendixC}

\backmatter
\bibliographystyle{ecs}
\bibliography{ECS}
\end{document}
%% ----------------------------------------------------------------


\appendix
\include{appendix/AppendixB}
 %% ----------------------------------------------------------------
%% Progress.tex
%% ---------------------------------------------------------------- 
\documentclass{ecsprogress}    % Use the progress Style
\graphicspath{{../figs/}}   % Location of your graphics files
    \usepackage{natbib}            % Use Natbib style for the refs.
\hypersetup{colorlinks=true}   % Set to false for black/white printing
\input{Definitions}            % Include your abbreviations



\usepackage{enumitem}% http://ctan.org/pkg/enumitem
\usepackage{multirow}
\usepackage{float}
\usepackage{amsmath}
\usepackage{multicol}
\usepackage{amssymb}
\usepackage[normalem]{ulem}
\useunder{\uline}{\ul}{}
\usepackage{wrapfig}


\usepackage[table,xcdraw]{xcolor}


%% ----------------------------------------------------------------
\begin{document}
\frontmatter
\title      {Heterogeneous Agent-based Model for Supermarket Competition}
\authors    {\texorpdfstring
             {\href{mailto:sc22g13@ecs.soton.ac.uk}{Stefan J. Collier}}
             {Stefan J. Collier}
            }
\addresses  {\groupname\\\deptname\\\univname}
\date       {\today}
\subject    {}
\keywords   {}
\supervisor {Dr. Maria Polukarov}
\examiner   {Professor Sheng Chen}

\maketitle
\begin{abstract}
This project aim was to model and analyse the effects of competitive pricing behaviors of grocery retailers on the British market. 

This was achieved by creating a multi-agent model, containing retailer and consumer agents. The heterogeneous crowd of retailers employs either a uniform pricing strategy or a ‘local price flexing’ strategy. The actions of these retailers are chosen by predicting the profit of each action, using a perceptron. Following on from the consideration of different economic models, a discrete model was developed so that software agents have a discrete environment to operate within. Within the model, it has been observed how supermarkets with differing behaviors affect a heterogeneous crowd of consumer agents. The model was implemented in Java with Python used to evaluate the results. 

The simulation displays good acceptance with real grocery market behavior, i.e. captures the performance of British retailers thus can be used to determine the impact of changes in their behavior on their competitors and consumers.Furthermore it can be used to provide insight into sustainability of volatile pricing strategies, providing a useful insight in volatility of British supermarket retail industry. 
\end{abstract}
\acknowledgements{
I would like to express my sincere gratitude to Dr Maria Polukarov for her guidance and support which provided me the freedom to take this research in the direction of my interest.\\
\\
I would also like to thank my family and friends for their encouragement and support. To those who quietly listened to my software complaints. To those who worked throughout the nights with me. To those who helped me write what I couldn't say. I cannot thank you enough.
}

\declaration{
I, Stefan Collier, declare that this dissertation and the work presented in it are my own and has been generated by me as the result of my own original research.\\
I confirm that:\\
1. This work was done wholly or mainly while in candidature for a degree at this University;\\
2. Where any part of this dissertation has previously been submitted for any other qualification at this University or any other institution, this has been clearly stated;\\
3. Where I have consulted the published work of others, this is always clearly attributed;\\
4. Where I have quoted from the work of others, the source is always given. With the exception of such quotations, this dissertation is entirely my own work;\\
5. I have acknowledged all main sources of help;\\
6. Where the thesis is based on work done by myself jointly with others, I have made clear exactly what was done by others and what I have contributed myself;\\
7. Either none of this work has been published before submission, or parts of this work have been published by :\\
\\
Stefan Collier\\
April 2016
}
\tableofcontents
\listoffigures
\listoftables

\mainmatter
%% ----------------------------------------------------------------
%\include{Introduction}
%\include{Conclusions}
\include{chapters/1Project/main}
\include{chapters/2Lit/main}
\include{chapters/3Design/HighLevel}
\include{chapters/3Design/InDepth}
\include{chapters/4Impl/main}

\include{chapters/5Experiments/1/main}
\include{chapters/5Experiments/2/main}
\include{chapters/5Experiments/3/main}
\include{chapters/5Experiments/4/main}

\include{chapters/6Conclusion/main}

\appendix
\include{appendix/AppendixB}
\include{appendix/D/main}
\include{appendix/AppendixC}

\backmatter
\bibliographystyle{ecs}
\bibliography{ECS}
\end{document}
%% ----------------------------------------------------------------

\include{appendix/AppendixC}

\backmatter
\bibliographystyle{ecs}
\bibliography{ECS}
\end{document}
%% ----------------------------------------------------------------

 %% ----------------------------------------------------------------
%% Progress.tex
%% ---------------------------------------------------------------- 
\documentclass{ecsprogress}    % Use the progress Style
\graphicspath{{../figs/}}   % Location of your graphics files
    \usepackage{natbib}            % Use Natbib style for the refs.
\hypersetup{colorlinks=true}   % Set to false for black/white printing
\input{Definitions}            % Include your abbreviations



\usepackage{enumitem}% http://ctan.org/pkg/enumitem
\usepackage{multirow}
\usepackage{float}
\usepackage{amsmath}
\usepackage{multicol}
\usepackage{amssymb}
\usepackage[normalem]{ulem}
\useunder{\uline}{\ul}{}
\usepackage{wrapfig}


\usepackage[table,xcdraw]{xcolor}


%% ----------------------------------------------------------------
\begin{document}
\frontmatter
\title      {Heterogeneous Agent-based Model for Supermarket Competition}
\authors    {\texorpdfstring
             {\href{mailto:sc22g13@ecs.soton.ac.uk}{Stefan J. Collier}}
             {Stefan J. Collier}
            }
\addresses  {\groupname\\\deptname\\\univname}
\date       {\today}
\subject    {}
\keywords   {}
\supervisor {Dr. Maria Polukarov}
\examiner   {Professor Sheng Chen}

\maketitle
\begin{abstract}
This project aim was to model and analyse the effects of competitive pricing behaviors of grocery retailers on the British market. 

This was achieved by creating a multi-agent model, containing retailer and consumer agents. The heterogeneous crowd of retailers employs either a uniform pricing strategy or a ‘local price flexing’ strategy. The actions of these retailers are chosen by predicting the profit of each action, using a perceptron. Following on from the consideration of different economic models, a discrete model was developed so that software agents have a discrete environment to operate within. Within the model, it has been observed how supermarkets with differing behaviors affect a heterogeneous crowd of consumer agents. The model was implemented in Java with Python used to evaluate the results. 

The simulation displays good acceptance with real grocery market behavior, i.e. captures the performance of British retailers thus can be used to determine the impact of changes in their behavior on their competitors and consumers.Furthermore it can be used to provide insight into sustainability of volatile pricing strategies, providing a useful insight in volatility of British supermarket retail industry. 
\end{abstract}
\acknowledgements{
I would like to express my sincere gratitude to Dr Maria Polukarov for her guidance and support which provided me the freedom to take this research in the direction of my interest.\\
\\
I would also like to thank my family and friends for their encouragement and support. To those who quietly listened to my software complaints. To those who worked throughout the nights with me. To those who helped me write what I couldn't say. I cannot thank you enough.
}

\declaration{
I, Stefan Collier, declare that this dissertation and the work presented in it are my own and has been generated by me as the result of my own original research.\\
I confirm that:\\
1. This work was done wholly or mainly while in candidature for a degree at this University;\\
2. Where any part of this dissertation has previously been submitted for any other qualification at this University or any other institution, this has been clearly stated;\\
3. Where I have consulted the published work of others, this is always clearly attributed;\\
4. Where I have quoted from the work of others, the source is always given. With the exception of such quotations, this dissertation is entirely my own work;\\
5. I have acknowledged all main sources of help;\\
6. Where the thesis is based on work done by myself jointly with others, I have made clear exactly what was done by others and what I have contributed myself;\\
7. Either none of this work has been published before submission, or parts of this work have been published by :\\
\\
Stefan Collier\\
April 2016
}
\tableofcontents
\listoffigures
\listoftables

\mainmatter
%% ----------------------------------------------------------------
%\include{Introduction}
%\include{Conclusions}
 %% ----------------------------------------------------------------
%% Progress.tex
%% ---------------------------------------------------------------- 
\documentclass{ecsprogress}    % Use the progress Style
\graphicspath{{../figs/}}   % Location of your graphics files
    \usepackage{natbib}            % Use Natbib style for the refs.
\hypersetup{colorlinks=true}   % Set to false for black/white printing
\input{Definitions}            % Include your abbreviations



\usepackage{enumitem}% http://ctan.org/pkg/enumitem
\usepackage{multirow}
\usepackage{float}
\usepackage{amsmath}
\usepackage{multicol}
\usepackage{amssymb}
\usepackage[normalem]{ulem}
\useunder{\uline}{\ul}{}
\usepackage{wrapfig}


\usepackage[table,xcdraw]{xcolor}


%% ----------------------------------------------------------------
\begin{document}
\frontmatter
\title      {Heterogeneous Agent-based Model for Supermarket Competition}
\authors    {\texorpdfstring
             {\href{mailto:sc22g13@ecs.soton.ac.uk}{Stefan J. Collier}}
             {Stefan J. Collier}
            }
\addresses  {\groupname\\\deptname\\\univname}
\date       {\today}
\subject    {}
\keywords   {}
\supervisor {Dr. Maria Polukarov}
\examiner   {Professor Sheng Chen}

\maketitle
\begin{abstract}
This project aim was to model and analyse the effects of competitive pricing behaviors of grocery retailers on the British market. 

This was achieved by creating a multi-agent model, containing retailer and consumer agents. The heterogeneous crowd of retailers employs either a uniform pricing strategy or a ‘local price flexing’ strategy. The actions of these retailers are chosen by predicting the profit of each action, using a perceptron. Following on from the consideration of different economic models, a discrete model was developed so that software agents have a discrete environment to operate within. Within the model, it has been observed how supermarkets with differing behaviors affect a heterogeneous crowd of consumer agents. The model was implemented in Java with Python used to evaluate the results. 

The simulation displays good acceptance with real grocery market behavior, i.e. captures the performance of British retailers thus can be used to determine the impact of changes in their behavior on their competitors and consumers.Furthermore it can be used to provide insight into sustainability of volatile pricing strategies, providing a useful insight in volatility of British supermarket retail industry. 
\end{abstract}
\acknowledgements{
I would like to express my sincere gratitude to Dr Maria Polukarov for her guidance and support which provided me the freedom to take this research in the direction of my interest.\\
\\
I would also like to thank my family and friends for their encouragement and support. To those who quietly listened to my software complaints. To those who worked throughout the nights with me. To those who helped me write what I couldn't say. I cannot thank you enough.
}

\declaration{
I, Stefan Collier, declare that this dissertation and the work presented in it are my own and has been generated by me as the result of my own original research.\\
I confirm that:\\
1. This work was done wholly or mainly while in candidature for a degree at this University;\\
2. Where any part of this dissertation has previously been submitted for any other qualification at this University or any other institution, this has been clearly stated;\\
3. Where I have consulted the published work of others, this is always clearly attributed;\\
4. Where I have quoted from the work of others, the source is always given. With the exception of such quotations, this dissertation is entirely my own work;\\
5. I have acknowledged all main sources of help;\\
6. Where the thesis is based on work done by myself jointly with others, I have made clear exactly what was done by others and what I have contributed myself;\\
7. Either none of this work has been published before submission, or parts of this work have been published by :\\
\\
Stefan Collier\\
April 2016
}
\tableofcontents
\listoffigures
\listoftables

\mainmatter
%% ----------------------------------------------------------------
%\include{Introduction}
%\include{Conclusions}
\include{chapters/1Project/main}
\include{chapters/2Lit/main}
\include{chapters/3Design/HighLevel}
\include{chapters/3Design/InDepth}
\include{chapters/4Impl/main}

\include{chapters/5Experiments/1/main}
\include{chapters/5Experiments/2/main}
\include{chapters/5Experiments/3/main}
\include{chapters/5Experiments/4/main}

\include{chapters/6Conclusion/main}

\appendix
\include{appendix/AppendixB}
\include{appendix/D/main}
\include{appendix/AppendixC}

\backmatter
\bibliographystyle{ecs}
\bibliography{ECS}
\end{document}
%% ----------------------------------------------------------------

 %% ----------------------------------------------------------------
%% Progress.tex
%% ---------------------------------------------------------------- 
\documentclass{ecsprogress}    % Use the progress Style
\graphicspath{{../figs/}}   % Location of your graphics files
    \usepackage{natbib}            % Use Natbib style for the refs.
\hypersetup{colorlinks=true}   % Set to false for black/white printing
\input{Definitions}            % Include your abbreviations



\usepackage{enumitem}% http://ctan.org/pkg/enumitem
\usepackage{multirow}
\usepackage{float}
\usepackage{amsmath}
\usepackage{multicol}
\usepackage{amssymb}
\usepackage[normalem]{ulem}
\useunder{\uline}{\ul}{}
\usepackage{wrapfig}


\usepackage[table,xcdraw]{xcolor}


%% ----------------------------------------------------------------
\begin{document}
\frontmatter
\title      {Heterogeneous Agent-based Model for Supermarket Competition}
\authors    {\texorpdfstring
             {\href{mailto:sc22g13@ecs.soton.ac.uk}{Stefan J. Collier}}
             {Stefan J. Collier}
            }
\addresses  {\groupname\\\deptname\\\univname}
\date       {\today}
\subject    {}
\keywords   {}
\supervisor {Dr. Maria Polukarov}
\examiner   {Professor Sheng Chen}

\maketitle
\begin{abstract}
This project aim was to model and analyse the effects of competitive pricing behaviors of grocery retailers on the British market. 

This was achieved by creating a multi-agent model, containing retailer and consumer agents. The heterogeneous crowd of retailers employs either a uniform pricing strategy or a ‘local price flexing’ strategy. The actions of these retailers are chosen by predicting the profit of each action, using a perceptron. Following on from the consideration of different economic models, a discrete model was developed so that software agents have a discrete environment to operate within. Within the model, it has been observed how supermarkets with differing behaviors affect a heterogeneous crowd of consumer agents. The model was implemented in Java with Python used to evaluate the results. 

The simulation displays good acceptance with real grocery market behavior, i.e. captures the performance of British retailers thus can be used to determine the impact of changes in their behavior on their competitors and consumers.Furthermore it can be used to provide insight into sustainability of volatile pricing strategies, providing a useful insight in volatility of British supermarket retail industry. 
\end{abstract}
\acknowledgements{
I would like to express my sincere gratitude to Dr Maria Polukarov for her guidance and support which provided me the freedom to take this research in the direction of my interest.\\
\\
I would also like to thank my family and friends for their encouragement and support. To those who quietly listened to my software complaints. To those who worked throughout the nights with me. To those who helped me write what I couldn't say. I cannot thank you enough.
}

\declaration{
I, Stefan Collier, declare that this dissertation and the work presented in it are my own and has been generated by me as the result of my own original research.\\
I confirm that:\\
1. This work was done wholly or mainly while in candidature for a degree at this University;\\
2. Where any part of this dissertation has previously been submitted for any other qualification at this University or any other institution, this has been clearly stated;\\
3. Where I have consulted the published work of others, this is always clearly attributed;\\
4. Where I have quoted from the work of others, the source is always given. With the exception of such quotations, this dissertation is entirely my own work;\\
5. I have acknowledged all main sources of help;\\
6. Where the thesis is based on work done by myself jointly with others, I have made clear exactly what was done by others and what I have contributed myself;\\
7. Either none of this work has been published before submission, or parts of this work have been published by :\\
\\
Stefan Collier\\
April 2016
}
\tableofcontents
\listoffigures
\listoftables

\mainmatter
%% ----------------------------------------------------------------
%\include{Introduction}
%\include{Conclusions}
\include{chapters/1Project/main}
\include{chapters/2Lit/main}
\include{chapters/3Design/HighLevel}
\include{chapters/3Design/InDepth}
\include{chapters/4Impl/main}

\include{chapters/5Experiments/1/main}
\include{chapters/5Experiments/2/main}
\include{chapters/5Experiments/3/main}
\include{chapters/5Experiments/4/main}

\include{chapters/6Conclusion/main}

\appendix
\include{appendix/AppendixB}
\include{appendix/D/main}
\include{appendix/AppendixC}

\backmatter
\bibliographystyle{ecs}
\bibliography{ECS}
\end{document}
%% ----------------------------------------------------------------

\include{chapters/3Design/HighLevel}
\include{chapters/3Design/InDepth}
 %% ----------------------------------------------------------------
%% Progress.tex
%% ---------------------------------------------------------------- 
\documentclass{ecsprogress}    % Use the progress Style
\graphicspath{{../figs/}}   % Location of your graphics files
    \usepackage{natbib}            % Use Natbib style for the refs.
\hypersetup{colorlinks=true}   % Set to false for black/white printing
\input{Definitions}            % Include your abbreviations



\usepackage{enumitem}% http://ctan.org/pkg/enumitem
\usepackage{multirow}
\usepackage{float}
\usepackage{amsmath}
\usepackage{multicol}
\usepackage{amssymb}
\usepackage[normalem]{ulem}
\useunder{\uline}{\ul}{}
\usepackage{wrapfig}


\usepackage[table,xcdraw]{xcolor}


%% ----------------------------------------------------------------
\begin{document}
\frontmatter
\title      {Heterogeneous Agent-based Model for Supermarket Competition}
\authors    {\texorpdfstring
             {\href{mailto:sc22g13@ecs.soton.ac.uk}{Stefan J. Collier}}
             {Stefan J. Collier}
            }
\addresses  {\groupname\\\deptname\\\univname}
\date       {\today}
\subject    {}
\keywords   {}
\supervisor {Dr. Maria Polukarov}
\examiner   {Professor Sheng Chen}

\maketitle
\begin{abstract}
This project aim was to model and analyse the effects of competitive pricing behaviors of grocery retailers on the British market. 

This was achieved by creating a multi-agent model, containing retailer and consumer agents. The heterogeneous crowd of retailers employs either a uniform pricing strategy or a ‘local price flexing’ strategy. The actions of these retailers are chosen by predicting the profit of each action, using a perceptron. Following on from the consideration of different economic models, a discrete model was developed so that software agents have a discrete environment to operate within. Within the model, it has been observed how supermarkets with differing behaviors affect a heterogeneous crowd of consumer agents. The model was implemented in Java with Python used to evaluate the results. 

The simulation displays good acceptance with real grocery market behavior, i.e. captures the performance of British retailers thus can be used to determine the impact of changes in their behavior on their competitors and consumers.Furthermore it can be used to provide insight into sustainability of volatile pricing strategies, providing a useful insight in volatility of British supermarket retail industry. 
\end{abstract}
\acknowledgements{
I would like to express my sincere gratitude to Dr Maria Polukarov for her guidance and support which provided me the freedom to take this research in the direction of my interest.\\
\\
I would also like to thank my family and friends for their encouragement and support. To those who quietly listened to my software complaints. To those who worked throughout the nights with me. To those who helped me write what I couldn't say. I cannot thank you enough.
}

\declaration{
I, Stefan Collier, declare that this dissertation and the work presented in it are my own and has been generated by me as the result of my own original research.\\
I confirm that:\\
1. This work was done wholly or mainly while in candidature for a degree at this University;\\
2. Where any part of this dissertation has previously been submitted for any other qualification at this University or any other institution, this has been clearly stated;\\
3. Where I have consulted the published work of others, this is always clearly attributed;\\
4. Where I have quoted from the work of others, the source is always given. With the exception of such quotations, this dissertation is entirely my own work;\\
5. I have acknowledged all main sources of help;\\
6. Where the thesis is based on work done by myself jointly with others, I have made clear exactly what was done by others and what I have contributed myself;\\
7. Either none of this work has been published before submission, or parts of this work have been published by :\\
\\
Stefan Collier\\
April 2016
}
\tableofcontents
\listoffigures
\listoftables

\mainmatter
%% ----------------------------------------------------------------
%\include{Introduction}
%\include{Conclusions}
\include{chapters/1Project/main}
\include{chapters/2Lit/main}
\include{chapters/3Design/HighLevel}
\include{chapters/3Design/InDepth}
\include{chapters/4Impl/main}

\include{chapters/5Experiments/1/main}
\include{chapters/5Experiments/2/main}
\include{chapters/5Experiments/3/main}
\include{chapters/5Experiments/4/main}

\include{chapters/6Conclusion/main}

\appendix
\include{appendix/AppendixB}
\include{appendix/D/main}
\include{appendix/AppendixC}

\backmatter
\bibliographystyle{ecs}
\bibliography{ECS}
\end{document}
%% ----------------------------------------------------------------


 %% ----------------------------------------------------------------
%% Progress.tex
%% ---------------------------------------------------------------- 
\documentclass{ecsprogress}    % Use the progress Style
\graphicspath{{../figs/}}   % Location of your graphics files
    \usepackage{natbib}            % Use Natbib style for the refs.
\hypersetup{colorlinks=true}   % Set to false for black/white printing
\input{Definitions}            % Include your abbreviations



\usepackage{enumitem}% http://ctan.org/pkg/enumitem
\usepackage{multirow}
\usepackage{float}
\usepackage{amsmath}
\usepackage{multicol}
\usepackage{amssymb}
\usepackage[normalem]{ulem}
\useunder{\uline}{\ul}{}
\usepackage{wrapfig}


\usepackage[table,xcdraw]{xcolor}


%% ----------------------------------------------------------------
\begin{document}
\frontmatter
\title      {Heterogeneous Agent-based Model for Supermarket Competition}
\authors    {\texorpdfstring
             {\href{mailto:sc22g13@ecs.soton.ac.uk}{Stefan J. Collier}}
             {Stefan J. Collier}
            }
\addresses  {\groupname\\\deptname\\\univname}
\date       {\today}
\subject    {}
\keywords   {}
\supervisor {Dr. Maria Polukarov}
\examiner   {Professor Sheng Chen}

\maketitle
\begin{abstract}
This project aim was to model and analyse the effects of competitive pricing behaviors of grocery retailers on the British market. 

This was achieved by creating a multi-agent model, containing retailer and consumer agents. The heterogeneous crowd of retailers employs either a uniform pricing strategy or a ‘local price flexing’ strategy. The actions of these retailers are chosen by predicting the profit of each action, using a perceptron. Following on from the consideration of different economic models, a discrete model was developed so that software agents have a discrete environment to operate within. Within the model, it has been observed how supermarkets with differing behaviors affect a heterogeneous crowd of consumer agents. The model was implemented in Java with Python used to evaluate the results. 

The simulation displays good acceptance with real grocery market behavior, i.e. captures the performance of British retailers thus can be used to determine the impact of changes in their behavior on their competitors and consumers.Furthermore it can be used to provide insight into sustainability of volatile pricing strategies, providing a useful insight in volatility of British supermarket retail industry. 
\end{abstract}
\acknowledgements{
I would like to express my sincere gratitude to Dr Maria Polukarov for her guidance and support which provided me the freedom to take this research in the direction of my interest.\\
\\
I would also like to thank my family and friends for their encouragement and support. To those who quietly listened to my software complaints. To those who worked throughout the nights with me. To those who helped me write what I couldn't say. I cannot thank you enough.
}

\declaration{
I, Stefan Collier, declare that this dissertation and the work presented in it are my own and has been generated by me as the result of my own original research.\\
I confirm that:\\
1. This work was done wholly or mainly while in candidature for a degree at this University;\\
2. Where any part of this dissertation has previously been submitted for any other qualification at this University or any other institution, this has been clearly stated;\\
3. Where I have consulted the published work of others, this is always clearly attributed;\\
4. Where I have quoted from the work of others, the source is always given. With the exception of such quotations, this dissertation is entirely my own work;\\
5. I have acknowledged all main sources of help;\\
6. Where the thesis is based on work done by myself jointly with others, I have made clear exactly what was done by others and what I have contributed myself;\\
7. Either none of this work has been published before submission, or parts of this work have been published by :\\
\\
Stefan Collier\\
April 2016
}
\tableofcontents
\listoffigures
\listoftables

\mainmatter
%% ----------------------------------------------------------------
%\include{Introduction}
%\include{Conclusions}
\include{chapters/1Project/main}
\include{chapters/2Lit/main}
\include{chapters/3Design/HighLevel}
\include{chapters/3Design/InDepth}
\include{chapters/4Impl/main}

\include{chapters/5Experiments/1/main}
\include{chapters/5Experiments/2/main}
\include{chapters/5Experiments/3/main}
\include{chapters/5Experiments/4/main}

\include{chapters/6Conclusion/main}

\appendix
\include{appendix/AppendixB}
\include{appendix/D/main}
\include{appendix/AppendixC}

\backmatter
\bibliographystyle{ecs}
\bibliography{ECS}
\end{document}
%% ----------------------------------------------------------------

 %% ----------------------------------------------------------------
%% Progress.tex
%% ---------------------------------------------------------------- 
\documentclass{ecsprogress}    % Use the progress Style
\graphicspath{{../figs/}}   % Location of your graphics files
    \usepackage{natbib}            % Use Natbib style for the refs.
\hypersetup{colorlinks=true}   % Set to false for black/white printing
\input{Definitions}            % Include your abbreviations



\usepackage{enumitem}% http://ctan.org/pkg/enumitem
\usepackage{multirow}
\usepackage{float}
\usepackage{amsmath}
\usepackage{multicol}
\usepackage{amssymb}
\usepackage[normalem]{ulem}
\useunder{\uline}{\ul}{}
\usepackage{wrapfig}


\usepackage[table,xcdraw]{xcolor}


%% ----------------------------------------------------------------
\begin{document}
\frontmatter
\title      {Heterogeneous Agent-based Model for Supermarket Competition}
\authors    {\texorpdfstring
             {\href{mailto:sc22g13@ecs.soton.ac.uk}{Stefan J. Collier}}
             {Stefan J. Collier}
            }
\addresses  {\groupname\\\deptname\\\univname}
\date       {\today}
\subject    {}
\keywords   {}
\supervisor {Dr. Maria Polukarov}
\examiner   {Professor Sheng Chen}

\maketitle
\begin{abstract}
This project aim was to model and analyse the effects of competitive pricing behaviors of grocery retailers on the British market. 

This was achieved by creating a multi-agent model, containing retailer and consumer agents. The heterogeneous crowd of retailers employs either a uniform pricing strategy or a ‘local price flexing’ strategy. The actions of these retailers are chosen by predicting the profit of each action, using a perceptron. Following on from the consideration of different economic models, a discrete model was developed so that software agents have a discrete environment to operate within. Within the model, it has been observed how supermarkets with differing behaviors affect a heterogeneous crowd of consumer agents. The model was implemented in Java with Python used to evaluate the results. 

The simulation displays good acceptance with real grocery market behavior, i.e. captures the performance of British retailers thus can be used to determine the impact of changes in their behavior on their competitors and consumers.Furthermore it can be used to provide insight into sustainability of volatile pricing strategies, providing a useful insight in volatility of British supermarket retail industry. 
\end{abstract}
\acknowledgements{
I would like to express my sincere gratitude to Dr Maria Polukarov for her guidance and support which provided me the freedom to take this research in the direction of my interest.\\
\\
I would also like to thank my family and friends for their encouragement and support. To those who quietly listened to my software complaints. To those who worked throughout the nights with me. To those who helped me write what I couldn't say. I cannot thank you enough.
}

\declaration{
I, Stefan Collier, declare that this dissertation and the work presented in it are my own and has been generated by me as the result of my own original research.\\
I confirm that:\\
1. This work was done wholly or mainly while in candidature for a degree at this University;\\
2. Where any part of this dissertation has previously been submitted for any other qualification at this University or any other institution, this has been clearly stated;\\
3. Where I have consulted the published work of others, this is always clearly attributed;\\
4. Where I have quoted from the work of others, the source is always given. With the exception of such quotations, this dissertation is entirely my own work;\\
5. I have acknowledged all main sources of help;\\
6. Where the thesis is based on work done by myself jointly with others, I have made clear exactly what was done by others and what I have contributed myself;\\
7. Either none of this work has been published before submission, or parts of this work have been published by :\\
\\
Stefan Collier\\
April 2016
}
\tableofcontents
\listoffigures
\listoftables

\mainmatter
%% ----------------------------------------------------------------
%\include{Introduction}
%\include{Conclusions}
\include{chapters/1Project/main}
\include{chapters/2Lit/main}
\include{chapters/3Design/HighLevel}
\include{chapters/3Design/InDepth}
\include{chapters/4Impl/main}

\include{chapters/5Experiments/1/main}
\include{chapters/5Experiments/2/main}
\include{chapters/5Experiments/3/main}
\include{chapters/5Experiments/4/main}

\include{chapters/6Conclusion/main}

\appendix
\include{appendix/AppendixB}
\include{appendix/D/main}
\include{appendix/AppendixC}

\backmatter
\bibliographystyle{ecs}
\bibliography{ECS}
\end{document}
%% ----------------------------------------------------------------

 %% ----------------------------------------------------------------
%% Progress.tex
%% ---------------------------------------------------------------- 
\documentclass{ecsprogress}    % Use the progress Style
\graphicspath{{../figs/}}   % Location of your graphics files
    \usepackage{natbib}            % Use Natbib style for the refs.
\hypersetup{colorlinks=true}   % Set to false for black/white printing
\input{Definitions}            % Include your abbreviations



\usepackage{enumitem}% http://ctan.org/pkg/enumitem
\usepackage{multirow}
\usepackage{float}
\usepackage{amsmath}
\usepackage{multicol}
\usepackage{amssymb}
\usepackage[normalem]{ulem}
\useunder{\uline}{\ul}{}
\usepackage{wrapfig}


\usepackage[table,xcdraw]{xcolor}


%% ----------------------------------------------------------------
\begin{document}
\frontmatter
\title      {Heterogeneous Agent-based Model for Supermarket Competition}
\authors    {\texorpdfstring
             {\href{mailto:sc22g13@ecs.soton.ac.uk}{Stefan J. Collier}}
             {Stefan J. Collier}
            }
\addresses  {\groupname\\\deptname\\\univname}
\date       {\today}
\subject    {}
\keywords   {}
\supervisor {Dr. Maria Polukarov}
\examiner   {Professor Sheng Chen}

\maketitle
\begin{abstract}
This project aim was to model and analyse the effects of competitive pricing behaviors of grocery retailers on the British market. 

This was achieved by creating a multi-agent model, containing retailer and consumer agents. The heterogeneous crowd of retailers employs either a uniform pricing strategy or a ‘local price flexing’ strategy. The actions of these retailers are chosen by predicting the profit of each action, using a perceptron. Following on from the consideration of different economic models, a discrete model was developed so that software agents have a discrete environment to operate within. Within the model, it has been observed how supermarkets with differing behaviors affect a heterogeneous crowd of consumer agents. The model was implemented in Java with Python used to evaluate the results. 

The simulation displays good acceptance with real grocery market behavior, i.e. captures the performance of British retailers thus can be used to determine the impact of changes in their behavior on their competitors and consumers.Furthermore it can be used to provide insight into sustainability of volatile pricing strategies, providing a useful insight in volatility of British supermarket retail industry. 
\end{abstract}
\acknowledgements{
I would like to express my sincere gratitude to Dr Maria Polukarov for her guidance and support which provided me the freedom to take this research in the direction of my interest.\\
\\
I would also like to thank my family and friends for their encouragement and support. To those who quietly listened to my software complaints. To those who worked throughout the nights with me. To those who helped me write what I couldn't say. I cannot thank you enough.
}

\declaration{
I, Stefan Collier, declare that this dissertation and the work presented in it are my own and has been generated by me as the result of my own original research.\\
I confirm that:\\
1. This work was done wholly or mainly while in candidature for a degree at this University;\\
2. Where any part of this dissertation has previously been submitted for any other qualification at this University or any other institution, this has been clearly stated;\\
3. Where I have consulted the published work of others, this is always clearly attributed;\\
4. Where I have quoted from the work of others, the source is always given. With the exception of such quotations, this dissertation is entirely my own work;\\
5. I have acknowledged all main sources of help;\\
6. Where the thesis is based on work done by myself jointly with others, I have made clear exactly what was done by others and what I have contributed myself;\\
7. Either none of this work has been published before submission, or parts of this work have been published by :\\
\\
Stefan Collier\\
April 2016
}
\tableofcontents
\listoffigures
\listoftables

\mainmatter
%% ----------------------------------------------------------------
%\include{Introduction}
%\include{Conclusions}
\include{chapters/1Project/main}
\include{chapters/2Lit/main}
\include{chapters/3Design/HighLevel}
\include{chapters/3Design/InDepth}
\include{chapters/4Impl/main}

\include{chapters/5Experiments/1/main}
\include{chapters/5Experiments/2/main}
\include{chapters/5Experiments/3/main}
\include{chapters/5Experiments/4/main}

\include{chapters/6Conclusion/main}

\appendix
\include{appendix/AppendixB}
\include{appendix/D/main}
\include{appendix/AppendixC}

\backmatter
\bibliographystyle{ecs}
\bibliography{ECS}
\end{document}
%% ----------------------------------------------------------------

 %% ----------------------------------------------------------------
%% Progress.tex
%% ---------------------------------------------------------------- 
\documentclass{ecsprogress}    % Use the progress Style
\graphicspath{{../figs/}}   % Location of your graphics files
    \usepackage{natbib}            % Use Natbib style for the refs.
\hypersetup{colorlinks=true}   % Set to false for black/white printing
\input{Definitions}            % Include your abbreviations



\usepackage{enumitem}% http://ctan.org/pkg/enumitem
\usepackage{multirow}
\usepackage{float}
\usepackage{amsmath}
\usepackage{multicol}
\usepackage{amssymb}
\usepackage[normalem]{ulem}
\useunder{\uline}{\ul}{}
\usepackage{wrapfig}


\usepackage[table,xcdraw]{xcolor}


%% ----------------------------------------------------------------
\begin{document}
\frontmatter
\title      {Heterogeneous Agent-based Model for Supermarket Competition}
\authors    {\texorpdfstring
             {\href{mailto:sc22g13@ecs.soton.ac.uk}{Stefan J. Collier}}
             {Stefan J. Collier}
            }
\addresses  {\groupname\\\deptname\\\univname}
\date       {\today}
\subject    {}
\keywords   {}
\supervisor {Dr. Maria Polukarov}
\examiner   {Professor Sheng Chen}

\maketitle
\begin{abstract}
This project aim was to model and analyse the effects of competitive pricing behaviors of grocery retailers on the British market. 

This was achieved by creating a multi-agent model, containing retailer and consumer agents. The heterogeneous crowd of retailers employs either a uniform pricing strategy or a ‘local price flexing’ strategy. The actions of these retailers are chosen by predicting the profit of each action, using a perceptron. Following on from the consideration of different economic models, a discrete model was developed so that software agents have a discrete environment to operate within. Within the model, it has been observed how supermarkets with differing behaviors affect a heterogeneous crowd of consumer agents. The model was implemented in Java with Python used to evaluate the results. 

The simulation displays good acceptance with real grocery market behavior, i.e. captures the performance of British retailers thus can be used to determine the impact of changes in their behavior on their competitors and consumers.Furthermore it can be used to provide insight into sustainability of volatile pricing strategies, providing a useful insight in volatility of British supermarket retail industry. 
\end{abstract}
\acknowledgements{
I would like to express my sincere gratitude to Dr Maria Polukarov for her guidance and support which provided me the freedom to take this research in the direction of my interest.\\
\\
I would also like to thank my family and friends for their encouragement and support. To those who quietly listened to my software complaints. To those who worked throughout the nights with me. To those who helped me write what I couldn't say. I cannot thank you enough.
}

\declaration{
I, Stefan Collier, declare that this dissertation and the work presented in it are my own and has been generated by me as the result of my own original research.\\
I confirm that:\\
1. This work was done wholly or mainly while in candidature for a degree at this University;\\
2. Where any part of this dissertation has previously been submitted for any other qualification at this University or any other institution, this has been clearly stated;\\
3. Where I have consulted the published work of others, this is always clearly attributed;\\
4. Where I have quoted from the work of others, the source is always given. With the exception of such quotations, this dissertation is entirely my own work;\\
5. I have acknowledged all main sources of help;\\
6. Where the thesis is based on work done by myself jointly with others, I have made clear exactly what was done by others and what I have contributed myself;\\
7. Either none of this work has been published before submission, or parts of this work have been published by :\\
\\
Stefan Collier\\
April 2016
}
\tableofcontents
\listoffigures
\listoftables

\mainmatter
%% ----------------------------------------------------------------
%\include{Introduction}
%\include{Conclusions}
\include{chapters/1Project/main}
\include{chapters/2Lit/main}
\include{chapters/3Design/HighLevel}
\include{chapters/3Design/InDepth}
\include{chapters/4Impl/main}

\include{chapters/5Experiments/1/main}
\include{chapters/5Experiments/2/main}
\include{chapters/5Experiments/3/main}
\include{chapters/5Experiments/4/main}

\include{chapters/6Conclusion/main}

\appendix
\include{appendix/AppendixB}
\include{appendix/D/main}
\include{appendix/AppendixC}

\backmatter
\bibliographystyle{ecs}
\bibliography{ECS}
\end{document}
%% ----------------------------------------------------------------


 %% ----------------------------------------------------------------
%% Progress.tex
%% ---------------------------------------------------------------- 
\documentclass{ecsprogress}    % Use the progress Style
\graphicspath{{../figs/}}   % Location of your graphics files
    \usepackage{natbib}            % Use Natbib style for the refs.
\hypersetup{colorlinks=true}   % Set to false for black/white printing
\input{Definitions}            % Include your abbreviations



\usepackage{enumitem}% http://ctan.org/pkg/enumitem
\usepackage{multirow}
\usepackage{float}
\usepackage{amsmath}
\usepackage{multicol}
\usepackage{amssymb}
\usepackage[normalem]{ulem}
\useunder{\uline}{\ul}{}
\usepackage{wrapfig}


\usepackage[table,xcdraw]{xcolor}


%% ----------------------------------------------------------------
\begin{document}
\frontmatter
\title      {Heterogeneous Agent-based Model for Supermarket Competition}
\authors    {\texorpdfstring
             {\href{mailto:sc22g13@ecs.soton.ac.uk}{Stefan J. Collier}}
             {Stefan J. Collier}
            }
\addresses  {\groupname\\\deptname\\\univname}
\date       {\today}
\subject    {}
\keywords   {}
\supervisor {Dr. Maria Polukarov}
\examiner   {Professor Sheng Chen}

\maketitle
\begin{abstract}
This project aim was to model and analyse the effects of competitive pricing behaviors of grocery retailers on the British market. 

This was achieved by creating a multi-agent model, containing retailer and consumer agents. The heterogeneous crowd of retailers employs either a uniform pricing strategy or a ‘local price flexing’ strategy. The actions of these retailers are chosen by predicting the profit of each action, using a perceptron. Following on from the consideration of different economic models, a discrete model was developed so that software agents have a discrete environment to operate within. Within the model, it has been observed how supermarkets with differing behaviors affect a heterogeneous crowd of consumer agents. The model was implemented in Java with Python used to evaluate the results. 

The simulation displays good acceptance with real grocery market behavior, i.e. captures the performance of British retailers thus can be used to determine the impact of changes in their behavior on their competitors and consumers.Furthermore it can be used to provide insight into sustainability of volatile pricing strategies, providing a useful insight in volatility of British supermarket retail industry. 
\end{abstract}
\acknowledgements{
I would like to express my sincere gratitude to Dr Maria Polukarov for her guidance and support which provided me the freedom to take this research in the direction of my interest.\\
\\
I would also like to thank my family and friends for their encouragement and support. To those who quietly listened to my software complaints. To those who worked throughout the nights with me. To those who helped me write what I couldn't say. I cannot thank you enough.
}

\declaration{
I, Stefan Collier, declare that this dissertation and the work presented in it are my own and has been generated by me as the result of my own original research.\\
I confirm that:\\
1. This work was done wholly or mainly while in candidature for a degree at this University;\\
2. Where any part of this dissertation has previously been submitted for any other qualification at this University or any other institution, this has been clearly stated;\\
3. Where I have consulted the published work of others, this is always clearly attributed;\\
4. Where I have quoted from the work of others, the source is always given. With the exception of such quotations, this dissertation is entirely my own work;\\
5. I have acknowledged all main sources of help;\\
6. Where the thesis is based on work done by myself jointly with others, I have made clear exactly what was done by others and what I have contributed myself;\\
7. Either none of this work has been published before submission, or parts of this work have been published by :\\
\\
Stefan Collier\\
April 2016
}
\tableofcontents
\listoffigures
\listoftables

\mainmatter
%% ----------------------------------------------------------------
%\include{Introduction}
%\include{Conclusions}
\include{chapters/1Project/main}
\include{chapters/2Lit/main}
\include{chapters/3Design/HighLevel}
\include{chapters/3Design/InDepth}
\include{chapters/4Impl/main}

\include{chapters/5Experiments/1/main}
\include{chapters/5Experiments/2/main}
\include{chapters/5Experiments/3/main}
\include{chapters/5Experiments/4/main}

\include{chapters/6Conclusion/main}

\appendix
\include{appendix/AppendixB}
\include{appendix/D/main}
\include{appendix/AppendixC}

\backmatter
\bibliographystyle{ecs}
\bibliography{ECS}
\end{document}
%% ----------------------------------------------------------------


\appendix
\include{appendix/AppendixB}
 %% ----------------------------------------------------------------
%% Progress.tex
%% ---------------------------------------------------------------- 
\documentclass{ecsprogress}    % Use the progress Style
\graphicspath{{../figs/}}   % Location of your graphics files
    \usepackage{natbib}            % Use Natbib style for the refs.
\hypersetup{colorlinks=true}   % Set to false for black/white printing
\input{Definitions}            % Include your abbreviations



\usepackage{enumitem}% http://ctan.org/pkg/enumitem
\usepackage{multirow}
\usepackage{float}
\usepackage{amsmath}
\usepackage{multicol}
\usepackage{amssymb}
\usepackage[normalem]{ulem}
\useunder{\uline}{\ul}{}
\usepackage{wrapfig}


\usepackage[table,xcdraw]{xcolor}


%% ----------------------------------------------------------------
\begin{document}
\frontmatter
\title      {Heterogeneous Agent-based Model for Supermarket Competition}
\authors    {\texorpdfstring
             {\href{mailto:sc22g13@ecs.soton.ac.uk}{Stefan J. Collier}}
             {Stefan J. Collier}
            }
\addresses  {\groupname\\\deptname\\\univname}
\date       {\today}
\subject    {}
\keywords   {}
\supervisor {Dr. Maria Polukarov}
\examiner   {Professor Sheng Chen}

\maketitle
\begin{abstract}
This project aim was to model and analyse the effects of competitive pricing behaviors of grocery retailers on the British market. 

This was achieved by creating a multi-agent model, containing retailer and consumer agents. The heterogeneous crowd of retailers employs either a uniform pricing strategy or a ‘local price flexing’ strategy. The actions of these retailers are chosen by predicting the profit of each action, using a perceptron. Following on from the consideration of different economic models, a discrete model was developed so that software agents have a discrete environment to operate within. Within the model, it has been observed how supermarkets with differing behaviors affect a heterogeneous crowd of consumer agents. The model was implemented in Java with Python used to evaluate the results. 

The simulation displays good acceptance with real grocery market behavior, i.e. captures the performance of British retailers thus can be used to determine the impact of changes in their behavior on their competitors and consumers.Furthermore it can be used to provide insight into sustainability of volatile pricing strategies, providing a useful insight in volatility of British supermarket retail industry. 
\end{abstract}
\acknowledgements{
I would like to express my sincere gratitude to Dr Maria Polukarov for her guidance and support which provided me the freedom to take this research in the direction of my interest.\\
\\
I would also like to thank my family and friends for their encouragement and support. To those who quietly listened to my software complaints. To those who worked throughout the nights with me. To those who helped me write what I couldn't say. I cannot thank you enough.
}

\declaration{
I, Stefan Collier, declare that this dissertation and the work presented in it are my own and has been generated by me as the result of my own original research.\\
I confirm that:\\
1. This work was done wholly or mainly while in candidature for a degree at this University;\\
2. Where any part of this dissertation has previously been submitted for any other qualification at this University or any other institution, this has been clearly stated;\\
3. Where I have consulted the published work of others, this is always clearly attributed;\\
4. Where I have quoted from the work of others, the source is always given. With the exception of such quotations, this dissertation is entirely my own work;\\
5. I have acknowledged all main sources of help;\\
6. Where the thesis is based on work done by myself jointly with others, I have made clear exactly what was done by others and what I have contributed myself;\\
7. Either none of this work has been published before submission, or parts of this work have been published by :\\
\\
Stefan Collier\\
April 2016
}
\tableofcontents
\listoffigures
\listoftables

\mainmatter
%% ----------------------------------------------------------------
%\include{Introduction}
%\include{Conclusions}
\include{chapters/1Project/main}
\include{chapters/2Lit/main}
\include{chapters/3Design/HighLevel}
\include{chapters/3Design/InDepth}
\include{chapters/4Impl/main}

\include{chapters/5Experiments/1/main}
\include{chapters/5Experiments/2/main}
\include{chapters/5Experiments/3/main}
\include{chapters/5Experiments/4/main}

\include{chapters/6Conclusion/main}

\appendix
\include{appendix/AppendixB}
\include{appendix/D/main}
\include{appendix/AppendixC}

\backmatter
\bibliographystyle{ecs}
\bibliography{ECS}
\end{document}
%% ----------------------------------------------------------------

\include{appendix/AppendixC}

\backmatter
\bibliographystyle{ecs}
\bibliography{ECS}
\end{document}
%% ----------------------------------------------------------------


 %% ----------------------------------------------------------------
%% Progress.tex
%% ---------------------------------------------------------------- 
\documentclass{ecsprogress}    % Use the progress Style
\graphicspath{{../figs/}}   % Location of your graphics files
    \usepackage{natbib}            % Use Natbib style for the refs.
\hypersetup{colorlinks=true}   % Set to false for black/white printing
\input{Definitions}            % Include your abbreviations



\usepackage{enumitem}% http://ctan.org/pkg/enumitem
\usepackage{multirow}
\usepackage{float}
\usepackage{amsmath}
\usepackage{multicol}
\usepackage{amssymb}
\usepackage[normalem]{ulem}
\useunder{\uline}{\ul}{}
\usepackage{wrapfig}


\usepackage[table,xcdraw]{xcolor}


%% ----------------------------------------------------------------
\begin{document}
\frontmatter
\title      {Heterogeneous Agent-based Model for Supermarket Competition}
\authors    {\texorpdfstring
             {\href{mailto:sc22g13@ecs.soton.ac.uk}{Stefan J. Collier}}
             {Stefan J. Collier}
            }
\addresses  {\groupname\\\deptname\\\univname}
\date       {\today}
\subject    {}
\keywords   {}
\supervisor {Dr. Maria Polukarov}
\examiner   {Professor Sheng Chen}

\maketitle
\begin{abstract}
This project aim was to model and analyse the effects of competitive pricing behaviors of grocery retailers on the British market. 

This was achieved by creating a multi-agent model, containing retailer and consumer agents. The heterogeneous crowd of retailers employs either a uniform pricing strategy or a ‘local price flexing’ strategy. The actions of these retailers are chosen by predicting the profit of each action, using a perceptron. Following on from the consideration of different economic models, a discrete model was developed so that software agents have a discrete environment to operate within. Within the model, it has been observed how supermarkets with differing behaviors affect a heterogeneous crowd of consumer agents. The model was implemented in Java with Python used to evaluate the results. 

The simulation displays good acceptance with real grocery market behavior, i.e. captures the performance of British retailers thus can be used to determine the impact of changes in their behavior on their competitors and consumers.Furthermore it can be used to provide insight into sustainability of volatile pricing strategies, providing a useful insight in volatility of British supermarket retail industry. 
\end{abstract}
\acknowledgements{
I would like to express my sincere gratitude to Dr Maria Polukarov for her guidance and support which provided me the freedom to take this research in the direction of my interest.\\
\\
I would also like to thank my family and friends for their encouragement and support. To those who quietly listened to my software complaints. To those who worked throughout the nights with me. To those who helped me write what I couldn't say. I cannot thank you enough.
}

\declaration{
I, Stefan Collier, declare that this dissertation and the work presented in it are my own and has been generated by me as the result of my own original research.\\
I confirm that:\\
1. This work was done wholly or mainly while in candidature for a degree at this University;\\
2. Where any part of this dissertation has previously been submitted for any other qualification at this University or any other institution, this has been clearly stated;\\
3. Where I have consulted the published work of others, this is always clearly attributed;\\
4. Where I have quoted from the work of others, the source is always given. With the exception of such quotations, this dissertation is entirely my own work;\\
5. I have acknowledged all main sources of help;\\
6. Where the thesis is based on work done by myself jointly with others, I have made clear exactly what was done by others and what I have contributed myself;\\
7. Either none of this work has been published before submission, or parts of this work have been published by :\\
\\
Stefan Collier\\
April 2016
}
\tableofcontents
\listoffigures
\listoftables

\mainmatter
%% ----------------------------------------------------------------
%\include{Introduction}
%\include{Conclusions}
 %% ----------------------------------------------------------------
%% Progress.tex
%% ---------------------------------------------------------------- 
\documentclass{ecsprogress}    % Use the progress Style
\graphicspath{{../figs/}}   % Location of your graphics files
    \usepackage{natbib}            % Use Natbib style for the refs.
\hypersetup{colorlinks=true}   % Set to false for black/white printing
\input{Definitions}            % Include your abbreviations



\usepackage{enumitem}% http://ctan.org/pkg/enumitem
\usepackage{multirow}
\usepackage{float}
\usepackage{amsmath}
\usepackage{multicol}
\usepackage{amssymb}
\usepackage[normalem]{ulem}
\useunder{\uline}{\ul}{}
\usepackage{wrapfig}


\usepackage[table,xcdraw]{xcolor}


%% ----------------------------------------------------------------
\begin{document}
\frontmatter
\title      {Heterogeneous Agent-based Model for Supermarket Competition}
\authors    {\texorpdfstring
             {\href{mailto:sc22g13@ecs.soton.ac.uk}{Stefan J. Collier}}
             {Stefan J. Collier}
            }
\addresses  {\groupname\\\deptname\\\univname}
\date       {\today}
\subject    {}
\keywords   {}
\supervisor {Dr. Maria Polukarov}
\examiner   {Professor Sheng Chen}

\maketitle
\begin{abstract}
This project aim was to model and analyse the effects of competitive pricing behaviors of grocery retailers on the British market. 

This was achieved by creating a multi-agent model, containing retailer and consumer agents. The heterogeneous crowd of retailers employs either a uniform pricing strategy or a ‘local price flexing’ strategy. The actions of these retailers are chosen by predicting the profit of each action, using a perceptron. Following on from the consideration of different economic models, a discrete model was developed so that software agents have a discrete environment to operate within. Within the model, it has been observed how supermarkets with differing behaviors affect a heterogeneous crowd of consumer agents. The model was implemented in Java with Python used to evaluate the results. 

The simulation displays good acceptance with real grocery market behavior, i.e. captures the performance of British retailers thus can be used to determine the impact of changes in their behavior on their competitors and consumers.Furthermore it can be used to provide insight into sustainability of volatile pricing strategies, providing a useful insight in volatility of British supermarket retail industry. 
\end{abstract}
\acknowledgements{
I would like to express my sincere gratitude to Dr Maria Polukarov for her guidance and support which provided me the freedom to take this research in the direction of my interest.\\
\\
I would also like to thank my family and friends for their encouragement and support. To those who quietly listened to my software complaints. To those who worked throughout the nights with me. To those who helped me write what I couldn't say. I cannot thank you enough.
}

\declaration{
I, Stefan Collier, declare that this dissertation and the work presented in it are my own and has been generated by me as the result of my own original research.\\
I confirm that:\\
1. This work was done wholly or mainly while in candidature for a degree at this University;\\
2. Where any part of this dissertation has previously been submitted for any other qualification at this University or any other institution, this has been clearly stated;\\
3. Where I have consulted the published work of others, this is always clearly attributed;\\
4. Where I have quoted from the work of others, the source is always given. With the exception of such quotations, this dissertation is entirely my own work;\\
5. I have acknowledged all main sources of help;\\
6. Where the thesis is based on work done by myself jointly with others, I have made clear exactly what was done by others and what I have contributed myself;\\
7. Either none of this work has been published before submission, or parts of this work have been published by :\\
\\
Stefan Collier\\
April 2016
}
\tableofcontents
\listoffigures
\listoftables

\mainmatter
%% ----------------------------------------------------------------
%\include{Introduction}
%\include{Conclusions}
\include{chapters/1Project/main}
\include{chapters/2Lit/main}
\include{chapters/3Design/HighLevel}
\include{chapters/3Design/InDepth}
\include{chapters/4Impl/main}

\include{chapters/5Experiments/1/main}
\include{chapters/5Experiments/2/main}
\include{chapters/5Experiments/3/main}
\include{chapters/5Experiments/4/main}

\include{chapters/6Conclusion/main}

\appendix
\include{appendix/AppendixB}
\include{appendix/D/main}
\include{appendix/AppendixC}

\backmatter
\bibliographystyle{ecs}
\bibliography{ECS}
\end{document}
%% ----------------------------------------------------------------

 %% ----------------------------------------------------------------
%% Progress.tex
%% ---------------------------------------------------------------- 
\documentclass{ecsprogress}    % Use the progress Style
\graphicspath{{../figs/}}   % Location of your graphics files
    \usepackage{natbib}            % Use Natbib style for the refs.
\hypersetup{colorlinks=true}   % Set to false for black/white printing
\input{Definitions}            % Include your abbreviations



\usepackage{enumitem}% http://ctan.org/pkg/enumitem
\usepackage{multirow}
\usepackage{float}
\usepackage{amsmath}
\usepackage{multicol}
\usepackage{amssymb}
\usepackage[normalem]{ulem}
\useunder{\uline}{\ul}{}
\usepackage{wrapfig}


\usepackage[table,xcdraw]{xcolor}


%% ----------------------------------------------------------------
\begin{document}
\frontmatter
\title      {Heterogeneous Agent-based Model for Supermarket Competition}
\authors    {\texorpdfstring
             {\href{mailto:sc22g13@ecs.soton.ac.uk}{Stefan J. Collier}}
             {Stefan J. Collier}
            }
\addresses  {\groupname\\\deptname\\\univname}
\date       {\today}
\subject    {}
\keywords   {}
\supervisor {Dr. Maria Polukarov}
\examiner   {Professor Sheng Chen}

\maketitle
\begin{abstract}
This project aim was to model and analyse the effects of competitive pricing behaviors of grocery retailers on the British market. 

This was achieved by creating a multi-agent model, containing retailer and consumer agents. The heterogeneous crowd of retailers employs either a uniform pricing strategy or a ‘local price flexing’ strategy. The actions of these retailers are chosen by predicting the profit of each action, using a perceptron. Following on from the consideration of different economic models, a discrete model was developed so that software agents have a discrete environment to operate within. Within the model, it has been observed how supermarkets with differing behaviors affect a heterogeneous crowd of consumer agents. The model was implemented in Java with Python used to evaluate the results. 

The simulation displays good acceptance with real grocery market behavior, i.e. captures the performance of British retailers thus can be used to determine the impact of changes in their behavior on their competitors and consumers.Furthermore it can be used to provide insight into sustainability of volatile pricing strategies, providing a useful insight in volatility of British supermarket retail industry. 
\end{abstract}
\acknowledgements{
I would like to express my sincere gratitude to Dr Maria Polukarov for her guidance and support which provided me the freedom to take this research in the direction of my interest.\\
\\
I would also like to thank my family and friends for their encouragement and support. To those who quietly listened to my software complaints. To those who worked throughout the nights with me. To those who helped me write what I couldn't say. I cannot thank you enough.
}

\declaration{
I, Stefan Collier, declare that this dissertation and the work presented in it are my own and has been generated by me as the result of my own original research.\\
I confirm that:\\
1. This work was done wholly or mainly while in candidature for a degree at this University;\\
2. Where any part of this dissertation has previously been submitted for any other qualification at this University or any other institution, this has been clearly stated;\\
3. Where I have consulted the published work of others, this is always clearly attributed;\\
4. Where I have quoted from the work of others, the source is always given. With the exception of such quotations, this dissertation is entirely my own work;\\
5. I have acknowledged all main sources of help;\\
6. Where the thesis is based on work done by myself jointly with others, I have made clear exactly what was done by others and what I have contributed myself;\\
7. Either none of this work has been published before submission, or parts of this work have been published by :\\
\\
Stefan Collier\\
April 2016
}
\tableofcontents
\listoffigures
\listoftables

\mainmatter
%% ----------------------------------------------------------------
%\include{Introduction}
%\include{Conclusions}
\include{chapters/1Project/main}
\include{chapters/2Lit/main}
\include{chapters/3Design/HighLevel}
\include{chapters/3Design/InDepth}
\include{chapters/4Impl/main}

\include{chapters/5Experiments/1/main}
\include{chapters/5Experiments/2/main}
\include{chapters/5Experiments/3/main}
\include{chapters/5Experiments/4/main}

\include{chapters/6Conclusion/main}

\appendix
\include{appendix/AppendixB}
\include{appendix/D/main}
\include{appendix/AppendixC}

\backmatter
\bibliographystyle{ecs}
\bibliography{ECS}
\end{document}
%% ----------------------------------------------------------------

\include{chapters/3Design/HighLevel}
\include{chapters/3Design/InDepth}
 %% ----------------------------------------------------------------
%% Progress.tex
%% ---------------------------------------------------------------- 
\documentclass{ecsprogress}    % Use the progress Style
\graphicspath{{../figs/}}   % Location of your graphics files
    \usepackage{natbib}            % Use Natbib style for the refs.
\hypersetup{colorlinks=true}   % Set to false for black/white printing
\input{Definitions}            % Include your abbreviations



\usepackage{enumitem}% http://ctan.org/pkg/enumitem
\usepackage{multirow}
\usepackage{float}
\usepackage{amsmath}
\usepackage{multicol}
\usepackage{amssymb}
\usepackage[normalem]{ulem}
\useunder{\uline}{\ul}{}
\usepackage{wrapfig}


\usepackage[table,xcdraw]{xcolor}


%% ----------------------------------------------------------------
\begin{document}
\frontmatter
\title      {Heterogeneous Agent-based Model for Supermarket Competition}
\authors    {\texorpdfstring
             {\href{mailto:sc22g13@ecs.soton.ac.uk}{Stefan J. Collier}}
             {Stefan J. Collier}
            }
\addresses  {\groupname\\\deptname\\\univname}
\date       {\today}
\subject    {}
\keywords   {}
\supervisor {Dr. Maria Polukarov}
\examiner   {Professor Sheng Chen}

\maketitle
\begin{abstract}
This project aim was to model and analyse the effects of competitive pricing behaviors of grocery retailers on the British market. 

This was achieved by creating a multi-agent model, containing retailer and consumer agents. The heterogeneous crowd of retailers employs either a uniform pricing strategy or a ‘local price flexing’ strategy. The actions of these retailers are chosen by predicting the profit of each action, using a perceptron. Following on from the consideration of different economic models, a discrete model was developed so that software agents have a discrete environment to operate within. Within the model, it has been observed how supermarkets with differing behaviors affect a heterogeneous crowd of consumer agents. The model was implemented in Java with Python used to evaluate the results. 

The simulation displays good acceptance with real grocery market behavior, i.e. captures the performance of British retailers thus can be used to determine the impact of changes in their behavior on their competitors and consumers.Furthermore it can be used to provide insight into sustainability of volatile pricing strategies, providing a useful insight in volatility of British supermarket retail industry. 
\end{abstract}
\acknowledgements{
I would like to express my sincere gratitude to Dr Maria Polukarov for her guidance and support which provided me the freedom to take this research in the direction of my interest.\\
\\
I would also like to thank my family and friends for their encouragement and support. To those who quietly listened to my software complaints. To those who worked throughout the nights with me. To those who helped me write what I couldn't say. I cannot thank you enough.
}

\declaration{
I, Stefan Collier, declare that this dissertation and the work presented in it are my own and has been generated by me as the result of my own original research.\\
I confirm that:\\
1. This work was done wholly or mainly while in candidature for a degree at this University;\\
2. Where any part of this dissertation has previously been submitted for any other qualification at this University or any other institution, this has been clearly stated;\\
3. Where I have consulted the published work of others, this is always clearly attributed;\\
4. Where I have quoted from the work of others, the source is always given. With the exception of such quotations, this dissertation is entirely my own work;\\
5. I have acknowledged all main sources of help;\\
6. Where the thesis is based on work done by myself jointly with others, I have made clear exactly what was done by others and what I have contributed myself;\\
7. Either none of this work has been published before submission, or parts of this work have been published by :\\
\\
Stefan Collier\\
April 2016
}
\tableofcontents
\listoffigures
\listoftables

\mainmatter
%% ----------------------------------------------------------------
%\include{Introduction}
%\include{Conclusions}
\include{chapters/1Project/main}
\include{chapters/2Lit/main}
\include{chapters/3Design/HighLevel}
\include{chapters/3Design/InDepth}
\include{chapters/4Impl/main}

\include{chapters/5Experiments/1/main}
\include{chapters/5Experiments/2/main}
\include{chapters/5Experiments/3/main}
\include{chapters/5Experiments/4/main}

\include{chapters/6Conclusion/main}

\appendix
\include{appendix/AppendixB}
\include{appendix/D/main}
\include{appendix/AppendixC}

\backmatter
\bibliographystyle{ecs}
\bibliography{ECS}
\end{document}
%% ----------------------------------------------------------------


 %% ----------------------------------------------------------------
%% Progress.tex
%% ---------------------------------------------------------------- 
\documentclass{ecsprogress}    % Use the progress Style
\graphicspath{{../figs/}}   % Location of your graphics files
    \usepackage{natbib}            % Use Natbib style for the refs.
\hypersetup{colorlinks=true}   % Set to false for black/white printing
\input{Definitions}            % Include your abbreviations



\usepackage{enumitem}% http://ctan.org/pkg/enumitem
\usepackage{multirow}
\usepackage{float}
\usepackage{amsmath}
\usepackage{multicol}
\usepackage{amssymb}
\usepackage[normalem]{ulem}
\useunder{\uline}{\ul}{}
\usepackage{wrapfig}


\usepackage[table,xcdraw]{xcolor}


%% ----------------------------------------------------------------
\begin{document}
\frontmatter
\title      {Heterogeneous Agent-based Model for Supermarket Competition}
\authors    {\texorpdfstring
             {\href{mailto:sc22g13@ecs.soton.ac.uk}{Stefan J. Collier}}
             {Stefan J. Collier}
            }
\addresses  {\groupname\\\deptname\\\univname}
\date       {\today}
\subject    {}
\keywords   {}
\supervisor {Dr. Maria Polukarov}
\examiner   {Professor Sheng Chen}

\maketitle
\begin{abstract}
This project aim was to model and analyse the effects of competitive pricing behaviors of grocery retailers on the British market. 

This was achieved by creating a multi-agent model, containing retailer and consumer agents. The heterogeneous crowd of retailers employs either a uniform pricing strategy or a ‘local price flexing’ strategy. The actions of these retailers are chosen by predicting the profit of each action, using a perceptron. Following on from the consideration of different economic models, a discrete model was developed so that software agents have a discrete environment to operate within. Within the model, it has been observed how supermarkets with differing behaviors affect a heterogeneous crowd of consumer agents. The model was implemented in Java with Python used to evaluate the results. 

The simulation displays good acceptance with real grocery market behavior, i.e. captures the performance of British retailers thus can be used to determine the impact of changes in their behavior on their competitors and consumers.Furthermore it can be used to provide insight into sustainability of volatile pricing strategies, providing a useful insight in volatility of British supermarket retail industry. 
\end{abstract}
\acknowledgements{
I would like to express my sincere gratitude to Dr Maria Polukarov for her guidance and support which provided me the freedom to take this research in the direction of my interest.\\
\\
I would also like to thank my family and friends for their encouragement and support. To those who quietly listened to my software complaints. To those who worked throughout the nights with me. To those who helped me write what I couldn't say. I cannot thank you enough.
}

\declaration{
I, Stefan Collier, declare that this dissertation and the work presented in it are my own and has been generated by me as the result of my own original research.\\
I confirm that:\\
1. This work was done wholly or mainly while in candidature for a degree at this University;\\
2. Where any part of this dissertation has previously been submitted for any other qualification at this University or any other institution, this has been clearly stated;\\
3. Where I have consulted the published work of others, this is always clearly attributed;\\
4. Where I have quoted from the work of others, the source is always given. With the exception of such quotations, this dissertation is entirely my own work;\\
5. I have acknowledged all main sources of help;\\
6. Where the thesis is based on work done by myself jointly with others, I have made clear exactly what was done by others and what I have contributed myself;\\
7. Either none of this work has been published before submission, or parts of this work have been published by :\\
\\
Stefan Collier\\
April 2016
}
\tableofcontents
\listoffigures
\listoftables

\mainmatter
%% ----------------------------------------------------------------
%\include{Introduction}
%\include{Conclusions}
\include{chapters/1Project/main}
\include{chapters/2Lit/main}
\include{chapters/3Design/HighLevel}
\include{chapters/3Design/InDepth}
\include{chapters/4Impl/main}

\include{chapters/5Experiments/1/main}
\include{chapters/5Experiments/2/main}
\include{chapters/5Experiments/3/main}
\include{chapters/5Experiments/4/main}

\include{chapters/6Conclusion/main}

\appendix
\include{appendix/AppendixB}
\include{appendix/D/main}
\include{appendix/AppendixC}

\backmatter
\bibliographystyle{ecs}
\bibliography{ECS}
\end{document}
%% ----------------------------------------------------------------

 %% ----------------------------------------------------------------
%% Progress.tex
%% ---------------------------------------------------------------- 
\documentclass{ecsprogress}    % Use the progress Style
\graphicspath{{../figs/}}   % Location of your graphics files
    \usepackage{natbib}            % Use Natbib style for the refs.
\hypersetup{colorlinks=true}   % Set to false for black/white printing
\input{Definitions}            % Include your abbreviations



\usepackage{enumitem}% http://ctan.org/pkg/enumitem
\usepackage{multirow}
\usepackage{float}
\usepackage{amsmath}
\usepackage{multicol}
\usepackage{amssymb}
\usepackage[normalem]{ulem}
\useunder{\uline}{\ul}{}
\usepackage{wrapfig}


\usepackage[table,xcdraw]{xcolor}


%% ----------------------------------------------------------------
\begin{document}
\frontmatter
\title      {Heterogeneous Agent-based Model for Supermarket Competition}
\authors    {\texorpdfstring
             {\href{mailto:sc22g13@ecs.soton.ac.uk}{Stefan J. Collier}}
             {Stefan J. Collier}
            }
\addresses  {\groupname\\\deptname\\\univname}
\date       {\today}
\subject    {}
\keywords   {}
\supervisor {Dr. Maria Polukarov}
\examiner   {Professor Sheng Chen}

\maketitle
\begin{abstract}
This project aim was to model and analyse the effects of competitive pricing behaviors of grocery retailers on the British market. 

This was achieved by creating a multi-agent model, containing retailer and consumer agents. The heterogeneous crowd of retailers employs either a uniform pricing strategy or a ‘local price flexing’ strategy. The actions of these retailers are chosen by predicting the profit of each action, using a perceptron. Following on from the consideration of different economic models, a discrete model was developed so that software agents have a discrete environment to operate within. Within the model, it has been observed how supermarkets with differing behaviors affect a heterogeneous crowd of consumer agents. The model was implemented in Java with Python used to evaluate the results. 

The simulation displays good acceptance with real grocery market behavior, i.e. captures the performance of British retailers thus can be used to determine the impact of changes in their behavior on their competitors and consumers.Furthermore it can be used to provide insight into sustainability of volatile pricing strategies, providing a useful insight in volatility of British supermarket retail industry. 
\end{abstract}
\acknowledgements{
I would like to express my sincere gratitude to Dr Maria Polukarov for her guidance and support which provided me the freedom to take this research in the direction of my interest.\\
\\
I would also like to thank my family and friends for their encouragement and support. To those who quietly listened to my software complaints. To those who worked throughout the nights with me. To those who helped me write what I couldn't say. I cannot thank you enough.
}

\declaration{
I, Stefan Collier, declare that this dissertation and the work presented in it are my own and has been generated by me as the result of my own original research.\\
I confirm that:\\
1. This work was done wholly or mainly while in candidature for a degree at this University;\\
2. Where any part of this dissertation has previously been submitted for any other qualification at this University or any other institution, this has been clearly stated;\\
3. Where I have consulted the published work of others, this is always clearly attributed;\\
4. Where I have quoted from the work of others, the source is always given. With the exception of such quotations, this dissertation is entirely my own work;\\
5. I have acknowledged all main sources of help;\\
6. Where the thesis is based on work done by myself jointly with others, I have made clear exactly what was done by others and what I have contributed myself;\\
7. Either none of this work has been published before submission, or parts of this work have been published by :\\
\\
Stefan Collier\\
April 2016
}
\tableofcontents
\listoffigures
\listoftables

\mainmatter
%% ----------------------------------------------------------------
%\include{Introduction}
%\include{Conclusions}
\include{chapters/1Project/main}
\include{chapters/2Lit/main}
\include{chapters/3Design/HighLevel}
\include{chapters/3Design/InDepth}
\include{chapters/4Impl/main}

\include{chapters/5Experiments/1/main}
\include{chapters/5Experiments/2/main}
\include{chapters/5Experiments/3/main}
\include{chapters/5Experiments/4/main}

\include{chapters/6Conclusion/main}

\appendix
\include{appendix/AppendixB}
\include{appendix/D/main}
\include{appendix/AppendixC}

\backmatter
\bibliographystyle{ecs}
\bibliography{ECS}
\end{document}
%% ----------------------------------------------------------------

 %% ----------------------------------------------------------------
%% Progress.tex
%% ---------------------------------------------------------------- 
\documentclass{ecsprogress}    % Use the progress Style
\graphicspath{{../figs/}}   % Location of your graphics files
    \usepackage{natbib}            % Use Natbib style for the refs.
\hypersetup{colorlinks=true}   % Set to false for black/white printing
\input{Definitions}            % Include your abbreviations



\usepackage{enumitem}% http://ctan.org/pkg/enumitem
\usepackage{multirow}
\usepackage{float}
\usepackage{amsmath}
\usepackage{multicol}
\usepackage{amssymb}
\usepackage[normalem]{ulem}
\useunder{\uline}{\ul}{}
\usepackage{wrapfig}


\usepackage[table,xcdraw]{xcolor}


%% ----------------------------------------------------------------
\begin{document}
\frontmatter
\title      {Heterogeneous Agent-based Model for Supermarket Competition}
\authors    {\texorpdfstring
             {\href{mailto:sc22g13@ecs.soton.ac.uk}{Stefan J. Collier}}
             {Stefan J. Collier}
            }
\addresses  {\groupname\\\deptname\\\univname}
\date       {\today}
\subject    {}
\keywords   {}
\supervisor {Dr. Maria Polukarov}
\examiner   {Professor Sheng Chen}

\maketitle
\begin{abstract}
This project aim was to model and analyse the effects of competitive pricing behaviors of grocery retailers on the British market. 

This was achieved by creating a multi-agent model, containing retailer and consumer agents. The heterogeneous crowd of retailers employs either a uniform pricing strategy or a ‘local price flexing’ strategy. The actions of these retailers are chosen by predicting the profit of each action, using a perceptron. Following on from the consideration of different economic models, a discrete model was developed so that software agents have a discrete environment to operate within. Within the model, it has been observed how supermarkets with differing behaviors affect a heterogeneous crowd of consumer agents. The model was implemented in Java with Python used to evaluate the results. 

The simulation displays good acceptance with real grocery market behavior, i.e. captures the performance of British retailers thus can be used to determine the impact of changes in their behavior on their competitors and consumers.Furthermore it can be used to provide insight into sustainability of volatile pricing strategies, providing a useful insight in volatility of British supermarket retail industry. 
\end{abstract}
\acknowledgements{
I would like to express my sincere gratitude to Dr Maria Polukarov for her guidance and support which provided me the freedom to take this research in the direction of my interest.\\
\\
I would also like to thank my family and friends for their encouragement and support. To those who quietly listened to my software complaints. To those who worked throughout the nights with me. To those who helped me write what I couldn't say. I cannot thank you enough.
}

\declaration{
I, Stefan Collier, declare that this dissertation and the work presented in it are my own and has been generated by me as the result of my own original research.\\
I confirm that:\\
1. This work was done wholly or mainly while in candidature for a degree at this University;\\
2. Where any part of this dissertation has previously been submitted for any other qualification at this University or any other institution, this has been clearly stated;\\
3. Where I have consulted the published work of others, this is always clearly attributed;\\
4. Where I have quoted from the work of others, the source is always given. With the exception of such quotations, this dissertation is entirely my own work;\\
5. I have acknowledged all main sources of help;\\
6. Where the thesis is based on work done by myself jointly with others, I have made clear exactly what was done by others and what I have contributed myself;\\
7. Either none of this work has been published before submission, or parts of this work have been published by :\\
\\
Stefan Collier\\
April 2016
}
\tableofcontents
\listoffigures
\listoftables

\mainmatter
%% ----------------------------------------------------------------
%\include{Introduction}
%\include{Conclusions}
\include{chapters/1Project/main}
\include{chapters/2Lit/main}
\include{chapters/3Design/HighLevel}
\include{chapters/3Design/InDepth}
\include{chapters/4Impl/main}

\include{chapters/5Experiments/1/main}
\include{chapters/5Experiments/2/main}
\include{chapters/5Experiments/3/main}
\include{chapters/5Experiments/4/main}

\include{chapters/6Conclusion/main}

\appendix
\include{appendix/AppendixB}
\include{appendix/D/main}
\include{appendix/AppendixC}

\backmatter
\bibliographystyle{ecs}
\bibliography{ECS}
\end{document}
%% ----------------------------------------------------------------

 %% ----------------------------------------------------------------
%% Progress.tex
%% ---------------------------------------------------------------- 
\documentclass{ecsprogress}    % Use the progress Style
\graphicspath{{../figs/}}   % Location of your graphics files
    \usepackage{natbib}            % Use Natbib style for the refs.
\hypersetup{colorlinks=true}   % Set to false for black/white printing
\input{Definitions}            % Include your abbreviations



\usepackage{enumitem}% http://ctan.org/pkg/enumitem
\usepackage{multirow}
\usepackage{float}
\usepackage{amsmath}
\usepackage{multicol}
\usepackage{amssymb}
\usepackage[normalem]{ulem}
\useunder{\uline}{\ul}{}
\usepackage{wrapfig}


\usepackage[table,xcdraw]{xcolor}


%% ----------------------------------------------------------------
\begin{document}
\frontmatter
\title      {Heterogeneous Agent-based Model for Supermarket Competition}
\authors    {\texorpdfstring
             {\href{mailto:sc22g13@ecs.soton.ac.uk}{Stefan J. Collier}}
             {Stefan J. Collier}
            }
\addresses  {\groupname\\\deptname\\\univname}
\date       {\today}
\subject    {}
\keywords   {}
\supervisor {Dr. Maria Polukarov}
\examiner   {Professor Sheng Chen}

\maketitle
\begin{abstract}
This project aim was to model and analyse the effects of competitive pricing behaviors of grocery retailers on the British market. 

This was achieved by creating a multi-agent model, containing retailer and consumer agents. The heterogeneous crowd of retailers employs either a uniform pricing strategy or a ‘local price flexing’ strategy. The actions of these retailers are chosen by predicting the profit of each action, using a perceptron. Following on from the consideration of different economic models, a discrete model was developed so that software agents have a discrete environment to operate within. Within the model, it has been observed how supermarkets with differing behaviors affect a heterogeneous crowd of consumer agents. The model was implemented in Java with Python used to evaluate the results. 

The simulation displays good acceptance with real grocery market behavior, i.e. captures the performance of British retailers thus can be used to determine the impact of changes in their behavior on their competitors and consumers.Furthermore it can be used to provide insight into sustainability of volatile pricing strategies, providing a useful insight in volatility of British supermarket retail industry. 
\end{abstract}
\acknowledgements{
I would like to express my sincere gratitude to Dr Maria Polukarov for her guidance and support which provided me the freedom to take this research in the direction of my interest.\\
\\
I would also like to thank my family and friends for their encouragement and support. To those who quietly listened to my software complaints. To those who worked throughout the nights with me. To those who helped me write what I couldn't say. I cannot thank you enough.
}

\declaration{
I, Stefan Collier, declare that this dissertation and the work presented in it are my own and has been generated by me as the result of my own original research.\\
I confirm that:\\
1. This work was done wholly or mainly while in candidature for a degree at this University;\\
2. Where any part of this dissertation has previously been submitted for any other qualification at this University or any other institution, this has been clearly stated;\\
3. Where I have consulted the published work of others, this is always clearly attributed;\\
4. Where I have quoted from the work of others, the source is always given. With the exception of such quotations, this dissertation is entirely my own work;\\
5. I have acknowledged all main sources of help;\\
6. Where the thesis is based on work done by myself jointly with others, I have made clear exactly what was done by others and what I have contributed myself;\\
7. Either none of this work has been published before submission, or parts of this work have been published by :\\
\\
Stefan Collier\\
April 2016
}
\tableofcontents
\listoffigures
\listoftables

\mainmatter
%% ----------------------------------------------------------------
%\include{Introduction}
%\include{Conclusions}
\include{chapters/1Project/main}
\include{chapters/2Lit/main}
\include{chapters/3Design/HighLevel}
\include{chapters/3Design/InDepth}
\include{chapters/4Impl/main}

\include{chapters/5Experiments/1/main}
\include{chapters/5Experiments/2/main}
\include{chapters/5Experiments/3/main}
\include{chapters/5Experiments/4/main}

\include{chapters/6Conclusion/main}

\appendix
\include{appendix/AppendixB}
\include{appendix/D/main}
\include{appendix/AppendixC}

\backmatter
\bibliographystyle{ecs}
\bibliography{ECS}
\end{document}
%% ----------------------------------------------------------------


 %% ----------------------------------------------------------------
%% Progress.tex
%% ---------------------------------------------------------------- 
\documentclass{ecsprogress}    % Use the progress Style
\graphicspath{{../figs/}}   % Location of your graphics files
    \usepackage{natbib}            % Use Natbib style for the refs.
\hypersetup{colorlinks=true}   % Set to false for black/white printing
\input{Definitions}            % Include your abbreviations



\usepackage{enumitem}% http://ctan.org/pkg/enumitem
\usepackage{multirow}
\usepackage{float}
\usepackage{amsmath}
\usepackage{multicol}
\usepackage{amssymb}
\usepackage[normalem]{ulem}
\useunder{\uline}{\ul}{}
\usepackage{wrapfig}


\usepackage[table,xcdraw]{xcolor}


%% ----------------------------------------------------------------
\begin{document}
\frontmatter
\title      {Heterogeneous Agent-based Model for Supermarket Competition}
\authors    {\texorpdfstring
             {\href{mailto:sc22g13@ecs.soton.ac.uk}{Stefan J. Collier}}
             {Stefan J. Collier}
            }
\addresses  {\groupname\\\deptname\\\univname}
\date       {\today}
\subject    {}
\keywords   {}
\supervisor {Dr. Maria Polukarov}
\examiner   {Professor Sheng Chen}

\maketitle
\begin{abstract}
This project aim was to model and analyse the effects of competitive pricing behaviors of grocery retailers on the British market. 

This was achieved by creating a multi-agent model, containing retailer and consumer agents. The heterogeneous crowd of retailers employs either a uniform pricing strategy or a ‘local price flexing’ strategy. The actions of these retailers are chosen by predicting the profit of each action, using a perceptron. Following on from the consideration of different economic models, a discrete model was developed so that software agents have a discrete environment to operate within. Within the model, it has been observed how supermarkets with differing behaviors affect a heterogeneous crowd of consumer agents. The model was implemented in Java with Python used to evaluate the results. 

The simulation displays good acceptance with real grocery market behavior, i.e. captures the performance of British retailers thus can be used to determine the impact of changes in their behavior on their competitors and consumers.Furthermore it can be used to provide insight into sustainability of volatile pricing strategies, providing a useful insight in volatility of British supermarket retail industry. 
\end{abstract}
\acknowledgements{
I would like to express my sincere gratitude to Dr Maria Polukarov for her guidance and support which provided me the freedom to take this research in the direction of my interest.\\
\\
I would also like to thank my family and friends for their encouragement and support. To those who quietly listened to my software complaints. To those who worked throughout the nights with me. To those who helped me write what I couldn't say. I cannot thank you enough.
}

\declaration{
I, Stefan Collier, declare that this dissertation and the work presented in it are my own and has been generated by me as the result of my own original research.\\
I confirm that:\\
1. This work was done wholly or mainly while in candidature for a degree at this University;\\
2. Where any part of this dissertation has previously been submitted for any other qualification at this University or any other institution, this has been clearly stated;\\
3. Where I have consulted the published work of others, this is always clearly attributed;\\
4. Where I have quoted from the work of others, the source is always given. With the exception of such quotations, this dissertation is entirely my own work;\\
5. I have acknowledged all main sources of help;\\
6. Where the thesis is based on work done by myself jointly with others, I have made clear exactly what was done by others and what I have contributed myself;\\
7. Either none of this work has been published before submission, or parts of this work have been published by :\\
\\
Stefan Collier\\
April 2016
}
\tableofcontents
\listoffigures
\listoftables

\mainmatter
%% ----------------------------------------------------------------
%\include{Introduction}
%\include{Conclusions}
\include{chapters/1Project/main}
\include{chapters/2Lit/main}
\include{chapters/3Design/HighLevel}
\include{chapters/3Design/InDepth}
\include{chapters/4Impl/main}

\include{chapters/5Experiments/1/main}
\include{chapters/5Experiments/2/main}
\include{chapters/5Experiments/3/main}
\include{chapters/5Experiments/4/main}

\include{chapters/6Conclusion/main}

\appendix
\include{appendix/AppendixB}
\include{appendix/D/main}
\include{appendix/AppendixC}

\backmatter
\bibliographystyle{ecs}
\bibliography{ECS}
\end{document}
%% ----------------------------------------------------------------


\appendix
\include{appendix/AppendixB}
 %% ----------------------------------------------------------------
%% Progress.tex
%% ---------------------------------------------------------------- 
\documentclass{ecsprogress}    % Use the progress Style
\graphicspath{{../figs/}}   % Location of your graphics files
    \usepackage{natbib}            % Use Natbib style for the refs.
\hypersetup{colorlinks=true}   % Set to false for black/white printing
\input{Definitions}            % Include your abbreviations



\usepackage{enumitem}% http://ctan.org/pkg/enumitem
\usepackage{multirow}
\usepackage{float}
\usepackage{amsmath}
\usepackage{multicol}
\usepackage{amssymb}
\usepackage[normalem]{ulem}
\useunder{\uline}{\ul}{}
\usepackage{wrapfig}


\usepackage[table,xcdraw]{xcolor}


%% ----------------------------------------------------------------
\begin{document}
\frontmatter
\title      {Heterogeneous Agent-based Model for Supermarket Competition}
\authors    {\texorpdfstring
             {\href{mailto:sc22g13@ecs.soton.ac.uk}{Stefan J. Collier}}
             {Stefan J. Collier}
            }
\addresses  {\groupname\\\deptname\\\univname}
\date       {\today}
\subject    {}
\keywords   {}
\supervisor {Dr. Maria Polukarov}
\examiner   {Professor Sheng Chen}

\maketitle
\begin{abstract}
This project aim was to model and analyse the effects of competitive pricing behaviors of grocery retailers on the British market. 

This was achieved by creating a multi-agent model, containing retailer and consumer agents. The heterogeneous crowd of retailers employs either a uniform pricing strategy or a ‘local price flexing’ strategy. The actions of these retailers are chosen by predicting the profit of each action, using a perceptron. Following on from the consideration of different economic models, a discrete model was developed so that software agents have a discrete environment to operate within. Within the model, it has been observed how supermarkets with differing behaviors affect a heterogeneous crowd of consumer agents. The model was implemented in Java with Python used to evaluate the results. 

The simulation displays good acceptance with real grocery market behavior, i.e. captures the performance of British retailers thus can be used to determine the impact of changes in their behavior on their competitors and consumers.Furthermore it can be used to provide insight into sustainability of volatile pricing strategies, providing a useful insight in volatility of British supermarket retail industry. 
\end{abstract}
\acknowledgements{
I would like to express my sincere gratitude to Dr Maria Polukarov for her guidance and support which provided me the freedom to take this research in the direction of my interest.\\
\\
I would also like to thank my family and friends for their encouragement and support. To those who quietly listened to my software complaints. To those who worked throughout the nights with me. To those who helped me write what I couldn't say. I cannot thank you enough.
}

\declaration{
I, Stefan Collier, declare that this dissertation and the work presented in it are my own and has been generated by me as the result of my own original research.\\
I confirm that:\\
1. This work was done wholly or mainly while in candidature for a degree at this University;\\
2. Where any part of this dissertation has previously been submitted for any other qualification at this University or any other institution, this has been clearly stated;\\
3. Where I have consulted the published work of others, this is always clearly attributed;\\
4. Where I have quoted from the work of others, the source is always given. With the exception of such quotations, this dissertation is entirely my own work;\\
5. I have acknowledged all main sources of help;\\
6. Where the thesis is based on work done by myself jointly with others, I have made clear exactly what was done by others and what I have contributed myself;\\
7. Either none of this work has been published before submission, or parts of this work have been published by :\\
\\
Stefan Collier\\
April 2016
}
\tableofcontents
\listoffigures
\listoftables

\mainmatter
%% ----------------------------------------------------------------
%\include{Introduction}
%\include{Conclusions}
\include{chapters/1Project/main}
\include{chapters/2Lit/main}
\include{chapters/3Design/HighLevel}
\include{chapters/3Design/InDepth}
\include{chapters/4Impl/main}

\include{chapters/5Experiments/1/main}
\include{chapters/5Experiments/2/main}
\include{chapters/5Experiments/3/main}
\include{chapters/5Experiments/4/main}

\include{chapters/6Conclusion/main}

\appendix
\include{appendix/AppendixB}
\include{appendix/D/main}
\include{appendix/AppendixC}

\backmatter
\bibliographystyle{ecs}
\bibliography{ECS}
\end{document}
%% ----------------------------------------------------------------

\include{appendix/AppendixC}

\backmatter
\bibliographystyle{ecs}
\bibliography{ECS}
\end{document}
%% ----------------------------------------------------------------


\appendix
\include{appendix/AppendixB}
 %% ----------------------------------------------------------------
%% Progress.tex
%% ---------------------------------------------------------------- 
\documentclass{ecsprogress}    % Use the progress Style
\graphicspath{{../figs/}}   % Location of your graphics files
    \usepackage{natbib}            % Use Natbib style for the refs.
\hypersetup{colorlinks=true}   % Set to false for black/white printing
\input{Definitions}            % Include your abbreviations



\usepackage{enumitem}% http://ctan.org/pkg/enumitem
\usepackage{multirow}
\usepackage{float}
\usepackage{amsmath}
\usepackage{multicol}
\usepackage{amssymb}
\usepackage[normalem]{ulem}
\useunder{\uline}{\ul}{}
\usepackage{wrapfig}


\usepackage[table,xcdraw]{xcolor}


%% ----------------------------------------------------------------
\begin{document}
\frontmatter
\title      {Heterogeneous Agent-based Model for Supermarket Competition}
\authors    {\texorpdfstring
             {\href{mailto:sc22g13@ecs.soton.ac.uk}{Stefan J. Collier}}
             {Stefan J. Collier}
            }
\addresses  {\groupname\\\deptname\\\univname}
\date       {\today}
\subject    {}
\keywords   {}
\supervisor {Dr. Maria Polukarov}
\examiner   {Professor Sheng Chen}

\maketitle
\begin{abstract}
This project aim was to model and analyse the effects of competitive pricing behaviors of grocery retailers on the British market. 

This was achieved by creating a multi-agent model, containing retailer and consumer agents. The heterogeneous crowd of retailers employs either a uniform pricing strategy or a ‘local price flexing’ strategy. The actions of these retailers are chosen by predicting the profit of each action, using a perceptron. Following on from the consideration of different economic models, a discrete model was developed so that software agents have a discrete environment to operate within. Within the model, it has been observed how supermarkets with differing behaviors affect a heterogeneous crowd of consumer agents. The model was implemented in Java with Python used to evaluate the results. 

The simulation displays good acceptance with real grocery market behavior, i.e. captures the performance of British retailers thus can be used to determine the impact of changes in their behavior on their competitors and consumers.Furthermore it can be used to provide insight into sustainability of volatile pricing strategies, providing a useful insight in volatility of British supermarket retail industry. 
\end{abstract}
\acknowledgements{
I would like to express my sincere gratitude to Dr Maria Polukarov for her guidance and support which provided me the freedom to take this research in the direction of my interest.\\
\\
I would also like to thank my family and friends for their encouragement and support. To those who quietly listened to my software complaints. To those who worked throughout the nights with me. To those who helped me write what I couldn't say. I cannot thank you enough.
}

\declaration{
I, Stefan Collier, declare that this dissertation and the work presented in it are my own and has been generated by me as the result of my own original research.\\
I confirm that:\\
1. This work was done wholly or mainly while in candidature for a degree at this University;\\
2. Where any part of this dissertation has previously been submitted for any other qualification at this University or any other institution, this has been clearly stated;\\
3. Where I have consulted the published work of others, this is always clearly attributed;\\
4. Where I have quoted from the work of others, the source is always given. With the exception of such quotations, this dissertation is entirely my own work;\\
5. I have acknowledged all main sources of help;\\
6. Where the thesis is based on work done by myself jointly with others, I have made clear exactly what was done by others and what I have contributed myself;\\
7. Either none of this work has been published before submission, or parts of this work have been published by :\\
\\
Stefan Collier\\
April 2016
}
\tableofcontents
\listoffigures
\listoftables

\mainmatter
%% ----------------------------------------------------------------
%\include{Introduction}
%\include{Conclusions}
 %% ----------------------------------------------------------------
%% Progress.tex
%% ---------------------------------------------------------------- 
\documentclass{ecsprogress}    % Use the progress Style
\graphicspath{{../figs/}}   % Location of your graphics files
    \usepackage{natbib}            % Use Natbib style for the refs.
\hypersetup{colorlinks=true}   % Set to false for black/white printing
\input{Definitions}            % Include your abbreviations



\usepackage{enumitem}% http://ctan.org/pkg/enumitem
\usepackage{multirow}
\usepackage{float}
\usepackage{amsmath}
\usepackage{multicol}
\usepackage{amssymb}
\usepackage[normalem]{ulem}
\useunder{\uline}{\ul}{}
\usepackage{wrapfig}


\usepackage[table,xcdraw]{xcolor}


%% ----------------------------------------------------------------
\begin{document}
\frontmatter
\title      {Heterogeneous Agent-based Model for Supermarket Competition}
\authors    {\texorpdfstring
             {\href{mailto:sc22g13@ecs.soton.ac.uk}{Stefan J. Collier}}
             {Stefan J. Collier}
            }
\addresses  {\groupname\\\deptname\\\univname}
\date       {\today}
\subject    {}
\keywords   {}
\supervisor {Dr. Maria Polukarov}
\examiner   {Professor Sheng Chen}

\maketitle
\begin{abstract}
This project aim was to model and analyse the effects of competitive pricing behaviors of grocery retailers on the British market. 

This was achieved by creating a multi-agent model, containing retailer and consumer agents. The heterogeneous crowd of retailers employs either a uniform pricing strategy or a ‘local price flexing’ strategy. The actions of these retailers are chosen by predicting the profit of each action, using a perceptron. Following on from the consideration of different economic models, a discrete model was developed so that software agents have a discrete environment to operate within. Within the model, it has been observed how supermarkets with differing behaviors affect a heterogeneous crowd of consumer agents. The model was implemented in Java with Python used to evaluate the results. 

The simulation displays good acceptance with real grocery market behavior, i.e. captures the performance of British retailers thus can be used to determine the impact of changes in their behavior on their competitors and consumers.Furthermore it can be used to provide insight into sustainability of volatile pricing strategies, providing a useful insight in volatility of British supermarket retail industry. 
\end{abstract}
\acknowledgements{
I would like to express my sincere gratitude to Dr Maria Polukarov for her guidance and support which provided me the freedom to take this research in the direction of my interest.\\
\\
I would also like to thank my family and friends for their encouragement and support. To those who quietly listened to my software complaints. To those who worked throughout the nights with me. To those who helped me write what I couldn't say. I cannot thank you enough.
}

\declaration{
I, Stefan Collier, declare that this dissertation and the work presented in it are my own and has been generated by me as the result of my own original research.\\
I confirm that:\\
1. This work was done wholly or mainly while in candidature for a degree at this University;\\
2. Where any part of this dissertation has previously been submitted for any other qualification at this University or any other institution, this has been clearly stated;\\
3. Where I have consulted the published work of others, this is always clearly attributed;\\
4. Where I have quoted from the work of others, the source is always given. With the exception of such quotations, this dissertation is entirely my own work;\\
5. I have acknowledged all main sources of help;\\
6. Where the thesis is based on work done by myself jointly with others, I have made clear exactly what was done by others and what I have contributed myself;\\
7. Either none of this work has been published before submission, or parts of this work have been published by :\\
\\
Stefan Collier\\
April 2016
}
\tableofcontents
\listoffigures
\listoftables

\mainmatter
%% ----------------------------------------------------------------
%\include{Introduction}
%\include{Conclusions}
\include{chapters/1Project/main}
\include{chapters/2Lit/main}
\include{chapters/3Design/HighLevel}
\include{chapters/3Design/InDepth}
\include{chapters/4Impl/main}

\include{chapters/5Experiments/1/main}
\include{chapters/5Experiments/2/main}
\include{chapters/5Experiments/3/main}
\include{chapters/5Experiments/4/main}

\include{chapters/6Conclusion/main}

\appendix
\include{appendix/AppendixB}
\include{appendix/D/main}
\include{appendix/AppendixC}

\backmatter
\bibliographystyle{ecs}
\bibliography{ECS}
\end{document}
%% ----------------------------------------------------------------

 %% ----------------------------------------------------------------
%% Progress.tex
%% ---------------------------------------------------------------- 
\documentclass{ecsprogress}    % Use the progress Style
\graphicspath{{../figs/}}   % Location of your graphics files
    \usepackage{natbib}            % Use Natbib style for the refs.
\hypersetup{colorlinks=true}   % Set to false for black/white printing
\input{Definitions}            % Include your abbreviations



\usepackage{enumitem}% http://ctan.org/pkg/enumitem
\usepackage{multirow}
\usepackage{float}
\usepackage{amsmath}
\usepackage{multicol}
\usepackage{amssymb}
\usepackage[normalem]{ulem}
\useunder{\uline}{\ul}{}
\usepackage{wrapfig}


\usepackage[table,xcdraw]{xcolor}


%% ----------------------------------------------------------------
\begin{document}
\frontmatter
\title      {Heterogeneous Agent-based Model for Supermarket Competition}
\authors    {\texorpdfstring
             {\href{mailto:sc22g13@ecs.soton.ac.uk}{Stefan J. Collier}}
             {Stefan J. Collier}
            }
\addresses  {\groupname\\\deptname\\\univname}
\date       {\today}
\subject    {}
\keywords   {}
\supervisor {Dr. Maria Polukarov}
\examiner   {Professor Sheng Chen}

\maketitle
\begin{abstract}
This project aim was to model and analyse the effects of competitive pricing behaviors of grocery retailers on the British market. 

This was achieved by creating a multi-agent model, containing retailer and consumer agents. The heterogeneous crowd of retailers employs either a uniform pricing strategy or a ‘local price flexing’ strategy. The actions of these retailers are chosen by predicting the profit of each action, using a perceptron. Following on from the consideration of different economic models, a discrete model was developed so that software agents have a discrete environment to operate within. Within the model, it has been observed how supermarkets with differing behaviors affect a heterogeneous crowd of consumer agents. The model was implemented in Java with Python used to evaluate the results. 

The simulation displays good acceptance with real grocery market behavior, i.e. captures the performance of British retailers thus can be used to determine the impact of changes in their behavior on their competitors and consumers.Furthermore it can be used to provide insight into sustainability of volatile pricing strategies, providing a useful insight in volatility of British supermarket retail industry. 
\end{abstract}
\acknowledgements{
I would like to express my sincere gratitude to Dr Maria Polukarov for her guidance and support which provided me the freedom to take this research in the direction of my interest.\\
\\
I would also like to thank my family and friends for their encouragement and support. To those who quietly listened to my software complaints. To those who worked throughout the nights with me. To those who helped me write what I couldn't say. I cannot thank you enough.
}

\declaration{
I, Stefan Collier, declare that this dissertation and the work presented in it are my own and has been generated by me as the result of my own original research.\\
I confirm that:\\
1. This work was done wholly or mainly while in candidature for a degree at this University;\\
2. Where any part of this dissertation has previously been submitted for any other qualification at this University or any other institution, this has been clearly stated;\\
3. Where I have consulted the published work of others, this is always clearly attributed;\\
4. Where I have quoted from the work of others, the source is always given. With the exception of such quotations, this dissertation is entirely my own work;\\
5. I have acknowledged all main sources of help;\\
6. Where the thesis is based on work done by myself jointly with others, I have made clear exactly what was done by others and what I have contributed myself;\\
7. Either none of this work has been published before submission, or parts of this work have been published by :\\
\\
Stefan Collier\\
April 2016
}
\tableofcontents
\listoffigures
\listoftables

\mainmatter
%% ----------------------------------------------------------------
%\include{Introduction}
%\include{Conclusions}
\include{chapters/1Project/main}
\include{chapters/2Lit/main}
\include{chapters/3Design/HighLevel}
\include{chapters/3Design/InDepth}
\include{chapters/4Impl/main}

\include{chapters/5Experiments/1/main}
\include{chapters/5Experiments/2/main}
\include{chapters/5Experiments/3/main}
\include{chapters/5Experiments/4/main}

\include{chapters/6Conclusion/main}

\appendix
\include{appendix/AppendixB}
\include{appendix/D/main}
\include{appendix/AppendixC}

\backmatter
\bibliographystyle{ecs}
\bibliography{ECS}
\end{document}
%% ----------------------------------------------------------------

\include{chapters/3Design/HighLevel}
\include{chapters/3Design/InDepth}
 %% ----------------------------------------------------------------
%% Progress.tex
%% ---------------------------------------------------------------- 
\documentclass{ecsprogress}    % Use the progress Style
\graphicspath{{../figs/}}   % Location of your graphics files
    \usepackage{natbib}            % Use Natbib style for the refs.
\hypersetup{colorlinks=true}   % Set to false for black/white printing
\input{Definitions}            % Include your abbreviations



\usepackage{enumitem}% http://ctan.org/pkg/enumitem
\usepackage{multirow}
\usepackage{float}
\usepackage{amsmath}
\usepackage{multicol}
\usepackage{amssymb}
\usepackage[normalem]{ulem}
\useunder{\uline}{\ul}{}
\usepackage{wrapfig}


\usepackage[table,xcdraw]{xcolor}


%% ----------------------------------------------------------------
\begin{document}
\frontmatter
\title      {Heterogeneous Agent-based Model for Supermarket Competition}
\authors    {\texorpdfstring
             {\href{mailto:sc22g13@ecs.soton.ac.uk}{Stefan J. Collier}}
             {Stefan J. Collier}
            }
\addresses  {\groupname\\\deptname\\\univname}
\date       {\today}
\subject    {}
\keywords   {}
\supervisor {Dr. Maria Polukarov}
\examiner   {Professor Sheng Chen}

\maketitle
\begin{abstract}
This project aim was to model and analyse the effects of competitive pricing behaviors of grocery retailers on the British market. 

This was achieved by creating a multi-agent model, containing retailer and consumer agents. The heterogeneous crowd of retailers employs either a uniform pricing strategy or a ‘local price flexing’ strategy. The actions of these retailers are chosen by predicting the profit of each action, using a perceptron. Following on from the consideration of different economic models, a discrete model was developed so that software agents have a discrete environment to operate within. Within the model, it has been observed how supermarkets with differing behaviors affect a heterogeneous crowd of consumer agents. The model was implemented in Java with Python used to evaluate the results. 

The simulation displays good acceptance with real grocery market behavior, i.e. captures the performance of British retailers thus can be used to determine the impact of changes in their behavior on their competitors and consumers.Furthermore it can be used to provide insight into sustainability of volatile pricing strategies, providing a useful insight in volatility of British supermarket retail industry. 
\end{abstract}
\acknowledgements{
I would like to express my sincere gratitude to Dr Maria Polukarov for her guidance and support which provided me the freedom to take this research in the direction of my interest.\\
\\
I would also like to thank my family and friends for their encouragement and support. To those who quietly listened to my software complaints. To those who worked throughout the nights with me. To those who helped me write what I couldn't say. I cannot thank you enough.
}

\declaration{
I, Stefan Collier, declare that this dissertation and the work presented in it are my own and has been generated by me as the result of my own original research.\\
I confirm that:\\
1. This work was done wholly or mainly while in candidature for a degree at this University;\\
2. Where any part of this dissertation has previously been submitted for any other qualification at this University or any other institution, this has been clearly stated;\\
3. Where I have consulted the published work of others, this is always clearly attributed;\\
4. Where I have quoted from the work of others, the source is always given. With the exception of such quotations, this dissertation is entirely my own work;\\
5. I have acknowledged all main sources of help;\\
6. Where the thesis is based on work done by myself jointly with others, I have made clear exactly what was done by others and what I have contributed myself;\\
7. Either none of this work has been published before submission, or parts of this work have been published by :\\
\\
Stefan Collier\\
April 2016
}
\tableofcontents
\listoffigures
\listoftables

\mainmatter
%% ----------------------------------------------------------------
%\include{Introduction}
%\include{Conclusions}
\include{chapters/1Project/main}
\include{chapters/2Lit/main}
\include{chapters/3Design/HighLevel}
\include{chapters/3Design/InDepth}
\include{chapters/4Impl/main}

\include{chapters/5Experiments/1/main}
\include{chapters/5Experiments/2/main}
\include{chapters/5Experiments/3/main}
\include{chapters/5Experiments/4/main}

\include{chapters/6Conclusion/main}

\appendix
\include{appendix/AppendixB}
\include{appendix/D/main}
\include{appendix/AppendixC}

\backmatter
\bibliographystyle{ecs}
\bibliography{ECS}
\end{document}
%% ----------------------------------------------------------------


 %% ----------------------------------------------------------------
%% Progress.tex
%% ---------------------------------------------------------------- 
\documentclass{ecsprogress}    % Use the progress Style
\graphicspath{{../figs/}}   % Location of your graphics files
    \usepackage{natbib}            % Use Natbib style for the refs.
\hypersetup{colorlinks=true}   % Set to false for black/white printing
\input{Definitions}            % Include your abbreviations



\usepackage{enumitem}% http://ctan.org/pkg/enumitem
\usepackage{multirow}
\usepackage{float}
\usepackage{amsmath}
\usepackage{multicol}
\usepackage{amssymb}
\usepackage[normalem]{ulem}
\useunder{\uline}{\ul}{}
\usepackage{wrapfig}


\usepackage[table,xcdraw]{xcolor}


%% ----------------------------------------------------------------
\begin{document}
\frontmatter
\title      {Heterogeneous Agent-based Model for Supermarket Competition}
\authors    {\texorpdfstring
             {\href{mailto:sc22g13@ecs.soton.ac.uk}{Stefan J. Collier}}
             {Stefan J. Collier}
            }
\addresses  {\groupname\\\deptname\\\univname}
\date       {\today}
\subject    {}
\keywords   {}
\supervisor {Dr. Maria Polukarov}
\examiner   {Professor Sheng Chen}

\maketitle
\begin{abstract}
This project aim was to model and analyse the effects of competitive pricing behaviors of grocery retailers on the British market. 

This was achieved by creating a multi-agent model, containing retailer and consumer agents. The heterogeneous crowd of retailers employs either a uniform pricing strategy or a ‘local price flexing’ strategy. The actions of these retailers are chosen by predicting the profit of each action, using a perceptron. Following on from the consideration of different economic models, a discrete model was developed so that software agents have a discrete environment to operate within. Within the model, it has been observed how supermarkets with differing behaviors affect a heterogeneous crowd of consumer agents. The model was implemented in Java with Python used to evaluate the results. 

The simulation displays good acceptance with real grocery market behavior, i.e. captures the performance of British retailers thus can be used to determine the impact of changes in their behavior on their competitors and consumers.Furthermore it can be used to provide insight into sustainability of volatile pricing strategies, providing a useful insight in volatility of British supermarket retail industry. 
\end{abstract}
\acknowledgements{
I would like to express my sincere gratitude to Dr Maria Polukarov for her guidance and support which provided me the freedom to take this research in the direction of my interest.\\
\\
I would also like to thank my family and friends for their encouragement and support. To those who quietly listened to my software complaints. To those who worked throughout the nights with me. To those who helped me write what I couldn't say. I cannot thank you enough.
}

\declaration{
I, Stefan Collier, declare that this dissertation and the work presented in it are my own and has been generated by me as the result of my own original research.\\
I confirm that:\\
1. This work was done wholly or mainly while in candidature for a degree at this University;\\
2. Where any part of this dissertation has previously been submitted for any other qualification at this University or any other institution, this has been clearly stated;\\
3. Where I have consulted the published work of others, this is always clearly attributed;\\
4. Where I have quoted from the work of others, the source is always given. With the exception of such quotations, this dissertation is entirely my own work;\\
5. I have acknowledged all main sources of help;\\
6. Where the thesis is based on work done by myself jointly with others, I have made clear exactly what was done by others and what I have contributed myself;\\
7. Either none of this work has been published before submission, or parts of this work have been published by :\\
\\
Stefan Collier\\
April 2016
}
\tableofcontents
\listoffigures
\listoftables

\mainmatter
%% ----------------------------------------------------------------
%\include{Introduction}
%\include{Conclusions}
\include{chapters/1Project/main}
\include{chapters/2Lit/main}
\include{chapters/3Design/HighLevel}
\include{chapters/3Design/InDepth}
\include{chapters/4Impl/main}

\include{chapters/5Experiments/1/main}
\include{chapters/5Experiments/2/main}
\include{chapters/5Experiments/3/main}
\include{chapters/5Experiments/4/main}

\include{chapters/6Conclusion/main}

\appendix
\include{appendix/AppendixB}
\include{appendix/D/main}
\include{appendix/AppendixC}

\backmatter
\bibliographystyle{ecs}
\bibliography{ECS}
\end{document}
%% ----------------------------------------------------------------

 %% ----------------------------------------------------------------
%% Progress.tex
%% ---------------------------------------------------------------- 
\documentclass{ecsprogress}    % Use the progress Style
\graphicspath{{../figs/}}   % Location of your graphics files
    \usepackage{natbib}            % Use Natbib style for the refs.
\hypersetup{colorlinks=true}   % Set to false for black/white printing
\input{Definitions}            % Include your abbreviations



\usepackage{enumitem}% http://ctan.org/pkg/enumitem
\usepackage{multirow}
\usepackage{float}
\usepackage{amsmath}
\usepackage{multicol}
\usepackage{amssymb}
\usepackage[normalem]{ulem}
\useunder{\uline}{\ul}{}
\usepackage{wrapfig}


\usepackage[table,xcdraw]{xcolor}


%% ----------------------------------------------------------------
\begin{document}
\frontmatter
\title      {Heterogeneous Agent-based Model for Supermarket Competition}
\authors    {\texorpdfstring
             {\href{mailto:sc22g13@ecs.soton.ac.uk}{Stefan J. Collier}}
             {Stefan J. Collier}
            }
\addresses  {\groupname\\\deptname\\\univname}
\date       {\today}
\subject    {}
\keywords   {}
\supervisor {Dr. Maria Polukarov}
\examiner   {Professor Sheng Chen}

\maketitle
\begin{abstract}
This project aim was to model and analyse the effects of competitive pricing behaviors of grocery retailers on the British market. 

This was achieved by creating a multi-agent model, containing retailer and consumer agents. The heterogeneous crowd of retailers employs either a uniform pricing strategy or a ‘local price flexing’ strategy. The actions of these retailers are chosen by predicting the profit of each action, using a perceptron. Following on from the consideration of different economic models, a discrete model was developed so that software agents have a discrete environment to operate within. Within the model, it has been observed how supermarkets with differing behaviors affect a heterogeneous crowd of consumer agents. The model was implemented in Java with Python used to evaluate the results. 

The simulation displays good acceptance with real grocery market behavior, i.e. captures the performance of British retailers thus can be used to determine the impact of changes in their behavior on their competitors and consumers.Furthermore it can be used to provide insight into sustainability of volatile pricing strategies, providing a useful insight in volatility of British supermarket retail industry. 
\end{abstract}
\acknowledgements{
I would like to express my sincere gratitude to Dr Maria Polukarov for her guidance and support which provided me the freedom to take this research in the direction of my interest.\\
\\
I would also like to thank my family and friends for their encouragement and support. To those who quietly listened to my software complaints. To those who worked throughout the nights with me. To those who helped me write what I couldn't say. I cannot thank you enough.
}

\declaration{
I, Stefan Collier, declare that this dissertation and the work presented in it are my own and has been generated by me as the result of my own original research.\\
I confirm that:\\
1. This work was done wholly or mainly while in candidature for a degree at this University;\\
2. Where any part of this dissertation has previously been submitted for any other qualification at this University or any other institution, this has been clearly stated;\\
3. Where I have consulted the published work of others, this is always clearly attributed;\\
4. Where I have quoted from the work of others, the source is always given. With the exception of such quotations, this dissertation is entirely my own work;\\
5. I have acknowledged all main sources of help;\\
6. Where the thesis is based on work done by myself jointly with others, I have made clear exactly what was done by others and what I have contributed myself;\\
7. Either none of this work has been published before submission, or parts of this work have been published by :\\
\\
Stefan Collier\\
April 2016
}
\tableofcontents
\listoffigures
\listoftables

\mainmatter
%% ----------------------------------------------------------------
%\include{Introduction}
%\include{Conclusions}
\include{chapters/1Project/main}
\include{chapters/2Lit/main}
\include{chapters/3Design/HighLevel}
\include{chapters/3Design/InDepth}
\include{chapters/4Impl/main}

\include{chapters/5Experiments/1/main}
\include{chapters/5Experiments/2/main}
\include{chapters/5Experiments/3/main}
\include{chapters/5Experiments/4/main}

\include{chapters/6Conclusion/main}

\appendix
\include{appendix/AppendixB}
\include{appendix/D/main}
\include{appendix/AppendixC}

\backmatter
\bibliographystyle{ecs}
\bibliography{ECS}
\end{document}
%% ----------------------------------------------------------------

 %% ----------------------------------------------------------------
%% Progress.tex
%% ---------------------------------------------------------------- 
\documentclass{ecsprogress}    % Use the progress Style
\graphicspath{{../figs/}}   % Location of your graphics files
    \usepackage{natbib}            % Use Natbib style for the refs.
\hypersetup{colorlinks=true}   % Set to false for black/white printing
\input{Definitions}            % Include your abbreviations



\usepackage{enumitem}% http://ctan.org/pkg/enumitem
\usepackage{multirow}
\usepackage{float}
\usepackage{amsmath}
\usepackage{multicol}
\usepackage{amssymb}
\usepackage[normalem]{ulem}
\useunder{\uline}{\ul}{}
\usepackage{wrapfig}


\usepackage[table,xcdraw]{xcolor}


%% ----------------------------------------------------------------
\begin{document}
\frontmatter
\title      {Heterogeneous Agent-based Model for Supermarket Competition}
\authors    {\texorpdfstring
             {\href{mailto:sc22g13@ecs.soton.ac.uk}{Stefan J. Collier}}
             {Stefan J. Collier}
            }
\addresses  {\groupname\\\deptname\\\univname}
\date       {\today}
\subject    {}
\keywords   {}
\supervisor {Dr. Maria Polukarov}
\examiner   {Professor Sheng Chen}

\maketitle
\begin{abstract}
This project aim was to model and analyse the effects of competitive pricing behaviors of grocery retailers on the British market. 

This was achieved by creating a multi-agent model, containing retailer and consumer agents. The heterogeneous crowd of retailers employs either a uniform pricing strategy or a ‘local price flexing’ strategy. The actions of these retailers are chosen by predicting the profit of each action, using a perceptron. Following on from the consideration of different economic models, a discrete model was developed so that software agents have a discrete environment to operate within. Within the model, it has been observed how supermarkets with differing behaviors affect a heterogeneous crowd of consumer agents. The model was implemented in Java with Python used to evaluate the results. 

The simulation displays good acceptance with real grocery market behavior, i.e. captures the performance of British retailers thus can be used to determine the impact of changes in their behavior on their competitors and consumers.Furthermore it can be used to provide insight into sustainability of volatile pricing strategies, providing a useful insight in volatility of British supermarket retail industry. 
\end{abstract}
\acknowledgements{
I would like to express my sincere gratitude to Dr Maria Polukarov for her guidance and support which provided me the freedom to take this research in the direction of my interest.\\
\\
I would also like to thank my family and friends for their encouragement and support. To those who quietly listened to my software complaints. To those who worked throughout the nights with me. To those who helped me write what I couldn't say. I cannot thank you enough.
}

\declaration{
I, Stefan Collier, declare that this dissertation and the work presented in it are my own and has been generated by me as the result of my own original research.\\
I confirm that:\\
1. This work was done wholly or mainly while in candidature for a degree at this University;\\
2. Where any part of this dissertation has previously been submitted for any other qualification at this University or any other institution, this has been clearly stated;\\
3. Where I have consulted the published work of others, this is always clearly attributed;\\
4. Where I have quoted from the work of others, the source is always given. With the exception of such quotations, this dissertation is entirely my own work;\\
5. I have acknowledged all main sources of help;\\
6. Where the thesis is based on work done by myself jointly with others, I have made clear exactly what was done by others and what I have contributed myself;\\
7. Either none of this work has been published before submission, or parts of this work have been published by :\\
\\
Stefan Collier\\
April 2016
}
\tableofcontents
\listoffigures
\listoftables

\mainmatter
%% ----------------------------------------------------------------
%\include{Introduction}
%\include{Conclusions}
\include{chapters/1Project/main}
\include{chapters/2Lit/main}
\include{chapters/3Design/HighLevel}
\include{chapters/3Design/InDepth}
\include{chapters/4Impl/main}

\include{chapters/5Experiments/1/main}
\include{chapters/5Experiments/2/main}
\include{chapters/5Experiments/3/main}
\include{chapters/5Experiments/4/main}

\include{chapters/6Conclusion/main}

\appendix
\include{appendix/AppendixB}
\include{appendix/D/main}
\include{appendix/AppendixC}

\backmatter
\bibliographystyle{ecs}
\bibliography{ECS}
\end{document}
%% ----------------------------------------------------------------

 %% ----------------------------------------------------------------
%% Progress.tex
%% ---------------------------------------------------------------- 
\documentclass{ecsprogress}    % Use the progress Style
\graphicspath{{../figs/}}   % Location of your graphics files
    \usepackage{natbib}            % Use Natbib style for the refs.
\hypersetup{colorlinks=true}   % Set to false for black/white printing
\input{Definitions}            % Include your abbreviations



\usepackage{enumitem}% http://ctan.org/pkg/enumitem
\usepackage{multirow}
\usepackage{float}
\usepackage{amsmath}
\usepackage{multicol}
\usepackage{amssymb}
\usepackage[normalem]{ulem}
\useunder{\uline}{\ul}{}
\usepackage{wrapfig}


\usepackage[table,xcdraw]{xcolor}


%% ----------------------------------------------------------------
\begin{document}
\frontmatter
\title      {Heterogeneous Agent-based Model for Supermarket Competition}
\authors    {\texorpdfstring
             {\href{mailto:sc22g13@ecs.soton.ac.uk}{Stefan J. Collier}}
             {Stefan J. Collier}
            }
\addresses  {\groupname\\\deptname\\\univname}
\date       {\today}
\subject    {}
\keywords   {}
\supervisor {Dr. Maria Polukarov}
\examiner   {Professor Sheng Chen}

\maketitle
\begin{abstract}
This project aim was to model and analyse the effects of competitive pricing behaviors of grocery retailers on the British market. 

This was achieved by creating a multi-agent model, containing retailer and consumer agents. The heterogeneous crowd of retailers employs either a uniform pricing strategy or a ‘local price flexing’ strategy. The actions of these retailers are chosen by predicting the profit of each action, using a perceptron. Following on from the consideration of different economic models, a discrete model was developed so that software agents have a discrete environment to operate within. Within the model, it has been observed how supermarkets with differing behaviors affect a heterogeneous crowd of consumer agents. The model was implemented in Java with Python used to evaluate the results. 

The simulation displays good acceptance with real grocery market behavior, i.e. captures the performance of British retailers thus can be used to determine the impact of changes in their behavior on their competitors and consumers.Furthermore it can be used to provide insight into sustainability of volatile pricing strategies, providing a useful insight in volatility of British supermarket retail industry. 
\end{abstract}
\acknowledgements{
I would like to express my sincere gratitude to Dr Maria Polukarov for her guidance and support which provided me the freedom to take this research in the direction of my interest.\\
\\
I would also like to thank my family and friends for their encouragement and support. To those who quietly listened to my software complaints. To those who worked throughout the nights with me. To those who helped me write what I couldn't say. I cannot thank you enough.
}

\declaration{
I, Stefan Collier, declare that this dissertation and the work presented in it are my own and has been generated by me as the result of my own original research.\\
I confirm that:\\
1. This work was done wholly or mainly while in candidature for a degree at this University;\\
2. Where any part of this dissertation has previously been submitted for any other qualification at this University or any other institution, this has been clearly stated;\\
3. Where I have consulted the published work of others, this is always clearly attributed;\\
4. Where I have quoted from the work of others, the source is always given. With the exception of such quotations, this dissertation is entirely my own work;\\
5. I have acknowledged all main sources of help;\\
6. Where the thesis is based on work done by myself jointly with others, I have made clear exactly what was done by others and what I have contributed myself;\\
7. Either none of this work has been published before submission, or parts of this work have been published by :\\
\\
Stefan Collier\\
April 2016
}
\tableofcontents
\listoffigures
\listoftables

\mainmatter
%% ----------------------------------------------------------------
%\include{Introduction}
%\include{Conclusions}
\include{chapters/1Project/main}
\include{chapters/2Lit/main}
\include{chapters/3Design/HighLevel}
\include{chapters/3Design/InDepth}
\include{chapters/4Impl/main}

\include{chapters/5Experiments/1/main}
\include{chapters/5Experiments/2/main}
\include{chapters/5Experiments/3/main}
\include{chapters/5Experiments/4/main}

\include{chapters/6Conclusion/main}

\appendix
\include{appendix/AppendixB}
\include{appendix/D/main}
\include{appendix/AppendixC}

\backmatter
\bibliographystyle{ecs}
\bibliography{ECS}
\end{document}
%% ----------------------------------------------------------------


 %% ----------------------------------------------------------------
%% Progress.tex
%% ---------------------------------------------------------------- 
\documentclass{ecsprogress}    % Use the progress Style
\graphicspath{{../figs/}}   % Location of your graphics files
    \usepackage{natbib}            % Use Natbib style for the refs.
\hypersetup{colorlinks=true}   % Set to false for black/white printing
\input{Definitions}            % Include your abbreviations



\usepackage{enumitem}% http://ctan.org/pkg/enumitem
\usepackage{multirow}
\usepackage{float}
\usepackage{amsmath}
\usepackage{multicol}
\usepackage{amssymb}
\usepackage[normalem]{ulem}
\useunder{\uline}{\ul}{}
\usepackage{wrapfig}


\usepackage[table,xcdraw]{xcolor}


%% ----------------------------------------------------------------
\begin{document}
\frontmatter
\title      {Heterogeneous Agent-based Model for Supermarket Competition}
\authors    {\texorpdfstring
             {\href{mailto:sc22g13@ecs.soton.ac.uk}{Stefan J. Collier}}
             {Stefan J. Collier}
            }
\addresses  {\groupname\\\deptname\\\univname}
\date       {\today}
\subject    {}
\keywords   {}
\supervisor {Dr. Maria Polukarov}
\examiner   {Professor Sheng Chen}

\maketitle
\begin{abstract}
This project aim was to model and analyse the effects of competitive pricing behaviors of grocery retailers on the British market. 

This was achieved by creating a multi-agent model, containing retailer and consumer agents. The heterogeneous crowd of retailers employs either a uniform pricing strategy or a ‘local price flexing’ strategy. The actions of these retailers are chosen by predicting the profit of each action, using a perceptron. Following on from the consideration of different economic models, a discrete model was developed so that software agents have a discrete environment to operate within. Within the model, it has been observed how supermarkets with differing behaviors affect a heterogeneous crowd of consumer agents. The model was implemented in Java with Python used to evaluate the results. 

The simulation displays good acceptance with real grocery market behavior, i.e. captures the performance of British retailers thus can be used to determine the impact of changes in their behavior on their competitors and consumers.Furthermore it can be used to provide insight into sustainability of volatile pricing strategies, providing a useful insight in volatility of British supermarket retail industry. 
\end{abstract}
\acknowledgements{
I would like to express my sincere gratitude to Dr Maria Polukarov for her guidance and support which provided me the freedom to take this research in the direction of my interest.\\
\\
I would also like to thank my family and friends for their encouragement and support. To those who quietly listened to my software complaints. To those who worked throughout the nights with me. To those who helped me write what I couldn't say. I cannot thank you enough.
}

\declaration{
I, Stefan Collier, declare that this dissertation and the work presented in it are my own and has been generated by me as the result of my own original research.\\
I confirm that:\\
1. This work was done wholly or mainly while in candidature for a degree at this University;\\
2. Where any part of this dissertation has previously been submitted for any other qualification at this University or any other institution, this has been clearly stated;\\
3. Where I have consulted the published work of others, this is always clearly attributed;\\
4. Where I have quoted from the work of others, the source is always given. With the exception of such quotations, this dissertation is entirely my own work;\\
5. I have acknowledged all main sources of help;\\
6. Where the thesis is based on work done by myself jointly with others, I have made clear exactly what was done by others and what I have contributed myself;\\
7. Either none of this work has been published before submission, or parts of this work have been published by :\\
\\
Stefan Collier\\
April 2016
}
\tableofcontents
\listoffigures
\listoftables

\mainmatter
%% ----------------------------------------------------------------
%\include{Introduction}
%\include{Conclusions}
\include{chapters/1Project/main}
\include{chapters/2Lit/main}
\include{chapters/3Design/HighLevel}
\include{chapters/3Design/InDepth}
\include{chapters/4Impl/main}

\include{chapters/5Experiments/1/main}
\include{chapters/5Experiments/2/main}
\include{chapters/5Experiments/3/main}
\include{chapters/5Experiments/4/main}

\include{chapters/6Conclusion/main}

\appendix
\include{appendix/AppendixB}
\include{appendix/D/main}
\include{appendix/AppendixC}

\backmatter
\bibliographystyle{ecs}
\bibliography{ECS}
\end{document}
%% ----------------------------------------------------------------


\appendix
\include{appendix/AppendixB}
 %% ----------------------------------------------------------------
%% Progress.tex
%% ---------------------------------------------------------------- 
\documentclass{ecsprogress}    % Use the progress Style
\graphicspath{{../figs/}}   % Location of your graphics files
    \usepackage{natbib}            % Use Natbib style for the refs.
\hypersetup{colorlinks=true}   % Set to false for black/white printing
\input{Definitions}            % Include your abbreviations



\usepackage{enumitem}% http://ctan.org/pkg/enumitem
\usepackage{multirow}
\usepackage{float}
\usepackage{amsmath}
\usepackage{multicol}
\usepackage{amssymb}
\usepackage[normalem]{ulem}
\useunder{\uline}{\ul}{}
\usepackage{wrapfig}


\usepackage[table,xcdraw]{xcolor}


%% ----------------------------------------------------------------
\begin{document}
\frontmatter
\title      {Heterogeneous Agent-based Model for Supermarket Competition}
\authors    {\texorpdfstring
             {\href{mailto:sc22g13@ecs.soton.ac.uk}{Stefan J. Collier}}
             {Stefan J. Collier}
            }
\addresses  {\groupname\\\deptname\\\univname}
\date       {\today}
\subject    {}
\keywords   {}
\supervisor {Dr. Maria Polukarov}
\examiner   {Professor Sheng Chen}

\maketitle
\begin{abstract}
This project aim was to model and analyse the effects of competitive pricing behaviors of grocery retailers on the British market. 

This was achieved by creating a multi-agent model, containing retailer and consumer agents. The heterogeneous crowd of retailers employs either a uniform pricing strategy or a ‘local price flexing’ strategy. The actions of these retailers are chosen by predicting the profit of each action, using a perceptron. Following on from the consideration of different economic models, a discrete model was developed so that software agents have a discrete environment to operate within. Within the model, it has been observed how supermarkets with differing behaviors affect a heterogeneous crowd of consumer agents. The model was implemented in Java with Python used to evaluate the results. 

The simulation displays good acceptance with real grocery market behavior, i.e. captures the performance of British retailers thus can be used to determine the impact of changes in their behavior on their competitors and consumers.Furthermore it can be used to provide insight into sustainability of volatile pricing strategies, providing a useful insight in volatility of British supermarket retail industry. 
\end{abstract}
\acknowledgements{
I would like to express my sincere gratitude to Dr Maria Polukarov for her guidance and support which provided me the freedom to take this research in the direction of my interest.\\
\\
I would also like to thank my family and friends for their encouragement and support. To those who quietly listened to my software complaints. To those who worked throughout the nights with me. To those who helped me write what I couldn't say. I cannot thank you enough.
}

\declaration{
I, Stefan Collier, declare that this dissertation and the work presented in it are my own and has been generated by me as the result of my own original research.\\
I confirm that:\\
1. This work was done wholly or mainly while in candidature for a degree at this University;\\
2. Where any part of this dissertation has previously been submitted for any other qualification at this University or any other institution, this has been clearly stated;\\
3. Where I have consulted the published work of others, this is always clearly attributed;\\
4. Where I have quoted from the work of others, the source is always given. With the exception of such quotations, this dissertation is entirely my own work;\\
5. I have acknowledged all main sources of help;\\
6. Where the thesis is based on work done by myself jointly with others, I have made clear exactly what was done by others and what I have contributed myself;\\
7. Either none of this work has been published before submission, or parts of this work have been published by :\\
\\
Stefan Collier\\
April 2016
}
\tableofcontents
\listoffigures
\listoftables

\mainmatter
%% ----------------------------------------------------------------
%\include{Introduction}
%\include{Conclusions}
\include{chapters/1Project/main}
\include{chapters/2Lit/main}
\include{chapters/3Design/HighLevel}
\include{chapters/3Design/InDepth}
\include{chapters/4Impl/main}

\include{chapters/5Experiments/1/main}
\include{chapters/5Experiments/2/main}
\include{chapters/5Experiments/3/main}
\include{chapters/5Experiments/4/main}

\include{chapters/6Conclusion/main}

\appendix
\include{appendix/AppendixB}
\include{appendix/D/main}
\include{appendix/AppendixC}

\backmatter
\bibliographystyle{ecs}
\bibliography{ECS}
\end{document}
%% ----------------------------------------------------------------

\include{appendix/AppendixC}

\backmatter
\bibliographystyle{ecs}
\bibliography{ECS}
\end{document}
%% ----------------------------------------------------------------

\include{appendix/AppendixC}

\backmatter
\bibliographystyle{ecs}
\bibliography{ECS}
\end{document}
%% ----------------------------------------------------------------


\appendix
\include{appendix/AppendixB}
 %% ----------------------------------------------------------------
%% Progress.tex
%% ---------------------------------------------------------------- 
\documentclass{ecsprogress}    % Use the progress Style
\graphicspath{{../figs/}}   % Location of your graphics files
    \usepackage{natbib}            % Use Natbib style for the refs.
\hypersetup{colorlinks=true}   % Set to false for black/white printing
\input{Definitions}            % Include your abbreviations



\usepackage{enumitem}% http://ctan.org/pkg/enumitem
\usepackage{multirow}
\usepackage{float}
\usepackage{amsmath}
\usepackage{multicol}
\usepackage{amssymb}
\usepackage[normalem]{ulem}
\useunder{\uline}{\ul}{}
\usepackage{wrapfig}


\usepackage[table,xcdraw]{xcolor}


%% ----------------------------------------------------------------
\begin{document}
\frontmatter
\title      {Heterogeneous Agent-based Model for Supermarket Competition}
\authors    {\texorpdfstring
             {\href{mailto:sc22g13@ecs.soton.ac.uk}{Stefan J. Collier}}
             {Stefan J. Collier}
            }
\addresses  {\groupname\\\deptname\\\univname}
\date       {\today}
\subject    {}
\keywords   {}
\supervisor {Dr. Maria Polukarov}
\examiner   {Professor Sheng Chen}

\maketitle
\begin{abstract}
This project aim was to model and analyse the effects of competitive pricing behaviors of grocery retailers on the British market. 

This was achieved by creating a multi-agent model, containing retailer and consumer agents. The heterogeneous crowd of retailers employs either a uniform pricing strategy or a ‘local price flexing’ strategy. The actions of these retailers are chosen by predicting the profit of each action, using a perceptron. Following on from the consideration of different economic models, a discrete model was developed so that software agents have a discrete environment to operate within. Within the model, it has been observed how supermarkets with differing behaviors affect a heterogeneous crowd of consumer agents. The model was implemented in Java with Python used to evaluate the results. 

The simulation displays good acceptance with real grocery market behavior, i.e. captures the performance of British retailers thus can be used to determine the impact of changes in their behavior on their competitors and consumers.Furthermore it can be used to provide insight into sustainability of volatile pricing strategies, providing a useful insight in volatility of British supermarket retail industry. 
\end{abstract}
\acknowledgements{
I would like to express my sincere gratitude to Dr Maria Polukarov for her guidance and support which provided me the freedom to take this research in the direction of my interest.\\
\\
I would also like to thank my family and friends for their encouragement and support. To those who quietly listened to my software complaints. To those who worked throughout the nights with me. To those who helped me write what I couldn't say. I cannot thank you enough.
}

\declaration{
I, Stefan Collier, declare that this dissertation and the work presented in it are my own and has been generated by me as the result of my own original research.\\
I confirm that:\\
1. This work was done wholly or mainly while in candidature for a degree at this University;\\
2. Where any part of this dissertation has previously been submitted for any other qualification at this University or any other institution, this has been clearly stated;\\
3. Where I have consulted the published work of others, this is always clearly attributed;\\
4. Where I have quoted from the work of others, the source is always given. With the exception of such quotations, this dissertation is entirely my own work;\\
5. I have acknowledged all main sources of help;\\
6. Where the thesis is based on work done by myself jointly with others, I have made clear exactly what was done by others and what I have contributed myself;\\
7. Either none of this work has been published before submission, or parts of this work have been published by :\\
\\
Stefan Collier\\
April 2016
}
\tableofcontents
\listoffigures
\listoftables

\mainmatter
%% ----------------------------------------------------------------
%\include{Introduction}
%\include{Conclusions}
 %% ----------------------------------------------------------------
%% Progress.tex
%% ---------------------------------------------------------------- 
\documentclass{ecsprogress}    % Use the progress Style
\graphicspath{{../figs/}}   % Location of your graphics files
    \usepackage{natbib}            % Use Natbib style for the refs.
\hypersetup{colorlinks=true}   % Set to false for black/white printing
\input{Definitions}            % Include your abbreviations



\usepackage{enumitem}% http://ctan.org/pkg/enumitem
\usepackage{multirow}
\usepackage{float}
\usepackage{amsmath}
\usepackage{multicol}
\usepackage{amssymb}
\usepackage[normalem]{ulem}
\useunder{\uline}{\ul}{}
\usepackage{wrapfig}


\usepackage[table,xcdraw]{xcolor}


%% ----------------------------------------------------------------
\begin{document}
\frontmatter
\title      {Heterogeneous Agent-based Model for Supermarket Competition}
\authors    {\texorpdfstring
             {\href{mailto:sc22g13@ecs.soton.ac.uk}{Stefan J. Collier}}
             {Stefan J. Collier}
            }
\addresses  {\groupname\\\deptname\\\univname}
\date       {\today}
\subject    {}
\keywords   {}
\supervisor {Dr. Maria Polukarov}
\examiner   {Professor Sheng Chen}

\maketitle
\begin{abstract}
This project aim was to model and analyse the effects of competitive pricing behaviors of grocery retailers on the British market. 

This was achieved by creating a multi-agent model, containing retailer and consumer agents. The heterogeneous crowd of retailers employs either a uniform pricing strategy or a ‘local price flexing’ strategy. The actions of these retailers are chosen by predicting the profit of each action, using a perceptron. Following on from the consideration of different economic models, a discrete model was developed so that software agents have a discrete environment to operate within. Within the model, it has been observed how supermarkets with differing behaviors affect a heterogeneous crowd of consumer agents. The model was implemented in Java with Python used to evaluate the results. 

The simulation displays good acceptance with real grocery market behavior, i.e. captures the performance of British retailers thus can be used to determine the impact of changes in their behavior on their competitors and consumers.Furthermore it can be used to provide insight into sustainability of volatile pricing strategies, providing a useful insight in volatility of British supermarket retail industry. 
\end{abstract}
\acknowledgements{
I would like to express my sincere gratitude to Dr Maria Polukarov for her guidance and support which provided me the freedom to take this research in the direction of my interest.\\
\\
I would also like to thank my family and friends for their encouragement and support. To those who quietly listened to my software complaints. To those who worked throughout the nights with me. To those who helped me write what I couldn't say. I cannot thank you enough.
}

\declaration{
I, Stefan Collier, declare that this dissertation and the work presented in it are my own and has been generated by me as the result of my own original research.\\
I confirm that:\\
1. This work was done wholly or mainly while in candidature for a degree at this University;\\
2. Where any part of this dissertation has previously been submitted for any other qualification at this University or any other institution, this has been clearly stated;\\
3. Where I have consulted the published work of others, this is always clearly attributed;\\
4. Where I have quoted from the work of others, the source is always given. With the exception of such quotations, this dissertation is entirely my own work;\\
5. I have acknowledged all main sources of help;\\
6. Where the thesis is based on work done by myself jointly with others, I have made clear exactly what was done by others and what I have contributed myself;\\
7. Either none of this work has been published before submission, or parts of this work have been published by :\\
\\
Stefan Collier\\
April 2016
}
\tableofcontents
\listoffigures
\listoftables

\mainmatter
%% ----------------------------------------------------------------
%\include{Introduction}
%\include{Conclusions}
 %% ----------------------------------------------------------------
%% Progress.tex
%% ---------------------------------------------------------------- 
\documentclass{ecsprogress}    % Use the progress Style
\graphicspath{{../figs/}}   % Location of your graphics files
    \usepackage{natbib}            % Use Natbib style for the refs.
\hypersetup{colorlinks=true}   % Set to false for black/white printing
\input{Definitions}            % Include your abbreviations



\usepackage{enumitem}% http://ctan.org/pkg/enumitem
\usepackage{multirow}
\usepackage{float}
\usepackage{amsmath}
\usepackage{multicol}
\usepackage{amssymb}
\usepackage[normalem]{ulem}
\useunder{\uline}{\ul}{}
\usepackage{wrapfig}


\usepackage[table,xcdraw]{xcolor}


%% ----------------------------------------------------------------
\begin{document}
\frontmatter
\title      {Heterogeneous Agent-based Model for Supermarket Competition}
\authors    {\texorpdfstring
             {\href{mailto:sc22g13@ecs.soton.ac.uk}{Stefan J. Collier}}
             {Stefan J. Collier}
            }
\addresses  {\groupname\\\deptname\\\univname}
\date       {\today}
\subject    {}
\keywords   {}
\supervisor {Dr. Maria Polukarov}
\examiner   {Professor Sheng Chen}

\maketitle
\begin{abstract}
This project aim was to model and analyse the effects of competitive pricing behaviors of grocery retailers on the British market. 

This was achieved by creating a multi-agent model, containing retailer and consumer agents. The heterogeneous crowd of retailers employs either a uniform pricing strategy or a ‘local price flexing’ strategy. The actions of these retailers are chosen by predicting the profit of each action, using a perceptron. Following on from the consideration of different economic models, a discrete model was developed so that software agents have a discrete environment to operate within. Within the model, it has been observed how supermarkets with differing behaviors affect a heterogeneous crowd of consumer agents. The model was implemented in Java with Python used to evaluate the results. 

The simulation displays good acceptance with real grocery market behavior, i.e. captures the performance of British retailers thus can be used to determine the impact of changes in their behavior on their competitors and consumers.Furthermore it can be used to provide insight into sustainability of volatile pricing strategies, providing a useful insight in volatility of British supermarket retail industry. 
\end{abstract}
\acknowledgements{
I would like to express my sincere gratitude to Dr Maria Polukarov for her guidance and support which provided me the freedom to take this research in the direction of my interest.\\
\\
I would also like to thank my family and friends for their encouragement and support. To those who quietly listened to my software complaints. To those who worked throughout the nights with me. To those who helped me write what I couldn't say. I cannot thank you enough.
}

\declaration{
I, Stefan Collier, declare that this dissertation and the work presented in it are my own and has been generated by me as the result of my own original research.\\
I confirm that:\\
1. This work was done wholly or mainly while in candidature for a degree at this University;\\
2. Where any part of this dissertation has previously been submitted for any other qualification at this University or any other institution, this has been clearly stated;\\
3. Where I have consulted the published work of others, this is always clearly attributed;\\
4. Where I have quoted from the work of others, the source is always given. With the exception of such quotations, this dissertation is entirely my own work;\\
5. I have acknowledged all main sources of help;\\
6. Where the thesis is based on work done by myself jointly with others, I have made clear exactly what was done by others and what I have contributed myself;\\
7. Either none of this work has been published before submission, or parts of this work have been published by :\\
\\
Stefan Collier\\
April 2016
}
\tableofcontents
\listoffigures
\listoftables

\mainmatter
%% ----------------------------------------------------------------
%\include{Introduction}
%\include{Conclusions}
\include{chapters/1Project/main}
\include{chapters/2Lit/main}
\include{chapters/3Design/HighLevel}
\include{chapters/3Design/InDepth}
\include{chapters/4Impl/main}

\include{chapters/5Experiments/1/main}
\include{chapters/5Experiments/2/main}
\include{chapters/5Experiments/3/main}
\include{chapters/5Experiments/4/main}

\include{chapters/6Conclusion/main}

\appendix
\include{appendix/AppendixB}
\include{appendix/D/main}
\include{appendix/AppendixC}

\backmatter
\bibliographystyle{ecs}
\bibliography{ECS}
\end{document}
%% ----------------------------------------------------------------

 %% ----------------------------------------------------------------
%% Progress.tex
%% ---------------------------------------------------------------- 
\documentclass{ecsprogress}    % Use the progress Style
\graphicspath{{../figs/}}   % Location of your graphics files
    \usepackage{natbib}            % Use Natbib style for the refs.
\hypersetup{colorlinks=true}   % Set to false for black/white printing
\input{Definitions}            % Include your abbreviations



\usepackage{enumitem}% http://ctan.org/pkg/enumitem
\usepackage{multirow}
\usepackage{float}
\usepackage{amsmath}
\usepackage{multicol}
\usepackage{amssymb}
\usepackage[normalem]{ulem}
\useunder{\uline}{\ul}{}
\usepackage{wrapfig}


\usepackage[table,xcdraw]{xcolor}


%% ----------------------------------------------------------------
\begin{document}
\frontmatter
\title      {Heterogeneous Agent-based Model for Supermarket Competition}
\authors    {\texorpdfstring
             {\href{mailto:sc22g13@ecs.soton.ac.uk}{Stefan J. Collier}}
             {Stefan J. Collier}
            }
\addresses  {\groupname\\\deptname\\\univname}
\date       {\today}
\subject    {}
\keywords   {}
\supervisor {Dr. Maria Polukarov}
\examiner   {Professor Sheng Chen}

\maketitle
\begin{abstract}
This project aim was to model and analyse the effects of competitive pricing behaviors of grocery retailers on the British market. 

This was achieved by creating a multi-agent model, containing retailer and consumer agents. The heterogeneous crowd of retailers employs either a uniform pricing strategy or a ‘local price flexing’ strategy. The actions of these retailers are chosen by predicting the profit of each action, using a perceptron. Following on from the consideration of different economic models, a discrete model was developed so that software agents have a discrete environment to operate within. Within the model, it has been observed how supermarkets with differing behaviors affect a heterogeneous crowd of consumer agents. The model was implemented in Java with Python used to evaluate the results. 

The simulation displays good acceptance with real grocery market behavior, i.e. captures the performance of British retailers thus can be used to determine the impact of changes in their behavior on their competitors and consumers.Furthermore it can be used to provide insight into sustainability of volatile pricing strategies, providing a useful insight in volatility of British supermarket retail industry. 
\end{abstract}
\acknowledgements{
I would like to express my sincere gratitude to Dr Maria Polukarov for her guidance and support which provided me the freedom to take this research in the direction of my interest.\\
\\
I would also like to thank my family and friends for their encouragement and support. To those who quietly listened to my software complaints. To those who worked throughout the nights with me. To those who helped me write what I couldn't say. I cannot thank you enough.
}

\declaration{
I, Stefan Collier, declare that this dissertation and the work presented in it are my own and has been generated by me as the result of my own original research.\\
I confirm that:\\
1. This work was done wholly or mainly while in candidature for a degree at this University;\\
2. Where any part of this dissertation has previously been submitted for any other qualification at this University or any other institution, this has been clearly stated;\\
3. Where I have consulted the published work of others, this is always clearly attributed;\\
4. Where I have quoted from the work of others, the source is always given. With the exception of such quotations, this dissertation is entirely my own work;\\
5. I have acknowledged all main sources of help;\\
6. Where the thesis is based on work done by myself jointly with others, I have made clear exactly what was done by others and what I have contributed myself;\\
7. Either none of this work has been published before submission, or parts of this work have been published by :\\
\\
Stefan Collier\\
April 2016
}
\tableofcontents
\listoffigures
\listoftables

\mainmatter
%% ----------------------------------------------------------------
%\include{Introduction}
%\include{Conclusions}
\include{chapters/1Project/main}
\include{chapters/2Lit/main}
\include{chapters/3Design/HighLevel}
\include{chapters/3Design/InDepth}
\include{chapters/4Impl/main}

\include{chapters/5Experiments/1/main}
\include{chapters/5Experiments/2/main}
\include{chapters/5Experiments/3/main}
\include{chapters/5Experiments/4/main}

\include{chapters/6Conclusion/main}

\appendix
\include{appendix/AppendixB}
\include{appendix/D/main}
\include{appendix/AppendixC}

\backmatter
\bibliographystyle{ecs}
\bibliography{ECS}
\end{document}
%% ----------------------------------------------------------------

\include{chapters/3Design/HighLevel}
\include{chapters/3Design/InDepth}
 %% ----------------------------------------------------------------
%% Progress.tex
%% ---------------------------------------------------------------- 
\documentclass{ecsprogress}    % Use the progress Style
\graphicspath{{../figs/}}   % Location of your graphics files
    \usepackage{natbib}            % Use Natbib style for the refs.
\hypersetup{colorlinks=true}   % Set to false for black/white printing
\input{Definitions}            % Include your abbreviations



\usepackage{enumitem}% http://ctan.org/pkg/enumitem
\usepackage{multirow}
\usepackage{float}
\usepackage{amsmath}
\usepackage{multicol}
\usepackage{amssymb}
\usepackage[normalem]{ulem}
\useunder{\uline}{\ul}{}
\usepackage{wrapfig}


\usepackage[table,xcdraw]{xcolor}


%% ----------------------------------------------------------------
\begin{document}
\frontmatter
\title      {Heterogeneous Agent-based Model for Supermarket Competition}
\authors    {\texorpdfstring
             {\href{mailto:sc22g13@ecs.soton.ac.uk}{Stefan J. Collier}}
             {Stefan J. Collier}
            }
\addresses  {\groupname\\\deptname\\\univname}
\date       {\today}
\subject    {}
\keywords   {}
\supervisor {Dr. Maria Polukarov}
\examiner   {Professor Sheng Chen}

\maketitle
\begin{abstract}
This project aim was to model and analyse the effects of competitive pricing behaviors of grocery retailers on the British market. 

This was achieved by creating a multi-agent model, containing retailer and consumer agents. The heterogeneous crowd of retailers employs either a uniform pricing strategy or a ‘local price flexing’ strategy. The actions of these retailers are chosen by predicting the profit of each action, using a perceptron. Following on from the consideration of different economic models, a discrete model was developed so that software agents have a discrete environment to operate within. Within the model, it has been observed how supermarkets with differing behaviors affect a heterogeneous crowd of consumer agents. The model was implemented in Java with Python used to evaluate the results. 

The simulation displays good acceptance with real grocery market behavior, i.e. captures the performance of British retailers thus can be used to determine the impact of changes in their behavior on their competitors and consumers.Furthermore it can be used to provide insight into sustainability of volatile pricing strategies, providing a useful insight in volatility of British supermarket retail industry. 
\end{abstract}
\acknowledgements{
I would like to express my sincere gratitude to Dr Maria Polukarov for her guidance and support which provided me the freedom to take this research in the direction of my interest.\\
\\
I would also like to thank my family and friends for their encouragement and support. To those who quietly listened to my software complaints. To those who worked throughout the nights with me. To those who helped me write what I couldn't say. I cannot thank you enough.
}

\declaration{
I, Stefan Collier, declare that this dissertation and the work presented in it are my own and has been generated by me as the result of my own original research.\\
I confirm that:\\
1. This work was done wholly or mainly while in candidature for a degree at this University;\\
2. Where any part of this dissertation has previously been submitted for any other qualification at this University or any other institution, this has been clearly stated;\\
3. Where I have consulted the published work of others, this is always clearly attributed;\\
4. Where I have quoted from the work of others, the source is always given. With the exception of such quotations, this dissertation is entirely my own work;\\
5. I have acknowledged all main sources of help;\\
6. Where the thesis is based on work done by myself jointly with others, I have made clear exactly what was done by others and what I have contributed myself;\\
7. Either none of this work has been published before submission, or parts of this work have been published by :\\
\\
Stefan Collier\\
April 2016
}
\tableofcontents
\listoffigures
\listoftables

\mainmatter
%% ----------------------------------------------------------------
%\include{Introduction}
%\include{Conclusions}
\include{chapters/1Project/main}
\include{chapters/2Lit/main}
\include{chapters/3Design/HighLevel}
\include{chapters/3Design/InDepth}
\include{chapters/4Impl/main}

\include{chapters/5Experiments/1/main}
\include{chapters/5Experiments/2/main}
\include{chapters/5Experiments/3/main}
\include{chapters/5Experiments/4/main}

\include{chapters/6Conclusion/main}

\appendix
\include{appendix/AppendixB}
\include{appendix/D/main}
\include{appendix/AppendixC}

\backmatter
\bibliographystyle{ecs}
\bibliography{ECS}
\end{document}
%% ----------------------------------------------------------------


 %% ----------------------------------------------------------------
%% Progress.tex
%% ---------------------------------------------------------------- 
\documentclass{ecsprogress}    % Use the progress Style
\graphicspath{{../figs/}}   % Location of your graphics files
    \usepackage{natbib}            % Use Natbib style for the refs.
\hypersetup{colorlinks=true}   % Set to false for black/white printing
\input{Definitions}            % Include your abbreviations



\usepackage{enumitem}% http://ctan.org/pkg/enumitem
\usepackage{multirow}
\usepackage{float}
\usepackage{amsmath}
\usepackage{multicol}
\usepackage{amssymb}
\usepackage[normalem]{ulem}
\useunder{\uline}{\ul}{}
\usepackage{wrapfig}


\usepackage[table,xcdraw]{xcolor}


%% ----------------------------------------------------------------
\begin{document}
\frontmatter
\title      {Heterogeneous Agent-based Model for Supermarket Competition}
\authors    {\texorpdfstring
             {\href{mailto:sc22g13@ecs.soton.ac.uk}{Stefan J. Collier}}
             {Stefan J. Collier}
            }
\addresses  {\groupname\\\deptname\\\univname}
\date       {\today}
\subject    {}
\keywords   {}
\supervisor {Dr. Maria Polukarov}
\examiner   {Professor Sheng Chen}

\maketitle
\begin{abstract}
This project aim was to model and analyse the effects of competitive pricing behaviors of grocery retailers on the British market. 

This was achieved by creating a multi-agent model, containing retailer and consumer agents. The heterogeneous crowd of retailers employs either a uniform pricing strategy or a ‘local price flexing’ strategy. The actions of these retailers are chosen by predicting the profit of each action, using a perceptron. Following on from the consideration of different economic models, a discrete model was developed so that software agents have a discrete environment to operate within. Within the model, it has been observed how supermarkets with differing behaviors affect a heterogeneous crowd of consumer agents. The model was implemented in Java with Python used to evaluate the results. 

The simulation displays good acceptance with real grocery market behavior, i.e. captures the performance of British retailers thus can be used to determine the impact of changes in their behavior on their competitors and consumers.Furthermore it can be used to provide insight into sustainability of volatile pricing strategies, providing a useful insight in volatility of British supermarket retail industry. 
\end{abstract}
\acknowledgements{
I would like to express my sincere gratitude to Dr Maria Polukarov for her guidance and support which provided me the freedom to take this research in the direction of my interest.\\
\\
I would also like to thank my family and friends for their encouragement and support. To those who quietly listened to my software complaints. To those who worked throughout the nights with me. To those who helped me write what I couldn't say. I cannot thank you enough.
}

\declaration{
I, Stefan Collier, declare that this dissertation and the work presented in it are my own and has been generated by me as the result of my own original research.\\
I confirm that:\\
1. This work was done wholly or mainly while in candidature for a degree at this University;\\
2. Where any part of this dissertation has previously been submitted for any other qualification at this University or any other institution, this has been clearly stated;\\
3. Where I have consulted the published work of others, this is always clearly attributed;\\
4. Where I have quoted from the work of others, the source is always given. With the exception of such quotations, this dissertation is entirely my own work;\\
5. I have acknowledged all main sources of help;\\
6. Where the thesis is based on work done by myself jointly with others, I have made clear exactly what was done by others and what I have contributed myself;\\
7. Either none of this work has been published before submission, or parts of this work have been published by :\\
\\
Stefan Collier\\
April 2016
}
\tableofcontents
\listoffigures
\listoftables

\mainmatter
%% ----------------------------------------------------------------
%\include{Introduction}
%\include{Conclusions}
\include{chapters/1Project/main}
\include{chapters/2Lit/main}
\include{chapters/3Design/HighLevel}
\include{chapters/3Design/InDepth}
\include{chapters/4Impl/main}

\include{chapters/5Experiments/1/main}
\include{chapters/5Experiments/2/main}
\include{chapters/5Experiments/3/main}
\include{chapters/5Experiments/4/main}

\include{chapters/6Conclusion/main}

\appendix
\include{appendix/AppendixB}
\include{appendix/D/main}
\include{appendix/AppendixC}

\backmatter
\bibliographystyle{ecs}
\bibliography{ECS}
\end{document}
%% ----------------------------------------------------------------

 %% ----------------------------------------------------------------
%% Progress.tex
%% ---------------------------------------------------------------- 
\documentclass{ecsprogress}    % Use the progress Style
\graphicspath{{../figs/}}   % Location of your graphics files
    \usepackage{natbib}            % Use Natbib style for the refs.
\hypersetup{colorlinks=true}   % Set to false for black/white printing
\input{Definitions}            % Include your abbreviations



\usepackage{enumitem}% http://ctan.org/pkg/enumitem
\usepackage{multirow}
\usepackage{float}
\usepackage{amsmath}
\usepackage{multicol}
\usepackage{amssymb}
\usepackage[normalem]{ulem}
\useunder{\uline}{\ul}{}
\usepackage{wrapfig}


\usepackage[table,xcdraw]{xcolor}


%% ----------------------------------------------------------------
\begin{document}
\frontmatter
\title      {Heterogeneous Agent-based Model for Supermarket Competition}
\authors    {\texorpdfstring
             {\href{mailto:sc22g13@ecs.soton.ac.uk}{Stefan J. Collier}}
             {Stefan J. Collier}
            }
\addresses  {\groupname\\\deptname\\\univname}
\date       {\today}
\subject    {}
\keywords   {}
\supervisor {Dr. Maria Polukarov}
\examiner   {Professor Sheng Chen}

\maketitle
\begin{abstract}
This project aim was to model and analyse the effects of competitive pricing behaviors of grocery retailers on the British market. 

This was achieved by creating a multi-agent model, containing retailer and consumer agents. The heterogeneous crowd of retailers employs either a uniform pricing strategy or a ‘local price flexing’ strategy. The actions of these retailers are chosen by predicting the profit of each action, using a perceptron. Following on from the consideration of different economic models, a discrete model was developed so that software agents have a discrete environment to operate within. Within the model, it has been observed how supermarkets with differing behaviors affect a heterogeneous crowd of consumer agents. The model was implemented in Java with Python used to evaluate the results. 

The simulation displays good acceptance with real grocery market behavior, i.e. captures the performance of British retailers thus can be used to determine the impact of changes in their behavior on their competitors and consumers.Furthermore it can be used to provide insight into sustainability of volatile pricing strategies, providing a useful insight in volatility of British supermarket retail industry. 
\end{abstract}
\acknowledgements{
I would like to express my sincere gratitude to Dr Maria Polukarov for her guidance and support which provided me the freedom to take this research in the direction of my interest.\\
\\
I would also like to thank my family and friends for their encouragement and support. To those who quietly listened to my software complaints. To those who worked throughout the nights with me. To those who helped me write what I couldn't say. I cannot thank you enough.
}

\declaration{
I, Stefan Collier, declare that this dissertation and the work presented in it are my own and has been generated by me as the result of my own original research.\\
I confirm that:\\
1. This work was done wholly or mainly while in candidature for a degree at this University;\\
2. Where any part of this dissertation has previously been submitted for any other qualification at this University or any other institution, this has been clearly stated;\\
3. Where I have consulted the published work of others, this is always clearly attributed;\\
4. Where I have quoted from the work of others, the source is always given. With the exception of such quotations, this dissertation is entirely my own work;\\
5. I have acknowledged all main sources of help;\\
6. Where the thesis is based on work done by myself jointly with others, I have made clear exactly what was done by others and what I have contributed myself;\\
7. Either none of this work has been published before submission, or parts of this work have been published by :\\
\\
Stefan Collier\\
April 2016
}
\tableofcontents
\listoffigures
\listoftables

\mainmatter
%% ----------------------------------------------------------------
%\include{Introduction}
%\include{Conclusions}
\include{chapters/1Project/main}
\include{chapters/2Lit/main}
\include{chapters/3Design/HighLevel}
\include{chapters/3Design/InDepth}
\include{chapters/4Impl/main}

\include{chapters/5Experiments/1/main}
\include{chapters/5Experiments/2/main}
\include{chapters/5Experiments/3/main}
\include{chapters/5Experiments/4/main}

\include{chapters/6Conclusion/main}

\appendix
\include{appendix/AppendixB}
\include{appendix/D/main}
\include{appendix/AppendixC}

\backmatter
\bibliographystyle{ecs}
\bibliography{ECS}
\end{document}
%% ----------------------------------------------------------------

 %% ----------------------------------------------------------------
%% Progress.tex
%% ---------------------------------------------------------------- 
\documentclass{ecsprogress}    % Use the progress Style
\graphicspath{{../figs/}}   % Location of your graphics files
    \usepackage{natbib}            % Use Natbib style for the refs.
\hypersetup{colorlinks=true}   % Set to false for black/white printing
\input{Definitions}            % Include your abbreviations



\usepackage{enumitem}% http://ctan.org/pkg/enumitem
\usepackage{multirow}
\usepackage{float}
\usepackage{amsmath}
\usepackage{multicol}
\usepackage{amssymb}
\usepackage[normalem]{ulem}
\useunder{\uline}{\ul}{}
\usepackage{wrapfig}


\usepackage[table,xcdraw]{xcolor}


%% ----------------------------------------------------------------
\begin{document}
\frontmatter
\title      {Heterogeneous Agent-based Model for Supermarket Competition}
\authors    {\texorpdfstring
             {\href{mailto:sc22g13@ecs.soton.ac.uk}{Stefan J. Collier}}
             {Stefan J. Collier}
            }
\addresses  {\groupname\\\deptname\\\univname}
\date       {\today}
\subject    {}
\keywords   {}
\supervisor {Dr. Maria Polukarov}
\examiner   {Professor Sheng Chen}

\maketitle
\begin{abstract}
This project aim was to model and analyse the effects of competitive pricing behaviors of grocery retailers on the British market. 

This was achieved by creating a multi-agent model, containing retailer and consumer agents. The heterogeneous crowd of retailers employs either a uniform pricing strategy or a ‘local price flexing’ strategy. The actions of these retailers are chosen by predicting the profit of each action, using a perceptron. Following on from the consideration of different economic models, a discrete model was developed so that software agents have a discrete environment to operate within. Within the model, it has been observed how supermarkets with differing behaviors affect a heterogeneous crowd of consumer agents. The model was implemented in Java with Python used to evaluate the results. 

The simulation displays good acceptance with real grocery market behavior, i.e. captures the performance of British retailers thus can be used to determine the impact of changes in their behavior on their competitors and consumers.Furthermore it can be used to provide insight into sustainability of volatile pricing strategies, providing a useful insight in volatility of British supermarket retail industry. 
\end{abstract}
\acknowledgements{
I would like to express my sincere gratitude to Dr Maria Polukarov for her guidance and support which provided me the freedom to take this research in the direction of my interest.\\
\\
I would also like to thank my family and friends for their encouragement and support. To those who quietly listened to my software complaints. To those who worked throughout the nights with me. To those who helped me write what I couldn't say. I cannot thank you enough.
}

\declaration{
I, Stefan Collier, declare that this dissertation and the work presented in it are my own and has been generated by me as the result of my own original research.\\
I confirm that:\\
1. This work was done wholly or mainly while in candidature for a degree at this University;\\
2. Where any part of this dissertation has previously been submitted for any other qualification at this University or any other institution, this has been clearly stated;\\
3. Where I have consulted the published work of others, this is always clearly attributed;\\
4. Where I have quoted from the work of others, the source is always given. With the exception of such quotations, this dissertation is entirely my own work;\\
5. I have acknowledged all main sources of help;\\
6. Where the thesis is based on work done by myself jointly with others, I have made clear exactly what was done by others and what I have contributed myself;\\
7. Either none of this work has been published before submission, or parts of this work have been published by :\\
\\
Stefan Collier\\
April 2016
}
\tableofcontents
\listoffigures
\listoftables

\mainmatter
%% ----------------------------------------------------------------
%\include{Introduction}
%\include{Conclusions}
\include{chapters/1Project/main}
\include{chapters/2Lit/main}
\include{chapters/3Design/HighLevel}
\include{chapters/3Design/InDepth}
\include{chapters/4Impl/main}

\include{chapters/5Experiments/1/main}
\include{chapters/5Experiments/2/main}
\include{chapters/5Experiments/3/main}
\include{chapters/5Experiments/4/main}

\include{chapters/6Conclusion/main}

\appendix
\include{appendix/AppendixB}
\include{appendix/D/main}
\include{appendix/AppendixC}

\backmatter
\bibliographystyle{ecs}
\bibliography{ECS}
\end{document}
%% ----------------------------------------------------------------

 %% ----------------------------------------------------------------
%% Progress.tex
%% ---------------------------------------------------------------- 
\documentclass{ecsprogress}    % Use the progress Style
\graphicspath{{../figs/}}   % Location of your graphics files
    \usepackage{natbib}            % Use Natbib style for the refs.
\hypersetup{colorlinks=true}   % Set to false for black/white printing
\input{Definitions}            % Include your abbreviations



\usepackage{enumitem}% http://ctan.org/pkg/enumitem
\usepackage{multirow}
\usepackage{float}
\usepackage{amsmath}
\usepackage{multicol}
\usepackage{amssymb}
\usepackage[normalem]{ulem}
\useunder{\uline}{\ul}{}
\usepackage{wrapfig}


\usepackage[table,xcdraw]{xcolor}


%% ----------------------------------------------------------------
\begin{document}
\frontmatter
\title      {Heterogeneous Agent-based Model for Supermarket Competition}
\authors    {\texorpdfstring
             {\href{mailto:sc22g13@ecs.soton.ac.uk}{Stefan J. Collier}}
             {Stefan J. Collier}
            }
\addresses  {\groupname\\\deptname\\\univname}
\date       {\today}
\subject    {}
\keywords   {}
\supervisor {Dr. Maria Polukarov}
\examiner   {Professor Sheng Chen}

\maketitle
\begin{abstract}
This project aim was to model and analyse the effects of competitive pricing behaviors of grocery retailers on the British market. 

This was achieved by creating a multi-agent model, containing retailer and consumer agents. The heterogeneous crowd of retailers employs either a uniform pricing strategy or a ‘local price flexing’ strategy. The actions of these retailers are chosen by predicting the profit of each action, using a perceptron. Following on from the consideration of different economic models, a discrete model was developed so that software agents have a discrete environment to operate within. Within the model, it has been observed how supermarkets with differing behaviors affect a heterogeneous crowd of consumer agents. The model was implemented in Java with Python used to evaluate the results. 

The simulation displays good acceptance with real grocery market behavior, i.e. captures the performance of British retailers thus can be used to determine the impact of changes in their behavior on their competitors and consumers.Furthermore it can be used to provide insight into sustainability of volatile pricing strategies, providing a useful insight in volatility of British supermarket retail industry. 
\end{abstract}
\acknowledgements{
I would like to express my sincere gratitude to Dr Maria Polukarov for her guidance and support which provided me the freedom to take this research in the direction of my interest.\\
\\
I would also like to thank my family and friends for their encouragement and support. To those who quietly listened to my software complaints. To those who worked throughout the nights with me. To those who helped me write what I couldn't say. I cannot thank you enough.
}

\declaration{
I, Stefan Collier, declare that this dissertation and the work presented in it are my own and has been generated by me as the result of my own original research.\\
I confirm that:\\
1. This work was done wholly or mainly while in candidature for a degree at this University;\\
2. Where any part of this dissertation has previously been submitted for any other qualification at this University or any other institution, this has been clearly stated;\\
3. Where I have consulted the published work of others, this is always clearly attributed;\\
4. Where I have quoted from the work of others, the source is always given. With the exception of such quotations, this dissertation is entirely my own work;\\
5. I have acknowledged all main sources of help;\\
6. Where the thesis is based on work done by myself jointly with others, I have made clear exactly what was done by others and what I have contributed myself;\\
7. Either none of this work has been published before submission, or parts of this work have been published by :\\
\\
Stefan Collier\\
April 2016
}
\tableofcontents
\listoffigures
\listoftables

\mainmatter
%% ----------------------------------------------------------------
%\include{Introduction}
%\include{Conclusions}
\include{chapters/1Project/main}
\include{chapters/2Lit/main}
\include{chapters/3Design/HighLevel}
\include{chapters/3Design/InDepth}
\include{chapters/4Impl/main}

\include{chapters/5Experiments/1/main}
\include{chapters/5Experiments/2/main}
\include{chapters/5Experiments/3/main}
\include{chapters/5Experiments/4/main}

\include{chapters/6Conclusion/main}

\appendix
\include{appendix/AppendixB}
\include{appendix/D/main}
\include{appendix/AppendixC}

\backmatter
\bibliographystyle{ecs}
\bibliography{ECS}
\end{document}
%% ----------------------------------------------------------------


 %% ----------------------------------------------------------------
%% Progress.tex
%% ---------------------------------------------------------------- 
\documentclass{ecsprogress}    % Use the progress Style
\graphicspath{{../figs/}}   % Location of your graphics files
    \usepackage{natbib}            % Use Natbib style for the refs.
\hypersetup{colorlinks=true}   % Set to false for black/white printing
\input{Definitions}            % Include your abbreviations



\usepackage{enumitem}% http://ctan.org/pkg/enumitem
\usepackage{multirow}
\usepackage{float}
\usepackage{amsmath}
\usepackage{multicol}
\usepackage{amssymb}
\usepackage[normalem]{ulem}
\useunder{\uline}{\ul}{}
\usepackage{wrapfig}


\usepackage[table,xcdraw]{xcolor}


%% ----------------------------------------------------------------
\begin{document}
\frontmatter
\title      {Heterogeneous Agent-based Model for Supermarket Competition}
\authors    {\texorpdfstring
             {\href{mailto:sc22g13@ecs.soton.ac.uk}{Stefan J. Collier}}
             {Stefan J. Collier}
            }
\addresses  {\groupname\\\deptname\\\univname}
\date       {\today}
\subject    {}
\keywords   {}
\supervisor {Dr. Maria Polukarov}
\examiner   {Professor Sheng Chen}

\maketitle
\begin{abstract}
This project aim was to model and analyse the effects of competitive pricing behaviors of grocery retailers on the British market. 

This was achieved by creating a multi-agent model, containing retailer and consumer agents. The heterogeneous crowd of retailers employs either a uniform pricing strategy or a ‘local price flexing’ strategy. The actions of these retailers are chosen by predicting the profit of each action, using a perceptron. Following on from the consideration of different economic models, a discrete model was developed so that software agents have a discrete environment to operate within. Within the model, it has been observed how supermarkets with differing behaviors affect a heterogeneous crowd of consumer agents. The model was implemented in Java with Python used to evaluate the results. 

The simulation displays good acceptance with real grocery market behavior, i.e. captures the performance of British retailers thus can be used to determine the impact of changes in their behavior on their competitors and consumers.Furthermore it can be used to provide insight into sustainability of volatile pricing strategies, providing a useful insight in volatility of British supermarket retail industry. 
\end{abstract}
\acknowledgements{
I would like to express my sincere gratitude to Dr Maria Polukarov for her guidance and support which provided me the freedom to take this research in the direction of my interest.\\
\\
I would also like to thank my family and friends for their encouragement and support. To those who quietly listened to my software complaints. To those who worked throughout the nights with me. To those who helped me write what I couldn't say. I cannot thank you enough.
}

\declaration{
I, Stefan Collier, declare that this dissertation and the work presented in it are my own and has been generated by me as the result of my own original research.\\
I confirm that:\\
1. This work was done wholly or mainly while in candidature for a degree at this University;\\
2. Where any part of this dissertation has previously been submitted for any other qualification at this University or any other institution, this has been clearly stated;\\
3. Where I have consulted the published work of others, this is always clearly attributed;\\
4. Where I have quoted from the work of others, the source is always given. With the exception of such quotations, this dissertation is entirely my own work;\\
5. I have acknowledged all main sources of help;\\
6. Where the thesis is based on work done by myself jointly with others, I have made clear exactly what was done by others and what I have contributed myself;\\
7. Either none of this work has been published before submission, or parts of this work have been published by :\\
\\
Stefan Collier\\
April 2016
}
\tableofcontents
\listoffigures
\listoftables

\mainmatter
%% ----------------------------------------------------------------
%\include{Introduction}
%\include{Conclusions}
\include{chapters/1Project/main}
\include{chapters/2Lit/main}
\include{chapters/3Design/HighLevel}
\include{chapters/3Design/InDepth}
\include{chapters/4Impl/main}

\include{chapters/5Experiments/1/main}
\include{chapters/5Experiments/2/main}
\include{chapters/5Experiments/3/main}
\include{chapters/5Experiments/4/main}

\include{chapters/6Conclusion/main}

\appendix
\include{appendix/AppendixB}
\include{appendix/D/main}
\include{appendix/AppendixC}

\backmatter
\bibliographystyle{ecs}
\bibliography{ECS}
\end{document}
%% ----------------------------------------------------------------


\appendix
\include{appendix/AppendixB}
 %% ----------------------------------------------------------------
%% Progress.tex
%% ---------------------------------------------------------------- 
\documentclass{ecsprogress}    % Use the progress Style
\graphicspath{{../figs/}}   % Location of your graphics files
    \usepackage{natbib}            % Use Natbib style for the refs.
\hypersetup{colorlinks=true}   % Set to false for black/white printing
\input{Definitions}            % Include your abbreviations



\usepackage{enumitem}% http://ctan.org/pkg/enumitem
\usepackage{multirow}
\usepackage{float}
\usepackage{amsmath}
\usepackage{multicol}
\usepackage{amssymb}
\usepackage[normalem]{ulem}
\useunder{\uline}{\ul}{}
\usepackage{wrapfig}


\usepackage[table,xcdraw]{xcolor}


%% ----------------------------------------------------------------
\begin{document}
\frontmatter
\title      {Heterogeneous Agent-based Model for Supermarket Competition}
\authors    {\texorpdfstring
             {\href{mailto:sc22g13@ecs.soton.ac.uk}{Stefan J. Collier}}
             {Stefan J. Collier}
            }
\addresses  {\groupname\\\deptname\\\univname}
\date       {\today}
\subject    {}
\keywords   {}
\supervisor {Dr. Maria Polukarov}
\examiner   {Professor Sheng Chen}

\maketitle
\begin{abstract}
This project aim was to model and analyse the effects of competitive pricing behaviors of grocery retailers on the British market. 

This was achieved by creating a multi-agent model, containing retailer and consumer agents. The heterogeneous crowd of retailers employs either a uniform pricing strategy or a ‘local price flexing’ strategy. The actions of these retailers are chosen by predicting the profit of each action, using a perceptron. Following on from the consideration of different economic models, a discrete model was developed so that software agents have a discrete environment to operate within. Within the model, it has been observed how supermarkets with differing behaviors affect a heterogeneous crowd of consumer agents. The model was implemented in Java with Python used to evaluate the results. 

The simulation displays good acceptance with real grocery market behavior, i.e. captures the performance of British retailers thus can be used to determine the impact of changes in their behavior on their competitors and consumers.Furthermore it can be used to provide insight into sustainability of volatile pricing strategies, providing a useful insight in volatility of British supermarket retail industry. 
\end{abstract}
\acknowledgements{
I would like to express my sincere gratitude to Dr Maria Polukarov for her guidance and support which provided me the freedom to take this research in the direction of my interest.\\
\\
I would also like to thank my family and friends for their encouragement and support. To those who quietly listened to my software complaints. To those who worked throughout the nights with me. To those who helped me write what I couldn't say. I cannot thank you enough.
}

\declaration{
I, Stefan Collier, declare that this dissertation and the work presented in it are my own and has been generated by me as the result of my own original research.\\
I confirm that:\\
1. This work was done wholly or mainly while in candidature for a degree at this University;\\
2. Where any part of this dissertation has previously been submitted for any other qualification at this University or any other institution, this has been clearly stated;\\
3. Where I have consulted the published work of others, this is always clearly attributed;\\
4. Where I have quoted from the work of others, the source is always given. With the exception of such quotations, this dissertation is entirely my own work;\\
5. I have acknowledged all main sources of help;\\
6. Where the thesis is based on work done by myself jointly with others, I have made clear exactly what was done by others and what I have contributed myself;\\
7. Either none of this work has been published before submission, or parts of this work have been published by :\\
\\
Stefan Collier\\
April 2016
}
\tableofcontents
\listoffigures
\listoftables

\mainmatter
%% ----------------------------------------------------------------
%\include{Introduction}
%\include{Conclusions}
\include{chapters/1Project/main}
\include{chapters/2Lit/main}
\include{chapters/3Design/HighLevel}
\include{chapters/3Design/InDepth}
\include{chapters/4Impl/main}

\include{chapters/5Experiments/1/main}
\include{chapters/5Experiments/2/main}
\include{chapters/5Experiments/3/main}
\include{chapters/5Experiments/4/main}

\include{chapters/6Conclusion/main}

\appendix
\include{appendix/AppendixB}
\include{appendix/D/main}
\include{appendix/AppendixC}

\backmatter
\bibliographystyle{ecs}
\bibliography{ECS}
\end{document}
%% ----------------------------------------------------------------

\include{appendix/AppendixC}

\backmatter
\bibliographystyle{ecs}
\bibliography{ECS}
\end{document}
%% ----------------------------------------------------------------

 %% ----------------------------------------------------------------
%% Progress.tex
%% ---------------------------------------------------------------- 
\documentclass{ecsprogress}    % Use the progress Style
\graphicspath{{../figs/}}   % Location of your graphics files
    \usepackage{natbib}            % Use Natbib style for the refs.
\hypersetup{colorlinks=true}   % Set to false for black/white printing
\input{Definitions}            % Include your abbreviations



\usepackage{enumitem}% http://ctan.org/pkg/enumitem
\usepackage{multirow}
\usepackage{float}
\usepackage{amsmath}
\usepackage{multicol}
\usepackage{amssymb}
\usepackage[normalem]{ulem}
\useunder{\uline}{\ul}{}
\usepackage{wrapfig}


\usepackage[table,xcdraw]{xcolor}


%% ----------------------------------------------------------------
\begin{document}
\frontmatter
\title      {Heterogeneous Agent-based Model for Supermarket Competition}
\authors    {\texorpdfstring
             {\href{mailto:sc22g13@ecs.soton.ac.uk}{Stefan J. Collier}}
             {Stefan J. Collier}
            }
\addresses  {\groupname\\\deptname\\\univname}
\date       {\today}
\subject    {}
\keywords   {}
\supervisor {Dr. Maria Polukarov}
\examiner   {Professor Sheng Chen}

\maketitle
\begin{abstract}
This project aim was to model and analyse the effects of competitive pricing behaviors of grocery retailers on the British market. 

This was achieved by creating a multi-agent model, containing retailer and consumer agents. The heterogeneous crowd of retailers employs either a uniform pricing strategy or a ‘local price flexing’ strategy. The actions of these retailers are chosen by predicting the profit of each action, using a perceptron. Following on from the consideration of different economic models, a discrete model was developed so that software agents have a discrete environment to operate within. Within the model, it has been observed how supermarkets with differing behaviors affect a heterogeneous crowd of consumer agents. The model was implemented in Java with Python used to evaluate the results. 

The simulation displays good acceptance with real grocery market behavior, i.e. captures the performance of British retailers thus can be used to determine the impact of changes in their behavior on their competitors and consumers.Furthermore it can be used to provide insight into sustainability of volatile pricing strategies, providing a useful insight in volatility of British supermarket retail industry. 
\end{abstract}
\acknowledgements{
I would like to express my sincere gratitude to Dr Maria Polukarov for her guidance and support which provided me the freedom to take this research in the direction of my interest.\\
\\
I would also like to thank my family and friends for their encouragement and support. To those who quietly listened to my software complaints. To those who worked throughout the nights with me. To those who helped me write what I couldn't say. I cannot thank you enough.
}

\declaration{
I, Stefan Collier, declare that this dissertation and the work presented in it are my own and has been generated by me as the result of my own original research.\\
I confirm that:\\
1. This work was done wholly or mainly while in candidature for a degree at this University;\\
2. Where any part of this dissertation has previously been submitted for any other qualification at this University or any other institution, this has been clearly stated;\\
3. Where I have consulted the published work of others, this is always clearly attributed;\\
4. Where I have quoted from the work of others, the source is always given. With the exception of such quotations, this dissertation is entirely my own work;\\
5. I have acknowledged all main sources of help;\\
6. Where the thesis is based on work done by myself jointly with others, I have made clear exactly what was done by others and what I have contributed myself;\\
7. Either none of this work has been published before submission, or parts of this work have been published by :\\
\\
Stefan Collier\\
April 2016
}
\tableofcontents
\listoffigures
\listoftables

\mainmatter
%% ----------------------------------------------------------------
%\include{Introduction}
%\include{Conclusions}
 %% ----------------------------------------------------------------
%% Progress.tex
%% ---------------------------------------------------------------- 
\documentclass{ecsprogress}    % Use the progress Style
\graphicspath{{../figs/}}   % Location of your graphics files
    \usepackage{natbib}            % Use Natbib style for the refs.
\hypersetup{colorlinks=true}   % Set to false for black/white printing
\input{Definitions}            % Include your abbreviations



\usepackage{enumitem}% http://ctan.org/pkg/enumitem
\usepackage{multirow}
\usepackage{float}
\usepackage{amsmath}
\usepackage{multicol}
\usepackage{amssymb}
\usepackage[normalem]{ulem}
\useunder{\uline}{\ul}{}
\usepackage{wrapfig}


\usepackage[table,xcdraw]{xcolor}


%% ----------------------------------------------------------------
\begin{document}
\frontmatter
\title      {Heterogeneous Agent-based Model for Supermarket Competition}
\authors    {\texorpdfstring
             {\href{mailto:sc22g13@ecs.soton.ac.uk}{Stefan J. Collier}}
             {Stefan J. Collier}
            }
\addresses  {\groupname\\\deptname\\\univname}
\date       {\today}
\subject    {}
\keywords   {}
\supervisor {Dr. Maria Polukarov}
\examiner   {Professor Sheng Chen}

\maketitle
\begin{abstract}
This project aim was to model and analyse the effects of competitive pricing behaviors of grocery retailers on the British market. 

This was achieved by creating a multi-agent model, containing retailer and consumer agents. The heterogeneous crowd of retailers employs either a uniform pricing strategy or a ‘local price flexing’ strategy. The actions of these retailers are chosen by predicting the profit of each action, using a perceptron. Following on from the consideration of different economic models, a discrete model was developed so that software agents have a discrete environment to operate within. Within the model, it has been observed how supermarkets with differing behaviors affect a heterogeneous crowd of consumer agents. The model was implemented in Java with Python used to evaluate the results. 

The simulation displays good acceptance with real grocery market behavior, i.e. captures the performance of British retailers thus can be used to determine the impact of changes in their behavior on their competitors and consumers.Furthermore it can be used to provide insight into sustainability of volatile pricing strategies, providing a useful insight in volatility of British supermarket retail industry. 
\end{abstract}
\acknowledgements{
I would like to express my sincere gratitude to Dr Maria Polukarov for her guidance and support which provided me the freedom to take this research in the direction of my interest.\\
\\
I would also like to thank my family and friends for their encouragement and support. To those who quietly listened to my software complaints. To those who worked throughout the nights with me. To those who helped me write what I couldn't say. I cannot thank you enough.
}

\declaration{
I, Stefan Collier, declare that this dissertation and the work presented in it are my own and has been generated by me as the result of my own original research.\\
I confirm that:\\
1. This work was done wholly or mainly while in candidature for a degree at this University;\\
2. Where any part of this dissertation has previously been submitted for any other qualification at this University or any other institution, this has been clearly stated;\\
3. Where I have consulted the published work of others, this is always clearly attributed;\\
4. Where I have quoted from the work of others, the source is always given. With the exception of such quotations, this dissertation is entirely my own work;\\
5. I have acknowledged all main sources of help;\\
6. Where the thesis is based on work done by myself jointly with others, I have made clear exactly what was done by others and what I have contributed myself;\\
7. Either none of this work has been published before submission, or parts of this work have been published by :\\
\\
Stefan Collier\\
April 2016
}
\tableofcontents
\listoffigures
\listoftables

\mainmatter
%% ----------------------------------------------------------------
%\include{Introduction}
%\include{Conclusions}
\include{chapters/1Project/main}
\include{chapters/2Lit/main}
\include{chapters/3Design/HighLevel}
\include{chapters/3Design/InDepth}
\include{chapters/4Impl/main}

\include{chapters/5Experiments/1/main}
\include{chapters/5Experiments/2/main}
\include{chapters/5Experiments/3/main}
\include{chapters/5Experiments/4/main}

\include{chapters/6Conclusion/main}

\appendix
\include{appendix/AppendixB}
\include{appendix/D/main}
\include{appendix/AppendixC}

\backmatter
\bibliographystyle{ecs}
\bibliography{ECS}
\end{document}
%% ----------------------------------------------------------------

 %% ----------------------------------------------------------------
%% Progress.tex
%% ---------------------------------------------------------------- 
\documentclass{ecsprogress}    % Use the progress Style
\graphicspath{{../figs/}}   % Location of your graphics files
    \usepackage{natbib}            % Use Natbib style for the refs.
\hypersetup{colorlinks=true}   % Set to false for black/white printing
\input{Definitions}            % Include your abbreviations



\usepackage{enumitem}% http://ctan.org/pkg/enumitem
\usepackage{multirow}
\usepackage{float}
\usepackage{amsmath}
\usepackage{multicol}
\usepackage{amssymb}
\usepackage[normalem]{ulem}
\useunder{\uline}{\ul}{}
\usepackage{wrapfig}


\usepackage[table,xcdraw]{xcolor}


%% ----------------------------------------------------------------
\begin{document}
\frontmatter
\title      {Heterogeneous Agent-based Model for Supermarket Competition}
\authors    {\texorpdfstring
             {\href{mailto:sc22g13@ecs.soton.ac.uk}{Stefan J. Collier}}
             {Stefan J. Collier}
            }
\addresses  {\groupname\\\deptname\\\univname}
\date       {\today}
\subject    {}
\keywords   {}
\supervisor {Dr. Maria Polukarov}
\examiner   {Professor Sheng Chen}

\maketitle
\begin{abstract}
This project aim was to model and analyse the effects of competitive pricing behaviors of grocery retailers on the British market. 

This was achieved by creating a multi-agent model, containing retailer and consumer agents. The heterogeneous crowd of retailers employs either a uniform pricing strategy or a ‘local price flexing’ strategy. The actions of these retailers are chosen by predicting the profit of each action, using a perceptron. Following on from the consideration of different economic models, a discrete model was developed so that software agents have a discrete environment to operate within. Within the model, it has been observed how supermarkets with differing behaviors affect a heterogeneous crowd of consumer agents. The model was implemented in Java with Python used to evaluate the results. 

The simulation displays good acceptance with real grocery market behavior, i.e. captures the performance of British retailers thus can be used to determine the impact of changes in their behavior on their competitors and consumers.Furthermore it can be used to provide insight into sustainability of volatile pricing strategies, providing a useful insight in volatility of British supermarket retail industry. 
\end{abstract}
\acknowledgements{
I would like to express my sincere gratitude to Dr Maria Polukarov for her guidance and support which provided me the freedom to take this research in the direction of my interest.\\
\\
I would also like to thank my family and friends for their encouragement and support. To those who quietly listened to my software complaints. To those who worked throughout the nights with me. To those who helped me write what I couldn't say. I cannot thank you enough.
}

\declaration{
I, Stefan Collier, declare that this dissertation and the work presented in it are my own and has been generated by me as the result of my own original research.\\
I confirm that:\\
1. This work was done wholly or mainly while in candidature for a degree at this University;\\
2. Where any part of this dissertation has previously been submitted for any other qualification at this University or any other institution, this has been clearly stated;\\
3. Where I have consulted the published work of others, this is always clearly attributed;\\
4. Where I have quoted from the work of others, the source is always given. With the exception of such quotations, this dissertation is entirely my own work;\\
5. I have acknowledged all main sources of help;\\
6. Where the thesis is based on work done by myself jointly with others, I have made clear exactly what was done by others and what I have contributed myself;\\
7. Either none of this work has been published before submission, or parts of this work have been published by :\\
\\
Stefan Collier\\
April 2016
}
\tableofcontents
\listoffigures
\listoftables

\mainmatter
%% ----------------------------------------------------------------
%\include{Introduction}
%\include{Conclusions}
\include{chapters/1Project/main}
\include{chapters/2Lit/main}
\include{chapters/3Design/HighLevel}
\include{chapters/3Design/InDepth}
\include{chapters/4Impl/main}

\include{chapters/5Experiments/1/main}
\include{chapters/5Experiments/2/main}
\include{chapters/5Experiments/3/main}
\include{chapters/5Experiments/4/main}

\include{chapters/6Conclusion/main}

\appendix
\include{appendix/AppendixB}
\include{appendix/D/main}
\include{appendix/AppendixC}

\backmatter
\bibliographystyle{ecs}
\bibliography{ECS}
\end{document}
%% ----------------------------------------------------------------

\include{chapters/3Design/HighLevel}
\include{chapters/3Design/InDepth}
 %% ----------------------------------------------------------------
%% Progress.tex
%% ---------------------------------------------------------------- 
\documentclass{ecsprogress}    % Use the progress Style
\graphicspath{{../figs/}}   % Location of your graphics files
    \usepackage{natbib}            % Use Natbib style for the refs.
\hypersetup{colorlinks=true}   % Set to false for black/white printing
\input{Definitions}            % Include your abbreviations



\usepackage{enumitem}% http://ctan.org/pkg/enumitem
\usepackage{multirow}
\usepackage{float}
\usepackage{amsmath}
\usepackage{multicol}
\usepackage{amssymb}
\usepackage[normalem]{ulem}
\useunder{\uline}{\ul}{}
\usepackage{wrapfig}


\usepackage[table,xcdraw]{xcolor}


%% ----------------------------------------------------------------
\begin{document}
\frontmatter
\title      {Heterogeneous Agent-based Model for Supermarket Competition}
\authors    {\texorpdfstring
             {\href{mailto:sc22g13@ecs.soton.ac.uk}{Stefan J. Collier}}
             {Stefan J. Collier}
            }
\addresses  {\groupname\\\deptname\\\univname}
\date       {\today}
\subject    {}
\keywords   {}
\supervisor {Dr. Maria Polukarov}
\examiner   {Professor Sheng Chen}

\maketitle
\begin{abstract}
This project aim was to model and analyse the effects of competitive pricing behaviors of grocery retailers on the British market. 

This was achieved by creating a multi-agent model, containing retailer and consumer agents. The heterogeneous crowd of retailers employs either a uniform pricing strategy or a ‘local price flexing’ strategy. The actions of these retailers are chosen by predicting the profit of each action, using a perceptron. Following on from the consideration of different economic models, a discrete model was developed so that software agents have a discrete environment to operate within. Within the model, it has been observed how supermarkets with differing behaviors affect a heterogeneous crowd of consumer agents. The model was implemented in Java with Python used to evaluate the results. 

The simulation displays good acceptance with real grocery market behavior, i.e. captures the performance of British retailers thus can be used to determine the impact of changes in their behavior on their competitors and consumers.Furthermore it can be used to provide insight into sustainability of volatile pricing strategies, providing a useful insight in volatility of British supermarket retail industry. 
\end{abstract}
\acknowledgements{
I would like to express my sincere gratitude to Dr Maria Polukarov for her guidance and support which provided me the freedom to take this research in the direction of my interest.\\
\\
I would also like to thank my family and friends for their encouragement and support. To those who quietly listened to my software complaints. To those who worked throughout the nights with me. To those who helped me write what I couldn't say. I cannot thank you enough.
}

\declaration{
I, Stefan Collier, declare that this dissertation and the work presented in it are my own and has been generated by me as the result of my own original research.\\
I confirm that:\\
1. This work was done wholly or mainly while in candidature for a degree at this University;\\
2. Where any part of this dissertation has previously been submitted for any other qualification at this University or any other institution, this has been clearly stated;\\
3. Where I have consulted the published work of others, this is always clearly attributed;\\
4. Where I have quoted from the work of others, the source is always given. With the exception of such quotations, this dissertation is entirely my own work;\\
5. I have acknowledged all main sources of help;\\
6. Where the thesis is based on work done by myself jointly with others, I have made clear exactly what was done by others and what I have contributed myself;\\
7. Either none of this work has been published before submission, or parts of this work have been published by :\\
\\
Stefan Collier\\
April 2016
}
\tableofcontents
\listoffigures
\listoftables

\mainmatter
%% ----------------------------------------------------------------
%\include{Introduction}
%\include{Conclusions}
\include{chapters/1Project/main}
\include{chapters/2Lit/main}
\include{chapters/3Design/HighLevel}
\include{chapters/3Design/InDepth}
\include{chapters/4Impl/main}

\include{chapters/5Experiments/1/main}
\include{chapters/5Experiments/2/main}
\include{chapters/5Experiments/3/main}
\include{chapters/5Experiments/4/main}

\include{chapters/6Conclusion/main}

\appendix
\include{appendix/AppendixB}
\include{appendix/D/main}
\include{appendix/AppendixC}

\backmatter
\bibliographystyle{ecs}
\bibliography{ECS}
\end{document}
%% ----------------------------------------------------------------


 %% ----------------------------------------------------------------
%% Progress.tex
%% ---------------------------------------------------------------- 
\documentclass{ecsprogress}    % Use the progress Style
\graphicspath{{../figs/}}   % Location of your graphics files
    \usepackage{natbib}            % Use Natbib style for the refs.
\hypersetup{colorlinks=true}   % Set to false for black/white printing
\input{Definitions}            % Include your abbreviations



\usepackage{enumitem}% http://ctan.org/pkg/enumitem
\usepackage{multirow}
\usepackage{float}
\usepackage{amsmath}
\usepackage{multicol}
\usepackage{amssymb}
\usepackage[normalem]{ulem}
\useunder{\uline}{\ul}{}
\usepackage{wrapfig}


\usepackage[table,xcdraw]{xcolor}


%% ----------------------------------------------------------------
\begin{document}
\frontmatter
\title      {Heterogeneous Agent-based Model for Supermarket Competition}
\authors    {\texorpdfstring
             {\href{mailto:sc22g13@ecs.soton.ac.uk}{Stefan J. Collier}}
             {Stefan J. Collier}
            }
\addresses  {\groupname\\\deptname\\\univname}
\date       {\today}
\subject    {}
\keywords   {}
\supervisor {Dr. Maria Polukarov}
\examiner   {Professor Sheng Chen}

\maketitle
\begin{abstract}
This project aim was to model and analyse the effects of competitive pricing behaviors of grocery retailers on the British market. 

This was achieved by creating a multi-agent model, containing retailer and consumer agents. The heterogeneous crowd of retailers employs either a uniform pricing strategy or a ‘local price flexing’ strategy. The actions of these retailers are chosen by predicting the profit of each action, using a perceptron. Following on from the consideration of different economic models, a discrete model was developed so that software agents have a discrete environment to operate within. Within the model, it has been observed how supermarkets with differing behaviors affect a heterogeneous crowd of consumer agents. The model was implemented in Java with Python used to evaluate the results. 

The simulation displays good acceptance with real grocery market behavior, i.e. captures the performance of British retailers thus can be used to determine the impact of changes in their behavior on their competitors and consumers.Furthermore it can be used to provide insight into sustainability of volatile pricing strategies, providing a useful insight in volatility of British supermarket retail industry. 
\end{abstract}
\acknowledgements{
I would like to express my sincere gratitude to Dr Maria Polukarov for her guidance and support which provided me the freedom to take this research in the direction of my interest.\\
\\
I would also like to thank my family and friends for their encouragement and support. To those who quietly listened to my software complaints. To those who worked throughout the nights with me. To those who helped me write what I couldn't say. I cannot thank you enough.
}

\declaration{
I, Stefan Collier, declare that this dissertation and the work presented in it are my own and has been generated by me as the result of my own original research.\\
I confirm that:\\
1. This work was done wholly or mainly while in candidature for a degree at this University;\\
2. Where any part of this dissertation has previously been submitted for any other qualification at this University or any other institution, this has been clearly stated;\\
3. Where I have consulted the published work of others, this is always clearly attributed;\\
4. Where I have quoted from the work of others, the source is always given. With the exception of such quotations, this dissertation is entirely my own work;\\
5. I have acknowledged all main sources of help;\\
6. Where the thesis is based on work done by myself jointly with others, I have made clear exactly what was done by others and what I have contributed myself;\\
7. Either none of this work has been published before submission, or parts of this work have been published by :\\
\\
Stefan Collier\\
April 2016
}
\tableofcontents
\listoffigures
\listoftables

\mainmatter
%% ----------------------------------------------------------------
%\include{Introduction}
%\include{Conclusions}
\include{chapters/1Project/main}
\include{chapters/2Lit/main}
\include{chapters/3Design/HighLevel}
\include{chapters/3Design/InDepth}
\include{chapters/4Impl/main}

\include{chapters/5Experiments/1/main}
\include{chapters/5Experiments/2/main}
\include{chapters/5Experiments/3/main}
\include{chapters/5Experiments/4/main}

\include{chapters/6Conclusion/main}

\appendix
\include{appendix/AppendixB}
\include{appendix/D/main}
\include{appendix/AppendixC}

\backmatter
\bibliographystyle{ecs}
\bibliography{ECS}
\end{document}
%% ----------------------------------------------------------------

 %% ----------------------------------------------------------------
%% Progress.tex
%% ---------------------------------------------------------------- 
\documentclass{ecsprogress}    % Use the progress Style
\graphicspath{{../figs/}}   % Location of your graphics files
    \usepackage{natbib}            % Use Natbib style for the refs.
\hypersetup{colorlinks=true}   % Set to false for black/white printing
\input{Definitions}            % Include your abbreviations



\usepackage{enumitem}% http://ctan.org/pkg/enumitem
\usepackage{multirow}
\usepackage{float}
\usepackage{amsmath}
\usepackage{multicol}
\usepackage{amssymb}
\usepackage[normalem]{ulem}
\useunder{\uline}{\ul}{}
\usepackage{wrapfig}


\usepackage[table,xcdraw]{xcolor}


%% ----------------------------------------------------------------
\begin{document}
\frontmatter
\title      {Heterogeneous Agent-based Model for Supermarket Competition}
\authors    {\texorpdfstring
             {\href{mailto:sc22g13@ecs.soton.ac.uk}{Stefan J. Collier}}
             {Stefan J. Collier}
            }
\addresses  {\groupname\\\deptname\\\univname}
\date       {\today}
\subject    {}
\keywords   {}
\supervisor {Dr. Maria Polukarov}
\examiner   {Professor Sheng Chen}

\maketitle
\begin{abstract}
This project aim was to model and analyse the effects of competitive pricing behaviors of grocery retailers on the British market. 

This was achieved by creating a multi-agent model, containing retailer and consumer agents. The heterogeneous crowd of retailers employs either a uniform pricing strategy or a ‘local price flexing’ strategy. The actions of these retailers are chosen by predicting the profit of each action, using a perceptron. Following on from the consideration of different economic models, a discrete model was developed so that software agents have a discrete environment to operate within. Within the model, it has been observed how supermarkets with differing behaviors affect a heterogeneous crowd of consumer agents. The model was implemented in Java with Python used to evaluate the results. 

The simulation displays good acceptance with real grocery market behavior, i.e. captures the performance of British retailers thus can be used to determine the impact of changes in their behavior on their competitors and consumers.Furthermore it can be used to provide insight into sustainability of volatile pricing strategies, providing a useful insight in volatility of British supermarket retail industry. 
\end{abstract}
\acknowledgements{
I would like to express my sincere gratitude to Dr Maria Polukarov for her guidance and support which provided me the freedom to take this research in the direction of my interest.\\
\\
I would also like to thank my family and friends for their encouragement and support. To those who quietly listened to my software complaints. To those who worked throughout the nights with me. To those who helped me write what I couldn't say. I cannot thank you enough.
}

\declaration{
I, Stefan Collier, declare that this dissertation and the work presented in it are my own and has been generated by me as the result of my own original research.\\
I confirm that:\\
1. This work was done wholly or mainly while in candidature for a degree at this University;\\
2. Where any part of this dissertation has previously been submitted for any other qualification at this University or any other institution, this has been clearly stated;\\
3. Where I have consulted the published work of others, this is always clearly attributed;\\
4. Where I have quoted from the work of others, the source is always given. With the exception of such quotations, this dissertation is entirely my own work;\\
5. I have acknowledged all main sources of help;\\
6. Where the thesis is based on work done by myself jointly with others, I have made clear exactly what was done by others and what I have contributed myself;\\
7. Either none of this work has been published before submission, or parts of this work have been published by :\\
\\
Stefan Collier\\
April 2016
}
\tableofcontents
\listoffigures
\listoftables

\mainmatter
%% ----------------------------------------------------------------
%\include{Introduction}
%\include{Conclusions}
\include{chapters/1Project/main}
\include{chapters/2Lit/main}
\include{chapters/3Design/HighLevel}
\include{chapters/3Design/InDepth}
\include{chapters/4Impl/main}

\include{chapters/5Experiments/1/main}
\include{chapters/5Experiments/2/main}
\include{chapters/5Experiments/3/main}
\include{chapters/5Experiments/4/main}

\include{chapters/6Conclusion/main}

\appendix
\include{appendix/AppendixB}
\include{appendix/D/main}
\include{appendix/AppendixC}

\backmatter
\bibliographystyle{ecs}
\bibliography{ECS}
\end{document}
%% ----------------------------------------------------------------

 %% ----------------------------------------------------------------
%% Progress.tex
%% ---------------------------------------------------------------- 
\documentclass{ecsprogress}    % Use the progress Style
\graphicspath{{../figs/}}   % Location of your graphics files
    \usepackage{natbib}            % Use Natbib style for the refs.
\hypersetup{colorlinks=true}   % Set to false for black/white printing
\input{Definitions}            % Include your abbreviations



\usepackage{enumitem}% http://ctan.org/pkg/enumitem
\usepackage{multirow}
\usepackage{float}
\usepackage{amsmath}
\usepackage{multicol}
\usepackage{amssymb}
\usepackage[normalem]{ulem}
\useunder{\uline}{\ul}{}
\usepackage{wrapfig}


\usepackage[table,xcdraw]{xcolor}


%% ----------------------------------------------------------------
\begin{document}
\frontmatter
\title      {Heterogeneous Agent-based Model for Supermarket Competition}
\authors    {\texorpdfstring
             {\href{mailto:sc22g13@ecs.soton.ac.uk}{Stefan J. Collier}}
             {Stefan J. Collier}
            }
\addresses  {\groupname\\\deptname\\\univname}
\date       {\today}
\subject    {}
\keywords   {}
\supervisor {Dr. Maria Polukarov}
\examiner   {Professor Sheng Chen}

\maketitle
\begin{abstract}
This project aim was to model and analyse the effects of competitive pricing behaviors of grocery retailers on the British market. 

This was achieved by creating a multi-agent model, containing retailer and consumer agents. The heterogeneous crowd of retailers employs either a uniform pricing strategy or a ‘local price flexing’ strategy. The actions of these retailers are chosen by predicting the profit of each action, using a perceptron. Following on from the consideration of different economic models, a discrete model was developed so that software agents have a discrete environment to operate within. Within the model, it has been observed how supermarkets with differing behaviors affect a heterogeneous crowd of consumer agents. The model was implemented in Java with Python used to evaluate the results. 

The simulation displays good acceptance with real grocery market behavior, i.e. captures the performance of British retailers thus can be used to determine the impact of changes in their behavior on their competitors and consumers.Furthermore it can be used to provide insight into sustainability of volatile pricing strategies, providing a useful insight in volatility of British supermarket retail industry. 
\end{abstract}
\acknowledgements{
I would like to express my sincere gratitude to Dr Maria Polukarov for her guidance and support which provided me the freedom to take this research in the direction of my interest.\\
\\
I would also like to thank my family and friends for their encouragement and support. To those who quietly listened to my software complaints. To those who worked throughout the nights with me. To those who helped me write what I couldn't say. I cannot thank you enough.
}

\declaration{
I, Stefan Collier, declare that this dissertation and the work presented in it are my own and has been generated by me as the result of my own original research.\\
I confirm that:\\
1. This work was done wholly or mainly while in candidature for a degree at this University;\\
2. Where any part of this dissertation has previously been submitted for any other qualification at this University or any other institution, this has been clearly stated;\\
3. Where I have consulted the published work of others, this is always clearly attributed;\\
4. Where I have quoted from the work of others, the source is always given. With the exception of such quotations, this dissertation is entirely my own work;\\
5. I have acknowledged all main sources of help;\\
6. Where the thesis is based on work done by myself jointly with others, I have made clear exactly what was done by others and what I have contributed myself;\\
7. Either none of this work has been published before submission, or parts of this work have been published by :\\
\\
Stefan Collier\\
April 2016
}
\tableofcontents
\listoffigures
\listoftables

\mainmatter
%% ----------------------------------------------------------------
%\include{Introduction}
%\include{Conclusions}
\include{chapters/1Project/main}
\include{chapters/2Lit/main}
\include{chapters/3Design/HighLevel}
\include{chapters/3Design/InDepth}
\include{chapters/4Impl/main}

\include{chapters/5Experiments/1/main}
\include{chapters/5Experiments/2/main}
\include{chapters/5Experiments/3/main}
\include{chapters/5Experiments/4/main}

\include{chapters/6Conclusion/main}

\appendix
\include{appendix/AppendixB}
\include{appendix/D/main}
\include{appendix/AppendixC}

\backmatter
\bibliographystyle{ecs}
\bibliography{ECS}
\end{document}
%% ----------------------------------------------------------------

 %% ----------------------------------------------------------------
%% Progress.tex
%% ---------------------------------------------------------------- 
\documentclass{ecsprogress}    % Use the progress Style
\graphicspath{{../figs/}}   % Location of your graphics files
    \usepackage{natbib}            % Use Natbib style for the refs.
\hypersetup{colorlinks=true}   % Set to false for black/white printing
\input{Definitions}            % Include your abbreviations



\usepackage{enumitem}% http://ctan.org/pkg/enumitem
\usepackage{multirow}
\usepackage{float}
\usepackage{amsmath}
\usepackage{multicol}
\usepackage{amssymb}
\usepackage[normalem]{ulem}
\useunder{\uline}{\ul}{}
\usepackage{wrapfig}


\usepackage[table,xcdraw]{xcolor}


%% ----------------------------------------------------------------
\begin{document}
\frontmatter
\title      {Heterogeneous Agent-based Model for Supermarket Competition}
\authors    {\texorpdfstring
             {\href{mailto:sc22g13@ecs.soton.ac.uk}{Stefan J. Collier}}
             {Stefan J. Collier}
            }
\addresses  {\groupname\\\deptname\\\univname}
\date       {\today}
\subject    {}
\keywords   {}
\supervisor {Dr. Maria Polukarov}
\examiner   {Professor Sheng Chen}

\maketitle
\begin{abstract}
This project aim was to model and analyse the effects of competitive pricing behaviors of grocery retailers on the British market. 

This was achieved by creating a multi-agent model, containing retailer and consumer agents. The heterogeneous crowd of retailers employs either a uniform pricing strategy or a ‘local price flexing’ strategy. The actions of these retailers are chosen by predicting the profit of each action, using a perceptron. Following on from the consideration of different economic models, a discrete model was developed so that software agents have a discrete environment to operate within. Within the model, it has been observed how supermarkets with differing behaviors affect a heterogeneous crowd of consumer agents. The model was implemented in Java with Python used to evaluate the results. 

The simulation displays good acceptance with real grocery market behavior, i.e. captures the performance of British retailers thus can be used to determine the impact of changes in their behavior on their competitors and consumers.Furthermore it can be used to provide insight into sustainability of volatile pricing strategies, providing a useful insight in volatility of British supermarket retail industry. 
\end{abstract}
\acknowledgements{
I would like to express my sincere gratitude to Dr Maria Polukarov for her guidance and support which provided me the freedom to take this research in the direction of my interest.\\
\\
I would also like to thank my family and friends for their encouragement and support. To those who quietly listened to my software complaints. To those who worked throughout the nights with me. To those who helped me write what I couldn't say. I cannot thank you enough.
}

\declaration{
I, Stefan Collier, declare that this dissertation and the work presented in it are my own and has been generated by me as the result of my own original research.\\
I confirm that:\\
1. This work was done wholly or mainly while in candidature for a degree at this University;\\
2. Where any part of this dissertation has previously been submitted for any other qualification at this University or any other institution, this has been clearly stated;\\
3. Where I have consulted the published work of others, this is always clearly attributed;\\
4. Where I have quoted from the work of others, the source is always given. With the exception of such quotations, this dissertation is entirely my own work;\\
5. I have acknowledged all main sources of help;\\
6. Where the thesis is based on work done by myself jointly with others, I have made clear exactly what was done by others and what I have contributed myself;\\
7. Either none of this work has been published before submission, or parts of this work have been published by :\\
\\
Stefan Collier\\
April 2016
}
\tableofcontents
\listoffigures
\listoftables

\mainmatter
%% ----------------------------------------------------------------
%\include{Introduction}
%\include{Conclusions}
\include{chapters/1Project/main}
\include{chapters/2Lit/main}
\include{chapters/3Design/HighLevel}
\include{chapters/3Design/InDepth}
\include{chapters/4Impl/main}

\include{chapters/5Experiments/1/main}
\include{chapters/5Experiments/2/main}
\include{chapters/5Experiments/3/main}
\include{chapters/5Experiments/4/main}

\include{chapters/6Conclusion/main}

\appendix
\include{appendix/AppendixB}
\include{appendix/D/main}
\include{appendix/AppendixC}

\backmatter
\bibliographystyle{ecs}
\bibliography{ECS}
\end{document}
%% ----------------------------------------------------------------


 %% ----------------------------------------------------------------
%% Progress.tex
%% ---------------------------------------------------------------- 
\documentclass{ecsprogress}    % Use the progress Style
\graphicspath{{../figs/}}   % Location of your graphics files
    \usepackage{natbib}            % Use Natbib style for the refs.
\hypersetup{colorlinks=true}   % Set to false for black/white printing
\input{Definitions}            % Include your abbreviations



\usepackage{enumitem}% http://ctan.org/pkg/enumitem
\usepackage{multirow}
\usepackage{float}
\usepackage{amsmath}
\usepackage{multicol}
\usepackage{amssymb}
\usepackage[normalem]{ulem}
\useunder{\uline}{\ul}{}
\usepackage{wrapfig}


\usepackage[table,xcdraw]{xcolor}


%% ----------------------------------------------------------------
\begin{document}
\frontmatter
\title      {Heterogeneous Agent-based Model for Supermarket Competition}
\authors    {\texorpdfstring
             {\href{mailto:sc22g13@ecs.soton.ac.uk}{Stefan J. Collier}}
             {Stefan J. Collier}
            }
\addresses  {\groupname\\\deptname\\\univname}
\date       {\today}
\subject    {}
\keywords   {}
\supervisor {Dr. Maria Polukarov}
\examiner   {Professor Sheng Chen}

\maketitle
\begin{abstract}
This project aim was to model and analyse the effects of competitive pricing behaviors of grocery retailers on the British market. 

This was achieved by creating a multi-agent model, containing retailer and consumer agents. The heterogeneous crowd of retailers employs either a uniform pricing strategy or a ‘local price flexing’ strategy. The actions of these retailers are chosen by predicting the profit of each action, using a perceptron. Following on from the consideration of different economic models, a discrete model was developed so that software agents have a discrete environment to operate within. Within the model, it has been observed how supermarkets with differing behaviors affect a heterogeneous crowd of consumer agents. The model was implemented in Java with Python used to evaluate the results. 

The simulation displays good acceptance with real grocery market behavior, i.e. captures the performance of British retailers thus can be used to determine the impact of changes in their behavior on their competitors and consumers.Furthermore it can be used to provide insight into sustainability of volatile pricing strategies, providing a useful insight in volatility of British supermarket retail industry. 
\end{abstract}
\acknowledgements{
I would like to express my sincere gratitude to Dr Maria Polukarov for her guidance and support which provided me the freedom to take this research in the direction of my interest.\\
\\
I would also like to thank my family and friends for their encouragement and support. To those who quietly listened to my software complaints. To those who worked throughout the nights with me. To those who helped me write what I couldn't say. I cannot thank you enough.
}

\declaration{
I, Stefan Collier, declare that this dissertation and the work presented in it are my own and has been generated by me as the result of my own original research.\\
I confirm that:\\
1. This work was done wholly or mainly while in candidature for a degree at this University;\\
2. Where any part of this dissertation has previously been submitted for any other qualification at this University or any other institution, this has been clearly stated;\\
3. Where I have consulted the published work of others, this is always clearly attributed;\\
4. Where I have quoted from the work of others, the source is always given. With the exception of such quotations, this dissertation is entirely my own work;\\
5. I have acknowledged all main sources of help;\\
6. Where the thesis is based on work done by myself jointly with others, I have made clear exactly what was done by others and what I have contributed myself;\\
7. Either none of this work has been published before submission, or parts of this work have been published by :\\
\\
Stefan Collier\\
April 2016
}
\tableofcontents
\listoffigures
\listoftables

\mainmatter
%% ----------------------------------------------------------------
%\include{Introduction}
%\include{Conclusions}
\include{chapters/1Project/main}
\include{chapters/2Lit/main}
\include{chapters/3Design/HighLevel}
\include{chapters/3Design/InDepth}
\include{chapters/4Impl/main}

\include{chapters/5Experiments/1/main}
\include{chapters/5Experiments/2/main}
\include{chapters/5Experiments/3/main}
\include{chapters/5Experiments/4/main}

\include{chapters/6Conclusion/main}

\appendix
\include{appendix/AppendixB}
\include{appendix/D/main}
\include{appendix/AppendixC}

\backmatter
\bibliographystyle{ecs}
\bibliography{ECS}
\end{document}
%% ----------------------------------------------------------------


\appendix
\include{appendix/AppendixB}
 %% ----------------------------------------------------------------
%% Progress.tex
%% ---------------------------------------------------------------- 
\documentclass{ecsprogress}    % Use the progress Style
\graphicspath{{../figs/}}   % Location of your graphics files
    \usepackage{natbib}            % Use Natbib style for the refs.
\hypersetup{colorlinks=true}   % Set to false for black/white printing
\input{Definitions}            % Include your abbreviations



\usepackage{enumitem}% http://ctan.org/pkg/enumitem
\usepackage{multirow}
\usepackage{float}
\usepackage{amsmath}
\usepackage{multicol}
\usepackage{amssymb}
\usepackage[normalem]{ulem}
\useunder{\uline}{\ul}{}
\usepackage{wrapfig}


\usepackage[table,xcdraw]{xcolor}


%% ----------------------------------------------------------------
\begin{document}
\frontmatter
\title      {Heterogeneous Agent-based Model for Supermarket Competition}
\authors    {\texorpdfstring
             {\href{mailto:sc22g13@ecs.soton.ac.uk}{Stefan J. Collier}}
             {Stefan J. Collier}
            }
\addresses  {\groupname\\\deptname\\\univname}
\date       {\today}
\subject    {}
\keywords   {}
\supervisor {Dr. Maria Polukarov}
\examiner   {Professor Sheng Chen}

\maketitle
\begin{abstract}
This project aim was to model and analyse the effects of competitive pricing behaviors of grocery retailers on the British market. 

This was achieved by creating a multi-agent model, containing retailer and consumer agents. The heterogeneous crowd of retailers employs either a uniform pricing strategy or a ‘local price flexing’ strategy. The actions of these retailers are chosen by predicting the profit of each action, using a perceptron. Following on from the consideration of different economic models, a discrete model was developed so that software agents have a discrete environment to operate within. Within the model, it has been observed how supermarkets with differing behaviors affect a heterogeneous crowd of consumer agents. The model was implemented in Java with Python used to evaluate the results. 

The simulation displays good acceptance with real grocery market behavior, i.e. captures the performance of British retailers thus can be used to determine the impact of changes in their behavior on their competitors and consumers.Furthermore it can be used to provide insight into sustainability of volatile pricing strategies, providing a useful insight in volatility of British supermarket retail industry. 
\end{abstract}
\acknowledgements{
I would like to express my sincere gratitude to Dr Maria Polukarov for her guidance and support which provided me the freedom to take this research in the direction of my interest.\\
\\
I would also like to thank my family and friends for their encouragement and support. To those who quietly listened to my software complaints. To those who worked throughout the nights with me. To those who helped me write what I couldn't say. I cannot thank you enough.
}

\declaration{
I, Stefan Collier, declare that this dissertation and the work presented in it are my own and has been generated by me as the result of my own original research.\\
I confirm that:\\
1. This work was done wholly or mainly while in candidature for a degree at this University;\\
2. Where any part of this dissertation has previously been submitted for any other qualification at this University or any other institution, this has been clearly stated;\\
3. Where I have consulted the published work of others, this is always clearly attributed;\\
4. Where I have quoted from the work of others, the source is always given. With the exception of such quotations, this dissertation is entirely my own work;\\
5. I have acknowledged all main sources of help;\\
6. Where the thesis is based on work done by myself jointly with others, I have made clear exactly what was done by others and what I have contributed myself;\\
7. Either none of this work has been published before submission, or parts of this work have been published by :\\
\\
Stefan Collier\\
April 2016
}
\tableofcontents
\listoffigures
\listoftables

\mainmatter
%% ----------------------------------------------------------------
%\include{Introduction}
%\include{Conclusions}
\include{chapters/1Project/main}
\include{chapters/2Lit/main}
\include{chapters/3Design/HighLevel}
\include{chapters/3Design/InDepth}
\include{chapters/4Impl/main}

\include{chapters/5Experiments/1/main}
\include{chapters/5Experiments/2/main}
\include{chapters/5Experiments/3/main}
\include{chapters/5Experiments/4/main}

\include{chapters/6Conclusion/main}

\appendix
\include{appendix/AppendixB}
\include{appendix/D/main}
\include{appendix/AppendixC}

\backmatter
\bibliographystyle{ecs}
\bibliography{ECS}
\end{document}
%% ----------------------------------------------------------------

\include{appendix/AppendixC}

\backmatter
\bibliographystyle{ecs}
\bibliography{ECS}
\end{document}
%% ----------------------------------------------------------------

\include{chapters/3Design/HighLevel}
\include{chapters/3Design/InDepth}
 %% ----------------------------------------------------------------
%% Progress.tex
%% ---------------------------------------------------------------- 
\documentclass{ecsprogress}    % Use the progress Style
\graphicspath{{../figs/}}   % Location of your graphics files
    \usepackage{natbib}            % Use Natbib style for the refs.
\hypersetup{colorlinks=true}   % Set to false for black/white printing
\input{Definitions}            % Include your abbreviations



\usepackage{enumitem}% http://ctan.org/pkg/enumitem
\usepackage{multirow}
\usepackage{float}
\usepackage{amsmath}
\usepackage{multicol}
\usepackage{amssymb}
\usepackage[normalem]{ulem}
\useunder{\uline}{\ul}{}
\usepackage{wrapfig}


\usepackage[table,xcdraw]{xcolor}


%% ----------------------------------------------------------------
\begin{document}
\frontmatter
\title      {Heterogeneous Agent-based Model for Supermarket Competition}
\authors    {\texorpdfstring
             {\href{mailto:sc22g13@ecs.soton.ac.uk}{Stefan J. Collier}}
             {Stefan J. Collier}
            }
\addresses  {\groupname\\\deptname\\\univname}
\date       {\today}
\subject    {}
\keywords   {}
\supervisor {Dr. Maria Polukarov}
\examiner   {Professor Sheng Chen}

\maketitle
\begin{abstract}
This project aim was to model and analyse the effects of competitive pricing behaviors of grocery retailers on the British market. 

This was achieved by creating a multi-agent model, containing retailer and consumer agents. The heterogeneous crowd of retailers employs either a uniform pricing strategy or a ‘local price flexing’ strategy. The actions of these retailers are chosen by predicting the profit of each action, using a perceptron. Following on from the consideration of different economic models, a discrete model was developed so that software agents have a discrete environment to operate within. Within the model, it has been observed how supermarkets with differing behaviors affect a heterogeneous crowd of consumer agents. The model was implemented in Java with Python used to evaluate the results. 

The simulation displays good acceptance with real grocery market behavior, i.e. captures the performance of British retailers thus can be used to determine the impact of changes in their behavior on their competitors and consumers.Furthermore it can be used to provide insight into sustainability of volatile pricing strategies, providing a useful insight in volatility of British supermarket retail industry. 
\end{abstract}
\acknowledgements{
I would like to express my sincere gratitude to Dr Maria Polukarov for her guidance and support which provided me the freedom to take this research in the direction of my interest.\\
\\
I would also like to thank my family and friends for their encouragement and support. To those who quietly listened to my software complaints. To those who worked throughout the nights with me. To those who helped me write what I couldn't say. I cannot thank you enough.
}

\declaration{
I, Stefan Collier, declare that this dissertation and the work presented in it are my own and has been generated by me as the result of my own original research.\\
I confirm that:\\
1. This work was done wholly or mainly while in candidature for a degree at this University;\\
2. Where any part of this dissertation has previously been submitted for any other qualification at this University or any other institution, this has been clearly stated;\\
3. Where I have consulted the published work of others, this is always clearly attributed;\\
4. Where I have quoted from the work of others, the source is always given. With the exception of such quotations, this dissertation is entirely my own work;\\
5. I have acknowledged all main sources of help;\\
6. Where the thesis is based on work done by myself jointly with others, I have made clear exactly what was done by others and what I have contributed myself;\\
7. Either none of this work has been published before submission, or parts of this work have been published by :\\
\\
Stefan Collier\\
April 2016
}
\tableofcontents
\listoffigures
\listoftables

\mainmatter
%% ----------------------------------------------------------------
%\include{Introduction}
%\include{Conclusions}
 %% ----------------------------------------------------------------
%% Progress.tex
%% ---------------------------------------------------------------- 
\documentclass{ecsprogress}    % Use the progress Style
\graphicspath{{../figs/}}   % Location of your graphics files
    \usepackage{natbib}            % Use Natbib style for the refs.
\hypersetup{colorlinks=true}   % Set to false for black/white printing
\input{Definitions}            % Include your abbreviations



\usepackage{enumitem}% http://ctan.org/pkg/enumitem
\usepackage{multirow}
\usepackage{float}
\usepackage{amsmath}
\usepackage{multicol}
\usepackage{amssymb}
\usepackage[normalem]{ulem}
\useunder{\uline}{\ul}{}
\usepackage{wrapfig}


\usepackage[table,xcdraw]{xcolor}


%% ----------------------------------------------------------------
\begin{document}
\frontmatter
\title      {Heterogeneous Agent-based Model for Supermarket Competition}
\authors    {\texorpdfstring
             {\href{mailto:sc22g13@ecs.soton.ac.uk}{Stefan J. Collier}}
             {Stefan J. Collier}
            }
\addresses  {\groupname\\\deptname\\\univname}
\date       {\today}
\subject    {}
\keywords   {}
\supervisor {Dr. Maria Polukarov}
\examiner   {Professor Sheng Chen}

\maketitle
\begin{abstract}
This project aim was to model and analyse the effects of competitive pricing behaviors of grocery retailers on the British market. 

This was achieved by creating a multi-agent model, containing retailer and consumer agents. The heterogeneous crowd of retailers employs either a uniform pricing strategy or a ‘local price flexing’ strategy. The actions of these retailers are chosen by predicting the profit of each action, using a perceptron. Following on from the consideration of different economic models, a discrete model was developed so that software agents have a discrete environment to operate within. Within the model, it has been observed how supermarkets with differing behaviors affect a heterogeneous crowd of consumer agents. The model was implemented in Java with Python used to evaluate the results. 

The simulation displays good acceptance with real grocery market behavior, i.e. captures the performance of British retailers thus can be used to determine the impact of changes in their behavior on their competitors and consumers.Furthermore it can be used to provide insight into sustainability of volatile pricing strategies, providing a useful insight in volatility of British supermarket retail industry. 
\end{abstract}
\acknowledgements{
I would like to express my sincere gratitude to Dr Maria Polukarov for her guidance and support which provided me the freedom to take this research in the direction of my interest.\\
\\
I would also like to thank my family and friends for their encouragement and support. To those who quietly listened to my software complaints. To those who worked throughout the nights with me. To those who helped me write what I couldn't say. I cannot thank you enough.
}

\declaration{
I, Stefan Collier, declare that this dissertation and the work presented in it are my own and has been generated by me as the result of my own original research.\\
I confirm that:\\
1. This work was done wholly or mainly while in candidature for a degree at this University;\\
2. Where any part of this dissertation has previously been submitted for any other qualification at this University or any other institution, this has been clearly stated;\\
3. Where I have consulted the published work of others, this is always clearly attributed;\\
4. Where I have quoted from the work of others, the source is always given. With the exception of such quotations, this dissertation is entirely my own work;\\
5. I have acknowledged all main sources of help;\\
6. Where the thesis is based on work done by myself jointly with others, I have made clear exactly what was done by others and what I have contributed myself;\\
7. Either none of this work has been published before submission, or parts of this work have been published by :\\
\\
Stefan Collier\\
April 2016
}
\tableofcontents
\listoffigures
\listoftables

\mainmatter
%% ----------------------------------------------------------------
%\include{Introduction}
%\include{Conclusions}
\include{chapters/1Project/main}
\include{chapters/2Lit/main}
\include{chapters/3Design/HighLevel}
\include{chapters/3Design/InDepth}
\include{chapters/4Impl/main}

\include{chapters/5Experiments/1/main}
\include{chapters/5Experiments/2/main}
\include{chapters/5Experiments/3/main}
\include{chapters/5Experiments/4/main}

\include{chapters/6Conclusion/main}

\appendix
\include{appendix/AppendixB}
\include{appendix/D/main}
\include{appendix/AppendixC}

\backmatter
\bibliographystyle{ecs}
\bibliography{ECS}
\end{document}
%% ----------------------------------------------------------------

 %% ----------------------------------------------------------------
%% Progress.tex
%% ---------------------------------------------------------------- 
\documentclass{ecsprogress}    % Use the progress Style
\graphicspath{{../figs/}}   % Location of your graphics files
    \usepackage{natbib}            % Use Natbib style for the refs.
\hypersetup{colorlinks=true}   % Set to false for black/white printing
\input{Definitions}            % Include your abbreviations



\usepackage{enumitem}% http://ctan.org/pkg/enumitem
\usepackage{multirow}
\usepackage{float}
\usepackage{amsmath}
\usepackage{multicol}
\usepackage{amssymb}
\usepackage[normalem]{ulem}
\useunder{\uline}{\ul}{}
\usepackage{wrapfig}


\usepackage[table,xcdraw]{xcolor}


%% ----------------------------------------------------------------
\begin{document}
\frontmatter
\title      {Heterogeneous Agent-based Model for Supermarket Competition}
\authors    {\texorpdfstring
             {\href{mailto:sc22g13@ecs.soton.ac.uk}{Stefan J. Collier}}
             {Stefan J. Collier}
            }
\addresses  {\groupname\\\deptname\\\univname}
\date       {\today}
\subject    {}
\keywords   {}
\supervisor {Dr. Maria Polukarov}
\examiner   {Professor Sheng Chen}

\maketitle
\begin{abstract}
This project aim was to model and analyse the effects of competitive pricing behaviors of grocery retailers on the British market. 

This was achieved by creating a multi-agent model, containing retailer and consumer agents. The heterogeneous crowd of retailers employs either a uniform pricing strategy or a ‘local price flexing’ strategy. The actions of these retailers are chosen by predicting the profit of each action, using a perceptron. Following on from the consideration of different economic models, a discrete model was developed so that software agents have a discrete environment to operate within. Within the model, it has been observed how supermarkets with differing behaviors affect a heterogeneous crowd of consumer agents. The model was implemented in Java with Python used to evaluate the results. 

The simulation displays good acceptance with real grocery market behavior, i.e. captures the performance of British retailers thus can be used to determine the impact of changes in their behavior on their competitors and consumers.Furthermore it can be used to provide insight into sustainability of volatile pricing strategies, providing a useful insight in volatility of British supermarket retail industry. 
\end{abstract}
\acknowledgements{
I would like to express my sincere gratitude to Dr Maria Polukarov for her guidance and support which provided me the freedom to take this research in the direction of my interest.\\
\\
I would also like to thank my family and friends for their encouragement and support. To those who quietly listened to my software complaints. To those who worked throughout the nights with me. To those who helped me write what I couldn't say. I cannot thank you enough.
}

\declaration{
I, Stefan Collier, declare that this dissertation and the work presented in it are my own and has been generated by me as the result of my own original research.\\
I confirm that:\\
1. This work was done wholly or mainly while in candidature for a degree at this University;\\
2. Where any part of this dissertation has previously been submitted for any other qualification at this University or any other institution, this has been clearly stated;\\
3. Where I have consulted the published work of others, this is always clearly attributed;\\
4. Where I have quoted from the work of others, the source is always given. With the exception of such quotations, this dissertation is entirely my own work;\\
5. I have acknowledged all main sources of help;\\
6. Where the thesis is based on work done by myself jointly with others, I have made clear exactly what was done by others and what I have contributed myself;\\
7. Either none of this work has been published before submission, or parts of this work have been published by :\\
\\
Stefan Collier\\
April 2016
}
\tableofcontents
\listoffigures
\listoftables

\mainmatter
%% ----------------------------------------------------------------
%\include{Introduction}
%\include{Conclusions}
\include{chapters/1Project/main}
\include{chapters/2Lit/main}
\include{chapters/3Design/HighLevel}
\include{chapters/3Design/InDepth}
\include{chapters/4Impl/main}

\include{chapters/5Experiments/1/main}
\include{chapters/5Experiments/2/main}
\include{chapters/5Experiments/3/main}
\include{chapters/5Experiments/4/main}

\include{chapters/6Conclusion/main}

\appendix
\include{appendix/AppendixB}
\include{appendix/D/main}
\include{appendix/AppendixC}

\backmatter
\bibliographystyle{ecs}
\bibliography{ECS}
\end{document}
%% ----------------------------------------------------------------

\include{chapters/3Design/HighLevel}
\include{chapters/3Design/InDepth}
 %% ----------------------------------------------------------------
%% Progress.tex
%% ---------------------------------------------------------------- 
\documentclass{ecsprogress}    % Use the progress Style
\graphicspath{{../figs/}}   % Location of your graphics files
    \usepackage{natbib}            % Use Natbib style for the refs.
\hypersetup{colorlinks=true}   % Set to false for black/white printing
\input{Definitions}            % Include your abbreviations



\usepackage{enumitem}% http://ctan.org/pkg/enumitem
\usepackage{multirow}
\usepackage{float}
\usepackage{amsmath}
\usepackage{multicol}
\usepackage{amssymb}
\usepackage[normalem]{ulem}
\useunder{\uline}{\ul}{}
\usepackage{wrapfig}


\usepackage[table,xcdraw]{xcolor}


%% ----------------------------------------------------------------
\begin{document}
\frontmatter
\title      {Heterogeneous Agent-based Model for Supermarket Competition}
\authors    {\texorpdfstring
             {\href{mailto:sc22g13@ecs.soton.ac.uk}{Stefan J. Collier}}
             {Stefan J. Collier}
            }
\addresses  {\groupname\\\deptname\\\univname}
\date       {\today}
\subject    {}
\keywords   {}
\supervisor {Dr. Maria Polukarov}
\examiner   {Professor Sheng Chen}

\maketitle
\begin{abstract}
This project aim was to model and analyse the effects of competitive pricing behaviors of grocery retailers on the British market. 

This was achieved by creating a multi-agent model, containing retailer and consumer agents. The heterogeneous crowd of retailers employs either a uniform pricing strategy or a ‘local price flexing’ strategy. The actions of these retailers are chosen by predicting the profit of each action, using a perceptron. Following on from the consideration of different economic models, a discrete model was developed so that software agents have a discrete environment to operate within. Within the model, it has been observed how supermarkets with differing behaviors affect a heterogeneous crowd of consumer agents. The model was implemented in Java with Python used to evaluate the results. 

The simulation displays good acceptance with real grocery market behavior, i.e. captures the performance of British retailers thus can be used to determine the impact of changes in their behavior on their competitors and consumers.Furthermore it can be used to provide insight into sustainability of volatile pricing strategies, providing a useful insight in volatility of British supermarket retail industry. 
\end{abstract}
\acknowledgements{
I would like to express my sincere gratitude to Dr Maria Polukarov for her guidance and support which provided me the freedom to take this research in the direction of my interest.\\
\\
I would also like to thank my family and friends for their encouragement and support. To those who quietly listened to my software complaints. To those who worked throughout the nights with me. To those who helped me write what I couldn't say. I cannot thank you enough.
}

\declaration{
I, Stefan Collier, declare that this dissertation and the work presented in it are my own and has been generated by me as the result of my own original research.\\
I confirm that:\\
1. This work was done wholly or mainly while in candidature for a degree at this University;\\
2. Where any part of this dissertation has previously been submitted for any other qualification at this University or any other institution, this has been clearly stated;\\
3. Where I have consulted the published work of others, this is always clearly attributed;\\
4. Where I have quoted from the work of others, the source is always given. With the exception of such quotations, this dissertation is entirely my own work;\\
5. I have acknowledged all main sources of help;\\
6. Where the thesis is based on work done by myself jointly with others, I have made clear exactly what was done by others and what I have contributed myself;\\
7. Either none of this work has been published before submission, or parts of this work have been published by :\\
\\
Stefan Collier\\
April 2016
}
\tableofcontents
\listoffigures
\listoftables

\mainmatter
%% ----------------------------------------------------------------
%\include{Introduction}
%\include{Conclusions}
\include{chapters/1Project/main}
\include{chapters/2Lit/main}
\include{chapters/3Design/HighLevel}
\include{chapters/3Design/InDepth}
\include{chapters/4Impl/main}

\include{chapters/5Experiments/1/main}
\include{chapters/5Experiments/2/main}
\include{chapters/5Experiments/3/main}
\include{chapters/5Experiments/4/main}

\include{chapters/6Conclusion/main}

\appendix
\include{appendix/AppendixB}
\include{appendix/D/main}
\include{appendix/AppendixC}

\backmatter
\bibliographystyle{ecs}
\bibliography{ECS}
\end{document}
%% ----------------------------------------------------------------


 %% ----------------------------------------------------------------
%% Progress.tex
%% ---------------------------------------------------------------- 
\documentclass{ecsprogress}    % Use the progress Style
\graphicspath{{../figs/}}   % Location of your graphics files
    \usepackage{natbib}            % Use Natbib style for the refs.
\hypersetup{colorlinks=true}   % Set to false for black/white printing
\input{Definitions}            % Include your abbreviations



\usepackage{enumitem}% http://ctan.org/pkg/enumitem
\usepackage{multirow}
\usepackage{float}
\usepackage{amsmath}
\usepackage{multicol}
\usepackage{amssymb}
\usepackage[normalem]{ulem}
\useunder{\uline}{\ul}{}
\usepackage{wrapfig}


\usepackage[table,xcdraw]{xcolor}


%% ----------------------------------------------------------------
\begin{document}
\frontmatter
\title      {Heterogeneous Agent-based Model for Supermarket Competition}
\authors    {\texorpdfstring
             {\href{mailto:sc22g13@ecs.soton.ac.uk}{Stefan J. Collier}}
             {Stefan J. Collier}
            }
\addresses  {\groupname\\\deptname\\\univname}
\date       {\today}
\subject    {}
\keywords   {}
\supervisor {Dr. Maria Polukarov}
\examiner   {Professor Sheng Chen}

\maketitle
\begin{abstract}
This project aim was to model and analyse the effects of competitive pricing behaviors of grocery retailers on the British market. 

This was achieved by creating a multi-agent model, containing retailer and consumer agents. The heterogeneous crowd of retailers employs either a uniform pricing strategy or a ‘local price flexing’ strategy. The actions of these retailers are chosen by predicting the profit of each action, using a perceptron. Following on from the consideration of different economic models, a discrete model was developed so that software agents have a discrete environment to operate within. Within the model, it has been observed how supermarkets with differing behaviors affect a heterogeneous crowd of consumer agents. The model was implemented in Java with Python used to evaluate the results. 

The simulation displays good acceptance with real grocery market behavior, i.e. captures the performance of British retailers thus can be used to determine the impact of changes in their behavior on their competitors and consumers.Furthermore it can be used to provide insight into sustainability of volatile pricing strategies, providing a useful insight in volatility of British supermarket retail industry. 
\end{abstract}
\acknowledgements{
I would like to express my sincere gratitude to Dr Maria Polukarov for her guidance and support which provided me the freedom to take this research in the direction of my interest.\\
\\
I would also like to thank my family and friends for their encouragement and support. To those who quietly listened to my software complaints. To those who worked throughout the nights with me. To those who helped me write what I couldn't say. I cannot thank you enough.
}

\declaration{
I, Stefan Collier, declare that this dissertation and the work presented in it are my own and has been generated by me as the result of my own original research.\\
I confirm that:\\
1. This work was done wholly or mainly while in candidature for a degree at this University;\\
2. Where any part of this dissertation has previously been submitted for any other qualification at this University or any other institution, this has been clearly stated;\\
3. Where I have consulted the published work of others, this is always clearly attributed;\\
4. Where I have quoted from the work of others, the source is always given. With the exception of such quotations, this dissertation is entirely my own work;\\
5. I have acknowledged all main sources of help;\\
6. Where the thesis is based on work done by myself jointly with others, I have made clear exactly what was done by others and what I have contributed myself;\\
7. Either none of this work has been published before submission, or parts of this work have been published by :\\
\\
Stefan Collier\\
April 2016
}
\tableofcontents
\listoffigures
\listoftables

\mainmatter
%% ----------------------------------------------------------------
%\include{Introduction}
%\include{Conclusions}
\include{chapters/1Project/main}
\include{chapters/2Lit/main}
\include{chapters/3Design/HighLevel}
\include{chapters/3Design/InDepth}
\include{chapters/4Impl/main}

\include{chapters/5Experiments/1/main}
\include{chapters/5Experiments/2/main}
\include{chapters/5Experiments/3/main}
\include{chapters/5Experiments/4/main}

\include{chapters/6Conclusion/main}

\appendix
\include{appendix/AppendixB}
\include{appendix/D/main}
\include{appendix/AppendixC}

\backmatter
\bibliographystyle{ecs}
\bibliography{ECS}
\end{document}
%% ----------------------------------------------------------------

 %% ----------------------------------------------------------------
%% Progress.tex
%% ---------------------------------------------------------------- 
\documentclass{ecsprogress}    % Use the progress Style
\graphicspath{{../figs/}}   % Location of your graphics files
    \usepackage{natbib}            % Use Natbib style for the refs.
\hypersetup{colorlinks=true}   % Set to false for black/white printing
\input{Definitions}            % Include your abbreviations



\usepackage{enumitem}% http://ctan.org/pkg/enumitem
\usepackage{multirow}
\usepackage{float}
\usepackage{amsmath}
\usepackage{multicol}
\usepackage{amssymb}
\usepackage[normalem]{ulem}
\useunder{\uline}{\ul}{}
\usepackage{wrapfig}


\usepackage[table,xcdraw]{xcolor}


%% ----------------------------------------------------------------
\begin{document}
\frontmatter
\title      {Heterogeneous Agent-based Model for Supermarket Competition}
\authors    {\texorpdfstring
             {\href{mailto:sc22g13@ecs.soton.ac.uk}{Stefan J. Collier}}
             {Stefan J. Collier}
            }
\addresses  {\groupname\\\deptname\\\univname}
\date       {\today}
\subject    {}
\keywords   {}
\supervisor {Dr. Maria Polukarov}
\examiner   {Professor Sheng Chen}

\maketitle
\begin{abstract}
This project aim was to model and analyse the effects of competitive pricing behaviors of grocery retailers on the British market. 

This was achieved by creating a multi-agent model, containing retailer and consumer agents. The heterogeneous crowd of retailers employs either a uniform pricing strategy or a ‘local price flexing’ strategy. The actions of these retailers are chosen by predicting the profit of each action, using a perceptron. Following on from the consideration of different economic models, a discrete model was developed so that software agents have a discrete environment to operate within. Within the model, it has been observed how supermarkets with differing behaviors affect a heterogeneous crowd of consumer agents. The model was implemented in Java with Python used to evaluate the results. 

The simulation displays good acceptance with real grocery market behavior, i.e. captures the performance of British retailers thus can be used to determine the impact of changes in their behavior on their competitors and consumers.Furthermore it can be used to provide insight into sustainability of volatile pricing strategies, providing a useful insight in volatility of British supermarket retail industry. 
\end{abstract}
\acknowledgements{
I would like to express my sincere gratitude to Dr Maria Polukarov for her guidance and support which provided me the freedom to take this research in the direction of my interest.\\
\\
I would also like to thank my family and friends for their encouragement and support. To those who quietly listened to my software complaints. To those who worked throughout the nights with me. To those who helped me write what I couldn't say. I cannot thank you enough.
}

\declaration{
I, Stefan Collier, declare that this dissertation and the work presented in it are my own and has been generated by me as the result of my own original research.\\
I confirm that:\\
1. This work was done wholly or mainly while in candidature for a degree at this University;\\
2. Where any part of this dissertation has previously been submitted for any other qualification at this University or any other institution, this has been clearly stated;\\
3. Where I have consulted the published work of others, this is always clearly attributed;\\
4. Where I have quoted from the work of others, the source is always given. With the exception of such quotations, this dissertation is entirely my own work;\\
5. I have acknowledged all main sources of help;\\
6. Where the thesis is based on work done by myself jointly with others, I have made clear exactly what was done by others and what I have contributed myself;\\
7. Either none of this work has been published before submission, or parts of this work have been published by :\\
\\
Stefan Collier\\
April 2016
}
\tableofcontents
\listoffigures
\listoftables

\mainmatter
%% ----------------------------------------------------------------
%\include{Introduction}
%\include{Conclusions}
\include{chapters/1Project/main}
\include{chapters/2Lit/main}
\include{chapters/3Design/HighLevel}
\include{chapters/3Design/InDepth}
\include{chapters/4Impl/main}

\include{chapters/5Experiments/1/main}
\include{chapters/5Experiments/2/main}
\include{chapters/5Experiments/3/main}
\include{chapters/5Experiments/4/main}

\include{chapters/6Conclusion/main}

\appendix
\include{appendix/AppendixB}
\include{appendix/D/main}
\include{appendix/AppendixC}

\backmatter
\bibliographystyle{ecs}
\bibliography{ECS}
\end{document}
%% ----------------------------------------------------------------

 %% ----------------------------------------------------------------
%% Progress.tex
%% ---------------------------------------------------------------- 
\documentclass{ecsprogress}    % Use the progress Style
\graphicspath{{../figs/}}   % Location of your graphics files
    \usepackage{natbib}            % Use Natbib style for the refs.
\hypersetup{colorlinks=true}   % Set to false for black/white printing
\input{Definitions}            % Include your abbreviations



\usepackage{enumitem}% http://ctan.org/pkg/enumitem
\usepackage{multirow}
\usepackage{float}
\usepackage{amsmath}
\usepackage{multicol}
\usepackage{amssymb}
\usepackage[normalem]{ulem}
\useunder{\uline}{\ul}{}
\usepackage{wrapfig}


\usepackage[table,xcdraw]{xcolor}


%% ----------------------------------------------------------------
\begin{document}
\frontmatter
\title      {Heterogeneous Agent-based Model for Supermarket Competition}
\authors    {\texorpdfstring
             {\href{mailto:sc22g13@ecs.soton.ac.uk}{Stefan J. Collier}}
             {Stefan J. Collier}
            }
\addresses  {\groupname\\\deptname\\\univname}
\date       {\today}
\subject    {}
\keywords   {}
\supervisor {Dr. Maria Polukarov}
\examiner   {Professor Sheng Chen}

\maketitle
\begin{abstract}
This project aim was to model and analyse the effects of competitive pricing behaviors of grocery retailers on the British market. 

This was achieved by creating a multi-agent model, containing retailer and consumer agents. The heterogeneous crowd of retailers employs either a uniform pricing strategy or a ‘local price flexing’ strategy. The actions of these retailers are chosen by predicting the profit of each action, using a perceptron. Following on from the consideration of different economic models, a discrete model was developed so that software agents have a discrete environment to operate within. Within the model, it has been observed how supermarkets with differing behaviors affect a heterogeneous crowd of consumer agents. The model was implemented in Java with Python used to evaluate the results. 

The simulation displays good acceptance with real grocery market behavior, i.e. captures the performance of British retailers thus can be used to determine the impact of changes in their behavior on their competitors and consumers.Furthermore it can be used to provide insight into sustainability of volatile pricing strategies, providing a useful insight in volatility of British supermarket retail industry. 
\end{abstract}
\acknowledgements{
I would like to express my sincere gratitude to Dr Maria Polukarov for her guidance and support which provided me the freedom to take this research in the direction of my interest.\\
\\
I would also like to thank my family and friends for their encouragement and support. To those who quietly listened to my software complaints. To those who worked throughout the nights with me. To those who helped me write what I couldn't say. I cannot thank you enough.
}

\declaration{
I, Stefan Collier, declare that this dissertation and the work presented in it are my own and has been generated by me as the result of my own original research.\\
I confirm that:\\
1. This work was done wholly or mainly while in candidature for a degree at this University;\\
2. Where any part of this dissertation has previously been submitted for any other qualification at this University or any other institution, this has been clearly stated;\\
3. Where I have consulted the published work of others, this is always clearly attributed;\\
4. Where I have quoted from the work of others, the source is always given. With the exception of such quotations, this dissertation is entirely my own work;\\
5. I have acknowledged all main sources of help;\\
6. Where the thesis is based on work done by myself jointly with others, I have made clear exactly what was done by others and what I have contributed myself;\\
7. Either none of this work has been published before submission, or parts of this work have been published by :\\
\\
Stefan Collier\\
April 2016
}
\tableofcontents
\listoffigures
\listoftables

\mainmatter
%% ----------------------------------------------------------------
%\include{Introduction}
%\include{Conclusions}
\include{chapters/1Project/main}
\include{chapters/2Lit/main}
\include{chapters/3Design/HighLevel}
\include{chapters/3Design/InDepth}
\include{chapters/4Impl/main}

\include{chapters/5Experiments/1/main}
\include{chapters/5Experiments/2/main}
\include{chapters/5Experiments/3/main}
\include{chapters/5Experiments/4/main}

\include{chapters/6Conclusion/main}

\appendix
\include{appendix/AppendixB}
\include{appendix/D/main}
\include{appendix/AppendixC}

\backmatter
\bibliographystyle{ecs}
\bibliography{ECS}
\end{document}
%% ----------------------------------------------------------------

 %% ----------------------------------------------------------------
%% Progress.tex
%% ---------------------------------------------------------------- 
\documentclass{ecsprogress}    % Use the progress Style
\graphicspath{{../figs/}}   % Location of your graphics files
    \usepackage{natbib}            % Use Natbib style for the refs.
\hypersetup{colorlinks=true}   % Set to false for black/white printing
\input{Definitions}            % Include your abbreviations



\usepackage{enumitem}% http://ctan.org/pkg/enumitem
\usepackage{multirow}
\usepackage{float}
\usepackage{amsmath}
\usepackage{multicol}
\usepackage{amssymb}
\usepackage[normalem]{ulem}
\useunder{\uline}{\ul}{}
\usepackage{wrapfig}


\usepackage[table,xcdraw]{xcolor}


%% ----------------------------------------------------------------
\begin{document}
\frontmatter
\title      {Heterogeneous Agent-based Model for Supermarket Competition}
\authors    {\texorpdfstring
             {\href{mailto:sc22g13@ecs.soton.ac.uk}{Stefan J. Collier}}
             {Stefan J. Collier}
            }
\addresses  {\groupname\\\deptname\\\univname}
\date       {\today}
\subject    {}
\keywords   {}
\supervisor {Dr. Maria Polukarov}
\examiner   {Professor Sheng Chen}

\maketitle
\begin{abstract}
This project aim was to model and analyse the effects of competitive pricing behaviors of grocery retailers on the British market. 

This was achieved by creating a multi-agent model, containing retailer and consumer agents. The heterogeneous crowd of retailers employs either a uniform pricing strategy or a ‘local price flexing’ strategy. The actions of these retailers are chosen by predicting the profit of each action, using a perceptron. Following on from the consideration of different economic models, a discrete model was developed so that software agents have a discrete environment to operate within. Within the model, it has been observed how supermarkets with differing behaviors affect a heterogeneous crowd of consumer agents. The model was implemented in Java with Python used to evaluate the results. 

The simulation displays good acceptance with real grocery market behavior, i.e. captures the performance of British retailers thus can be used to determine the impact of changes in their behavior on their competitors and consumers.Furthermore it can be used to provide insight into sustainability of volatile pricing strategies, providing a useful insight in volatility of British supermarket retail industry. 
\end{abstract}
\acknowledgements{
I would like to express my sincere gratitude to Dr Maria Polukarov for her guidance and support which provided me the freedom to take this research in the direction of my interest.\\
\\
I would also like to thank my family and friends for their encouragement and support. To those who quietly listened to my software complaints. To those who worked throughout the nights with me. To those who helped me write what I couldn't say. I cannot thank you enough.
}

\declaration{
I, Stefan Collier, declare that this dissertation and the work presented in it are my own and has been generated by me as the result of my own original research.\\
I confirm that:\\
1. This work was done wholly or mainly while in candidature for a degree at this University;\\
2. Where any part of this dissertation has previously been submitted for any other qualification at this University or any other institution, this has been clearly stated;\\
3. Where I have consulted the published work of others, this is always clearly attributed;\\
4. Where I have quoted from the work of others, the source is always given. With the exception of such quotations, this dissertation is entirely my own work;\\
5. I have acknowledged all main sources of help;\\
6. Where the thesis is based on work done by myself jointly with others, I have made clear exactly what was done by others and what I have contributed myself;\\
7. Either none of this work has been published before submission, or parts of this work have been published by :\\
\\
Stefan Collier\\
April 2016
}
\tableofcontents
\listoffigures
\listoftables

\mainmatter
%% ----------------------------------------------------------------
%\include{Introduction}
%\include{Conclusions}
\include{chapters/1Project/main}
\include{chapters/2Lit/main}
\include{chapters/3Design/HighLevel}
\include{chapters/3Design/InDepth}
\include{chapters/4Impl/main}

\include{chapters/5Experiments/1/main}
\include{chapters/5Experiments/2/main}
\include{chapters/5Experiments/3/main}
\include{chapters/5Experiments/4/main}

\include{chapters/6Conclusion/main}

\appendix
\include{appendix/AppendixB}
\include{appendix/D/main}
\include{appendix/AppendixC}

\backmatter
\bibliographystyle{ecs}
\bibliography{ECS}
\end{document}
%% ----------------------------------------------------------------


 %% ----------------------------------------------------------------
%% Progress.tex
%% ---------------------------------------------------------------- 
\documentclass{ecsprogress}    % Use the progress Style
\graphicspath{{../figs/}}   % Location of your graphics files
    \usepackage{natbib}            % Use Natbib style for the refs.
\hypersetup{colorlinks=true}   % Set to false for black/white printing
\input{Definitions}            % Include your abbreviations



\usepackage{enumitem}% http://ctan.org/pkg/enumitem
\usepackage{multirow}
\usepackage{float}
\usepackage{amsmath}
\usepackage{multicol}
\usepackage{amssymb}
\usepackage[normalem]{ulem}
\useunder{\uline}{\ul}{}
\usepackage{wrapfig}


\usepackage[table,xcdraw]{xcolor}


%% ----------------------------------------------------------------
\begin{document}
\frontmatter
\title      {Heterogeneous Agent-based Model for Supermarket Competition}
\authors    {\texorpdfstring
             {\href{mailto:sc22g13@ecs.soton.ac.uk}{Stefan J. Collier}}
             {Stefan J. Collier}
            }
\addresses  {\groupname\\\deptname\\\univname}
\date       {\today}
\subject    {}
\keywords   {}
\supervisor {Dr. Maria Polukarov}
\examiner   {Professor Sheng Chen}

\maketitle
\begin{abstract}
This project aim was to model and analyse the effects of competitive pricing behaviors of grocery retailers on the British market. 

This was achieved by creating a multi-agent model, containing retailer and consumer agents. The heterogeneous crowd of retailers employs either a uniform pricing strategy or a ‘local price flexing’ strategy. The actions of these retailers are chosen by predicting the profit of each action, using a perceptron. Following on from the consideration of different economic models, a discrete model was developed so that software agents have a discrete environment to operate within. Within the model, it has been observed how supermarkets with differing behaviors affect a heterogeneous crowd of consumer agents. The model was implemented in Java with Python used to evaluate the results. 

The simulation displays good acceptance with real grocery market behavior, i.e. captures the performance of British retailers thus can be used to determine the impact of changes in their behavior on their competitors and consumers.Furthermore it can be used to provide insight into sustainability of volatile pricing strategies, providing a useful insight in volatility of British supermarket retail industry. 
\end{abstract}
\acknowledgements{
I would like to express my sincere gratitude to Dr Maria Polukarov for her guidance and support which provided me the freedom to take this research in the direction of my interest.\\
\\
I would also like to thank my family and friends for their encouragement and support. To those who quietly listened to my software complaints. To those who worked throughout the nights with me. To those who helped me write what I couldn't say. I cannot thank you enough.
}

\declaration{
I, Stefan Collier, declare that this dissertation and the work presented in it are my own and has been generated by me as the result of my own original research.\\
I confirm that:\\
1. This work was done wholly or mainly while in candidature for a degree at this University;\\
2. Where any part of this dissertation has previously been submitted for any other qualification at this University or any other institution, this has been clearly stated;\\
3. Where I have consulted the published work of others, this is always clearly attributed;\\
4. Where I have quoted from the work of others, the source is always given. With the exception of such quotations, this dissertation is entirely my own work;\\
5. I have acknowledged all main sources of help;\\
6. Where the thesis is based on work done by myself jointly with others, I have made clear exactly what was done by others and what I have contributed myself;\\
7. Either none of this work has been published before submission, or parts of this work have been published by :\\
\\
Stefan Collier\\
April 2016
}
\tableofcontents
\listoffigures
\listoftables

\mainmatter
%% ----------------------------------------------------------------
%\include{Introduction}
%\include{Conclusions}
\include{chapters/1Project/main}
\include{chapters/2Lit/main}
\include{chapters/3Design/HighLevel}
\include{chapters/3Design/InDepth}
\include{chapters/4Impl/main}

\include{chapters/5Experiments/1/main}
\include{chapters/5Experiments/2/main}
\include{chapters/5Experiments/3/main}
\include{chapters/5Experiments/4/main}

\include{chapters/6Conclusion/main}

\appendix
\include{appendix/AppendixB}
\include{appendix/D/main}
\include{appendix/AppendixC}

\backmatter
\bibliographystyle{ecs}
\bibliography{ECS}
\end{document}
%% ----------------------------------------------------------------


\appendix
\include{appendix/AppendixB}
 %% ----------------------------------------------------------------
%% Progress.tex
%% ---------------------------------------------------------------- 
\documentclass{ecsprogress}    % Use the progress Style
\graphicspath{{../figs/}}   % Location of your graphics files
    \usepackage{natbib}            % Use Natbib style for the refs.
\hypersetup{colorlinks=true}   % Set to false for black/white printing
\input{Definitions}            % Include your abbreviations



\usepackage{enumitem}% http://ctan.org/pkg/enumitem
\usepackage{multirow}
\usepackage{float}
\usepackage{amsmath}
\usepackage{multicol}
\usepackage{amssymb}
\usepackage[normalem]{ulem}
\useunder{\uline}{\ul}{}
\usepackage{wrapfig}


\usepackage[table,xcdraw]{xcolor}


%% ----------------------------------------------------------------
\begin{document}
\frontmatter
\title      {Heterogeneous Agent-based Model for Supermarket Competition}
\authors    {\texorpdfstring
             {\href{mailto:sc22g13@ecs.soton.ac.uk}{Stefan J. Collier}}
             {Stefan J. Collier}
            }
\addresses  {\groupname\\\deptname\\\univname}
\date       {\today}
\subject    {}
\keywords   {}
\supervisor {Dr. Maria Polukarov}
\examiner   {Professor Sheng Chen}

\maketitle
\begin{abstract}
This project aim was to model and analyse the effects of competitive pricing behaviors of grocery retailers on the British market. 

This was achieved by creating a multi-agent model, containing retailer and consumer agents. The heterogeneous crowd of retailers employs either a uniform pricing strategy or a ‘local price flexing’ strategy. The actions of these retailers are chosen by predicting the profit of each action, using a perceptron. Following on from the consideration of different economic models, a discrete model was developed so that software agents have a discrete environment to operate within. Within the model, it has been observed how supermarkets with differing behaviors affect a heterogeneous crowd of consumer agents. The model was implemented in Java with Python used to evaluate the results. 

The simulation displays good acceptance with real grocery market behavior, i.e. captures the performance of British retailers thus can be used to determine the impact of changes in their behavior on their competitors and consumers.Furthermore it can be used to provide insight into sustainability of volatile pricing strategies, providing a useful insight in volatility of British supermarket retail industry. 
\end{abstract}
\acknowledgements{
I would like to express my sincere gratitude to Dr Maria Polukarov for her guidance and support which provided me the freedom to take this research in the direction of my interest.\\
\\
I would also like to thank my family and friends for their encouragement and support. To those who quietly listened to my software complaints. To those who worked throughout the nights with me. To those who helped me write what I couldn't say. I cannot thank you enough.
}

\declaration{
I, Stefan Collier, declare that this dissertation and the work presented in it are my own and has been generated by me as the result of my own original research.\\
I confirm that:\\
1. This work was done wholly or mainly while in candidature for a degree at this University;\\
2. Where any part of this dissertation has previously been submitted for any other qualification at this University or any other institution, this has been clearly stated;\\
3. Where I have consulted the published work of others, this is always clearly attributed;\\
4. Where I have quoted from the work of others, the source is always given. With the exception of such quotations, this dissertation is entirely my own work;\\
5. I have acknowledged all main sources of help;\\
6. Where the thesis is based on work done by myself jointly with others, I have made clear exactly what was done by others and what I have contributed myself;\\
7. Either none of this work has been published before submission, or parts of this work have been published by :\\
\\
Stefan Collier\\
April 2016
}
\tableofcontents
\listoffigures
\listoftables

\mainmatter
%% ----------------------------------------------------------------
%\include{Introduction}
%\include{Conclusions}
\include{chapters/1Project/main}
\include{chapters/2Lit/main}
\include{chapters/3Design/HighLevel}
\include{chapters/3Design/InDepth}
\include{chapters/4Impl/main}

\include{chapters/5Experiments/1/main}
\include{chapters/5Experiments/2/main}
\include{chapters/5Experiments/3/main}
\include{chapters/5Experiments/4/main}

\include{chapters/6Conclusion/main}

\appendix
\include{appendix/AppendixB}
\include{appendix/D/main}
\include{appendix/AppendixC}

\backmatter
\bibliographystyle{ecs}
\bibliography{ECS}
\end{document}
%% ----------------------------------------------------------------

\include{appendix/AppendixC}

\backmatter
\bibliographystyle{ecs}
\bibliography{ECS}
\end{document}
%% ----------------------------------------------------------------


 %% ----------------------------------------------------------------
%% Progress.tex
%% ---------------------------------------------------------------- 
\documentclass{ecsprogress}    % Use the progress Style
\graphicspath{{../figs/}}   % Location of your graphics files
    \usepackage{natbib}            % Use Natbib style for the refs.
\hypersetup{colorlinks=true}   % Set to false for black/white printing
\input{Definitions}            % Include your abbreviations



\usepackage{enumitem}% http://ctan.org/pkg/enumitem
\usepackage{multirow}
\usepackage{float}
\usepackage{amsmath}
\usepackage{multicol}
\usepackage{amssymb}
\usepackage[normalem]{ulem}
\useunder{\uline}{\ul}{}
\usepackage{wrapfig}


\usepackage[table,xcdraw]{xcolor}


%% ----------------------------------------------------------------
\begin{document}
\frontmatter
\title      {Heterogeneous Agent-based Model for Supermarket Competition}
\authors    {\texorpdfstring
             {\href{mailto:sc22g13@ecs.soton.ac.uk}{Stefan J. Collier}}
             {Stefan J. Collier}
            }
\addresses  {\groupname\\\deptname\\\univname}
\date       {\today}
\subject    {}
\keywords   {}
\supervisor {Dr. Maria Polukarov}
\examiner   {Professor Sheng Chen}

\maketitle
\begin{abstract}
This project aim was to model and analyse the effects of competitive pricing behaviors of grocery retailers on the British market. 

This was achieved by creating a multi-agent model, containing retailer and consumer agents. The heterogeneous crowd of retailers employs either a uniform pricing strategy or a ‘local price flexing’ strategy. The actions of these retailers are chosen by predicting the profit of each action, using a perceptron. Following on from the consideration of different economic models, a discrete model was developed so that software agents have a discrete environment to operate within. Within the model, it has been observed how supermarkets with differing behaviors affect a heterogeneous crowd of consumer agents. The model was implemented in Java with Python used to evaluate the results. 

The simulation displays good acceptance with real grocery market behavior, i.e. captures the performance of British retailers thus can be used to determine the impact of changes in their behavior on their competitors and consumers.Furthermore it can be used to provide insight into sustainability of volatile pricing strategies, providing a useful insight in volatility of British supermarket retail industry. 
\end{abstract}
\acknowledgements{
I would like to express my sincere gratitude to Dr Maria Polukarov for her guidance and support which provided me the freedom to take this research in the direction of my interest.\\
\\
I would also like to thank my family and friends for their encouragement and support. To those who quietly listened to my software complaints. To those who worked throughout the nights with me. To those who helped me write what I couldn't say. I cannot thank you enough.
}

\declaration{
I, Stefan Collier, declare that this dissertation and the work presented in it are my own and has been generated by me as the result of my own original research.\\
I confirm that:\\
1. This work was done wholly or mainly while in candidature for a degree at this University;\\
2. Where any part of this dissertation has previously been submitted for any other qualification at this University or any other institution, this has been clearly stated;\\
3. Where I have consulted the published work of others, this is always clearly attributed;\\
4. Where I have quoted from the work of others, the source is always given. With the exception of such quotations, this dissertation is entirely my own work;\\
5. I have acknowledged all main sources of help;\\
6. Where the thesis is based on work done by myself jointly with others, I have made clear exactly what was done by others and what I have contributed myself;\\
7. Either none of this work has been published before submission, or parts of this work have been published by :\\
\\
Stefan Collier\\
April 2016
}
\tableofcontents
\listoffigures
\listoftables

\mainmatter
%% ----------------------------------------------------------------
%\include{Introduction}
%\include{Conclusions}
 %% ----------------------------------------------------------------
%% Progress.tex
%% ---------------------------------------------------------------- 
\documentclass{ecsprogress}    % Use the progress Style
\graphicspath{{../figs/}}   % Location of your graphics files
    \usepackage{natbib}            % Use Natbib style for the refs.
\hypersetup{colorlinks=true}   % Set to false for black/white printing
\input{Definitions}            % Include your abbreviations



\usepackage{enumitem}% http://ctan.org/pkg/enumitem
\usepackage{multirow}
\usepackage{float}
\usepackage{amsmath}
\usepackage{multicol}
\usepackage{amssymb}
\usepackage[normalem]{ulem}
\useunder{\uline}{\ul}{}
\usepackage{wrapfig}


\usepackage[table,xcdraw]{xcolor}


%% ----------------------------------------------------------------
\begin{document}
\frontmatter
\title      {Heterogeneous Agent-based Model for Supermarket Competition}
\authors    {\texorpdfstring
             {\href{mailto:sc22g13@ecs.soton.ac.uk}{Stefan J. Collier}}
             {Stefan J. Collier}
            }
\addresses  {\groupname\\\deptname\\\univname}
\date       {\today}
\subject    {}
\keywords   {}
\supervisor {Dr. Maria Polukarov}
\examiner   {Professor Sheng Chen}

\maketitle
\begin{abstract}
This project aim was to model and analyse the effects of competitive pricing behaviors of grocery retailers on the British market. 

This was achieved by creating a multi-agent model, containing retailer and consumer agents. The heterogeneous crowd of retailers employs either a uniform pricing strategy or a ‘local price flexing’ strategy. The actions of these retailers are chosen by predicting the profit of each action, using a perceptron. Following on from the consideration of different economic models, a discrete model was developed so that software agents have a discrete environment to operate within. Within the model, it has been observed how supermarkets with differing behaviors affect a heterogeneous crowd of consumer agents. The model was implemented in Java with Python used to evaluate the results. 

The simulation displays good acceptance with real grocery market behavior, i.e. captures the performance of British retailers thus can be used to determine the impact of changes in their behavior on their competitors and consumers.Furthermore it can be used to provide insight into sustainability of volatile pricing strategies, providing a useful insight in volatility of British supermarket retail industry. 
\end{abstract}
\acknowledgements{
I would like to express my sincere gratitude to Dr Maria Polukarov for her guidance and support which provided me the freedom to take this research in the direction of my interest.\\
\\
I would also like to thank my family and friends for their encouragement and support. To those who quietly listened to my software complaints. To those who worked throughout the nights with me. To those who helped me write what I couldn't say. I cannot thank you enough.
}

\declaration{
I, Stefan Collier, declare that this dissertation and the work presented in it are my own and has been generated by me as the result of my own original research.\\
I confirm that:\\
1. This work was done wholly or mainly while in candidature for a degree at this University;\\
2. Where any part of this dissertation has previously been submitted for any other qualification at this University or any other institution, this has been clearly stated;\\
3. Where I have consulted the published work of others, this is always clearly attributed;\\
4. Where I have quoted from the work of others, the source is always given. With the exception of such quotations, this dissertation is entirely my own work;\\
5. I have acknowledged all main sources of help;\\
6. Where the thesis is based on work done by myself jointly with others, I have made clear exactly what was done by others and what I have contributed myself;\\
7. Either none of this work has been published before submission, or parts of this work have been published by :\\
\\
Stefan Collier\\
April 2016
}
\tableofcontents
\listoffigures
\listoftables

\mainmatter
%% ----------------------------------------------------------------
%\include{Introduction}
%\include{Conclusions}
\include{chapters/1Project/main}
\include{chapters/2Lit/main}
\include{chapters/3Design/HighLevel}
\include{chapters/3Design/InDepth}
\include{chapters/4Impl/main}

\include{chapters/5Experiments/1/main}
\include{chapters/5Experiments/2/main}
\include{chapters/5Experiments/3/main}
\include{chapters/5Experiments/4/main}

\include{chapters/6Conclusion/main}

\appendix
\include{appendix/AppendixB}
\include{appendix/D/main}
\include{appendix/AppendixC}

\backmatter
\bibliographystyle{ecs}
\bibliography{ECS}
\end{document}
%% ----------------------------------------------------------------

 %% ----------------------------------------------------------------
%% Progress.tex
%% ---------------------------------------------------------------- 
\documentclass{ecsprogress}    % Use the progress Style
\graphicspath{{../figs/}}   % Location of your graphics files
    \usepackage{natbib}            % Use Natbib style for the refs.
\hypersetup{colorlinks=true}   % Set to false for black/white printing
\input{Definitions}            % Include your abbreviations



\usepackage{enumitem}% http://ctan.org/pkg/enumitem
\usepackage{multirow}
\usepackage{float}
\usepackage{amsmath}
\usepackage{multicol}
\usepackage{amssymb}
\usepackage[normalem]{ulem}
\useunder{\uline}{\ul}{}
\usepackage{wrapfig}


\usepackage[table,xcdraw]{xcolor}


%% ----------------------------------------------------------------
\begin{document}
\frontmatter
\title      {Heterogeneous Agent-based Model for Supermarket Competition}
\authors    {\texorpdfstring
             {\href{mailto:sc22g13@ecs.soton.ac.uk}{Stefan J. Collier}}
             {Stefan J. Collier}
            }
\addresses  {\groupname\\\deptname\\\univname}
\date       {\today}
\subject    {}
\keywords   {}
\supervisor {Dr. Maria Polukarov}
\examiner   {Professor Sheng Chen}

\maketitle
\begin{abstract}
This project aim was to model and analyse the effects of competitive pricing behaviors of grocery retailers on the British market. 

This was achieved by creating a multi-agent model, containing retailer and consumer agents. The heterogeneous crowd of retailers employs either a uniform pricing strategy or a ‘local price flexing’ strategy. The actions of these retailers are chosen by predicting the profit of each action, using a perceptron. Following on from the consideration of different economic models, a discrete model was developed so that software agents have a discrete environment to operate within. Within the model, it has been observed how supermarkets with differing behaviors affect a heterogeneous crowd of consumer agents. The model was implemented in Java with Python used to evaluate the results. 

The simulation displays good acceptance with real grocery market behavior, i.e. captures the performance of British retailers thus can be used to determine the impact of changes in their behavior on their competitors and consumers.Furthermore it can be used to provide insight into sustainability of volatile pricing strategies, providing a useful insight in volatility of British supermarket retail industry. 
\end{abstract}
\acknowledgements{
I would like to express my sincere gratitude to Dr Maria Polukarov for her guidance and support which provided me the freedom to take this research in the direction of my interest.\\
\\
I would also like to thank my family and friends for their encouragement and support. To those who quietly listened to my software complaints. To those who worked throughout the nights with me. To those who helped me write what I couldn't say. I cannot thank you enough.
}

\declaration{
I, Stefan Collier, declare that this dissertation and the work presented in it are my own and has been generated by me as the result of my own original research.\\
I confirm that:\\
1. This work was done wholly or mainly while in candidature for a degree at this University;\\
2. Where any part of this dissertation has previously been submitted for any other qualification at this University or any other institution, this has been clearly stated;\\
3. Where I have consulted the published work of others, this is always clearly attributed;\\
4. Where I have quoted from the work of others, the source is always given. With the exception of such quotations, this dissertation is entirely my own work;\\
5. I have acknowledged all main sources of help;\\
6. Where the thesis is based on work done by myself jointly with others, I have made clear exactly what was done by others and what I have contributed myself;\\
7. Either none of this work has been published before submission, or parts of this work have been published by :\\
\\
Stefan Collier\\
April 2016
}
\tableofcontents
\listoffigures
\listoftables

\mainmatter
%% ----------------------------------------------------------------
%\include{Introduction}
%\include{Conclusions}
\include{chapters/1Project/main}
\include{chapters/2Lit/main}
\include{chapters/3Design/HighLevel}
\include{chapters/3Design/InDepth}
\include{chapters/4Impl/main}

\include{chapters/5Experiments/1/main}
\include{chapters/5Experiments/2/main}
\include{chapters/5Experiments/3/main}
\include{chapters/5Experiments/4/main}

\include{chapters/6Conclusion/main}

\appendix
\include{appendix/AppendixB}
\include{appendix/D/main}
\include{appendix/AppendixC}

\backmatter
\bibliographystyle{ecs}
\bibliography{ECS}
\end{document}
%% ----------------------------------------------------------------

\include{chapters/3Design/HighLevel}
\include{chapters/3Design/InDepth}
 %% ----------------------------------------------------------------
%% Progress.tex
%% ---------------------------------------------------------------- 
\documentclass{ecsprogress}    % Use the progress Style
\graphicspath{{../figs/}}   % Location of your graphics files
    \usepackage{natbib}            % Use Natbib style for the refs.
\hypersetup{colorlinks=true}   % Set to false for black/white printing
\input{Definitions}            % Include your abbreviations



\usepackage{enumitem}% http://ctan.org/pkg/enumitem
\usepackage{multirow}
\usepackage{float}
\usepackage{amsmath}
\usepackage{multicol}
\usepackage{amssymb}
\usepackage[normalem]{ulem}
\useunder{\uline}{\ul}{}
\usepackage{wrapfig}


\usepackage[table,xcdraw]{xcolor}


%% ----------------------------------------------------------------
\begin{document}
\frontmatter
\title      {Heterogeneous Agent-based Model for Supermarket Competition}
\authors    {\texorpdfstring
             {\href{mailto:sc22g13@ecs.soton.ac.uk}{Stefan J. Collier}}
             {Stefan J. Collier}
            }
\addresses  {\groupname\\\deptname\\\univname}
\date       {\today}
\subject    {}
\keywords   {}
\supervisor {Dr. Maria Polukarov}
\examiner   {Professor Sheng Chen}

\maketitle
\begin{abstract}
This project aim was to model and analyse the effects of competitive pricing behaviors of grocery retailers on the British market. 

This was achieved by creating a multi-agent model, containing retailer and consumer agents. The heterogeneous crowd of retailers employs either a uniform pricing strategy or a ‘local price flexing’ strategy. The actions of these retailers are chosen by predicting the profit of each action, using a perceptron. Following on from the consideration of different economic models, a discrete model was developed so that software agents have a discrete environment to operate within. Within the model, it has been observed how supermarkets with differing behaviors affect a heterogeneous crowd of consumer agents. The model was implemented in Java with Python used to evaluate the results. 

The simulation displays good acceptance with real grocery market behavior, i.e. captures the performance of British retailers thus can be used to determine the impact of changes in their behavior on their competitors and consumers.Furthermore it can be used to provide insight into sustainability of volatile pricing strategies, providing a useful insight in volatility of British supermarket retail industry. 
\end{abstract}
\acknowledgements{
I would like to express my sincere gratitude to Dr Maria Polukarov for her guidance and support which provided me the freedom to take this research in the direction of my interest.\\
\\
I would also like to thank my family and friends for their encouragement and support. To those who quietly listened to my software complaints. To those who worked throughout the nights with me. To those who helped me write what I couldn't say. I cannot thank you enough.
}

\declaration{
I, Stefan Collier, declare that this dissertation and the work presented in it are my own and has been generated by me as the result of my own original research.\\
I confirm that:\\
1. This work was done wholly or mainly while in candidature for a degree at this University;\\
2. Where any part of this dissertation has previously been submitted for any other qualification at this University or any other institution, this has been clearly stated;\\
3. Where I have consulted the published work of others, this is always clearly attributed;\\
4. Where I have quoted from the work of others, the source is always given. With the exception of such quotations, this dissertation is entirely my own work;\\
5. I have acknowledged all main sources of help;\\
6. Where the thesis is based on work done by myself jointly with others, I have made clear exactly what was done by others and what I have contributed myself;\\
7. Either none of this work has been published before submission, or parts of this work have been published by :\\
\\
Stefan Collier\\
April 2016
}
\tableofcontents
\listoffigures
\listoftables

\mainmatter
%% ----------------------------------------------------------------
%\include{Introduction}
%\include{Conclusions}
\include{chapters/1Project/main}
\include{chapters/2Lit/main}
\include{chapters/3Design/HighLevel}
\include{chapters/3Design/InDepth}
\include{chapters/4Impl/main}

\include{chapters/5Experiments/1/main}
\include{chapters/5Experiments/2/main}
\include{chapters/5Experiments/3/main}
\include{chapters/5Experiments/4/main}

\include{chapters/6Conclusion/main}

\appendix
\include{appendix/AppendixB}
\include{appendix/D/main}
\include{appendix/AppendixC}

\backmatter
\bibliographystyle{ecs}
\bibliography{ECS}
\end{document}
%% ----------------------------------------------------------------


 %% ----------------------------------------------------------------
%% Progress.tex
%% ---------------------------------------------------------------- 
\documentclass{ecsprogress}    % Use the progress Style
\graphicspath{{../figs/}}   % Location of your graphics files
    \usepackage{natbib}            % Use Natbib style for the refs.
\hypersetup{colorlinks=true}   % Set to false for black/white printing
\input{Definitions}            % Include your abbreviations



\usepackage{enumitem}% http://ctan.org/pkg/enumitem
\usepackage{multirow}
\usepackage{float}
\usepackage{amsmath}
\usepackage{multicol}
\usepackage{amssymb}
\usepackage[normalem]{ulem}
\useunder{\uline}{\ul}{}
\usepackage{wrapfig}


\usepackage[table,xcdraw]{xcolor}


%% ----------------------------------------------------------------
\begin{document}
\frontmatter
\title      {Heterogeneous Agent-based Model for Supermarket Competition}
\authors    {\texorpdfstring
             {\href{mailto:sc22g13@ecs.soton.ac.uk}{Stefan J. Collier}}
             {Stefan J. Collier}
            }
\addresses  {\groupname\\\deptname\\\univname}
\date       {\today}
\subject    {}
\keywords   {}
\supervisor {Dr. Maria Polukarov}
\examiner   {Professor Sheng Chen}

\maketitle
\begin{abstract}
This project aim was to model and analyse the effects of competitive pricing behaviors of grocery retailers on the British market. 

This was achieved by creating a multi-agent model, containing retailer and consumer agents. The heterogeneous crowd of retailers employs either a uniform pricing strategy or a ‘local price flexing’ strategy. The actions of these retailers are chosen by predicting the profit of each action, using a perceptron. Following on from the consideration of different economic models, a discrete model was developed so that software agents have a discrete environment to operate within. Within the model, it has been observed how supermarkets with differing behaviors affect a heterogeneous crowd of consumer agents. The model was implemented in Java with Python used to evaluate the results. 

The simulation displays good acceptance with real grocery market behavior, i.e. captures the performance of British retailers thus can be used to determine the impact of changes in their behavior on their competitors and consumers.Furthermore it can be used to provide insight into sustainability of volatile pricing strategies, providing a useful insight in volatility of British supermarket retail industry. 
\end{abstract}
\acknowledgements{
I would like to express my sincere gratitude to Dr Maria Polukarov for her guidance and support which provided me the freedom to take this research in the direction of my interest.\\
\\
I would also like to thank my family and friends for their encouragement and support. To those who quietly listened to my software complaints. To those who worked throughout the nights with me. To those who helped me write what I couldn't say. I cannot thank you enough.
}

\declaration{
I, Stefan Collier, declare that this dissertation and the work presented in it are my own and has been generated by me as the result of my own original research.\\
I confirm that:\\
1. This work was done wholly or mainly while in candidature for a degree at this University;\\
2. Where any part of this dissertation has previously been submitted for any other qualification at this University or any other institution, this has been clearly stated;\\
3. Where I have consulted the published work of others, this is always clearly attributed;\\
4. Where I have quoted from the work of others, the source is always given. With the exception of such quotations, this dissertation is entirely my own work;\\
5. I have acknowledged all main sources of help;\\
6. Where the thesis is based on work done by myself jointly with others, I have made clear exactly what was done by others and what I have contributed myself;\\
7. Either none of this work has been published before submission, or parts of this work have been published by :\\
\\
Stefan Collier\\
April 2016
}
\tableofcontents
\listoffigures
\listoftables

\mainmatter
%% ----------------------------------------------------------------
%\include{Introduction}
%\include{Conclusions}
\include{chapters/1Project/main}
\include{chapters/2Lit/main}
\include{chapters/3Design/HighLevel}
\include{chapters/3Design/InDepth}
\include{chapters/4Impl/main}

\include{chapters/5Experiments/1/main}
\include{chapters/5Experiments/2/main}
\include{chapters/5Experiments/3/main}
\include{chapters/5Experiments/4/main}

\include{chapters/6Conclusion/main}

\appendix
\include{appendix/AppendixB}
\include{appendix/D/main}
\include{appendix/AppendixC}

\backmatter
\bibliographystyle{ecs}
\bibliography{ECS}
\end{document}
%% ----------------------------------------------------------------

 %% ----------------------------------------------------------------
%% Progress.tex
%% ---------------------------------------------------------------- 
\documentclass{ecsprogress}    % Use the progress Style
\graphicspath{{../figs/}}   % Location of your graphics files
    \usepackage{natbib}            % Use Natbib style for the refs.
\hypersetup{colorlinks=true}   % Set to false for black/white printing
\input{Definitions}            % Include your abbreviations



\usepackage{enumitem}% http://ctan.org/pkg/enumitem
\usepackage{multirow}
\usepackage{float}
\usepackage{amsmath}
\usepackage{multicol}
\usepackage{amssymb}
\usepackage[normalem]{ulem}
\useunder{\uline}{\ul}{}
\usepackage{wrapfig}


\usepackage[table,xcdraw]{xcolor}


%% ----------------------------------------------------------------
\begin{document}
\frontmatter
\title      {Heterogeneous Agent-based Model for Supermarket Competition}
\authors    {\texorpdfstring
             {\href{mailto:sc22g13@ecs.soton.ac.uk}{Stefan J. Collier}}
             {Stefan J. Collier}
            }
\addresses  {\groupname\\\deptname\\\univname}
\date       {\today}
\subject    {}
\keywords   {}
\supervisor {Dr. Maria Polukarov}
\examiner   {Professor Sheng Chen}

\maketitle
\begin{abstract}
This project aim was to model and analyse the effects of competitive pricing behaviors of grocery retailers on the British market. 

This was achieved by creating a multi-agent model, containing retailer and consumer agents. The heterogeneous crowd of retailers employs either a uniform pricing strategy or a ‘local price flexing’ strategy. The actions of these retailers are chosen by predicting the profit of each action, using a perceptron. Following on from the consideration of different economic models, a discrete model was developed so that software agents have a discrete environment to operate within. Within the model, it has been observed how supermarkets with differing behaviors affect a heterogeneous crowd of consumer agents. The model was implemented in Java with Python used to evaluate the results. 

The simulation displays good acceptance with real grocery market behavior, i.e. captures the performance of British retailers thus can be used to determine the impact of changes in their behavior on their competitors and consumers.Furthermore it can be used to provide insight into sustainability of volatile pricing strategies, providing a useful insight in volatility of British supermarket retail industry. 
\end{abstract}
\acknowledgements{
I would like to express my sincere gratitude to Dr Maria Polukarov for her guidance and support which provided me the freedom to take this research in the direction of my interest.\\
\\
I would also like to thank my family and friends for their encouragement and support. To those who quietly listened to my software complaints. To those who worked throughout the nights with me. To those who helped me write what I couldn't say. I cannot thank you enough.
}

\declaration{
I, Stefan Collier, declare that this dissertation and the work presented in it are my own and has been generated by me as the result of my own original research.\\
I confirm that:\\
1. This work was done wholly or mainly while in candidature for a degree at this University;\\
2. Where any part of this dissertation has previously been submitted for any other qualification at this University or any other institution, this has been clearly stated;\\
3. Where I have consulted the published work of others, this is always clearly attributed;\\
4. Where I have quoted from the work of others, the source is always given. With the exception of such quotations, this dissertation is entirely my own work;\\
5. I have acknowledged all main sources of help;\\
6. Where the thesis is based on work done by myself jointly with others, I have made clear exactly what was done by others and what I have contributed myself;\\
7. Either none of this work has been published before submission, or parts of this work have been published by :\\
\\
Stefan Collier\\
April 2016
}
\tableofcontents
\listoffigures
\listoftables

\mainmatter
%% ----------------------------------------------------------------
%\include{Introduction}
%\include{Conclusions}
\include{chapters/1Project/main}
\include{chapters/2Lit/main}
\include{chapters/3Design/HighLevel}
\include{chapters/3Design/InDepth}
\include{chapters/4Impl/main}

\include{chapters/5Experiments/1/main}
\include{chapters/5Experiments/2/main}
\include{chapters/5Experiments/3/main}
\include{chapters/5Experiments/4/main}

\include{chapters/6Conclusion/main}

\appendix
\include{appendix/AppendixB}
\include{appendix/D/main}
\include{appendix/AppendixC}

\backmatter
\bibliographystyle{ecs}
\bibliography{ECS}
\end{document}
%% ----------------------------------------------------------------

 %% ----------------------------------------------------------------
%% Progress.tex
%% ---------------------------------------------------------------- 
\documentclass{ecsprogress}    % Use the progress Style
\graphicspath{{../figs/}}   % Location of your graphics files
    \usepackage{natbib}            % Use Natbib style for the refs.
\hypersetup{colorlinks=true}   % Set to false for black/white printing
\input{Definitions}            % Include your abbreviations



\usepackage{enumitem}% http://ctan.org/pkg/enumitem
\usepackage{multirow}
\usepackage{float}
\usepackage{amsmath}
\usepackage{multicol}
\usepackage{amssymb}
\usepackage[normalem]{ulem}
\useunder{\uline}{\ul}{}
\usepackage{wrapfig}


\usepackage[table,xcdraw]{xcolor}


%% ----------------------------------------------------------------
\begin{document}
\frontmatter
\title      {Heterogeneous Agent-based Model for Supermarket Competition}
\authors    {\texorpdfstring
             {\href{mailto:sc22g13@ecs.soton.ac.uk}{Stefan J. Collier}}
             {Stefan J. Collier}
            }
\addresses  {\groupname\\\deptname\\\univname}
\date       {\today}
\subject    {}
\keywords   {}
\supervisor {Dr. Maria Polukarov}
\examiner   {Professor Sheng Chen}

\maketitle
\begin{abstract}
This project aim was to model and analyse the effects of competitive pricing behaviors of grocery retailers on the British market. 

This was achieved by creating a multi-agent model, containing retailer and consumer agents. The heterogeneous crowd of retailers employs either a uniform pricing strategy or a ‘local price flexing’ strategy. The actions of these retailers are chosen by predicting the profit of each action, using a perceptron. Following on from the consideration of different economic models, a discrete model was developed so that software agents have a discrete environment to operate within. Within the model, it has been observed how supermarkets with differing behaviors affect a heterogeneous crowd of consumer agents. The model was implemented in Java with Python used to evaluate the results. 

The simulation displays good acceptance with real grocery market behavior, i.e. captures the performance of British retailers thus can be used to determine the impact of changes in their behavior on their competitors and consumers.Furthermore it can be used to provide insight into sustainability of volatile pricing strategies, providing a useful insight in volatility of British supermarket retail industry. 
\end{abstract}
\acknowledgements{
I would like to express my sincere gratitude to Dr Maria Polukarov for her guidance and support which provided me the freedom to take this research in the direction of my interest.\\
\\
I would also like to thank my family and friends for their encouragement and support. To those who quietly listened to my software complaints. To those who worked throughout the nights with me. To those who helped me write what I couldn't say. I cannot thank you enough.
}

\declaration{
I, Stefan Collier, declare that this dissertation and the work presented in it are my own and has been generated by me as the result of my own original research.\\
I confirm that:\\
1. This work was done wholly or mainly while in candidature for a degree at this University;\\
2. Where any part of this dissertation has previously been submitted for any other qualification at this University or any other institution, this has been clearly stated;\\
3. Where I have consulted the published work of others, this is always clearly attributed;\\
4. Where I have quoted from the work of others, the source is always given. With the exception of such quotations, this dissertation is entirely my own work;\\
5. I have acknowledged all main sources of help;\\
6. Where the thesis is based on work done by myself jointly with others, I have made clear exactly what was done by others and what I have contributed myself;\\
7. Either none of this work has been published before submission, or parts of this work have been published by :\\
\\
Stefan Collier\\
April 2016
}
\tableofcontents
\listoffigures
\listoftables

\mainmatter
%% ----------------------------------------------------------------
%\include{Introduction}
%\include{Conclusions}
\include{chapters/1Project/main}
\include{chapters/2Lit/main}
\include{chapters/3Design/HighLevel}
\include{chapters/3Design/InDepth}
\include{chapters/4Impl/main}

\include{chapters/5Experiments/1/main}
\include{chapters/5Experiments/2/main}
\include{chapters/5Experiments/3/main}
\include{chapters/5Experiments/4/main}

\include{chapters/6Conclusion/main}

\appendix
\include{appendix/AppendixB}
\include{appendix/D/main}
\include{appendix/AppendixC}

\backmatter
\bibliographystyle{ecs}
\bibliography{ECS}
\end{document}
%% ----------------------------------------------------------------

 %% ----------------------------------------------------------------
%% Progress.tex
%% ---------------------------------------------------------------- 
\documentclass{ecsprogress}    % Use the progress Style
\graphicspath{{../figs/}}   % Location of your graphics files
    \usepackage{natbib}            % Use Natbib style for the refs.
\hypersetup{colorlinks=true}   % Set to false for black/white printing
\input{Definitions}            % Include your abbreviations



\usepackage{enumitem}% http://ctan.org/pkg/enumitem
\usepackage{multirow}
\usepackage{float}
\usepackage{amsmath}
\usepackage{multicol}
\usepackage{amssymb}
\usepackage[normalem]{ulem}
\useunder{\uline}{\ul}{}
\usepackage{wrapfig}


\usepackage[table,xcdraw]{xcolor}


%% ----------------------------------------------------------------
\begin{document}
\frontmatter
\title      {Heterogeneous Agent-based Model for Supermarket Competition}
\authors    {\texorpdfstring
             {\href{mailto:sc22g13@ecs.soton.ac.uk}{Stefan J. Collier}}
             {Stefan J. Collier}
            }
\addresses  {\groupname\\\deptname\\\univname}
\date       {\today}
\subject    {}
\keywords   {}
\supervisor {Dr. Maria Polukarov}
\examiner   {Professor Sheng Chen}

\maketitle
\begin{abstract}
This project aim was to model and analyse the effects of competitive pricing behaviors of grocery retailers on the British market. 

This was achieved by creating a multi-agent model, containing retailer and consumer agents. The heterogeneous crowd of retailers employs either a uniform pricing strategy or a ‘local price flexing’ strategy. The actions of these retailers are chosen by predicting the profit of each action, using a perceptron. Following on from the consideration of different economic models, a discrete model was developed so that software agents have a discrete environment to operate within. Within the model, it has been observed how supermarkets with differing behaviors affect a heterogeneous crowd of consumer agents. The model was implemented in Java with Python used to evaluate the results. 

The simulation displays good acceptance with real grocery market behavior, i.e. captures the performance of British retailers thus can be used to determine the impact of changes in their behavior on their competitors and consumers.Furthermore it can be used to provide insight into sustainability of volatile pricing strategies, providing a useful insight in volatility of British supermarket retail industry. 
\end{abstract}
\acknowledgements{
I would like to express my sincere gratitude to Dr Maria Polukarov for her guidance and support which provided me the freedom to take this research in the direction of my interest.\\
\\
I would also like to thank my family and friends for their encouragement and support. To those who quietly listened to my software complaints. To those who worked throughout the nights with me. To those who helped me write what I couldn't say. I cannot thank you enough.
}

\declaration{
I, Stefan Collier, declare that this dissertation and the work presented in it are my own and has been generated by me as the result of my own original research.\\
I confirm that:\\
1. This work was done wholly or mainly while in candidature for a degree at this University;\\
2. Where any part of this dissertation has previously been submitted for any other qualification at this University or any other institution, this has been clearly stated;\\
3. Where I have consulted the published work of others, this is always clearly attributed;\\
4. Where I have quoted from the work of others, the source is always given. With the exception of such quotations, this dissertation is entirely my own work;\\
5. I have acknowledged all main sources of help;\\
6. Where the thesis is based on work done by myself jointly with others, I have made clear exactly what was done by others and what I have contributed myself;\\
7. Either none of this work has been published before submission, or parts of this work have been published by :\\
\\
Stefan Collier\\
April 2016
}
\tableofcontents
\listoffigures
\listoftables

\mainmatter
%% ----------------------------------------------------------------
%\include{Introduction}
%\include{Conclusions}
\include{chapters/1Project/main}
\include{chapters/2Lit/main}
\include{chapters/3Design/HighLevel}
\include{chapters/3Design/InDepth}
\include{chapters/4Impl/main}

\include{chapters/5Experiments/1/main}
\include{chapters/5Experiments/2/main}
\include{chapters/5Experiments/3/main}
\include{chapters/5Experiments/4/main}

\include{chapters/6Conclusion/main}

\appendix
\include{appendix/AppendixB}
\include{appendix/D/main}
\include{appendix/AppendixC}

\backmatter
\bibliographystyle{ecs}
\bibliography{ECS}
\end{document}
%% ----------------------------------------------------------------


 %% ----------------------------------------------------------------
%% Progress.tex
%% ---------------------------------------------------------------- 
\documentclass{ecsprogress}    % Use the progress Style
\graphicspath{{../figs/}}   % Location of your graphics files
    \usepackage{natbib}            % Use Natbib style for the refs.
\hypersetup{colorlinks=true}   % Set to false for black/white printing
\input{Definitions}            % Include your abbreviations



\usepackage{enumitem}% http://ctan.org/pkg/enumitem
\usepackage{multirow}
\usepackage{float}
\usepackage{amsmath}
\usepackage{multicol}
\usepackage{amssymb}
\usepackage[normalem]{ulem}
\useunder{\uline}{\ul}{}
\usepackage{wrapfig}


\usepackage[table,xcdraw]{xcolor}


%% ----------------------------------------------------------------
\begin{document}
\frontmatter
\title      {Heterogeneous Agent-based Model for Supermarket Competition}
\authors    {\texorpdfstring
             {\href{mailto:sc22g13@ecs.soton.ac.uk}{Stefan J. Collier}}
             {Stefan J. Collier}
            }
\addresses  {\groupname\\\deptname\\\univname}
\date       {\today}
\subject    {}
\keywords   {}
\supervisor {Dr. Maria Polukarov}
\examiner   {Professor Sheng Chen}

\maketitle
\begin{abstract}
This project aim was to model and analyse the effects of competitive pricing behaviors of grocery retailers on the British market. 

This was achieved by creating a multi-agent model, containing retailer and consumer agents. The heterogeneous crowd of retailers employs either a uniform pricing strategy or a ‘local price flexing’ strategy. The actions of these retailers are chosen by predicting the profit of each action, using a perceptron. Following on from the consideration of different economic models, a discrete model was developed so that software agents have a discrete environment to operate within. Within the model, it has been observed how supermarkets with differing behaviors affect a heterogeneous crowd of consumer agents. The model was implemented in Java with Python used to evaluate the results. 

The simulation displays good acceptance with real grocery market behavior, i.e. captures the performance of British retailers thus can be used to determine the impact of changes in their behavior on their competitors and consumers.Furthermore it can be used to provide insight into sustainability of volatile pricing strategies, providing a useful insight in volatility of British supermarket retail industry. 
\end{abstract}
\acknowledgements{
I would like to express my sincere gratitude to Dr Maria Polukarov for her guidance and support which provided me the freedom to take this research in the direction of my interest.\\
\\
I would also like to thank my family and friends for their encouragement and support. To those who quietly listened to my software complaints. To those who worked throughout the nights with me. To those who helped me write what I couldn't say. I cannot thank you enough.
}

\declaration{
I, Stefan Collier, declare that this dissertation and the work presented in it are my own and has been generated by me as the result of my own original research.\\
I confirm that:\\
1. This work was done wholly or mainly while in candidature for a degree at this University;\\
2. Where any part of this dissertation has previously been submitted for any other qualification at this University or any other institution, this has been clearly stated;\\
3. Where I have consulted the published work of others, this is always clearly attributed;\\
4. Where I have quoted from the work of others, the source is always given. With the exception of such quotations, this dissertation is entirely my own work;\\
5. I have acknowledged all main sources of help;\\
6. Where the thesis is based on work done by myself jointly with others, I have made clear exactly what was done by others and what I have contributed myself;\\
7. Either none of this work has been published before submission, or parts of this work have been published by :\\
\\
Stefan Collier\\
April 2016
}
\tableofcontents
\listoffigures
\listoftables

\mainmatter
%% ----------------------------------------------------------------
%\include{Introduction}
%\include{Conclusions}
\include{chapters/1Project/main}
\include{chapters/2Lit/main}
\include{chapters/3Design/HighLevel}
\include{chapters/3Design/InDepth}
\include{chapters/4Impl/main}

\include{chapters/5Experiments/1/main}
\include{chapters/5Experiments/2/main}
\include{chapters/5Experiments/3/main}
\include{chapters/5Experiments/4/main}

\include{chapters/6Conclusion/main}

\appendix
\include{appendix/AppendixB}
\include{appendix/D/main}
\include{appendix/AppendixC}

\backmatter
\bibliographystyle{ecs}
\bibliography{ECS}
\end{document}
%% ----------------------------------------------------------------


\appendix
\include{appendix/AppendixB}
 %% ----------------------------------------------------------------
%% Progress.tex
%% ---------------------------------------------------------------- 
\documentclass{ecsprogress}    % Use the progress Style
\graphicspath{{../figs/}}   % Location of your graphics files
    \usepackage{natbib}            % Use Natbib style for the refs.
\hypersetup{colorlinks=true}   % Set to false for black/white printing
\input{Definitions}            % Include your abbreviations



\usepackage{enumitem}% http://ctan.org/pkg/enumitem
\usepackage{multirow}
\usepackage{float}
\usepackage{amsmath}
\usepackage{multicol}
\usepackage{amssymb}
\usepackage[normalem]{ulem}
\useunder{\uline}{\ul}{}
\usepackage{wrapfig}


\usepackage[table,xcdraw]{xcolor}


%% ----------------------------------------------------------------
\begin{document}
\frontmatter
\title      {Heterogeneous Agent-based Model for Supermarket Competition}
\authors    {\texorpdfstring
             {\href{mailto:sc22g13@ecs.soton.ac.uk}{Stefan J. Collier}}
             {Stefan J. Collier}
            }
\addresses  {\groupname\\\deptname\\\univname}
\date       {\today}
\subject    {}
\keywords   {}
\supervisor {Dr. Maria Polukarov}
\examiner   {Professor Sheng Chen}

\maketitle
\begin{abstract}
This project aim was to model and analyse the effects of competitive pricing behaviors of grocery retailers on the British market. 

This was achieved by creating a multi-agent model, containing retailer and consumer agents. The heterogeneous crowd of retailers employs either a uniform pricing strategy or a ‘local price flexing’ strategy. The actions of these retailers are chosen by predicting the profit of each action, using a perceptron. Following on from the consideration of different economic models, a discrete model was developed so that software agents have a discrete environment to operate within. Within the model, it has been observed how supermarkets with differing behaviors affect a heterogeneous crowd of consumer agents. The model was implemented in Java with Python used to evaluate the results. 

The simulation displays good acceptance with real grocery market behavior, i.e. captures the performance of British retailers thus can be used to determine the impact of changes in their behavior on their competitors and consumers.Furthermore it can be used to provide insight into sustainability of volatile pricing strategies, providing a useful insight in volatility of British supermarket retail industry. 
\end{abstract}
\acknowledgements{
I would like to express my sincere gratitude to Dr Maria Polukarov for her guidance and support which provided me the freedom to take this research in the direction of my interest.\\
\\
I would also like to thank my family and friends for their encouragement and support. To those who quietly listened to my software complaints. To those who worked throughout the nights with me. To those who helped me write what I couldn't say. I cannot thank you enough.
}

\declaration{
I, Stefan Collier, declare that this dissertation and the work presented in it are my own and has been generated by me as the result of my own original research.\\
I confirm that:\\
1. This work was done wholly or mainly while in candidature for a degree at this University;\\
2. Where any part of this dissertation has previously been submitted for any other qualification at this University or any other institution, this has been clearly stated;\\
3. Where I have consulted the published work of others, this is always clearly attributed;\\
4. Where I have quoted from the work of others, the source is always given. With the exception of such quotations, this dissertation is entirely my own work;\\
5. I have acknowledged all main sources of help;\\
6. Where the thesis is based on work done by myself jointly with others, I have made clear exactly what was done by others and what I have contributed myself;\\
7. Either none of this work has been published before submission, or parts of this work have been published by :\\
\\
Stefan Collier\\
April 2016
}
\tableofcontents
\listoffigures
\listoftables

\mainmatter
%% ----------------------------------------------------------------
%\include{Introduction}
%\include{Conclusions}
\include{chapters/1Project/main}
\include{chapters/2Lit/main}
\include{chapters/3Design/HighLevel}
\include{chapters/3Design/InDepth}
\include{chapters/4Impl/main}

\include{chapters/5Experiments/1/main}
\include{chapters/5Experiments/2/main}
\include{chapters/5Experiments/3/main}
\include{chapters/5Experiments/4/main}

\include{chapters/6Conclusion/main}

\appendix
\include{appendix/AppendixB}
\include{appendix/D/main}
\include{appendix/AppendixC}

\backmatter
\bibliographystyle{ecs}
\bibliography{ECS}
\end{document}
%% ----------------------------------------------------------------

\include{appendix/AppendixC}

\backmatter
\bibliographystyle{ecs}
\bibliography{ECS}
\end{document}
%% ----------------------------------------------------------------

 %% ----------------------------------------------------------------
%% Progress.tex
%% ---------------------------------------------------------------- 
\documentclass{ecsprogress}    % Use the progress Style
\graphicspath{{../figs/}}   % Location of your graphics files
    \usepackage{natbib}            % Use Natbib style for the refs.
\hypersetup{colorlinks=true}   % Set to false for black/white printing
\input{Definitions}            % Include your abbreviations



\usepackage{enumitem}% http://ctan.org/pkg/enumitem
\usepackage{multirow}
\usepackage{float}
\usepackage{amsmath}
\usepackage{multicol}
\usepackage{amssymb}
\usepackage[normalem]{ulem}
\useunder{\uline}{\ul}{}
\usepackage{wrapfig}


\usepackage[table,xcdraw]{xcolor}


%% ----------------------------------------------------------------
\begin{document}
\frontmatter
\title      {Heterogeneous Agent-based Model for Supermarket Competition}
\authors    {\texorpdfstring
             {\href{mailto:sc22g13@ecs.soton.ac.uk}{Stefan J. Collier}}
             {Stefan J. Collier}
            }
\addresses  {\groupname\\\deptname\\\univname}
\date       {\today}
\subject    {}
\keywords   {}
\supervisor {Dr. Maria Polukarov}
\examiner   {Professor Sheng Chen}

\maketitle
\begin{abstract}
This project aim was to model and analyse the effects of competitive pricing behaviors of grocery retailers on the British market. 

This was achieved by creating a multi-agent model, containing retailer and consumer agents. The heterogeneous crowd of retailers employs either a uniform pricing strategy or a ‘local price flexing’ strategy. The actions of these retailers are chosen by predicting the profit of each action, using a perceptron. Following on from the consideration of different economic models, a discrete model was developed so that software agents have a discrete environment to operate within. Within the model, it has been observed how supermarkets with differing behaviors affect a heterogeneous crowd of consumer agents. The model was implemented in Java with Python used to evaluate the results. 

The simulation displays good acceptance with real grocery market behavior, i.e. captures the performance of British retailers thus can be used to determine the impact of changes in their behavior on their competitors and consumers.Furthermore it can be used to provide insight into sustainability of volatile pricing strategies, providing a useful insight in volatility of British supermarket retail industry. 
\end{abstract}
\acknowledgements{
I would like to express my sincere gratitude to Dr Maria Polukarov for her guidance and support which provided me the freedom to take this research in the direction of my interest.\\
\\
I would also like to thank my family and friends for their encouragement and support. To those who quietly listened to my software complaints. To those who worked throughout the nights with me. To those who helped me write what I couldn't say. I cannot thank you enough.
}

\declaration{
I, Stefan Collier, declare that this dissertation and the work presented in it are my own and has been generated by me as the result of my own original research.\\
I confirm that:\\
1. This work was done wholly or mainly while in candidature for a degree at this University;\\
2. Where any part of this dissertation has previously been submitted for any other qualification at this University or any other institution, this has been clearly stated;\\
3. Where I have consulted the published work of others, this is always clearly attributed;\\
4. Where I have quoted from the work of others, the source is always given. With the exception of such quotations, this dissertation is entirely my own work;\\
5. I have acknowledged all main sources of help;\\
6. Where the thesis is based on work done by myself jointly with others, I have made clear exactly what was done by others and what I have contributed myself;\\
7. Either none of this work has been published before submission, or parts of this work have been published by :\\
\\
Stefan Collier\\
April 2016
}
\tableofcontents
\listoffigures
\listoftables

\mainmatter
%% ----------------------------------------------------------------
%\include{Introduction}
%\include{Conclusions}
 %% ----------------------------------------------------------------
%% Progress.tex
%% ---------------------------------------------------------------- 
\documentclass{ecsprogress}    % Use the progress Style
\graphicspath{{../figs/}}   % Location of your graphics files
    \usepackage{natbib}            % Use Natbib style for the refs.
\hypersetup{colorlinks=true}   % Set to false for black/white printing
\input{Definitions}            % Include your abbreviations



\usepackage{enumitem}% http://ctan.org/pkg/enumitem
\usepackage{multirow}
\usepackage{float}
\usepackage{amsmath}
\usepackage{multicol}
\usepackage{amssymb}
\usepackage[normalem]{ulem}
\useunder{\uline}{\ul}{}
\usepackage{wrapfig}


\usepackage[table,xcdraw]{xcolor}


%% ----------------------------------------------------------------
\begin{document}
\frontmatter
\title      {Heterogeneous Agent-based Model for Supermarket Competition}
\authors    {\texorpdfstring
             {\href{mailto:sc22g13@ecs.soton.ac.uk}{Stefan J. Collier}}
             {Stefan J. Collier}
            }
\addresses  {\groupname\\\deptname\\\univname}
\date       {\today}
\subject    {}
\keywords   {}
\supervisor {Dr. Maria Polukarov}
\examiner   {Professor Sheng Chen}

\maketitle
\begin{abstract}
This project aim was to model and analyse the effects of competitive pricing behaviors of grocery retailers on the British market. 

This was achieved by creating a multi-agent model, containing retailer and consumer agents. The heterogeneous crowd of retailers employs either a uniform pricing strategy or a ‘local price flexing’ strategy. The actions of these retailers are chosen by predicting the profit of each action, using a perceptron. Following on from the consideration of different economic models, a discrete model was developed so that software agents have a discrete environment to operate within. Within the model, it has been observed how supermarkets with differing behaviors affect a heterogeneous crowd of consumer agents. The model was implemented in Java with Python used to evaluate the results. 

The simulation displays good acceptance with real grocery market behavior, i.e. captures the performance of British retailers thus can be used to determine the impact of changes in their behavior on their competitors and consumers.Furthermore it can be used to provide insight into sustainability of volatile pricing strategies, providing a useful insight in volatility of British supermarket retail industry. 
\end{abstract}
\acknowledgements{
I would like to express my sincere gratitude to Dr Maria Polukarov for her guidance and support which provided me the freedom to take this research in the direction of my interest.\\
\\
I would also like to thank my family and friends for their encouragement and support. To those who quietly listened to my software complaints. To those who worked throughout the nights with me. To those who helped me write what I couldn't say. I cannot thank you enough.
}

\declaration{
I, Stefan Collier, declare that this dissertation and the work presented in it are my own and has been generated by me as the result of my own original research.\\
I confirm that:\\
1. This work was done wholly or mainly while in candidature for a degree at this University;\\
2. Where any part of this dissertation has previously been submitted for any other qualification at this University or any other institution, this has been clearly stated;\\
3. Where I have consulted the published work of others, this is always clearly attributed;\\
4. Where I have quoted from the work of others, the source is always given. With the exception of such quotations, this dissertation is entirely my own work;\\
5. I have acknowledged all main sources of help;\\
6. Where the thesis is based on work done by myself jointly with others, I have made clear exactly what was done by others and what I have contributed myself;\\
7. Either none of this work has been published before submission, or parts of this work have been published by :\\
\\
Stefan Collier\\
April 2016
}
\tableofcontents
\listoffigures
\listoftables

\mainmatter
%% ----------------------------------------------------------------
%\include{Introduction}
%\include{Conclusions}
\include{chapters/1Project/main}
\include{chapters/2Lit/main}
\include{chapters/3Design/HighLevel}
\include{chapters/3Design/InDepth}
\include{chapters/4Impl/main}

\include{chapters/5Experiments/1/main}
\include{chapters/5Experiments/2/main}
\include{chapters/5Experiments/3/main}
\include{chapters/5Experiments/4/main}

\include{chapters/6Conclusion/main}

\appendix
\include{appendix/AppendixB}
\include{appendix/D/main}
\include{appendix/AppendixC}

\backmatter
\bibliographystyle{ecs}
\bibliography{ECS}
\end{document}
%% ----------------------------------------------------------------

 %% ----------------------------------------------------------------
%% Progress.tex
%% ---------------------------------------------------------------- 
\documentclass{ecsprogress}    % Use the progress Style
\graphicspath{{../figs/}}   % Location of your graphics files
    \usepackage{natbib}            % Use Natbib style for the refs.
\hypersetup{colorlinks=true}   % Set to false for black/white printing
\input{Definitions}            % Include your abbreviations



\usepackage{enumitem}% http://ctan.org/pkg/enumitem
\usepackage{multirow}
\usepackage{float}
\usepackage{amsmath}
\usepackage{multicol}
\usepackage{amssymb}
\usepackage[normalem]{ulem}
\useunder{\uline}{\ul}{}
\usepackage{wrapfig}


\usepackage[table,xcdraw]{xcolor}


%% ----------------------------------------------------------------
\begin{document}
\frontmatter
\title      {Heterogeneous Agent-based Model for Supermarket Competition}
\authors    {\texorpdfstring
             {\href{mailto:sc22g13@ecs.soton.ac.uk}{Stefan J. Collier}}
             {Stefan J. Collier}
            }
\addresses  {\groupname\\\deptname\\\univname}
\date       {\today}
\subject    {}
\keywords   {}
\supervisor {Dr. Maria Polukarov}
\examiner   {Professor Sheng Chen}

\maketitle
\begin{abstract}
This project aim was to model and analyse the effects of competitive pricing behaviors of grocery retailers on the British market. 

This was achieved by creating a multi-agent model, containing retailer and consumer agents. The heterogeneous crowd of retailers employs either a uniform pricing strategy or a ‘local price flexing’ strategy. The actions of these retailers are chosen by predicting the profit of each action, using a perceptron. Following on from the consideration of different economic models, a discrete model was developed so that software agents have a discrete environment to operate within. Within the model, it has been observed how supermarkets with differing behaviors affect a heterogeneous crowd of consumer agents. The model was implemented in Java with Python used to evaluate the results. 

The simulation displays good acceptance with real grocery market behavior, i.e. captures the performance of British retailers thus can be used to determine the impact of changes in their behavior on their competitors and consumers.Furthermore it can be used to provide insight into sustainability of volatile pricing strategies, providing a useful insight in volatility of British supermarket retail industry. 
\end{abstract}
\acknowledgements{
I would like to express my sincere gratitude to Dr Maria Polukarov for her guidance and support which provided me the freedom to take this research in the direction of my interest.\\
\\
I would also like to thank my family and friends for their encouragement and support. To those who quietly listened to my software complaints. To those who worked throughout the nights with me. To those who helped me write what I couldn't say. I cannot thank you enough.
}

\declaration{
I, Stefan Collier, declare that this dissertation and the work presented in it are my own and has been generated by me as the result of my own original research.\\
I confirm that:\\
1. This work was done wholly or mainly while in candidature for a degree at this University;\\
2. Where any part of this dissertation has previously been submitted for any other qualification at this University or any other institution, this has been clearly stated;\\
3. Where I have consulted the published work of others, this is always clearly attributed;\\
4. Where I have quoted from the work of others, the source is always given. With the exception of such quotations, this dissertation is entirely my own work;\\
5. I have acknowledged all main sources of help;\\
6. Where the thesis is based on work done by myself jointly with others, I have made clear exactly what was done by others and what I have contributed myself;\\
7. Either none of this work has been published before submission, or parts of this work have been published by :\\
\\
Stefan Collier\\
April 2016
}
\tableofcontents
\listoffigures
\listoftables

\mainmatter
%% ----------------------------------------------------------------
%\include{Introduction}
%\include{Conclusions}
\include{chapters/1Project/main}
\include{chapters/2Lit/main}
\include{chapters/3Design/HighLevel}
\include{chapters/3Design/InDepth}
\include{chapters/4Impl/main}

\include{chapters/5Experiments/1/main}
\include{chapters/5Experiments/2/main}
\include{chapters/5Experiments/3/main}
\include{chapters/5Experiments/4/main}

\include{chapters/6Conclusion/main}

\appendix
\include{appendix/AppendixB}
\include{appendix/D/main}
\include{appendix/AppendixC}

\backmatter
\bibliographystyle{ecs}
\bibliography{ECS}
\end{document}
%% ----------------------------------------------------------------

\include{chapters/3Design/HighLevel}
\include{chapters/3Design/InDepth}
 %% ----------------------------------------------------------------
%% Progress.tex
%% ---------------------------------------------------------------- 
\documentclass{ecsprogress}    % Use the progress Style
\graphicspath{{../figs/}}   % Location of your graphics files
    \usepackage{natbib}            % Use Natbib style for the refs.
\hypersetup{colorlinks=true}   % Set to false for black/white printing
\input{Definitions}            % Include your abbreviations



\usepackage{enumitem}% http://ctan.org/pkg/enumitem
\usepackage{multirow}
\usepackage{float}
\usepackage{amsmath}
\usepackage{multicol}
\usepackage{amssymb}
\usepackage[normalem]{ulem}
\useunder{\uline}{\ul}{}
\usepackage{wrapfig}


\usepackage[table,xcdraw]{xcolor}


%% ----------------------------------------------------------------
\begin{document}
\frontmatter
\title      {Heterogeneous Agent-based Model for Supermarket Competition}
\authors    {\texorpdfstring
             {\href{mailto:sc22g13@ecs.soton.ac.uk}{Stefan J. Collier}}
             {Stefan J. Collier}
            }
\addresses  {\groupname\\\deptname\\\univname}
\date       {\today}
\subject    {}
\keywords   {}
\supervisor {Dr. Maria Polukarov}
\examiner   {Professor Sheng Chen}

\maketitle
\begin{abstract}
This project aim was to model and analyse the effects of competitive pricing behaviors of grocery retailers on the British market. 

This was achieved by creating a multi-agent model, containing retailer and consumer agents. The heterogeneous crowd of retailers employs either a uniform pricing strategy or a ‘local price flexing’ strategy. The actions of these retailers are chosen by predicting the profit of each action, using a perceptron. Following on from the consideration of different economic models, a discrete model was developed so that software agents have a discrete environment to operate within. Within the model, it has been observed how supermarkets with differing behaviors affect a heterogeneous crowd of consumer agents. The model was implemented in Java with Python used to evaluate the results. 

The simulation displays good acceptance with real grocery market behavior, i.e. captures the performance of British retailers thus can be used to determine the impact of changes in their behavior on their competitors and consumers.Furthermore it can be used to provide insight into sustainability of volatile pricing strategies, providing a useful insight in volatility of British supermarket retail industry. 
\end{abstract}
\acknowledgements{
I would like to express my sincere gratitude to Dr Maria Polukarov for her guidance and support which provided me the freedom to take this research in the direction of my interest.\\
\\
I would also like to thank my family and friends for their encouragement and support. To those who quietly listened to my software complaints. To those who worked throughout the nights with me. To those who helped me write what I couldn't say. I cannot thank you enough.
}

\declaration{
I, Stefan Collier, declare that this dissertation and the work presented in it are my own and has been generated by me as the result of my own original research.\\
I confirm that:\\
1. This work was done wholly or mainly while in candidature for a degree at this University;\\
2. Where any part of this dissertation has previously been submitted for any other qualification at this University or any other institution, this has been clearly stated;\\
3. Where I have consulted the published work of others, this is always clearly attributed;\\
4. Where I have quoted from the work of others, the source is always given. With the exception of such quotations, this dissertation is entirely my own work;\\
5. I have acknowledged all main sources of help;\\
6. Where the thesis is based on work done by myself jointly with others, I have made clear exactly what was done by others and what I have contributed myself;\\
7. Either none of this work has been published before submission, or parts of this work have been published by :\\
\\
Stefan Collier\\
April 2016
}
\tableofcontents
\listoffigures
\listoftables

\mainmatter
%% ----------------------------------------------------------------
%\include{Introduction}
%\include{Conclusions}
\include{chapters/1Project/main}
\include{chapters/2Lit/main}
\include{chapters/3Design/HighLevel}
\include{chapters/3Design/InDepth}
\include{chapters/4Impl/main}

\include{chapters/5Experiments/1/main}
\include{chapters/5Experiments/2/main}
\include{chapters/5Experiments/3/main}
\include{chapters/5Experiments/4/main}

\include{chapters/6Conclusion/main}

\appendix
\include{appendix/AppendixB}
\include{appendix/D/main}
\include{appendix/AppendixC}

\backmatter
\bibliographystyle{ecs}
\bibliography{ECS}
\end{document}
%% ----------------------------------------------------------------


 %% ----------------------------------------------------------------
%% Progress.tex
%% ---------------------------------------------------------------- 
\documentclass{ecsprogress}    % Use the progress Style
\graphicspath{{../figs/}}   % Location of your graphics files
    \usepackage{natbib}            % Use Natbib style for the refs.
\hypersetup{colorlinks=true}   % Set to false for black/white printing
\input{Definitions}            % Include your abbreviations



\usepackage{enumitem}% http://ctan.org/pkg/enumitem
\usepackage{multirow}
\usepackage{float}
\usepackage{amsmath}
\usepackage{multicol}
\usepackage{amssymb}
\usepackage[normalem]{ulem}
\useunder{\uline}{\ul}{}
\usepackage{wrapfig}


\usepackage[table,xcdraw]{xcolor}


%% ----------------------------------------------------------------
\begin{document}
\frontmatter
\title      {Heterogeneous Agent-based Model for Supermarket Competition}
\authors    {\texorpdfstring
             {\href{mailto:sc22g13@ecs.soton.ac.uk}{Stefan J. Collier}}
             {Stefan J. Collier}
            }
\addresses  {\groupname\\\deptname\\\univname}
\date       {\today}
\subject    {}
\keywords   {}
\supervisor {Dr. Maria Polukarov}
\examiner   {Professor Sheng Chen}

\maketitle
\begin{abstract}
This project aim was to model and analyse the effects of competitive pricing behaviors of grocery retailers on the British market. 

This was achieved by creating a multi-agent model, containing retailer and consumer agents. The heterogeneous crowd of retailers employs either a uniform pricing strategy or a ‘local price flexing’ strategy. The actions of these retailers are chosen by predicting the profit of each action, using a perceptron. Following on from the consideration of different economic models, a discrete model was developed so that software agents have a discrete environment to operate within. Within the model, it has been observed how supermarkets with differing behaviors affect a heterogeneous crowd of consumer agents. The model was implemented in Java with Python used to evaluate the results. 

The simulation displays good acceptance with real grocery market behavior, i.e. captures the performance of British retailers thus can be used to determine the impact of changes in their behavior on their competitors and consumers.Furthermore it can be used to provide insight into sustainability of volatile pricing strategies, providing a useful insight in volatility of British supermarket retail industry. 
\end{abstract}
\acknowledgements{
I would like to express my sincere gratitude to Dr Maria Polukarov for her guidance and support which provided me the freedom to take this research in the direction of my interest.\\
\\
I would also like to thank my family and friends for their encouragement and support. To those who quietly listened to my software complaints. To those who worked throughout the nights with me. To those who helped me write what I couldn't say. I cannot thank you enough.
}

\declaration{
I, Stefan Collier, declare that this dissertation and the work presented in it are my own and has been generated by me as the result of my own original research.\\
I confirm that:\\
1. This work was done wholly or mainly while in candidature for a degree at this University;\\
2. Where any part of this dissertation has previously been submitted for any other qualification at this University or any other institution, this has been clearly stated;\\
3. Where I have consulted the published work of others, this is always clearly attributed;\\
4. Where I have quoted from the work of others, the source is always given. With the exception of such quotations, this dissertation is entirely my own work;\\
5. I have acknowledged all main sources of help;\\
6. Where the thesis is based on work done by myself jointly with others, I have made clear exactly what was done by others and what I have contributed myself;\\
7. Either none of this work has been published before submission, or parts of this work have been published by :\\
\\
Stefan Collier\\
April 2016
}
\tableofcontents
\listoffigures
\listoftables

\mainmatter
%% ----------------------------------------------------------------
%\include{Introduction}
%\include{Conclusions}
\include{chapters/1Project/main}
\include{chapters/2Lit/main}
\include{chapters/3Design/HighLevel}
\include{chapters/3Design/InDepth}
\include{chapters/4Impl/main}

\include{chapters/5Experiments/1/main}
\include{chapters/5Experiments/2/main}
\include{chapters/5Experiments/3/main}
\include{chapters/5Experiments/4/main}

\include{chapters/6Conclusion/main}

\appendix
\include{appendix/AppendixB}
\include{appendix/D/main}
\include{appendix/AppendixC}

\backmatter
\bibliographystyle{ecs}
\bibliography{ECS}
\end{document}
%% ----------------------------------------------------------------

 %% ----------------------------------------------------------------
%% Progress.tex
%% ---------------------------------------------------------------- 
\documentclass{ecsprogress}    % Use the progress Style
\graphicspath{{../figs/}}   % Location of your graphics files
    \usepackage{natbib}            % Use Natbib style for the refs.
\hypersetup{colorlinks=true}   % Set to false for black/white printing
\input{Definitions}            % Include your abbreviations



\usepackage{enumitem}% http://ctan.org/pkg/enumitem
\usepackage{multirow}
\usepackage{float}
\usepackage{amsmath}
\usepackage{multicol}
\usepackage{amssymb}
\usepackage[normalem]{ulem}
\useunder{\uline}{\ul}{}
\usepackage{wrapfig}


\usepackage[table,xcdraw]{xcolor}


%% ----------------------------------------------------------------
\begin{document}
\frontmatter
\title      {Heterogeneous Agent-based Model for Supermarket Competition}
\authors    {\texorpdfstring
             {\href{mailto:sc22g13@ecs.soton.ac.uk}{Stefan J. Collier}}
             {Stefan J. Collier}
            }
\addresses  {\groupname\\\deptname\\\univname}
\date       {\today}
\subject    {}
\keywords   {}
\supervisor {Dr. Maria Polukarov}
\examiner   {Professor Sheng Chen}

\maketitle
\begin{abstract}
This project aim was to model and analyse the effects of competitive pricing behaviors of grocery retailers on the British market. 

This was achieved by creating a multi-agent model, containing retailer and consumer agents. The heterogeneous crowd of retailers employs either a uniform pricing strategy or a ‘local price flexing’ strategy. The actions of these retailers are chosen by predicting the profit of each action, using a perceptron. Following on from the consideration of different economic models, a discrete model was developed so that software agents have a discrete environment to operate within. Within the model, it has been observed how supermarkets with differing behaviors affect a heterogeneous crowd of consumer agents. The model was implemented in Java with Python used to evaluate the results. 

The simulation displays good acceptance with real grocery market behavior, i.e. captures the performance of British retailers thus can be used to determine the impact of changes in their behavior on their competitors and consumers.Furthermore it can be used to provide insight into sustainability of volatile pricing strategies, providing a useful insight in volatility of British supermarket retail industry. 
\end{abstract}
\acknowledgements{
I would like to express my sincere gratitude to Dr Maria Polukarov for her guidance and support which provided me the freedom to take this research in the direction of my interest.\\
\\
I would also like to thank my family and friends for their encouragement and support. To those who quietly listened to my software complaints. To those who worked throughout the nights with me. To those who helped me write what I couldn't say. I cannot thank you enough.
}

\declaration{
I, Stefan Collier, declare that this dissertation and the work presented in it are my own and has been generated by me as the result of my own original research.\\
I confirm that:\\
1. This work was done wholly or mainly while in candidature for a degree at this University;\\
2. Where any part of this dissertation has previously been submitted for any other qualification at this University or any other institution, this has been clearly stated;\\
3. Where I have consulted the published work of others, this is always clearly attributed;\\
4. Where I have quoted from the work of others, the source is always given. With the exception of such quotations, this dissertation is entirely my own work;\\
5. I have acknowledged all main sources of help;\\
6. Where the thesis is based on work done by myself jointly with others, I have made clear exactly what was done by others and what I have contributed myself;\\
7. Either none of this work has been published before submission, or parts of this work have been published by :\\
\\
Stefan Collier\\
April 2016
}
\tableofcontents
\listoffigures
\listoftables

\mainmatter
%% ----------------------------------------------------------------
%\include{Introduction}
%\include{Conclusions}
\include{chapters/1Project/main}
\include{chapters/2Lit/main}
\include{chapters/3Design/HighLevel}
\include{chapters/3Design/InDepth}
\include{chapters/4Impl/main}

\include{chapters/5Experiments/1/main}
\include{chapters/5Experiments/2/main}
\include{chapters/5Experiments/3/main}
\include{chapters/5Experiments/4/main}

\include{chapters/6Conclusion/main}

\appendix
\include{appendix/AppendixB}
\include{appendix/D/main}
\include{appendix/AppendixC}

\backmatter
\bibliographystyle{ecs}
\bibliography{ECS}
\end{document}
%% ----------------------------------------------------------------

 %% ----------------------------------------------------------------
%% Progress.tex
%% ---------------------------------------------------------------- 
\documentclass{ecsprogress}    % Use the progress Style
\graphicspath{{../figs/}}   % Location of your graphics files
    \usepackage{natbib}            % Use Natbib style for the refs.
\hypersetup{colorlinks=true}   % Set to false for black/white printing
\input{Definitions}            % Include your abbreviations



\usepackage{enumitem}% http://ctan.org/pkg/enumitem
\usepackage{multirow}
\usepackage{float}
\usepackage{amsmath}
\usepackage{multicol}
\usepackage{amssymb}
\usepackage[normalem]{ulem}
\useunder{\uline}{\ul}{}
\usepackage{wrapfig}


\usepackage[table,xcdraw]{xcolor}


%% ----------------------------------------------------------------
\begin{document}
\frontmatter
\title      {Heterogeneous Agent-based Model for Supermarket Competition}
\authors    {\texorpdfstring
             {\href{mailto:sc22g13@ecs.soton.ac.uk}{Stefan J. Collier}}
             {Stefan J. Collier}
            }
\addresses  {\groupname\\\deptname\\\univname}
\date       {\today}
\subject    {}
\keywords   {}
\supervisor {Dr. Maria Polukarov}
\examiner   {Professor Sheng Chen}

\maketitle
\begin{abstract}
This project aim was to model and analyse the effects of competitive pricing behaviors of grocery retailers on the British market. 

This was achieved by creating a multi-agent model, containing retailer and consumer agents. The heterogeneous crowd of retailers employs either a uniform pricing strategy or a ‘local price flexing’ strategy. The actions of these retailers are chosen by predicting the profit of each action, using a perceptron. Following on from the consideration of different economic models, a discrete model was developed so that software agents have a discrete environment to operate within. Within the model, it has been observed how supermarkets with differing behaviors affect a heterogeneous crowd of consumer agents. The model was implemented in Java with Python used to evaluate the results. 

The simulation displays good acceptance with real grocery market behavior, i.e. captures the performance of British retailers thus can be used to determine the impact of changes in their behavior on their competitors and consumers.Furthermore it can be used to provide insight into sustainability of volatile pricing strategies, providing a useful insight in volatility of British supermarket retail industry. 
\end{abstract}
\acknowledgements{
I would like to express my sincere gratitude to Dr Maria Polukarov for her guidance and support which provided me the freedom to take this research in the direction of my interest.\\
\\
I would also like to thank my family and friends for their encouragement and support. To those who quietly listened to my software complaints. To those who worked throughout the nights with me. To those who helped me write what I couldn't say. I cannot thank you enough.
}

\declaration{
I, Stefan Collier, declare that this dissertation and the work presented in it are my own and has been generated by me as the result of my own original research.\\
I confirm that:\\
1. This work was done wholly or mainly while in candidature for a degree at this University;\\
2. Where any part of this dissertation has previously been submitted for any other qualification at this University or any other institution, this has been clearly stated;\\
3. Where I have consulted the published work of others, this is always clearly attributed;\\
4. Where I have quoted from the work of others, the source is always given. With the exception of such quotations, this dissertation is entirely my own work;\\
5. I have acknowledged all main sources of help;\\
6. Where the thesis is based on work done by myself jointly with others, I have made clear exactly what was done by others and what I have contributed myself;\\
7. Either none of this work has been published before submission, or parts of this work have been published by :\\
\\
Stefan Collier\\
April 2016
}
\tableofcontents
\listoffigures
\listoftables

\mainmatter
%% ----------------------------------------------------------------
%\include{Introduction}
%\include{Conclusions}
\include{chapters/1Project/main}
\include{chapters/2Lit/main}
\include{chapters/3Design/HighLevel}
\include{chapters/3Design/InDepth}
\include{chapters/4Impl/main}

\include{chapters/5Experiments/1/main}
\include{chapters/5Experiments/2/main}
\include{chapters/5Experiments/3/main}
\include{chapters/5Experiments/4/main}

\include{chapters/6Conclusion/main}

\appendix
\include{appendix/AppendixB}
\include{appendix/D/main}
\include{appendix/AppendixC}

\backmatter
\bibliographystyle{ecs}
\bibliography{ECS}
\end{document}
%% ----------------------------------------------------------------

 %% ----------------------------------------------------------------
%% Progress.tex
%% ---------------------------------------------------------------- 
\documentclass{ecsprogress}    % Use the progress Style
\graphicspath{{../figs/}}   % Location of your graphics files
    \usepackage{natbib}            % Use Natbib style for the refs.
\hypersetup{colorlinks=true}   % Set to false for black/white printing
\input{Definitions}            % Include your abbreviations



\usepackage{enumitem}% http://ctan.org/pkg/enumitem
\usepackage{multirow}
\usepackage{float}
\usepackage{amsmath}
\usepackage{multicol}
\usepackage{amssymb}
\usepackage[normalem]{ulem}
\useunder{\uline}{\ul}{}
\usepackage{wrapfig}


\usepackage[table,xcdraw]{xcolor}


%% ----------------------------------------------------------------
\begin{document}
\frontmatter
\title      {Heterogeneous Agent-based Model for Supermarket Competition}
\authors    {\texorpdfstring
             {\href{mailto:sc22g13@ecs.soton.ac.uk}{Stefan J. Collier}}
             {Stefan J. Collier}
            }
\addresses  {\groupname\\\deptname\\\univname}
\date       {\today}
\subject    {}
\keywords   {}
\supervisor {Dr. Maria Polukarov}
\examiner   {Professor Sheng Chen}

\maketitle
\begin{abstract}
This project aim was to model and analyse the effects of competitive pricing behaviors of grocery retailers on the British market. 

This was achieved by creating a multi-agent model, containing retailer and consumer agents. The heterogeneous crowd of retailers employs either a uniform pricing strategy or a ‘local price flexing’ strategy. The actions of these retailers are chosen by predicting the profit of each action, using a perceptron. Following on from the consideration of different economic models, a discrete model was developed so that software agents have a discrete environment to operate within. Within the model, it has been observed how supermarkets with differing behaviors affect a heterogeneous crowd of consumer agents. The model was implemented in Java with Python used to evaluate the results. 

The simulation displays good acceptance with real grocery market behavior, i.e. captures the performance of British retailers thus can be used to determine the impact of changes in their behavior on their competitors and consumers.Furthermore it can be used to provide insight into sustainability of volatile pricing strategies, providing a useful insight in volatility of British supermarket retail industry. 
\end{abstract}
\acknowledgements{
I would like to express my sincere gratitude to Dr Maria Polukarov for her guidance and support which provided me the freedom to take this research in the direction of my interest.\\
\\
I would also like to thank my family and friends for their encouragement and support. To those who quietly listened to my software complaints. To those who worked throughout the nights with me. To those who helped me write what I couldn't say. I cannot thank you enough.
}

\declaration{
I, Stefan Collier, declare that this dissertation and the work presented in it are my own and has been generated by me as the result of my own original research.\\
I confirm that:\\
1. This work was done wholly or mainly while in candidature for a degree at this University;\\
2. Where any part of this dissertation has previously been submitted for any other qualification at this University or any other institution, this has been clearly stated;\\
3. Where I have consulted the published work of others, this is always clearly attributed;\\
4. Where I have quoted from the work of others, the source is always given. With the exception of such quotations, this dissertation is entirely my own work;\\
5. I have acknowledged all main sources of help;\\
6. Where the thesis is based on work done by myself jointly with others, I have made clear exactly what was done by others and what I have contributed myself;\\
7. Either none of this work has been published before submission, or parts of this work have been published by :\\
\\
Stefan Collier\\
April 2016
}
\tableofcontents
\listoffigures
\listoftables

\mainmatter
%% ----------------------------------------------------------------
%\include{Introduction}
%\include{Conclusions}
\include{chapters/1Project/main}
\include{chapters/2Lit/main}
\include{chapters/3Design/HighLevel}
\include{chapters/3Design/InDepth}
\include{chapters/4Impl/main}

\include{chapters/5Experiments/1/main}
\include{chapters/5Experiments/2/main}
\include{chapters/5Experiments/3/main}
\include{chapters/5Experiments/4/main}

\include{chapters/6Conclusion/main}

\appendix
\include{appendix/AppendixB}
\include{appendix/D/main}
\include{appendix/AppendixC}

\backmatter
\bibliographystyle{ecs}
\bibliography{ECS}
\end{document}
%% ----------------------------------------------------------------


 %% ----------------------------------------------------------------
%% Progress.tex
%% ---------------------------------------------------------------- 
\documentclass{ecsprogress}    % Use the progress Style
\graphicspath{{../figs/}}   % Location of your graphics files
    \usepackage{natbib}            % Use Natbib style for the refs.
\hypersetup{colorlinks=true}   % Set to false for black/white printing
\input{Definitions}            % Include your abbreviations



\usepackage{enumitem}% http://ctan.org/pkg/enumitem
\usepackage{multirow}
\usepackage{float}
\usepackage{amsmath}
\usepackage{multicol}
\usepackage{amssymb}
\usepackage[normalem]{ulem}
\useunder{\uline}{\ul}{}
\usepackage{wrapfig}


\usepackage[table,xcdraw]{xcolor}


%% ----------------------------------------------------------------
\begin{document}
\frontmatter
\title      {Heterogeneous Agent-based Model for Supermarket Competition}
\authors    {\texorpdfstring
             {\href{mailto:sc22g13@ecs.soton.ac.uk}{Stefan J. Collier}}
             {Stefan J. Collier}
            }
\addresses  {\groupname\\\deptname\\\univname}
\date       {\today}
\subject    {}
\keywords   {}
\supervisor {Dr. Maria Polukarov}
\examiner   {Professor Sheng Chen}

\maketitle
\begin{abstract}
This project aim was to model and analyse the effects of competitive pricing behaviors of grocery retailers on the British market. 

This was achieved by creating a multi-agent model, containing retailer and consumer agents. The heterogeneous crowd of retailers employs either a uniform pricing strategy or a ‘local price flexing’ strategy. The actions of these retailers are chosen by predicting the profit of each action, using a perceptron. Following on from the consideration of different economic models, a discrete model was developed so that software agents have a discrete environment to operate within. Within the model, it has been observed how supermarkets with differing behaviors affect a heterogeneous crowd of consumer agents. The model was implemented in Java with Python used to evaluate the results. 

The simulation displays good acceptance with real grocery market behavior, i.e. captures the performance of British retailers thus can be used to determine the impact of changes in their behavior on their competitors and consumers.Furthermore it can be used to provide insight into sustainability of volatile pricing strategies, providing a useful insight in volatility of British supermarket retail industry. 
\end{abstract}
\acknowledgements{
I would like to express my sincere gratitude to Dr Maria Polukarov for her guidance and support which provided me the freedom to take this research in the direction of my interest.\\
\\
I would also like to thank my family and friends for their encouragement and support. To those who quietly listened to my software complaints. To those who worked throughout the nights with me. To those who helped me write what I couldn't say. I cannot thank you enough.
}

\declaration{
I, Stefan Collier, declare that this dissertation and the work presented in it are my own and has been generated by me as the result of my own original research.\\
I confirm that:\\
1. This work was done wholly or mainly while in candidature for a degree at this University;\\
2. Where any part of this dissertation has previously been submitted for any other qualification at this University or any other institution, this has been clearly stated;\\
3. Where I have consulted the published work of others, this is always clearly attributed;\\
4. Where I have quoted from the work of others, the source is always given. With the exception of such quotations, this dissertation is entirely my own work;\\
5. I have acknowledged all main sources of help;\\
6. Where the thesis is based on work done by myself jointly with others, I have made clear exactly what was done by others and what I have contributed myself;\\
7. Either none of this work has been published before submission, or parts of this work have been published by :\\
\\
Stefan Collier\\
April 2016
}
\tableofcontents
\listoffigures
\listoftables

\mainmatter
%% ----------------------------------------------------------------
%\include{Introduction}
%\include{Conclusions}
\include{chapters/1Project/main}
\include{chapters/2Lit/main}
\include{chapters/3Design/HighLevel}
\include{chapters/3Design/InDepth}
\include{chapters/4Impl/main}

\include{chapters/5Experiments/1/main}
\include{chapters/5Experiments/2/main}
\include{chapters/5Experiments/3/main}
\include{chapters/5Experiments/4/main}

\include{chapters/6Conclusion/main}

\appendix
\include{appendix/AppendixB}
\include{appendix/D/main}
\include{appendix/AppendixC}

\backmatter
\bibliographystyle{ecs}
\bibliography{ECS}
\end{document}
%% ----------------------------------------------------------------


\appendix
\include{appendix/AppendixB}
 %% ----------------------------------------------------------------
%% Progress.tex
%% ---------------------------------------------------------------- 
\documentclass{ecsprogress}    % Use the progress Style
\graphicspath{{../figs/}}   % Location of your graphics files
    \usepackage{natbib}            % Use Natbib style for the refs.
\hypersetup{colorlinks=true}   % Set to false for black/white printing
\input{Definitions}            % Include your abbreviations



\usepackage{enumitem}% http://ctan.org/pkg/enumitem
\usepackage{multirow}
\usepackage{float}
\usepackage{amsmath}
\usepackage{multicol}
\usepackage{amssymb}
\usepackage[normalem]{ulem}
\useunder{\uline}{\ul}{}
\usepackage{wrapfig}


\usepackage[table,xcdraw]{xcolor}


%% ----------------------------------------------------------------
\begin{document}
\frontmatter
\title      {Heterogeneous Agent-based Model for Supermarket Competition}
\authors    {\texorpdfstring
             {\href{mailto:sc22g13@ecs.soton.ac.uk}{Stefan J. Collier}}
             {Stefan J. Collier}
            }
\addresses  {\groupname\\\deptname\\\univname}
\date       {\today}
\subject    {}
\keywords   {}
\supervisor {Dr. Maria Polukarov}
\examiner   {Professor Sheng Chen}

\maketitle
\begin{abstract}
This project aim was to model and analyse the effects of competitive pricing behaviors of grocery retailers on the British market. 

This was achieved by creating a multi-agent model, containing retailer and consumer agents. The heterogeneous crowd of retailers employs either a uniform pricing strategy or a ‘local price flexing’ strategy. The actions of these retailers are chosen by predicting the profit of each action, using a perceptron. Following on from the consideration of different economic models, a discrete model was developed so that software agents have a discrete environment to operate within. Within the model, it has been observed how supermarkets with differing behaviors affect a heterogeneous crowd of consumer agents. The model was implemented in Java with Python used to evaluate the results. 

The simulation displays good acceptance with real grocery market behavior, i.e. captures the performance of British retailers thus can be used to determine the impact of changes in their behavior on their competitors and consumers.Furthermore it can be used to provide insight into sustainability of volatile pricing strategies, providing a useful insight in volatility of British supermarket retail industry. 
\end{abstract}
\acknowledgements{
I would like to express my sincere gratitude to Dr Maria Polukarov for her guidance and support which provided me the freedom to take this research in the direction of my interest.\\
\\
I would also like to thank my family and friends for their encouragement and support. To those who quietly listened to my software complaints. To those who worked throughout the nights with me. To those who helped me write what I couldn't say. I cannot thank you enough.
}

\declaration{
I, Stefan Collier, declare that this dissertation and the work presented in it are my own and has been generated by me as the result of my own original research.\\
I confirm that:\\
1. This work was done wholly or mainly while in candidature for a degree at this University;\\
2. Where any part of this dissertation has previously been submitted for any other qualification at this University or any other institution, this has been clearly stated;\\
3. Where I have consulted the published work of others, this is always clearly attributed;\\
4. Where I have quoted from the work of others, the source is always given. With the exception of such quotations, this dissertation is entirely my own work;\\
5. I have acknowledged all main sources of help;\\
6. Where the thesis is based on work done by myself jointly with others, I have made clear exactly what was done by others and what I have contributed myself;\\
7. Either none of this work has been published before submission, or parts of this work have been published by :\\
\\
Stefan Collier\\
April 2016
}
\tableofcontents
\listoffigures
\listoftables

\mainmatter
%% ----------------------------------------------------------------
%\include{Introduction}
%\include{Conclusions}
\include{chapters/1Project/main}
\include{chapters/2Lit/main}
\include{chapters/3Design/HighLevel}
\include{chapters/3Design/InDepth}
\include{chapters/4Impl/main}

\include{chapters/5Experiments/1/main}
\include{chapters/5Experiments/2/main}
\include{chapters/5Experiments/3/main}
\include{chapters/5Experiments/4/main}

\include{chapters/6Conclusion/main}

\appendix
\include{appendix/AppendixB}
\include{appendix/D/main}
\include{appendix/AppendixC}

\backmatter
\bibliographystyle{ecs}
\bibliography{ECS}
\end{document}
%% ----------------------------------------------------------------

\include{appendix/AppendixC}

\backmatter
\bibliographystyle{ecs}
\bibliography{ECS}
\end{document}
%% ----------------------------------------------------------------

 %% ----------------------------------------------------------------
%% Progress.tex
%% ---------------------------------------------------------------- 
\documentclass{ecsprogress}    % Use the progress Style
\graphicspath{{../figs/}}   % Location of your graphics files
    \usepackage{natbib}            % Use Natbib style for the refs.
\hypersetup{colorlinks=true}   % Set to false for black/white printing
\input{Definitions}            % Include your abbreviations



\usepackage{enumitem}% http://ctan.org/pkg/enumitem
\usepackage{multirow}
\usepackage{float}
\usepackage{amsmath}
\usepackage{multicol}
\usepackage{amssymb}
\usepackage[normalem]{ulem}
\useunder{\uline}{\ul}{}
\usepackage{wrapfig}


\usepackage[table,xcdraw]{xcolor}


%% ----------------------------------------------------------------
\begin{document}
\frontmatter
\title      {Heterogeneous Agent-based Model for Supermarket Competition}
\authors    {\texorpdfstring
             {\href{mailto:sc22g13@ecs.soton.ac.uk}{Stefan J. Collier}}
             {Stefan J. Collier}
            }
\addresses  {\groupname\\\deptname\\\univname}
\date       {\today}
\subject    {}
\keywords   {}
\supervisor {Dr. Maria Polukarov}
\examiner   {Professor Sheng Chen}

\maketitle
\begin{abstract}
This project aim was to model and analyse the effects of competitive pricing behaviors of grocery retailers on the British market. 

This was achieved by creating a multi-agent model, containing retailer and consumer agents. The heterogeneous crowd of retailers employs either a uniform pricing strategy or a ‘local price flexing’ strategy. The actions of these retailers are chosen by predicting the profit of each action, using a perceptron. Following on from the consideration of different economic models, a discrete model was developed so that software agents have a discrete environment to operate within. Within the model, it has been observed how supermarkets with differing behaviors affect a heterogeneous crowd of consumer agents. The model was implemented in Java with Python used to evaluate the results. 

The simulation displays good acceptance with real grocery market behavior, i.e. captures the performance of British retailers thus can be used to determine the impact of changes in their behavior on their competitors and consumers.Furthermore it can be used to provide insight into sustainability of volatile pricing strategies, providing a useful insight in volatility of British supermarket retail industry. 
\end{abstract}
\acknowledgements{
I would like to express my sincere gratitude to Dr Maria Polukarov for her guidance and support which provided me the freedom to take this research in the direction of my interest.\\
\\
I would also like to thank my family and friends for their encouragement and support. To those who quietly listened to my software complaints. To those who worked throughout the nights with me. To those who helped me write what I couldn't say. I cannot thank you enough.
}

\declaration{
I, Stefan Collier, declare that this dissertation and the work presented in it are my own and has been generated by me as the result of my own original research.\\
I confirm that:\\
1. This work was done wholly or mainly while in candidature for a degree at this University;\\
2. Where any part of this dissertation has previously been submitted for any other qualification at this University or any other institution, this has been clearly stated;\\
3. Where I have consulted the published work of others, this is always clearly attributed;\\
4. Where I have quoted from the work of others, the source is always given. With the exception of such quotations, this dissertation is entirely my own work;\\
5. I have acknowledged all main sources of help;\\
6. Where the thesis is based on work done by myself jointly with others, I have made clear exactly what was done by others and what I have contributed myself;\\
7. Either none of this work has been published before submission, or parts of this work have been published by :\\
\\
Stefan Collier\\
April 2016
}
\tableofcontents
\listoffigures
\listoftables

\mainmatter
%% ----------------------------------------------------------------
%\include{Introduction}
%\include{Conclusions}
 %% ----------------------------------------------------------------
%% Progress.tex
%% ---------------------------------------------------------------- 
\documentclass{ecsprogress}    % Use the progress Style
\graphicspath{{../figs/}}   % Location of your graphics files
    \usepackage{natbib}            % Use Natbib style for the refs.
\hypersetup{colorlinks=true}   % Set to false for black/white printing
\input{Definitions}            % Include your abbreviations



\usepackage{enumitem}% http://ctan.org/pkg/enumitem
\usepackage{multirow}
\usepackage{float}
\usepackage{amsmath}
\usepackage{multicol}
\usepackage{amssymb}
\usepackage[normalem]{ulem}
\useunder{\uline}{\ul}{}
\usepackage{wrapfig}


\usepackage[table,xcdraw]{xcolor}


%% ----------------------------------------------------------------
\begin{document}
\frontmatter
\title      {Heterogeneous Agent-based Model for Supermarket Competition}
\authors    {\texorpdfstring
             {\href{mailto:sc22g13@ecs.soton.ac.uk}{Stefan J. Collier}}
             {Stefan J. Collier}
            }
\addresses  {\groupname\\\deptname\\\univname}
\date       {\today}
\subject    {}
\keywords   {}
\supervisor {Dr. Maria Polukarov}
\examiner   {Professor Sheng Chen}

\maketitle
\begin{abstract}
This project aim was to model and analyse the effects of competitive pricing behaviors of grocery retailers on the British market. 

This was achieved by creating a multi-agent model, containing retailer and consumer agents. The heterogeneous crowd of retailers employs either a uniform pricing strategy or a ‘local price flexing’ strategy. The actions of these retailers are chosen by predicting the profit of each action, using a perceptron. Following on from the consideration of different economic models, a discrete model was developed so that software agents have a discrete environment to operate within. Within the model, it has been observed how supermarkets with differing behaviors affect a heterogeneous crowd of consumer agents. The model was implemented in Java with Python used to evaluate the results. 

The simulation displays good acceptance with real grocery market behavior, i.e. captures the performance of British retailers thus can be used to determine the impact of changes in their behavior on their competitors and consumers.Furthermore it can be used to provide insight into sustainability of volatile pricing strategies, providing a useful insight in volatility of British supermarket retail industry. 
\end{abstract}
\acknowledgements{
I would like to express my sincere gratitude to Dr Maria Polukarov for her guidance and support which provided me the freedom to take this research in the direction of my interest.\\
\\
I would also like to thank my family and friends for their encouragement and support. To those who quietly listened to my software complaints. To those who worked throughout the nights with me. To those who helped me write what I couldn't say. I cannot thank you enough.
}

\declaration{
I, Stefan Collier, declare that this dissertation and the work presented in it are my own and has been generated by me as the result of my own original research.\\
I confirm that:\\
1. This work was done wholly or mainly while in candidature for a degree at this University;\\
2. Where any part of this dissertation has previously been submitted for any other qualification at this University or any other institution, this has been clearly stated;\\
3. Where I have consulted the published work of others, this is always clearly attributed;\\
4. Where I have quoted from the work of others, the source is always given. With the exception of such quotations, this dissertation is entirely my own work;\\
5. I have acknowledged all main sources of help;\\
6. Where the thesis is based on work done by myself jointly with others, I have made clear exactly what was done by others and what I have contributed myself;\\
7. Either none of this work has been published before submission, or parts of this work have been published by :\\
\\
Stefan Collier\\
April 2016
}
\tableofcontents
\listoffigures
\listoftables

\mainmatter
%% ----------------------------------------------------------------
%\include{Introduction}
%\include{Conclusions}
\include{chapters/1Project/main}
\include{chapters/2Lit/main}
\include{chapters/3Design/HighLevel}
\include{chapters/3Design/InDepth}
\include{chapters/4Impl/main}

\include{chapters/5Experiments/1/main}
\include{chapters/5Experiments/2/main}
\include{chapters/5Experiments/3/main}
\include{chapters/5Experiments/4/main}

\include{chapters/6Conclusion/main}

\appendix
\include{appendix/AppendixB}
\include{appendix/D/main}
\include{appendix/AppendixC}

\backmatter
\bibliographystyle{ecs}
\bibliography{ECS}
\end{document}
%% ----------------------------------------------------------------

 %% ----------------------------------------------------------------
%% Progress.tex
%% ---------------------------------------------------------------- 
\documentclass{ecsprogress}    % Use the progress Style
\graphicspath{{../figs/}}   % Location of your graphics files
    \usepackage{natbib}            % Use Natbib style for the refs.
\hypersetup{colorlinks=true}   % Set to false for black/white printing
\input{Definitions}            % Include your abbreviations



\usepackage{enumitem}% http://ctan.org/pkg/enumitem
\usepackage{multirow}
\usepackage{float}
\usepackage{amsmath}
\usepackage{multicol}
\usepackage{amssymb}
\usepackage[normalem]{ulem}
\useunder{\uline}{\ul}{}
\usepackage{wrapfig}


\usepackage[table,xcdraw]{xcolor}


%% ----------------------------------------------------------------
\begin{document}
\frontmatter
\title      {Heterogeneous Agent-based Model for Supermarket Competition}
\authors    {\texorpdfstring
             {\href{mailto:sc22g13@ecs.soton.ac.uk}{Stefan J. Collier}}
             {Stefan J. Collier}
            }
\addresses  {\groupname\\\deptname\\\univname}
\date       {\today}
\subject    {}
\keywords   {}
\supervisor {Dr. Maria Polukarov}
\examiner   {Professor Sheng Chen}

\maketitle
\begin{abstract}
This project aim was to model and analyse the effects of competitive pricing behaviors of grocery retailers on the British market. 

This was achieved by creating a multi-agent model, containing retailer and consumer agents. The heterogeneous crowd of retailers employs either a uniform pricing strategy or a ‘local price flexing’ strategy. The actions of these retailers are chosen by predicting the profit of each action, using a perceptron. Following on from the consideration of different economic models, a discrete model was developed so that software agents have a discrete environment to operate within. Within the model, it has been observed how supermarkets with differing behaviors affect a heterogeneous crowd of consumer agents. The model was implemented in Java with Python used to evaluate the results. 

The simulation displays good acceptance with real grocery market behavior, i.e. captures the performance of British retailers thus can be used to determine the impact of changes in their behavior on their competitors and consumers.Furthermore it can be used to provide insight into sustainability of volatile pricing strategies, providing a useful insight in volatility of British supermarket retail industry. 
\end{abstract}
\acknowledgements{
I would like to express my sincere gratitude to Dr Maria Polukarov for her guidance and support which provided me the freedom to take this research in the direction of my interest.\\
\\
I would also like to thank my family and friends for their encouragement and support. To those who quietly listened to my software complaints. To those who worked throughout the nights with me. To those who helped me write what I couldn't say. I cannot thank you enough.
}

\declaration{
I, Stefan Collier, declare that this dissertation and the work presented in it are my own and has been generated by me as the result of my own original research.\\
I confirm that:\\
1. This work was done wholly or mainly while in candidature for a degree at this University;\\
2. Where any part of this dissertation has previously been submitted for any other qualification at this University or any other institution, this has been clearly stated;\\
3. Where I have consulted the published work of others, this is always clearly attributed;\\
4. Where I have quoted from the work of others, the source is always given. With the exception of such quotations, this dissertation is entirely my own work;\\
5. I have acknowledged all main sources of help;\\
6. Where the thesis is based on work done by myself jointly with others, I have made clear exactly what was done by others and what I have contributed myself;\\
7. Either none of this work has been published before submission, or parts of this work have been published by :\\
\\
Stefan Collier\\
April 2016
}
\tableofcontents
\listoffigures
\listoftables

\mainmatter
%% ----------------------------------------------------------------
%\include{Introduction}
%\include{Conclusions}
\include{chapters/1Project/main}
\include{chapters/2Lit/main}
\include{chapters/3Design/HighLevel}
\include{chapters/3Design/InDepth}
\include{chapters/4Impl/main}

\include{chapters/5Experiments/1/main}
\include{chapters/5Experiments/2/main}
\include{chapters/5Experiments/3/main}
\include{chapters/5Experiments/4/main}

\include{chapters/6Conclusion/main}

\appendix
\include{appendix/AppendixB}
\include{appendix/D/main}
\include{appendix/AppendixC}

\backmatter
\bibliographystyle{ecs}
\bibliography{ECS}
\end{document}
%% ----------------------------------------------------------------

\include{chapters/3Design/HighLevel}
\include{chapters/3Design/InDepth}
 %% ----------------------------------------------------------------
%% Progress.tex
%% ---------------------------------------------------------------- 
\documentclass{ecsprogress}    % Use the progress Style
\graphicspath{{../figs/}}   % Location of your graphics files
    \usepackage{natbib}            % Use Natbib style for the refs.
\hypersetup{colorlinks=true}   % Set to false for black/white printing
\input{Definitions}            % Include your abbreviations



\usepackage{enumitem}% http://ctan.org/pkg/enumitem
\usepackage{multirow}
\usepackage{float}
\usepackage{amsmath}
\usepackage{multicol}
\usepackage{amssymb}
\usepackage[normalem]{ulem}
\useunder{\uline}{\ul}{}
\usepackage{wrapfig}


\usepackage[table,xcdraw]{xcolor}


%% ----------------------------------------------------------------
\begin{document}
\frontmatter
\title      {Heterogeneous Agent-based Model for Supermarket Competition}
\authors    {\texorpdfstring
             {\href{mailto:sc22g13@ecs.soton.ac.uk}{Stefan J. Collier}}
             {Stefan J. Collier}
            }
\addresses  {\groupname\\\deptname\\\univname}
\date       {\today}
\subject    {}
\keywords   {}
\supervisor {Dr. Maria Polukarov}
\examiner   {Professor Sheng Chen}

\maketitle
\begin{abstract}
This project aim was to model and analyse the effects of competitive pricing behaviors of grocery retailers on the British market. 

This was achieved by creating a multi-agent model, containing retailer and consumer agents. The heterogeneous crowd of retailers employs either a uniform pricing strategy or a ‘local price flexing’ strategy. The actions of these retailers are chosen by predicting the profit of each action, using a perceptron. Following on from the consideration of different economic models, a discrete model was developed so that software agents have a discrete environment to operate within. Within the model, it has been observed how supermarkets with differing behaviors affect a heterogeneous crowd of consumer agents. The model was implemented in Java with Python used to evaluate the results. 

The simulation displays good acceptance with real grocery market behavior, i.e. captures the performance of British retailers thus can be used to determine the impact of changes in their behavior on their competitors and consumers.Furthermore it can be used to provide insight into sustainability of volatile pricing strategies, providing a useful insight in volatility of British supermarket retail industry. 
\end{abstract}
\acknowledgements{
I would like to express my sincere gratitude to Dr Maria Polukarov for her guidance and support which provided me the freedom to take this research in the direction of my interest.\\
\\
I would also like to thank my family and friends for their encouragement and support. To those who quietly listened to my software complaints. To those who worked throughout the nights with me. To those who helped me write what I couldn't say. I cannot thank you enough.
}

\declaration{
I, Stefan Collier, declare that this dissertation and the work presented in it are my own and has been generated by me as the result of my own original research.\\
I confirm that:\\
1. This work was done wholly or mainly while in candidature for a degree at this University;\\
2. Where any part of this dissertation has previously been submitted for any other qualification at this University or any other institution, this has been clearly stated;\\
3. Where I have consulted the published work of others, this is always clearly attributed;\\
4. Where I have quoted from the work of others, the source is always given. With the exception of such quotations, this dissertation is entirely my own work;\\
5. I have acknowledged all main sources of help;\\
6. Where the thesis is based on work done by myself jointly with others, I have made clear exactly what was done by others and what I have contributed myself;\\
7. Either none of this work has been published before submission, or parts of this work have been published by :\\
\\
Stefan Collier\\
April 2016
}
\tableofcontents
\listoffigures
\listoftables

\mainmatter
%% ----------------------------------------------------------------
%\include{Introduction}
%\include{Conclusions}
\include{chapters/1Project/main}
\include{chapters/2Lit/main}
\include{chapters/3Design/HighLevel}
\include{chapters/3Design/InDepth}
\include{chapters/4Impl/main}

\include{chapters/5Experiments/1/main}
\include{chapters/5Experiments/2/main}
\include{chapters/5Experiments/3/main}
\include{chapters/5Experiments/4/main}

\include{chapters/6Conclusion/main}

\appendix
\include{appendix/AppendixB}
\include{appendix/D/main}
\include{appendix/AppendixC}

\backmatter
\bibliographystyle{ecs}
\bibliography{ECS}
\end{document}
%% ----------------------------------------------------------------


 %% ----------------------------------------------------------------
%% Progress.tex
%% ---------------------------------------------------------------- 
\documentclass{ecsprogress}    % Use the progress Style
\graphicspath{{../figs/}}   % Location of your graphics files
    \usepackage{natbib}            % Use Natbib style for the refs.
\hypersetup{colorlinks=true}   % Set to false for black/white printing
\input{Definitions}            % Include your abbreviations



\usepackage{enumitem}% http://ctan.org/pkg/enumitem
\usepackage{multirow}
\usepackage{float}
\usepackage{amsmath}
\usepackage{multicol}
\usepackage{amssymb}
\usepackage[normalem]{ulem}
\useunder{\uline}{\ul}{}
\usepackage{wrapfig}


\usepackage[table,xcdraw]{xcolor}


%% ----------------------------------------------------------------
\begin{document}
\frontmatter
\title      {Heterogeneous Agent-based Model for Supermarket Competition}
\authors    {\texorpdfstring
             {\href{mailto:sc22g13@ecs.soton.ac.uk}{Stefan J. Collier}}
             {Stefan J. Collier}
            }
\addresses  {\groupname\\\deptname\\\univname}
\date       {\today}
\subject    {}
\keywords   {}
\supervisor {Dr. Maria Polukarov}
\examiner   {Professor Sheng Chen}

\maketitle
\begin{abstract}
This project aim was to model and analyse the effects of competitive pricing behaviors of grocery retailers on the British market. 

This was achieved by creating a multi-agent model, containing retailer and consumer agents. The heterogeneous crowd of retailers employs either a uniform pricing strategy or a ‘local price flexing’ strategy. The actions of these retailers are chosen by predicting the profit of each action, using a perceptron. Following on from the consideration of different economic models, a discrete model was developed so that software agents have a discrete environment to operate within. Within the model, it has been observed how supermarkets with differing behaviors affect a heterogeneous crowd of consumer agents. The model was implemented in Java with Python used to evaluate the results. 

The simulation displays good acceptance with real grocery market behavior, i.e. captures the performance of British retailers thus can be used to determine the impact of changes in their behavior on their competitors and consumers.Furthermore it can be used to provide insight into sustainability of volatile pricing strategies, providing a useful insight in volatility of British supermarket retail industry. 
\end{abstract}
\acknowledgements{
I would like to express my sincere gratitude to Dr Maria Polukarov for her guidance and support which provided me the freedom to take this research in the direction of my interest.\\
\\
I would also like to thank my family and friends for their encouragement and support. To those who quietly listened to my software complaints. To those who worked throughout the nights with me. To those who helped me write what I couldn't say. I cannot thank you enough.
}

\declaration{
I, Stefan Collier, declare that this dissertation and the work presented in it are my own and has been generated by me as the result of my own original research.\\
I confirm that:\\
1. This work was done wholly or mainly while in candidature for a degree at this University;\\
2. Where any part of this dissertation has previously been submitted for any other qualification at this University or any other institution, this has been clearly stated;\\
3. Where I have consulted the published work of others, this is always clearly attributed;\\
4. Where I have quoted from the work of others, the source is always given. With the exception of such quotations, this dissertation is entirely my own work;\\
5. I have acknowledged all main sources of help;\\
6. Where the thesis is based on work done by myself jointly with others, I have made clear exactly what was done by others and what I have contributed myself;\\
7. Either none of this work has been published before submission, or parts of this work have been published by :\\
\\
Stefan Collier\\
April 2016
}
\tableofcontents
\listoffigures
\listoftables

\mainmatter
%% ----------------------------------------------------------------
%\include{Introduction}
%\include{Conclusions}
\include{chapters/1Project/main}
\include{chapters/2Lit/main}
\include{chapters/3Design/HighLevel}
\include{chapters/3Design/InDepth}
\include{chapters/4Impl/main}

\include{chapters/5Experiments/1/main}
\include{chapters/5Experiments/2/main}
\include{chapters/5Experiments/3/main}
\include{chapters/5Experiments/4/main}

\include{chapters/6Conclusion/main}

\appendix
\include{appendix/AppendixB}
\include{appendix/D/main}
\include{appendix/AppendixC}

\backmatter
\bibliographystyle{ecs}
\bibliography{ECS}
\end{document}
%% ----------------------------------------------------------------

 %% ----------------------------------------------------------------
%% Progress.tex
%% ---------------------------------------------------------------- 
\documentclass{ecsprogress}    % Use the progress Style
\graphicspath{{../figs/}}   % Location of your graphics files
    \usepackage{natbib}            % Use Natbib style for the refs.
\hypersetup{colorlinks=true}   % Set to false for black/white printing
\input{Definitions}            % Include your abbreviations



\usepackage{enumitem}% http://ctan.org/pkg/enumitem
\usepackage{multirow}
\usepackage{float}
\usepackage{amsmath}
\usepackage{multicol}
\usepackage{amssymb}
\usepackage[normalem]{ulem}
\useunder{\uline}{\ul}{}
\usepackage{wrapfig}


\usepackage[table,xcdraw]{xcolor}


%% ----------------------------------------------------------------
\begin{document}
\frontmatter
\title      {Heterogeneous Agent-based Model for Supermarket Competition}
\authors    {\texorpdfstring
             {\href{mailto:sc22g13@ecs.soton.ac.uk}{Stefan J. Collier}}
             {Stefan J. Collier}
            }
\addresses  {\groupname\\\deptname\\\univname}
\date       {\today}
\subject    {}
\keywords   {}
\supervisor {Dr. Maria Polukarov}
\examiner   {Professor Sheng Chen}

\maketitle
\begin{abstract}
This project aim was to model and analyse the effects of competitive pricing behaviors of grocery retailers on the British market. 

This was achieved by creating a multi-agent model, containing retailer and consumer agents. The heterogeneous crowd of retailers employs either a uniform pricing strategy or a ‘local price flexing’ strategy. The actions of these retailers are chosen by predicting the profit of each action, using a perceptron. Following on from the consideration of different economic models, a discrete model was developed so that software agents have a discrete environment to operate within. Within the model, it has been observed how supermarkets with differing behaviors affect a heterogeneous crowd of consumer agents. The model was implemented in Java with Python used to evaluate the results. 

The simulation displays good acceptance with real grocery market behavior, i.e. captures the performance of British retailers thus can be used to determine the impact of changes in their behavior on their competitors and consumers.Furthermore it can be used to provide insight into sustainability of volatile pricing strategies, providing a useful insight in volatility of British supermarket retail industry. 
\end{abstract}
\acknowledgements{
I would like to express my sincere gratitude to Dr Maria Polukarov for her guidance and support which provided me the freedom to take this research in the direction of my interest.\\
\\
I would also like to thank my family and friends for their encouragement and support. To those who quietly listened to my software complaints. To those who worked throughout the nights with me. To those who helped me write what I couldn't say. I cannot thank you enough.
}

\declaration{
I, Stefan Collier, declare that this dissertation and the work presented in it are my own and has been generated by me as the result of my own original research.\\
I confirm that:\\
1. This work was done wholly or mainly while in candidature for a degree at this University;\\
2. Where any part of this dissertation has previously been submitted for any other qualification at this University or any other institution, this has been clearly stated;\\
3. Where I have consulted the published work of others, this is always clearly attributed;\\
4. Where I have quoted from the work of others, the source is always given. With the exception of such quotations, this dissertation is entirely my own work;\\
5. I have acknowledged all main sources of help;\\
6. Where the thesis is based on work done by myself jointly with others, I have made clear exactly what was done by others and what I have contributed myself;\\
7. Either none of this work has been published before submission, or parts of this work have been published by :\\
\\
Stefan Collier\\
April 2016
}
\tableofcontents
\listoffigures
\listoftables

\mainmatter
%% ----------------------------------------------------------------
%\include{Introduction}
%\include{Conclusions}
\include{chapters/1Project/main}
\include{chapters/2Lit/main}
\include{chapters/3Design/HighLevel}
\include{chapters/3Design/InDepth}
\include{chapters/4Impl/main}

\include{chapters/5Experiments/1/main}
\include{chapters/5Experiments/2/main}
\include{chapters/5Experiments/3/main}
\include{chapters/5Experiments/4/main}

\include{chapters/6Conclusion/main}

\appendix
\include{appendix/AppendixB}
\include{appendix/D/main}
\include{appendix/AppendixC}

\backmatter
\bibliographystyle{ecs}
\bibliography{ECS}
\end{document}
%% ----------------------------------------------------------------

 %% ----------------------------------------------------------------
%% Progress.tex
%% ---------------------------------------------------------------- 
\documentclass{ecsprogress}    % Use the progress Style
\graphicspath{{../figs/}}   % Location of your graphics files
    \usepackage{natbib}            % Use Natbib style for the refs.
\hypersetup{colorlinks=true}   % Set to false for black/white printing
\input{Definitions}            % Include your abbreviations



\usepackage{enumitem}% http://ctan.org/pkg/enumitem
\usepackage{multirow}
\usepackage{float}
\usepackage{amsmath}
\usepackage{multicol}
\usepackage{amssymb}
\usepackage[normalem]{ulem}
\useunder{\uline}{\ul}{}
\usepackage{wrapfig}


\usepackage[table,xcdraw]{xcolor}


%% ----------------------------------------------------------------
\begin{document}
\frontmatter
\title      {Heterogeneous Agent-based Model for Supermarket Competition}
\authors    {\texorpdfstring
             {\href{mailto:sc22g13@ecs.soton.ac.uk}{Stefan J. Collier}}
             {Stefan J. Collier}
            }
\addresses  {\groupname\\\deptname\\\univname}
\date       {\today}
\subject    {}
\keywords   {}
\supervisor {Dr. Maria Polukarov}
\examiner   {Professor Sheng Chen}

\maketitle
\begin{abstract}
This project aim was to model and analyse the effects of competitive pricing behaviors of grocery retailers on the British market. 

This was achieved by creating a multi-agent model, containing retailer and consumer agents. The heterogeneous crowd of retailers employs either a uniform pricing strategy or a ‘local price flexing’ strategy. The actions of these retailers are chosen by predicting the profit of each action, using a perceptron. Following on from the consideration of different economic models, a discrete model was developed so that software agents have a discrete environment to operate within. Within the model, it has been observed how supermarkets with differing behaviors affect a heterogeneous crowd of consumer agents. The model was implemented in Java with Python used to evaluate the results. 

The simulation displays good acceptance with real grocery market behavior, i.e. captures the performance of British retailers thus can be used to determine the impact of changes in their behavior on their competitors and consumers.Furthermore it can be used to provide insight into sustainability of volatile pricing strategies, providing a useful insight in volatility of British supermarket retail industry. 
\end{abstract}
\acknowledgements{
I would like to express my sincere gratitude to Dr Maria Polukarov for her guidance and support which provided me the freedom to take this research in the direction of my interest.\\
\\
I would also like to thank my family and friends for their encouragement and support. To those who quietly listened to my software complaints. To those who worked throughout the nights with me. To those who helped me write what I couldn't say. I cannot thank you enough.
}

\declaration{
I, Stefan Collier, declare that this dissertation and the work presented in it are my own and has been generated by me as the result of my own original research.\\
I confirm that:\\
1. This work was done wholly or mainly while in candidature for a degree at this University;\\
2. Where any part of this dissertation has previously been submitted for any other qualification at this University or any other institution, this has been clearly stated;\\
3. Where I have consulted the published work of others, this is always clearly attributed;\\
4. Where I have quoted from the work of others, the source is always given. With the exception of such quotations, this dissertation is entirely my own work;\\
5. I have acknowledged all main sources of help;\\
6. Where the thesis is based on work done by myself jointly with others, I have made clear exactly what was done by others and what I have contributed myself;\\
7. Either none of this work has been published before submission, or parts of this work have been published by :\\
\\
Stefan Collier\\
April 2016
}
\tableofcontents
\listoffigures
\listoftables

\mainmatter
%% ----------------------------------------------------------------
%\include{Introduction}
%\include{Conclusions}
\include{chapters/1Project/main}
\include{chapters/2Lit/main}
\include{chapters/3Design/HighLevel}
\include{chapters/3Design/InDepth}
\include{chapters/4Impl/main}

\include{chapters/5Experiments/1/main}
\include{chapters/5Experiments/2/main}
\include{chapters/5Experiments/3/main}
\include{chapters/5Experiments/4/main}

\include{chapters/6Conclusion/main}

\appendix
\include{appendix/AppendixB}
\include{appendix/D/main}
\include{appendix/AppendixC}

\backmatter
\bibliographystyle{ecs}
\bibliography{ECS}
\end{document}
%% ----------------------------------------------------------------

 %% ----------------------------------------------------------------
%% Progress.tex
%% ---------------------------------------------------------------- 
\documentclass{ecsprogress}    % Use the progress Style
\graphicspath{{../figs/}}   % Location of your graphics files
    \usepackage{natbib}            % Use Natbib style for the refs.
\hypersetup{colorlinks=true}   % Set to false for black/white printing
\input{Definitions}            % Include your abbreviations



\usepackage{enumitem}% http://ctan.org/pkg/enumitem
\usepackage{multirow}
\usepackage{float}
\usepackage{amsmath}
\usepackage{multicol}
\usepackage{amssymb}
\usepackage[normalem]{ulem}
\useunder{\uline}{\ul}{}
\usepackage{wrapfig}


\usepackage[table,xcdraw]{xcolor}


%% ----------------------------------------------------------------
\begin{document}
\frontmatter
\title      {Heterogeneous Agent-based Model for Supermarket Competition}
\authors    {\texorpdfstring
             {\href{mailto:sc22g13@ecs.soton.ac.uk}{Stefan J. Collier}}
             {Stefan J. Collier}
            }
\addresses  {\groupname\\\deptname\\\univname}
\date       {\today}
\subject    {}
\keywords   {}
\supervisor {Dr. Maria Polukarov}
\examiner   {Professor Sheng Chen}

\maketitle
\begin{abstract}
This project aim was to model and analyse the effects of competitive pricing behaviors of grocery retailers on the British market. 

This was achieved by creating a multi-agent model, containing retailer and consumer agents. The heterogeneous crowd of retailers employs either a uniform pricing strategy or a ‘local price flexing’ strategy. The actions of these retailers are chosen by predicting the profit of each action, using a perceptron. Following on from the consideration of different economic models, a discrete model was developed so that software agents have a discrete environment to operate within. Within the model, it has been observed how supermarkets with differing behaviors affect a heterogeneous crowd of consumer agents. The model was implemented in Java with Python used to evaluate the results. 

The simulation displays good acceptance with real grocery market behavior, i.e. captures the performance of British retailers thus can be used to determine the impact of changes in their behavior on their competitors and consumers.Furthermore it can be used to provide insight into sustainability of volatile pricing strategies, providing a useful insight in volatility of British supermarket retail industry. 
\end{abstract}
\acknowledgements{
I would like to express my sincere gratitude to Dr Maria Polukarov for her guidance and support which provided me the freedom to take this research in the direction of my interest.\\
\\
I would also like to thank my family and friends for their encouragement and support. To those who quietly listened to my software complaints. To those who worked throughout the nights with me. To those who helped me write what I couldn't say. I cannot thank you enough.
}

\declaration{
I, Stefan Collier, declare that this dissertation and the work presented in it are my own and has been generated by me as the result of my own original research.\\
I confirm that:\\
1. This work was done wholly or mainly while in candidature for a degree at this University;\\
2. Where any part of this dissertation has previously been submitted for any other qualification at this University or any other institution, this has been clearly stated;\\
3. Where I have consulted the published work of others, this is always clearly attributed;\\
4. Where I have quoted from the work of others, the source is always given. With the exception of such quotations, this dissertation is entirely my own work;\\
5. I have acknowledged all main sources of help;\\
6. Where the thesis is based on work done by myself jointly with others, I have made clear exactly what was done by others and what I have contributed myself;\\
7. Either none of this work has been published before submission, or parts of this work have been published by :\\
\\
Stefan Collier\\
April 2016
}
\tableofcontents
\listoffigures
\listoftables

\mainmatter
%% ----------------------------------------------------------------
%\include{Introduction}
%\include{Conclusions}
\include{chapters/1Project/main}
\include{chapters/2Lit/main}
\include{chapters/3Design/HighLevel}
\include{chapters/3Design/InDepth}
\include{chapters/4Impl/main}

\include{chapters/5Experiments/1/main}
\include{chapters/5Experiments/2/main}
\include{chapters/5Experiments/3/main}
\include{chapters/5Experiments/4/main}

\include{chapters/6Conclusion/main}

\appendix
\include{appendix/AppendixB}
\include{appendix/D/main}
\include{appendix/AppendixC}

\backmatter
\bibliographystyle{ecs}
\bibliography{ECS}
\end{document}
%% ----------------------------------------------------------------


 %% ----------------------------------------------------------------
%% Progress.tex
%% ---------------------------------------------------------------- 
\documentclass{ecsprogress}    % Use the progress Style
\graphicspath{{../figs/}}   % Location of your graphics files
    \usepackage{natbib}            % Use Natbib style for the refs.
\hypersetup{colorlinks=true}   % Set to false for black/white printing
\input{Definitions}            % Include your abbreviations



\usepackage{enumitem}% http://ctan.org/pkg/enumitem
\usepackage{multirow}
\usepackage{float}
\usepackage{amsmath}
\usepackage{multicol}
\usepackage{amssymb}
\usepackage[normalem]{ulem}
\useunder{\uline}{\ul}{}
\usepackage{wrapfig}


\usepackage[table,xcdraw]{xcolor}


%% ----------------------------------------------------------------
\begin{document}
\frontmatter
\title      {Heterogeneous Agent-based Model for Supermarket Competition}
\authors    {\texorpdfstring
             {\href{mailto:sc22g13@ecs.soton.ac.uk}{Stefan J. Collier}}
             {Stefan J. Collier}
            }
\addresses  {\groupname\\\deptname\\\univname}
\date       {\today}
\subject    {}
\keywords   {}
\supervisor {Dr. Maria Polukarov}
\examiner   {Professor Sheng Chen}

\maketitle
\begin{abstract}
This project aim was to model and analyse the effects of competitive pricing behaviors of grocery retailers on the British market. 

This was achieved by creating a multi-agent model, containing retailer and consumer agents. The heterogeneous crowd of retailers employs either a uniform pricing strategy or a ‘local price flexing’ strategy. The actions of these retailers are chosen by predicting the profit of each action, using a perceptron. Following on from the consideration of different economic models, a discrete model was developed so that software agents have a discrete environment to operate within. Within the model, it has been observed how supermarkets with differing behaviors affect a heterogeneous crowd of consumer agents. The model was implemented in Java with Python used to evaluate the results. 

The simulation displays good acceptance with real grocery market behavior, i.e. captures the performance of British retailers thus can be used to determine the impact of changes in their behavior on their competitors and consumers.Furthermore it can be used to provide insight into sustainability of volatile pricing strategies, providing a useful insight in volatility of British supermarket retail industry. 
\end{abstract}
\acknowledgements{
I would like to express my sincere gratitude to Dr Maria Polukarov for her guidance and support which provided me the freedom to take this research in the direction of my interest.\\
\\
I would also like to thank my family and friends for their encouragement and support. To those who quietly listened to my software complaints. To those who worked throughout the nights with me. To those who helped me write what I couldn't say. I cannot thank you enough.
}

\declaration{
I, Stefan Collier, declare that this dissertation and the work presented in it are my own and has been generated by me as the result of my own original research.\\
I confirm that:\\
1. This work was done wholly or mainly while in candidature for a degree at this University;\\
2. Where any part of this dissertation has previously been submitted for any other qualification at this University or any other institution, this has been clearly stated;\\
3. Where I have consulted the published work of others, this is always clearly attributed;\\
4. Where I have quoted from the work of others, the source is always given. With the exception of such quotations, this dissertation is entirely my own work;\\
5. I have acknowledged all main sources of help;\\
6. Where the thesis is based on work done by myself jointly with others, I have made clear exactly what was done by others and what I have contributed myself;\\
7. Either none of this work has been published before submission, or parts of this work have been published by :\\
\\
Stefan Collier\\
April 2016
}
\tableofcontents
\listoffigures
\listoftables

\mainmatter
%% ----------------------------------------------------------------
%\include{Introduction}
%\include{Conclusions}
\include{chapters/1Project/main}
\include{chapters/2Lit/main}
\include{chapters/3Design/HighLevel}
\include{chapters/3Design/InDepth}
\include{chapters/4Impl/main}

\include{chapters/5Experiments/1/main}
\include{chapters/5Experiments/2/main}
\include{chapters/5Experiments/3/main}
\include{chapters/5Experiments/4/main}

\include{chapters/6Conclusion/main}

\appendix
\include{appendix/AppendixB}
\include{appendix/D/main}
\include{appendix/AppendixC}

\backmatter
\bibliographystyle{ecs}
\bibliography{ECS}
\end{document}
%% ----------------------------------------------------------------


\appendix
\include{appendix/AppendixB}
 %% ----------------------------------------------------------------
%% Progress.tex
%% ---------------------------------------------------------------- 
\documentclass{ecsprogress}    % Use the progress Style
\graphicspath{{../figs/}}   % Location of your graphics files
    \usepackage{natbib}            % Use Natbib style for the refs.
\hypersetup{colorlinks=true}   % Set to false for black/white printing
\input{Definitions}            % Include your abbreviations



\usepackage{enumitem}% http://ctan.org/pkg/enumitem
\usepackage{multirow}
\usepackage{float}
\usepackage{amsmath}
\usepackage{multicol}
\usepackage{amssymb}
\usepackage[normalem]{ulem}
\useunder{\uline}{\ul}{}
\usepackage{wrapfig}


\usepackage[table,xcdraw]{xcolor}


%% ----------------------------------------------------------------
\begin{document}
\frontmatter
\title      {Heterogeneous Agent-based Model for Supermarket Competition}
\authors    {\texorpdfstring
             {\href{mailto:sc22g13@ecs.soton.ac.uk}{Stefan J. Collier}}
             {Stefan J. Collier}
            }
\addresses  {\groupname\\\deptname\\\univname}
\date       {\today}
\subject    {}
\keywords   {}
\supervisor {Dr. Maria Polukarov}
\examiner   {Professor Sheng Chen}

\maketitle
\begin{abstract}
This project aim was to model and analyse the effects of competitive pricing behaviors of grocery retailers on the British market. 

This was achieved by creating a multi-agent model, containing retailer and consumer agents. The heterogeneous crowd of retailers employs either a uniform pricing strategy or a ‘local price flexing’ strategy. The actions of these retailers are chosen by predicting the profit of each action, using a perceptron. Following on from the consideration of different economic models, a discrete model was developed so that software agents have a discrete environment to operate within. Within the model, it has been observed how supermarkets with differing behaviors affect a heterogeneous crowd of consumer agents. The model was implemented in Java with Python used to evaluate the results. 

The simulation displays good acceptance with real grocery market behavior, i.e. captures the performance of British retailers thus can be used to determine the impact of changes in their behavior on their competitors and consumers.Furthermore it can be used to provide insight into sustainability of volatile pricing strategies, providing a useful insight in volatility of British supermarket retail industry. 
\end{abstract}
\acknowledgements{
I would like to express my sincere gratitude to Dr Maria Polukarov for her guidance and support which provided me the freedom to take this research in the direction of my interest.\\
\\
I would also like to thank my family and friends for their encouragement and support. To those who quietly listened to my software complaints. To those who worked throughout the nights with me. To those who helped me write what I couldn't say. I cannot thank you enough.
}

\declaration{
I, Stefan Collier, declare that this dissertation and the work presented in it are my own and has been generated by me as the result of my own original research.\\
I confirm that:\\
1. This work was done wholly or mainly while in candidature for a degree at this University;\\
2. Where any part of this dissertation has previously been submitted for any other qualification at this University or any other institution, this has been clearly stated;\\
3. Where I have consulted the published work of others, this is always clearly attributed;\\
4. Where I have quoted from the work of others, the source is always given. With the exception of such quotations, this dissertation is entirely my own work;\\
5. I have acknowledged all main sources of help;\\
6. Where the thesis is based on work done by myself jointly with others, I have made clear exactly what was done by others and what I have contributed myself;\\
7. Either none of this work has been published before submission, or parts of this work have been published by :\\
\\
Stefan Collier\\
April 2016
}
\tableofcontents
\listoffigures
\listoftables

\mainmatter
%% ----------------------------------------------------------------
%\include{Introduction}
%\include{Conclusions}
\include{chapters/1Project/main}
\include{chapters/2Lit/main}
\include{chapters/3Design/HighLevel}
\include{chapters/3Design/InDepth}
\include{chapters/4Impl/main}

\include{chapters/5Experiments/1/main}
\include{chapters/5Experiments/2/main}
\include{chapters/5Experiments/3/main}
\include{chapters/5Experiments/4/main}

\include{chapters/6Conclusion/main}

\appendix
\include{appendix/AppendixB}
\include{appendix/D/main}
\include{appendix/AppendixC}

\backmatter
\bibliographystyle{ecs}
\bibliography{ECS}
\end{document}
%% ----------------------------------------------------------------

\include{appendix/AppendixC}

\backmatter
\bibliographystyle{ecs}
\bibliography{ECS}
\end{document}
%% ----------------------------------------------------------------

 %% ----------------------------------------------------------------
%% Progress.tex
%% ---------------------------------------------------------------- 
\documentclass{ecsprogress}    % Use the progress Style
\graphicspath{{../figs/}}   % Location of your graphics files
    \usepackage{natbib}            % Use Natbib style for the refs.
\hypersetup{colorlinks=true}   % Set to false for black/white printing
\input{Definitions}            % Include your abbreviations



\usepackage{enumitem}% http://ctan.org/pkg/enumitem
\usepackage{multirow}
\usepackage{float}
\usepackage{amsmath}
\usepackage{multicol}
\usepackage{amssymb}
\usepackage[normalem]{ulem}
\useunder{\uline}{\ul}{}
\usepackage{wrapfig}


\usepackage[table,xcdraw]{xcolor}


%% ----------------------------------------------------------------
\begin{document}
\frontmatter
\title      {Heterogeneous Agent-based Model for Supermarket Competition}
\authors    {\texorpdfstring
             {\href{mailto:sc22g13@ecs.soton.ac.uk}{Stefan J. Collier}}
             {Stefan J. Collier}
            }
\addresses  {\groupname\\\deptname\\\univname}
\date       {\today}
\subject    {}
\keywords   {}
\supervisor {Dr. Maria Polukarov}
\examiner   {Professor Sheng Chen}

\maketitle
\begin{abstract}
This project aim was to model and analyse the effects of competitive pricing behaviors of grocery retailers on the British market. 

This was achieved by creating a multi-agent model, containing retailer and consumer agents. The heterogeneous crowd of retailers employs either a uniform pricing strategy or a ‘local price flexing’ strategy. The actions of these retailers are chosen by predicting the profit of each action, using a perceptron. Following on from the consideration of different economic models, a discrete model was developed so that software agents have a discrete environment to operate within. Within the model, it has been observed how supermarkets with differing behaviors affect a heterogeneous crowd of consumer agents. The model was implemented in Java with Python used to evaluate the results. 

The simulation displays good acceptance with real grocery market behavior, i.e. captures the performance of British retailers thus can be used to determine the impact of changes in their behavior on their competitors and consumers.Furthermore it can be used to provide insight into sustainability of volatile pricing strategies, providing a useful insight in volatility of British supermarket retail industry. 
\end{abstract}
\acknowledgements{
I would like to express my sincere gratitude to Dr Maria Polukarov for her guidance and support which provided me the freedom to take this research in the direction of my interest.\\
\\
I would also like to thank my family and friends for their encouragement and support. To those who quietly listened to my software complaints. To those who worked throughout the nights with me. To those who helped me write what I couldn't say. I cannot thank you enough.
}

\declaration{
I, Stefan Collier, declare that this dissertation and the work presented in it are my own and has been generated by me as the result of my own original research.\\
I confirm that:\\
1. This work was done wholly or mainly while in candidature for a degree at this University;\\
2. Where any part of this dissertation has previously been submitted for any other qualification at this University or any other institution, this has been clearly stated;\\
3. Where I have consulted the published work of others, this is always clearly attributed;\\
4. Where I have quoted from the work of others, the source is always given. With the exception of such quotations, this dissertation is entirely my own work;\\
5. I have acknowledged all main sources of help;\\
6. Where the thesis is based on work done by myself jointly with others, I have made clear exactly what was done by others and what I have contributed myself;\\
7. Either none of this work has been published before submission, or parts of this work have been published by :\\
\\
Stefan Collier\\
April 2016
}
\tableofcontents
\listoffigures
\listoftables

\mainmatter
%% ----------------------------------------------------------------
%\include{Introduction}
%\include{Conclusions}
 %% ----------------------------------------------------------------
%% Progress.tex
%% ---------------------------------------------------------------- 
\documentclass{ecsprogress}    % Use the progress Style
\graphicspath{{../figs/}}   % Location of your graphics files
    \usepackage{natbib}            % Use Natbib style for the refs.
\hypersetup{colorlinks=true}   % Set to false for black/white printing
\input{Definitions}            % Include your abbreviations



\usepackage{enumitem}% http://ctan.org/pkg/enumitem
\usepackage{multirow}
\usepackage{float}
\usepackage{amsmath}
\usepackage{multicol}
\usepackage{amssymb}
\usepackage[normalem]{ulem}
\useunder{\uline}{\ul}{}
\usepackage{wrapfig}


\usepackage[table,xcdraw]{xcolor}


%% ----------------------------------------------------------------
\begin{document}
\frontmatter
\title      {Heterogeneous Agent-based Model for Supermarket Competition}
\authors    {\texorpdfstring
             {\href{mailto:sc22g13@ecs.soton.ac.uk}{Stefan J. Collier}}
             {Stefan J. Collier}
            }
\addresses  {\groupname\\\deptname\\\univname}
\date       {\today}
\subject    {}
\keywords   {}
\supervisor {Dr. Maria Polukarov}
\examiner   {Professor Sheng Chen}

\maketitle
\begin{abstract}
This project aim was to model and analyse the effects of competitive pricing behaviors of grocery retailers on the British market. 

This was achieved by creating a multi-agent model, containing retailer and consumer agents. The heterogeneous crowd of retailers employs either a uniform pricing strategy or a ‘local price flexing’ strategy. The actions of these retailers are chosen by predicting the profit of each action, using a perceptron. Following on from the consideration of different economic models, a discrete model was developed so that software agents have a discrete environment to operate within. Within the model, it has been observed how supermarkets with differing behaviors affect a heterogeneous crowd of consumer agents. The model was implemented in Java with Python used to evaluate the results. 

The simulation displays good acceptance with real grocery market behavior, i.e. captures the performance of British retailers thus can be used to determine the impact of changes in their behavior on their competitors and consumers.Furthermore it can be used to provide insight into sustainability of volatile pricing strategies, providing a useful insight in volatility of British supermarket retail industry. 
\end{abstract}
\acknowledgements{
I would like to express my sincere gratitude to Dr Maria Polukarov for her guidance and support which provided me the freedom to take this research in the direction of my interest.\\
\\
I would also like to thank my family and friends for their encouragement and support. To those who quietly listened to my software complaints. To those who worked throughout the nights with me. To those who helped me write what I couldn't say. I cannot thank you enough.
}

\declaration{
I, Stefan Collier, declare that this dissertation and the work presented in it are my own and has been generated by me as the result of my own original research.\\
I confirm that:\\
1. This work was done wholly or mainly while in candidature for a degree at this University;\\
2. Where any part of this dissertation has previously been submitted for any other qualification at this University or any other institution, this has been clearly stated;\\
3. Where I have consulted the published work of others, this is always clearly attributed;\\
4. Where I have quoted from the work of others, the source is always given. With the exception of such quotations, this dissertation is entirely my own work;\\
5. I have acknowledged all main sources of help;\\
6. Where the thesis is based on work done by myself jointly with others, I have made clear exactly what was done by others and what I have contributed myself;\\
7. Either none of this work has been published before submission, or parts of this work have been published by :\\
\\
Stefan Collier\\
April 2016
}
\tableofcontents
\listoffigures
\listoftables

\mainmatter
%% ----------------------------------------------------------------
%\include{Introduction}
%\include{Conclusions}
\include{chapters/1Project/main}
\include{chapters/2Lit/main}
\include{chapters/3Design/HighLevel}
\include{chapters/3Design/InDepth}
\include{chapters/4Impl/main}

\include{chapters/5Experiments/1/main}
\include{chapters/5Experiments/2/main}
\include{chapters/5Experiments/3/main}
\include{chapters/5Experiments/4/main}

\include{chapters/6Conclusion/main}

\appendix
\include{appendix/AppendixB}
\include{appendix/D/main}
\include{appendix/AppendixC}

\backmatter
\bibliographystyle{ecs}
\bibliography{ECS}
\end{document}
%% ----------------------------------------------------------------

 %% ----------------------------------------------------------------
%% Progress.tex
%% ---------------------------------------------------------------- 
\documentclass{ecsprogress}    % Use the progress Style
\graphicspath{{../figs/}}   % Location of your graphics files
    \usepackage{natbib}            % Use Natbib style for the refs.
\hypersetup{colorlinks=true}   % Set to false for black/white printing
\input{Definitions}            % Include your abbreviations



\usepackage{enumitem}% http://ctan.org/pkg/enumitem
\usepackage{multirow}
\usepackage{float}
\usepackage{amsmath}
\usepackage{multicol}
\usepackage{amssymb}
\usepackage[normalem]{ulem}
\useunder{\uline}{\ul}{}
\usepackage{wrapfig}


\usepackage[table,xcdraw]{xcolor}


%% ----------------------------------------------------------------
\begin{document}
\frontmatter
\title      {Heterogeneous Agent-based Model for Supermarket Competition}
\authors    {\texorpdfstring
             {\href{mailto:sc22g13@ecs.soton.ac.uk}{Stefan J. Collier}}
             {Stefan J. Collier}
            }
\addresses  {\groupname\\\deptname\\\univname}
\date       {\today}
\subject    {}
\keywords   {}
\supervisor {Dr. Maria Polukarov}
\examiner   {Professor Sheng Chen}

\maketitle
\begin{abstract}
This project aim was to model and analyse the effects of competitive pricing behaviors of grocery retailers on the British market. 

This was achieved by creating a multi-agent model, containing retailer and consumer agents. The heterogeneous crowd of retailers employs either a uniform pricing strategy or a ‘local price flexing’ strategy. The actions of these retailers are chosen by predicting the profit of each action, using a perceptron. Following on from the consideration of different economic models, a discrete model was developed so that software agents have a discrete environment to operate within. Within the model, it has been observed how supermarkets with differing behaviors affect a heterogeneous crowd of consumer agents. The model was implemented in Java with Python used to evaluate the results. 

The simulation displays good acceptance with real grocery market behavior, i.e. captures the performance of British retailers thus can be used to determine the impact of changes in their behavior on their competitors and consumers.Furthermore it can be used to provide insight into sustainability of volatile pricing strategies, providing a useful insight in volatility of British supermarket retail industry. 
\end{abstract}
\acknowledgements{
I would like to express my sincere gratitude to Dr Maria Polukarov for her guidance and support which provided me the freedom to take this research in the direction of my interest.\\
\\
I would also like to thank my family and friends for their encouragement and support. To those who quietly listened to my software complaints. To those who worked throughout the nights with me. To those who helped me write what I couldn't say. I cannot thank you enough.
}

\declaration{
I, Stefan Collier, declare that this dissertation and the work presented in it are my own and has been generated by me as the result of my own original research.\\
I confirm that:\\
1. This work was done wholly or mainly while in candidature for a degree at this University;\\
2. Where any part of this dissertation has previously been submitted for any other qualification at this University or any other institution, this has been clearly stated;\\
3. Where I have consulted the published work of others, this is always clearly attributed;\\
4. Where I have quoted from the work of others, the source is always given. With the exception of such quotations, this dissertation is entirely my own work;\\
5. I have acknowledged all main sources of help;\\
6. Where the thesis is based on work done by myself jointly with others, I have made clear exactly what was done by others and what I have contributed myself;\\
7. Either none of this work has been published before submission, or parts of this work have been published by :\\
\\
Stefan Collier\\
April 2016
}
\tableofcontents
\listoffigures
\listoftables

\mainmatter
%% ----------------------------------------------------------------
%\include{Introduction}
%\include{Conclusions}
\include{chapters/1Project/main}
\include{chapters/2Lit/main}
\include{chapters/3Design/HighLevel}
\include{chapters/3Design/InDepth}
\include{chapters/4Impl/main}

\include{chapters/5Experiments/1/main}
\include{chapters/5Experiments/2/main}
\include{chapters/5Experiments/3/main}
\include{chapters/5Experiments/4/main}

\include{chapters/6Conclusion/main}

\appendix
\include{appendix/AppendixB}
\include{appendix/D/main}
\include{appendix/AppendixC}

\backmatter
\bibliographystyle{ecs}
\bibliography{ECS}
\end{document}
%% ----------------------------------------------------------------

\include{chapters/3Design/HighLevel}
\include{chapters/3Design/InDepth}
 %% ----------------------------------------------------------------
%% Progress.tex
%% ---------------------------------------------------------------- 
\documentclass{ecsprogress}    % Use the progress Style
\graphicspath{{../figs/}}   % Location of your graphics files
    \usepackage{natbib}            % Use Natbib style for the refs.
\hypersetup{colorlinks=true}   % Set to false for black/white printing
\input{Definitions}            % Include your abbreviations



\usepackage{enumitem}% http://ctan.org/pkg/enumitem
\usepackage{multirow}
\usepackage{float}
\usepackage{amsmath}
\usepackage{multicol}
\usepackage{amssymb}
\usepackage[normalem]{ulem}
\useunder{\uline}{\ul}{}
\usepackage{wrapfig}


\usepackage[table,xcdraw]{xcolor}


%% ----------------------------------------------------------------
\begin{document}
\frontmatter
\title      {Heterogeneous Agent-based Model for Supermarket Competition}
\authors    {\texorpdfstring
             {\href{mailto:sc22g13@ecs.soton.ac.uk}{Stefan J. Collier}}
             {Stefan J. Collier}
            }
\addresses  {\groupname\\\deptname\\\univname}
\date       {\today}
\subject    {}
\keywords   {}
\supervisor {Dr. Maria Polukarov}
\examiner   {Professor Sheng Chen}

\maketitle
\begin{abstract}
This project aim was to model and analyse the effects of competitive pricing behaviors of grocery retailers on the British market. 

This was achieved by creating a multi-agent model, containing retailer and consumer agents. The heterogeneous crowd of retailers employs either a uniform pricing strategy or a ‘local price flexing’ strategy. The actions of these retailers are chosen by predicting the profit of each action, using a perceptron. Following on from the consideration of different economic models, a discrete model was developed so that software agents have a discrete environment to operate within. Within the model, it has been observed how supermarkets with differing behaviors affect a heterogeneous crowd of consumer agents. The model was implemented in Java with Python used to evaluate the results. 

The simulation displays good acceptance with real grocery market behavior, i.e. captures the performance of British retailers thus can be used to determine the impact of changes in their behavior on their competitors and consumers.Furthermore it can be used to provide insight into sustainability of volatile pricing strategies, providing a useful insight in volatility of British supermarket retail industry. 
\end{abstract}
\acknowledgements{
I would like to express my sincere gratitude to Dr Maria Polukarov for her guidance and support which provided me the freedom to take this research in the direction of my interest.\\
\\
I would also like to thank my family and friends for their encouragement and support. To those who quietly listened to my software complaints. To those who worked throughout the nights with me. To those who helped me write what I couldn't say. I cannot thank you enough.
}

\declaration{
I, Stefan Collier, declare that this dissertation and the work presented in it are my own and has been generated by me as the result of my own original research.\\
I confirm that:\\
1. This work was done wholly or mainly while in candidature for a degree at this University;\\
2. Where any part of this dissertation has previously been submitted for any other qualification at this University or any other institution, this has been clearly stated;\\
3. Where I have consulted the published work of others, this is always clearly attributed;\\
4. Where I have quoted from the work of others, the source is always given. With the exception of such quotations, this dissertation is entirely my own work;\\
5. I have acknowledged all main sources of help;\\
6. Where the thesis is based on work done by myself jointly with others, I have made clear exactly what was done by others and what I have contributed myself;\\
7. Either none of this work has been published before submission, or parts of this work have been published by :\\
\\
Stefan Collier\\
April 2016
}
\tableofcontents
\listoffigures
\listoftables

\mainmatter
%% ----------------------------------------------------------------
%\include{Introduction}
%\include{Conclusions}
\include{chapters/1Project/main}
\include{chapters/2Lit/main}
\include{chapters/3Design/HighLevel}
\include{chapters/3Design/InDepth}
\include{chapters/4Impl/main}

\include{chapters/5Experiments/1/main}
\include{chapters/5Experiments/2/main}
\include{chapters/5Experiments/3/main}
\include{chapters/5Experiments/4/main}

\include{chapters/6Conclusion/main}

\appendix
\include{appendix/AppendixB}
\include{appendix/D/main}
\include{appendix/AppendixC}

\backmatter
\bibliographystyle{ecs}
\bibliography{ECS}
\end{document}
%% ----------------------------------------------------------------


 %% ----------------------------------------------------------------
%% Progress.tex
%% ---------------------------------------------------------------- 
\documentclass{ecsprogress}    % Use the progress Style
\graphicspath{{../figs/}}   % Location of your graphics files
    \usepackage{natbib}            % Use Natbib style for the refs.
\hypersetup{colorlinks=true}   % Set to false for black/white printing
\input{Definitions}            % Include your abbreviations



\usepackage{enumitem}% http://ctan.org/pkg/enumitem
\usepackage{multirow}
\usepackage{float}
\usepackage{amsmath}
\usepackage{multicol}
\usepackage{amssymb}
\usepackage[normalem]{ulem}
\useunder{\uline}{\ul}{}
\usepackage{wrapfig}


\usepackage[table,xcdraw]{xcolor}


%% ----------------------------------------------------------------
\begin{document}
\frontmatter
\title      {Heterogeneous Agent-based Model for Supermarket Competition}
\authors    {\texorpdfstring
             {\href{mailto:sc22g13@ecs.soton.ac.uk}{Stefan J. Collier}}
             {Stefan J. Collier}
            }
\addresses  {\groupname\\\deptname\\\univname}
\date       {\today}
\subject    {}
\keywords   {}
\supervisor {Dr. Maria Polukarov}
\examiner   {Professor Sheng Chen}

\maketitle
\begin{abstract}
This project aim was to model and analyse the effects of competitive pricing behaviors of grocery retailers on the British market. 

This was achieved by creating a multi-agent model, containing retailer and consumer agents. The heterogeneous crowd of retailers employs either a uniform pricing strategy or a ‘local price flexing’ strategy. The actions of these retailers are chosen by predicting the profit of each action, using a perceptron. Following on from the consideration of different economic models, a discrete model was developed so that software agents have a discrete environment to operate within. Within the model, it has been observed how supermarkets with differing behaviors affect a heterogeneous crowd of consumer agents. The model was implemented in Java with Python used to evaluate the results. 

The simulation displays good acceptance with real grocery market behavior, i.e. captures the performance of British retailers thus can be used to determine the impact of changes in their behavior on their competitors and consumers.Furthermore it can be used to provide insight into sustainability of volatile pricing strategies, providing a useful insight in volatility of British supermarket retail industry. 
\end{abstract}
\acknowledgements{
I would like to express my sincere gratitude to Dr Maria Polukarov for her guidance and support which provided me the freedom to take this research in the direction of my interest.\\
\\
I would also like to thank my family and friends for their encouragement and support. To those who quietly listened to my software complaints. To those who worked throughout the nights with me. To those who helped me write what I couldn't say. I cannot thank you enough.
}

\declaration{
I, Stefan Collier, declare that this dissertation and the work presented in it are my own and has been generated by me as the result of my own original research.\\
I confirm that:\\
1. This work was done wholly or mainly while in candidature for a degree at this University;\\
2. Where any part of this dissertation has previously been submitted for any other qualification at this University or any other institution, this has been clearly stated;\\
3. Where I have consulted the published work of others, this is always clearly attributed;\\
4. Where I have quoted from the work of others, the source is always given. With the exception of such quotations, this dissertation is entirely my own work;\\
5. I have acknowledged all main sources of help;\\
6. Where the thesis is based on work done by myself jointly with others, I have made clear exactly what was done by others and what I have contributed myself;\\
7. Either none of this work has been published before submission, or parts of this work have been published by :\\
\\
Stefan Collier\\
April 2016
}
\tableofcontents
\listoffigures
\listoftables

\mainmatter
%% ----------------------------------------------------------------
%\include{Introduction}
%\include{Conclusions}
\include{chapters/1Project/main}
\include{chapters/2Lit/main}
\include{chapters/3Design/HighLevel}
\include{chapters/3Design/InDepth}
\include{chapters/4Impl/main}

\include{chapters/5Experiments/1/main}
\include{chapters/5Experiments/2/main}
\include{chapters/5Experiments/3/main}
\include{chapters/5Experiments/4/main}

\include{chapters/6Conclusion/main}

\appendix
\include{appendix/AppendixB}
\include{appendix/D/main}
\include{appendix/AppendixC}

\backmatter
\bibliographystyle{ecs}
\bibliography{ECS}
\end{document}
%% ----------------------------------------------------------------

 %% ----------------------------------------------------------------
%% Progress.tex
%% ---------------------------------------------------------------- 
\documentclass{ecsprogress}    % Use the progress Style
\graphicspath{{../figs/}}   % Location of your graphics files
    \usepackage{natbib}            % Use Natbib style for the refs.
\hypersetup{colorlinks=true}   % Set to false for black/white printing
\input{Definitions}            % Include your abbreviations



\usepackage{enumitem}% http://ctan.org/pkg/enumitem
\usepackage{multirow}
\usepackage{float}
\usepackage{amsmath}
\usepackage{multicol}
\usepackage{amssymb}
\usepackage[normalem]{ulem}
\useunder{\uline}{\ul}{}
\usepackage{wrapfig}


\usepackage[table,xcdraw]{xcolor}


%% ----------------------------------------------------------------
\begin{document}
\frontmatter
\title      {Heterogeneous Agent-based Model for Supermarket Competition}
\authors    {\texorpdfstring
             {\href{mailto:sc22g13@ecs.soton.ac.uk}{Stefan J. Collier}}
             {Stefan J. Collier}
            }
\addresses  {\groupname\\\deptname\\\univname}
\date       {\today}
\subject    {}
\keywords   {}
\supervisor {Dr. Maria Polukarov}
\examiner   {Professor Sheng Chen}

\maketitle
\begin{abstract}
This project aim was to model and analyse the effects of competitive pricing behaviors of grocery retailers on the British market. 

This was achieved by creating a multi-agent model, containing retailer and consumer agents. The heterogeneous crowd of retailers employs either a uniform pricing strategy or a ‘local price flexing’ strategy. The actions of these retailers are chosen by predicting the profit of each action, using a perceptron. Following on from the consideration of different economic models, a discrete model was developed so that software agents have a discrete environment to operate within. Within the model, it has been observed how supermarkets with differing behaviors affect a heterogeneous crowd of consumer agents. The model was implemented in Java with Python used to evaluate the results. 

The simulation displays good acceptance with real grocery market behavior, i.e. captures the performance of British retailers thus can be used to determine the impact of changes in their behavior on their competitors and consumers.Furthermore it can be used to provide insight into sustainability of volatile pricing strategies, providing a useful insight in volatility of British supermarket retail industry. 
\end{abstract}
\acknowledgements{
I would like to express my sincere gratitude to Dr Maria Polukarov for her guidance and support which provided me the freedom to take this research in the direction of my interest.\\
\\
I would also like to thank my family and friends for their encouragement and support. To those who quietly listened to my software complaints. To those who worked throughout the nights with me. To those who helped me write what I couldn't say. I cannot thank you enough.
}

\declaration{
I, Stefan Collier, declare that this dissertation and the work presented in it are my own and has been generated by me as the result of my own original research.\\
I confirm that:\\
1. This work was done wholly or mainly while in candidature for a degree at this University;\\
2. Where any part of this dissertation has previously been submitted for any other qualification at this University or any other institution, this has been clearly stated;\\
3. Where I have consulted the published work of others, this is always clearly attributed;\\
4. Where I have quoted from the work of others, the source is always given. With the exception of such quotations, this dissertation is entirely my own work;\\
5. I have acknowledged all main sources of help;\\
6. Where the thesis is based on work done by myself jointly with others, I have made clear exactly what was done by others and what I have contributed myself;\\
7. Either none of this work has been published before submission, or parts of this work have been published by :\\
\\
Stefan Collier\\
April 2016
}
\tableofcontents
\listoffigures
\listoftables

\mainmatter
%% ----------------------------------------------------------------
%\include{Introduction}
%\include{Conclusions}
\include{chapters/1Project/main}
\include{chapters/2Lit/main}
\include{chapters/3Design/HighLevel}
\include{chapters/3Design/InDepth}
\include{chapters/4Impl/main}

\include{chapters/5Experiments/1/main}
\include{chapters/5Experiments/2/main}
\include{chapters/5Experiments/3/main}
\include{chapters/5Experiments/4/main}

\include{chapters/6Conclusion/main}

\appendix
\include{appendix/AppendixB}
\include{appendix/D/main}
\include{appendix/AppendixC}

\backmatter
\bibliographystyle{ecs}
\bibliography{ECS}
\end{document}
%% ----------------------------------------------------------------

 %% ----------------------------------------------------------------
%% Progress.tex
%% ---------------------------------------------------------------- 
\documentclass{ecsprogress}    % Use the progress Style
\graphicspath{{../figs/}}   % Location of your graphics files
    \usepackage{natbib}            % Use Natbib style for the refs.
\hypersetup{colorlinks=true}   % Set to false for black/white printing
\input{Definitions}            % Include your abbreviations



\usepackage{enumitem}% http://ctan.org/pkg/enumitem
\usepackage{multirow}
\usepackage{float}
\usepackage{amsmath}
\usepackage{multicol}
\usepackage{amssymb}
\usepackage[normalem]{ulem}
\useunder{\uline}{\ul}{}
\usepackage{wrapfig}


\usepackage[table,xcdraw]{xcolor}


%% ----------------------------------------------------------------
\begin{document}
\frontmatter
\title      {Heterogeneous Agent-based Model for Supermarket Competition}
\authors    {\texorpdfstring
             {\href{mailto:sc22g13@ecs.soton.ac.uk}{Stefan J. Collier}}
             {Stefan J. Collier}
            }
\addresses  {\groupname\\\deptname\\\univname}
\date       {\today}
\subject    {}
\keywords   {}
\supervisor {Dr. Maria Polukarov}
\examiner   {Professor Sheng Chen}

\maketitle
\begin{abstract}
This project aim was to model and analyse the effects of competitive pricing behaviors of grocery retailers on the British market. 

This was achieved by creating a multi-agent model, containing retailer and consumer agents. The heterogeneous crowd of retailers employs either a uniform pricing strategy or a ‘local price flexing’ strategy. The actions of these retailers are chosen by predicting the profit of each action, using a perceptron. Following on from the consideration of different economic models, a discrete model was developed so that software agents have a discrete environment to operate within. Within the model, it has been observed how supermarkets with differing behaviors affect a heterogeneous crowd of consumer agents. The model was implemented in Java with Python used to evaluate the results. 

The simulation displays good acceptance with real grocery market behavior, i.e. captures the performance of British retailers thus can be used to determine the impact of changes in their behavior on their competitors and consumers.Furthermore it can be used to provide insight into sustainability of volatile pricing strategies, providing a useful insight in volatility of British supermarket retail industry. 
\end{abstract}
\acknowledgements{
I would like to express my sincere gratitude to Dr Maria Polukarov for her guidance and support which provided me the freedom to take this research in the direction of my interest.\\
\\
I would also like to thank my family and friends for their encouragement and support. To those who quietly listened to my software complaints. To those who worked throughout the nights with me. To those who helped me write what I couldn't say. I cannot thank you enough.
}

\declaration{
I, Stefan Collier, declare that this dissertation and the work presented in it are my own and has been generated by me as the result of my own original research.\\
I confirm that:\\
1. This work was done wholly or mainly while in candidature for a degree at this University;\\
2. Where any part of this dissertation has previously been submitted for any other qualification at this University or any other institution, this has been clearly stated;\\
3. Where I have consulted the published work of others, this is always clearly attributed;\\
4. Where I have quoted from the work of others, the source is always given. With the exception of such quotations, this dissertation is entirely my own work;\\
5. I have acknowledged all main sources of help;\\
6. Where the thesis is based on work done by myself jointly with others, I have made clear exactly what was done by others and what I have contributed myself;\\
7. Either none of this work has been published before submission, or parts of this work have been published by :\\
\\
Stefan Collier\\
April 2016
}
\tableofcontents
\listoffigures
\listoftables

\mainmatter
%% ----------------------------------------------------------------
%\include{Introduction}
%\include{Conclusions}
\include{chapters/1Project/main}
\include{chapters/2Lit/main}
\include{chapters/3Design/HighLevel}
\include{chapters/3Design/InDepth}
\include{chapters/4Impl/main}

\include{chapters/5Experiments/1/main}
\include{chapters/5Experiments/2/main}
\include{chapters/5Experiments/3/main}
\include{chapters/5Experiments/4/main}

\include{chapters/6Conclusion/main}

\appendix
\include{appendix/AppendixB}
\include{appendix/D/main}
\include{appendix/AppendixC}

\backmatter
\bibliographystyle{ecs}
\bibliography{ECS}
\end{document}
%% ----------------------------------------------------------------

 %% ----------------------------------------------------------------
%% Progress.tex
%% ---------------------------------------------------------------- 
\documentclass{ecsprogress}    % Use the progress Style
\graphicspath{{../figs/}}   % Location of your graphics files
    \usepackage{natbib}            % Use Natbib style for the refs.
\hypersetup{colorlinks=true}   % Set to false for black/white printing
\input{Definitions}            % Include your abbreviations



\usepackage{enumitem}% http://ctan.org/pkg/enumitem
\usepackage{multirow}
\usepackage{float}
\usepackage{amsmath}
\usepackage{multicol}
\usepackage{amssymb}
\usepackage[normalem]{ulem}
\useunder{\uline}{\ul}{}
\usepackage{wrapfig}


\usepackage[table,xcdraw]{xcolor}


%% ----------------------------------------------------------------
\begin{document}
\frontmatter
\title      {Heterogeneous Agent-based Model for Supermarket Competition}
\authors    {\texorpdfstring
             {\href{mailto:sc22g13@ecs.soton.ac.uk}{Stefan J. Collier}}
             {Stefan J. Collier}
            }
\addresses  {\groupname\\\deptname\\\univname}
\date       {\today}
\subject    {}
\keywords   {}
\supervisor {Dr. Maria Polukarov}
\examiner   {Professor Sheng Chen}

\maketitle
\begin{abstract}
This project aim was to model and analyse the effects of competitive pricing behaviors of grocery retailers on the British market. 

This was achieved by creating a multi-agent model, containing retailer and consumer agents. The heterogeneous crowd of retailers employs either a uniform pricing strategy or a ‘local price flexing’ strategy. The actions of these retailers are chosen by predicting the profit of each action, using a perceptron. Following on from the consideration of different economic models, a discrete model was developed so that software agents have a discrete environment to operate within. Within the model, it has been observed how supermarkets with differing behaviors affect a heterogeneous crowd of consumer agents. The model was implemented in Java with Python used to evaluate the results. 

The simulation displays good acceptance with real grocery market behavior, i.e. captures the performance of British retailers thus can be used to determine the impact of changes in their behavior on their competitors and consumers.Furthermore it can be used to provide insight into sustainability of volatile pricing strategies, providing a useful insight in volatility of British supermarket retail industry. 
\end{abstract}
\acknowledgements{
I would like to express my sincere gratitude to Dr Maria Polukarov for her guidance and support which provided me the freedom to take this research in the direction of my interest.\\
\\
I would also like to thank my family and friends for their encouragement and support. To those who quietly listened to my software complaints. To those who worked throughout the nights with me. To those who helped me write what I couldn't say. I cannot thank you enough.
}

\declaration{
I, Stefan Collier, declare that this dissertation and the work presented in it are my own and has been generated by me as the result of my own original research.\\
I confirm that:\\
1. This work was done wholly or mainly while in candidature for a degree at this University;\\
2. Where any part of this dissertation has previously been submitted for any other qualification at this University or any other institution, this has been clearly stated;\\
3. Where I have consulted the published work of others, this is always clearly attributed;\\
4. Where I have quoted from the work of others, the source is always given. With the exception of such quotations, this dissertation is entirely my own work;\\
5. I have acknowledged all main sources of help;\\
6. Where the thesis is based on work done by myself jointly with others, I have made clear exactly what was done by others and what I have contributed myself;\\
7. Either none of this work has been published before submission, or parts of this work have been published by :\\
\\
Stefan Collier\\
April 2016
}
\tableofcontents
\listoffigures
\listoftables

\mainmatter
%% ----------------------------------------------------------------
%\include{Introduction}
%\include{Conclusions}
\include{chapters/1Project/main}
\include{chapters/2Lit/main}
\include{chapters/3Design/HighLevel}
\include{chapters/3Design/InDepth}
\include{chapters/4Impl/main}

\include{chapters/5Experiments/1/main}
\include{chapters/5Experiments/2/main}
\include{chapters/5Experiments/3/main}
\include{chapters/5Experiments/4/main}

\include{chapters/6Conclusion/main}

\appendix
\include{appendix/AppendixB}
\include{appendix/D/main}
\include{appendix/AppendixC}

\backmatter
\bibliographystyle{ecs}
\bibliography{ECS}
\end{document}
%% ----------------------------------------------------------------


 %% ----------------------------------------------------------------
%% Progress.tex
%% ---------------------------------------------------------------- 
\documentclass{ecsprogress}    % Use the progress Style
\graphicspath{{../figs/}}   % Location of your graphics files
    \usepackage{natbib}            % Use Natbib style for the refs.
\hypersetup{colorlinks=true}   % Set to false for black/white printing
\input{Definitions}            % Include your abbreviations



\usepackage{enumitem}% http://ctan.org/pkg/enumitem
\usepackage{multirow}
\usepackage{float}
\usepackage{amsmath}
\usepackage{multicol}
\usepackage{amssymb}
\usepackage[normalem]{ulem}
\useunder{\uline}{\ul}{}
\usepackage{wrapfig}


\usepackage[table,xcdraw]{xcolor}


%% ----------------------------------------------------------------
\begin{document}
\frontmatter
\title      {Heterogeneous Agent-based Model for Supermarket Competition}
\authors    {\texorpdfstring
             {\href{mailto:sc22g13@ecs.soton.ac.uk}{Stefan J. Collier}}
             {Stefan J. Collier}
            }
\addresses  {\groupname\\\deptname\\\univname}
\date       {\today}
\subject    {}
\keywords   {}
\supervisor {Dr. Maria Polukarov}
\examiner   {Professor Sheng Chen}

\maketitle
\begin{abstract}
This project aim was to model and analyse the effects of competitive pricing behaviors of grocery retailers on the British market. 

This was achieved by creating a multi-agent model, containing retailer and consumer agents. The heterogeneous crowd of retailers employs either a uniform pricing strategy or a ‘local price flexing’ strategy. The actions of these retailers are chosen by predicting the profit of each action, using a perceptron. Following on from the consideration of different economic models, a discrete model was developed so that software agents have a discrete environment to operate within. Within the model, it has been observed how supermarkets with differing behaviors affect a heterogeneous crowd of consumer agents. The model was implemented in Java with Python used to evaluate the results. 

The simulation displays good acceptance with real grocery market behavior, i.e. captures the performance of British retailers thus can be used to determine the impact of changes in their behavior on their competitors and consumers.Furthermore it can be used to provide insight into sustainability of volatile pricing strategies, providing a useful insight in volatility of British supermarket retail industry. 
\end{abstract}
\acknowledgements{
I would like to express my sincere gratitude to Dr Maria Polukarov for her guidance and support which provided me the freedom to take this research in the direction of my interest.\\
\\
I would also like to thank my family and friends for their encouragement and support. To those who quietly listened to my software complaints. To those who worked throughout the nights with me. To those who helped me write what I couldn't say. I cannot thank you enough.
}

\declaration{
I, Stefan Collier, declare that this dissertation and the work presented in it are my own and has been generated by me as the result of my own original research.\\
I confirm that:\\
1. This work was done wholly or mainly while in candidature for a degree at this University;\\
2. Where any part of this dissertation has previously been submitted for any other qualification at this University or any other institution, this has been clearly stated;\\
3. Where I have consulted the published work of others, this is always clearly attributed;\\
4. Where I have quoted from the work of others, the source is always given. With the exception of such quotations, this dissertation is entirely my own work;\\
5. I have acknowledged all main sources of help;\\
6. Where the thesis is based on work done by myself jointly with others, I have made clear exactly what was done by others and what I have contributed myself;\\
7. Either none of this work has been published before submission, or parts of this work have been published by :\\
\\
Stefan Collier\\
April 2016
}
\tableofcontents
\listoffigures
\listoftables

\mainmatter
%% ----------------------------------------------------------------
%\include{Introduction}
%\include{Conclusions}
\include{chapters/1Project/main}
\include{chapters/2Lit/main}
\include{chapters/3Design/HighLevel}
\include{chapters/3Design/InDepth}
\include{chapters/4Impl/main}

\include{chapters/5Experiments/1/main}
\include{chapters/5Experiments/2/main}
\include{chapters/5Experiments/3/main}
\include{chapters/5Experiments/4/main}

\include{chapters/6Conclusion/main}

\appendix
\include{appendix/AppendixB}
\include{appendix/D/main}
\include{appendix/AppendixC}

\backmatter
\bibliographystyle{ecs}
\bibliography{ECS}
\end{document}
%% ----------------------------------------------------------------


\appendix
\include{appendix/AppendixB}
 %% ----------------------------------------------------------------
%% Progress.tex
%% ---------------------------------------------------------------- 
\documentclass{ecsprogress}    % Use the progress Style
\graphicspath{{../figs/}}   % Location of your graphics files
    \usepackage{natbib}            % Use Natbib style for the refs.
\hypersetup{colorlinks=true}   % Set to false for black/white printing
\input{Definitions}            % Include your abbreviations



\usepackage{enumitem}% http://ctan.org/pkg/enumitem
\usepackage{multirow}
\usepackage{float}
\usepackage{amsmath}
\usepackage{multicol}
\usepackage{amssymb}
\usepackage[normalem]{ulem}
\useunder{\uline}{\ul}{}
\usepackage{wrapfig}


\usepackage[table,xcdraw]{xcolor}


%% ----------------------------------------------------------------
\begin{document}
\frontmatter
\title      {Heterogeneous Agent-based Model for Supermarket Competition}
\authors    {\texorpdfstring
             {\href{mailto:sc22g13@ecs.soton.ac.uk}{Stefan J. Collier}}
             {Stefan J. Collier}
            }
\addresses  {\groupname\\\deptname\\\univname}
\date       {\today}
\subject    {}
\keywords   {}
\supervisor {Dr. Maria Polukarov}
\examiner   {Professor Sheng Chen}

\maketitle
\begin{abstract}
This project aim was to model and analyse the effects of competitive pricing behaviors of grocery retailers on the British market. 

This was achieved by creating a multi-agent model, containing retailer and consumer agents. The heterogeneous crowd of retailers employs either a uniform pricing strategy or a ‘local price flexing’ strategy. The actions of these retailers are chosen by predicting the profit of each action, using a perceptron. Following on from the consideration of different economic models, a discrete model was developed so that software agents have a discrete environment to operate within. Within the model, it has been observed how supermarkets with differing behaviors affect a heterogeneous crowd of consumer agents. The model was implemented in Java with Python used to evaluate the results. 

The simulation displays good acceptance with real grocery market behavior, i.e. captures the performance of British retailers thus can be used to determine the impact of changes in their behavior on their competitors and consumers.Furthermore it can be used to provide insight into sustainability of volatile pricing strategies, providing a useful insight in volatility of British supermarket retail industry. 
\end{abstract}
\acknowledgements{
I would like to express my sincere gratitude to Dr Maria Polukarov for her guidance and support which provided me the freedom to take this research in the direction of my interest.\\
\\
I would also like to thank my family and friends for their encouragement and support. To those who quietly listened to my software complaints. To those who worked throughout the nights with me. To those who helped me write what I couldn't say. I cannot thank you enough.
}

\declaration{
I, Stefan Collier, declare that this dissertation and the work presented in it are my own and has been generated by me as the result of my own original research.\\
I confirm that:\\
1. This work was done wholly or mainly while in candidature for a degree at this University;\\
2. Where any part of this dissertation has previously been submitted for any other qualification at this University or any other institution, this has been clearly stated;\\
3. Where I have consulted the published work of others, this is always clearly attributed;\\
4. Where I have quoted from the work of others, the source is always given. With the exception of such quotations, this dissertation is entirely my own work;\\
5. I have acknowledged all main sources of help;\\
6. Where the thesis is based on work done by myself jointly with others, I have made clear exactly what was done by others and what I have contributed myself;\\
7. Either none of this work has been published before submission, or parts of this work have been published by :\\
\\
Stefan Collier\\
April 2016
}
\tableofcontents
\listoffigures
\listoftables

\mainmatter
%% ----------------------------------------------------------------
%\include{Introduction}
%\include{Conclusions}
\include{chapters/1Project/main}
\include{chapters/2Lit/main}
\include{chapters/3Design/HighLevel}
\include{chapters/3Design/InDepth}
\include{chapters/4Impl/main}

\include{chapters/5Experiments/1/main}
\include{chapters/5Experiments/2/main}
\include{chapters/5Experiments/3/main}
\include{chapters/5Experiments/4/main}

\include{chapters/6Conclusion/main}

\appendix
\include{appendix/AppendixB}
\include{appendix/D/main}
\include{appendix/AppendixC}

\backmatter
\bibliographystyle{ecs}
\bibliography{ECS}
\end{document}
%% ----------------------------------------------------------------

\include{appendix/AppendixC}

\backmatter
\bibliographystyle{ecs}
\bibliography{ECS}
\end{document}
%% ----------------------------------------------------------------


 %% ----------------------------------------------------------------
%% Progress.tex
%% ---------------------------------------------------------------- 
\documentclass{ecsprogress}    % Use the progress Style
\graphicspath{{../figs/}}   % Location of your graphics files
    \usepackage{natbib}            % Use Natbib style for the refs.
\hypersetup{colorlinks=true}   % Set to false for black/white printing
\input{Definitions}            % Include your abbreviations



\usepackage{enumitem}% http://ctan.org/pkg/enumitem
\usepackage{multirow}
\usepackage{float}
\usepackage{amsmath}
\usepackage{multicol}
\usepackage{amssymb}
\usepackage[normalem]{ulem}
\useunder{\uline}{\ul}{}
\usepackage{wrapfig}


\usepackage[table,xcdraw]{xcolor}


%% ----------------------------------------------------------------
\begin{document}
\frontmatter
\title      {Heterogeneous Agent-based Model for Supermarket Competition}
\authors    {\texorpdfstring
             {\href{mailto:sc22g13@ecs.soton.ac.uk}{Stefan J. Collier}}
             {Stefan J. Collier}
            }
\addresses  {\groupname\\\deptname\\\univname}
\date       {\today}
\subject    {}
\keywords   {}
\supervisor {Dr. Maria Polukarov}
\examiner   {Professor Sheng Chen}

\maketitle
\begin{abstract}
This project aim was to model and analyse the effects of competitive pricing behaviors of grocery retailers on the British market. 

This was achieved by creating a multi-agent model, containing retailer and consumer agents. The heterogeneous crowd of retailers employs either a uniform pricing strategy or a ‘local price flexing’ strategy. The actions of these retailers are chosen by predicting the profit of each action, using a perceptron. Following on from the consideration of different economic models, a discrete model was developed so that software agents have a discrete environment to operate within. Within the model, it has been observed how supermarkets with differing behaviors affect a heterogeneous crowd of consumer agents. The model was implemented in Java with Python used to evaluate the results. 

The simulation displays good acceptance with real grocery market behavior, i.e. captures the performance of British retailers thus can be used to determine the impact of changes in their behavior on their competitors and consumers.Furthermore it can be used to provide insight into sustainability of volatile pricing strategies, providing a useful insight in volatility of British supermarket retail industry. 
\end{abstract}
\acknowledgements{
I would like to express my sincere gratitude to Dr Maria Polukarov for her guidance and support which provided me the freedom to take this research in the direction of my interest.\\
\\
I would also like to thank my family and friends for their encouragement and support. To those who quietly listened to my software complaints. To those who worked throughout the nights with me. To those who helped me write what I couldn't say. I cannot thank you enough.
}

\declaration{
I, Stefan Collier, declare that this dissertation and the work presented in it are my own and has been generated by me as the result of my own original research.\\
I confirm that:\\
1. This work was done wholly or mainly while in candidature for a degree at this University;\\
2. Where any part of this dissertation has previously been submitted for any other qualification at this University or any other institution, this has been clearly stated;\\
3. Where I have consulted the published work of others, this is always clearly attributed;\\
4. Where I have quoted from the work of others, the source is always given. With the exception of such quotations, this dissertation is entirely my own work;\\
5. I have acknowledged all main sources of help;\\
6. Where the thesis is based on work done by myself jointly with others, I have made clear exactly what was done by others and what I have contributed myself;\\
7. Either none of this work has been published before submission, or parts of this work have been published by :\\
\\
Stefan Collier\\
April 2016
}
\tableofcontents
\listoffigures
\listoftables

\mainmatter
%% ----------------------------------------------------------------
%\include{Introduction}
%\include{Conclusions}
 %% ----------------------------------------------------------------
%% Progress.tex
%% ---------------------------------------------------------------- 
\documentclass{ecsprogress}    % Use the progress Style
\graphicspath{{../figs/}}   % Location of your graphics files
    \usepackage{natbib}            % Use Natbib style for the refs.
\hypersetup{colorlinks=true}   % Set to false for black/white printing
\input{Definitions}            % Include your abbreviations



\usepackage{enumitem}% http://ctan.org/pkg/enumitem
\usepackage{multirow}
\usepackage{float}
\usepackage{amsmath}
\usepackage{multicol}
\usepackage{amssymb}
\usepackage[normalem]{ulem}
\useunder{\uline}{\ul}{}
\usepackage{wrapfig}


\usepackage[table,xcdraw]{xcolor}


%% ----------------------------------------------------------------
\begin{document}
\frontmatter
\title      {Heterogeneous Agent-based Model for Supermarket Competition}
\authors    {\texorpdfstring
             {\href{mailto:sc22g13@ecs.soton.ac.uk}{Stefan J. Collier}}
             {Stefan J. Collier}
            }
\addresses  {\groupname\\\deptname\\\univname}
\date       {\today}
\subject    {}
\keywords   {}
\supervisor {Dr. Maria Polukarov}
\examiner   {Professor Sheng Chen}

\maketitle
\begin{abstract}
This project aim was to model and analyse the effects of competitive pricing behaviors of grocery retailers on the British market. 

This was achieved by creating a multi-agent model, containing retailer and consumer agents. The heterogeneous crowd of retailers employs either a uniform pricing strategy or a ‘local price flexing’ strategy. The actions of these retailers are chosen by predicting the profit of each action, using a perceptron. Following on from the consideration of different economic models, a discrete model was developed so that software agents have a discrete environment to operate within. Within the model, it has been observed how supermarkets with differing behaviors affect a heterogeneous crowd of consumer agents. The model was implemented in Java with Python used to evaluate the results. 

The simulation displays good acceptance with real grocery market behavior, i.e. captures the performance of British retailers thus can be used to determine the impact of changes in their behavior on their competitors and consumers.Furthermore it can be used to provide insight into sustainability of volatile pricing strategies, providing a useful insight in volatility of British supermarket retail industry. 
\end{abstract}
\acknowledgements{
I would like to express my sincere gratitude to Dr Maria Polukarov for her guidance and support which provided me the freedom to take this research in the direction of my interest.\\
\\
I would also like to thank my family and friends for their encouragement and support. To those who quietly listened to my software complaints. To those who worked throughout the nights with me. To those who helped me write what I couldn't say. I cannot thank you enough.
}

\declaration{
I, Stefan Collier, declare that this dissertation and the work presented in it are my own and has been generated by me as the result of my own original research.\\
I confirm that:\\
1. This work was done wholly or mainly while in candidature for a degree at this University;\\
2. Where any part of this dissertation has previously been submitted for any other qualification at this University or any other institution, this has been clearly stated;\\
3. Where I have consulted the published work of others, this is always clearly attributed;\\
4. Where I have quoted from the work of others, the source is always given. With the exception of such quotations, this dissertation is entirely my own work;\\
5. I have acknowledged all main sources of help;\\
6. Where the thesis is based on work done by myself jointly with others, I have made clear exactly what was done by others and what I have contributed myself;\\
7. Either none of this work has been published before submission, or parts of this work have been published by :\\
\\
Stefan Collier\\
April 2016
}
\tableofcontents
\listoffigures
\listoftables

\mainmatter
%% ----------------------------------------------------------------
%\include{Introduction}
%\include{Conclusions}
\include{chapters/1Project/main}
\include{chapters/2Lit/main}
\include{chapters/3Design/HighLevel}
\include{chapters/3Design/InDepth}
\include{chapters/4Impl/main}

\include{chapters/5Experiments/1/main}
\include{chapters/5Experiments/2/main}
\include{chapters/5Experiments/3/main}
\include{chapters/5Experiments/4/main}

\include{chapters/6Conclusion/main}

\appendix
\include{appendix/AppendixB}
\include{appendix/D/main}
\include{appendix/AppendixC}

\backmatter
\bibliographystyle{ecs}
\bibliography{ECS}
\end{document}
%% ----------------------------------------------------------------

 %% ----------------------------------------------------------------
%% Progress.tex
%% ---------------------------------------------------------------- 
\documentclass{ecsprogress}    % Use the progress Style
\graphicspath{{../figs/}}   % Location of your graphics files
    \usepackage{natbib}            % Use Natbib style for the refs.
\hypersetup{colorlinks=true}   % Set to false for black/white printing
\input{Definitions}            % Include your abbreviations



\usepackage{enumitem}% http://ctan.org/pkg/enumitem
\usepackage{multirow}
\usepackage{float}
\usepackage{amsmath}
\usepackage{multicol}
\usepackage{amssymb}
\usepackage[normalem]{ulem}
\useunder{\uline}{\ul}{}
\usepackage{wrapfig}


\usepackage[table,xcdraw]{xcolor}


%% ----------------------------------------------------------------
\begin{document}
\frontmatter
\title      {Heterogeneous Agent-based Model for Supermarket Competition}
\authors    {\texorpdfstring
             {\href{mailto:sc22g13@ecs.soton.ac.uk}{Stefan J. Collier}}
             {Stefan J. Collier}
            }
\addresses  {\groupname\\\deptname\\\univname}
\date       {\today}
\subject    {}
\keywords   {}
\supervisor {Dr. Maria Polukarov}
\examiner   {Professor Sheng Chen}

\maketitle
\begin{abstract}
This project aim was to model and analyse the effects of competitive pricing behaviors of grocery retailers on the British market. 

This was achieved by creating a multi-agent model, containing retailer and consumer agents. The heterogeneous crowd of retailers employs either a uniform pricing strategy or a ‘local price flexing’ strategy. The actions of these retailers are chosen by predicting the profit of each action, using a perceptron. Following on from the consideration of different economic models, a discrete model was developed so that software agents have a discrete environment to operate within. Within the model, it has been observed how supermarkets with differing behaviors affect a heterogeneous crowd of consumer agents. The model was implemented in Java with Python used to evaluate the results. 

The simulation displays good acceptance with real grocery market behavior, i.e. captures the performance of British retailers thus can be used to determine the impact of changes in their behavior on their competitors and consumers.Furthermore it can be used to provide insight into sustainability of volatile pricing strategies, providing a useful insight in volatility of British supermarket retail industry. 
\end{abstract}
\acknowledgements{
I would like to express my sincere gratitude to Dr Maria Polukarov for her guidance and support which provided me the freedom to take this research in the direction of my interest.\\
\\
I would also like to thank my family and friends for their encouragement and support. To those who quietly listened to my software complaints. To those who worked throughout the nights with me. To those who helped me write what I couldn't say. I cannot thank you enough.
}

\declaration{
I, Stefan Collier, declare that this dissertation and the work presented in it are my own and has been generated by me as the result of my own original research.\\
I confirm that:\\
1. This work was done wholly or mainly while in candidature for a degree at this University;\\
2. Where any part of this dissertation has previously been submitted for any other qualification at this University or any other institution, this has been clearly stated;\\
3. Where I have consulted the published work of others, this is always clearly attributed;\\
4. Where I have quoted from the work of others, the source is always given. With the exception of such quotations, this dissertation is entirely my own work;\\
5. I have acknowledged all main sources of help;\\
6. Where the thesis is based on work done by myself jointly with others, I have made clear exactly what was done by others and what I have contributed myself;\\
7. Either none of this work has been published before submission, or parts of this work have been published by :\\
\\
Stefan Collier\\
April 2016
}
\tableofcontents
\listoffigures
\listoftables

\mainmatter
%% ----------------------------------------------------------------
%\include{Introduction}
%\include{Conclusions}
\include{chapters/1Project/main}
\include{chapters/2Lit/main}
\include{chapters/3Design/HighLevel}
\include{chapters/3Design/InDepth}
\include{chapters/4Impl/main}

\include{chapters/5Experiments/1/main}
\include{chapters/5Experiments/2/main}
\include{chapters/5Experiments/3/main}
\include{chapters/5Experiments/4/main}

\include{chapters/6Conclusion/main}

\appendix
\include{appendix/AppendixB}
\include{appendix/D/main}
\include{appendix/AppendixC}

\backmatter
\bibliographystyle{ecs}
\bibliography{ECS}
\end{document}
%% ----------------------------------------------------------------

\include{chapters/3Design/HighLevel}
\include{chapters/3Design/InDepth}
 %% ----------------------------------------------------------------
%% Progress.tex
%% ---------------------------------------------------------------- 
\documentclass{ecsprogress}    % Use the progress Style
\graphicspath{{../figs/}}   % Location of your graphics files
    \usepackage{natbib}            % Use Natbib style for the refs.
\hypersetup{colorlinks=true}   % Set to false for black/white printing
\input{Definitions}            % Include your abbreviations



\usepackage{enumitem}% http://ctan.org/pkg/enumitem
\usepackage{multirow}
\usepackage{float}
\usepackage{amsmath}
\usepackage{multicol}
\usepackage{amssymb}
\usepackage[normalem]{ulem}
\useunder{\uline}{\ul}{}
\usepackage{wrapfig}


\usepackage[table,xcdraw]{xcolor}


%% ----------------------------------------------------------------
\begin{document}
\frontmatter
\title      {Heterogeneous Agent-based Model for Supermarket Competition}
\authors    {\texorpdfstring
             {\href{mailto:sc22g13@ecs.soton.ac.uk}{Stefan J. Collier}}
             {Stefan J. Collier}
            }
\addresses  {\groupname\\\deptname\\\univname}
\date       {\today}
\subject    {}
\keywords   {}
\supervisor {Dr. Maria Polukarov}
\examiner   {Professor Sheng Chen}

\maketitle
\begin{abstract}
This project aim was to model and analyse the effects of competitive pricing behaviors of grocery retailers on the British market. 

This was achieved by creating a multi-agent model, containing retailer and consumer agents. The heterogeneous crowd of retailers employs either a uniform pricing strategy or a ‘local price flexing’ strategy. The actions of these retailers are chosen by predicting the profit of each action, using a perceptron. Following on from the consideration of different economic models, a discrete model was developed so that software agents have a discrete environment to operate within. Within the model, it has been observed how supermarkets with differing behaviors affect a heterogeneous crowd of consumer agents. The model was implemented in Java with Python used to evaluate the results. 

The simulation displays good acceptance with real grocery market behavior, i.e. captures the performance of British retailers thus can be used to determine the impact of changes in their behavior on their competitors and consumers.Furthermore it can be used to provide insight into sustainability of volatile pricing strategies, providing a useful insight in volatility of British supermarket retail industry. 
\end{abstract}
\acknowledgements{
I would like to express my sincere gratitude to Dr Maria Polukarov for her guidance and support which provided me the freedom to take this research in the direction of my interest.\\
\\
I would also like to thank my family and friends for their encouragement and support. To those who quietly listened to my software complaints. To those who worked throughout the nights with me. To those who helped me write what I couldn't say. I cannot thank you enough.
}

\declaration{
I, Stefan Collier, declare that this dissertation and the work presented in it are my own and has been generated by me as the result of my own original research.\\
I confirm that:\\
1. This work was done wholly or mainly while in candidature for a degree at this University;\\
2. Where any part of this dissertation has previously been submitted for any other qualification at this University or any other institution, this has been clearly stated;\\
3. Where I have consulted the published work of others, this is always clearly attributed;\\
4. Where I have quoted from the work of others, the source is always given. With the exception of such quotations, this dissertation is entirely my own work;\\
5. I have acknowledged all main sources of help;\\
6. Where the thesis is based on work done by myself jointly with others, I have made clear exactly what was done by others and what I have contributed myself;\\
7. Either none of this work has been published before submission, or parts of this work have been published by :\\
\\
Stefan Collier\\
April 2016
}
\tableofcontents
\listoffigures
\listoftables

\mainmatter
%% ----------------------------------------------------------------
%\include{Introduction}
%\include{Conclusions}
\include{chapters/1Project/main}
\include{chapters/2Lit/main}
\include{chapters/3Design/HighLevel}
\include{chapters/3Design/InDepth}
\include{chapters/4Impl/main}

\include{chapters/5Experiments/1/main}
\include{chapters/5Experiments/2/main}
\include{chapters/5Experiments/3/main}
\include{chapters/5Experiments/4/main}

\include{chapters/6Conclusion/main}

\appendix
\include{appendix/AppendixB}
\include{appendix/D/main}
\include{appendix/AppendixC}

\backmatter
\bibliographystyle{ecs}
\bibliography{ECS}
\end{document}
%% ----------------------------------------------------------------


 %% ----------------------------------------------------------------
%% Progress.tex
%% ---------------------------------------------------------------- 
\documentclass{ecsprogress}    % Use the progress Style
\graphicspath{{../figs/}}   % Location of your graphics files
    \usepackage{natbib}            % Use Natbib style for the refs.
\hypersetup{colorlinks=true}   % Set to false for black/white printing
\input{Definitions}            % Include your abbreviations



\usepackage{enumitem}% http://ctan.org/pkg/enumitem
\usepackage{multirow}
\usepackage{float}
\usepackage{amsmath}
\usepackage{multicol}
\usepackage{amssymb}
\usepackage[normalem]{ulem}
\useunder{\uline}{\ul}{}
\usepackage{wrapfig}


\usepackage[table,xcdraw]{xcolor}


%% ----------------------------------------------------------------
\begin{document}
\frontmatter
\title      {Heterogeneous Agent-based Model for Supermarket Competition}
\authors    {\texorpdfstring
             {\href{mailto:sc22g13@ecs.soton.ac.uk}{Stefan J. Collier}}
             {Stefan J. Collier}
            }
\addresses  {\groupname\\\deptname\\\univname}
\date       {\today}
\subject    {}
\keywords   {}
\supervisor {Dr. Maria Polukarov}
\examiner   {Professor Sheng Chen}

\maketitle
\begin{abstract}
This project aim was to model and analyse the effects of competitive pricing behaviors of grocery retailers on the British market. 

This was achieved by creating a multi-agent model, containing retailer and consumer agents. The heterogeneous crowd of retailers employs either a uniform pricing strategy or a ‘local price flexing’ strategy. The actions of these retailers are chosen by predicting the profit of each action, using a perceptron. Following on from the consideration of different economic models, a discrete model was developed so that software agents have a discrete environment to operate within. Within the model, it has been observed how supermarkets with differing behaviors affect a heterogeneous crowd of consumer agents. The model was implemented in Java with Python used to evaluate the results. 

The simulation displays good acceptance with real grocery market behavior, i.e. captures the performance of British retailers thus can be used to determine the impact of changes in their behavior on their competitors and consumers.Furthermore it can be used to provide insight into sustainability of volatile pricing strategies, providing a useful insight in volatility of British supermarket retail industry. 
\end{abstract}
\acknowledgements{
I would like to express my sincere gratitude to Dr Maria Polukarov for her guidance and support which provided me the freedom to take this research in the direction of my interest.\\
\\
I would also like to thank my family and friends for their encouragement and support. To those who quietly listened to my software complaints. To those who worked throughout the nights with me. To those who helped me write what I couldn't say. I cannot thank you enough.
}

\declaration{
I, Stefan Collier, declare that this dissertation and the work presented in it are my own and has been generated by me as the result of my own original research.\\
I confirm that:\\
1. This work was done wholly or mainly while in candidature for a degree at this University;\\
2. Where any part of this dissertation has previously been submitted for any other qualification at this University or any other institution, this has been clearly stated;\\
3. Where I have consulted the published work of others, this is always clearly attributed;\\
4. Where I have quoted from the work of others, the source is always given. With the exception of such quotations, this dissertation is entirely my own work;\\
5. I have acknowledged all main sources of help;\\
6. Where the thesis is based on work done by myself jointly with others, I have made clear exactly what was done by others and what I have contributed myself;\\
7. Either none of this work has been published before submission, or parts of this work have been published by :\\
\\
Stefan Collier\\
April 2016
}
\tableofcontents
\listoffigures
\listoftables

\mainmatter
%% ----------------------------------------------------------------
%\include{Introduction}
%\include{Conclusions}
\include{chapters/1Project/main}
\include{chapters/2Lit/main}
\include{chapters/3Design/HighLevel}
\include{chapters/3Design/InDepth}
\include{chapters/4Impl/main}

\include{chapters/5Experiments/1/main}
\include{chapters/5Experiments/2/main}
\include{chapters/5Experiments/3/main}
\include{chapters/5Experiments/4/main}

\include{chapters/6Conclusion/main}

\appendix
\include{appendix/AppendixB}
\include{appendix/D/main}
\include{appendix/AppendixC}

\backmatter
\bibliographystyle{ecs}
\bibliography{ECS}
\end{document}
%% ----------------------------------------------------------------

 %% ----------------------------------------------------------------
%% Progress.tex
%% ---------------------------------------------------------------- 
\documentclass{ecsprogress}    % Use the progress Style
\graphicspath{{../figs/}}   % Location of your graphics files
    \usepackage{natbib}            % Use Natbib style for the refs.
\hypersetup{colorlinks=true}   % Set to false for black/white printing
\input{Definitions}            % Include your abbreviations



\usepackage{enumitem}% http://ctan.org/pkg/enumitem
\usepackage{multirow}
\usepackage{float}
\usepackage{amsmath}
\usepackage{multicol}
\usepackage{amssymb}
\usepackage[normalem]{ulem}
\useunder{\uline}{\ul}{}
\usepackage{wrapfig}


\usepackage[table,xcdraw]{xcolor}


%% ----------------------------------------------------------------
\begin{document}
\frontmatter
\title      {Heterogeneous Agent-based Model for Supermarket Competition}
\authors    {\texorpdfstring
             {\href{mailto:sc22g13@ecs.soton.ac.uk}{Stefan J. Collier}}
             {Stefan J. Collier}
            }
\addresses  {\groupname\\\deptname\\\univname}
\date       {\today}
\subject    {}
\keywords   {}
\supervisor {Dr. Maria Polukarov}
\examiner   {Professor Sheng Chen}

\maketitle
\begin{abstract}
This project aim was to model and analyse the effects of competitive pricing behaviors of grocery retailers on the British market. 

This was achieved by creating a multi-agent model, containing retailer and consumer agents. The heterogeneous crowd of retailers employs either a uniform pricing strategy or a ‘local price flexing’ strategy. The actions of these retailers are chosen by predicting the profit of each action, using a perceptron. Following on from the consideration of different economic models, a discrete model was developed so that software agents have a discrete environment to operate within. Within the model, it has been observed how supermarkets with differing behaviors affect a heterogeneous crowd of consumer agents. The model was implemented in Java with Python used to evaluate the results. 

The simulation displays good acceptance with real grocery market behavior, i.e. captures the performance of British retailers thus can be used to determine the impact of changes in their behavior on their competitors and consumers.Furthermore it can be used to provide insight into sustainability of volatile pricing strategies, providing a useful insight in volatility of British supermarket retail industry. 
\end{abstract}
\acknowledgements{
I would like to express my sincere gratitude to Dr Maria Polukarov for her guidance and support which provided me the freedom to take this research in the direction of my interest.\\
\\
I would also like to thank my family and friends for their encouragement and support. To those who quietly listened to my software complaints. To those who worked throughout the nights with me. To those who helped me write what I couldn't say. I cannot thank you enough.
}

\declaration{
I, Stefan Collier, declare that this dissertation and the work presented in it are my own and has been generated by me as the result of my own original research.\\
I confirm that:\\
1. This work was done wholly or mainly while in candidature for a degree at this University;\\
2. Where any part of this dissertation has previously been submitted for any other qualification at this University or any other institution, this has been clearly stated;\\
3. Where I have consulted the published work of others, this is always clearly attributed;\\
4. Where I have quoted from the work of others, the source is always given. With the exception of such quotations, this dissertation is entirely my own work;\\
5. I have acknowledged all main sources of help;\\
6. Where the thesis is based on work done by myself jointly with others, I have made clear exactly what was done by others and what I have contributed myself;\\
7. Either none of this work has been published before submission, or parts of this work have been published by :\\
\\
Stefan Collier\\
April 2016
}
\tableofcontents
\listoffigures
\listoftables

\mainmatter
%% ----------------------------------------------------------------
%\include{Introduction}
%\include{Conclusions}
\include{chapters/1Project/main}
\include{chapters/2Lit/main}
\include{chapters/3Design/HighLevel}
\include{chapters/3Design/InDepth}
\include{chapters/4Impl/main}

\include{chapters/5Experiments/1/main}
\include{chapters/5Experiments/2/main}
\include{chapters/5Experiments/3/main}
\include{chapters/5Experiments/4/main}

\include{chapters/6Conclusion/main}

\appendix
\include{appendix/AppendixB}
\include{appendix/D/main}
\include{appendix/AppendixC}

\backmatter
\bibliographystyle{ecs}
\bibliography{ECS}
\end{document}
%% ----------------------------------------------------------------

 %% ----------------------------------------------------------------
%% Progress.tex
%% ---------------------------------------------------------------- 
\documentclass{ecsprogress}    % Use the progress Style
\graphicspath{{../figs/}}   % Location of your graphics files
    \usepackage{natbib}            % Use Natbib style for the refs.
\hypersetup{colorlinks=true}   % Set to false for black/white printing
\input{Definitions}            % Include your abbreviations



\usepackage{enumitem}% http://ctan.org/pkg/enumitem
\usepackage{multirow}
\usepackage{float}
\usepackage{amsmath}
\usepackage{multicol}
\usepackage{amssymb}
\usepackage[normalem]{ulem}
\useunder{\uline}{\ul}{}
\usepackage{wrapfig}


\usepackage[table,xcdraw]{xcolor}


%% ----------------------------------------------------------------
\begin{document}
\frontmatter
\title      {Heterogeneous Agent-based Model for Supermarket Competition}
\authors    {\texorpdfstring
             {\href{mailto:sc22g13@ecs.soton.ac.uk}{Stefan J. Collier}}
             {Stefan J. Collier}
            }
\addresses  {\groupname\\\deptname\\\univname}
\date       {\today}
\subject    {}
\keywords   {}
\supervisor {Dr. Maria Polukarov}
\examiner   {Professor Sheng Chen}

\maketitle
\begin{abstract}
This project aim was to model and analyse the effects of competitive pricing behaviors of grocery retailers on the British market. 

This was achieved by creating a multi-agent model, containing retailer and consumer agents. The heterogeneous crowd of retailers employs either a uniform pricing strategy or a ‘local price flexing’ strategy. The actions of these retailers are chosen by predicting the profit of each action, using a perceptron. Following on from the consideration of different economic models, a discrete model was developed so that software agents have a discrete environment to operate within. Within the model, it has been observed how supermarkets with differing behaviors affect a heterogeneous crowd of consumer agents. The model was implemented in Java with Python used to evaluate the results. 

The simulation displays good acceptance with real grocery market behavior, i.e. captures the performance of British retailers thus can be used to determine the impact of changes in their behavior on their competitors and consumers.Furthermore it can be used to provide insight into sustainability of volatile pricing strategies, providing a useful insight in volatility of British supermarket retail industry. 
\end{abstract}
\acknowledgements{
I would like to express my sincere gratitude to Dr Maria Polukarov for her guidance and support which provided me the freedom to take this research in the direction of my interest.\\
\\
I would also like to thank my family and friends for their encouragement and support. To those who quietly listened to my software complaints. To those who worked throughout the nights with me. To those who helped me write what I couldn't say. I cannot thank you enough.
}

\declaration{
I, Stefan Collier, declare that this dissertation and the work presented in it are my own and has been generated by me as the result of my own original research.\\
I confirm that:\\
1. This work was done wholly or mainly while in candidature for a degree at this University;\\
2. Where any part of this dissertation has previously been submitted for any other qualification at this University or any other institution, this has been clearly stated;\\
3. Where I have consulted the published work of others, this is always clearly attributed;\\
4. Where I have quoted from the work of others, the source is always given. With the exception of such quotations, this dissertation is entirely my own work;\\
5. I have acknowledged all main sources of help;\\
6. Where the thesis is based on work done by myself jointly with others, I have made clear exactly what was done by others and what I have contributed myself;\\
7. Either none of this work has been published before submission, or parts of this work have been published by :\\
\\
Stefan Collier\\
April 2016
}
\tableofcontents
\listoffigures
\listoftables

\mainmatter
%% ----------------------------------------------------------------
%\include{Introduction}
%\include{Conclusions}
\include{chapters/1Project/main}
\include{chapters/2Lit/main}
\include{chapters/3Design/HighLevel}
\include{chapters/3Design/InDepth}
\include{chapters/4Impl/main}

\include{chapters/5Experiments/1/main}
\include{chapters/5Experiments/2/main}
\include{chapters/5Experiments/3/main}
\include{chapters/5Experiments/4/main}

\include{chapters/6Conclusion/main}

\appendix
\include{appendix/AppendixB}
\include{appendix/D/main}
\include{appendix/AppendixC}

\backmatter
\bibliographystyle{ecs}
\bibliography{ECS}
\end{document}
%% ----------------------------------------------------------------

 %% ----------------------------------------------------------------
%% Progress.tex
%% ---------------------------------------------------------------- 
\documentclass{ecsprogress}    % Use the progress Style
\graphicspath{{../figs/}}   % Location of your graphics files
    \usepackage{natbib}            % Use Natbib style for the refs.
\hypersetup{colorlinks=true}   % Set to false for black/white printing
\input{Definitions}            % Include your abbreviations



\usepackage{enumitem}% http://ctan.org/pkg/enumitem
\usepackage{multirow}
\usepackage{float}
\usepackage{amsmath}
\usepackage{multicol}
\usepackage{amssymb}
\usepackage[normalem]{ulem}
\useunder{\uline}{\ul}{}
\usepackage{wrapfig}


\usepackage[table,xcdraw]{xcolor}


%% ----------------------------------------------------------------
\begin{document}
\frontmatter
\title      {Heterogeneous Agent-based Model for Supermarket Competition}
\authors    {\texorpdfstring
             {\href{mailto:sc22g13@ecs.soton.ac.uk}{Stefan J. Collier}}
             {Stefan J. Collier}
            }
\addresses  {\groupname\\\deptname\\\univname}
\date       {\today}
\subject    {}
\keywords   {}
\supervisor {Dr. Maria Polukarov}
\examiner   {Professor Sheng Chen}

\maketitle
\begin{abstract}
This project aim was to model and analyse the effects of competitive pricing behaviors of grocery retailers on the British market. 

This was achieved by creating a multi-agent model, containing retailer and consumer agents. The heterogeneous crowd of retailers employs either a uniform pricing strategy or a ‘local price flexing’ strategy. The actions of these retailers are chosen by predicting the profit of each action, using a perceptron. Following on from the consideration of different economic models, a discrete model was developed so that software agents have a discrete environment to operate within. Within the model, it has been observed how supermarkets with differing behaviors affect a heterogeneous crowd of consumer agents. The model was implemented in Java with Python used to evaluate the results. 

The simulation displays good acceptance with real grocery market behavior, i.e. captures the performance of British retailers thus can be used to determine the impact of changes in their behavior on their competitors and consumers.Furthermore it can be used to provide insight into sustainability of volatile pricing strategies, providing a useful insight in volatility of British supermarket retail industry. 
\end{abstract}
\acknowledgements{
I would like to express my sincere gratitude to Dr Maria Polukarov for her guidance and support which provided me the freedom to take this research in the direction of my interest.\\
\\
I would also like to thank my family and friends for their encouragement and support. To those who quietly listened to my software complaints. To those who worked throughout the nights with me. To those who helped me write what I couldn't say. I cannot thank you enough.
}

\declaration{
I, Stefan Collier, declare that this dissertation and the work presented in it are my own and has been generated by me as the result of my own original research.\\
I confirm that:\\
1. This work was done wholly or mainly while in candidature for a degree at this University;\\
2. Where any part of this dissertation has previously been submitted for any other qualification at this University or any other institution, this has been clearly stated;\\
3. Where I have consulted the published work of others, this is always clearly attributed;\\
4. Where I have quoted from the work of others, the source is always given. With the exception of such quotations, this dissertation is entirely my own work;\\
5. I have acknowledged all main sources of help;\\
6. Where the thesis is based on work done by myself jointly with others, I have made clear exactly what was done by others and what I have contributed myself;\\
7. Either none of this work has been published before submission, or parts of this work have been published by :\\
\\
Stefan Collier\\
April 2016
}
\tableofcontents
\listoffigures
\listoftables

\mainmatter
%% ----------------------------------------------------------------
%\include{Introduction}
%\include{Conclusions}
\include{chapters/1Project/main}
\include{chapters/2Lit/main}
\include{chapters/3Design/HighLevel}
\include{chapters/3Design/InDepth}
\include{chapters/4Impl/main}

\include{chapters/5Experiments/1/main}
\include{chapters/5Experiments/2/main}
\include{chapters/5Experiments/3/main}
\include{chapters/5Experiments/4/main}

\include{chapters/6Conclusion/main}

\appendix
\include{appendix/AppendixB}
\include{appendix/D/main}
\include{appendix/AppendixC}

\backmatter
\bibliographystyle{ecs}
\bibliography{ECS}
\end{document}
%% ----------------------------------------------------------------


 %% ----------------------------------------------------------------
%% Progress.tex
%% ---------------------------------------------------------------- 
\documentclass{ecsprogress}    % Use the progress Style
\graphicspath{{../figs/}}   % Location of your graphics files
    \usepackage{natbib}            % Use Natbib style for the refs.
\hypersetup{colorlinks=true}   % Set to false for black/white printing
\input{Definitions}            % Include your abbreviations



\usepackage{enumitem}% http://ctan.org/pkg/enumitem
\usepackage{multirow}
\usepackage{float}
\usepackage{amsmath}
\usepackage{multicol}
\usepackage{amssymb}
\usepackage[normalem]{ulem}
\useunder{\uline}{\ul}{}
\usepackage{wrapfig}


\usepackage[table,xcdraw]{xcolor}


%% ----------------------------------------------------------------
\begin{document}
\frontmatter
\title      {Heterogeneous Agent-based Model for Supermarket Competition}
\authors    {\texorpdfstring
             {\href{mailto:sc22g13@ecs.soton.ac.uk}{Stefan J. Collier}}
             {Stefan J. Collier}
            }
\addresses  {\groupname\\\deptname\\\univname}
\date       {\today}
\subject    {}
\keywords   {}
\supervisor {Dr. Maria Polukarov}
\examiner   {Professor Sheng Chen}

\maketitle
\begin{abstract}
This project aim was to model and analyse the effects of competitive pricing behaviors of grocery retailers on the British market. 

This was achieved by creating a multi-agent model, containing retailer and consumer agents. The heterogeneous crowd of retailers employs either a uniform pricing strategy or a ‘local price flexing’ strategy. The actions of these retailers are chosen by predicting the profit of each action, using a perceptron. Following on from the consideration of different economic models, a discrete model was developed so that software agents have a discrete environment to operate within. Within the model, it has been observed how supermarkets with differing behaviors affect a heterogeneous crowd of consumer agents. The model was implemented in Java with Python used to evaluate the results. 

The simulation displays good acceptance with real grocery market behavior, i.e. captures the performance of British retailers thus can be used to determine the impact of changes in their behavior on their competitors and consumers.Furthermore it can be used to provide insight into sustainability of volatile pricing strategies, providing a useful insight in volatility of British supermarket retail industry. 
\end{abstract}
\acknowledgements{
I would like to express my sincere gratitude to Dr Maria Polukarov for her guidance and support which provided me the freedom to take this research in the direction of my interest.\\
\\
I would also like to thank my family and friends for their encouragement and support. To those who quietly listened to my software complaints. To those who worked throughout the nights with me. To those who helped me write what I couldn't say. I cannot thank you enough.
}

\declaration{
I, Stefan Collier, declare that this dissertation and the work presented in it are my own and has been generated by me as the result of my own original research.\\
I confirm that:\\
1. This work was done wholly or mainly while in candidature for a degree at this University;\\
2. Where any part of this dissertation has previously been submitted for any other qualification at this University or any other institution, this has been clearly stated;\\
3. Where I have consulted the published work of others, this is always clearly attributed;\\
4. Where I have quoted from the work of others, the source is always given. With the exception of such quotations, this dissertation is entirely my own work;\\
5. I have acknowledged all main sources of help;\\
6. Where the thesis is based on work done by myself jointly with others, I have made clear exactly what was done by others and what I have contributed myself;\\
7. Either none of this work has been published before submission, or parts of this work have been published by :\\
\\
Stefan Collier\\
April 2016
}
\tableofcontents
\listoffigures
\listoftables

\mainmatter
%% ----------------------------------------------------------------
%\include{Introduction}
%\include{Conclusions}
\include{chapters/1Project/main}
\include{chapters/2Lit/main}
\include{chapters/3Design/HighLevel}
\include{chapters/3Design/InDepth}
\include{chapters/4Impl/main}

\include{chapters/5Experiments/1/main}
\include{chapters/5Experiments/2/main}
\include{chapters/5Experiments/3/main}
\include{chapters/5Experiments/4/main}

\include{chapters/6Conclusion/main}

\appendix
\include{appendix/AppendixB}
\include{appendix/D/main}
\include{appendix/AppendixC}

\backmatter
\bibliographystyle{ecs}
\bibliography{ECS}
\end{document}
%% ----------------------------------------------------------------


\appendix
\include{appendix/AppendixB}
 %% ----------------------------------------------------------------
%% Progress.tex
%% ---------------------------------------------------------------- 
\documentclass{ecsprogress}    % Use the progress Style
\graphicspath{{../figs/}}   % Location of your graphics files
    \usepackage{natbib}            % Use Natbib style for the refs.
\hypersetup{colorlinks=true}   % Set to false for black/white printing
\input{Definitions}            % Include your abbreviations



\usepackage{enumitem}% http://ctan.org/pkg/enumitem
\usepackage{multirow}
\usepackage{float}
\usepackage{amsmath}
\usepackage{multicol}
\usepackage{amssymb}
\usepackage[normalem]{ulem}
\useunder{\uline}{\ul}{}
\usepackage{wrapfig}


\usepackage[table,xcdraw]{xcolor}


%% ----------------------------------------------------------------
\begin{document}
\frontmatter
\title      {Heterogeneous Agent-based Model for Supermarket Competition}
\authors    {\texorpdfstring
             {\href{mailto:sc22g13@ecs.soton.ac.uk}{Stefan J. Collier}}
             {Stefan J. Collier}
            }
\addresses  {\groupname\\\deptname\\\univname}
\date       {\today}
\subject    {}
\keywords   {}
\supervisor {Dr. Maria Polukarov}
\examiner   {Professor Sheng Chen}

\maketitle
\begin{abstract}
This project aim was to model and analyse the effects of competitive pricing behaviors of grocery retailers on the British market. 

This was achieved by creating a multi-agent model, containing retailer and consumer agents. The heterogeneous crowd of retailers employs either a uniform pricing strategy or a ‘local price flexing’ strategy. The actions of these retailers are chosen by predicting the profit of each action, using a perceptron. Following on from the consideration of different economic models, a discrete model was developed so that software agents have a discrete environment to operate within. Within the model, it has been observed how supermarkets with differing behaviors affect a heterogeneous crowd of consumer agents. The model was implemented in Java with Python used to evaluate the results. 

The simulation displays good acceptance with real grocery market behavior, i.e. captures the performance of British retailers thus can be used to determine the impact of changes in their behavior on their competitors and consumers.Furthermore it can be used to provide insight into sustainability of volatile pricing strategies, providing a useful insight in volatility of British supermarket retail industry. 
\end{abstract}
\acknowledgements{
I would like to express my sincere gratitude to Dr Maria Polukarov for her guidance and support which provided me the freedom to take this research in the direction of my interest.\\
\\
I would also like to thank my family and friends for their encouragement and support. To those who quietly listened to my software complaints. To those who worked throughout the nights with me. To those who helped me write what I couldn't say. I cannot thank you enough.
}

\declaration{
I, Stefan Collier, declare that this dissertation and the work presented in it are my own and has been generated by me as the result of my own original research.\\
I confirm that:\\
1. This work was done wholly or mainly while in candidature for a degree at this University;\\
2. Where any part of this dissertation has previously been submitted for any other qualification at this University or any other institution, this has been clearly stated;\\
3. Where I have consulted the published work of others, this is always clearly attributed;\\
4. Where I have quoted from the work of others, the source is always given. With the exception of such quotations, this dissertation is entirely my own work;\\
5. I have acknowledged all main sources of help;\\
6. Where the thesis is based on work done by myself jointly with others, I have made clear exactly what was done by others and what I have contributed myself;\\
7. Either none of this work has been published before submission, or parts of this work have been published by :\\
\\
Stefan Collier\\
April 2016
}
\tableofcontents
\listoffigures
\listoftables

\mainmatter
%% ----------------------------------------------------------------
%\include{Introduction}
%\include{Conclusions}
\include{chapters/1Project/main}
\include{chapters/2Lit/main}
\include{chapters/3Design/HighLevel}
\include{chapters/3Design/InDepth}
\include{chapters/4Impl/main}

\include{chapters/5Experiments/1/main}
\include{chapters/5Experiments/2/main}
\include{chapters/5Experiments/3/main}
\include{chapters/5Experiments/4/main}

\include{chapters/6Conclusion/main}

\appendix
\include{appendix/AppendixB}
\include{appendix/D/main}
\include{appendix/AppendixC}

\backmatter
\bibliographystyle{ecs}
\bibliography{ECS}
\end{document}
%% ----------------------------------------------------------------

\include{appendix/AppendixC}

\backmatter
\bibliographystyle{ecs}
\bibliography{ECS}
\end{document}
%% ----------------------------------------------------------------


\appendix
\include{appendix/AppendixB}
 %% ----------------------------------------------------------------
%% Progress.tex
%% ---------------------------------------------------------------- 
\documentclass{ecsprogress}    % Use the progress Style
\graphicspath{{../figs/}}   % Location of your graphics files
    \usepackage{natbib}            % Use Natbib style for the refs.
\hypersetup{colorlinks=true}   % Set to false for black/white printing
\input{Definitions}            % Include your abbreviations



\usepackage{enumitem}% http://ctan.org/pkg/enumitem
\usepackage{multirow}
\usepackage{float}
\usepackage{amsmath}
\usepackage{multicol}
\usepackage{amssymb}
\usepackage[normalem]{ulem}
\useunder{\uline}{\ul}{}
\usepackage{wrapfig}


\usepackage[table,xcdraw]{xcolor}


%% ----------------------------------------------------------------
\begin{document}
\frontmatter
\title      {Heterogeneous Agent-based Model for Supermarket Competition}
\authors    {\texorpdfstring
             {\href{mailto:sc22g13@ecs.soton.ac.uk}{Stefan J. Collier}}
             {Stefan J. Collier}
            }
\addresses  {\groupname\\\deptname\\\univname}
\date       {\today}
\subject    {}
\keywords   {}
\supervisor {Dr. Maria Polukarov}
\examiner   {Professor Sheng Chen}

\maketitle
\begin{abstract}
This project aim was to model and analyse the effects of competitive pricing behaviors of grocery retailers on the British market. 

This was achieved by creating a multi-agent model, containing retailer and consumer agents. The heterogeneous crowd of retailers employs either a uniform pricing strategy or a ‘local price flexing’ strategy. The actions of these retailers are chosen by predicting the profit of each action, using a perceptron. Following on from the consideration of different economic models, a discrete model was developed so that software agents have a discrete environment to operate within. Within the model, it has been observed how supermarkets with differing behaviors affect a heterogeneous crowd of consumer agents. The model was implemented in Java with Python used to evaluate the results. 

The simulation displays good acceptance with real grocery market behavior, i.e. captures the performance of British retailers thus can be used to determine the impact of changes in their behavior on their competitors and consumers.Furthermore it can be used to provide insight into sustainability of volatile pricing strategies, providing a useful insight in volatility of British supermarket retail industry. 
\end{abstract}
\acknowledgements{
I would like to express my sincere gratitude to Dr Maria Polukarov for her guidance and support which provided me the freedom to take this research in the direction of my interest.\\
\\
I would also like to thank my family and friends for their encouragement and support. To those who quietly listened to my software complaints. To those who worked throughout the nights with me. To those who helped me write what I couldn't say. I cannot thank you enough.
}

\declaration{
I, Stefan Collier, declare that this dissertation and the work presented in it are my own and has been generated by me as the result of my own original research.\\
I confirm that:\\
1. This work was done wholly or mainly while in candidature for a degree at this University;\\
2. Where any part of this dissertation has previously been submitted for any other qualification at this University or any other institution, this has been clearly stated;\\
3. Where I have consulted the published work of others, this is always clearly attributed;\\
4. Where I have quoted from the work of others, the source is always given. With the exception of such quotations, this dissertation is entirely my own work;\\
5. I have acknowledged all main sources of help;\\
6. Where the thesis is based on work done by myself jointly with others, I have made clear exactly what was done by others and what I have contributed myself;\\
7. Either none of this work has been published before submission, or parts of this work have been published by :\\
\\
Stefan Collier\\
April 2016
}
\tableofcontents
\listoffigures
\listoftables

\mainmatter
%% ----------------------------------------------------------------
%\include{Introduction}
%\include{Conclusions}
 %% ----------------------------------------------------------------
%% Progress.tex
%% ---------------------------------------------------------------- 
\documentclass{ecsprogress}    % Use the progress Style
\graphicspath{{../figs/}}   % Location of your graphics files
    \usepackage{natbib}            % Use Natbib style for the refs.
\hypersetup{colorlinks=true}   % Set to false for black/white printing
\input{Definitions}            % Include your abbreviations



\usepackage{enumitem}% http://ctan.org/pkg/enumitem
\usepackage{multirow}
\usepackage{float}
\usepackage{amsmath}
\usepackage{multicol}
\usepackage{amssymb}
\usepackage[normalem]{ulem}
\useunder{\uline}{\ul}{}
\usepackage{wrapfig}


\usepackage[table,xcdraw]{xcolor}


%% ----------------------------------------------------------------
\begin{document}
\frontmatter
\title      {Heterogeneous Agent-based Model for Supermarket Competition}
\authors    {\texorpdfstring
             {\href{mailto:sc22g13@ecs.soton.ac.uk}{Stefan J. Collier}}
             {Stefan J. Collier}
            }
\addresses  {\groupname\\\deptname\\\univname}
\date       {\today}
\subject    {}
\keywords   {}
\supervisor {Dr. Maria Polukarov}
\examiner   {Professor Sheng Chen}

\maketitle
\begin{abstract}
This project aim was to model and analyse the effects of competitive pricing behaviors of grocery retailers on the British market. 

This was achieved by creating a multi-agent model, containing retailer and consumer agents. The heterogeneous crowd of retailers employs either a uniform pricing strategy or a ‘local price flexing’ strategy. The actions of these retailers are chosen by predicting the profit of each action, using a perceptron. Following on from the consideration of different economic models, a discrete model was developed so that software agents have a discrete environment to operate within. Within the model, it has been observed how supermarkets with differing behaviors affect a heterogeneous crowd of consumer agents. The model was implemented in Java with Python used to evaluate the results. 

The simulation displays good acceptance with real grocery market behavior, i.e. captures the performance of British retailers thus can be used to determine the impact of changes in their behavior on their competitors and consumers.Furthermore it can be used to provide insight into sustainability of volatile pricing strategies, providing a useful insight in volatility of British supermarket retail industry. 
\end{abstract}
\acknowledgements{
I would like to express my sincere gratitude to Dr Maria Polukarov for her guidance and support which provided me the freedom to take this research in the direction of my interest.\\
\\
I would also like to thank my family and friends for their encouragement and support. To those who quietly listened to my software complaints. To those who worked throughout the nights with me. To those who helped me write what I couldn't say. I cannot thank you enough.
}

\declaration{
I, Stefan Collier, declare that this dissertation and the work presented in it are my own and has been generated by me as the result of my own original research.\\
I confirm that:\\
1. This work was done wholly or mainly while in candidature for a degree at this University;\\
2. Where any part of this dissertation has previously been submitted for any other qualification at this University or any other institution, this has been clearly stated;\\
3. Where I have consulted the published work of others, this is always clearly attributed;\\
4. Where I have quoted from the work of others, the source is always given. With the exception of such quotations, this dissertation is entirely my own work;\\
5. I have acknowledged all main sources of help;\\
6. Where the thesis is based on work done by myself jointly with others, I have made clear exactly what was done by others and what I have contributed myself;\\
7. Either none of this work has been published before submission, or parts of this work have been published by :\\
\\
Stefan Collier\\
April 2016
}
\tableofcontents
\listoffigures
\listoftables

\mainmatter
%% ----------------------------------------------------------------
%\include{Introduction}
%\include{Conclusions}
\include{chapters/1Project/main}
\include{chapters/2Lit/main}
\include{chapters/3Design/HighLevel}
\include{chapters/3Design/InDepth}
\include{chapters/4Impl/main}

\include{chapters/5Experiments/1/main}
\include{chapters/5Experiments/2/main}
\include{chapters/5Experiments/3/main}
\include{chapters/5Experiments/4/main}

\include{chapters/6Conclusion/main}

\appendix
\include{appendix/AppendixB}
\include{appendix/D/main}
\include{appendix/AppendixC}

\backmatter
\bibliographystyle{ecs}
\bibliography{ECS}
\end{document}
%% ----------------------------------------------------------------

 %% ----------------------------------------------------------------
%% Progress.tex
%% ---------------------------------------------------------------- 
\documentclass{ecsprogress}    % Use the progress Style
\graphicspath{{../figs/}}   % Location of your graphics files
    \usepackage{natbib}            % Use Natbib style for the refs.
\hypersetup{colorlinks=true}   % Set to false for black/white printing
\input{Definitions}            % Include your abbreviations



\usepackage{enumitem}% http://ctan.org/pkg/enumitem
\usepackage{multirow}
\usepackage{float}
\usepackage{amsmath}
\usepackage{multicol}
\usepackage{amssymb}
\usepackage[normalem]{ulem}
\useunder{\uline}{\ul}{}
\usepackage{wrapfig}


\usepackage[table,xcdraw]{xcolor}


%% ----------------------------------------------------------------
\begin{document}
\frontmatter
\title      {Heterogeneous Agent-based Model for Supermarket Competition}
\authors    {\texorpdfstring
             {\href{mailto:sc22g13@ecs.soton.ac.uk}{Stefan J. Collier}}
             {Stefan J. Collier}
            }
\addresses  {\groupname\\\deptname\\\univname}
\date       {\today}
\subject    {}
\keywords   {}
\supervisor {Dr. Maria Polukarov}
\examiner   {Professor Sheng Chen}

\maketitle
\begin{abstract}
This project aim was to model and analyse the effects of competitive pricing behaviors of grocery retailers on the British market. 

This was achieved by creating a multi-agent model, containing retailer and consumer agents. The heterogeneous crowd of retailers employs either a uniform pricing strategy or a ‘local price flexing’ strategy. The actions of these retailers are chosen by predicting the profit of each action, using a perceptron. Following on from the consideration of different economic models, a discrete model was developed so that software agents have a discrete environment to operate within. Within the model, it has been observed how supermarkets with differing behaviors affect a heterogeneous crowd of consumer agents. The model was implemented in Java with Python used to evaluate the results. 

The simulation displays good acceptance with real grocery market behavior, i.e. captures the performance of British retailers thus can be used to determine the impact of changes in their behavior on their competitors and consumers.Furthermore it can be used to provide insight into sustainability of volatile pricing strategies, providing a useful insight in volatility of British supermarket retail industry. 
\end{abstract}
\acknowledgements{
I would like to express my sincere gratitude to Dr Maria Polukarov for her guidance and support which provided me the freedom to take this research in the direction of my interest.\\
\\
I would also like to thank my family and friends for their encouragement and support. To those who quietly listened to my software complaints. To those who worked throughout the nights with me. To those who helped me write what I couldn't say. I cannot thank you enough.
}

\declaration{
I, Stefan Collier, declare that this dissertation and the work presented in it are my own and has been generated by me as the result of my own original research.\\
I confirm that:\\
1. This work was done wholly or mainly while in candidature for a degree at this University;\\
2. Where any part of this dissertation has previously been submitted for any other qualification at this University or any other institution, this has been clearly stated;\\
3. Where I have consulted the published work of others, this is always clearly attributed;\\
4. Where I have quoted from the work of others, the source is always given. With the exception of such quotations, this dissertation is entirely my own work;\\
5. I have acknowledged all main sources of help;\\
6. Where the thesis is based on work done by myself jointly with others, I have made clear exactly what was done by others and what I have contributed myself;\\
7. Either none of this work has been published before submission, or parts of this work have been published by :\\
\\
Stefan Collier\\
April 2016
}
\tableofcontents
\listoffigures
\listoftables

\mainmatter
%% ----------------------------------------------------------------
%\include{Introduction}
%\include{Conclusions}
\include{chapters/1Project/main}
\include{chapters/2Lit/main}
\include{chapters/3Design/HighLevel}
\include{chapters/3Design/InDepth}
\include{chapters/4Impl/main}

\include{chapters/5Experiments/1/main}
\include{chapters/5Experiments/2/main}
\include{chapters/5Experiments/3/main}
\include{chapters/5Experiments/4/main}

\include{chapters/6Conclusion/main}

\appendix
\include{appendix/AppendixB}
\include{appendix/D/main}
\include{appendix/AppendixC}

\backmatter
\bibliographystyle{ecs}
\bibliography{ECS}
\end{document}
%% ----------------------------------------------------------------

\include{chapters/3Design/HighLevel}
\include{chapters/3Design/InDepth}
 %% ----------------------------------------------------------------
%% Progress.tex
%% ---------------------------------------------------------------- 
\documentclass{ecsprogress}    % Use the progress Style
\graphicspath{{../figs/}}   % Location of your graphics files
    \usepackage{natbib}            % Use Natbib style for the refs.
\hypersetup{colorlinks=true}   % Set to false for black/white printing
\input{Definitions}            % Include your abbreviations



\usepackage{enumitem}% http://ctan.org/pkg/enumitem
\usepackage{multirow}
\usepackage{float}
\usepackage{amsmath}
\usepackage{multicol}
\usepackage{amssymb}
\usepackage[normalem]{ulem}
\useunder{\uline}{\ul}{}
\usepackage{wrapfig}


\usepackage[table,xcdraw]{xcolor}


%% ----------------------------------------------------------------
\begin{document}
\frontmatter
\title      {Heterogeneous Agent-based Model for Supermarket Competition}
\authors    {\texorpdfstring
             {\href{mailto:sc22g13@ecs.soton.ac.uk}{Stefan J. Collier}}
             {Stefan J. Collier}
            }
\addresses  {\groupname\\\deptname\\\univname}
\date       {\today}
\subject    {}
\keywords   {}
\supervisor {Dr. Maria Polukarov}
\examiner   {Professor Sheng Chen}

\maketitle
\begin{abstract}
This project aim was to model and analyse the effects of competitive pricing behaviors of grocery retailers on the British market. 

This was achieved by creating a multi-agent model, containing retailer and consumer agents. The heterogeneous crowd of retailers employs either a uniform pricing strategy or a ‘local price flexing’ strategy. The actions of these retailers are chosen by predicting the profit of each action, using a perceptron. Following on from the consideration of different economic models, a discrete model was developed so that software agents have a discrete environment to operate within. Within the model, it has been observed how supermarkets with differing behaviors affect a heterogeneous crowd of consumer agents. The model was implemented in Java with Python used to evaluate the results. 

The simulation displays good acceptance with real grocery market behavior, i.e. captures the performance of British retailers thus can be used to determine the impact of changes in their behavior on their competitors and consumers.Furthermore it can be used to provide insight into sustainability of volatile pricing strategies, providing a useful insight in volatility of British supermarket retail industry. 
\end{abstract}
\acknowledgements{
I would like to express my sincere gratitude to Dr Maria Polukarov for her guidance and support which provided me the freedom to take this research in the direction of my interest.\\
\\
I would also like to thank my family and friends for their encouragement and support. To those who quietly listened to my software complaints. To those who worked throughout the nights with me. To those who helped me write what I couldn't say. I cannot thank you enough.
}

\declaration{
I, Stefan Collier, declare that this dissertation and the work presented in it are my own and has been generated by me as the result of my own original research.\\
I confirm that:\\
1. This work was done wholly or mainly while in candidature for a degree at this University;\\
2. Where any part of this dissertation has previously been submitted for any other qualification at this University or any other institution, this has been clearly stated;\\
3. Where I have consulted the published work of others, this is always clearly attributed;\\
4. Where I have quoted from the work of others, the source is always given. With the exception of such quotations, this dissertation is entirely my own work;\\
5. I have acknowledged all main sources of help;\\
6. Where the thesis is based on work done by myself jointly with others, I have made clear exactly what was done by others and what I have contributed myself;\\
7. Either none of this work has been published before submission, or parts of this work have been published by :\\
\\
Stefan Collier\\
April 2016
}
\tableofcontents
\listoffigures
\listoftables

\mainmatter
%% ----------------------------------------------------------------
%\include{Introduction}
%\include{Conclusions}
\include{chapters/1Project/main}
\include{chapters/2Lit/main}
\include{chapters/3Design/HighLevel}
\include{chapters/3Design/InDepth}
\include{chapters/4Impl/main}

\include{chapters/5Experiments/1/main}
\include{chapters/5Experiments/2/main}
\include{chapters/5Experiments/3/main}
\include{chapters/5Experiments/4/main}

\include{chapters/6Conclusion/main}

\appendix
\include{appendix/AppendixB}
\include{appendix/D/main}
\include{appendix/AppendixC}

\backmatter
\bibliographystyle{ecs}
\bibliography{ECS}
\end{document}
%% ----------------------------------------------------------------


 %% ----------------------------------------------------------------
%% Progress.tex
%% ---------------------------------------------------------------- 
\documentclass{ecsprogress}    % Use the progress Style
\graphicspath{{../figs/}}   % Location of your graphics files
    \usepackage{natbib}            % Use Natbib style for the refs.
\hypersetup{colorlinks=true}   % Set to false for black/white printing
\input{Definitions}            % Include your abbreviations



\usepackage{enumitem}% http://ctan.org/pkg/enumitem
\usepackage{multirow}
\usepackage{float}
\usepackage{amsmath}
\usepackage{multicol}
\usepackage{amssymb}
\usepackage[normalem]{ulem}
\useunder{\uline}{\ul}{}
\usepackage{wrapfig}


\usepackage[table,xcdraw]{xcolor}


%% ----------------------------------------------------------------
\begin{document}
\frontmatter
\title      {Heterogeneous Agent-based Model for Supermarket Competition}
\authors    {\texorpdfstring
             {\href{mailto:sc22g13@ecs.soton.ac.uk}{Stefan J. Collier}}
             {Stefan J. Collier}
            }
\addresses  {\groupname\\\deptname\\\univname}
\date       {\today}
\subject    {}
\keywords   {}
\supervisor {Dr. Maria Polukarov}
\examiner   {Professor Sheng Chen}

\maketitle
\begin{abstract}
This project aim was to model and analyse the effects of competitive pricing behaviors of grocery retailers on the British market. 

This was achieved by creating a multi-agent model, containing retailer and consumer agents. The heterogeneous crowd of retailers employs either a uniform pricing strategy or a ‘local price flexing’ strategy. The actions of these retailers are chosen by predicting the profit of each action, using a perceptron. Following on from the consideration of different economic models, a discrete model was developed so that software agents have a discrete environment to operate within. Within the model, it has been observed how supermarkets with differing behaviors affect a heterogeneous crowd of consumer agents. The model was implemented in Java with Python used to evaluate the results. 

The simulation displays good acceptance with real grocery market behavior, i.e. captures the performance of British retailers thus can be used to determine the impact of changes in their behavior on their competitors and consumers.Furthermore it can be used to provide insight into sustainability of volatile pricing strategies, providing a useful insight in volatility of British supermarket retail industry. 
\end{abstract}
\acknowledgements{
I would like to express my sincere gratitude to Dr Maria Polukarov for her guidance and support which provided me the freedom to take this research in the direction of my interest.\\
\\
I would also like to thank my family and friends for their encouragement and support. To those who quietly listened to my software complaints. To those who worked throughout the nights with me. To those who helped me write what I couldn't say. I cannot thank you enough.
}

\declaration{
I, Stefan Collier, declare that this dissertation and the work presented in it are my own and has been generated by me as the result of my own original research.\\
I confirm that:\\
1. This work was done wholly or mainly while in candidature for a degree at this University;\\
2. Where any part of this dissertation has previously been submitted for any other qualification at this University or any other institution, this has been clearly stated;\\
3. Where I have consulted the published work of others, this is always clearly attributed;\\
4. Where I have quoted from the work of others, the source is always given. With the exception of such quotations, this dissertation is entirely my own work;\\
5. I have acknowledged all main sources of help;\\
6. Where the thesis is based on work done by myself jointly with others, I have made clear exactly what was done by others and what I have contributed myself;\\
7. Either none of this work has been published before submission, or parts of this work have been published by :\\
\\
Stefan Collier\\
April 2016
}
\tableofcontents
\listoffigures
\listoftables

\mainmatter
%% ----------------------------------------------------------------
%\include{Introduction}
%\include{Conclusions}
\include{chapters/1Project/main}
\include{chapters/2Lit/main}
\include{chapters/3Design/HighLevel}
\include{chapters/3Design/InDepth}
\include{chapters/4Impl/main}

\include{chapters/5Experiments/1/main}
\include{chapters/5Experiments/2/main}
\include{chapters/5Experiments/3/main}
\include{chapters/5Experiments/4/main}

\include{chapters/6Conclusion/main}

\appendix
\include{appendix/AppendixB}
\include{appendix/D/main}
\include{appendix/AppendixC}

\backmatter
\bibliographystyle{ecs}
\bibliography{ECS}
\end{document}
%% ----------------------------------------------------------------

 %% ----------------------------------------------------------------
%% Progress.tex
%% ---------------------------------------------------------------- 
\documentclass{ecsprogress}    % Use the progress Style
\graphicspath{{../figs/}}   % Location of your graphics files
    \usepackage{natbib}            % Use Natbib style for the refs.
\hypersetup{colorlinks=true}   % Set to false for black/white printing
\input{Definitions}            % Include your abbreviations



\usepackage{enumitem}% http://ctan.org/pkg/enumitem
\usepackage{multirow}
\usepackage{float}
\usepackage{amsmath}
\usepackage{multicol}
\usepackage{amssymb}
\usepackage[normalem]{ulem}
\useunder{\uline}{\ul}{}
\usepackage{wrapfig}


\usepackage[table,xcdraw]{xcolor}


%% ----------------------------------------------------------------
\begin{document}
\frontmatter
\title      {Heterogeneous Agent-based Model for Supermarket Competition}
\authors    {\texorpdfstring
             {\href{mailto:sc22g13@ecs.soton.ac.uk}{Stefan J. Collier}}
             {Stefan J. Collier}
            }
\addresses  {\groupname\\\deptname\\\univname}
\date       {\today}
\subject    {}
\keywords   {}
\supervisor {Dr. Maria Polukarov}
\examiner   {Professor Sheng Chen}

\maketitle
\begin{abstract}
This project aim was to model and analyse the effects of competitive pricing behaviors of grocery retailers on the British market. 

This was achieved by creating a multi-agent model, containing retailer and consumer agents. The heterogeneous crowd of retailers employs either a uniform pricing strategy or a ‘local price flexing’ strategy. The actions of these retailers are chosen by predicting the profit of each action, using a perceptron. Following on from the consideration of different economic models, a discrete model was developed so that software agents have a discrete environment to operate within. Within the model, it has been observed how supermarkets with differing behaviors affect a heterogeneous crowd of consumer agents. The model was implemented in Java with Python used to evaluate the results. 

The simulation displays good acceptance with real grocery market behavior, i.e. captures the performance of British retailers thus can be used to determine the impact of changes in their behavior on their competitors and consumers.Furthermore it can be used to provide insight into sustainability of volatile pricing strategies, providing a useful insight in volatility of British supermarket retail industry. 
\end{abstract}
\acknowledgements{
I would like to express my sincere gratitude to Dr Maria Polukarov for her guidance and support which provided me the freedom to take this research in the direction of my interest.\\
\\
I would also like to thank my family and friends for their encouragement and support. To those who quietly listened to my software complaints. To those who worked throughout the nights with me. To those who helped me write what I couldn't say. I cannot thank you enough.
}

\declaration{
I, Stefan Collier, declare that this dissertation and the work presented in it are my own and has been generated by me as the result of my own original research.\\
I confirm that:\\
1. This work was done wholly or mainly while in candidature for a degree at this University;\\
2. Where any part of this dissertation has previously been submitted for any other qualification at this University or any other institution, this has been clearly stated;\\
3. Where I have consulted the published work of others, this is always clearly attributed;\\
4. Where I have quoted from the work of others, the source is always given. With the exception of such quotations, this dissertation is entirely my own work;\\
5. I have acknowledged all main sources of help;\\
6. Where the thesis is based on work done by myself jointly with others, I have made clear exactly what was done by others and what I have contributed myself;\\
7. Either none of this work has been published before submission, or parts of this work have been published by :\\
\\
Stefan Collier\\
April 2016
}
\tableofcontents
\listoffigures
\listoftables

\mainmatter
%% ----------------------------------------------------------------
%\include{Introduction}
%\include{Conclusions}
\include{chapters/1Project/main}
\include{chapters/2Lit/main}
\include{chapters/3Design/HighLevel}
\include{chapters/3Design/InDepth}
\include{chapters/4Impl/main}

\include{chapters/5Experiments/1/main}
\include{chapters/5Experiments/2/main}
\include{chapters/5Experiments/3/main}
\include{chapters/5Experiments/4/main}

\include{chapters/6Conclusion/main}

\appendix
\include{appendix/AppendixB}
\include{appendix/D/main}
\include{appendix/AppendixC}

\backmatter
\bibliographystyle{ecs}
\bibliography{ECS}
\end{document}
%% ----------------------------------------------------------------

 %% ----------------------------------------------------------------
%% Progress.tex
%% ---------------------------------------------------------------- 
\documentclass{ecsprogress}    % Use the progress Style
\graphicspath{{../figs/}}   % Location of your graphics files
    \usepackage{natbib}            % Use Natbib style for the refs.
\hypersetup{colorlinks=true}   % Set to false for black/white printing
\input{Definitions}            % Include your abbreviations



\usepackage{enumitem}% http://ctan.org/pkg/enumitem
\usepackage{multirow}
\usepackage{float}
\usepackage{amsmath}
\usepackage{multicol}
\usepackage{amssymb}
\usepackage[normalem]{ulem}
\useunder{\uline}{\ul}{}
\usepackage{wrapfig}


\usepackage[table,xcdraw]{xcolor}


%% ----------------------------------------------------------------
\begin{document}
\frontmatter
\title      {Heterogeneous Agent-based Model for Supermarket Competition}
\authors    {\texorpdfstring
             {\href{mailto:sc22g13@ecs.soton.ac.uk}{Stefan J. Collier}}
             {Stefan J. Collier}
            }
\addresses  {\groupname\\\deptname\\\univname}
\date       {\today}
\subject    {}
\keywords   {}
\supervisor {Dr. Maria Polukarov}
\examiner   {Professor Sheng Chen}

\maketitle
\begin{abstract}
This project aim was to model and analyse the effects of competitive pricing behaviors of grocery retailers on the British market. 

This was achieved by creating a multi-agent model, containing retailer and consumer agents. The heterogeneous crowd of retailers employs either a uniform pricing strategy or a ‘local price flexing’ strategy. The actions of these retailers are chosen by predicting the profit of each action, using a perceptron. Following on from the consideration of different economic models, a discrete model was developed so that software agents have a discrete environment to operate within. Within the model, it has been observed how supermarkets with differing behaviors affect a heterogeneous crowd of consumer agents. The model was implemented in Java with Python used to evaluate the results. 

The simulation displays good acceptance with real grocery market behavior, i.e. captures the performance of British retailers thus can be used to determine the impact of changes in their behavior on their competitors and consumers.Furthermore it can be used to provide insight into sustainability of volatile pricing strategies, providing a useful insight in volatility of British supermarket retail industry. 
\end{abstract}
\acknowledgements{
I would like to express my sincere gratitude to Dr Maria Polukarov for her guidance and support which provided me the freedom to take this research in the direction of my interest.\\
\\
I would also like to thank my family and friends for their encouragement and support. To those who quietly listened to my software complaints. To those who worked throughout the nights with me. To those who helped me write what I couldn't say. I cannot thank you enough.
}

\declaration{
I, Stefan Collier, declare that this dissertation and the work presented in it are my own and has been generated by me as the result of my own original research.\\
I confirm that:\\
1. This work was done wholly or mainly while in candidature for a degree at this University;\\
2. Where any part of this dissertation has previously been submitted for any other qualification at this University or any other institution, this has been clearly stated;\\
3. Where I have consulted the published work of others, this is always clearly attributed;\\
4. Where I have quoted from the work of others, the source is always given. With the exception of such quotations, this dissertation is entirely my own work;\\
5. I have acknowledged all main sources of help;\\
6. Where the thesis is based on work done by myself jointly with others, I have made clear exactly what was done by others and what I have contributed myself;\\
7. Either none of this work has been published before submission, or parts of this work have been published by :\\
\\
Stefan Collier\\
April 2016
}
\tableofcontents
\listoffigures
\listoftables

\mainmatter
%% ----------------------------------------------------------------
%\include{Introduction}
%\include{Conclusions}
\include{chapters/1Project/main}
\include{chapters/2Lit/main}
\include{chapters/3Design/HighLevel}
\include{chapters/3Design/InDepth}
\include{chapters/4Impl/main}

\include{chapters/5Experiments/1/main}
\include{chapters/5Experiments/2/main}
\include{chapters/5Experiments/3/main}
\include{chapters/5Experiments/4/main}

\include{chapters/6Conclusion/main}

\appendix
\include{appendix/AppendixB}
\include{appendix/D/main}
\include{appendix/AppendixC}

\backmatter
\bibliographystyle{ecs}
\bibliography{ECS}
\end{document}
%% ----------------------------------------------------------------

 %% ----------------------------------------------------------------
%% Progress.tex
%% ---------------------------------------------------------------- 
\documentclass{ecsprogress}    % Use the progress Style
\graphicspath{{../figs/}}   % Location of your graphics files
    \usepackage{natbib}            % Use Natbib style for the refs.
\hypersetup{colorlinks=true}   % Set to false for black/white printing
\input{Definitions}            % Include your abbreviations



\usepackage{enumitem}% http://ctan.org/pkg/enumitem
\usepackage{multirow}
\usepackage{float}
\usepackage{amsmath}
\usepackage{multicol}
\usepackage{amssymb}
\usepackage[normalem]{ulem}
\useunder{\uline}{\ul}{}
\usepackage{wrapfig}


\usepackage[table,xcdraw]{xcolor}


%% ----------------------------------------------------------------
\begin{document}
\frontmatter
\title      {Heterogeneous Agent-based Model for Supermarket Competition}
\authors    {\texorpdfstring
             {\href{mailto:sc22g13@ecs.soton.ac.uk}{Stefan J. Collier}}
             {Stefan J. Collier}
            }
\addresses  {\groupname\\\deptname\\\univname}
\date       {\today}
\subject    {}
\keywords   {}
\supervisor {Dr. Maria Polukarov}
\examiner   {Professor Sheng Chen}

\maketitle
\begin{abstract}
This project aim was to model and analyse the effects of competitive pricing behaviors of grocery retailers on the British market. 

This was achieved by creating a multi-agent model, containing retailer and consumer agents. The heterogeneous crowd of retailers employs either a uniform pricing strategy or a ‘local price flexing’ strategy. The actions of these retailers are chosen by predicting the profit of each action, using a perceptron. Following on from the consideration of different economic models, a discrete model was developed so that software agents have a discrete environment to operate within. Within the model, it has been observed how supermarkets with differing behaviors affect a heterogeneous crowd of consumer agents. The model was implemented in Java with Python used to evaluate the results. 

The simulation displays good acceptance with real grocery market behavior, i.e. captures the performance of British retailers thus can be used to determine the impact of changes in their behavior on their competitors and consumers.Furthermore it can be used to provide insight into sustainability of volatile pricing strategies, providing a useful insight in volatility of British supermarket retail industry. 
\end{abstract}
\acknowledgements{
I would like to express my sincere gratitude to Dr Maria Polukarov for her guidance and support which provided me the freedom to take this research in the direction of my interest.\\
\\
I would also like to thank my family and friends for their encouragement and support. To those who quietly listened to my software complaints. To those who worked throughout the nights with me. To those who helped me write what I couldn't say. I cannot thank you enough.
}

\declaration{
I, Stefan Collier, declare that this dissertation and the work presented in it are my own and has been generated by me as the result of my own original research.\\
I confirm that:\\
1. This work was done wholly or mainly while in candidature for a degree at this University;\\
2. Where any part of this dissertation has previously been submitted for any other qualification at this University or any other institution, this has been clearly stated;\\
3. Where I have consulted the published work of others, this is always clearly attributed;\\
4. Where I have quoted from the work of others, the source is always given. With the exception of such quotations, this dissertation is entirely my own work;\\
5. I have acknowledged all main sources of help;\\
6. Where the thesis is based on work done by myself jointly with others, I have made clear exactly what was done by others and what I have contributed myself;\\
7. Either none of this work has been published before submission, or parts of this work have been published by :\\
\\
Stefan Collier\\
April 2016
}
\tableofcontents
\listoffigures
\listoftables

\mainmatter
%% ----------------------------------------------------------------
%\include{Introduction}
%\include{Conclusions}
\include{chapters/1Project/main}
\include{chapters/2Lit/main}
\include{chapters/3Design/HighLevel}
\include{chapters/3Design/InDepth}
\include{chapters/4Impl/main}

\include{chapters/5Experiments/1/main}
\include{chapters/5Experiments/2/main}
\include{chapters/5Experiments/3/main}
\include{chapters/5Experiments/4/main}

\include{chapters/6Conclusion/main}

\appendix
\include{appendix/AppendixB}
\include{appendix/D/main}
\include{appendix/AppendixC}

\backmatter
\bibliographystyle{ecs}
\bibliography{ECS}
\end{document}
%% ----------------------------------------------------------------


 %% ----------------------------------------------------------------
%% Progress.tex
%% ---------------------------------------------------------------- 
\documentclass{ecsprogress}    % Use the progress Style
\graphicspath{{../figs/}}   % Location of your graphics files
    \usepackage{natbib}            % Use Natbib style for the refs.
\hypersetup{colorlinks=true}   % Set to false for black/white printing
\input{Definitions}            % Include your abbreviations



\usepackage{enumitem}% http://ctan.org/pkg/enumitem
\usepackage{multirow}
\usepackage{float}
\usepackage{amsmath}
\usepackage{multicol}
\usepackage{amssymb}
\usepackage[normalem]{ulem}
\useunder{\uline}{\ul}{}
\usepackage{wrapfig}


\usepackage[table,xcdraw]{xcolor}


%% ----------------------------------------------------------------
\begin{document}
\frontmatter
\title      {Heterogeneous Agent-based Model for Supermarket Competition}
\authors    {\texorpdfstring
             {\href{mailto:sc22g13@ecs.soton.ac.uk}{Stefan J. Collier}}
             {Stefan J. Collier}
            }
\addresses  {\groupname\\\deptname\\\univname}
\date       {\today}
\subject    {}
\keywords   {}
\supervisor {Dr. Maria Polukarov}
\examiner   {Professor Sheng Chen}

\maketitle
\begin{abstract}
This project aim was to model and analyse the effects of competitive pricing behaviors of grocery retailers on the British market. 

This was achieved by creating a multi-agent model, containing retailer and consumer agents. The heterogeneous crowd of retailers employs either a uniform pricing strategy or a ‘local price flexing’ strategy. The actions of these retailers are chosen by predicting the profit of each action, using a perceptron. Following on from the consideration of different economic models, a discrete model was developed so that software agents have a discrete environment to operate within. Within the model, it has been observed how supermarkets with differing behaviors affect a heterogeneous crowd of consumer agents. The model was implemented in Java with Python used to evaluate the results. 

The simulation displays good acceptance with real grocery market behavior, i.e. captures the performance of British retailers thus can be used to determine the impact of changes in their behavior on their competitors and consumers.Furthermore it can be used to provide insight into sustainability of volatile pricing strategies, providing a useful insight in volatility of British supermarket retail industry. 
\end{abstract}
\acknowledgements{
I would like to express my sincere gratitude to Dr Maria Polukarov for her guidance and support which provided me the freedom to take this research in the direction of my interest.\\
\\
I would also like to thank my family and friends for their encouragement and support. To those who quietly listened to my software complaints. To those who worked throughout the nights with me. To those who helped me write what I couldn't say. I cannot thank you enough.
}

\declaration{
I, Stefan Collier, declare that this dissertation and the work presented in it are my own and has been generated by me as the result of my own original research.\\
I confirm that:\\
1. This work was done wholly or mainly while in candidature for a degree at this University;\\
2. Where any part of this dissertation has previously been submitted for any other qualification at this University or any other institution, this has been clearly stated;\\
3. Where I have consulted the published work of others, this is always clearly attributed;\\
4. Where I have quoted from the work of others, the source is always given. With the exception of such quotations, this dissertation is entirely my own work;\\
5. I have acknowledged all main sources of help;\\
6. Where the thesis is based on work done by myself jointly with others, I have made clear exactly what was done by others and what I have contributed myself;\\
7. Either none of this work has been published before submission, or parts of this work have been published by :\\
\\
Stefan Collier\\
April 2016
}
\tableofcontents
\listoffigures
\listoftables

\mainmatter
%% ----------------------------------------------------------------
%\include{Introduction}
%\include{Conclusions}
\include{chapters/1Project/main}
\include{chapters/2Lit/main}
\include{chapters/3Design/HighLevel}
\include{chapters/3Design/InDepth}
\include{chapters/4Impl/main}

\include{chapters/5Experiments/1/main}
\include{chapters/5Experiments/2/main}
\include{chapters/5Experiments/3/main}
\include{chapters/5Experiments/4/main}

\include{chapters/6Conclusion/main}

\appendix
\include{appendix/AppendixB}
\include{appendix/D/main}
\include{appendix/AppendixC}

\backmatter
\bibliographystyle{ecs}
\bibliography{ECS}
\end{document}
%% ----------------------------------------------------------------


\appendix
\include{appendix/AppendixB}
 %% ----------------------------------------------------------------
%% Progress.tex
%% ---------------------------------------------------------------- 
\documentclass{ecsprogress}    % Use the progress Style
\graphicspath{{../figs/}}   % Location of your graphics files
    \usepackage{natbib}            % Use Natbib style for the refs.
\hypersetup{colorlinks=true}   % Set to false for black/white printing
\input{Definitions}            % Include your abbreviations



\usepackage{enumitem}% http://ctan.org/pkg/enumitem
\usepackage{multirow}
\usepackage{float}
\usepackage{amsmath}
\usepackage{multicol}
\usepackage{amssymb}
\usepackage[normalem]{ulem}
\useunder{\uline}{\ul}{}
\usepackage{wrapfig}


\usepackage[table,xcdraw]{xcolor}


%% ----------------------------------------------------------------
\begin{document}
\frontmatter
\title      {Heterogeneous Agent-based Model for Supermarket Competition}
\authors    {\texorpdfstring
             {\href{mailto:sc22g13@ecs.soton.ac.uk}{Stefan J. Collier}}
             {Stefan J. Collier}
            }
\addresses  {\groupname\\\deptname\\\univname}
\date       {\today}
\subject    {}
\keywords   {}
\supervisor {Dr. Maria Polukarov}
\examiner   {Professor Sheng Chen}

\maketitle
\begin{abstract}
This project aim was to model and analyse the effects of competitive pricing behaviors of grocery retailers on the British market. 

This was achieved by creating a multi-agent model, containing retailer and consumer agents. The heterogeneous crowd of retailers employs either a uniform pricing strategy or a ‘local price flexing’ strategy. The actions of these retailers are chosen by predicting the profit of each action, using a perceptron. Following on from the consideration of different economic models, a discrete model was developed so that software agents have a discrete environment to operate within. Within the model, it has been observed how supermarkets with differing behaviors affect a heterogeneous crowd of consumer agents. The model was implemented in Java with Python used to evaluate the results. 

The simulation displays good acceptance with real grocery market behavior, i.e. captures the performance of British retailers thus can be used to determine the impact of changes in their behavior on their competitors and consumers.Furthermore it can be used to provide insight into sustainability of volatile pricing strategies, providing a useful insight in volatility of British supermarket retail industry. 
\end{abstract}
\acknowledgements{
I would like to express my sincere gratitude to Dr Maria Polukarov for her guidance and support which provided me the freedom to take this research in the direction of my interest.\\
\\
I would also like to thank my family and friends for their encouragement and support. To those who quietly listened to my software complaints. To those who worked throughout the nights with me. To those who helped me write what I couldn't say. I cannot thank you enough.
}

\declaration{
I, Stefan Collier, declare that this dissertation and the work presented in it are my own and has been generated by me as the result of my own original research.\\
I confirm that:\\
1. This work was done wholly or mainly while in candidature for a degree at this University;\\
2. Where any part of this dissertation has previously been submitted for any other qualification at this University or any other institution, this has been clearly stated;\\
3. Where I have consulted the published work of others, this is always clearly attributed;\\
4. Where I have quoted from the work of others, the source is always given. With the exception of such quotations, this dissertation is entirely my own work;\\
5. I have acknowledged all main sources of help;\\
6. Where the thesis is based on work done by myself jointly with others, I have made clear exactly what was done by others and what I have contributed myself;\\
7. Either none of this work has been published before submission, or parts of this work have been published by :\\
\\
Stefan Collier\\
April 2016
}
\tableofcontents
\listoffigures
\listoftables

\mainmatter
%% ----------------------------------------------------------------
%\include{Introduction}
%\include{Conclusions}
\include{chapters/1Project/main}
\include{chapters/2Lit/main}
\include{chapters/3Design/HighLevel}
\include{chapters/3Design/InDepth}
\include{chapters/4Impl/main}

\include{chapters/5Experiments/1/main}
\include{chapters/5Experiments/2/main}
\include{chapters/5Experiments/3/main}
\include{chapters/5Experiments/4/main}

\include{chapters/6Conclusion/main}

\appendix
\include{appendix/AppendixB}
\include{appendix/D/main}
\include{appendix/AppendixC}

\backmatter
\bibliographystyle{ecs}
\bibliography{ECS}
\end{document}
%% ----------------------------------------------------------------

\include{appendix/AppendixC}

\backmatter
\bibliographystyle{ecs}
\bibliography{ECS}
\end{document}
%% ----------------------------------------------------------------

\include{appendix/AppendixC}

\backmatter
\bibliographystyle{ecs}
\bibliography{ECS}
\end{document}
%% ----------------------------------------------------------------

\include{appendix/AppendixC}

\backmatter
\bibliographystyle{ecs}
\bibliography{ECS}
\end{document}
%% ----------------------------------------------------------------
